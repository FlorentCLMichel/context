% language=uk

% author    : Hans Hagen
% copyright : PRAGMA ADE & ConTeXt Development Team
% license   : Creative Commons Attribution ShareAlike 4.0 International
% reference : pragma-ade.nl | contextgarden.net | texlive (related) distributions
% origin    : the ConTeXt distribution
%
% comment   : Because this manual is distributed with TeX distributions it comes with a rather
%             liberal license. We try to adapt these documents to upgrades in the (sub)systems
%             that they describe. Using parts of the content otherwise can therefore conflict
%             with existing functionality and we cannot be held responsible for that. Many of
%             the manuals contain characteristic graphics and personal notes or examples that
%             make no sense when used out-of-context.

\usemodule[mag-01,abr-02]

\startbuffer[abstract]
    A not so widely known feature of the verbatim handler in \CONTEXT\ is the
    ability to add comments in another style and \MKIV\ even offers a bit more.
    Here some examples are shown.
\stopbuffer

\startdocument
  [title={Annotated Verbatim},
   author=Hans Hagen,
   affiliation=PRAGMA ADE,
   date=July 2011,
   number=1102 \MKIV]

\definetextbackground
  [example]
  [frame=on,
   framecolor=darkblue,
   location=paragraph,
   leftoffset=1ex,
   topoffset=1ex,
   bottomoffset=1ex]

Annotating verbatim content is done using a mechanism called escaping. For such
special cases it's often best to define a specific instance.

\startbuffer[define]
\definetyping
  [annotatedtyping]
  [escape=/,
   color=darkblue,
   before=,
   after=]
\stopbuffer

\startbuffer[example]
\startannotatedtyping
bla = test               /bgroup /sl oeps /egroup
                         /bgroup /bf some more /egroup
    | another test
    | somethingverylong  /bgroup /it oeps /egroup
\stopannotatedtyping
\stopbuffer

\typebuffer[define,example][option=TEX] \getbuffer[define]

\starttextbackground[example]
    \getbuffer[example]
\stoptextbackground

In this example the \type {/} now serves as an escape character. Of course you
can also use the normal backslash but then you need to use a command to specify
it.

\startbuffer[setup]
\setuptyping
  [annotatedtyping]
  [escape=\letterbackslash]
\stopbuffer

\typebuffer[setup][option=TEX] \getbuffer[setup]

Now we can say:

\startbuffer[example]
\startannotatedtyping
bla = test               \bgroup \sl oeps \egroup
                         \bgroup \bf some more \egroup
    | another test
    | somethingverylong  \bgroup \it oeps \egroup
\stopannotatedtyping
\stopbuffer

\typebuffer[example][option=TEX]

and get:

\starttextbackground[example]
    \getbuffer[example]
\stoptextbackground

You can also define an end symbol:

\startbuffer[setup]
\setuptyping
  [annotatedtyping]
  [escape={//,*},
   color=darkblue]

\definestartstop
  [cmt]
  [style=\rm\bf]
\stopbuffer

\typebuffer[setup][option=TEX] \getbuffer[setup]

Here the \type {//} starts the annotation and \type {*} ends it.

\startbuffer[example]
\startannotatedtyping
bla = test               // \black // \cmt{oeps} *
                         // \black // \cmt{some more} *
    | another test
    | somethingverylong  // \black // \cmt{oeps} *
\stopannotatedtyping
\stopbuffer

\typebuffer[example][option=TEX]

Contrary to the first example, all text in the annotation is treated as \TEX\
input:

\starttextbackground[example]
    \getbuffer[example]
\stoptextbackground

You can consider using more balanced tagging, as in:

\startbuffer[setup]
\setuptyping
  [annotatedtyping]
  [escape={<<,>>},
   color=darkblue]
\stopbuffer

\typebuffer[example][option=TEX]

Watch how we limit the annotation to part of the text:

\startbuffer[example]
\startannotatedtyping
bla = test               << \rm\bf first  >> test
                         << \rm\bf second >> test
    | test
    | somethingverylong  << \rm\bf fourth >> test
\stopannotatedtyping
\stopbuffer

\typebuffer[example][option=TEX]

The \type {test} a the end of the lines is verbatim again.

\starttextbackground[example]
    \getbuffer[example]
\stoptextbackground

If no end symbol is given, the end of the line is used instead:

\startbuffer[setup]
\setuptyping
  [annotatedtyping]
  [escape={//,},
   color=darkblue]
\stopbuffer

\typebuffer[setup][option=TEX] \getbuffer[setup]

Watch out: here we use \type {{//,}} and not just \type {//} (which would trigger
the escaped variant).

\definestartstop[cmt][style=\rm\bf]

\startbuffer[example]
\startannotatedtyping
bla = test               // \black // \cmt{oeps}
                         // \black // \cmt{some more}
    | test
    | somethingverylong  // \black // \cmt{oeps}
\stopannotatedtyping
\stopbuffer

\typebuffer[example][option=TEX]

The result is:

\starttextbackground[example]
    \getbuffer[example]
\stoptextbackground

This can also be done easier by abusing the \type {style} option of \type {cmt}:

\startbuffer[setup]
\definestartstop
  [cmt]
  [color=black,
   style=\black //\rm\bf\space]
\stopbuffer

\typebuffer[setup][option=TEX] \getbuffer[setup]

When we give:

\startbuffer[example]
\startannotatedtyping
bla = test               // \cmt{oeps}
                         // \cmt{some more}
    | test
    | somethingverylong  // \cmt{oeps}
\stopannotatedtyping
\stopbuffer

\typebuffer[example][option=TEX]

We get:

\starttextbackground[example]
    \getbuffer[example]
\stoptextbackground

For cases like this, where we want to specify a somewhat detailed way to deal
with a situation, we can use processors: \footnote {More mechanisms in \CONTEXT\
\MKIV\ will use that feature.}

\startbuffer[setup]
\defineprocessor
  [escape]
  [style=bold,
   color=black,
   left=(,right=)]
\stopbuffer

\typebuffer[setup][option=TEX] \getbuffer[setup]

The previous definition of the annotation now becomes:

\startbuffer[setup]
\setuptyping
  [annotatedtyping]
  [escape=escape->{//,},
   color=darkblue]
\stopbuffer

\typebuffer[setup][option=TEX] \getbuffer[setup]

This time no commands are needed in the annotation:

\startbuffer[example]
\startannotatedtyping
bla = test               // first
                         // second
    | test
    | somethingverylong  // fourth
\stopannotatedtyping
\stopbuffer

\typebuffer[example][option=TEX]

The processor is applied to all text following the \type {//}. Spaces before the
text are stripped.

\starttextbackground[example]
    \getbuffer[example]
\stoptextbackground

As some characters are special to \TEX, sometimes you need to escape the boundary
sequence:

\startbuffer[setup]
\defineprocessor
  [myescape]
  [style=\rm\tf,
   color=black]

\setuptyping
  [annotatedtyping]
  [escape=myescape->{\letterhash\letterhash,},
   color=darkgreen]
\stopbuffer

\typebuffer[setup][option=TEX] \getbuffer[setup]

All text between the double hashes and the end of the line is now treated as
annotation:

\startbuffer[example]
\startannotatedtyping
bla = test               ## first \bf test
                         ## second \sl test
    | test
    | somethingverylong  ## third \it test
\stopannotatedtyping
\stopbuffer

\typebuffer[example][option=TEX]

So we get:

\starttextbackground[example]
    \getbuffer[example]
\stoptextbackground

We can beautify \TEX\ commenting as follows:

\startbuffer[setup]
\defineprocessor
  [comment]
  [style=\rm,
   color=black,
   left={\tttf\letterpercent\space}]

\setuptyping
  [annotatedtyping]
  [escape=comment->{\letterpercent\letterpercent,},
   color=darkblue]
\stopbuffer

\typebuffer[setup][option=TEX] \getbuffer[setup]

Here the double comments are turned into a single one and the text after it is
typeset in a regular font:

\startbuffer[example]
\startannotatedtyping
bla = test               %% first \bf test
                         %% second \sl test
    | test
    | somethingverylong  %% third \it test
\stopannotatedtyping
\stopbuffer

\typebuffer[example][option=TEX]

This gives:

\starttextbackground[example]
    \getbuffer[example]
\stoptextbackground

It is possible to define several escapes. Let's start with the delimited variant:

\startbuffer[setup]
\defineprocessor
  [escape_a]
  [style=bold,
   color=darkred,
   left=(,
   right=)]

\defineprocessor
  [escape_b]
  [style=bold,
   color=darkgreen,
   left=(,
   right=)]

\setuptyping
  [annotatedtyping]
  [escape={escape_a->{[[,]]},escape_b->{[(,)]}},
   color=darkblue]
\stopbuffer

\typebuffer[setup][option=TEX] \getbuffer[setup]

We can now alternate comments:

\startbuffer[example]
\startannotatedtyping
bla = test               [[ first  ]] test [( first  )]
                         [[ second ]] test [( second )]
    | test
    | somethingverylong  [[ fourth ]] test [( fourth )]
\stopannotatedtyping
\stopbuffer

\typebuffer[example][option=TEX]

When typeset this looks as follows:

\starttextbackground[example]
    \getbuffer[example]
\stoptextbackground

The line terminated variant can also have multiple escapes.

\startbuffer[setup]
\defineprocessor
  [annotated_bf]
  [style=\rm\bf,
   color=darkred]

\defineprocessor
  [annotated_bs]
  [style=\rm\bs,
   color=darkyellow]

\setuptyping
  [annotatedtyping]
  [escape={annotated_bf->{!bf,},annotated_bs->{!bs,}},
   color=darkblue]
\stopbuffer

\typebuffer[setup][option=TEX] \getbuffer[setup]

So this time we have two ways to enter regular \TEX\ mode:

\startbuffer[example]
\startannotatedtyping
bla = test               !bf one {\em again}
                         !bs two {\em again}
    | test
    | somethingverylong  !bf three {\em again}
\stopannotatedtyping
\stopbuffer

\typebuffer[example][option=TEX]

These somewhat meaningful tags result in:

\starttextbackground[example]
    \getbuffer[example]
\stoptextbackground

\stopdocument
