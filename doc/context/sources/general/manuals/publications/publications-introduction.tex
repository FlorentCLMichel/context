\environment publications-style

\startcomponent publications-introduction

\startchapter[title=Introduction]

\startsubject[title=How Hans got involved]

This manual is dedicated to Taco \Name{Hoekwater}{T.} who in a previous century
implemented the first \BIBTEX\ module and saw it morph into a \TEX||\LUA\ hybrid
in this century. The fact that there was support for bibliographies made it
possible for users to use \CONTEXT\ in an academic environment, dominated by
bibliographic databases encoded in the \BIBTEX\ format.

This manual describes how \MKIV\ now handles bibliographies. Support in \CONTEXT\
started in \MKII\ for \BIBTEX, as mentioned above, using a module written by Taco
\Name {Hoekwater} {T.}. Later his code was adapted to \MKIV, but because users
demanded more, I decided that reimplementing made more sense than patching. In
particular, through the use of \LUA, the \BIBTEX\ data files can be easily
directly parsed, thus liberating \CONTEXT\ from the dependency on an external
\BIBTEX\ executable. The \Index{CritEd project} (by Thomas \Name {Schmitz} {T.},
Alan \Name {Braslau} {A.}, Luigi \Name {Scarso} {L.} and \name {Hagen}
{H.}myself) was a good reason to undertake this rewrite. As part that project
users were invited to come up with ideas about extensions. Not all of them are
(yet) honored, but the rewrite makes more functionality possible.

The subsystem described here is one of the most complex and messy of all
\CONTEXT\ subsystems. This has to do with the fact that it combines (multiple)
lists and (multiple) forward and backward references, all kind of rendering of
the citation as well as the entry in the list, rather complex interactivity,
multiple databases, datasets and renderings and of course combinations of this.
The implementation uses a mix of \TEX\ and \LUA\ code with so called setups as
rendering specifications. At the cost of complexity (and some runtime penalty)
this provides a lot of freedom and flexibility.

% \startlines
% Hans \Name {Hagen} {H.}
% PRAGMA ADE
% Hasselt NL
% \stoplines

\stopsubject

\startsubject[title=How Alan got involved]

Bibliographies and citations are of utmost importance in any scholarly work.
Nevertheless, the production of bibliography lists and the insertion of
citations, just like the production of an index, is a task that is often
postponed to a later stage in the writing of an article, a book, or a manual.
Perhaps this is because it can be more important to create than to refer, but
maybe the necessary tools are found to be insufficient or unnatural.

A computerized typesetting system should help an author produce a text, not
impose any preset format or unnecessary constraint. In a referenced work, a
bibliography system should be flexible enough to adapt to very different styles
and practices. Creating such a system is quite a challenge.

\CONTEXT\ \MKII\ implemented a system that was based on a use of \BIBTEX, an
external program that built upon basic bibliographic macros introduced in \LATEX.
\CONTEXT\ \MKIV\ moved away from this dependency, opening up many possibilities
for new functionality and, we hoped, providing more natural and flexible tools
for authors. For my own use, the most important of which is a very powerful
search and match mechanism that has been made possible though the use of \LUA.

I had started by asking simple questions on details of the workings of this new
system and making \quotation{wouldn't it be nice to} requests for functionality
that I knew was somehow buried in the inner workings of \CONTEXT. As a result of
these inquiries, I got drawn into the project to make this new system a reality.

% \startlines
% Alan \Name{Braslau}{A.}
% Paris, France
% \stoplines

\stopsubject

\startsubject[title=How you can be involved]

Bibliography management is indeed one of the most complex subsystems in \CONTEXT,
and many, many design decisions had to be made during its development.
Experimental features were added, some of which were later abandoned as being
inappropriate or else superseded by some better mechanism. The effort (and time)
that we spent in reimplementing the treatment of bibliographies was much greater
than any of us had anticipated when we undertook this project. Hopefully, now the
system is stable enough to be more widely used and this manual is an attempt to
make it accessible to all users.

There are \CONTEXT\ users who will just use whatever the bibliograpy modules
provide by default. For many, the \APA\ style is good enough; others may have
specific needs. This manual should provide insight on how to adapt the system to
new styles. But sometimes users will ask questions on the mailing list that are
not answered here. Feel free to come up with additional examples that can be
added to the test suite, or when we consider them to be of general use, to this
manual.

\stopsubject

\startsubject[title=Hyperlinks]

Please note \startfootnote Footnotes are placed at the end of each chapter.
\stopfootnote that this document contains hyperlinks that are not highlighted for
aesthetic reasons. In addition to standard interaction (table of contents, index,
and cross|-|references, some external web sources are selectable.

\stopsubject

\stopchapter

\stopcomponent
