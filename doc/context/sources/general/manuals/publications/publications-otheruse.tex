\pushoverloadmode

\environment publications-style

\startcomponent publications-otheruse

\startchapter
   [reference=ch:duane,
    title=Other use of datasets]

Because a bibliography is just a kind of database, you can use the publications
mechanism for other purposes as well.

\startsection
    [title=Images]

During the re|-|implementation \name {Miklavec}{M.}Mojca came up with the
following definitions:

\startbuffer
\startbuffer[duane]
@IMAGE {tug2013,
    title       = "TUG 2013",
    url         = "https://tug.org/tug2013/",
    url_image   = "https://tug.org/tug2013/tug2013-color-300.jpg",
    url_thumb   = "https://tug.org/tug2013/t2013-thumb.jpg",
    description = "Official drawing of the TUG 2013 conference",
    author      = "Duane Bibby",
    year        = 2013,
    copyright   = "TUG",
}

@IMAGE {tug2014,
    title       = "TUG 2014",
    url         = "https://tug.org/tug2014/",
    url_image   = "https://tug.org/art/tug2014-color.jpg",
    url_thumb   = "https://tug.org/tug2014/t2014-thumb.jpg",
    description = "Official drawing of the TUG 2014 conference",
    author      = "Duane Bibby",
    year        = 2014,
    copyright   = "TUG",
}
\stopbuffer
\stopbuffer

\typeBTXbuffer \name{Bibby}{D.}\getbuffer

For documentation purposes we can have a definition in a buffer so that we can
show it verbatim but also load it. The following code defines a dataset, loads
the buffer and sets up a rendering.

\startbuffer
\definebtxdataset
  [duane]

\usebtxdataset
  [duane]
  [duane.buffer]

\definebtx
  [duane]
  [default=,
   specification=duane,
   authorconversion=normal]

\definebtx
  [duane:list]
  [duane]

\definebtx
  [duane:list:author]
  [duane:list]

\definebtx
  [duane:list:image]
  [duane:list]

\definebtxrendering
  [duane]
  [specification=duane,
   group=duane, % so not: group=default,
   dataset=duane,
   method=dataset,
   numbering=no,
   criterium=all]
\stopbuffer

\cindex{definebtxdataset}
\cindex{usebtxdataset}
\cindex{definebtx}
\cindex{definebtxrendering}

\typeTEXbuffer \getbuffer

Instead of for instance \TEXcode {apa} we create a specification named \TEXcode
{duane}. Because categories are rendered with a setup, we can define the
following:

\startbuffer
\startsetups btx:duane:list:image
  \tbox \bgroup
    \bTABLE[offset=1ex]
      \bTR
        \bTD[ny=4]
          \dontleavehmode
          \externalfigure[\btxflush{url_thumb}][width=3cm]
        \eTD
        \bTD \btxflush{title} \eTD
      \eTR
      \bTR
        \bTD \btxflush{author} \eTD
      \eTR
      \bTR
        \bTD \btxflush{description} \eTD
      \eTR
      \bTR
        \bTD
          \goto{high res variant}[url(\btxflush{url_image})]
        \eTD
      \eTR
    \eTABLE
  \egroup
\stopsetups

\placebtxrendering[duane][criterium=all]
\stopbuffer

\cindex{startsetups}
\cindex{stopsetups}
\cindex{placebtxrendering}
\cindex{btxflush}

\typeTEXbuffer \getbuffer

An alternative rendering is:

\startbuffer
\startsetups btx:duane:list:image
  \bgroup
    \bTABLE[offset=1ex]
      \bTR
        \bTD
          \dontleavehmode
          \goto{\externalfigure[\btxflush{url_thumb}][width=3cm]}
            [url(\btxflush{url_image})]
        \eTD
        \bTD
          \bold{\btxflush{title}}
          \blank
          \btxflush{description}
          \blank [3*line]
          \btxflush{author}
        \eTD
      \eTR
    \eTABLE
  \egroup
\stopsetups

\placebtxrendering[duane]
\stopbuffer

\cindex{startsetups}
\cindex{stopsetups}
\cindex{placebtxrendering}
\cindex{btxflush}

\typeTEXbuffer \getbuffer

We only get the second rendering because we specified \TEXcode {criterium} as
\TEXcode {all}. Future version of \CONTEXT\ will probably provide \Index{sorting}
options and ways to plug in additional functionality.

\stopsection

\startsection
    [title=Chemistry]

To give further ideas on how the \quote {publications} database system can be
effectively used in a document, consider any object such as an image (as seen
above), an equation, a chemical compound, or other entity having various
properties and that might be used repeatedly. Here we will give an almost
realistic example of a dataset containing various chemical compounds.

Like above, we shall define a named dataset and load it with definitions having
nothing to do with publications, rather a number of chemical compounds of
formula \chemical {C_6H_6O} (there are some 57 such molecules).

\startbuffer
\definebtxdataset
  [chemistry]

\usebtxdataset
  [chemistry]
  [C6H6O.bib]
\stopbuffer

\cindex{definebtxdataset}
\cindex{usebtxdataset}
\typebuffer [option=TEX]
\getbuffer

We can visualize the contents of the dataset using, for example:

\startbuffer
\showbtxdatasetcompleteness
  [dataset=chemistry,
   specification=chemistry]
\stopbuffer

\cindex{showbtxdatasetcompleteness}
\typebuffer [option=TEX]

Note that the specification \quote {chemistry} is undefined (as yet) so the log file shows:

\starttyping
publications    > no data definition file 'publ-imp-chemistry.lua' for 'chemistry'
\stoptyping

This is harmless as a missing (lua) definition file makes \emphasis {all} data categories and fields
\quote {optional}, allowing their access and use.

The dataset (containing just 4 of the 57 entries) is:

\startparagraph [style=small]
\getbuffer
\stopparagraph

The next step is to activate the specification and dataset and to define a list rendering, as
follows:

\startbuffer
\setupbtx
  [specification=chemistry,
   dataset=chemistry]

\definebtxrendering
  [chemistry]
  [specification=chemistry,
   group=chemistry,
   dataset=chemistry]
\stopbuffer

\cindex{setupbtx}
\cindex{definebtxrendering}
\typebuffer [option=TEX]
\getbuffer

This list rendering can then be placed, but first one needs to create a setup
describing how to handle the entry (or category) \type {@chemical{}} contained
in the file \type {C6H6O.bib}. As in the earlier example, we shall use a table
structure for the list text:

\startbuffer
\startsetups btx:chemistry:list:chemical
  \startframed [frame=off,location=middle]
  \bTABLE [frame=off]
    \bTR
      \bTD [width=.2\textwidth,ny=2]
        \goto
        {\startchemical [scale=small]
           \chemical [\btxflush{chemical}]
         \stopchemical} [program(\btxflush{rendered})]
      \eTD
      \bTD [width=.2\textwidth] \chemical{\btxflush{formula}} \eTD
      \bTD [width=.4\textwidth] \btxflush{iupac} \eTD
    \eTR
    \bTR
      \bTD \btxflush{mw}~g/mole \eTD
      \bTD \btxflush{cas} \eTD
    \eTR
  \eTABLE
  \stopframed
\stopsetups

\placebtxrendering [method=dataset]
\stopbuffer

\cindex{startsetups}
\cindex{stopsetups}
\typebuffer [option=TEX]

Note that not all fields of the dataset are used.

\getbuffer

The real usefulness of the dataset is the possibility of referring to entries
and the ability to extract data fields. Setups can be defined for those fields
requiring any formatting beyond a simple flushing, for example those requiring
the \Cindex {chemical} command.

\startbuffer
\startsetups btx:chemistry:cite:formula
  \chemical {\currentbtxfirst}
\stopsetups

\startsetups btx:chemistry:cite:chemical
  \startchemical [scale=600]
    \chemical [\currentbtxfirst]
  \stopchemical
\stopsetups

\cite [name]        [108-95-2] |<|or|>|
\cite [formula]     [108-95-2] |<|or|>|
\cite [chemical]    [108-95-2] |<|or|>|
\cite [molarvolume] [108-95-2]
\stopbuffer

\cindex{startsetups}
\cindex{stopsetups}
\cindex{cite}
\typebuffer [option=TEX]
\getbuffer

\startbuffer
\cite [name]        [566-33-5] |<|or|>|
\cite [formula]     [566-33-5] |<|or|>|
\cite [chemical]    [566-33-5] |<|or|>|
\cite [molarvolume] [566-33-5]
\stopbuffer

\cindex{cite}
\typebuffer [option=TEX]
\getbuffer

\stopsection

\startsection[title=\METAPOST]

Because it's a variant on previous examples, we just show the whole set
of definitions and the result.

\startbuffer
\startbuffer[sample]
\startMPpage
    fill fullcircle scaled 4cm withcolor "darkgreen";
\stopMPpage
\stopbuffer

\savebuffer[list={sample},file={metafun-sample-2.tex},prefix=]

\startbuffer[samples]
@IMAGE {sample1,
    title       = "sample-1",
    author      = "My Self",
    mpcode      = {fill fullcircle scaled 4cm withcolor "darkblue";}
}

@IMAGE {sample2,
    title       = "sample-2",
    author      = "My Self",
    mpfile      = "metafun-sample-2.tex"
}
\stopbuffer

\definebtxdataset
  [samples]

\usebtxdataset
  [samples]
  [samples.buffer]

\definebtx
  [samples]
  [default=,
   specification=samples]

\definebtx
  [samples:list:image]
  [samples:list]

\definebtxrendering
  [samples]
  [specification=samples,
   group=samples,
   dataset=samples,
   method=dataset,
   numbering=no,
   criterium=all]

\startsetups btx:samples:list:image
  \tbox \bgroup
    \bTABLE[offset=1ex]
      \bTR
        \bTD[ny=4]
          \dontleavehmode
          \doifelsesomething{\btxflush{mpfile}} {
            \externalfigure[\btxflush{mpfile}]
          } {
            \MPcode{
                \btxflush{mpcode}
            }
          }
        \eTD
        \bTD
            \btxflush{title}
        \eTD
      \eTR
    \eTABLE
  \egroup
\stopsetups

\placebtxrendering[samples][criterium=all]
\stopbuffer

\typebuffer[option=TEX]

\getbuffer

\stopsection

\stopchapter

\stopcomponent
