% language=us

% \enabletrackers[metapost.svg.result]
% \enabletrackers[metapost.svg.path]

\enabledirectives[pdf.compact]

% \nopdfcompression

\def\showSVGcode#1#2%
  {\ctxlua{metapost.showsvgpage{
        filename = "#1",
        index    = tonumber(#2),
        method   = "code"
    }}}

\def\showSVGcodeG#1#2%
  {\ctxlua{metapost.showsvgpage{
        filename = "#1",
        index    = tonumber(#2),
        method   = "code",
        x        = 0,
        y        = 1000,
        width    = 1000,
        height   = 1000,
        noclip   = true
    }}}

\def\typeSVGcode#1#2%
  {\ctxlua{metapost.typesvgpage{
        filename = "#1",
        index    = tonumber(#2)
    }}}

\usemodule[abbreviations-logos]
\usemodule[scite]

\setuppapersize
  [A4,landscape]

\setuplayout
  [backspace=1cm,
   cutspace=2cm,
   topspace=1cm,
   width=middle,
   height=middle,
   header=0pt,
   footer=0pt]

\setupbodyfont
  [dejavu,12pt]

\setupwhitespace
  [big]

\setuphead
  [chapter]
  [style=\bfc,
  color=darkblue]

\setuphead
  [section]
  [style=\bfb,
   color=darkblue]

\starttext

\startMPpage
    drawlineoptions   (withpen pencircle scaled 0.4 withcolor darkgray) ;
    drawpointoptions  (withpen pencircle scaled 0.8 withcolor darkred) ;
    drawcontroloptions(withpen pencircle scaled 0.6 withcolor darkgreen) ;
    drawpathoptions   (withpen pencircle scaled 1.0 withcolor darkblue) ;

    drawoptionsfactor := .5 ;
    detailpaths ;

    StartPage ;

        fill Page withcolor darkblue ;

        draw lmt_svg [
            filename = "svglogo.svg",
            origin   = true,
        ]
            ysized (PaperHeight -64mm)
            shifted (12mm,52mm) ;

        draw textext.llft ("\strut in context")
            ysized 34mm
            shifted lrcorner Page
            shifted (-8mm,84mm)
            withcolor white ;

        draw textext.llft ("\strut and metafun xl")
            ysized 34mm
            shifted lrcorner Page
            shifted (-8mm,44mm)
            withcolor white ;


    StopPage ;
\stopMPpage

\starttitle[title=Introduction]

This document is about using \SVG, an \XML\ based format for describing graphics
and colorful font shapes in \CONTEXT. It's one of the external figure formats.
Where we can use \METAPOST\ for all kind of systematic graphics, bitmap images
and artistic outlines come from outside. Inclusion of \SVG\ using the methods
discussed here is quite efficient and will work for many graphics, but when it
doesn't you can always fall back on a conversion by \INKSCAPE. It's work in
progress anyway.

The document is made for viewing on the screen and has a bunch of examples taken
from websites. We might add more in due time. The cover page has the \SVG\ logo
taken from \WIKIPEDIA\ but with some details added. It's not a nice cover image
but it will do for our purpose. Feel free to suggest additional examples.

\startlines
Hans Hagen
Hasselt NL
October 2019\high{+}
\stoplines

\stoptitle

\starttitle[title=The \SVG\ format]

\startsection[title=What it is]

The Scalable Vector Graphics format (\SVG) showed up around the turn of this
century. I remember looking into it and wondering to what extent it was a fresh
development and not some kind of application format turned \XML. Most elements
are empty elements and data lives in attributes. What I found most puzzling is
that a path definition was an attribute and not just content, especially because
it can be a pretty large blob of numbers and commands. Anyway, at that time I
played a bit with conversion but in the end decided to just consider it an
external format for which conversion to (say) \PDF\ by an external program was a
reasonable. At some point that external program became \INKSCAPE\ and \CONTEXT\
uses that to convert \SVG\ images to \PDF\ runtime (with caching).

In the meantime edition one turned edition two and the advance of \HTML\ and
\CSS\ has crept features into the format, thereby not making it look better. But,
because viewers support rendering \SVG, we now also see graphics showing up. The
ones that I have to deal with are educational graphics, and when you look into
the files, they can be curiously inconsistent in the way parts of graphics are
made. For instance, the numbers along an axis of a mathematical graphic can be a
mix of references to a font (\type {<text/>}), references to symbols \type
{<symbol/>} that have paths (\type {<path/>}) or just paths \type {<path/>}.
Using a tool that can spit out something structured doesn't mean that all its
users will structure.

The \SVG\ format provides lines, rectangles, circles, ellipses, polylines,
polygons and paths. Paths are defines as a sequence of moves, lines, cubic and
quadratic curves, arcs, collected in the \type {d} attribute (a funny short name
compared to the length of its content and the verbosity of other attribute
names). They can be open or closed, and use different winding rules. Positions
are absolute or relative. This all leaves a lot of room for error and confusion.
When a path looks bad, it can be produced bad, or the interpretation can be bad.
Interpretation can even be such that errors are catched which makes it hard to
figure out what is really wrong. And as usual, bugs (and supposed catches) can
become features in the end. So it might take a while before this kind of support
in \CONTEXT\ becomes stable but once it is, normally we're okay for a while. And,
one nice side effect of \XML\ is that it can't really crash processing as it's
just data.

\stopsection

\startsection[title=Color fonts]

Then color fonts showed up in \OPENTYPE\ and \SVG\ is one of the used
sub|-|formats in that. Again it was convenient enough to rely on \INKSCAPE\ to do
the conversion to \PDF\ blobs, but after a while I decided that a more native
(built|-|in) support start making sense. A lot had happened since 2000, most
noticeably the arrival of \LUATEX\ and \CONTEXT\ \MKIV\ followed by \LUAMETATEX\
and \CONTEXT\ \LMTX, so a more direct support because more feasible. A more
direct support has the advantage that we don't need to call an external program
and cache the results (think of Emoji fonts with thousands of glyphs in \SVG\
format). Also, direct conversion makes it possible to tweak colors and such,
simply because the data goes through the \CONTEXT\ internals as part of the
typesetting process. So, as a prelude to the \CONTEXT\ 2019 meeting a preliminary
converter was made, color font support was partially redone, and afterward the
converter got completed to the level needed for embedding more fancy graphics,
including relabeling.

\stopsection

\startsection[title=In practice]

In the end all is about paths or glyphs, plus some optional clipping and
transformations. The rendering is controlled by attributes: color, transparency,
line thickness, the way lines join and end, etc. Now, in the original
specification that was done only with attributes, which is a clean and robust way
of doing it, but later styles and classes were introduced and we now have a whole
chain to consider when resolving a to be used attribute.

\startitemize[packed]
    \startitem attributes explicitly set by keys to an element \stopitem
    \startitem attributes set in the \type {style} attribute \stopitem
    \startitem attributes set via one or more \type {class} assignments \stopitem
    \startitem attributes set for the specific element \stopitem
    \startitem attributes inherited from an ancestor (somewhat vague) \stopitem
    \startitem redundant (nested) attributes (text styling) \stopitem
\stopitemize

Where examples are often hand codes and therefore look ok, graphics that get
generated can look quite horrible: the same parameters being set with different
methods, even inconsistently, to mention one. But also, graphics can be read in,
tweaked and saved again which in itself generates artifacts, etc. One can of
course argue that \XML\ is not for human consumption but personally I tend to
conclude that when a source file looks bad, the likelyhood is great that what it
encodes looks bad too. And for instance \INKSCAPE\ provides ways to inspect and
tweak the \XML\ in the editor.

\stopsection

\startsection[title=The conversion]

This brings us to the conversion. As we need \PDF\ operators one method is to
directly go from \SVG\ to \PDF. There is the issue of fonts, but as we delegate
that to \TEX\ anyway, because that is kind of an abstraction. Such a conversion
is comparable with going from \METAPOST\ to \PDF. However, for practical reasons
an intermediate step has been chosen: we go from \SVG\ to \METAPOST\ first. This
has the benefit that we need little code for color and transparency because
\METAPOST\ (read: \METAFUN) already deals with that. We also don't need that much
for text, as we deal with that in \METAPOST\ too, and that way we can even
overload and reposition for instance labels in graphics relatively easy.

Another advantage of the intermediate step is that we can combine \SVG\ graphics
with \METAPOST\ code. Of course we can already combine external graphics with
\METAPOST, but there is a big advantage here: the output is quite efficient. When
we transform paths and pens in \METAPOST, the end result is often just a path,
but where we to do a direct conversion to \PDF, we would either have to do
calculations on our own, or we would have to use lots of transformation
directives. In the end, especially because \METAPOST\ is so fast, the indirect
route pays of well (and I haven't even optimized it yet).

\stopsection

\startsection[title=Remark]

In the perspective if using \TEX\ and \METAPOST\ it makes sense to keep an eye on
consistency. You can make quite structured \SVG\ images if you want to. When you
use a graphical editor you can even consider using a normal text editor to clean
up the code occasionally. The cleaner the code, the more predictable the outcome
will become. Looking at the code might also give an impression of what features
not to use of use differently. Of course this makes most sense in situations
where there are many graphics and long|-|term (re)use is needed.

\stopsection

\stoptitle

\starttitle[title=Embedding graphics]

\startsection[title=External figures]

At least for now, the default \SVG\ inclusions is done via an external converter
but you can use the internal one by specifying a conversion. The next example
demonstrates that it works like any external figure:

\startbuffer
\startcombination[4*1]
    {\externalfigure[mozilla-tiger.svg][conversion=mp]}                      {1}
    {\externalfigure[mozilla-tiger.svg][conversion=mp,height=1cm]}           {2}
    {\externalfigure[mozilla-tiger.svg][conversion=mp,height=3cm,width=1cm]} {3}
    {\externalfigure[mozilla-tiger.svg][conversion=mp,height=1cm,width=8cm]} {4}
\stopcombination
\stopbuffer

\typebuffer[option=TEX]

We get:

\startlinecorrection
    \getbuffer
\stoplinecorrection

\stopsection

\startsection[title=Internal figures]

You can put some \SVG\ code in a buffer:

\startbuffer
\startbuffer[svgtest]
    <svg>
        <rect
            x="0" y="0" width="80" height="20"
            fill="blue" stroke="red" stroke-width="3"
            stroke-linejoin="miter"
            transform="rotate(10)"
        />
    </svg>
\stopbuffer
\stopbuffer

\typebuffer[option=TEX] \getbuffer

In the future more options might be added but for now there's only an offset
possible:

\startbuffer
\startcombination[2*1]
    {\framed[offset=overlay]{\includesvgbuffer[svgtest]}}             {default}
    {\framed[offset=overlay]{\includesvgbuffer[svgtest][offset=2bp]}} {some offset}
\stopcombination
\stopbuffer

\typebuffer[option=TEX] \getbuffer

There is a companion command \type {\includesvgfile} which accepts a filename
and also supports offsets. Sometimes the offset is needed to prevent unwanted
clipping.

\stopsection

\startsection[title=Mixing in \METAFUN]

An \SVG\ image can be directly included in an \METAFUN\ image. This makes it
possible to enhance (or manipulate) such an image, as in:

\startbuffer
\startMPcode
    draw lmt_svg [
        filename = "mozilla-tiger.svg",
        origin   = true,
    ] rotated 45 slanted .75 ysized 2cm ;

    setbounds currentpicture to
        boundingbox currentpicture
        enlarged 1mm ;

    addbackground
        withcolor "darkgray" ;
\stopMPcode
\stopbuffer

\typebuffer[option=TEX]

An \SVG\ image included this way becomes a regular \METAPOST\ picture, so a
collection of paths. Because \METAPOST\ on the average produces rather compact
output the \SVG\ image normally also is efficiently embedded. You don't need to
worry about loosing quality, because \METAPOST\ is quite accurate and we use so
called \quote {double} number mode anyway.

\startlinecorrection
    \getbuffer
\stoplinecorrection

Another trick is to inline the code:

\startbuffer
\startMPcode
    draw svg "<svg>
        <circle
            cx='50' cy='50' r='40'
            stroke='green' stroke-width='10' stroke-opacity='0.3'
            fill='red' fill-opacity='0.3'
        />
        <circle
            cx='150' cy='50' r='40'
            stroke='green' stroke-width='10'
            fill='red'
            opacity='0.3'
        />
    </svg>" ;
\stopMPcode
\stopbuffer

It doesn't really make sense as \METAPOST\ code is just as simple but
it looks cool:

\startlinecorrection
    \getbuffer
\stoplinecorrection

And actually it's less code (which internally of course expands to
more):

\startbuffer
\startMPcode
    pickup pencircle scaled 10;
    path p ; p := fullcircle scaled 80 ;
    fill p shifted (50,50) withcolor blue
        withtransparency(1,0.3) ;
    draw p shifted (50,50) withcolor yellow
        withtransparency(1,0.3) ;
    draw image (
        fill p shifted (150,50) withcolor blue ;
        draw p shifted (150,50) withcolor yellow ;
        setgroup currentpicture to boundingbox currentpicture
            withtransparency(1,0.3) ;
    ) ;
\stopMPcode
\stopbuffer

\typebuffer[option=TEX]

It's all a matter of taste. Watch the grouping trick!

\startlinecorrection
    \getbuffer
\stoplinecorrection

\stopsection

\startsection[title=Fonts]

{\em This is still experimental.}

\stopsection

\startsection[title=Labels]

{\em This is still experimental.}

\stopsection

\startsection[title=Tracing]

{\em This is still experimental.}

\stopsection

\stoptitle

\starttitle[title=Mozilla test snippets]

The Mozilla documentation pages for \SVG\ are pretty good and contain snippets
that can be used for testing. More examples might be added in due time.

\dorecurse{38}{
    \page
    \startsection[title=Snippet #1]
        \framed
          [offset=overlay]
          {\scale[height=4cm]{\showSVGcode{svg-lmtx-mozilla.lua}{#1}}}
        \blank
        \start
            \switchtobodyfont[10pt]
            \setupalign[flushleft,verytolerant,broad]
            \typeSVGcode{svg-lmtx-mozilla.lua}{#1}
            \par
        \stop
    \stopsection
    \page
}

\stoptitle

\starttitle[title=Microsoft test snippets]

These snippets come from the \MICROSOFT\ typography pages that discuss \OPENTYPE\
and \SVG. Because these are actually examples of glyphs, we need to set some
defaults:

\starttabulate[|cT|rT|]
\NC x      \NC    0 \NC \NR
\NC y      \NC 1000 \NC \NR
\NC width  \NC 1000 \NC \NR
\NC height \NC 1000 \NC \NR
\stoptabulate

in order to get the right placement. This has to do with the fact that the
vertical \SVG\ coordinates go in the other direction compared to \METAPOST\ and
\PDF.

\dorecurse{8}{
    \page
    \startsection[title=Snippet #1]
        \framed
          [offset=overlay]
          {\scale[height=4cm]{\showSVGcodeG{svg-lmtx-microsoft.lua}{#1}}}
        \blank
        \start
            \switchtobodyfont[10pt]
            \setupalign[flushleft,verytolerant,broad]
            \typeSVGcode{svg-lmtx-microsoft.lua}{#1}
            \par
        \stop
        \page
    \stopsection
    \page
}

\stoptitle

\starttitle[title=Xah Lee test snippets]

These snippets come from the \type {http://xahlee.info/js/svg_path_spec.html},
which gives a nice overview of \SVG. Not all examples are here. There are some
nice interactive examples there plus info about using fonts.

\dorecurse{38}{
    \page
    \startsection[title=Snippet #1]
        \framed
          [offset=overlay]
          {\scale[height=4cm]{\showSVGcodeG{svg-lmtx-xahlee.lua}{#1}}}
        \blank
        \start
            \switchtobodyfont[10pt]
            \setupalign[flushleft,verytolerant,broad]
            \typeSVGcode{svg-lmtx-xahlee.lua}{#1}
            \par
        \stop
        \page
    \stopsection
    \page
}

\stoptitle

\starttitle[title=Our own snippets]

These snippets were made as part if testing. I had some 1500 \SVG\ graphics that
internally were quite messy (it's surprising what some applications export) so I
sometimes had to extract bits and pieces and make my own tests to figure out how
to deal with it.

\dorecurse{2}{
    \page
    \startsection[title=Snippet #1]
        \framed
          [offset=overlay]
          {\scale[height=4cm]{\showSVGcode{svg-lmtx-context.lua}{#1}}}
        \blank
        \start
            \switchtobodyfont[10pt]
            \setupalign[flushleft,verytolerant,broad]
            \typeSVGcode{svg-lmtx-context.lua}{#1}
            \par
        \stop
        \page
    \stopsection
    \page
}

\stoptitle

\stoptext

% After some contemplating, while listening to Benmont Tench's solo album (2014),
% after first listening to a nice long interview, which I hit after following some
% Hammond links, I finally decided that it made sense to write this manual. Life is
% too short for delays.
