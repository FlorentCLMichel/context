% language=us runpath=texruns:manuals/metafun
%
% copyright=pragma-ade readme=readme.pdf licence=cc-by-nc-sa

\startcomponent metafun-examples

\environment metafun-environment

\startchapter[title={A few applications}]

\startintro

For those who need to be inspired, we will demonstrate how \METAPOST\ can be used
to enhance your document with simple graphics. In these examples we will try to
be not too clever, simply because we lack the experience to be that clever. The
real tricks can be found in the files that come with \METAPOST.

\stopintro

\startsection[reference={sec:coils,sec:springs},title={Simple drawings}]

In the words of John Hobby, the creator of \METAPOST, \quotation {\METAPOST\ is
particularly well||suited for generating figures for technical documents where
some aspects of a picture may be controlled by mathematical or geometrical
constraints that are best expressed symbolically. In other words, \METAPOST\ is
not meant to take the place of a freehand drawing tool or even an interactive
graphics editor}.

An example of such a picture is the following one, which is dedicated to David
Arnold, who asked me once how to draw a spring. So, imagine that we want to draw
a schematic view of a system of four springs.

\startbuffer[a]
def spring (expr a, b, w, h, n) =
  ( ( (0,0) -- (0,h) --
      for i=1 upto n-1: (if odd(i) : - fi w/2,i+h) -- endfor
      (0,n+h) -- (0,n+2h) )
    yscaled ((xpart (b-a) ++ ypart (b-a))/(n+2h))
    rotatedaround(origin,-90+angle(b-a))
    shifted a )
enddef ;
\stopbuffer

\startbuffer[b]
vardef spring (expr a, b, w, h, n) =
  pair vec ; path pat ; numeric len ; numeric ang ;
  vec := (b-a) ;
  pat := for i=1 upto n-1: (if odd(i):-fi w/2,i)--endfor (0,n) ;
  pat := (0,0)--(0,h)-- pat shifted (0,h)--(0,n+h)--(0,n+2h) ;
  len := (xpart vec ++ ypart vec)/(n+2h) ;
  ang := -90+angle(vec) ;
  ( pat yscaled len rotatedaround(origin,ang) shifted a )
enddef ;
\stopbuffer

\startbuffer[c]
path p ; p :=
  (0,0)--spring((.5cm,0),(2.5cm,0),.5cm,0,10)--(3cm,0) ;

draw p withpen pencircle scaled 2pt ;
draw p withpen pencircle scaled 1pt withcolor .8white;
\stopbuffer

\startbuffer[d]
z1 = (+2cm,0) ; z2 = (0,+2cm) ;
z3 = (-2cm,0) ; z4 = (0,-2cm) ;

pickup pencircle scaled 1.5pt ;

drawoptions (withcolor .625red) ;

draw spring (z1, z2, .75cm, 2, 10) ; draw z1 -- 1.5 z1 ;
draw spring (z2, z3, .75cm, 2,  9) ; draw z2 -- 1.1 z2 ;
draw spring (z3, z4, .75cm, 2,  8) ; draw z3 -- 1.5 z3 ;
draw spring (z4, z1, .75cm, 2,  7) ; draw z4 -- 1.1 z4 ;
\stopbuffer

\startbuffer[e]
drawarrow
  (0,0)--spring((.5cm,0),(2.5cm,0),.5cm,0,10)--(3cm,0)
  withpen pencircle scaled 2pt withcolor .625red ;
\stopbuffer

\startbuffer[f]
numeric u ; u := 1mm ; pickup pencircle scaled (u/2) ;
drawoptions (withcolor .625red) ;
draw (0,0)--spring((5u,0),(25u,0),5u,0,10)--(30u,0) ;
drawoptions (dashed evenly withcolor .5white) ;
draw (0,0)--spring((5u,0),(35u,0),(25/35)*5u,0,10)--(40u,0) ;
\stopbuffer

\startlinecorrection[blank]
\processMPbuffer[a,d]
\stoplinecorrection

A rather natural way to define such a system is:

\typebuffer[d]

Here, the macro \type {spring} takes 5~arguments: two points, the width of the
winding, the length of the connecting pieces, and the number of elements (half
windings). The definition of \type {spring} is less complicated than readable.

\typebuffer[a]

First we build a path starting in the origin, going left or right depending on
the counter being an odd number.

\starttyping
pat := (0,0) ;
for i=1 upto n-1:
  if odd(i) :
    pat := pat -- (-w/2,i) ;
  else :
    pat := pat -- (+w/2,i) ;
  fi ;
endfor ;
pat := pat -- (0,n) ;
\stoptyping

Once you are accustomed to the way \METAPOST\ interprets (specialists may say
expand) the source code, you will start using \type {if} and \type {for}
statements in assignments. The previous code can be converted in a one liner,
using the pattern:

\starttyping
pat := for i=1 upto n-1: (x,y)-- endfor (0,n) ;
\stoptyping

The loop splits out a series of \type {(x,y)--} but the last point is added
outside the loop. Otherwise \type {pat} would have ended with a dangling \type
{--}. Of course we need to replace \type {(x,y)} by something meaningful, so we
get:

\starttyping
pat := for i=1 upto n-1: (if odd(i):-fi w/2,i)--endfor (0,n) ;
\stoptyping

We scale this path to the length needed. The expression $b-a$ calculates a
vector, starting at $a$ and ending at $b$. In \METAPOST, the expression \type
{a++b} is identical to $\sqrt{a^2+b^2}$. Thus, the expression \typ {xpart (b-a)
++ ypart (b-a)} calculates the length of the vector $b-a$. Because the unscaled
spring has length $n+2h$, scaling by the expression \typ {((xpart (b-a) ++ ypart
(b-a)) / (n+2h))} gives the spring the same length as the vector $b-a$.

Because we have drawn our spring in the vertical position, we first rotate it 90
degrees clockwise to a horizontal position, and then rotate it through an angle
equal to the angle in which the vector $b-a$ is pointing. After that, we shift it
to the first point. The main complications are that we also want to draw
connecting lines at the beginning and end, as well as support springs that
connect arbitrary points. Since no check is done on the parameters, you should be
careful in using this macro.

When we want to improve the readability, we have to use intermediate variables.
Since the macro is expected to return a path, we must make sure that the content
matches this expectation.

\typebuffer[b]

If you use \type {vardef}, then the last statement is the return value. Here,
when \typ {p := spring (z1, z2, .75cm, 2, 10)} is being parsed, the macro is
expanded, the variables are kept invisible for the assignment, and the path at
the end is considered to be the return value. In a \type {def} the whole body of
the macro is \quote {pasted} in the text, while in a \type {vardef} only the last
line is visible. We will demonstrate this with a simple example.

\starttyping
def one = (n,n) ; n := n+1 ; enddef ;
def two = n := n + 1 ; (n,n) enddef ;
\stoptyping

Now, when we say:

\starttyping
pair a, b ; numeric n ; n= 10 ; a := one ; b := two ;
\stoptyping

we definitely get an error message. This is because, when macro \type {two} is
expanded, \METAPOST\ sees something:

\starttyping
b := n := n + 1 ;
\stoptyping

By changing the second definition in

\starttyping
vardef two = n := n + 1 ; (n,n) enddef ;
\stoptyping

the increment is expanded out of sight for \type {b :=} and the pair \type
{(n,n)} is returned.

We can draw a slightly better looking spring by drawing twice with a different
pen. The following commands use the spring macro implemented by the \type
{vardef}.

\typebuffer[c]

This time we get:

\startlinecorrection[blank]
\processMPbuffer[a,c]
\stoplinecorrection

Since the \type {spring} macro returns a path, you can do whatever is possible
with a path, like drawing an arrow:

\startlinecorrection[blank]
\processMPbuffer[a,e]
\stoplinecorrection

Or even (watch how we use the neutral unit \type {u} to specify the dimensions):

\startlinecorrection[blank]
\processMPbuffer[a,f]
\stoplinecorrection

This was keyed in as:

\typebuffer[e]

and:

\typebuffer[f]

\stopsection

\startsection[reference=sec:free labels,title={Free labels}]

\index {labels}

The \METAPOST\ label macro enables you to position text at certain points. This
macro is kind of special, because it also enables you to influence the
positioning. For that purpose it uses a special kind of syntax which we will not
discuss here in detail.

\startbuffer[a]
pickup pencircle scaled 1mm ;
path p ; p := fullcircle scaled 3cm ;
draw p withcolor .625yellow ;
dotlabel.rt   ("right"       , point 0 of p) ;
dotlabel.urt  ("upper right" , point 1 of p) ;
dotlabel.top  ("top"         , point 2 of p) ;
dotlabel.ulft ("upper left"  , point 3 of p) ;
dotlabel.lft  ("left"        , point 4 of p) ;
dotlabel.llft ("lower left"  , point 5 of p) ;
dotlabel.bot  ("bottom"      , point 6 of p) ;
dotlabel.lrt  ("lower right" , point 7 of p) ;
\stopbuffer

\typebuffer[a]

The \type {label} command just typesets a text, while \type {dotlabel} also draws
a dot at the position of the label. The \type {thelabel} (not shown here) command
returns a picture.

\startlinecorrection[blank]
\processMPbuffer[a]
\stoplinecorrection

There is a numeric constant \type {labeloffset} that can be set to influence the
distance between the point given and the content of the label. When we set the
offset to zero, we get the following output.

\startbuffer[x]
interim labeloffset := 0pt ; % local assignment
\stopbuffer

\startlinecorrection[blank]
\processMPbuffer[x,a]
\stoplinecorrection

This kind of positioning works well as long as we know where we want the label to
be placed. However, when we place labels automatically, for instance in a macro,
we have to apply a few clever tricks. There are for sure many ways to accomplish
this goal, but here we will follow the mathless method.

\startbuffer[a]
pickup pencircle scaled 1mm ;
path p ; p := fullcircle scaled 3cm ;
draw p withcolor .625yellow ;
vardef do (expr str) =
  save currentpicture ; picture currentpicture ;
  currentpicture := thelabel(str,origin) ;
  draw boundingbox currentpicture withpen pencircle scaled .5pt ;
  currentpicture
enddef ;
\stopbuffer

\startbuffer[b]
dotlabel.rt   (do("right")       , point 0 of p) ;
dotlabel.urt  (do("upper right") , point 1 of p) ;
dotlabel.top  (do("top")         , point 2 of p) ;
dotlabel.ulft (do("upper left")  , point 3 of p) ;
dotlabel.lft  (do("left")        , point 4 of p) ;
dotlabel.llft (do("lower left")  , point 5 of p) ;
dotlabel.bot  (do("bottom")      , point 6 of p) ;
dotlabel.lrt  (do("lower right") , point 7 of p) ;
\stopbuffer

\startlinecorrection[blank]
\processMPbuffer[x,a,b]
\stoplinecorrection

The previous graphic visualizes the bounding box of the labels. This bounding box
is rather tight and therefore the placement of labels will always be suboptimal.
Compare the alignment of the left- and rightmost labels. The \type {btex}||\type
{etex} method is better, since then we can add struts, like:

\starttyping
btex \strut right etex
\stoptyping

to force labels with uniform depths and heights. The next graphic demonstrates
that this looks better indeed. Also, as \TEX\ does the typesetting we get the
current text font instead of the label font and the content will be properly
typeset; for instance kerning will be applied when applicable. Spending some time
on such details pays back in better graphics.

\startbuffer[b]
dotlabel.rt   (do(btex \strut right       etex) , point 0 of p) ;
dotlabel.urt  (do(btex \strut upper right etex) , point 1 of p) ;
dotlabel.top  (do(btex \strut top         etex) , point 2 of p) ;
dotlabel.ulft (do(btex \strut upper left  etex) , point 3 of p) ;
dotlabel.lft  (do(btex \strut left        etex) , point 4 of p) ;
dotlabel.llft (do(btex \strut lower left  etex) , point 5 of p) ;
dotlabel.bot  (do(btex \strut bottom      etex) , point 6 of p) ;
dotlabel.lrt  (do(btex \strut lower right etex) , point 7 of p) ;
\stopbuffer

\startlinecorrection[blank]
\processMPbuffer[x,a,b]
\stoplinecorrection

Now, what happens when we want to place labels in other positions? In the worst
case, given that we place the labels manually, we end up in vague arguments in
favour for one or the other placement.

\startbuffer[y]
p := p rotatedaround(center p, 22.5) ;
\stopbuffer

\startlinecorrection[blank]
\processMPbuffer[x,a,y,b]
\stoplinecorrection

Although any automatic mechanism will be sub||optimal, we can give it a try to
write a macro that deals with arbitrary locations. This macro will accept three
arguments and return a picture.

\starttyping
thefreelabel("some string or picture",a position,the origin)
\stoptyping

Our testcase is just a simple \type {for} loop that places a series of labels.
The \type {freedotlabel} macro is derived from \type {thefreelabel}.

\startbuffer[c]
pickup pencircle scaled 1mm ;
path p ; p := fullcircle scaled 3cm ;
draw p withcolor .625yellow ;
for i=0 step .5 until 7.5 :
  freedotlabel ("text" , point i of p, center p) ;
endfor ;
\stopbuffer

\typebuffer[c]

As a first step we will simply place the labels without any correction. We also
visualize the bounding box.

\startbuffer[b]
vardef freedotlabel (expr str, loc, ori) =
  drawdot loc ; draw thefreelabel(str,loc,ori) ;
enddef ;

vardef freelabel (expr str, loc, ori) =
  draw thefreelabel(str,loc,ori) ;
enddef ;
\stopbuffer

\startbuffer[a]
vardef thefreelabel (expr str, loc, ori) =
  save s ; picture s ; s := thelabel(str,loc) ;
  draw boundingbox s withpen pencircle scaled .5pt ;
  s
enddef ;
\stopbuffer

\typebuffer[a]

To make our lives more easy, we also define a macro that draws the dot as well as
a macro that draws the label.

\typebuffer[b]

Now we get:

\startlinecorrection[blank]
\processMPbuffer[x,a,b,c]
\stoplinecorrection

The original label macros permits us to align the label at positions, 4~corners
and 4~points halfway the sides. It happens that circles are also composed of
8~points. Because in most cases the label is to be positioned in the direction of
the center of a curve and the point at hand, it makes sense to take circles as
the starting points for positioning the labels.

To help us in positioning, we define a special square path, \type {freesquare}.
This path is constructed out of 8~points that match the positions that are used
to align labels.

\startbuffer[d]
path freesquare ;

freesquare := ((-1,0)--(-1,-1)--(0,-1)--(+1,-1)--
               (+1,0)--(+1,+1)--(0,+1)--(-1,+1)--cycle) scaled .5 ;
\stopbuffer

\typebuffer[d]

We now show this free path together with a circle, using the following
definitions:

\startbuffer[e]
drawpath        fullcircle scaled 3cm ;
drawpoints      fullcircle scaled 3cm ;
drawpointlabels fullcircle scaled 3cm ;
currentpicture := currentpicture shifted (5cm,0) ;
drawpath        freesquare scaled 3cm ;
drawpoints      freesquare scaled 3cm ;
drawpointlabels freesquare scaled 3cm ;
\stopbuffer

\typebuffer[e]

We use two drawing macros that are part of the suite of visual debugging macros.

\startlinecorrection[blank]
\processMPbuffer[x,d,e]
\stoplinecorrection

As you can see, point~1 is the corner point that suits best for alignment when a
label is put at point~1 of the circle. We will now rewrite \type {thefreelabel}
in such a way that the appropriate point of the associated \type {freesquare} is
found.

\startbuffer[a]
vardef thefreelabel (expr str, loc, ori) =
  save s, p, q, l ; picture s ; path p, q ; pair l ;
  s := thelabel(str,loc) ;
  p := fullcircle scaled (2*length(loc-ori)) shifted ori ;
  q := freesquare xyscaled (urcorner s - llcorner s) ;
  l := point (xpart (p intersectiontimes (ori--loc))) of q ;
  draw q shifted loc withpen pencircle scaled .5pt ;
  draw l shifted loc withcolor .625yellow ;
  draw loc withcolor .625red ;
  s
enddef ;
\stopbuffer

\typebuffer[a]

The macro xyscaled is part of \METAFUN\ and scales in two directions at once. The
\METAPOST\ primitive \type {intersectiontimes} returns a pair of time values of
the point where two paths intersect. The first part of the pair concerns the
first path.

\startlinecorrection[blank]
\processMPbuffer[x,a,b,c]
\stoplinecorrection

We are now a small step from the exact placement. If we change the last line of
the macro into:

\starttyping
(s shifted -l)
\stoptyping

we get the displacement we want. Although the final look and feel is also
determined by the text itself, the average result is quite acceptable.

\startbuffer[a]
vardef thefreelabel (expr str, loc, ori) =
  save s, p, q, l ; picture s ; path p, q ; pair l ;
  s := thelabel(str,loc) ;
  p := fullcircle scaled (2*length(loc-ori)) shifted ori ;
  q := freesquare xyscaled (urcorner s - llcorner s) ;
  l := point (xpart (p intersectiontimes (ori--loc))) of q ;
  draw q shifted loc withpen pencircle scaled .5pt ;
  draw l shifted loc withcolor .625yellow ;
  draw loc withcolor .625red ;
  (s shifted -l)
enddef ;
\stopbuffer

\startlinecorrection[blank]
\processMPbuffer[x,a,b,c]
\stoplinecorrection

Because we also want to pass pictures, and add a bit of offset too, the final
implementation is slightly more complicated. The picture is handled with an
additional condition, and the offset with the \METAFUN\ macro \type {enlarged}.

\startbuffer[a]
newinternal freelabeloffset ; freelabeloffset := 3pt ;

vardef thefreelabel (expr str, loc, ori) =
  save s, p, q, l ; picture s ; path p, q ; pair l ;
  interim labeloffset := freelabeloffset ;
  s := if string str : thelabel(str,loc)
       else          : str shifted -center str shifted loc fi ;
  setbounds s to boundingbox s enlarged freelabeloffset ;
  p := fullcircle scaled (2*length(loc-ori)) shifted ori ;
  q := freesquare xyscaled (urcorner s - llcorner s) ;
  l := point (xpart (p intersectiontimes (ori--loc))) of q ;
  setbounds s to boundingbox s enlarged -freelabeloffset ;
  (s shifted -l)
enddef ;
\stopbuffer

\typebuffer[a]

Watch how we temporarily enlarge the bounding box of the typeset label text. We
will now test this macro on a slightly rotated circle, using labels typeset by
\TEX. The \type {reverse} is there purely for cosmetic reasons, to suit the label
texts.

\startbuffer[b]
pickup pencircle scaled 1mm ;
path p ; p := reverse fullcircle rotated -25 scaled 3cm ;
draw p withcolor .625yellow ; pair cp ; cp := center p ;
freedotlabel (btex \strut We can    etex, point 0 of p, cp) ;
freedotlabel (btex \strut go on     etex, point 1 of p, cp) ;
freedotlabel (btex \strut and on    etex, point 2 of p, cp) ;
freedotlabel (btex \strut in        etex, point 3 of p, cp) ;
freedotlabel (btex \strut defining  etex, point 4 of p, cp) ;
freedotlabel (btex \strut funny     etex, point 5 of p, cp) ;
freedotlabel (btex \strut macros.   etex, point 6 of p, cp) ;
freedotlabel (btex \strut Can't we? etex, point 7 of p, cp) ;
\stopbuffer

\startlinecorrection[blank]
\processMPbuffer[a,b]
\stoplinecorrection

\typebuffer[b]

Unfortunately we can run into problems due to rounding errors. Therefore we use a
less readable but more safe expression for calculating the intersection points.
Instead of using point \type {loc} as endpoint we use \type {loc} shifted over a
very small distance into the direction \type {loc} from \type{ori}. In the
assignment to~\type {l} we replace \type {loc} by:

\starttyping
 ( (1+eps) * arclength(ori--loc) * unitvector(loc-ori) )
\stoptyping

\stopsection

\startsection[title={Marking angles}]

\index{angles}

A convenient \METAPOST\ macro is \type {unitvector}. When we draw a line segment
from the origin to the point returned by this macro, the segment has a length of
1~base point. This macro has a wide range of applications, but some basic
knowledge of vector algebra is handy. The following lines of \METAPOST\ code
demonstrate the basics behind unitvectors.

\startbuffer
pair uv ; pickup pencircle scaled 1mm ; autoarrows := true ;
draw fullcircle scaled 2cm withcolor .625red    ;
for i=(10,35), (-40,-20), (85,-15) :
  draw origin--i dashed evenly withcolor .625white ;
  drawarrow origin--unitvector(i) scaled 1cm withcolor .625yellow ;
endfor ;
draw origin withcolor .625red ;
\stopbuffer

\typebuffer

The circle has a radius of 1cm, and the three line segments are drawn from the
origin in the direction of the points that are passed as arguments. Because the
vector has length of~1, we scale it to the radius to let it touch the circle. By
setting \type {autoarrows} we make sure that the arrowheads are scaled
proportionally to the linewidth of 1~mm.

\startlinecorrection[blank]
\processMPbuffer
\stoplinecorrection

An application of this macro is drawing the angle between two lines. In the
\METAPOST\ manual you can find two macros for drawing angles: \type {mark_angle}
and \type {mark_rt_angle}. You may want to take a look at their definitions
before we start developing our own alternatives.

\startbuffer[x]
pickup pencircle scaled 1mm ; autoarrows := true ;
drawoptions(withcolor .625white) ;
\stopbuffer

\startbuffer[a]
def anglebetween (expr a, b) =
  (unitvector(a){a rotated 90} .. unitvector(b))
enddef ;
\stopbuffer

\startbuffer[b]
pair a, b ; a := (2cm,-1cm) ; b := (3cm,1cm) ;
drawarrow origin--a ; drawarrow origin--b ;
drawarrow anglebetween(a,b) scaled 1cm withcolor .625red ;
\stopbuffer

\startlinecorrection[blank]
\processMPbuffer[x,a,b]
\stoplinecorrection

The previous graphic demonstrates what we want to accomplish: a circular curve
indicating the angle between two straight lines. The lines and curve are drawn
with the code:

\typebuffer[b]

where \type {anglebetween} is defined as:

\typebuffer[a]

Both unitvectors return just a point on the line positioned 1~unit (later scaled
to 1cm) from the origin. We connect these points by a curve that starts in the
direction at the first point. If we omit the \type {a rotated 90} direction
specifier, we get:

\startbuffer[a]
def anglebetween (expr a, b) =
  (unitvector(a) .. unitvector(b))
enddef ;
\stopbuffer

\startlinecorrection[blank]
\processMPbuffer[x,a,b]
\stoplinecorrection

These definitions of \type {anglebetween} are far from perfect. If we don't start
in the origin, we get the curve in the wrong place and when we swap both points,
we get the wrong curve.

\startbuffer[a]
def anglebetween (expr endofa, endofb, common, length) =
  (unitvector (endofa-common){(endofa-common) rotated 90} ..
   unitvector (endofb-common)) scaled length shifted common
enddef ;
\stopbuffer

\startbuffer[b]
pair a, b, c ; a := (2cm,-1cm) ; b := (3cm,1cm) ; c := (-1cm,.5cm) ;
drawarrow c--a ; drawarrow c--b ;
drawarrow anglebetween(a,b,c,1cm) withcolor .625red ;
\stopbuffer

The solution for the displacement is given in the \METAPOST\ manual and looks
like this (we package the macro a bit different):

\typebuffer[a]

As you can see, we compensate for the origin of both vectors. This macro is
called with a few more parameters. We need to pass the length, since we want to
add the shift to the macro and the shift takes place after the scaling.

\typebuffer[b]

That the results are indeed correct, is demonstrated by the output of following
example:

\startlinecorrection[blank]
\processMPbuffer[x,a,b]
\stoplinecorrection

However, when we swap the points, we get:

\startbuffer[a]
def anglebetween (expr endofb, endofa, common, length) =
  (unitvector (endofa-common){(endofa-common) rotated 90} ..
   unitvector (endofb-common)) scaled length shifted common
enddef ;
\stopbuffer

\startlinecorrection[blank]
\processMPbuffer[x,a,b]
\stoplinecorrection

This means that instead of rotating over $90$ degrees, we have to rotate over
$-90$ or $270$ degrees. That way the arrow will also point in the other
direction. There are undoubtedly more ways to determine the direction, but the
following method also demonstrates the use of \type {turningnumber}, which
reports the direction of a path. For this purpose we compose a dummy cyclic path.

\startbuffer[a]
vardef anglebetween (expr endofa, endofb, common, length) =
  save tn ; tn := turningnumber(common--endofa--endofb--cycle) ;
show tn ;
  (unitvector(endofa-common){(endofa-common) rotated (tn*90)} ..
   unitvector(endofb-common)) scaled length shifted common
enddef ;
\stopbuffer

\typebuffer[a]

Because we use an intermediate variable, just to keep things readable, we have to
use \type {vardef} to hide the assignment for the outside world. We demonstrate
this macro using the following code:

\startbuffer[b]
pair a, b, c ; a := (2cm,-1cm) ; b := (3cm,1cm) ; c := (-1cm,.5cm) ;
drawarrow c--a ; drawarrow c--b ;
drawarrow anglebetween(a,b,c,0.75cm) withcolor .625red ;
drawarrow anglebetween(b,a,c,1.50cm) withcolor .625red ;
\stopbuffer

\typebuffer[b]

Watch how both arrows point in the direction of the line that is determined by
the second point.

\startlinecorrection[blank]
\processMPbuffer[x,a,b]
\stoplinecorrection

We now have the framework of an angle drawing macro ready and can start working
placing the label.

\startbuffer[a]
vardef anglebetween (expr endofa, endofb, common, length, str) =
  save curve, where ; path curve ; numeric where ;
  where := turningnumber (common--endofa--endofb--cycle) ;
  curve := (unitvector(endofa-common){(endofa-common) rotated (where*90)}
         .. unitvector(endofb-common)) scaled length shifted common ;
  draw thefreelabel(str,point .5 of curve,common) withcolor black ;
  curve
enddef ;
\stopbuffer

\typebuffer[a]

The macro \type {thefreelabel} is part of \METAFUN\ and is explained in detail in
\in {section} [sec:free labels]. This macro tries to place the label as good as
possible without user interference.

\startbuffer[b]
pair a ; a := (2cm,-1cm) ; drawarrow origin--a ;
pair b ; b := (3cm, 1cm) ; drawarrow origin--b ;
drawarrow
  anglebetween(a,b,origin,1cm,btex $\alpha$ etex)
  withcolor .625red ;
\stopbuffer

\typebuffer[b]

Instead of a picture we may also pass a string, but using \TEX\ by means of \type
{btex}||\type {etex} often leads to better results.

\startlinecorrection[blank]
\processMPbuffer[x,a,b]
\stoplinecorrection

Because in most cases we want the length to be consistent between figures and
because passing two paths is more convenient than passing three points, the final
definition looks slightly different.

\startbuffer[a]
numeric anglelength ; anglelength := 20pt ;

vardef anglebetween (expr a, b, str) = % path path string
  save endofa, endofb, common, curve, where ;
  pair endofa, endofb, common ; path curve ; numeric where ;
  endofa := point length(a) of a ;
  endofb := point length(b) of b ;
  if round point 0 of a = round point 0 of b :
    common := point 0 of a ;
  else :
    common := a intersectionpoint b ;
  fi ;
  where := turningnumber (common--endofa--endofb--cycle) ;
  curve := (unitvector (endofa-common){(endofa-common) rotated (where*90)} ..
            unitvector (endofb-common)) scaled anglelength shifted common ;
  draw thefreelabel(str,point .5 of curve,common) withcolor black ;
  curve
enddef ;
\stopbuffer

\typebuffer[a]

This macro has a few more \type {if}'s than its predecessor. First we test if the
label is a string, and if so, we calculate the picture ourselves, otherwise we
leave this to the user.

\startbuffer[b]
path a, b, c, d, e, f ;
a := origin--( 2cm, 1cm) ; b := origin--( 1cm, 2cm) ;
c := origin--(-2cm, 2cm) ; d := origin--(-2cm,-1cm) ;
e := origin--(-1cm,-2cm) ; f := origin--( 1cm,-2cm) ;
for i=a, b, c, d, e, f : drawarrow i ; endfor ;
anglelength := 1.0cm ; drawoptions(withcolor .625red) ;
drawarrow         anglebetween(a,b,btex $\alpha  $ etex) ;
drawarrow         anglebetween(c,d,btex $\gamma  $ etex) ;
drawarrow         anglebetween(e,f,btex $\epsilon$ etex) ;
anglelength := 1.5cm ; drawoptions(withcolor .625yellow) ;
drawdblarrow      anglebetween(b,c,btex $\beta   $ etex) ;
drawarrow reverse anglebetween(d,e,btex $\delta  $ etex) ;
drawarrow         anglebetween(a,f,btex $\zeta   $ etex) ;
\stopbuffer

\typebuffer[b]

Because \type {anglebetween} returns a path, you can apply transformations to it,
like reversing. Close reading of the previous code learns that the macro handles
both directions.

\startlinecorrection[blank]
\processMPbuffer[x,a,b]
\stoplinecorrection

Multiples of 90 degrees are often identified by a rectangular symbol. We will now
extend the previously defined macro in such a way that more types can be drawn.

\startbuffer[a]
numeric anglelength ; anglelength := 20pt ;
numeric anglemethod ; anglemethod :=    1 ;

vardef anglebetween (expr a, b, str) = % path path string
  save pointa, pointb, common, middle, offset ;
  pair pointa, pointb, common, middle, offset ;
  save curve ; path curve ;
  save where ; numeric where ;
  if round point 0 of a = round point 0 of b :
    common := point 0 of a ;
  else :
    common := a intersectionpoint b ;
  fi ;
  pointa := point anglelength on a ;
  pointb := point anglelength on b ;
  where  := turningnumber (common--pointa--pointb--cycle) ;
  middle := ((common--pointa) rotatedaround (pointa,-where*90))
                            intersectionpoint
            ((common--pointb) rotatedaround (pointb, where*90)) ;
  if     anglemethod = 1 :
    curve  := pointa{unitvector(middle-pointa)}.. pointb;
    middle := point .5 along curve ;
  elseif anglemethod = 2 :
    middle := common rotatedaround(.5[pointa,pointb],180) ;
    curve  := pointa--middle--pointb ;
  elseif anglemethod = 3 :
    curve  := pointa--middle--pointb ;
  elseif anglemethod = 4 :
    curve  := pointa..controls middle..pointb ;
    middle := point .5 along curve ;
  fi ;
  draw thefreelabel(str, middle, common) withcolor black ;
  curve
enddef ;
\stopbuffer

\typebuffer[a]

\startbuffer[p]
anglemethod := 1 ;
\stopbuffer

\startbuffer[q]
anglemethod := 2 ;
\stopbuffer

\startbuffer[r]
anglemethod := 3 ;
\stopbuffer

\startbuffer
\startcombination[3*1]
  {\processMPbuffer[x,a,p,b]} {method 1}
  {\processMPbuffer[x,a,q,b]} {method 2}
  {\processMPbuffer[x,a,r,b]} {method 3}
\stopcombination
\stopbuffer

\placefigure
  [here][fig:three methods]
  {Three ways of marking angles.}
  {\getbuffer}

\in {Figure} [fig:three methods] shows the first three alternative methods
implemented here. Instead of using \typ {unitvectors}, we now calculate the
points using the \typ {arctime} and \typ {arclength} primitives. Instead of
complicated expressions, we use the \METAFUN\ operators \type {along} and \type
{on}. The following expressions are equivalent.

\starttyping
pointa := point anglelength on a ;
middle := point .5 along curve ;

pointa := point (arctime anglelength of a) of a ;
middle := arctime (.5(arclength curve)) of curve) of curve ;
\stoptyping

The third method can be implemented in different, more math intensive ways, but
the current implementation suits rather well and is understood by the author.

\stopsection

\startsection[reference=sec:color circles,title={Color circles}]

\index{color}

In \in {chapter} [sec:embedding] we showed a few color circles. Drawing such a
graphic can be done in several ways, and here we will show a few methods. First
we will demonstrate how you can apply \type {cutafter} and \type {cutbefore},
next we will show how the \METAPOST\ macro \type {buildpath} can be used, and
finally we will present a clean solution using \type {subpath}. We will assume
that the circle is called with the macro:

\starttyping
colorcircle (4cm, red, green, blue) ;
\stoptyping

\startbuffer[circle]
vardef colorcircle (expr size, red, green, blue) =
  save r, g, b, rr, gg, bb, cc, mm, yy ;
  save b_r, b_g, g_r, g_b ;
  save radius ;

  path r, g, b, rr, bb, gg, cc, mm, yy ;
  pair b_r, b_g, g_r, g_b ;

  numeric radius ; radius := 3cm ;

  pickup pencircle scaled (radius/20) ;

  r := g := b := fullcircle scaled radius shifted (0,radius/4);

  r := r rotatedaround(origin, 15) ; % drawarrow r withcolor red   ;
  g := g rotatedaround(origin,135) ; % drawarrow g withcolor green ;
  b := b rotatedaround(origin,255) ; % drawarrow b withcolor blue  ;

  b_r :=         b intersectionpoint r ; % draw b_r ;
  b_g :=         b intersectionpoint g ; % draw b_g ;
  g_r := reverse g intersectionpoint r ; % draw g_r ;
  g_b := reverse g intersectionpoint b ; % draw g_b ;

  bb := b cutafter  b_r ; bb := bb cutbefore b_g ; % drawarrow bb ;
  gg := g cutbefore b_g ; gg := gg cutafter  g_r ; % drawarrow gg ;
  rr := r cutbefore g_r &        r cutafter  b_r ; % drawarrow rr ;

  cc := b cutbefore b_r ; cc := cc cutafter  g_b ; % drawarrow br ;
  yy := g cutbefore g_r ; yy := yy cutafter  g_b ; % drawarrow rg ;
  mm := r cutbefore g_r &        r cutafter  b_r ; % drawarrow gb ;

  bb := gg -- rr -- reverse bb -- cycle ;
  gg := bb rotatedaround(origin,120) ;
  rr := bb rotatedaround(origin,240) ;

  cc := mm -- cc -- reverse yy -- cycle ;
  yy := cc rotatedaround(origin,120) ;
  mm := cc rotatedaround(origin,240) ;

  fill fullcircle scaled radius withcolor white ;

  fill rr withcolor red   ; fill cc withcolor white-red   ;
  fill gg withcolor green ; fill mm withcolor white-green ;
  fill bb withcolor blue  ; fill yy withcolor white-blue  ;

  for i = rr,gg,bb,cc,mm,yy : draw i withcolor .5white ; endfor ;

  currentpicture := currentpicture xsized size ;
enddef ;
\stopbuffer

We need to calculate seven paths. The first implementation does all the handywork
itself and thereby is rather long, complicated and unreadable. It does not really
use the strength of \METAPOST\ yet.

\typebuffer[circle]

In determining the right intersection points, you need to know where the path
starts and in what direction it moves. In case of doubt, drawing the path as an
arrow helps. If you want to see the small paths used, you need to comment the
lines with the \type {fill}'s and uncomment the lines with \type {draw}'s. Due to
the symmetry and the fact that we keep the figure centered around the origin, we
only need to calculate two paths since we can rotate them.

There are for sure more (efficient) ways to draw such a figure, but this one
demonstrates a few new tricks, like grouping. We use grouping here because we
want to use \type {mm} to indicate the magenta path, and \type {mm} normally
means millimeter. Within a group, you can save variables. These get their old
values when the group is left.

With \type {for} we process multiple paths after each other. In this case it
hardly saves tokens, but it looks more clever.

One of the more efficient methods is using the \type {buildcycle} macro. This
macro takes two or more paths and calculates the combined path. Although this is
a rather clever macro, you should be prepared to help it a bit when paths have
multiple intersection points. Again, we could follow a more secure mathematical
method, but the next one took only a few minutes of trial and error. To save some
memory, we redefine the \type {colors} graphic.

\startbuffer[demo]
colorcircle(4cm, red, green, blue) ;
\stopbuffer

When we call this macro as:

\typebuffer[demo]

we get:

\startlinecorrection[blank]
\processMPbuffer[circle,demo]
\stoplinecorrection

Of course this macro is only used for demonstration purposes and has no real use.

\startbuffer[circle]
vardef colorcircle (expr size, red, green, blue) =
  save r, g, b, rr, gg, bb, cc, mm, yy ; save radius ;
  path r, g, b, rr, bb, gg, cc, mm, yy ; numeric radius ;

  radius := 5cm ; pickup pencircle scaled (radius/25) ;

  r := g := b := fullcircle scaled radius shifted (0,radius/4) ;

  r := r rotatedaround (origin, 15) ;
  g := g rotatedaround (origin,135) ;
  b := b rotatedaround (origin,255) ;

  r := r rotatedaround(center r,-90) ;
  g := g rotatedaround(center g, 90) ;

  gg := buildcycle(buildcycle(reverse r,b),g) ;
  cc := buildcycle(buildcycle(b,reverse g),r) ;

  rr := gg rotatedaround(origin,120) ;
  bb := gg rotatedaround(origin,240) ;

  yy := cc rotatedaround(origin,120) ;
  mm := cc rotatedaround(origin,240) ;

  fill fullcircle scaled radius withcolor white ;

  fill rr withcolor red   ; fill cc withcolor white-red   ;
  fill gg withcolor green ; fill mm withcolor white-green ;
  fill bb withcolor blue  ; fill yy withcolor white-blue  ;

  for i = rr,gg,bb,cc,mm,yy : draw i withcolor .5white ; endfor ;

  currentpicture := currentpicture xsized size ;
enddef ;
\stopbuffer

\typebuffer [circle]

Since we don't want to duplicate a graphic, this time we show the dark
alternatives.

\startbuffer[demo]
colorcircle(4cm, .5red, .5green, .5blue) ;
\stopbuffer

\typebuffer[demo]

This kind of unsafe path calculations are very sensitive to breaking. Changing
the \type {radius/4} into something else demonstrates this but we will not
challenge this macro that much. Therefore, the 50\% color circle shows up as:

\startlinecorrection[blank]
\processMPbuffer[circle,demo]
\stoplinecorrection

This command is part of \METAFUN\ and can be used to determine nice color
combinations by also looking at their complementary colors.

\startbuffer[demo]
colorcircle (4cm, .7red, .5green, .3blue) ;
\stopbuffer

\typebuffer[demo]

\startlinecorrection[blank]
\processMPbuffer[circle,demo]
\stoplinecorrection

The next circle that we draw shows the three main colors used in this document.
This circle is not that beautiful.

\startbuffer[demo]
colorcircle(4cm,.625red,.625yellow,.625white) ;
\stopbuffer

\typebuffer[demo]

\startlinecorrection[blank]
\processMPbuffer[circle,demo]
\stoplinecorrection

This definition can be cleaned up a bit by using \type {transform}, but the fuzzy
\type {buildcycle}'s remain.

\startbuffer[circle]
vardef colorcircle (expr size, red, green, blue) =
  save r, g, b, rr, gg, bb, cc, mm, yy ; save radius ;
  path r, g, b, rr, bb, gg, cc, mm, yy ; numeric radius ;

  radius := 5cm ; pickup pencircle scaled (radius/25) ;

  transform t ; t := identity rotatedaround(origin,120) ;

  r := fullcircle scaled radius
   shifted (0,radius/4) rotatedaround(origin,15) ;

  g := r transformed t ; b := g transformed t ;

  r := r rotatedaround(center r,-90) ;
  g := g rotatedaround(center g, 90) ;

  gg := buildcycle(buildcycle(reverse r,b),g) ;
  cc := buildcycle(buildcycle(b,reverse g),r) ;

  rr := gg transformed t ; bb := rr transformed t ;
  yy := cc transformed t ; mm := yy transformed t ;

  fill fullcircle scaled radius withcolor white ;

  fill rr withcolor red   ; fill cc withcolor white-red   ;
  fill gg withcolor green ; fill mm withcolor white-green ;
  fill bb withcolor blue  ; fill yy withcolor white-blue  ;

  for i = rr,gg,bb,cc,mm,yy : draw i withcolor .5white ; endfor ;

  currentpicture := currentpicture xsized size ;
enddef ;
\stopbuffer

\typebuffer [circle]

\startbuffer[demo]
colorcircle(4cm,(.4,.6,.8),(.8,.4,.6),(.6,.8,.4));
\stopbuffer

\startlinecorrection[blank]
\processMPbuffer[circle,demo]
\stoplinecorrection

This rather nice circle is defined as:

\typebuffer[demo]

The final implementation, which is part of \METAFUN, is slightly more efficient.

\startbuffer[circle]
vardef colorcircle (expr size, red, green, blue) =
  save r, g, b, c, m, y, w ; save radius ;
  path r, g, b, c, m, y, w ; numeric radius ;

  radius := 5cm ; pickup pencircle scaled (radius/25) ;

  transform t ; t := identity rotatedaround(origin,120) ;

  r := fullcircle rotated 90 scaled radius
         shifted (0,radius/4) rotatedaround(origin,135) ;

  b := r transformed t ; g := b transformed t ;

  c := buildcycle(subpath(1,7) of g, subpath(1,7) of b) ;
  y := c transformed t ; m := y transformed t ;

  w := buildcycle(subpath(3,5) of r,
         subpath(3,5) of g, subpath(3,5) of b) ;

  pushcurrentpicture ;

  fill r withcolor red   ;
  fill g withcolor green ;
  fill b withcolor blue  ;
  fill c withcolor white-red   ;
  fill m withcolor white-green ;
  fill y withcolor white-blue  ;
  fill w withcolor white ;

  for i = r,g,b,c,m,y : draw i withcolor .5white ; endfor ;

  currentpicture := currentpicture xsized size ;

  popcurrentpicture ;
enddef ;
\stopbuffer

\typebuffer [circle]

Here, we first fill the primary circles, next we fill the secondary ones. These
also cover the center, which is why finally we fill the center with white.

\startbuffer[demo]
colorcircle(4cm,(.2,.5,.8),(.8,.2,.5),(.5,.8,.2));
\stopbuffer

\startlinecorrection[blank]
\processMPbuffer[circle,demo]
\stoplinecorrection

The circle uses the following colors:

\typebuffer[demo]

The next graphic demonstrates how the subpaths look that build the shapes.

\startbuffer[circle]
vardef colorcircle (expr size, red, green, blue) =
  save r, g, b, c, m, y, w ; save radius ;
  path r, g, b, c, m, y, w ; numeric radius ;

  radius := 5cm ; pickup pencircle scaled (radius/25) ;

  transform t ; t := identity rotatedaround(origin,120) ;

  r := fullcircle rotated 90 scaled radius
         shifted (0,radius/4) rotatedaround(origin,135) ;

  b := r transformed t ; g := b transformed t ;

  c := buildcycle(subpath(1,7) of g,subpath(1,7) of b) ;
  y := c transformed t ; m := y transformed t ;

  w := buildcycle(subpath(3,5) of r,
         subpath(3,5) of g, subpath(3,5) of b) ;

  pushcurrentpicture ;

  def do_it =
    fill r withcolor red   ;
    fill g withcolor green ;
    fill b withcolor blue  ;
    fill c withcolor white-red   ;
    fill m withcolor white-green ;
    fill y withcolor white-blue  ;
    fill w withcolor white ;
    for i = r,g,b,c,m,y : draw i withcolor .5white ; endfor ;
  enddef ;

  autoarrows := true ;

  do_it ;
  for i=r,g,b : drawarrow i withcolor black ; endfor ;
  currentpicture := currentpicture shifted (-2radius,0) ;

  do_it ;
  for i=r,g,b : drawarrow subpath(1,7) of i withcolor black ; endfor ;
  currentpicture := currentpicture shifted (-2radius,0) ;

  do_it ;
  for i=r,g,b : drawarrow subpath(3,5) of i withcolor black ; endfor ;
  currentpicture := currentpicture shifted (+4radius,2radius) ;

  drawarrow r withpen pencircle scaled (radius/10) withcolor red   ;
  drawarrow g withpen pencircle scaled (radius/20) withcolor green ;
  drawarrow b withpen pencircle scaled (radius/40) withcolor blue  ;
  currentpicture := currentpicture shifted (-2radius,0) ;

  drawarrow c withpen pencircle scaled (radius/10) withcolor white-red   ;
  drawarrow m withpen pencircle scaled (radius/20) withcolor white-green ;
  drawarrow y withpen pencircle scaled (radius/40) withcolor white-blue  ;
  currentpicture := currentpicture shifted (-2radius,0) ;

  drawarrow w withcolor black ;

  currentpicture := currentpicture xsized 3size ;

  popcurrentpicture ;
enddef ;
\stopbuffer

\startlinecorrection[blank]
\processMPbuffer[circle,demo]
\stoplinecorrection

We did not mention what the push and pop commands are responsible for. Scaling
the current picture is well defined as long as we deal with one graphic. However,
if the current picture already has some content, this content is also scaled. The
push and pop commands let us add content to the current picture as well as
manipulating the picture as a whole without any side effects. The final result is
put on top of the already drawn content. Instead of the sequence:

\starttyping
pushcurrentpicture ;
  ...
  currentpicture := currentpicture ... transformations ... ;
popcurrentpicture ;
\stoptyping

you can say:

\starttyping
pushcurrentpicture ;
  ...
popcurrentpicture ... transformations ... ;
\stoptyping

Both are equivalent to:

\starttyping
draw image ( ... ) ... transformations ... ;
\stoptyping

For larger sequences of commands, the push||pop alternative gives a bit more more
readable code.

\stopsection

\startsection[title={Fool yourself}]

When doing a literature search on the human perception of black||white edges, I
ran into several articles with graphics that I remember having seen before in
books on psychology, physiology and|/|or ergonomics. One of the articles was by
Edward H.~Adelson of MIT and we will use a few of his example graphics in our
exploration to what extend \METAPOST\ can be of help in those disciplines. Since
such graphics normally occur in typeset documents, we will define them in the
document source.

\startbuffer[a]
\startbuffer
interim linecap := butt ; numeric u ; u := 1cm ;
pickup pencircle scaled .5u ;
for i=1u step u until 5u :
  draw (0,i) -- (5u,i) ;
endfor ;
for i=2u step u until 4u :
  draw   (u,i) -- (2u,i)                   withcolor .5white ;
  draw ((3u,i) -- (4u,i)) shifted (0,-.5u) withcolor .5white ;
endfor ;
\stopbuffer
\stopbuffer

\startbuffer[b]
\placefigure
  [here][fig:tricked 1]
  {White's illusion.}
  {\processMPbuffer}
\stopbuffer

\getbuffer[a,b]

Unless you belong to the happy few whose visual capabilities are not distorted by
neural optimizations, in \in {figure} [fig:tricked 1] the gray rectangles at the
left look lighter than those on the right. Alas, you can fool yourself, but
\METAPOST\ does not cheat. This graphic, referred to as White's illusion, is
defined as follows.

\typebuffer[a]

Watch how we include the code directly. We have packaged this graphic in a buffer
which we include as a floating figure.

\typebuffer[b]

When passed to \METAPOST, this code is encapsulated in its \type {beginfig} and
\type {endfig} macros and thereby grouped. But any change to a variable that is
not explicitly saved, migrates to the outer level. In order to prevent all
successive graphics to have butt'd linecaps, we have to change this line
characteristic locally. Because \type {linecap} is defined as an internal
variable, we have to use \type {interim} to overload its value. Because \type {u}
is a rather commonly used scratch variable, we don't save its value.

Watch how we use \type {u} as the loop step. In spite of what your eyes tell you,
this graphic only has two explicit color directives, both being 50\% black. In
the next example we will use some real colors.

\startbuffer[a]
\startuseMPgraphic{first}
  numeric size, delta ;
  size := 2.5cm ; delta := size/3 ;
  color mainshade, topshade, bottomshade, centershade ;
  mainshade := \MPcolor{funcolor} ;
  topshade := .9mainshade ; bottomshade := .5mainshade ;
  centershade := .5[topshade,bottomshade] ;
\stopuseMPgraphic
\stopbuffer

\startbuffer[b]
\startuseMPgraphic{second}
  \includeMPgraphic{first}
  fill fullsquare scaled size  withcolor topshade ;
  fill fullsquare scaled delta withcolor centershade ;
\stopuseMPgraphic
\stopbuffer

\startbuffer[c]
\startuseMPgraphic{third}
  \includeMPgraphic{first}
  fill fullsquare scaled size  withcolor bottomshade ;
  fill fullsquare scaled delta withcolor centershade ;
\stopuseMPgraphic
\stopbuffer

\getbuffer[a,b,c]

\startbuffer[d]
\startcombination[5*2]
  {\definecolor[funcolor][red]   \useMPgraphic{second}} {}
  {\definecolor[funcolor][green] \useMPgraphic{second}} {}
  {\definecolor[funcolor][blue]  \useMPgraphic{second}} {}
  {\definecolor[funcolor][yellow]\useMPgraphic{second}} {}
  {\definecolor[funcolor][white] \useMPgraphic{second}} {}
  {\definecolor[funcolor][red]   \useMPgraphic{third}}  {}
  {\definecolor[funcolor][green] \useMPgraphic{third}}  {}
  {\definecolor[funcolor][blue]  \useMPgraphic{third}}  {}
  {\definecolor[funcolor][yellow]\useMPgraphic{third}}  {}
  {\definecolor[funcolor][white] \useMPgraphic{third}}  {}
\stopcombination
\stopbuffer

\placefigure
  [here][fig:tricked 2]
  {The simultaneous contrast effect.}
  {\getbuffer[d]}

In \in {figure} [fig:tricked 2] the small squares in the center of each colored
pair of big squares have the same shade, but the way we perceive them are
influenced by their surroundings. Both sets of squares are defined using usable
graphics. The top squares are defined as:

\typebuffer[b]

and the bottom squares are coded as:

\typebuffer[c]

Because both graphics share code, we have defined that code as a separate
graphic, that we include. The only point of interest in this definition is the
fact that we let \METAPOST\ interpolate between the two colors using \type {.5[
]}.

\typebuffer[a]

The color \type {funcolor} is provided by \CONTEXT, and since we want to use this
graphic with different colors, this kind of mapping is quite convenient. The
bunch of graphics is packaged in a combination with empty captions. Note how we
set the color before we include the graphic.

\typebuffer[d]

We use a similar arrangement for the following graphic, where we have replaced
the definitions of \type {first}, \type {second} and \type {third} by new
definitions.

\startbuffer[a]
\startuseMPgraphic{first}
  numeric height, width, radius, gap ; gap := 1mm ;
  height = 2.5cm ; width := height/2 ; radius := height/2.5 ;
  color mainshade, leftshade, rightshade, centershade ;
  mainshade := \MPcolor{funcolor} ;
  leftshade := .9mainshade ; rightshade := .5mainshade ;
  centershade := .5[leftshade,rightshade] ;
  fill unitsquare xyscaled ( width,height) withcolor leftshade ;
  fill unitsquare xyscaled (-width,height) withcolor rightshade ;
  draw (fullcircle scaled radius) shifted (0,height/2)
    withpen pencircle scaled (radius/2) withcolor centershade ;
\stopuseMPgraphic
\stopbuffer

\startbuffer[b]
\startuseMPgraphic{second}
  \includeMPgraphic{first}
  interim linecap := butt ; pickup pencircle scaled gap ;
  draw (0,0) -- (0,height) withcolor white ;
\stopuseMPgraphic
\stopbuffer

\startbuffer[c]
\startuseMPgraphic{third}
  \includeMPgraphic{first}
  picture p, q ; p := q := currentpicture ;
  clip p to unitsquare xscaled  width yscaled height ;
  clip q to unitsquare xscaled -width yscaled height ;
  currentpicture := p ;
  addto currentpicture also q shifted (0,radius/2) ;
\stopuseMPgraphic
\stopbuffer

\getbuffer[a,b,c]

\startbuffer[d]
\startcombination[5*3]
  {\definecolor[funcolor][red]   \useMPgraphic{first}}  {}
  {\definecolor[funcolor][green] \useMPgraphic{first}}  {}
  {\definecolor[funcolor][blue]  \useMPgraphic{first}}  {}
  {\definecolor[funcolor][yellow]\useMPgraphic{first}}  {}
  {\definecolor[funcolor][white] \useMPgraphic{first}}  {}
  {\definecolor[funcolor][red]   \useMPgraphic{second}} {}
  {\definecolor[funcolor][green] \useMPgraphic{second}} {}
  {\definecolor[funcolor][blue]  \useMPgraphic{second}} {}
  {\definecolor[funcolor][yellow]\useMPgraphic{second}} {}
  {\definecolor[funcolor][white] \useMPgraphic{second}} {}
  {\definecolor[funcolor][red]   \useMPgraphic{third}}  {}
  {\definecolor[funcolor][green] \useMPgraphic{third}}  {}
  {\definecolor[funcolor][blue]  \useMPgraphic{third}}  {}
  {\definecolor[funcolor][yellow]\useMPgraphic{third}}  {}
  {\definecolor[funcolor][white] \useMPgraphic{third}}  {}
\stopcombination
\stopbuffer

\placefigure
  [here][fig:tricked 3]
  {Koffka's examples of manipulating contrast by changing
   the spatial configuration.}
  {\getbuffer[d]}

The definition of the first row of \in {figure} [fig:tricked 3] is used in the
second and third and therefore is the most complicated. We use quite some scratch
variables to reach a high level of abstraction. The \type {xyscaled} operator is
a \METAFUN\ macro.

\typebuffer[a]

The graphics of the second row extends those of the first by drawing a white line
through the middle. In this example setting the linecap is not really needed,
because rounded top and bottoms in white are invisible and the part that extends
beyond the points does not count in calculating the bounding box.

\typebuffer[b]

The third row graphics again extend the first graphic. First we copy the picture
constructed so far. Watch the double assignment. Next we clip the pictures in
half, and shift the right half down over the width of the circle.

\typebuffer[c]

\stopsection

% \startsection[title={Puzzles}]
%
% {\em Maybe.}
%
% \stopsection
%
% \startsection[title={Flow charts}]
%
% {\em Instead of starting anew every time, you can use predefined macros, like
% those in the flow chart module. Let's see how we can influence the \METAPOST\
% code. Maybe not here.}
%
% \stopsection
%
% \startsection[title={Chemistry}]
%
% {\em \METAPOST\ can do it's work unseen, as in the chemistry module that comes
% with \CONTEXT. There, \METAPOST\ is used as one of the graphical plug||ins. It
% demonstrates that we can put \METAPOST\ to work without seeing any code.}
%
% \stopsection

\startsection[title={Growing graphics}]

Although \METAPOST\ is not really suited as a simulation engine, it is possible
to build graphics that are built and displayed incrementally with a sequence of
mouse clicks. The following example is the result of an email discussion David
Arnold and the author had while \METAFUN\ evolved.

Instead of defining the graphics in a separate \METAPOST\ file, we will
incorporate them in the document source in which they are used. We can use
several methods.

\startitemize[n]
\startitem
    Define macros and figures in a separate file and include the graphics as
    external graphics.
\stopitem
\startitem
    Define everything in the document source as usable graphics and include the
    graphics using \type {\useMPgraphic}.
\stopitem
\startitem
    Package the graphic components in buffers and paste those together as
    graphics that can be processed at run time.
\stopitem
\stopitemize

The first method is the most independent one, which has its advantages if we want
to use the graphics in other applications too. The second method works well in
graphics where parts of the definitions change between invocations of the
graphic. This method follows the template:

\starttyping
\startuseMPgraphic{whatever}
 ...
\stopuseMPgraphic

\startuseMPgraphic{result}
  ...
  \includeMPgraphic{whatever}
  ...
\stopuseMPgraphic

\useMPgraphic{result}
\stoptyping

The disadvantage of this method is that it cannot be combined with \type
{btex}||\type {etex} since it is nearly impossible to determine when, how, and to
what extent the content of a graphic should be expanded before writing it to the
temporary \METAPOST\ file.

Therefore, we will demonstrate how buffers can be used. This third method closely
parallels the first way of defining graphics. A nice side effect is that we can
easily typeset these buffers verbatim, which we did to typeset this document.

We are going to do a classic compass and straightedge construction, the bisection
of a line segment joining two arbitrary points. We will construct five graphics,
where each one displays one step of the construction. We will embed each graphic
in a start||stop command. Later we will see the advantage of this strategy.

\startbuffer
\startbuffer[a]
def start_everything = enddef ;
def stop_everything  = enddef ;
\stopbuffer
\stopbuffer

\typebuffer \getbuffer

\startbuffer
\startbuffer[b]
numeric u, w ; u := .5cm ; w := 1pt ;

pickup pencircle scaled w ;

def draw_dot expr p =
  draw p withpen pencircle scaled 3w ;
enddef ;

def stand_out =
  drawoptions(withcolor .625red) ;
enddef ;
\stopbuffer
\stopbuffer

We are going to draw a few dots, and to force consistency we first define a macro
\type {draw_dot}. The current step will be highlighted in red using \type
{stand_out}.

\typebuffer \getbuffer

\startbuffer
\startbuffer[c]
def draw_basics =
  pair pointA, pointB ; path lineAB ;
  pointA := origin ; pointB := pointA shifted (5u,0) ;
  lineAB := pointA -- pointB ;
  draw lineAB ;
  draw_dot pointA ; label.lft(btex A etex, pointA) ;
  draw_dot pointB ; label.rt (btex B etex, pointB) ;
enddef ;
\stopbuffer
\stopbuffer

First, we construct the macro that will plot two points $A$ and $B$ and connect
them with a line segment.

\typebuffer \getbuffer

\startbuffer
\startbuffer[1]
start_everything ;
  stand_out ; draw_basics ;
stop_everything ;
\stopbuffer
\stopbuffer

The code in this buffer executes the preceding macros. The \type {..._everything}
commands are still undefined, but later we can use these hooks for special
purposes.

\typebuffer \getbuffer

This graphic can now be embedded by the \CONTEXT\ command
\type {\processMPbuffer}. This command, like the ordinary
buffer inclusion commands, accepts a list of buffers.

\startbuffer
\startlinecorrection[blank]
\ruledhbox{\processMPbuffer[a,b,c,1]}
\stoplinecorrection
\stopbuffer

\typebuffer

We use \type {\ruledhbox} to show the tight bounding box of the graphic. The line
correction takes care of proper spacing around non textual content, like
graphics. \footnote {These spacing commands try to get the spacing around the
content visually compatible, and take the height and depth of the preceding and
following text into account.} This is only needed when the graphic is part of the
text flow!

\getbuffer

Next, we draw two circles of equal radius, one centered at point $A$, the other
at point $B$.

\startbuffer
\startbuffer[d]
def draw_circles =
  path circleA, circleB ; numeric radius, distance ;
  distance := (xpart pointB) - (xpart pointA) ;
  radius := 2/3 * distance ;
  circleA := fullcircle scaled (2*radius) ;
  circleB := circleA shifted pointB ;
  draw circleA ;
  draw circleB ;
enddef ;
\stopbuffer
\stopbuffer

\typebuffer \getbuffer

\startbuffer
\startbuffer[2]
start_everything ;
  draw_basics ; stand_out ; draw_circles ;
stop_everything ;
\stopbuffer
\stopbuffer

As you can see, we move down the \type {stand_out} macro so that only the
additions are colored red.

\typebuffer \getbuffer

We now use \type{\processMPbuffer[a,b,c,d,2]} to include the latest step.

\startlinecorrection[blank]
\ruledhbox{\processMPbuffer[a,b,c,d,2]}
\stoplinecorrection

The next step in the construction of the perpendicular bisector requires that we
find and label the points of intersection of the two circles centered at points
$A$ and $B$. The intersection points are calculated as follows. Watch the \type
{reverse} operation, which makes sure that we get the second intersection point.

\startbuffer
\startbuffer[e]
def draw_intersection =
  pair pointC, pointD ;
  pointC := circleA intersectionpoint circleB ;
  pointD := (reverse circleA) intersectionpoint (reverse circleB) ;
  draw_dot pointC ; label.lft(btex C etex, pointC shifted (-2w,0)) ;
  draw_dot pointD ; label.lft(btex D etex, pointD shifted (-2w,0)) ;
enddef ;
\stopbuffer
\stopbuffer

\typebuffer \getbuffer

\startbuffer
\startbuffer[3]
start_everything ;
  draw_basics ; draw_circles ; stand_out ; draw_intersection ;
stop_everything ;
\stopbuffer
\stopbuffer

In placing the label, we must make sure that the text runs free of the lines and
curves. Again, move the \type {stand_out} macro just prior to \type
{draw_intersection} macro, so that this step is highlighted in the drawing color,
while prior steps are drawn in the default color (in this case black).

\typebuffer \getbuffer

\startlinecorrection[blank]
\ruledhbox{\processMPbuffer[a,b,c,d,e,3]}
\stoplinecorrection

The line drawn through points $C$ and $D$ will be the perpendicular bisector of
the line segment connecting points $A$ and $B$. In the next step we will draw a
line using the plain \METAPOST\ \type {drawdblarrow} macro that draws arrowheads
at each end of a path.

\startbuffer
\startbuffer[f]
def draw_bisector =
  path lineCD ;
  lineCD := origin -- origin shifted (2*distance,0) ;
  lineCD := lineCD rotated 90 shifted 0.5[pointA,pointB] ;
  lineCD := lineCD shifted (0,-distance) ;
  drawdblarrow lineCD ;
enddef ;
\stopbuffer

\startbuffer[4]
start_everything ;
  draw_basics ; draw_circles ; draw_intersection ; stand_out ;
  draw_bisector ;
stop_everything ;
\stopbuffer
\stopbuffer

\typebuffer \getbuffer

\startlinecorrection[blank]
\ruledhbox{\processMPbuffer[a,b,c,d,e,f,4]}
\stoplinecorrection

The following code draws the intersection of line $C-D$ and line segment $A-B$,
which can be shown to be the midpoint of segment $A-B$.

\startbuffer
\startbuffer[g]
def draw_midpoint =
  pair pointM ;
  pointM := lineCD intersectionpoint lineAB ;
  draw_dot pointM ; label.llft(btex M etex, pointM) ;
enddef ;
\stopbuffer

\startbuffer[5]
start_everything ;
  draw_basics ; draw_circles ; draw_intersection ; draw_bisector ;
  stand_out ; draw_midpoint ;
stop_everything ;
\stopbuffer
\stopbuffer

\typebuffer \getbuffer

\startlinecorrection[blank]
\ruledhbox{\processMPbuffer[a,b,c,d,e,f,g,5]}
\stoplinecorrection

As long as we place the graphics as individual insertions in our document,
everything is fine. However, if we wish to place them all at once, or as we shall
see later, place them on top of one another in a fieldstack, it makes sense to
give them all the same bounding box. We can do this by completing the \type
{start_everything} and \type {stop_everything} commands.

\startbuffer
\startbuffer[a]
def start_everything =
  path bb ;
  draw_basics ;
  draw_circles ;
  draw_intersection ;
  draw_bisector ;
  draw_midpoint ;
  bb := boundingbox currentpicture ;
  currentpicture := nullpicture ;
enddef ;

def stop_everything =
  setbounds currentpicture to bb ;
enddef ;
\stopbuffer
\stopbuffer

\typebuffer \getbuffer

In \in {figure} [fig:1 till 5] we demonstrate the effect of this redefinition.
For this purpose we scale down the graphic to a comfortable 40\%, of course by
using an additional buffer. We also visualize the bounding box.

\startbuffer
\startbuffer[h]
def stop_everything =
  setbounds currentpicture to bb ;
  draw bb withpen pencircle scaled .5pt withcolor .625yellow ;
  currentpicture := currentpicture scaled .4 ;
enddef ;
\stopbuffer
\stopbuffer

\typebuffer \getbuffer

The graphic itself is defined as follows. Watch how we use the default buffer to
keep the definitions readable.

\startbuffer
\startbuffer
\startcombination[5*1]
  {\processMPbuffer[a,b,c,h,d,e,f,g,1]} {step 1}
  {\processMPbuffer[a,b,c,h,d,e,f,g,2]} {step 2}
  {\processMPbuffer[a,b,c,h,d,e,f,g,3]} {step 3}
  {\processMPbuffer[a,b,c,h,d,e,f,g,4]} {step 4}
  {\processMPbuffer[a,b,c,h,d,e,f,g,5]} {step 5}
\stopcombination
\stopbuffer

\placefigure
  [here][fig:1 till 5]
  {The five graphics, each with the same bounding box.}
  {\getbuffer}
\stopbuffer

\typebuffer \getbuffer

As the original purpose of these graphics was not to show them side by side, but
to present them as field stack in a document to be viewed at the computer screen.
For this purpose we have to define the graphics as symbols.

\startbuffer
\definesymbol[step 1][{\processMPbuffer[a,b,c,d,e,f,g,1]}]
\definesymbol[step 2][{\processMPbuffer[a,b,c,d,e,f,g,2]}]
\definesymbol[step 3][{\processMPbuffer[a,b,c,d,e,f,g,3]}]
\definesymbol[step 4][{\processMPbuffer[a,b,c,d,e,f,g,4]}]
\definesymbol[step 5][{\processMPbuffer[a,b,c,d,e,f,g,5]}]
\stopbuffer

\typebuffer \getbuffer

A field stack is a sequence of overlayed graphics. We will arrange these to cycle
manually, with clicks of the mouse, through the sequence of graphs depicting the
construction of the midpoint of segment $A-B$. So, in fact we are dealing with a
manual simulation. The definition of such a stack is as follows:

\startbuffer
\definefieldstack
  [midpoint construction]
  [step 1, step 2, step 3, step 4, step 5]
  [frame=on,offset=3pt,framecolor=darkyellow,rulethickness=1pt]
\stopbuffer

\typebuffer \getbuffer

The first argument is to be a unique identifier, the second argument takes a list
of symbols, while the third argument accepts settings. More on this command can
be found in the \CONTEXT\ manuals.

The stack is shown as \in {figure} [fig:steps]. Its caption provides a button,
which enables the reader to cycle through the stack. We call this a stack because
the graphics are positioned on top of each other. Only one of them is visible at
any time.

\startbuffer
\placefigure
  [here][fig:steps]
  {Bisecting a line segment with compass and straightedge? Just
   click \goto {here} [JS(Walk_Field{midpoint construction})] to
   walk through the construction! (This stack is only visible
   in a \PDF\ viewer that supports widgets.)}
  {\framed{\startoverlay
     {\symbol[step 1]}
     {\fieldstack[midpoint construction]}
   \stopoverlay}}
\stopbuffer

\typebuffer

We cheat a bit and overlay the stack over the first symbol because otherwise
nothing shows up in print (nowadays I mostly use sumatrapdf).

{\setupinteraction[color=darkred,contrastcolor=darkred]\getbuffer}

At the start of this section, we mentioned three methods. When we use the first
method of putting all the graphics in an external \METAPOST\ file, the following
framework suits. We assume that the file is called \type {step.mp} and that it is
kept by the user along with his document source. We start with the definitions of
the graphic steps. These are the same as the ones shown previously.

\starttyping
def draw_basics       = ... enddef ;
def draw_circles      = ... enddef ;
def draw_intersection = ... enddef ;
def draw_bisector     = ... enddef ;
def draw_midpoint     = ... enddef ;
def stand_out         = ... enddef ;
\stoptyping

We can safe some code by letting the \type {..._everything} take care of the
\type {beginfig} and \type {endfig} macros.

\starttyping
def start_everything (expr n) = beginfig(n) ; ... enddef ;
def stop_everything           =    ... ; endfig ; enddef ;
\stoptyping

The five graphics now become:

\starttyping
start_everything (1) ;
  stand_out ; draw_basics ;
stop_everything ;

start_everything (2) ;
  draw_basics ; stand_out ; draw_circles ;
stop_everything ;

start_everything (3) ;
  draw_basics ; draw_circles ; stand_out ; draw_intersection ;
stop_everything ;

start_everything (4) ;
  draw_basics ; draw_circles ; draw_intersection ; stand_out ;
  draw_bisector ;
stop_everything ;

start_everything (5) ;
  draw_basics ; draw_circles ; draw_intersection ; draw_bisector ;
  stand_out ; draw_midpoint ;
stop_everything ;
\stoptyping

The definitions of the symbols now refer to an external
figure.

\starttyping
\definesymbol[step 1][{\externalfigure[step.1]}]
\definesymbol[step 2][{\externalfigure[step.2]}]
\definesymbol[step 3][{\externalfigure[step.3]}]
\definesymbol[step 4][{\externalfigure[step.4]}]
\definesymbol[step 5][{\externalfigure[step.5]}]
\stoptyping

Which method is used, depends on the way the graphics are used. In this example
we wanted to change the definition of \type {..._everything}, so here the third
method was quite useful.

\stopsection

\startsection[title={Simple Logos}]

\startbuffer[ns]
numeric width, height, line, delta ;
width = 5cm ; height = width/2 ; line = height/4 ; delta = line ;

linejoin := mitered ; pickup pencircle scaled line ;

color nsblue   ; nsblue   := (0,0,1) ;
color nsyellow ; nsyellow := (1,1,0) ;

z1 = (0, height/2) ;
z2 = (width/2-height/4, y1) ;
z3 = (width/2+height/4, y4) ;
z4 = (width, 0) ;

z5 = (x4+height/2, y1) ;
z6 = (x4, 2y1) ;
z7 = 1.5[z5,z6] ;

path p ; p := z1--z2--z3--z4 ; path q ; q := z3--z4--z5--z7 ;

numeric d, lx, ly, ux, uy ; d = line/2 ;

lx = -3d - d/3  ; ly = -d ; ux = rt x5 + d/3 ; uy = top y6 ;

path r ; r := (lx,ly)--(ux,ly)--(ux,uy)--(lx,uy)--cycle;

lx := lx-delta ; ly := ly-delta ; ux := ux+delta ; uy := uy+delta ;

path s ; s := (lx,ly)--(ux,ly)--(ux,uy)--(lx,uy)--cycle;

draw p withcolor nsblue ; draw q withcolor nsblue ;

addto currentpicture also currentpicture
  rotatedaround (.5[z2,z3],180) shifted (height/4,height/2) ;

picture savedpicture ; savedpicture := currentpicture ;

clip currentpicture to r ;
setbounds currentpicture to r ;

savedpicture := currentpicture ; currentpicture := nullpicture ;

fill s withcolor nsyellow ;
addto currentpicture also savedpicture ;
\stopbuffer

Many company logos earn their beauty from their simplicity. One of the logos that
most Dutch people have imprinted in their mind is that of the Dutch Railway
Company (NS). An interesting feature of this logo is that, although it is widely
known, drawing it on a piece of paper from mind is a task that many people fail.

\startlinecorrection[blank]
\processMPbuffer[ns]
\stoplinecorrection

This logo makes a good candidate for demonstrating a few fine points of drawing
graphics, like using linear equations, setting line drawing characteristics,
clipping and manipulating bounding boxes.

The implementation below is quite certainly not according to the official
specifications, but it can nevertheless serve as an example of defining such
logos.

\startbuffer[a]
numeric width  ; width  = 3cm ;
numeric height ; height = width/2 ;
numeric line   ; line   = height/4 ;
\stopbuffer

As always, we need to determine the dimensions first. Here, both the height and
line width depend on the width of the graphic.

Instead of calculating the blue shape such that it will be a filled outline, we
will draw the logo shape using line segments. This is why we need the \type
{line} parameter.

\typebuffer[a]

We want sharp corners which can be achieved by setting \type {linejoin} to \type
{mitered}.

\startbuffer[b]
linejoin := mitered ; pickup pencircle scaled line ;
\stopbuffer

\typebuffer[b]

The colors are rather primary blue and yellow. At the time of writing this
manual, Dutch trains are still painted yellow, so we will use that shade as
background color.

\startbuffer[c]
color nsblue   ; nsblue   := (0,0,1) ;
color nsyellow ; nsyellow := (1,1,0) ;
\stopbuffer

\typebuffer[c]

We will now describe the main curves. Although these expressions are not that
advanced, they demonstrate that we can express relationships instead of using
assignments.

\startbuffer[d]
z1 = (0, height/2) ;
z2 = (width/2-height/4, y1) ;
z3 = (width/2+height/4, y4) ;
z4 = (width, 0) ;

path p ; p := z1--z2--z3--z4 ;
\stopbuffer

\typebuffer[d]

Although it is accepted to consider \type {z} to be a variable, it is in fact a
\type {vardef} macro, that expands into a pair \type {(x,y)}. This means that the
previous definitions internally become:

\starttyping
(x1,y1) = (0, height/2) ;
(x2,y2) = (width/2-height/4, y1) ;
(x3,y3) = (width/2+height/4, y4) ;
(x4,y4) = (width, 0) ;
\stoptyping

These 8 relations can be solved by \METAPOST, since all dependencies are known.

\starttyping
x1 = 0                ; y1 = height/2 ;
x2 = width/2-height/4 ; y2 = y1       ;
x3 = width/2+height/4 ; y3 = y4       ;
x4 = width            ; y4 = 0        ;
\stoptyping

Since we express the variables \type {x} and \type {y} in terms of relations, we
cannot reuse them, because that would mean that inconsistent relations occur. So,
the following lines will lead to an error message:

\starttyping
z1 = (10,20) ; z1 = (30,50) ;
\stoptyping

For similar reasons, we may not assign a value (using \type {:=}) to such a \type
{z} variable. Within a \METAPOST\ figure, \type {z} variables are automatically
saved, which means that they can be reused for each figure.

\startbuffer[x]
drawpath p ; drawpoints p ; drawpointlabels p ;
\stopbuffer

So far, we have defined the following segment of the logo.

\startlinecorrection[blank]
\processMPbuffer[a,b,c,d,x]
\stoplinecorrection

\startbuffer[e]
z5 = (x4+height/2, y1) ;
z6 = (x4, 2y1) ;
z7 = 1.5[z5,z6] ;

path q ; q := z3--z4--z5--z7 ;
\stopbuffer

The next expressions are used to define the second segment. The third expression
determines \type {z7} to be positioned on the line \type {z5--z6}, where we
extend this line by 50\%.

\typebuffer[e]

\startbuffer[x]
drawpath q ; drawpoints q ; drawpointlabels q ;
\stopbuffer

\startlinecorrection[blank]
\processMPbuffer[a,b,c,d,e,x]
\stoplinecorrection

If we combine these two segments, we get:

\startbuffer[x]
drawpath p ; drawpoints p ; drawpointlabels p ;
swappointlabels := true ;
drawpath q ; drawpoints q ; drawpointlabels q ;
\stopbuffer

\startlinecorrection[blank]
\processMPbuffer[a,b,c,d,e,x]
\stoplinecorrection

However, when we draw them using the right linewidth and color, you will notice
that we're not yet done:

\startbuffer[f]
draw p withcolor nsblue ; draw q withcolor nsblue ;
\stopbuffer

\startlinecorrection[blank]
\processMPbuffer[a,b,c,d,e,f]
\stoplinecorrection

The second curve is similar to the first one, but rotated over 180 degrees.

\startbuffer[g]
addto currentpicture also currentpicture
  rotatedaround (.5[z2,z3],180) shifted (height/4,height/2) ;
\stopbuffer

\typebuffer[g]

\startlinecorrection[blank]
\processMPbuffer[a,b,c,d,e,f,g]
\stoplinecorrection

In order to get the sharp edges, we need to clip off part of
the curves and at first sight, we may consider using a
scaled bounding box. However, when we show the natural
bounding box, you will notice that a more complicated bit of
calculations is needed.

\startbuffer[x]
draw boundingbox currentpicture
  withpen pencircle scaled .5mm withcolor .625white ;
\stopbuffer

\startlinecorrection[blank]
\processMPbuffer[a,b,c,d,e,f,g,x]
\stoplinecorrection

The right clip path is calculated using the following expressions. Watch how we
use \type {rt} and \type {top} to correct for the linewidth.

\startbuffer[h]
numeric d, lx, ly, ux, uy ; d = line/2 ;

lx = -3d - d/3  ; ly = -d ; ux = rt x5 + d/3 ; uy = top y6 ;

path r ; r := (lx,ly)--(ux,ly)--(ux,uy)--(lx,uy)--cycle;
\stopbuffer

\typebuffer[h]

The clipping path is applied by saying:

\startbuffer[i]
clip currentpicture to r ;
\stopbuffer

\typebuffer[i]

The result is quite acceptable:

\startlinecorrection[blank]
\processMPbuffer[a,b,c,d,e,f,g,h,i]
\stoplinecorrection

But, if you watch closely to how this graphic extends into to left margin of this
document, you will see that the bounding box is not yet right.

\startlinecorrection[blank]
\processMPbuffer[a,b,c,d,e,f,g,h,i,x]
\stoplinecorrection

\startbuffer[j]
setbounds currentpicture to r ;
\stopbuffer

\typebuffer[j]

We use the same path \type {r} to correct the bounding box.

\startlinecorrection[blank]
\processMPbuffer[a,b,c,d,e,f,g,h,i,j,x]
\stoplinecorrection

There are a few subtle points involved, like setting the \type {linejoin}
variable. If we had not set it to \type {mitered}, we would have got round
corners. We don't set the \type {linecap}, because a flat cap would not extend
far enough into the touching curve and would have left a small hole. The next
example shows what happens if we set these variables to the wrong values:

\startbuffer[bb]
linejoin := rounded ; linecap := mitered ;
\stopbuffer

\startlinecorrection[blank]
\processMPbuffer[a,b,bb,c,d,e,f,g,h,i,j]
\stoplinecorrection

In fact we misuse the fact that both curves overlay each other.

\startbuffer[f]
draw p withcolor nsblue ; draw q withcolor .625white ;
\stopbuffer

\startlinecorrection[blank]
\processMPbuffer[a,b,c,d,e,f,g,h,i,j]
\stoplinecorrection

The complete logo definition is a bit more extensive because we also want to add
a background. Because we need to clip the blue foreground graphic, we must
temporarily store it when we fill the background.

\typebuffer[ns]

For practical use it makes sense to package this definition in a macro to which
we pass the dimensions.

\stopsection

\startsection[title={Music sheets}]

The next example demonstrates quite some features. Imagine that we want to make
us a couple of sheets so that we can write a musical masterpiece. Let's also
forget that \TEX\ can draw lines, which means that somehow we need to use
\METAPOST.

Drawing a bar is not that complicated as the following code demonstrates.

\startbuffer
\startusableMPgraphic{bar}
  vardef MusicBar (expr width, gap, linewidth, barwidth) =
    image
      ( interim linecap := butt ;
        for i=1 upto 5 :
          draw ((0,0)--(width,0)) shifted (0,(i-1)*gap)
            withpen pencircle scaled linewidth ;
        endfor ;
        for i=llcorner currentpicture -- ulcorner currentpicture ,
              lrcorner currentpicture -- urcorner currentpicture :
          draw i withpen pencircle scaled barwidth ;
        endfor ; )
  enddef ;
\stopusableMPgraphic
\stopbuffer

\typebuffer \getbuffer

We can define the sidebars a bit more efficient using two predefined subpaths:

\starttyping
for i=leftboundary currentpicture, rightboundary currentpicture :
\stoptyping

We define a macro \type {MusicBar} that takes four arguments. The first two
determine the dimensions, the last two concern the line widths. Now watch how we
can use this macro:

\startbuffer
\includeMPgraphic{bar} ;
draw MusicBar (200pt, 6pt, 1pt, 2pt) ;
draw MusicBar (300pt, 6pt, 1pt, 2pt) shifted (0,-30pt) ;
\stopbuffer

\typebuffer

\startlinecorrection[blank]
\processMPbuffer
\stoplinecorrection

As you can see in this example, the bar is a picture that can be transformed
(shifted in our case). However, a close look at the macro teaches us that it does
a couple of draws too. This is possible because we wrap the whole in an image
using the \type {image} macro. This macro temporary saves the current picture,
and at the end puts the old \type {currentpicture} under the new one.

We wrap the whole in a \type {vardef}. This means that the image is returned as
if it was a variable. Actually, the last thing in a \type {vardef} should be a
proper return value, in our case a picture. This also means that we may not end
the \type {vardef} with a semi colon. So, when the content of the \type {vardef}
is expanded, we get something

\starttyping
draw some_picture ... ;
\stoptyping

Because we are still drawing something, we can add transform directives and set
attributes, like the color.

The second \type {for} loop demonstrates two nice features. Instead of repeating
the draw operation by copying code, we apply it to a list, in our case a list of
paths. This list contains two simple line paths. Because an \type {image} starts
with a fresh \type {currentpicture}, we can safely use the bounding box data to
determine the height of the line.

The next step in producing the sheets of paper is to put several bars on a page,
preferable with the width of the current text. This time we will use a reusable
graphic, because each bar is the same.

\startbuffer
\startreusableMPgraphic{bars}
  \includeMPgraphic{bar} ;
  draw MusicBar (TextWidth, 6pt, 1pt, 2pt) withcolor .625yellow ;
\stopreusableMPgraphic
\stopbuffer

\typebuffer \getbuffer

\startlinecorrection[blank]
\reuseMPgraphic{bars}
\stoplinecorrection

Instead of going through the trouble of letting \METAPOST\ calculate the positions
of the bars, we will use \TEX. We put 12 bars on a page and let \TEX\ take care
of the inter||bar spacing. Because we only want stretchable space between bars,
called glue in \TEX, we need to remove the last added glue.

\startnotmode[screen]

\startbuffer[music]
\startstandardmakeup[doublesided=no,page=]
  \dorecurse{15}{\reuseMPgraphic{bars}\vfill}\removelastskip
\stopstandardmakeup
\stopbuffer

\stopnotmode

\startmode[screen]

\startbuffer[music]
\startstandardmakeup[doublesided=no,page=]
  \dorecurse{10}{\reuseMPgraphic{bars}\vfill}\removelastskip
\stopstandardmakeup
\stopbuffer

\stopmode

\typebuffer[music]

\startusableMPgraphic{bar}
  vardef MusicBar (expr width, gap, linewidth, barwidth) =
    image
      ( interim linecap := butt ;
        for i=1 upto 5 :
          draw ((0,0)--(width,0))
            randomized (1pt,1.5pt)
            shifted (0,(i-1)*gap)
            withpen pencircle scaled linewidth ;
        endfor ;
        for i=llcorner currentpicture -- ulcorner currentpicture ,
              lrcorner currentpicture -- urcorner currentpicture :
          draw i randomized 2pt shifted (0,-1pt)
            withpen pencircle scaled barwidth ;
        endfor ; )
  enddef ;
\stopusableMPgraphic

\startreusableMPgraphic{bars} % trigger a new one
  \includeMPgraphic{bar} ;
  draw MusicBar (TextWidth, 6pt, 1pt, 2pt) withcolor .625yellow ;
\stopreusableMPgraphic

It may add to the atmosphere of handy||work if you slightly randomize the lines.
We leave it up to the reader to figure out how the code should be changed to
accomplish this.

\startlinecorrection[blank]
\reuseMPgraphic{bars}
\stoplinecorrection

The complete result is shown on the next page.

\startpostponing
\getbuffer[music]
\stoppostponing

\stopsection

\startsection[title={The euro symbol}]

When Patrick Gundlach posted a nice \METAPOST\ version of the euro symbol to the
\CONTEXT\ discussion list, he added the comment \quotation {The official
construction is ambiguous: how thick are the horizontal bars? How much do they
stick out to the left? Is this thing a circle or what? Are the angles on the left
side of the bars the same as the one on the right side? \unknown} The alternative
below is probably not as official as his, but permits a finetuning. You are
warned: whatever you try, the euro {\em is} and {\em will remain} an ugly symbol.

We use a couple of global variables to control the euro shape within reasonable
bounds. Then we define two circles. Next we define a vertical line that we use in
a couple of cut and paste operations. Watch how the top left point of the outer
circle determines the slant of the line that we use to slice the vertical bars.

\startbuffer[euro]
boolean trace_euro ; trace_euro := false ;

vardef euro_symbol = image ( % begin_of_euro

if unknown euro_radius   : euro_radius   := 2cm             ; fi ;
if unknown euro_width    : euro_width    := 3euro_radius/16 ; fi ;
if unknown euro_r_offset : euro_r_offset := euro_width      ; fi ;
if unknown euro_l_offset : euro_l_offset := euro_radius/32  ; fi ;
if unknown euro_l_shift  : euro_l_shift  := euro_r_offset   ; fi ;
if unknown euro_v_delta  : euro_v_delta  := euro_width/4    ; fi ;

save
  outer_circle, inner_circle, hor_bar,
  right_line, right_slant, top_slant,  bot_slant,
  euro_circle, euro_topbar, euro_botbar ;

path
  outer_circle, inner_circle, hor_bar,
  right_line, right_slant, top_slant, bot_slant,
  euro_circle, euro_topbar, euro_botbar ;

outer_circle := fullcircle scaled euro_radius ;
inner_circle := fullcircle scaled (euro_radius-euro_width) ;

if trace_euro : for i = outer_circle, inner_circle :
  draw i withpen pencircle scaled 1pt withcolor .5white ;
endfor ; fi ;

right_line :=
  (lrcorner outer_circle -- urcorner outer_circle)
  shifted (-euro_r_offset,0) ;

outer_circle := outer_circle cutbefore right_line ;

right_slant :=
  point 0 of outer_circle
  -- origin shifted (0,ypart lrcorner outer_circle) ;

euro_circle := buildcycle(outer_circle, right_line,
  reverse inner_circle, reverse right_slant) ;

hor_bar := (-euro_radius,0) -- (euro_radius,0) ;

top_slant :=
  right_slant shifted (-euro_radius+euro_r_offset-euro_l_offset,0) ;

bot_slant :=
  top_slant shifted (0,-euro_l_shift) ;

if trace_euro : for i = right_line, right_slant, top_slant, bot_slant :
  draw i withpen pencircle scaled 1pt withcolor .5white ;
endfor ; fi ;

euro_topbar := buildcycle
  (top_slant, hor_bar shifted (0, euro_v_delta),
   right_slant, hor_bar shifted (0, euro_v_delta+euro_width/2)) ;

euro_botbar := buildcycle
  (bot_slant, hor_bar shifted (0,-euro_v_delta),
   right_slant, hor_bar shifted (0,-euro_v_delta-euro_width/2)) ;

for i = euro_circle, euro_topbar, euro_botbar :
  draw i withpen pencircle scaled 0 ;
endfor ;
for i = euro_circle, euro_topbar, euro_botbar :
  fill i withpen pencircle scaled 0 ;
endfor ;

if trace_euro :
  drawpoints euro_circle withcolor red   ;
  drawpoints euro_topbar withcolor green ;
  drawpoints euro_botbar withcolor blue  ;
fi ;

) enddef ; % end_of_euro
\stopbuffer

\typebuffer[euro]

We only set a parameter when it is not yet set. This has
the advantage that we don't have to set them when we change
one. This way of manipulating paths (cutting and building)
does not always work well because of rounding errors, but
here it does work.

\startbuffer[demo]
euro_radius := 4cm ; trace_euro := true ; draw euro_symbol ;
\stopbuffer

\typebuffer[demo]

For educational purposes, we have added a bit of
tracing. When enabled, the euro shows up as:

\startlinecorrection[blank]
\processMPbuffer[euro,demo]
\stoplinecorrection

Of course it would be best to define the euro as one shape, but we won't go
though that process right now. By packaging the combined paths in an image, we
can conveniently color the euro symbol:

\startbuffer[demo]
draw euro_symbol withcolor .625red ;
\stopbuffer

\typebuffer[demo]

\startlinecorrection[blank]
\processMPbuffer[euro,demo]
\stoplinecorrection

You may wonder why we both draw and fill the euro, using a pen with zero width.
We've done this in order to demonstrate the \type {redraw} and \type {refill}
macros.

\startbuffer[extra]
redraw currentpicture withpen pencircle scaled 4pt withcolor .625yellow ;
refill currentpicture withcolor .625white ;
setbounds currentpicture to boundingbox currentpicture enlarged 2pt ;
\stopbuffer

\typebuffer[extra]

\startlinecorrection[blank]
\processMPbuffer[euro,demo,extra]
\stoplinecorrection

\stopsection

\startsection[title={Killing time}]

Not seldom \TEX\ users want to use this program and its meta||relatives as
general purpose tools, even at the cost of quite some effort or suboptimal
results. Imagine that you are under way from our planet to Mars. After a long
period of sleep you wake up and start wondering on what track you are. You even
start questioning the experts that send you on your way, so you pop open your
laptop, launch your editor and start metaposting.

First you need to determine the begin and end points of your journey. For now it
is enough to know the relative angle of the paths that both planets follow as
well as the path themselves. We assume circular paths.

\startbuffer
path a ; a := fullcircle scaled 3cm ;
path b ; b := fullcircle scaled 2cm rotated 120 ;

draw a withpen pencircle scaled 1mm withcolor .625red ;
draw b withpen pencircle scaled 1mm withcolor .625yellow ;

draw point 0 of a withpen pencircle scaled 2mm ;
draw point 0 of b withpen pencircle scaled 2mm ;
\stopbuffer

\typebuffer

The rotation 120 can be calculated from the relative starting points and time the
journey will take. Alternatively we can use the time along the path, but this
would be a bit more fuzzy later on. \footnote {In case you wonder why \METAPOST\
talks about the time on a path, you now have a cue.}

\startlinecorrection[blank]
\processMPbuffer
\stoplinecorrection

After a bit of playing with drawing paths between the two points, you decide to
make a macro. We want to feed the angle between the paths but also the connecting
path. So, we have to pass a path, but unfortunately we don't have direct access
to the points. By splitting the argument definition we can pass an expression
first, and a wildcard argument next.

\startbuffer
\startuseMPgraphic{gamble}
def Gamble (expr rot) (text track) =
  path a ; a := fullcircle scaled 3cm ;
  path b ; b := fullcircle scaled 2cm rotated rot ;

  pair aa ; aa := point 0 of a ;
  pair bb ; bb := point 0 of b ;
  path ab ; ab := track ;

  draw a  withpen pencircle scaled 1mm withcolor .625red ;
  draw b  withpen pencircle scaled 1mm withcolor .625yellow ;

  draw aa withpen pencircle scaled 2mm ;
  draw bb withpen pencircle scaled 2mm ;

  drawarrow ab withpen pencircle scaled 1mm withcolor .625white ;

  setbounds currentpicture to boundingbox a enlarged 2mm ;
  draw boundingbox currentpicture withpen pencircle scaled .25mm ;
enddef ;
\stopuseMPgraphic
\stopbuffer

\typebuffer \getbuffer

Because at this distance nobody will bother us with the thickness of the pen and
colors, we code them the hard way. We create our own universe by setting a fixed
boundingbox.

We leave the Earth in the most popular way, straight upwards and after a few
cycles, we leave it parallel to the surface. The path drawn reminds much of the
trajectories shown in popular magazines.

\startbuffer
\startMPcode
\includeMPgraphic{gamble} ;
Gamble(120, aa {(0,1)} .. bb) ;
\stopMPcode
\stopbuffer

\typebuffer

\startlinecorrection[blank] \getbuffer \stoplinecorrection

According to \METAPOST, when we leave the Earth straight upwards and want a
smooth trajectory, we have to pass through outer space.

\startbuffer
\startMPcode
\includeMPgraphic{gamble} ;
Gamble(120,aa {(1,0)} .. bb) ;
\stopMPcode
\stopbuffer

\typebuffer

\startlinecorrection[blank] \getbuffer \stoplinecorrection

Given that we want a smooth path as well as a short journey, we can best follow
Mars' path. Here we face the risk that when we travel slower than Mars does, we
have a problem.

\startbuffer
\startMPcode
\includeMPgraphic{gamble} ;
Gamble(120,aa {dir 90} .. {precontrol 0 of b rotated 90} bb) ;
\stopMPcode
\stopbuffer

\typebuffer

\startlinecorrection[blank] \getbuffer \stoplinecorrection

We can even travel a shorter path when we leave Earth at the surface that faces
the point of arrival.

\startbuffer
\startMPcode
\includeMPgraphic{gamble} ;
Gamble(120,aa .. {precontrol 0 of b rotated 90} bb) ;
\stopMPcode
\stopbuffer

\typebuffer

\startlinecorrection[blank] \getbuffer \stoplinecorrection

In the end we decide that although the trajectories look impressive, we will not
trust our lives to \METAPOST. A beautiful path is not neccessarily a good path.
But even then, this macro provides a nice way to experiment with directions,
controls and tensions.

\stopsection

% \startsection[title={Animations}]
%
% {\em Although \METAPOST\ is not that well suited for free hand drawings, you can
% use it to make stylistics animations.}
%
% \stopsection

\startsection[title={Selective randomization}]

In this document we have used a lot of randomization. Because \CONTEXT\ often
needs multiple runs to sort out cross references, positions, tables of contents,
and so on, being real random every run would result in endless runs to get things
right, because the size of graphics changes. This is prevented by storing the
random seed betweeen runs. You can remove the \type {tuc} file to get a new seed
(or run \type {context --purgeall}).

Here is another example of randomization. This time we only randomize the control
points so the main shape sort of remains intact which can be handy when you use
such random shapes around text but still want a predictable size.

\startbuffer
\startMPcode
fill fullcircle scaled 2cm
    randomizedcontrols 0.1cm
    withcolor darkred
    withtransparency (1,.5) ;
fill ((1cm,0)--(0,1cm)--(-1cm,0)--cycle)
    randomizedcontrols 0.1cm
    withcolor darkblue
    withtransparency (1,.5) ;
\stopMPcode
\stopbuffer

\typebuffer

\startlinecorrection[blank] \getbuffer \stoplinecorrection

\startbuffer
\startMPcode
draw image (
    fill fullcircle scaled 2cm
        withcolor darkred
        withtransparency (1,.5) ;
    fill ((1cm,0)--(0,1cm)--(-1cm,0)--cycle)
        withcolor darkblue
        withtransparency (1,.5) ;
) randomizedcontrols 0.1cm ;
\stopMPcode
\stopbuffer

\typebuffer

\startlinecorrection[blank] \getbuffer \stoplinecorrection

\stopsection

\startsection[title=Snapping]

There are quite some helpers in \METAFUN\ and I must admit that I forgot
about most. Some just ended up in the core because they can be useful, others
serve as illustration. Here's one of them: \type {snapped}. First we define
a few helpers that we then use to check out a few shapes.

\startbuffer
\startMPdefinitions
def ShowSnapGrid(text shape) =
    fill (shape xsized 77mm) withcolor white/3 ;
    draw image (
        for i=10mm step 5mm until 100mm :
            draw fullsquare scaled i ;
        endfor ;
    ) withcolor 2white/3 ;
    drawpoints (shape xsized 77mm) withcolor black ;
enddef ;

vardef SnapShape expr shape =
    image (
        draw       shape ;
        drawpoints shape ;
    )
enddef ;

vardef ShowSnapShape expr shape =
    ShowSnapGrid(shape);

    draw SnapShape(shape xsized 77mm snapped -5mm      ) withcolor red ;
    draw SnapShape(shape xsized 77mm snapped  5mm      ) withcolor red ;
    draw SnapShape(shape xsized 77mm snapped (5mm,10mm)) withcolor green ;
    draw SnapShape(shape xsized 77mm snapped (5mm,15mm)) withcolor blue ;
    draw SnapShape(shape xsized 77mm snapped (5mm,20mm)) withcolor yellow ;
enddef ;
\stopMPdefinitions
\stopbuffer

\typebuffer \getbuffer

In \in {figures} [fig:snapper:1], \in [fig:snapper:2] and \in [fig:snapper:3] we
see how the original shape gets snapped on the grid. Of course in more complex
images the direction of the snapping can change the result in an unwanted way,
like overlapping shapes that obscure others, but normally this snapping is only
useful for simple predictable cases (like title pages).

\startplacefigure[reference=fig:snapper:1,title={Messing with \type{fullcircle}.}]
\startMPcode
    ShowSnapShape(fullcircle);
\stopMPcode
\stopplacefigure

\startplacefigure[reference=fig:snapper:2,title={\type{fullsquare}}]
\startMPcode
    ShowSnapShape(fullsquare);
\stopMPcode
\stopplacefigure

\startplacefigure[reference=fig:snapper:3,title={\type{fulltriangle}}]
\startMPcode
    ShowSnapShape(fulltriangle);
\stopMPcode
\stopplacefigure

\stopsection

\startsection[title=Arrowheads]

Arrows are actually drawn quite well in \METAPOST, as the arrowheads nicely adapt
to the direction of the point where the arrowhead is attached. There are however
some limitations as the following examples demonstrate: arrows don't work well
with transparency and you can probably figure out why. Alan Braslau came up with
an extension that allows to set the dimple of the head. You can see all this
in \in {figure} [fig:arrowheads].

\startbuffer[a]
numeric unit ; unit := mm ;

drawoptions(withcolor .6blue withtransparency (1,.25)) ;

pickup pencircle scaled 2unit ; ahlength := 6unit ;

picture p ; p := image (
    drawarrow reverse fullcircle rotated - 5 scaled 50unit ;
    drawarrow reverse fullcircle rotated -10 scaled 30unit ;
) shifted ( -45unit, 0unit) ;

for i=0 step 90 until 360 : draw p rotated i ; endfor ;

currentpicture := currentpicture shifted - center currentpicture ;

p := currentpicture ; p := image (
    draw llcorner p -- center p ;
    drawarrow llcorner p -- 0.875[llcorner p,center p] ;
) ;

for i=0 step 90 until 360 : draw p rotated i ; endfor ;

clip currentpicture to boundingbox (fullcircle scaled 80unit) ;

if lua.mp.mode("screen") :
    currentpicture := currentpicture ysized .4TextHeight ;
else :
    currentpicture := currentpicture xsized .4TextWidth ;
fi ;
\stopbuffer

\typebuffer[a]

\startbuffer[b]
    resetarrows ;
\stopbuffer

\startbuffer[a1]
    ahvariant := 1 ;
\stopbuffer
\startbuffer[a2]
    ahvariant := 2 ;
\stopbuffer

\startbuffer[a3]
    ahvariant := 1 ; ahdimple  := 1/2 ;
\stopbuffer
\startbuffer[a4]
    ahvariant := 1 ; ahdimple  := 1 ;
\stopbuffer
\startbuffer[a5]
    ahvariant := 1 ; ahdimple  := 5/2 ;
\stopbuffer

\startplacefigure[reference=fig:arrowheads,title=The way arrowheads are constructed.]
    \doifelsemode {screen} {
        \setupcombination[nx=3,ny=2]
    } {
        \setupcombination[nx=2,ny=3]
    }
    \startcombination[distance=2em]
        {\processMPbuffer[a,   b]} {\tttf ahvariant=0}
        {\processMPbuffer[a1,a,b]} {\tttf ahvariant=1}
        {\processMPbuffer[a2,a,b]} {\tttf ahvariant=2}
        {\processMPbuffer[a3,a,b]} {\tttf ahvariant=1, ahdimple=1/2}
        {\processMPbuffer[a4,a,b]} {\tttf ahvariant=1, ahdimple=1}
        {\processMPbuffer[a5,a,b]} {\tttf ahvariant=1, ahdimple=5/2}
    \stopcombination
\stopplacefigure

\stopsection

\startsection[title=Teaser]

Sometimes, when playing with \METAPOST\ you run into interesting cases. Here is
one. The result is shown in \in {figure} [fig:teaser:1].

\startbuffer
\startusableMPgraphic{BackgroundTeaser}
    fill OverlayBox enlarged 1mm withcolor darkyellow ; % bleed
    path p ; p := OverlayBox enlarged  -5mm ;
    path q ; q := OverlayBox enlarged -10mm ;
    fill q withcolor white ;
    drawoptions(withcolor darkred) ;
    fill reverse topboundary    q -- topboundary    p -- cycle ;
    fill reverse bottomboundary q -- bottomboundary p -- cycle ;
    drawoptions(withcolor darkgreen) ;
    fill reverse leftboundary   q -- leftboundary   p -- cycle ;
    fill reverse rightboundary  q -- rightboundary  p -- cycle ;
\stopusableMPgraphic

\defineoverlay
  [BackgroundTeaser]
  [\useMPgraphic{BackgroundTeaser}]

\framed
  [frame=off,
   offset=15mm,
   background=BackgroundTeaser,
   align=normal]
  {\input knuth }
\stopbuffer

\typebuffer

\startplacefigure[reference=fig:teaser:1,title=Can you guess what happens here?]
    \getbuffer
\stopplacefigure

\stopsection

\startsection[title={Lists}]

For some specific purpose I needed to sort a list of paths and therefore
\METAFUN\ comes with a quick sort macro. Its working can be demonstrated by an
example.

\startbuffer[a]
pair p[], pp[] ; numeric n ; n := 25 ;
for i=1 upto n : p[i] := origin randomized 4cm ; endfor ;
\stopbuffer

\startbuffer[b]
copylist(p,pp) ; % unsorted
drawarrow listtolines(pp) shifted (  0,0) withcolor darkblue ;
\stopbuffer

\startbuffer[c]
copylist(p,pp) ; sortlist(pp)() ; % sorted
drawarrow listtolines(pp) shifted (300,0) withcolor darkyellow ;
\stopbuffer

\startbuffer[d]
copylist(p,pp) ; sortlist(pp)(xpart) ;
drawarrow listtolines(pp) shifted (100,0) withcolor darkred ;

\stopbuffer
\startbuffer[e]
copylist(p,pp) ; sortlist(pp)(ypart) ;
drawarrow listtolines(pp) shifted (200,0) withcolor darkgreen ;
\stopbuffer

\startbuffer[f]
vardef whow expr p = (xpart p + ypart p) enddef ;

copylist(p,pp) ; sortlist(pp)(whow) ;
drawarrow listtolines(pp) shifted (400,0) withcolor darkcyan ;
\stopbuffer

\startbuffer[g]
vardef whow expr p = (xpart p ++ ypart p) enddef ;

copylist(p,pp) ; sortlist(pp)(whow) ;
drawarrow listtolines(pp) shifted (500,0) withcolor darkmagenta ;
\stopbuffer

\typebuffer[a,b,c,d,e,f,g]

The result of this code is shown in \in {figure} [fig:sorting].

\startplacefigure[reference=fig:sorting,title={Using the sorter.}]
    \startcombination[3*2]
        {\processMPbuffer[a,b]} {\tttf unsorted}
        {\processMPbuffer[a,c]} {\tttf sorted}
        {\processMPbuffer[a,d]} {\tttf xpart}
        {\processMPbuffer[a,e]} {\tttf ypath}
        {\processMPbuffer[a,f]} {\tttf xpart p + ypart p}
        {\processMPbuffer[a,g]} {\tttf xpart p ++ ypart p}
    \stopcombination
\stopplacefigure

There is a helper that converts a list of paths into a shape that covers all
of them. In \in {figure} [fig:shapedlist] three shaped lists are shown.

\startbuffer[a]
    def ShowShape(expr e) =
        draw image (

            save p ; path p[] ;

            def MakeShape(expr i,w,h,x,y) =
                p[i] := e
                    xysized ((w,h) randomized (2mm,1mm))
                    shifted ((x,y) randomized (2mm,1mm)) ;
            enddef ;

            MakeShape(1,40mm,6mm,10mm,  0mm) ;
            MakeShape(2,50mm,5mm, 5mm,-10mm) ;
            MakeShape(3,20mm,8mm,30mm,-20mm) ;
            MakeShape(4,55mm,5mm,10mm,-30mm) ;
            MakeShape(5,55mm,5mm, 5mm,-50mm) ;

            save s ; path s ; s := shapedlist(p) ; drawarrow  s ;

            linejoin := butt ;

            for i=1 upto 5 :
                fill p[i] withcolor .75white withtransparency (1,.5) ;
                draw thetextext("\tttf " & decimal i, center p[i]) ;
            endfor ;

        ) ysized 4cm ;
    enddef ;
\stopbuffer

\typebuffer[a]

\startbuffer[b]
ShowShape(unitsquare)
\stopbuffer

\startbuffer[c]
ShowShape(unitcircle)
\stopbuffer

\startbuffer[d]
ShowShape(unittriangle)
\stopbuffer

\startplacefigure[reference=fig:shapedlist,title={The \type {shapedlist} macro returns the envelope that covers all the paths in the list.}]
    \startcombination[3*1]
        {\processMPbuffer[a,b]} {\tttf unitsquare}
        {\processMPbuffer[a,c]} {\tttf unitcircle}
        {\processMPbuffer[a,d]} {\tttf unittriangle}
    \stopcombination
\stopplacefigure

\stopsection

\startsection[title=Table cells]

Sometimes a standard \CONTEXT\ feature doesn't work out as expected. Take the
following table:

\startbuffer
\bTABLE[frame=on,framecolor=blue,rulethickness=1pt]
    \bTR
        \bTD test \eTD
        \bTD test \eTD
        \bTD test \eTD
        \bTD[framecolor=magenta] test \eTD
        \bTD test \eTD
        \bTD test \eTD
    \eTR
    \bTR
        \bTD test \eTD
        \bTD[framecolor=red] test \eTD
        \bTD test \eTD
        \bTD test \eTD
        \bTD test \eTD
        \bTD[framecolor=green] test \eTD
    \eTR
\eTABLE
\stopbuffer

\typebuffer

Because cells are drawn top|-|down and left|-|right a next cell border
overruns the previous one.

\startlinecorrection
\getbuffer
\stoplinecorrection

\startbuffer
\bTABLE[frame=on,framecolor=blue,rulethickness=1pt]
    \bTR
        \bTD test \eTD
        \bTD test \eTD
        \bTD test \eTD
        \bTD[framecolor=magenta,frameoffset=-.5pt] test \eTD
        \bTD test \eTD
        \bTD test \eTD
    \eTR
    \bTR
        \bTD test \eTD
        \bTD[framecolor=red,frameoffset=-.5pt] test \eTD
        \bTD test \eTD
        \bTD test \eTD
        \bTD test \eTD
        \bTD[framecolor=green,frameoffset=-.5pt] test \eTD
    \eTR
\eTABLE
\stopbuffer

You can try this:

\typebuffer

which gives us something that is not okay either for cells that touch an edge:

\startlinecorrection
\getbuffer
\stoplinecorrection

but we can cheat:

\startlinecorrection
\framed
  [offset=overlay,
   frameoffset=.5pt,
   framecolor=blue,
   rulethickness=1pt]
  {\getbuffer}
\stoplinecorrection

This is achieved by framing the whole table:

\starttyping
\framed
  [offset=overlay,
   frameoffset=.5pt,
   framecolor=blue,
   rulethickness=1pt]
  {...}
\stoptyping

\startbuffer
\startuseMPgraphic{cell:innerframe}{innercolor}
    draw OverlayBox enlarged -1.5OverlayLineWidth
        withpen pensquare scaled OverlayLineWidth
        withcolor \MPvar{innercolor} ;
\stopuseMPgraphic

\defineoverlay
  [innerframe]
  [{\uniqueMPgraphic{cell:innerframe}%
      {innercolor=\framedparameter{innercolor}}}]

\bTABLE[frame=on,framecolor=blue,rulethickness=1pt,innercolor=magenta]
    \bTR
        \bTD test \eTD
        \bTD test \eTD
        \bTD test \eTD
        \bTD[background=innerframe] test \eTD
        \bTD test \eTD
        \bTD test \eTD
    \eTR
    \bTR
        \bTD test \eTD
        \bTD[background=innerframe,innercolor=red] test \eTD
        \bTD test \eTD
        \bTD test \eTD
        \bTD test \eTD
        \bTD[background=innerframe,innercolor=green] test \eTD
    \eTR
\eTABLE
\stopbuffer

A \METAPOST\ alternative is also possible and it gives a bit nicer
interface too:

\typebuffer

We get:

\startlinecorrection
\getbuffer
\stoplinecorrection

\stopsection

\startsection[title=Educational]

I made this example long ago, when some family member had to learn tables by
heart. For some reason, at school, this is made into a complex issue, with tricks
and such, even it if only involves only a few numbers. My own experience (if I
remember right) was that some of these numbers are trivial, and that there is
quite some symmetry, so in practice only a quarter needs to be remembered. And,
assuming that you can easily deduct the trivial cases, one can just calculate the
rest if needed.

\startbuffer
\start \ttbf \startMPcode
  def MyDraw(expr i, j, c) =
    fill fullsquare shifted (i,j) withcolor c withtransparency (1,.5) ;
  enddef ;

  for i = 1 upto 10 :
  for j = 1 upto 10 : MyDraw( i,  -j, "middlered"   ) ; endfor ; endfor ;

  for j = 1 upto 10 : MyDraw( 1,  -j, "middleblue"  ) ; endfor ;
  for i = 1 upto 10 : MyDraw( i, -10, "middlegreen" ) ; endfor ;
  for i = 1 upto 10 : MyDraw( i,  -1, "middleyellow") ; endfor ;
  for j = 1 upto 10 : MyDraw(10,  -j, "middlecyan"  ) ; endfor ;
  for j = 1 upto 10 : MyDraw( 5,  -j, "middlegray"  ) ; endfor ;
  for j = 1 upto 10 : MyDraw( j,  -j, "middlegray"  ) ; endfor ;

  draw image ( for i = 1 upto 10 : for j = 1 upto 10 :
    draw textext(decimal (i*j)) ysized .25 shifted (i,-j) ;
  endfor ; endfor ; ) withcolor white ;

  currentpicture := currentpicture ysized 10cm ;
\stopMPcode \stop
\stopbuffer

\typebuffer

\startplacefigure[title=Overlapping glyphs,reference=fig:tentable]
    \getbuffer
\stopplacefigure

Here we use both transparency and colors to stress the reduction of cases. The
named colors resolve to ones defined at the \TEX\ end. We see the redndering
\in {figure} [fig:tentable].

\startsection[title=Glyph magic]

The next example is the result of a tread on the mailing list. After Henri Menke
posted a some glyph overlap code, I made a variant that more suited the way we do
it in \METAFUN. In the meantime Floris van Maanen had found out that some glyphs
need a more inventive solution so after that the code evolved. By then the \type
{outlinetext}, \type {drawoutlinetext} and \type {filloutlinetext} helpers had
been added to the code base.

Because this is complicated stuff, we just show the two solutions. The first one
is a relative simple one, the second one uses an approach suggested by Alan
Braslau and therefore uses some of the code that can be found in the \type
{crossingunder} macro.

\startbuffer
\startMPdefinitions
def ShowOverlapInOutlinesA(expr first, second) =
  path p_i, p_j, s_i, s_j ;
  numeric n_i, n_j, index ;
  pair found ;
  index := 0 ;
  for i within first :
    for j within second :
      p_i := pathpart i ; n_i := length(p_i) ;
      p_j := pathpart j ; n_j := length(p_j) ;
      for ii = 0 upto n_i - 1 :
        s_i := subpath(ii,ii+1) of p_i ;
        for jj = 0 upto n_j - 1 :
          s_j := subpath(jj,jj+1) of p_j ;
          found := s_i intersection_point s_j ;
          if intersection_found :
            index := index + 1 ;
            drawdot found
              withpen pencircle scaled 4 withcolor white ;
            draw textext("\strut\ttbf " & decimal index) ysized 3
              shifted found ;
          fi ;
        endfor ;
      endfor ;
    endfor ;
  endfor ;
enddef ;
\stopMPdefinitions
\stopbuffer

\typebuffer \getbuffer

This is the solution based on \type {crossingunder}, a macro that has been
introduced as part of Alan's neat node module.

\startbuffer
\startMPdefinitions
def ShowOverlapInOutlinesB(expr first, second) =
  begingroup ;
    save p, q, n, t, a, b, c, bcuttings, hold, found ;
    path p, q ;
    numeric n, hold ;
    path a, b, c, bcuttings ;
    pair found ;
    c := makepath(currentpen scaled crossingscale) ;
    for f within first :
      numeric t[];
      path hold[];
      t[0] := n := hold := 0 ;
      for s within second :
        p := pathpart f ;
        q := pathpart s ;
        a := p ;
        for i=1 upto crossingnumbermax : % safeguard
          clearxy ; z = a intersectiontimes q ;
          if x < 0 :
            exitif hold < 1 ;
            a := hold[hold] ; hold := hold - 1 ;
            clearxy ; z = a intersectiontimes q ;
          fi
          (t[incr n], whatever) = p intersectiontimes point x of a ;
          if x = 0 :
            a := a cutbefore c shifted point x of a ;
          elseif x = length a :
            a := a cutafter  c shifted point x of a ;
          else : % before or after?
            b := subpath (0,x)        of a cutafter  c shifted point x of a ;
            bcuttings := cuttings ;
            a := subpath (x,length a) of a cutbefore c shifted point x of a ;
            clearxy ; z = a intersectiontimes q ;
            if x < 0 :
              a := b ;
              cuttings := bcuttings ;
            elseif length bcuttings > 0 :
              clearxy ; z = b intersectiontimes q ;
              if x >= 0 :
                hold[incr hold] := b ;
              fi
            fi
          fi
          if length cuttings = 0 :
            exitif hold < 1 ;
            a := hold[hold] ; hold := hold - 1 ;
          fi
        endfor ;
      endfor ;
      t[incr n] = length p ;
      for i=1 upto n :
        found := point t[i] of p ;
        drawdot found
          withpen pencircle scaled 4 withcolor white ;
        draw textext("\strut\ttbf " & decimal i) ysized 3
          shifted found ;
      endfor ;
    endfor ;
  endgroup ;
enddef ;
\stopMPdefinitions
\stopbuffer

\typebuffer \getbuffer

We demonstrate the differences with an example. The result can be seen in
\in {figure} [fig:overlapping:a].

\startbuffer
\startcombination
  {\startMPcode
      picture first, second ;
     first  := outlinetext.p("N") ; first  := first  scaled 10 ;
     second := outlinetext.p("T") ; second := second scaled 10 ;
     second := second rotatedaround(center second, 5) shifted (1,-1) ;
     filloutlinetext(first ) withcolor .5[darkblue,white] ;
     filloutlinetext(second) withcolor .5[darkred,white] ;
     drawoutlinetext(first ) ;
     drawoutlinetext(second) ;
     ShowOverlapInOutlinesA(first, second) ;
     addbackground withcolor darkgray ;
     currentpicture := currentpicture scaled 2.5 ;
   \stopMPcode} {Method A}
   {\startMPcode
      picture first, second ;
      first  := outlinetext.p("N") ; first  := first  scaled 10 ;
      second := outlinetext.p("T") ; second := second scaled 10 ;
      second := second rotatedaround(center second, 5) shifted (1,-1) ;
      filloutlinetext(first ) withcolor .5[darkgreen,white] ;
      filloutlinetext(second) withcolor .5[darkyellow,white] ;
      drawoutlinetext(first ) ;
      drawoutlinetext(second) ;
      ShowOverlapInOutlinesB(first, second) ;
      addbackground withcolor darkgray ;
      currentpicture := currentpicture scaled 2.5 ;
    \stopMPcode} {Method B}
\stopcombination
\stopbuffer

\typebuffer

\startplacefigure[title=Overlapping glyphs,reference=fig:overlapping:a]
    \getbuffer
\stopplacefigure

We duplicate some code because the pictures will change in the process of
analyzing. Let's make a helper for that:

\startbuffer
\startMPdefinitions
def ShowOverlap(expr f, s, m) =
  picture first, second ;
  first  := outlinetext.p(f) ; first  := first  scaled 10 ;
  second := outlinetext.p(s) ; second := second scaled 10 ;

  filloutlinetext(first ) withcolor darkblue ;
  drawoutlinetext(first ) ;

  filloutlinetext(second) withcolor darkred  ;
  drawoutlinetext(second) ;

  if m == 2 :
    ShowOverlapInOutlinesB
  else :
    ShowOverlapInOutlinesA
  fi (first, second) ;

  addbackground withcolor darkgray ;
  currentpicture := currentpicture ysized 4cm ;
enddef ;
\stopMPdefinitions
\stopbuffer

\typebuffer \getbuffer

Again we demonstrate the differences with some examples. The result can be seen in
\in {figure} [fig:overlapping:b].

\startbuffer
  \startcombination[nx=3,ny=2]
    {\startMPcode ShowOverlap("N","T",1)                 ; \stopMPcode} {Method 1}
    {\startMPcode ShowOverlap("\$","Q",1)                ; \stopMPcode} {Method 1}
    {\startMPcode ShowOverlap("\tttf ABC","\tttf PQR",1) ; \stopMPcode} {Method 1}
    {\startMPcode ShowOverlap("N","T",2)                 ; \stopMPcode} {Method 2}
    {\startMPcode ShowOverlap("\$","Q",2)                ; \stopMPcode} {Method 2}
    {\startMPcode ShowOverlap("\tttf ABC","\tttf PQR",2) ; \stopMPcode} {Method 2}
  \stopcombination
\stopbuffer

\typebuffer

\startplacefigure[title=Overlapping glyphs,reference=fig:overlapping:b]
    \getbuffer
\stopplacefigure

\stopsection

\stopchapter

\stopcomponent
