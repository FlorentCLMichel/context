% language=us

\environment luametafun-style

\startcomponent luametafun-outline

\startchapter[title={Outline}]

In a regular text you can have outline characters by setting a (pseudo) font
feature but sometimes you want to play a bit more with this. In \METAFUN\ we
always had that option. In \MKII\ we call \type {pstoedit} to turn text into
outlines, in \MKIV\ we do that by manipulating the shapes directly. And, as with
some other extensions, in \LMTX\ a new interface has been added, but the
underlying code is the same as in \MKIV.

\startbuffer[1a]
\startMPcode{doublefun}
    draw lmt_outline [
        text      = "hello"
        kind      = "draw",
        drawcolor = "darkblue",
    ] xsized .45TextWidth ;
\stopMPcode
\stopbuffer

\startbuffer[1b]
\startMPcode{doublefun}
    draw lmt_outline [
        text          = "hello",
        kind          = "both",
        fillcolor     = "middlegray",
        drawcolor     = "darkgreen",
        rulethickness = 1/5,
    ] xsized .45TextWidth ;
\stopMPcode
\stopbuffer

In \in {figure} [outline:1] we see two examples:

\typebuffer[1a][option=TEX]

and

\typebuffer[1b][option=TEX]

\startplacefigure[reference=outline:1,title={Drawing and|/|or filling an outline.}]
    \startcombination
        {\getbuffer[1a]} {\type }
        {\getbuffer[1b]} {\type }
    \stopcombination
\stopplacefigure

Normally the fill ends up below the draw but we can reverse the order, as in
\in {figure} [outline:2], where we coded the leftmost example as:

\startbuffer[2a]
\startMPcode{doublefun}
    draw lmt_outline [
        text          = "hello",
        kind          = "reverse",
        fillcolor     = "darkred",
        drawcolor     = "darkblue",
        rulethickness = 1/2,
    ] xsized .45TextWidth ;
\stopMPcode
\stopbuffer

\startbuffer[2b]
\startMPcode{doublefun}
    draw lmt_outline [
        text          = "hello",
        kind          = "both",
        fillcolor     = "darkred",
        drawcolor     = "darkblue",
        rulethickness = 1/2,
    ] xsized .45TextWidth ;
\stopMPcode
\stopbuffer

\typebuffer[2a][option=TEX]

\startplacefigure[reference=outline:2,title={Reversing the order of drawing and filling.}]
    \startcombination
        {\getbuffer[2a]} {\type }
        {\getbuffer[2b]} {\type }
    \stopcombination
\stopplacefigure

It is possible to fill and draw in one operation, in which case the same color is
used for both, see \in {figure} [outline:3] for an example fo this. This is a low
level optimization where the shape is only output once.

\startbuffer[3a]
\startMPcode{doublefun}
    draw lmt_outline [
        text          = "hello",
        kind          = "fillup",
        fillcolor     = "darkgreen",
        rulethickness = 1/5,
    ] xsized .45TextWidth ;
\stopMPcode
\stopbuffer

\startbuffer[3b]
\startMPcode{doublefun}
    draw lmt_outline [
        text          = "hello",
        kind          = "fill",
        fillcolor     = "darkgreen",
        rulethickness = 1/5,
    ] xsized .45TextWidth ;
\stopMPcode
\stopbuffer

\startplacefigure[reference=outline:3,title={Combining a fill with a draw in the same color.}]
    \startcombination
        {\getbuffer[3a]} {\type }
        {\getbuffer[3b]} {\type }
    \stopcombination
\stopplacefigure


This interface is much nicer than the one where each variant (the parameter \type
{kind} above) had its own macro due to the need to group properties of the
outline and fill. Let's show some more:

\startbuffer[4]
\startMPcode{doublefun}
    draw lmt_outline [
        text      = "\obeydiscretionaries\samplefile{tufte}",
        align     = "normal",
        kind      = "draw",
        drawcolor = "darkblue",
    ] xsized TextWidth ;
\stopMPcode
\stopbuffer

\typebuffer[4][option=TEX]

In this case we feed the text into the \type {\framed} macro so that we get a
properly aligned paragraph of text, as demonstrated in \in {figure} [outline:4]
\in {and} [outline:5]. If you want more trickery you can of course use any
\CONTEXT\ command (including \type {\framed} with all kind of options) in the
text.

\startplacefigure[reference=outline:4,title={Outlining a paragraph of text.}]
    \getbuffer[4]
\stopplacefigure

\startbuffer[5]
\startMPcode{doublefun}
    draw lmt_outline [
        text      = "\obeydiscretionaries\samplefile{ward}",
        align     = "normal,tolerant",
        style     = "bold",
        width     = 10cm,
        kind      = "draw",
        drawcolor = "darkblue",
    ] xsized TextWidth ;
\stopMPcode
\stopbuffer

\typebuffer[5][option=TEX]

\startplacefigure[reference=outline:4,title={Outlining a paragraph of text with a specific width.}]
    \getbuffer[5]
\stopplacefigure

We summarize the parameters:

\starttabulate[|T|T|T|p|]
\FL
\BC name          \BC type    \BC default \BC comment \NC \NR
\ML
\NC text          \NC string  \NC         \NC \NC \NR
\NC kind          \NC string  \NC draw    \NC One of \type {draw}, \type {fill}, \type {both}, \type {reverse} and \type {fillup}. \NC \NR
\NC fillcolor     \NC string  \NC         \NC \NC \NR
\NC drawcolor     \NC string  \NC         \NC \NC \NR
\NC rulethickness \NC numeric \NC 1/10    \NC \NC \NR
\NC align         \NC string  \NC         \NC \NC \NR
\NC style         \NC string  \NC         \NC \NC \NR
\NC width         \NC numeric \NC         \NC \NC \NR
\LL
\stoptabulate

\stopchapter

\stopcomponent
