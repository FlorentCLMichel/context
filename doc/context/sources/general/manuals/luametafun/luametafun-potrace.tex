% language=us runpath=texruns:manuals/luametafun

\environment luametafun-style

\startcomponent luametafun-potrace

\startchapter[title={Potrace}]

\startsection[title=Introduction]

The potrace connection targets at bitmaps. You can think of logos that only exist
as bitmaps while outlines are preferred, but in this case we actually think more
of bitmaps that the user lays out. In order to give an impression what we are
talking about I give three simple examples:

\startbuffer[potraced]
01111111111111111111111111111100
11000000000000000000000000000110
11000000000000000000000000000011
11000000000000000000000000000011
11000000000000000000000000000011
01100000000000000000000000000011
00111111111111111111111111111110
\stopbuffer

\savebuffer[list=potraced,file=potraced.txt,prefix=no]

\startbuffer[1]
\startMPcode
    fill
        lmt_potraced [ bytes =
            "01111111111111111111111111111100
             11000000000000000000000000000110
             11000000000000000000000000000011
             11000000000000000000000000000011
             11000000000000000000000000000011
             01100000000000000000000000000011
             00111111111111111111111111111110",
        ] ysized 1cm
        withcolor "darkblue"
        withpen pencircle scaled 1 ;
\stopMPcode
\stopbuffer

\startbuffer[2]
\startMPcode
    fill
        lmt_potraced [
            filename = "potraced.txt",
        ] ysized 1cm
        withcolor "darkgreen"
        withpen pencircle scaled 1 ;
\stopMPcode
\stopbuffer

\startbuffer[3]
\startMPcode
    fill
        lmt_potraced [
            buffer = "potraced",
        ] ysized 1cm
        withcolor "darkred"
        withpen pencircle scaled 1 ;
\stopMPcode
\stopbuffer

\startlinecorrection
\startcombination[nx=3,ny=1]
    {\getbuffer[1]} {\type {bytes}}
    {\getbuffer[2]} {\type {buffer}}
    {\getbuffer[3]} {\type {filename}}
\stopcombination
\stoplinecorrection

Here we vectorize bitmaps with Peter Selingers potrace library, that we built in
\LUAMETATEX. We can directly feed bytes in a \METAFUN\ blob:

\typebuffer[1][option=TEX]

But we can also go via a file that has the same data:

\typebuffer[2][option=TEX]

Of course we can also use buffers:

\typebuffer[3][option=TEX]

You feed a bitmap specification and get back a \METAPOST\ path, likely multiple
subpaths sewed together. You can of course draw and fill that path, or store it
in a path variable and then do both.

In the following sections we will explore the various options and some tricks.
The main message in this section is that you need to look at bitmaps with
vectorized eyes because that is what you get in the end: a vector representation.

\stopsection

\startsection[title=Functions]

{\em todo}

\stopsection

\startsection[title=Icons]

When Mikael Sundqvist and I were playing with potrace in \METAFUN\ his girls came
up with this pattern.

\startbuffer
\startMPcode
fill
    lmt_potraced [ bytes =
        "001111100
         010000010
         100000001
         101101101
         100000001
         101000101
         100111001
         010000010
         001111100",
         size = 1,
    ] xysized (3cm,3cm)
    withcolor "middleorange" ;
\stopMPcode
\stopbuffer

\typebuffer[option=TEX]

This produces the following icon. The somewhat asymmetrical shape gives it a
charm, and it is surprising how little code is needed. This picture inspired
Willi Egger to make a ten by ten composition gadget for the attendants of the
2023 \CONTEXT\ meeting that was used in a tutorial.

\startlinecorrection \getbuffer \stoplinecorrection

We use this to demonstrate a few more features of the interface:

\startbuffer
\startMPcode
draw
    lmt_potraced [ bytes =
        "..11111..
         .1.....1.
         1.......1
         1.11.11.1
         1.......1
         1.1...1.1
         1..111..1
         .1.....1.
         ..11111..",
         polygon = true,
         size = 1,
    ] xysized (3cm,3cm)
    withcolor "darkblue"
    withpen pencircle scaled 1mm ;
\stopMPcode
\stopbuffer

\typebuffer[option=TEX]

This contour is actually accurate:

\startlinecorrection \getbuffer \stoplinecorrection

We can color some components:

\startbuffer
\startMPcode
draw image (
    lmt_startpotraced [ bytes =
        "..11111..
         .1.....1.
         1.......1
         1.22.22.1
         1.......1
         1.3...3.1
         1..333..1
         .1.....1.
         ..11111.."
    ] ;
    fill lmt_potraced [ value = "1", size = 1 ]
        withcolor "darkred" ;
    fill lmt_potraced [ value = "3", size = 1 ]
        withcolor "darkgreen" ;
    fill lmt_potraced [ value = "2", size = 0 ]
        withcolor "darkblue" ;
    lmt_stoppotraced ;
) xysized (3cm,3cm) ;
\stopMPcode
\stopbuffer

\typebuffer[option=TEX]

Of course there must be enough distinction (white space) between the
shapes:

\startlinecorrection \getbuffer \stoplinecorrection

Again we show the polygons:

\startbuffer
\startMPcode
draw image (
    lmt_startpotraced [ bytes =
        "..11111..
         .1.....1.
         1.......1
         1.22.22.1
         1.......1
         1.3...3.1
         1..333..1
         .1.....1.
         ..11111.."
    ] ;
    draw lmt_potraced [ value = "1", size = 1, polygon = true ]
        withcolor "darkred" ;
    draw lmt_potraced [ value = "3", size = 1, polygon = true ]
        withcolor "darkgreen" ;
    draw lmt_potraced [ value = "2", size = 0, polygon = true ]
        withcolor "darkblue" ;
    lmt_stoppotraced ;
)
    xysized (3cm,3cm)
    withpen pencircle scaled 1mm ;
\stopMPcode
\stopbuffer

\typebuffer[option=TEX]

Gives:

\startlinecorrection \getbuffer \stoplinecorrection

We can do the same with data defined in \LUA:

\startbuffer
\startluacode
io.savedata("temp.txt",[[
..11111..
.1.....1.
1.......1
1.22.22.1
1.......1
1.3...3.1
1..333..1
.1.....1.
..11111..
]])
\stopluacode
\stopbuffer

\typebuffer[option=TEX]

With:

\startbuffer
\startMPcode
draw image (
    lmt_startpotraced [ filename = "temp.txt" ] ;
        fill lmt_potraced [ value = "1", size = 1 ]
            withcolor "darkcyan" ;
        fill lmt_potraced [ value = "3", size = 1 ]
            withcolor "darkmagenta" ;
        fill lmt_potraced [ value = "2", size = 0 ]
            withcolor "darkyellow" ;
    lmt_stoppotraced ;
) xysized (3cm,3cm) ;
\stopMPcode
\stopbuffer

\typebuffer[option=TEX]

Indeed we get:

\startlinecorrection \getbuffer \stoplinecorrection

\stopsection

\startsection[title=Fonts]

{\em maybe}

\stopsection

\stopchapter

\stopcomponent

