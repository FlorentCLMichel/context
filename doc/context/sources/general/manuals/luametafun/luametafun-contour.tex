% language=us

\environment luametafun-style

\startcomponent luametafun-contour

\startchapter[title={Contour}]

This feature started out as experiment triggered by a request on the mailing
list. In the end it was a nice exploration of what is possible with a bit of
\LUA. In a sense it is more subsystem than a simple \METAPOST\ macro because
quite some \LUA\ code is involved and more might be used in the future. It's part
of the fun.

A contour is a line through equivalent values $z$ that result from applying a
function to two variables $x$ and $y$. There is quite a bit of analysis needed
to get these lines. In \METAFUN\ we currently support three methods for generating
a colorful background and three for putting lines on top:

One solution is to use the the isolines and isobands methods are described on the
marching squares page of wikipedia:

\starttyping
https://en.wikipedia.org/wiki/Marching_squares
\stoptyping

This method is relative efficient as we don't do much optimization, simply
because it takes time and the gain is not that much relevant. Because we support
filling of multiple curves in one go, we get efficient paths anyway without side
effects that normally can occur from many small paths alongside. In these days of
multi megabyte movies and sound clips a request of making a \PDF\ file small is kind
of strange anyway. In practice the penalty is not that large.

As background we can use a bitmap. This method is also quite efficient because we
use indexed colors which results in a very good compression. We use a simple
mapping on a range of values.

A third method is derived from the one that is distributed as \CCODE\ source
file at:

\starttyping
https://physiology.arizona.edu/people/secomb/contours
https://github.com/secomb/GreensV4
\stoptyping

We can create a background image, which uses a sequence of closed curves
\footnote {I have to figure out how to improve it a bit so that multiple path
don't get connected.}. It can also provide two variants of lines around the
contours (we tag them shape and shade). It's all a matter of taste. In the
meantime I managed to optimize the code a bit and I suppose that when I buy a new
computer (the code was developed on an 8 year old machine) performance is
probably acceptable.

In order of useability you can think of isoband (band) with isolines (cell),
bitmap (bitmap) with isolines (cell) and finally shapes (shape) with edges
(edge). But let's start with a couple of examples.

\startbuffer[1]
\startMPcode{doublefun}
    draw lmt_contour [
        xmin =  0, xmax = 4*pi, xstep = .05,
        ymin = -6, ymax = 6,    ystep = .05,

        levels     = 7,
        height     = 5cm,
        preamble   = "local sin, cos = math.sin, math.cos",
        function   = "cos(x) + sin(y)",
        background = "bitmap",
        foreground = "edge",
        linewidth  = 1/2,
        cache      = true,
    ] ;
\stopMPcode
\stopbuffer

\startplacefigure[reference=contour:1]
    \getbuffer[1]
\stopplacefigure

\typebuffer[1][option=TEX]

In \in {figure} [contour:1] we see the result. There is a in this case black and
white image generated and on top of that we see lines. The step determines the
resolution of the image. In practice using a bitmap is quite okay and also rather
efficient: we use an indexed colorspace and, as already was mentioned, because
the number of colors is limited such an image compresses well. A different
rendering is seen in \in {figure} [contour:2] where we use the shape method for
the background. That method creates outlines but is much slower, and when you use
a high resolution (small step) it can take quite a while to identify the shapes.
This is why we set the cache flag.

\startbuffer[2]
\startMPcode{doublefun}
    draw lmt_contour [
        xmin =  0, xmax = 4*pi, xstep = .10,
        ymin = -6, ymax = 6,    ystep = .10,

        levels     = 7,
        preamble   = "local sin, cos = math.sin, math.cos",
        function   = "cos(x) - sin(y)",
        background = "shape",
        foreground = "shape",
        linewidth  = 1/2,
        cache      = true,
    ] ;
\stopMPcode
\stopbuffer

\typebuffer[2][option=TEX]

\startplacefigure[reference=contour:2]
    \getbuffer[2]
\stopplacefigure

We mentioned colorspace but haven't seen any color yet, so let's set some in \in
{figure} [contour:3]. Two variants are shown: a background \type {shape} with
foreground \type {shape} and a background \type {bitmap} with a foreground \type
{edge}. The bitmap renders quite fast, definitely when we compare with the shape,
while the quality is as good at this size.

\startbuffer[3a]
\startMPcode{doublefun}
    draw lmt_contour [
        xmin = -10, xmax = 10, xstep =  .1,
        ymin = -10, ymax = 10, ystep =  .1,

        levels     = 10,
        height     = 7cm,
        color      = "shade({1/2,1/2,0},{0,0,1/2})",
        function   = "x^2 + y^2",
        background = "shape",
        foreground = "shape",
        linewidth  = 1/2,
        cache      = true,
    ] xsized .45TextWidth ;
\stopMPcode
\stopbuffer

\startbuffer[3b]
\startMPcode{doublefun}
    draw lmt_contour [
        xmin = -10, xmax = 10, xstep =  .1,
        ymin = -10, ymax = 10, ystep =  .1,

        levels     = 10,
        height     = 7cm,
        color      = "shade({1/2,0,0},{0,0,1/2})",
        function   = "x^2 + y^2",
        background = "bitmap",
        foreground = "edge",
        linewidth  = 1/2,
        cache      = true,
    ] xsized .45TextWidth ;
\stopMPcode
\stopbuffer

\typebuffer[3a][option=TEX]

\startplacefigure[reference=contour:3]
    \startcombination
        {\getbuffer[3a]} {\bf shape}
        {\getbuffer[3b]} {\bf bitmap}
    \stopcombination
\stopplacefigure

We use the \type {doublefun} instance because we need to be sure that we don't
run into issues with scaled numbers, the default model in \METAPOST. The
function that gets passed is {\em not} using \METAPOST\ but \LUA, so basically
you can do very complex things. Here we directly pass code, but you can for
instance also do this:

\starttyping[option=TEX]
\startluacode
    function document.MyContourA(x,y)
        return x^2 + y^2
    end
\stopluacode
\stoptyping

and then \type {function = "document.MyContourA(x,y)"}. As long as the function
returns a valid number we're okay. When you pass code directly you can use the
\type {preamble} key to set local shortcuts. In the previous examples we took
\type {sin} and \type {cos} from the math library but you can also roll out your
own functions and|/|or use the more elaborate \type {xmath} library. The color
parameter is also a function, one that returns one or three arguments. In the
next example we use \type {lin} to calculate a fraction of the current level and
total number of levels.

\startbuffer[4a]
\startMPcode{doublefun}
    draw lmt_contour [
        xmin = -3, xmax = 3, xstep =  .01,
        ymin = -1, ymax = 1, ystep =  .01,

        levels     = 10,
        default    = .5,
        height     = 5cm,
        function   = "x^2 + y^2 + x + y/2",
        color      = "lin(l), 0, 1/2",
        background = "bitmap"
        foreground = "none",
        cache      = true,
    ] xsized TextWidth ;
\stopMPcode
\stopbuffer

\typebuffer[4a][option=TEX]

\startplacefigure[reference=contour:4a]
    \getbuffer[4a]
\stopplacefigure

Instead of a bitmap we can use an isoband, which boils down to a set of tiny
shapes that make up a bigger one. This is shown in \in {figure} [contour:4b].

\startbuffer[4b]
\startMPcode{doublefun}
    draw lmt_contour [
        xmin = -3, xmax = 3, xstep =  .01,
        ymin = -1, ymax = 1, ystep =  .01,

        levels     = 10,
        default    = .5,
        height     = 5cm,
        function   = "x^2 + y^2 + x + y/2",
        color      = "lin(l), 1/2, 0",
        background = "band",
        foreground = "none",
        cache      = true,
    ] xsized TextWidth ;
\stopMPcode
\stopbuffer

\typebuffer[4b][option=TEX]

\startplacefigure[reference=contour:4b]
    \getbuffer[4b]
\stopplacefigure

You can draw several functions and see where they overlap:

\startbuffer[5]
\startMPcode{doublefun}
    draw lmt_contour [
        xmin = -pi, xmax = 4*pi, xstep = .1,
        ymin = -3,  ymax = 3,    ystep = .1,

        range      = { -.1, .1 },
        preamble   = "local sin, cos = math.sin, math.cos",
        functions  = {
            "sin(x) + sin(y)", "sin(x) + cos(y)",
            "cos(x) + sin(y)", "cos(x) + cos(y)"
        },
        background = "bitmap",
        linecolor  = "black",
        linewidth  = 1/10,
        color      = "shade({1,1,0},{0,0,1})"
        cache      = true,
    ] xsized TextWidth ;
\stopMPcode
\stopbuffer

\typebuffer[5][option=TEX]

\startplacefigure[reference=contour:5]
    \getbuffer[5]
\stopplacefigure

The range determines the $z$ value(s) that we take into account. You can also
pass a list of colors to be used. In \in {figure} [contour:6] this is
demonstrated. There we also show a variant foreground \type {cell}, which uses a
bit different method for calculating the edges. \footnote {This a bit of a
playground: more variants might show up in due time.}

\startbuffer[6]
\startMPcode{doublefun}
    draw lmt_contour [
        xmin = -2*pi, xmax =  2*pi, xstep = .01,
        ymin = -3,    ymax =  3,    ystep = .01,

        range      = { -.1, .1 },
        preamble   = "local sin, cos = math.sin, math.cos",
        functions  = { "sin(x) + sin(y)", "sin(x) + cos(y)" },
        background = "bitmap",
        foreground = "cell",
        linecolor  = "white",
        linewidth  = 1/10,
        colors     = { (1/2,1/2,1/2), red, green, blue }
        level      = 3,
        linewidth  = 6,
        cache      = true,
    ] xsized TextWidth ;
\stopMPcode
\stopbuffer

\typebuffer[6][option=TEX]

Here the number of levels depends on the number of functions as each can overlap
with another; for instance the outcome of two functions can overlap or not which
means 3 cases, and with a value not being seen that gives 4 different cases.

\startplacefigure[reference=contour:6]
    \getbuffer[6]
\stopplacefigure

\startbuffer[7]
\startMPcode{doublefun}
    draw lmt_contour [
        xmin = -2*pi, xmax =  2*pi, xstep = .01,
        ymin = -3,    ymax =  3,    ystep = .01,

        range      = { -.1, .1 },
        preamble   = "local sin, cos = math.sin, math.cos",
        functions  = {
            "sin(x) + sin(y)",
            "sin(x) + cos(y)",
            "cos(x) + sin(y)",
            "cos(x) + cos(y)"
        },
        background = "bitmap",
        foreground = "none",
        level      = 3,
        color      = "shade({2/3,0,0},{2/3,1,2/3})"
        cache      = true,
    ] xsized TextWidth ;
\stopMPcode
\stopbuffer

\typebuffer[7][option=TEX]

Of course one can wonder how useful showing many functions but it can give nice
pictures, as shown in \in {figure} [contour:7].

\startplacefigure[reference=contour:7]
    \getbuffer[7]
\stopplacefigure

\startbuffer[8]
\startMPcode{doublefun}
    draw lmt_contour [
        xmin = -2*pi, xmax =  2*pi, xstep = .01,
        ymin = -3,    ymax =  3,    ystep = .01,

        range      = { -.3, .3 },
        preamble   = "local sin, cos = math.sin, math.cos",
        functions  = {
            "sin(x) + sin(y)",
            "sin(x) + cos(y)",
            "cos(x) + sin(y)",
            "cos(x) + cos(y)"
        },
        background = "bitmap",
        foreground = "none",
        level      = 3,
        color      = "shade({1,0,0},{0,1,0})"
        cache      = true,
    ] xsized TextWidth ;
\stopMPcode
\stopbuffer

\typebuffer[8][option=TEX]

We can enlargen the window, which is demonstrated in \in {figure} [contour:8]. I
suppose that such images only make sense in educational settings.

\startplacefigure[reference=contour:8]
    \getbuffer[8]
\stopplacefigure

% \startbuffer[9a]
% \startMPcode{doublefun}
%     draw lmt_contour [
%         xmin = -10, xmax = 10, xstep = 1,
%         ymin = -10, ymax = 10, ystep = 1,
%         function  = "math.random(1,3)", levels    = 3,
%         linecolor = "white",            linewidth = 1/10,
%         width     = .3TextWidth,        legend    = "none",
%         color     = "shade({1/2,1/2,0},{0,0,1})"
%
%         background = "bitmap", foreground = "edge",
%     ] ;
% \stopMPcode
% \stopbuffer
%
% \typebuffer[9a][option=TEX]
%
% \startbuffer[9b]
% \startMPcode{doublefun}
%     draw lmt_contour [
%         xmin = -10, xmax = 10, xstep = 1,
%         ymin = -10, ymax = 10, ystep = 1,
%         function  = "math.random(1,3)", levels    = 3,
%         linecolor = "white",            linewidth = 1/10,
%         width     = .3TextWidth,        legend    = "none",
%         color     = "shade({1/2,1/2,0},{0,0,1/2})"
%
%         background = "bitmap", foreground = "cell",
%     ] ;
% \stopMPcode
% \stopbuffer
%
% \startbuffer[9c]
% \startMPcode{doublefun}
%     draw lmt_contour [
%         xmin = -10, xmax = 10, xstep = 1,
%         ymin = -10, ymax = 10, ystep = 1,
%         function  = "math.random(1,3)", levels    = 3,
%         linecolor = "white",            linewidth = 1/10,
%         width     = .3TextWidth,        legend    = "none",
%         color     = "shade({1/2,1/2,0},{0,0,1})"
%
%         background = "bitmap", foreground = "none",
%     ] ;
% \stopMPcode
% \stopbuffer
%
% \startbuffer[9d]
% \startMPcode{doublefun}
%     draw lmt_contour [
%         xmin = -10, xmax = 10, xstep = 1,
%         ymin = -10, ymax = 10, ystep = 1,
%         function  = "math.random(1,3)", levels    = 3,
%         linecolor = "white",            linewidth = 1/10,
%         width     = .3TextWidth,        legend    = "none",
%         color     = "shade({1/2,1/2,0},{0,0,1})"
%
%         background = "shape", foreground = "shape",
%     ] ;
% \stopMPcode
% \stopbuffer
%
% \startbuffer[9e]
% \startMPcode{doublefun}
%     draw lmt_contour [
%         xmin = -10, xmax = 10, xstep = 1,
%         ymin = -10, ymax = 10, ystep = 1,
%         function  = "math.random(1,3)", levels    = 3,
%         linecolor = "white",            linewidth = 1/10,
%         width     = .3TextWidth,        legend    = "none",
%         color     = "shade({1/2,1/2,0},{0,0,1})"
%
%         background = "shape", foreground = "edge",
%     ] ;
% \stopMPcode
% \stopbuffer
%
% \startbuffer[9f]
% \startMPcode{doublefun}
%     draw lmt_contour [
%         xmin = -10, xmax = 10, xstep = 1,
%         ymin = -10, ymax = 10, ystep = 1,
%         function  = "math.random(1,3)", levels    = 3,
%         linecolor = "white",            linewidth = 1/10,
%         width     = .3TextWidth,        legend    = "none",
%         color     = "shade({1/2,1/2,0},{0,0,1})"
%
%         background = "shape", foreground = "none",
%     ] ;
% \stopMPcode
% \stopbuffer
%
% \startbuffer[9g]
% \startMPcode{doublefun}
%     draw lmt_contour [
%         xmin = -10, xmax = 10, xstep = 1,
%         ymin = -10, ymax = 10, ystep = 1,
%         function  = "math.random(1,3)", levels    = 3,
%         linecolor = "white",            linewidth = 1/10,
%         width     = .3TextWidth,        legend    = "none",
%         color     = "shade({1/2,1/2,0},{0,0,1})"
%
%         background = "band", foreground = "edge",
%     ] ;
% \stopMPcode
% \stopbuffer
%
% \startbuffer[9h]
% \startMPcode{doublefun}
%     draw lmt_contour [
%         xmin = -10, xmax = 10, xstep = 1,
%         ymin = -10, ymax = 10, ystep = 1,
%         function  = "math.random(1,3)", levels    = 3,
%         linecolor = "white",            linewidth = 1/10,
%         width     = .3TextWidth,        legend    = "none",
%         color     = "shade({1/2,1/2,0},{0,0,1})"
%
%         background = "band", foreground = "cell",
%     ] ;
% \stopMPcode
% \stopbuffer
%
% \startbuffer[9i]
% \startMPcode{doublefun}
%     draw lmt_contour [
%         xmin = -10, xmax = 10, xstep = 1,
%         ymin = -10, ymax = 10, ystep = 1,
%         function  = "math.random(1,3)", levels    = 3,
%         linecolor = "white",            linewidth = 1/10,
%         width     = .3TextWidth,        legend    = "none",
%         color     = "shade({1/2,1/2,0},{0,0,1})"
%
%         background = "band", foreground = "none",
%     ] ;
% \stopMPcode
% \stopbuffer
%
% In \in {figure} [contour:9] we see that using the shape option doesn't work out
% too well here, which again demonstrates that using the bitmap method is not that
% bad. In that example we use random numbers, just to show the erratic behavior. In
% \in {figure} [contour:10] a more sane image is used. The band and bitmap examples
% are generated quite fast so no caching is used there. We only show one definition:
%
% \typebuffer[9i][option=TEX]
%
% \startplacefigure[reference=contour:9]
%     \startcombination[3*3]
%         {\pushrandomseed\setrandomseed{1}\getbuffer[9a]\poprandomseed} {\bf bitmap edge}
%         {\pushrandomseed\setrandomseed{1}\getbuffer[9b]\poprandomseed} {\bf bitmap cell}
%         {\pushrandomseed\setrandomseed{1}\getbuffer[9c]\poprandomseed} {\bf bitmap none}
%         {\pushrandomseed\setrandomseed{1}\getbuffer[9d]\poprandomseed} {\bf shape  shape}
%         {\pushrandomseed\setrandomseed{1}\getbuffer[9e]\poprandomseed} {\bf shape  edge}
%         {\pushrandomseed\setrandomseed{1}\getbuffer[9f]\poprandomseed} {\bf shape  none}
%         {\pushrandomseed\setrandomseed{1}\getbuffer[9g]\poprandomseed} {\bf band   edge}
%         {\pushrandomseed\setrandomseed{1}\getbuffer[9h]\poprandomseed} {\bf band   cell}
%         {\pushrandomseed\setrandomseed{1}\getbuffer[9i]\poprandomseed} {\bf band   none}
%     \stopcombination
% \stopplacefigure

In \in {figure} [contour:10] we see different combinations of backgrounds (in color)
and foregrounds (edges) in action.

\startbuffer[10a]
\startMPcode{doublefun}
    draw lmt_contour [
        xmin =  0, xmax = 4*pi, xstep = 0,
        ymin = -6, ymax = 6,    ystep = 0,

        levels = 5, legend = false, linewidth  = 1/2,

        preamble   = "local sin, cos = math.sin, math.cos",
        function   = "cos(x) - sin(y)",
        color      = "shade({1/2,0,0},{0,0,1/2})",

        background = "bitmap", foreground = "edge",
    ] xsized .3TextWidth ;
\stopMPcode
\stopbuffer

\startbuffer[10b]
\startMPcode{doublefun}
    draw lmt_contour [
        xmin =  0, xmax = 4*pi, xstep = 0,
        ymin = -6, ymax = 6,    ystep = 0,

        levels = 5, legend = false, linewidth  = 1/2,

        preamble   = "local sin, cos = math.sin, math.cos",
        function   = "cos(x) - sin(y)",
        color      = "shade({1/2,0,0},{0,0,1/2})",

        background = "bitmap", foreground = "cell",
    ] xsized .3TextWidth ;
\stopMPcode
\stopbuffer

\startbuffer[10c]
\startMPcode{doublefun}
    draw lmt_contour [
        xmin =  0, xmax = 4*pi, xstep = 0,
        ymin = -6, ymax = 6,    ystep = 0,

        levels = 5, legend = false, linewidth  = 1/2,

        preamble   = "local sin, cos = math.sin, math.cos",
        function   = "cos(x) - sin(y)",
        color      = "shade({1/2,0,0},{0,0,1/2})",

        background = "bitmap", foreground = "none",
    ] xsized .3TextWidth ;
\stopMPcode
\stopbuffer

\startbuffer[10d]
\startMPcode{doublefun}
    draw lmt_contour [
        xmin =  0, xmax = 4*pi, xstep = 0,
        ymin = -6, ymax = 6,    ystep = 0,

        levels = 5, legend = false, linewidth  = 1/2,

        preamble   = "local sin, cos = math.sin, math.cos",
        function   = "cos(x) - sin(y)",
        color      = "shade({1/2,0,0},{0,0,1/2})",

        background = "shape",  foreground = "shape", cache = true,
    ] xsized .3TextWidth ;
\stopMPcode
\stopbuffer

\startbuffer[10e]
\startMPcode{doublefun}
    draw lmt_contour [
        xmin =  0, xmax = 4*pi, xstep = 0,
        ymin = -6, ymax = 6,    ystep = 0,

        levels = 5, legend = false, linewidth  = 1/2,

        preamble   = "local sin, cos = math.sin, math.cos",
        function   = "cos(x) - sin(y)",
        color      = "shade({1/2,0,0},{0,0,1/2})",

        background = "shape",  foreground = "edge", cache = true,
    ] xsized .3TextWidth ;
\stopMPcode
\stopbuffer

\startbuffer[10f]
\startMPcode{doublefun}
    draw lmt_contour [
        xmin =  0, xmax = 4*pi, xstep = 0,
        ymin = -6, ymax = 6,    ystep = 0,

        levels = 5, legend = false, linewidth  = 1/2,

        preamble   = "local sin, cos = math.sin, math.cos",
        function   = "cos(x) - sin(y)",
        color      = "shade({1/2,0,0},{0,0,1/2})",

        background = "shape",  foreground = "none", cache = true,
    ] xsized .3TextWidth ;
\stopMPcode
\stopbuffer

\startbuffer[10g]
\startMPcode{doublefun}
    draw lmt_contour [
        xmin =  0, xmax = 4*pi, xstep = 0,
        ymin = -6, ymax = 6,    ystep = 0,

        levels = 5, legend = false, linewidth  = 1/2,

        preamble   = "local sin, cos = math.sin, math.cos",
        function   = "cos(x) - sin(y)",
        color      = "shade({1/2,0,0},{0,0,1/2})",

        background = "band",   foreground = "edge",
    ] xsized .3TextWidth ;
\stopMPcode
\stopbuffer

\startbuffer[10h]
\startMPcode{doublefun}
    draw lmt_contour [
        xmin =  0, xmax = 4*pi, xstep = 0,
        ymin = -6, ymax = 6,    ystep = 0,

        levels = 5, legend = false, linewidth  = 1/2,

        preamble   = "local sin, cos = math.sin, math.cos",
        function   = "cos(x) - sin(y)",
        color      = "shade({1/2,0,0},{0,0,1/2})",

        background = "band",   foreground = "cell",
    ] xsized .3TextWidth ;
\stopMPcode
\stopbuffer

\startbuffer[10i]
\startMPcode{doublefun}
    draw lmt_contour [
        xmin =  0, xmax = 4*pi, xstep = 0,
        ymin = -6, ymax = 6,    ystep = 0,

        levels = 5, legend = false, linewidth  = 1/2,

        preamble   = "local sin, cos = math.sin, math.cos",
        function   = "cos(x) - sin(y)",
        color      = "shade({1/2,0,0},{0,0,1/2})",

        background = "band",   foreground = "none",
    ] xsized .3TextWidth ;
\stopMPcode
\stopbuffer

\typebuffer[10b][option=TEX]

% \page

There are quite some settings. Some deal with the background, some with the
foreground and quite some deal with the legend.

\starttabulate[|T|T|T|p|]
\FL
\BC name             \BC type    \BC default    \BC comment \NC \NR
\ML
\NC xmin             \NC numeric \NC 0          \NC needs to be set \NC \NR
\NC xmax             \NC numeric \NC 0          \NC needs to be set \NC \NR
\NC ymin             \NC numeric \NC 0          \NC needs to be set \NC \NR
\NC ymax             \NC numeric \NC 0          \NC needs to be set \NC \NR
\NC xstep            \NC numeric \NC 0          \NC auto 1/200 when zero \NC \NR
\NC ystep            \NC numeric \NC 0          \NC auto 1/200 when zero \NC \NR
\NC checkresult      \NC boolean \NC false      \NC checks for overflow and NaN \NC \NR
\NC defaultnan       \NC numeric \NC 0          \NC the value to be used when NaN \NC \NR
\NC defaultinf       \NC numeric \NC 0          \NC the value to be used when overflow \NC \NR
\ML
\NC levels           \NC numeric \NC 10         \NC number of different levels to show \NC \NR
\NC level            \NC numeric \NC            \NC only show this level (foreground) \NC \NR
\ML
\NC preamble         \NC string  \NC            \NC shortcuts \NC \NR
\NC function         \NC string  \NC x + y      \NC the result z value \NC \NR
\NC functions        \NC list    \NC            \NC multiple functions (overlapping levels) \NC \NR
\NC color            \NC string  \NC lin(l)     \NC the result color value for level l (1 or 3 values) \NC \NR
\NC colors           \NC numeric \NC            \NC used when set \NC \NR
\ML
\NC background       \NC string  \NC bitmap     \NC band, bitmap, shape \NC \NR
\NC foreground       \NC string  \NC auto       \NC cell, edge, shape auto \NC \NR
\ML
\NC linewidth        \NC numeric \NC .25        \NC \NC \NR
%NC backgroundcolor  \NC string  \NC black      \NC \NC \NR
\NC linecolor        \NC string  \NC gray       \NC \NC \NR
\ML
\NC width            \NC numeric \NC 0          \NC automatic when zero \NC \NR
\NC height           \NC numeric \NC 0          \NC automatic when zero \NC \NR
\ML
\NC trace            \NC boolean \NC false      \NC \NC \NR
\ML
\NC legend           \NC string  \NC all        \NC x y z function range all \NC \NR
\NC legendheight     \NC numeric \NC LineHeight \NC \NC \NR
\NC legendwidth      \NC numeric \NC LineHeight \NC \NC \NR
\NC legendgap        \NC numeric \NC 0          \NC \NC \NR
\NC legenddistance   \NC numeric \NC EmWidth    \NC \NC \NR
\NC textdistance     \NC numeric \NC 2EmWidth/3 \NC \NC \NR
\NC functiondistance \NC numeric \NC ExHeight   \NC \NC \NR
\NC functionstyle    \NC string  \NC            \NC  \CONTEXT\ style name \NC \NR
\NC xformat          \NC string  \NC @0.2N      \NC number format template \NC \NR
\NC yformat          \NC string  \NC @0.2N      \NC number format template \NC \NR
\NC zformat          \NC string  \NC @0.2N      \NC number format template \NC \NR
\NC xstyle           \NC string  \NC            \NC \CONTEXT\ style name \NC \NR
\NC ystyle           \NC string  \NC            \NC \CONTEXT\ style name \NC \NR
\NC zstyle           \NC string  \NC            \NC \CONTEXT\ style name \NC \NR
\ML
\NC axisdistance     \NC numeric \NC ExHeight   \NC \NC \NR
\NC axislinewidth    \NC numeric \NC .25        \NC \NC \NR
\NC axisoffset       \NC numeric \NC ExHeight/4 \NC \NC \NR
\NC axiscolor        \NC string  \NC black      \NC \NC \NR
\NC ticklength       \NC numeric \NC ExHeight   \NC \NC \NR
\ML
\NC xtick            \NC numeric \NC 5          \NC \NC \NR
\NC ytick            \NC numeric \NC 5          \NC \NC \NR
\NC xlabel           \NC numeric \NC 5          \NC \NC \NR
\NC ylabel           \NC numeric \NC 5          \NC \NC \NR
\LL
\stoptabulate

\startplacefigure[reference=contour:10]
    \startcombination[3*3]
        {\getbuffer[10a]} {\bf bitmap edge}
        {\getbuffer[10b]} {\bf bitmap cell}
        {\getbuffer[10c]} {\bf bitmap none}
        {\getbuffer[10d]} {\bf shape  shape}
        {\getbuffer[10e]} {\bf shape  edge}
        {\getbuffer[10f]} {\bf shape  none}
        {\getbuffer[10g]} {\bf band   edge}
        {\getbuffer[10h]} {\bf band   cell}
        {\getbuffer[10i]} {\bf band   none}
    \stopcombination
\stopplacefigure

\stopchapter

\stopcomponent
