% language=uk

% todo:

% \starttext
%     \setupbodyfont[cambria]
%     \setupmathematics[kernpairs=yes]
%     $abcd$ % 𝑎𝑏
% \stoptext
%
% kernpairs = {
%     [0x1D44E] = {
%         [0x1D44F] = 1000, -- 𝑎𝑏 demo
%     }
% },

\startcomponent fonts-math

\environment fonts-environment

\startchapter[title=Math][color=darkmix-3]

\startsection[title=Introduction]

As one can expect, math support in \CONTEXT\ is to some extend modelled after
what plain \TEX\ provides, plus what was later decided to be standard. This
mostly concerns the way fonts behave and what names are used to access glyphs or
special constructs. It means that when you come from another macro package you
can stick to coding math the way you did before, at least the basic coding. In
addition to this, \CONTEXT\ gives control over fonts, structure and rendering and
most of that was either driven by personal need or user demand. To be honest,
many of the options are probably of not much interest to the average user.

As we focus on fonts we will only touch this aspect of math here. Right from when
we started with developing \LUATEX, cleaning op the math part of \CONTEXT\ was
part of the game. Some primitives were added that would make it possible to avoid
unnecessary complex macros to get certain glyphs rendered, like radicals, accents
and extensibles. This was made easy because we also support \OPENTYPE\ math and
because we knew that eventually the Latin Modern and Gyre fonts would also
support \OPENTYPE. In order to move forward and get rid of traditional eight bit
fonts \CONTEXT\ \MKIV\ can construct a virtual \OPENTYPE\ font from traditional
math fonts. It makes not much sense to discuss that here as by now this method is
only provided for reasons of compatibility and a reference to the past. As a lot
of time went into this it will always stay around if only to remind us of what
we went through to get where we are now.

\stopsection

\startsection[title=\UNICODE\ math]

Due to the limited amount of glyphs in a \TYPEONE\ font a macro package has to
jump through loops in order to get traditional \TEX\ engines behave well. As a
practical consequence these fonts are often a mixture of characters, symbols,
punctuation and snippets that make up larger shapes. The font dimensions in these
files have often special meanings too.

This has all changed considerably with math being part of \UNICODE. It was however
\MICROSOFT\ where the real action took place: the development of the first font that
combined \UNICODE\ with \OPENTYPE\ technology. The Cambria font can be considered
the benchmark for fonts that surfaced later. The characteristic of a math font are
the following:

\startitemize[packed]
    \startitem All math alphabets are present: latin as well as greek, in regular,
    italic, bold, fraktur and script variants as well as some combinations of these. \stopitem
    \startitem The symbols that make sense are present (read: the more obscure shapes
    can be omited). \stopitem
    \startitem For the characters that make sense, there are two variants that render well
    at smaller sizes: script and scriptscript. In the font they have the same size but
    the application will scale them down. This feature is named \type {ssty}. \stopitem
    \startitem Characters that can extend horizontally (for instance accents and arrows) or vertically
    (like radicals and fences) have associated larger variants and carries information about
    how to grow indefinitely. \stopitem
    \startitem There is a whole lot of special math dimensions. Most of the ones
    already used in \TEX\ are present. \stopitem
    \startitem Some glyphs come in variants in order to please special usage. There
    can also be variants for script or fraktur alphabets. \stopitem
\stopitemize

This means that in practice an \OPENTYPE\ math font is quite large. We easily
have thousands of glyphs. It also means that creating such a font involves some
expertise and this is one of the reasons why \TEX\ usergroups have joined forces
in developing a suite of fonts. There are also other initiatives in the \TEX\
community, of which Xits is an example. \footnote {This is a useable
variant of Stix fonts with proper math features, some extra glyphs and
experimental right||to||left shapes.} The well known Lucida Bright math font
package has also been upgraded to a set of \OPENTYPE\ math fonts.

The fact that there are not that many math fonts out there has a positive side as
well: \CONTEXT\ comes with them pre|-|configured. Because during the development
of \LUATEX\ we needed to have at least a couple of fonts for testing, and because
it makes no sense to waste time on traditional fonts, the Latin Modern, Palatino,
Times and a few more fonts were (and still are) provided as virtual \UNICODE\
fonts.

In a regular text font, what you key in is what you get out. So, when you've
chosen a font with an italic shape, you get italic shapes, even if the smallcaps
feature is enabled. In math, if you use the right unicodes you also get the shape
you expect. Because in this case italic shapes are situated in one of the
alphabets you explicitly choose a rendering. You can enter the right codepoints
directly, so for instance if you enter \UNICODE\ character \type {U+1D434} you
will get \mathematics {\utfchar {"1D434}}. In practice something like \type {$\bi
A$} should also give that character if only because that is what we have been
doing for over three decades. This means that the engine has to map a regular
\type {A} onto the bold italic alphabet. In a traditional approach you will use
math families for this, but in \CONTEXT\ \MKIV\ we simply use one font and one
family and let the \MKIV\ machinery do the rest.

In text mode we switch fonts styles in the following way:

\startbuffer
regular {\it italic} {\bf bold} {\bi bold italic} and so on
\stopbuffer

\typebuffer

The three commands shown here are shortcuts for font switches. This input is
converted into an internal representation and after whatever manipulations
are applied end up as:

\blank \getbuffer  \blank

If we look at what fonts we end up with we get:

\blank \markfonts{\getbuffer} \blank

Now lets do the same in math mode:

\startbuffer
$regular {\it italic} {\bf bold} {\bi bold italic} and so on$
\stopbuffer

\typebuffer

This time we get a different result:

\blank \getbuffer \blank

If again we analyze the fonts you see this:

\blank \markfonts{\getbuffer} \blank

All glyphs come from the same font. Instead of regular we get italic simply because
math characters are italic by nature. The two character style switches are not
really font switches but just make sure that the given input is mapped onto the
right alphabet.

A traditional approach using \TYPEONE\ fonts is to use a so called math family for
each alphabet. In that case each alphabet maps one||to||one onto the font: when
we switch to a bold family we just take the glyph that sits in that slot. In \MKIV\
we have all characters in one family so behind the screens a given character is
remapped.

Now take a look at the following example:

\startbuffer
$text^{script^{scriptscript}}$
\stopbuffer

\typebuffer

This renders to this, with the characters marked by font:

\blank \markfonts{\getbuffer} \blank

This time we have three different fonts: one for each of the three math sizes.
But this representation is not entirely honest, because indeed we have three font
instances for math, but the glyphs come from the same \OPENTYPE\ math font. We
just load the same font three times, once for each size. In fact we load the
font once, but use three copies, scaled accordingly to the relative scale the
font prescribes.

There is a whole bunch of commands to choose specific characters in math mode
using a regular input. These are state switching commands.

\def\SampleLine#1#2#3%
  {\NC \type{#1}\space
       \ifx#2\empty\else\type{#2}\fi\space
       \ifx#3\empty\else\type{#3}\fi
   \NC $#1 a$
   \NC $#1 A$
   \NC \NR}

\starttabulate[|||||]
\HL
\SampleLine \mr            \empty  \empty
\HL
\SampleLine \mathdefault   \empty  \empty
\SampleLine \mathscript    \empty  \empty
\SampleLine \mathfraktur   \empty  \empty
\SampleLine \mathblackboard\empty  \empty
\HL
\SampleLine \rm            \mathrm \empty
\SampleLine \ss            \mathss \empty
\SampleLine \tt            \mathtt \empty
\HL
\SampleLine \tf            \mathtf \tfmath
\SampleLine \sl            \mathsl \slmath
\SampleLine \it            \mathit \itmath
\HL
\SampleLine \bf            \mathbf \bfmath
\SampleLine \bs            \mathbs \bsmath
\SampleLine \bi            \mathbi \bimath
\HL
\stoptabulate

As you can see here, some commands have synonyms. The short commands adapt
themselves to text and mathmode, the longer ones are meant for use in math mode
only.

In text mode distinctive shapes are either a font property (the whole font looks
that way) or a stylistic alternate (an extra feature of a font). In math mode we
can have alternates, but in addition to the previously mentioned alphabet
switchers we have a few more:

\starttabulate[|||||]
\HL
\SampleLine \frak       \empty \empty
\SampleLine \cal        \empty \empty
\SampleLine \bbd        \empty \empty
\SampleLine \blackboard \empty \empty
\SampleLine \fraktur    \empty \empty
\SampleLine \gothic     \empty \empty
\HL
\stoptabulate

This chapter is not meant as an introduction to math but it is good to know
that math font support in \CONTEXT\ is rather flexible. There are several
mechanisms for remapping and converting characters and sequences into
others and more is possible. Here is one:

\startbuffer
\startformula
\reals {\mathbf R} \utfchar{"0211D} \utfchar{"1D411}
\stopformula
\stopbuffer

\typebuffer \blank \getbuffer \blank

Compare this to:

\startbuffer
\setupmathematics[symbolset=blackboard-to-bold]
\startformula
\reals {\mathbf R} \utfchar{"0211D} \utfchar{"1D411}
\stopformula
\stopbuffer

\typebuffer \blank \start \getbuffer \stop \blank

Greek is always troublesome because instead of regular text shapes math uses a
few variants. Because in \UNICODE\ characters are only included once, we have
gaps in the math alphabets but \MKIV\ will take care of this. \footnote {This is
a typical example of where exceptions in a standard force all applications that
deal with it have to implement tweaks.} Depending on the field an author has to
choose between upright and italic greek:

\startbuffer
$\nabla \alpha \mathgreekupright \nabla \alpha \mathgreekitalic \nabla \alpha$
\stopbuffer

\typebuffer \blank \start \getbuffer \stop \blank

By default \CONTEXT\ is set up as follows:

\starttyping
\setupmathematics
  [sygreek=normal,
   lcgreek=italic,
   ucgreek=normal]
\stoptyping

Again, these are not features of a font. The font just provides the glyphs and
the \TEX\ engine, controlled by \CONTEXT\ takes care of mapping characters to
glyphs and building special constructs. The same is true for spacing. Although
math fonts do have kerning information, most of the math spacing is controlled
by properties of characters and not by the font.

\unexpanded\def\SampleLine#1%
  {\NC
   \type{$a #1{+} b$}
   \NC
   \ruledhbox{$\mathsurround\zeropoint a#1{+}b$}
   \NC \NR}

\starttabulate[|||]
\SampleLine \mathord
\SampleLine \mathpunct
\SampleLine \mathinner
\SampleLine \mathop
\SampleLine \mathalpha
\SampleLine \mathnothing
\SampleLine \mathbin
\SampleLine \mathrel
\stoptabulate

As a user you don't have to worry about these issues because characters are tagged
according to their usage. \footnote {There are a few more commands, like \type
{\mathlimop}, \type {\mathnolop} and \type {mathbox} but these are used
differently.}

With \TEX\ being the oldest and still dominant math renderer it is no surprise
that \MICROSOFT\ modelled its math renderer after \TEX\ and Cambria quite well
suits the concept. In retrospect it is somewhat unfortunate that we're still
stuck with some left overs (or compromises) from the past with respect to spacing
built into the font. However, as long as this is consistent over fonts it's not
that relevant. You can always influence the spacing with the commands mentioned.

If you look at the low level definitions in for instance plain \TEX\ but also in
\CONTEXT\ \MKII\ that relate to prime symbols it probably takes a while before you
figure out what happens there. For instance, the prime symbol is triggered by a
quote and multiple in a row results in primes that are spaced tightly. In
\UNICODE\ we have slots for single, double and tripple primes. Therefore, in
\MKIV\ we have a mechanism that accepts different kinds of input that eventually
all end up in one of these three glyphs.

\unexpanded\def\SampleLine#1%
  {\NC \type{#1} \NC #1 \NC \NR}

\starttabulate[|||]
\SampleLine{$f^2$}
\SampleLine{$f\prime^2$}
\SampleLine{$f\prime\prime^2$}
\SampleLine{$f\prime\prime\prime^2$}
\SampleLine{$f{\prime}^2$}
\SampleLine{$f{\prime\prime}^2$}
\SampleLine{$f{\prime\prime\prime}^2$}
\SampleLine{$f'(x)$}
\SampleLine{$f''(x)$}
\SampleLine{$f'''(x)$}
\SampleLine{$f\utfchar{0x2032}(x)$}
\SampleLine{$f\utfchar{0x2033}(x)$}
\SampleLine{$f\utfchar{0x2034}(x)$}
\SampleLine{$f\utfchar{0x2032}\utfchar{0x2032}(x)$}
\SampleLine{$f\utfchar{0x2032}\utfchar{0x2032}\utfchar{0x2032}(x)$}
\SampleLine{$f\utfchar{0x2033}\utfchar{0x2032}(x)$}
\SampleLine{$f\utfchar{0x2032}\utfchar{0x2033}(x)$}
\stoptabulate

Again, this is not the same as ligature building features in text fonts, but
handled in a different way.

The \TEX\ engine understands the concept of italic correction. When an italic
shape is followed by for instance an upright shape, you can insert a \type {\/}
and the engine will add a correction as defined in the font. In \OPENTYPE\ we
don't have such corrections available but we can fake it, which is what the \type
{itlc} feature in \CONTEXT\ does. However, you need to enable this feature
explicitly. An example of a setup is:

\starttyping
\definefontfeature
  [default]
  [default]
  [itlc=yes,textitalics=yes]

\setupitaliccorrection
  [global,always]
\stoptyping

This will make sure that the right amount of correction is added between
italic shapes and non italics or boxes. Using \type {text} instead of
\type {always} would limit the correction to glyphs only and leaving out
the \type {global} would permit selective (grouped) usage at the cost
of more runtime. There is no need for the \type {\/} here.

In math we also can have italic correction but there it is built into the engine
and in traditional \TEX\ no directives are needed. Italic correction properties
in math fonts are somewhat troublesome as their application depends on what we're
dealing with: symbols, super- and subscripts, etc. Because early versions of
\LUATEX\ didn't handle all of it well, if only because the fonts were not yet okay,
the \MKIV\ math handler provides a bit of control.

\def\SampleLine#1#2%
  {\NC #1
   \NC \setupmathematics[italics=#1]\ruledhbox{$m$ t}
   \NC \setupmathematics[italics=#1]\ruledhbox{$m$ {\it t}}
   \NC \setupmathematics[italics=#1]\ruledhbox{t $m$ t $m$ {\it t}}
   \NC #2
   \NC \NR}

\starttabulate[||||||]
\SampleLine0{no correction}
\SampleLine1{only apply italics when the font carries them}
\SampleLine2{apply italics provided by the font or automatically calculated}
\SampleLine3{apply italics based on an emwidth and character properties}
\SampleLine4{use method 1 but fall back on 3 if needed}
\stoptabulate

Because we cannot rely on fonts too much, we default to method~3 which in practice
works out well, so the setup is:

\starttyping
\setupmathematics
  [italics=3]
\stoptyping

There are all kind of commands that can be used to build math constructs in such a
way that super- and subscripts are consistently rendered. It goes beyond this
chapter to discuss them and most users will never see or use those commands. The
main message of the examples above is that text and math use different fonts and
properties and therefore also different methods in rendering text or a formula.
Even if the names of mechanisms are the same (like italics) you cannot assume
that both modes do exactly the same.

\stopsection

\startsection[title=Bold math]

If you look at what \UNICODE\ provides you will notice that there are quite some
bold characters. First of all there are a bunch of alphabets and because bold is
not present in the text part of \UNICODE\ these alphabets have no holes. Then
there are some symbols that have special meaning.

\startluacode
local find = string.find
local NC, NR = context.NC, context.NR

context.starttabulate { "|Tl||l|" }
for unicode, entry in table.sortedhash(characters.data) do
    local description = entry.description
    if find(description,"^MATHEMATICAL BOLD") then
        NC() context("U+04X",unicode)
        NC() context.mathematics(utf.char(unicode))
        NC() context(description)
        NC() NR()
    end
end
context.stoptabulate()
\stopluacode

The biggest mistake one can make when discussing bold math is the assumption that
these bold alphabets are meant for section titles and other structural elements
that need some emphasis. This is not true, in that case we would expect the whole
formula to be bold and the bold symbols or variables would be even more bold.
Bold math boils down to {\em all} math being bold. The reason why we show the
list of bold characters on the previous pages is that it gives a good impression of
fact that we're mostly given alphabets in an otherwise regular font.

As Latin Modern (being derived from Computer Modern) has some bold extras in
\MKII\ to some extend we do support a complete bold math switch but mixing bold
formulas with regular ones has some limitations. Math typesetting consists of two
phases: first the input is translated into a special list where references to
fonts are not yet resolved. Instead families are used and each family has three
sizes: text, script and scriptscript. In a second pass the formula is typeset and
the families get translated into fonts. So, if we change the definition of a
family, say math italic into bold math italic, then the definition that is actual
when the second pass takes place is used.

Although \LUATEX\ provides for many more families and as a consequence we could
have replaced the \MKII\ mechanism with a more complete one, instead we just
forgot about it and stuck to one family for regular math and another one for bold
math. Okay, this is not entirely true as later on we added some more in order to
deal with bidirectional typesetting.

Only a few math fonts come with a bold variant. One of the objectives of the \TEX
Gyre math font project is to explore the possibilities of bold math companions,
but such a font will probably have less coverage, simply because no real complex
math will end up in for instance section titles.

When I wrote this down there were not that many math fonts that come with a real
(complete) bold variant. The \CONTEXT\ math font subsystem tries to fill this gap
as good as possible by using pseudo fonts. When a typeface doesn't define a math
bold variant a pseudo setup is used. When a real bold font is used, it could be
that not all alphabets are supported in which case a suitable alternative is
tried.

The Xits font, assembed from Stix and enhanced by Khaled Hosny, comes with a bold
variant but the coverage is not complete, at least not when I wrote this
paragraph. This can go unnoticed because \CONTEXT\ tries to work around this. On
the other hand, it definitely has bold properties, which can be seen from the
next example. You switch between regular and bold math with the \type {\mr} and
\type {\mb} commands.

\startbuffer
\switchtobodyfont[xitsbidi]

$             \sqrt{x } \quad
 \mb          \sqrt{mb} \quad
 \mathupright \sqrt{u } \quad
 \mr          \sqrt{mr} \quad
 \mathupright \sqrt{u } \quad
 \mathdefault \sqrt{d }
$
\stopbuffer

\typebuffer \blank \start \getbuffer \stop \blank

You can track some of what happens with:

\starttyping
\enabletrackers[math.remapping,math.families]
\stoptyping

You will get some information about remapping or when it fails if fallback
remapping is used. But no matter what happens with glyphs, you will notice in
this example that the radical symbol is bold indeed.

\stopsection

\startsection[title=Bidirectional math]

There is not that much to tell about bidirectional math typesetting, simply
because the fonts are still in development. However, Khaled Hosny added
some support to the Xits font. Of course you need to load this font first:

\starttyping
\switchtobodyfont[xitsbidi]
\stoptyping

In the previous chapter we mentioned bold math and as Xits also comes with
a bold variant which means that this command loads the whole lot (which is
fast enough anyway).

Easiest is to just show a few examples. When in left to right mode we get what we
are accustomed to:

\startbuffer
\setupmathematics[align=l2r]

\startformula
\sqrt{x^2\over 4x} \eqno(1)
\stopformula

\startformula
5 < 6 > 4
\stopformula

\startformula
5 \leq 6 \geq 7
\stopformula
\stopbuffer

\typebuffer \start \switchtobodyfont[xitsbidi] \getbuffer \stop

However, when we go the other way, we automatically get digits converted to
arabic.

\startbuffer
\setupmathematics[align=r2l,bidi=yes]

\startformula
\sqrt{ف^2\over 4ب} \eqno(1)
\stopformula

\startformula
5 < 6 > 4
\stopformula

\startformula
5 \leq 6 \geq 7
\stopformula
\stopbuffer

\typebuffer \start \switchtobodyfont[xitsbidi] \getbuffer \stop

You don't have to worry about how the font is set up, but not that much is needed
because \CONTEXT\ does it for you and the Xits typescripts carries the right
definitions. Just to give you an idea, we show a feature definition: The magic is
in the \type {rtlm} feature combined with \type {locl}.

\starttyping
\definefontfeature
  [mathematics-r2l]
  [mathematics]
  [language=ara,
   rtlm=yes,
   locl=yes]
\stoptyping

Some symbols are mirrored too:

\startbuffer
\setupmathematics[align=r2l,bidi=yes]

\startformula
\sum^\infty_{س=0} س^2 \eqno(2)
\stopformula
\stopbuffer

\typebuffer \start \switchtobodyfont[xitsbidi] \getbuffer \stop

And of course the extensible fences are done properly too:

\startbuffer
\setupmathematics[align=r2l,bidi=yes]

\startformula
\left(\root{2} \of{155}\right)
\stopformula

\startformula
\left[\int^{55}_{123} 666^3\right]
\qquad\textstyle\left[\int^{55}_{123} 666^3\right]
\stopformula

\startformula
\left\{\sum^{55}_{123} 666^3\right\}
\stopformula
\stopbuffer

\typebuffer \start \switchtobodyfont[xitsbidi] \getbuffer \stop

The real torture test is the radical sign. A mirrored shape is used
and it grows upwards as well as leftwards.

\startbuffer
\setupmathematics[align=r2l,bidi=yes]

\startformula
\sqrt{\sqrt{\sqrt{\sqrt{\sqrt{\sqrt{\sqrt{\sqrt{\sqrt{\sqrt{55}}}}}}}}}}
\stopformula
\stopbuffer

\typebuffer \start \switchtobodyfont[xitsbidi] \getbuffer \stop

\stopsection

\startsection[title=Styles]

In text mode you use font switches like \type {\sl} that switches the current font
to a slanted one. In math mode it is an alphabet switch in the same font. In
fact, there isn't much to choose from fonts there, apart from a massive switch
to bold, in which case \type {\bf} is just a bolder alphabet in that bolder font.

A lot of things in math mode happen automatically. There are for instance always
three instances of (the same) font active, each different in size: text, script
and the smallest, scriptscript and when you ask for instance for a superscript
the next smaller size is used.

\starttabulate[|l|l|l|]
\NC normal   \NC \type {\textstyle}         \NC $\textstyle         text^{script^{scriptscript}}$ \NC \NR
\NC smaller  \NC \type {\scriptstyle}       \NC $\scriptstyle       text^{script^{scriptscript}}$ \NC \NR
\NC smallest \NC \type {\scriptscriptstyle} \NC $\scriptscriptstyle text^{script^{scriptscript}}$ \NC \NR
\stoptabulate

In text style, superscripts can go twice smaller, but in script style only one
smaller size is left, and in scriptscript style you're stuck with one size. The
commands in the second column can be used to force a style.

The math formula builder has an important property: the formula is typeset after
it has been scanned completely. In a traditional setup that has some
consequences. Take this:

\starttyping
one \sl two \bf three \bi four
\stoptyping

In a traditional setup four so called families are used and each character gets
tagged with a family number. So we have (for instance):

\blank \start \tttf
o\low7n\low7e\low7 t\low6w\low6o\low6 t\low5h\low5r\low5e\low5e\low5 f\low9o\low9u\low9r\low9
\stop \blank

As the number of families was limited there could be at most 16 families. In
fact, the first four were traditionally reserved for math roman, math italic,
symbol and extensibles. Then, due to the limit of 256 characters per font,
another few were used for additional symbol fonts. So, adding a few more variants
could exhaust the family pool quite fast. You could argue that we could halfway
redefine a family but this will not work as there is a one to one relationship
between family numbers and fonts assigned to them when the formula has been
read in (the last value counts). And grouping won't help you either.

The actual (plain) situation is even more complex. As we have a limited number of
characters per font, most symbols are accessed by name, and the name relates to a
mathematical character definition using for instance \type {\mathchardef}. Such a
definition refers to a slot in a specific family number and therefore font. It
also puts a character in a so called math class. One of these, the alphanumeric
class, with number~7, is special. Characters that are input directly on the
keyboard (like \type {a}||\type {z} can also be tagged this way using \type
{\mathcode}.

When we switch a family, this will normally not affect a symbol defined as math
character, simply because we refer to a specific family|/|slot combination, but
when a character has class~7, then it will be taken from the current family. This
permits latin letters, digits and greek letters to be typeset in different
styles. So, in that traditional approach we have fonts that provide a bunch of
symbols as well as some alphabets. Think for instance of a font with additional
symbols where the regular alphabet slots contain blackboard shapes. The symbols
are accessed directly and the characters are accessed via the regular \type
{a}||\type {z} characters as these will adapt to the family and therefore font.
In practice users will not notice this complication as macro packages hide the
implementation details.

In \MKIV\ the situation is different as there we have one family (or a few more
if we use a full bold switch and|/|or bidirectional math). Although we no longer
have the limit of 16 fonts we actually don't need that many families, at least
not in the way we've set up \MKIV \footnote{A technical note: in principle the
\MKIV\ approach can have a speed penalty compared to a multi||family approach but
we don't care too much about it. Also, as we load less fonts the extra overhead gets
compensated nicely.}

\blank \start \tttf
o\low1n\low1e\low1 t\low1w\low1o\low1 t\low1h\low1r\low1e\low1e\low1 f\low1o\low1u\low1r\low1
\stop \blank

So how does this relate to styles? Each family has three fonts and we can use the
switch commands to choose any of these. In text mode we use the term style for a
font switch, while in math mode it's more than that: indeed we switch a font, but
only in size, but the spacing is also adapted. If a proper math font is used, the
smaller sizes are actually alternates in the font, visually adapted to suit their
use.

In text mode we do this in order to limit the scope of a switch:

\starttyping
normal {\bf bold {\it italic} bold} normalbracket
\stoptyping

This is the same as:

\starttyping
normal \bgroup \bf bold \bgroup \it italic\egroup
    \ bold\egroup \ normalbracket
\stoptyping

and:

\starttyping
normal \begingroup \bf bold \begingroup \it italic\endgroup
    \ bold\endgroup \ normalbracket
\stoptyping

The \CONTEXT\ distribution ships with a plain math definition file that also uses
one family but reassigns some math codes when we switch to another style. As the
number of characters that this applies to this is efficient enough for a modern
computer. A peek into \type {luatex-math.tex} gives an impression of what we deal
with. However, keep in mind that the implementation in \MKIV\ goes it differently
and is therefore more powerful. We also have hardly any definitions at the \TEX\
end and use information from \type {char-def.lua} instead.

In math mode there is a subtle difference in the way grouping works with styles:

\starttyping
text {\scriptstyle script} normal
\stoptyping

This is the same as:

\starttyping
text \bgroup\scriptstyle script\egroup\ normal
\stoptyping

but different from:

\starttyping
text \begingroup\scriptstyle script\endgroup\ script
\stoptyping

This has to do with the fact that a style switch is explicitly registered in the
math list and grouping like this is not limiting the scope. In math mode the
braced grouping mode actually does create a math group and there the scope of the
switch is limited to that group. In practice users will not run into this but
they can use macros that use \type {\begingroup}. Among other reasons, this is
why we have a special mathstyle mechanism.

\startbuffer
\ruledhbox{$x\begingroup\scriptstyle x\endgroup x$} \quad
\ruledhbox{$x\begingroup\setupmathstyle[script]x\endgroup x$} \quad
\ruledhbox{$x{\setupmathstyle[script]x}x$} \quad
\ruledhbox{$x\startmathstyle[script]x\stopmathstyle x$}
\stopbuffer

\typebuffer

This gives:

\startlinecorrection[blank] \dontleavehmode \getbuffer \stoplinecorrection

Mechanisms that support the \type {mathstyle} parameter know how to apply the
proper grouping so you don't have to worry there. You can best avoid using the
verbose grouping command and stick to braces or the \type {start}||\type {stop}
command. An example is the fence mechanism:

\startbuffer
\definemathfence
  [fancybracket] [bracket]
  [color=darkblue]
\definemathfence
  [smallbracket] [bracket]
  [command=yes,color=darkgreen,mathstyle=small]
\definemathfence
  [normalbracket] [bracket]
  [command=yes,color=darkred]
\stopbuffer

\typebuffer \getbuffer

We apply this to an example:

\startbuffer
$x \fenced[bar]{\frac{1}{x}} x$ \quad
$x \fenced[doublebar]{\frac{1}{x}} x$ \quad
$x \fenced[bracket]{\frac{1}{x}} x$ \quad
$x \fenced[fancybracket]{\frac{1}{x}} x$ \quad
$x \frac{1}{n} \normalbracket{\frac{1}{n}} \smallbracket{\frac{1}{s}} x$
\stopbuffer

\typebuffer

Of course these somewhat weird examples are not real but at least they
demonstrate the principles.

\startlinecorrection[blank] \dontleavehmode \getbuffer \stoplinecorrection

A math style is a combination of the following keys. Their effect can depend on
the current state, for instance you can switch cramp or size indepently.

\starttabulate[|T||]
\NC display          \NC display style, like text style but somewhat more spacy \NC \NR
\NC text             \NC text style, normally used inline \NC \NR
\NC script           \NC smaller than text cq. display style \NC \NR
\NC scriptscript     \NC smaller than script style \NC \NR
\NC cramped packed   \NC more tightly positioned superscripts \NC \NR
\NC uncramped normal \NC normal positioned superscripts \NC \NR
\NC small            \NC switch to the next smaller style but keep cramp state \NC \NR
\NC big              \NC switch to the next larger style but keep cramp state \NC \NR
\stoptabulate

Future versions of \MKIV\ will provide more features (like parameter sets driven
by keywords). As you might prefer a more symbolic approach we provide:

\starttyping
\definemathstyle[default][text,cramped]
\stoptyping

After this you can use the keyword \type {default} which has the advantage that
you only need to change one definition in order to get different rendering.

\stopsection

\startsection[title=Supported fonts]

As in \CONTEXT\ MKIV\ I wanted to go ahead with \UNICODE\ math as soon as the
first version of \LUATEX\ showed up. Because at that time only Cambria was
available I decided to provide virtual \UNICODE\ math fonts as a prelude to
proper replacements for the popular \TYPEONE\ math fonts. In the meantime Xits
came around and in 2012 we had quite useable math companions for the public Latin
Modern, Pagella and Termes fonts and the \TEX\ user groups started shipping
\OPENTYPE\ variants of Lucida. The virtual variants will still around so that we
can compare them with the new implementations. As the official specification of
\OPENTYPE\ math is not always clear from the beginning the \OPENTYPE\ fonts get
improved over time. In fact, this is true not only for math fonts. Just think of
this:

\startitemize

\startitem As \UNICODE\ gets extended, fonts might get more glyphs and possibly
alternate shapes. \stopitem

\startitem The more languages are supported, the more glyphs are to be available
and features have to get language dependent instances. \stopitem

\startitem The larger the font, the bigger the chance that mistakes get unnoticed
especially when contextual subtitutions and positioning are used. \stopitem

\startitem Math fonts can get more script and scriptscript alternates, more size
variants, more advanced extensibles, bidirectional support, etc. \stopitem

\stopitemize

So, like regular programs, \LUATEX\ and macro packages, we now have fonts as
component that needs occasional updating. Of course resources like hyphenation
patterns are also subjected to this, so it's not a new aspect. But still, best
keep en eye on font updates.

While there are lots of text fonts, there are not that many math fonts, so you
can safely assume that \CONTEXT\ ships with the proper setup for those fonts. Of
course you have to choose a specific instance when you set up your own
combination of fonts, but a peek into the typescripts shows the way.

In the font manual and on the wiki you can find more about typescript and what is
possible, so here we just take a look at one definition:

\startnarrowtyping
\starttypescript [serif] [dejavu] [name]
  \definefontsynonym [Serif]           [name:dejavuserif]           [features=default]
  \definefontsynonym [SerifBold]       [name:dejavuserifbold]       [features=default]
  \definefontsynonym [SerifItalic]     [name:dejavuserifitalic]     [features=default]
  \definefontsynonym [SerifBoldItalic] [name:dejavuserifbolditalic] [features=default]
\stoptypescript

\starttypescript [sans] [dejavu] [name]
  \definefontsynonym [Sans]           [name:dejavusans]            [features=default]
  \definefontsynonym [SansBold]       [name:dejavusansbold]        [features=default]
  \definefontsynonym [SansItalic]     [name:dejavusansoblique]     [features=default]
  \definefontsynonym [SansBoldItalic] [name:dejavusansboldoblique] [features=default]
\stoptypescript

\starttypescript [mono] [dejavu] [name]
  \definefontsynonym [Mono]           [name:dejavusansmono]            [features=none]
  \definefontsynonym [MonoBold]       [name:dejavusansmonobold]        [features=none]
  \definefontsynonym [MonoItalic]     [name:dejavusansmonooblique]     [features=none]
  \definefontsynonym [MonoBoldItalic] [name:dejavusansmonoboldoblique] [features=none]
\stoptypescript

\starttypescript[dejavu]
  \definetypeface [dejavu] [rm] [serif] [dejavu] [default]
  \definetypeface [dejavu] [ss] [sans]  [dejavu] [default]
  \definetypeface [dejavu] [tt] [mono]  [dejavu] [default]
  \definetypeface [dejavu] [mm] [math]  [xits]   [default] [scale=1.2]
\stoptypescript
\stopnarrowtyping

So, in many cases you can just copy this blob and replace the font names by your
own.

Loading a font, and Dejavu is a predefined one, is done as follows:

\starttyping
\setupbodyfont[dejavu]
\stoptyping

In a similar fashion you can enable \type {cambria}, \type {pagella}, \type
{termes}, \type {lucidaot}, etc.\ and if you don't use this command at all, you
get Latin Modern. These fonts are part of \TEX\ distributions, including
\CONTEXT\ stand||alone that can be downloaded from \CONTEXT\ garden.

If you want to use Lucida, all you have to do when you have bought the fonts, is
to put the \OPENTYPE\ files in a place where they can be found, for instance:

\starttyping
tex/texmf-fonts/fonts/data/lucida
\stoptyping

Of course you need to run \type {mtxrun --generate} afterwards so that the files
can be found.

\startnotabene
    Tracing and characters coverage will be discussed here as soon as the styles
    that are used for them are normalized.
\stopnotabene

\stopsection

\startsection[title={Stylistic alternates}]

Some fonts provide stylistic alternates. These can be described in goodies files
and the Lucida setup is a good example. Here we demonstrate the effects. We
disable the default math rendering (which takes the italic variants).

\startbuffer[sa:1]
\switchtobodyfont[lucidaot,14.4pt]
\setupmathrendering[lucidaot][it=]
$x
   ^{i \leftarrow 0 = ∅}
   _{i \leftarrow 0 = ∅}
$
\stopbuffer

\typebuffer[sa:1]

The next code enabled three alternatives:

\startbuffer[sa:2]
\switchtobodyfont[lucidaot,14.4pt]
\setupmathrendering[lucidaot][it=]
$x
   ^{i \leftarrow 0 = ∅}
   _{\setmathfontalternate{arrow}
     \setmathfontalternate{dotless}
     \setmathfontalternate{zero}
     i \leftarrow 0 = ∅}
$
\stopbuffer

\typebuffer[sa:2]

Here we set them in one go:

\startbuffer[sa:3]
\switchtobodyfont[lucidaot,14.4pt]
\setupmathrendering[lucidaot][it=]
$x
   ^{i \leftarrow 0 = ∅}
   _{\setmathfontalternate{arrow,dotless,zero}
     i \leftarrow 0 = ∅}
$
\stopbuffer
\
\typebuffer[sa:3]

The last example shows how to enable these features globally:

\startbuffer[sa:4]
\switchtobodyfont[lucidaot,14.4pt]
\setupmathrendering[lucidaot][it=]
\setupmathematics[stylealternative={arrow,dotless,zero}]
$x
   ^{i \leftarrow 0 = ∅}
   _{i \leftarrow 0 = ∅}
$
\stopbuffer

\typebuffer[sa:4]

The results are collected here:

\startlinecorrection[blank]
\startcombination[4*1]
    {\vbox{\hsize.2\hsize\midaligned{\nospacing\getbuffer[sa:1]}}} {\bf nothing}
    {\vbox{\hsize.2\hsize\midaligned{\nospacing\getbuffer[sa:2]}}} {\bf stepwise}
    {\vbox{\hsize.2\hsize\midaligned{\nospacing\getbuffer[sa:3]}}} {\bf combined}
    {\vbox{\hsize.2\hsize\midaligned{\nospacing\getbuffer[sa:4]}}} {\bf global}
\stopcombination
\stoplinecorrection

\stopsection

\startsection[title=Italics and limits]

An \OPENTYPE\ font treats italic correction differently from traditional fonts.
Officially the italic correction is used for placement above and below limits
where the scripts shift left and right half of the correction from the center of
the shape. Advanced kerns are then to be used for anchoring the scripts when they
are placed at the right side (so far no fonts seem to do this). Because we cannot
foresee if fonts compensate for correction then we can control placement a bit.
There is a parameter \type {\mathnolimitsmode} that controls the correction.

\definebodyfontenvironment[20pt]

\startlinecorrection
\startcombination[5*1]
    {\ruledhbox to .15\hsize{\hss\showglyphs\switchtobodyfont[20pt]\mathnolimitsmode0$\displaystyle\int\nolimits^0_1$\hss}} {\tttf 0}
    {\ruledhbox to .15\hsize{\hss\showglyphs\switchtobodyfont[20pt]\mathnolimitsmode1$\displaystyle\int\nolimits^0_1$\hss}} {\tttf 1}
    {\ruledhbox to .15\hsize{\hss\showglyphs\switchtobodyfont[20pt]\mathnolimitsmode2$\displaystyle\int\nolimits^0_1$\hss}} {\tttf 2}
    {\ruledhbox to .15\hsize{\hss\showglyphs\switchtobodyfont[20pt]\mathnolimitsmode3$\displaystyle\int\nolimits^0_1$\hss}} {\tttf 3}
    {\ruledhbox to .15\hsize{\hss\showglyphs\switchtobodyfont[20pt]\mathnolimitsmode4$\displaystyle\int\nolimits^0_1$\hss}} {\tttf 4}
\stopcombination
\stoplinecorrection

A value larger than 15 is interpreted as a factor (in the usual \TEX\ way 1000
means 1.0). We have some values left for future use when correction is to be
combined with kerns.

In \CONTEXT\ we set the value to 1 which means that the factors for super- and
subscript are set via math parameters (or constants in the font). We use a
default of \type {{0,800}} so we don't shift the superscript and the subscript we
shift less than the italic correction. This is driven by a feature but you can
change the values before loading a font, for instance with:

\starttyping
\adaptfontfeature[*math*][mathnolimitsmode={100,700}]
\stoptyping

The defaults come out as:

\startlinecorrection
\startcombination[5*1]
    {\ruledhbox to .15\hsize{\hss\showglyphs\switchtobodyfont[modern,  20pt]$\displaystyle\int\nolimits^0_1$\hss}} {\tttf modern}
    {\ruledhbox to .15\hsize{\hss\showglyphs\switchtobodyfont[xits,    20pt]$\displaystyle\int\nolimits^0_1$\hss}} {\tttf xits}
    {\ruledhbox to .15\hsize{\hss\showglyphs\switchtobodyfont[lucidaot,20pt]$\displaystyle\int\nolimits^0_1$\hss}} {\tttf lucidaot}
    {\ruledhbox to .15\hsize{\hss\showglyphs\switchtobodyfont[pagella, 20pt]$\displaystyle\int\nolimits^0_1$\hss}} {\tttf pagella}
    {\ruledhbox to .15\hsize{\hss\showglyphs\switchtobodyfont[cambria, 20pt]$\displaystyle\int\nolimits^0_1$\hss}} {\tttf cambria}
\stopcombination
\stoplinecorrection

\stopsection

\startsection[title=Kerning]

Math kerns in \OPENTYPE\ are quite advanced and use a staircase model left
and|/|or right of characters. However, hardly any math font implements them
(extensively). Therefore we provide a way to fine tune your fonts to your
preferences. You can test this mechanism by patching \type {cambria-math.lfg} by
adding this to the mathematics subtable:

\starttyping
kernpairs = {
    [0x1D44E] = {
        [0x1D44F] = 1000, -- 𝑎𝑏 demo
    }
}
\stoptyping

After that, the next example should work:

\starttyping
\starttext
    \setupbodyfont[cambria]
    \setupmathematics[kernpairs=yes]
    $abcd$
\stoptext
\stoptyping

There should be a gap between the $𝑎$ and $𝑏$. It is not shown here because I
don't want to mess up my goodie file.

\stopsection

\stopchapter

\stopcomponent
