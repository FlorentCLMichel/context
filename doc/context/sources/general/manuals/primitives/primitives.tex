% language=us runpath=texruns:manuals/primitives

\setupexternalfigures
  [location={default,global}]

\enableexperiments[fonts.compact]
\enableexperiments[fonts.accurate]

% \tracepositions

% The usual time stamp. This written while listening intermized to Fish (just ran
% into), Lazulli (some yt videos too, looking forward to a next live act) and
% because this all is boring checking out Sarah Coopers channel for new DT syncs
% every few hours. (Picking up writing this manual in 2023 makes me realize how
% time flies.)
%
% And now it's 2024, so time for a new time (musical) stamp when adding a few
% more descriptions after watching Tyler Visser perform Fear Inoculum in one take
% (think Pneuma): great to discover this in a time of artificial music popping up:
% an amazing drum only version; there is also an older equally amazing version.

% When you feel unhappy about the lack of detail in this manual, just keep in mind
% that you cannot really demand anything from volunteers: just hope for more (or
% pay for it). Friendly comments and corrections are of course always welcome. As
% we like what we're doing here, it all might eventually evolve to perfection, stay
% tuned.
%
% Hans Hagen | j.hagen @ xs4all . nl | ntg-context @ ntg . nl

% iflastnamedcs
% ignorerest

% \enableexperiments[fonts.compact]

% \string\Uabove               \quad:\quad\meaning\mathabove               \par
% \string\Uabovewithdelims     \quad:\quad\meaning\mathabovewithdelims     \par
% \string\Uatop                \quad:\quad\meaning\mathatop                \par
% \string\Uatopwithdelims      \quad:\quad\meaning\mathatopwithdelims      \par
% \string\Udelcode             \quad:\quad\meaning\mathdelcode             \par
% \string\Udelimited           \quad:\quad\meaning\mathdelimited           \par
% \string\Udelimiter           \quad:\quad\meaning\mathdelimiter           \par %
% \string\Udelimiterover       \quad:\quad\meaning\mathdelimiterover       \par
% \string\Udelimiterunder      \quad:\quad\meaning\mathdelimiterunder      \par
% \string\Uhextensible         \quad:\quad\meaning\mathhextensible         \par
% \string\Uleft                \quad:\quad\meaning\mathleft                \par
% \string\Umiddle              \quad:\quad\meaning\mathmiddle              \par %
% \string\Uoperator            \quad:\quad\meaning\mathoperator            \par %
% \string\Uover                \quad:\quad\meaning\mathover                \par %
% \string\Uoverdelimiter       \quad:\quad\meaning\mathoverdelimiter       \par
% \string\Uoverwithdelims      \quad:\quad\meaning\mathoverwithdelims      \par
% \string\Uradical             \quad:\quad\meaning\mathradical             \par %
% \string\Uright               \quad:\quad\meaning\mathright               \par
% \string\Uroot                \quad:\quad\meaning\mathroot                \par
% \string\Urooted              \quad:\quad\meaning\mathrooted              \par
% \string\Uskewed              \quad:\quad\meaning\mathskewed              \par
% \string\Uskewedwithdelims    \quad:\quad\meaning\mathskewedwithdelims    \par
% \string\Ustartdisplaymath    \quad:\quad\meaning\mathstartdisplaymath    \par
% \string\Ustartmath           \quad:\quad\meaning\mathstartmath           \par
% \string\Ustartmathmode       \quad:\quad\meaning\mathstartmathmode       \par
% \string\Ustopdisplaymath     \quad:\quad\meaning\mathstopdisplaymath     \par
% \string\Ustopmath            \quad:\quad\meaning\mathstopmath            \par
% \string\Ustopmathmode        \quad:\quad\meaning\mathstopmathmode        \par
% \string\Ustretched           \quad:\quad\meaning\mathstretched           \par
% \string\Ustretchedwithdelims \quad:\quad\meaning\mathstretchedwithdelims \par
% \string\Uunderdelimiter      \quad:\quad\meaning\mathunderdelimiter      \par
% \string\Uvextensible         \quad:\quad\meaning\mathvextensible         \par

% Umathaccent
% Umathchar
% Umathchardef

% Umathnolimitsubfactor
% Umathnolimitsupfactor
% Umathoverbarkern
% Umathoverbarrule
% Umathoverbarvgap
% Umathoverlinevariant
% Umathspaceafterscript
% Umathspacebeforescript
% Umathspacebetweenscript
% Umathunderbarkern
% Umathunderbarrule
% Umathunderbarvgap
% Umathunderlinevariant

\usemodule[system-syntax]
\usemodule[system-units]
\usemodule[article-basic]
\usemodule[abbreviations-logos]
\usemodule[scite]

\setupinteraction
  [state=start,
   color=,
   contrastcolor=,
   style=,
   contraststyle=]

\definecolor[maincolor][darkblue]
\definecolor[primcolor][darkblue]
\definecolor[nonecolor][darkgray]

\setuptyping
  [option=tex]

\setuptype
  [option=tex]

\setuphead
  [subject]
  [color=maincolor]

\definehead
  [newprimitive]
  [subsection]
  [color=maincolor]

\definehead
  [oldprimitive]
  [subsection]
  [%color=nonecolor,
   color=\ifcstok{\structureuservariable{obsolete}}{yes}darkred\else nonecolor\fi]

\setuplist
  [newprimitive]
  [textcolor=maincolor]

\setuplist
  [oldprimitive]
  [textcolor=nonecolor]

% We use the next one because we want to check what has been done. In a document
% like this using \type {\foo} makes more sense.

\protected\def\prm#1%
% {\doifmode{*bodypart}{\index{\tex{#1}}}\tex{#1}}
% {\ifmode{*bodypart}\index{\tex{#1!\string#1!}}\fi\tex{#1}}
  {\ctxlua{if string.find("#1","^U") then context.writestatus("!!!!!!","#1") end}%
   \ifmode{*bodypart}\index{\tex{#1}}\fi\tex{#1}}

\protected\def\stx#1%
  {\ctxlua{moduledata.engine.specification("#1")}} % only for checking

% This is why we need to tag bodymatter.

\starttext

\startbodymatter

\pushoverloadmode

\startMPpage
    fill Page
        withcolor "darkgray" ;
    draw textext("\sstf {\white new} primitives")
        xysized (.9bbwidth(Page),bbheight(Page)-2cm)
        shifted center Page
        withcolor "maincolor" ;
    draw textext.ulft("\sstf in luametatex")
        xysized (.9bbwidth(Page)/3,(bbheight(Page)-2cm)/6)
        shifted center lrcorner Page
        shifted (-.1bbwidth(Page),.05bbwidth(Page))
        withcolor "white" ;
    setbounds currentpicture to Page ;
\stopMPpage

\startsubject[title={Introduction}]

Here I will discuss some of the new primitives in \LUATEX\ and \LUAMETATEX, the
later being a successor that permits the \CONTEXT\ folks to experiment with new
features. The order is arbitrary. When you compare \LUATEX\ with \PDFTEX, there
are actually quite some differences. Some primitives that \PDFTEX\ introduced
have been dropped in \LUATEX\ because they can be done better in \LUA. Others
have been promoted to core primitives that no longer have a \type {pdf} prefix.
Then there are lots of new primitives, some introduce new concepts, some are a
side effect of for instance new math font technologies, and then there are those
that are handy extensions to the macro language. The \LUAMETATEX\ engine drops
quite some primitives, like those related to \PDFTEX\ specific f(r)ont or backend
features. It also adds some new primitives, mostly concerning the macro language.

We also discuss the primitives that fit into the macro programming scope that are
present in traditional \TEX\ and \ETEX\ but there are for sure better of
explanations out there already. Primitives that relate to typesetting, like those
controlling math, fonts, boxes, attributes, directions, catcodes, \LUA\
(functions) etc are not discussed or discussed in less detail here.

There are for instance primitives to create aliases to low level registers like
counters and dimensions, as well as other (semi|-|numeric) quantities like
characters, but normally these are wrapped into high level macros so that
definitions can't clash too much. Numbers, dimensions etc can be advanced,
multiplied and divided and there is a simple expression mechanism to deal with
them. We don't go into these details here: it's mostly an overview of what the
engine provides. If you are new to \TEX, you need to play a while with its mixed
bag of typesetting and programming features in order to understand the difference
between this macro language and other languages you might be familiar with.

\startcolumns
    \placelist
      [newprimitive,oldprimitive]
      [alternative=c]
\stopcolumns

In this document the section titles that discuss the \color [nonecolor] {original
\TEX\ and \ETEX\ primitives} have a different color those explaining the \color
[primcolor] {\LUATEX\ and \LUAMETATEX\ primitives}.

Primitives that extend typesetting related functionality, provide control over
subsystems (like math), allocate additional data types and resources, deal with
fonts and languages, manipulate boxes and glyphs, etc.\ are hardly discussed
here, only mentioned. Math for instance is a topic of its own. In this document
we concentrate on the programming aspects.

Most of the new primitives are discussed in specific manuals and often also
original primitives are covered there but the best explanations of the
traditional primitives can be found in The \TEX book by Donald Knuth and \TEX\ by
Topic from Victor Eijkhout. I see no need to try to improve on those.

% {\em Some new primitives in this list might be forgotten or already became
% obsolete. Let me know if you run into one.}

\stopsubject

% When writing this manual I also decided to merge some of the condition related
% code so that it dealt a bit more natural with the newer features. A usual side
% effects if writing manuals.

\startsubject[title=Primitives]

\startoldprimitive[title={\prm {<space>}}]

This original \TEX\ primitive is equivalent to the more verbose \prm
{explicitspace}.

\stopoldprimitive

\startoldprimitive[title={\prm {-}}]

This original \TEX\ primitive is equivalent to the more verbose \prm
{explicitdiscretionary}.

\stopoldprimitive

\startoldprimitive[title={\prm {/}}]

This original \TEX\ primitive is equivalent to the more verbose \prm
{explicititaliccorrection}.

\stopoldprimitive

% \startnewprimitive[title={\prm {Uabove}}]
% \stopnewprimitive

% \startnewprimitive[title={\prm {Uabovewithdelims}}]
% \stopnewprimitive

% \startnewprimitive[title={\prm {Uatop}}]
% \stopnewprimitive

% \startnewprimitive[title={\prm {Uatopwithdelims}}]
% \stopnewprimitive

% \startnewprimitive[title={\prm {Udelcode}}]
% \stopnewprimitive

% \startnewprimitive[title={\prm {Udelimited}}]
% \stopnewprimitive

% \startnewprimitive[title={\prm {Udelimiter}}]
% \stopnewprimitive

% \startnewprimitive[title={\prm {Udelimiterover}}]
% \stopnewprimitive

% \startnewprimitive[title={\prm {Udelimiterunder}}]
% \stopnewprimitive

% \startnewprimitive[title={\prm {Uhextensible}}]
% \stopnewprimitive

% \startnewprimitive[title={\prm {Uleft}}]
% \stopnewprimitive

% \startnewprimitive[title={\prm {Umathaccent}}]
% \stopnewprimitive

% \startnewprimitive[title={\prm {Umathaccentbasedepth}}]
% \stopnewprimitive

% \startnewprimitive[title={\prm {Umathaccentbaseheight}}]
% \stopnewprimitive

% \startnewprimitive[title={\prm {Umathaccentbottomovershoot}}]
% \stopnewprimitive

% \startnewprimitive[title={\prm {Umathaccentbottomshiftdown}}]
% \stopnewprimitive

% \startnewprimitive[title={\prm {Umathaccentextendmargin}}]
% \stopnewprimitive

% \startnewprimitive[title={\prm {Umathaccentsuperscriptdrop}}]
% \stopnewprimitive

% \startnewprimitive[title={\prm {Umathaccentsuperscriptpercent}}]
% \stopnewprimitive

% \startnewprimitive[title={\prm {Umathaccenttopovershoot}}]
% \stopnewprimitive

% \startnewprimitive[title={\prm {Umathaccenttopshiftup}}]
% \stopnewprimitive

% \startnewprimitive[title={\prm {Umathaccentvariant}}]
% \stopnewprimitive

% \startnewprimitive[title={\prm {Umathadapttoleft}}]
% \stopnewprimitive

% \startnewprimitive[title={\prm {Umathadapttoright}}]
% \stopnewprimitive

% \startnewprimitive[title={\prm {Umathaxis}}]
% \stopnewprimitive

% \startnewprimitive[title={\prm {Umathbottomaccentvariant}}]
% \stopnewprimitive

% \startnewprimitive[title={\prm {Umathchar}}]
% \stopnewprimitive

% \startnewprimitive[title={\prm {Umathchardef}}]
% \stopnewprimitive

% \startnewprimitive[title={\prm {Umathcode}}]
% \stopnewprimitive

% \startnewprimitive[title={\prm {Umathconnectoroverlapmin}}]
% \stopnewprimitive

% \startnewprimitive[title={\prm {Umathdegreevariant}}]
% \stopnewprimitive

% \startnewprimitive[title={\prm {Umathdelimiterextendmargin}}]
% \stopnewprimitive

% \startnewprimitive[title={\prm {Umathdelimiterovervariant}}]
% \stopnewprimitive

% \startnewprimitive[title={\prm {Umathdelimiterpercent}}]
% \stopnewprimitive

% \startnewprimitive[title={\prm {Umathdelimitershortfall}}]
% \stopnewprimitive

% \startnewprimitive[title={\prm {Umathdelimiterundervariant}}]
% \stopnewprimitive

% \startnewprimitive[title={\prm {Umathdenominatorvariant}}]
% \stopnewprimitive

% \startnewprimitive[title={\prm {Umathdictdef}}]
% \stopnewprimitive

% \startnewprimitive[title={\prm {Umathexheight}}]
% \stopnewprimitive

% \startnewprimitive[title={\prm {Umathextrasubpreshift}}]
% \stopnewprimitive

% \startnewprimitive[title={\prm {Umathextrasubprespace}}]
% \stopnewprimitive

% \startnewprimitive[title={\prm {Umathextrasubshift}}]
% \stopnewprimitive

% \startnewprimitive[title={\prm {Umathextrasubspace}}]
% \stopnewprimitive

% \startnewprimitive[title={\prm {Umathextrasuppreshift}}]
% \stopnewprimitive

% \startnewprimitive[title={\prm {Umathextrasupprespace}}]
% \stopnewprimitive

% \startnewprimitive[title={\prm {Umathextrasupshift}}]
% \stopnewprimitive

% \startnewprimitive[title={\prm {Umathextrasupspace}}]
% \stopnewprimitive

% \startnewprimitive[title={\prm {Umathflattenedaccentbasedepth}}]
% \stopnewprimitive

% \startnewprimitive[title={\prm {Umathflattenedaccentbaseheight}}]
% \stopnewprimitive

% \startnewprimitive[title={\prm {Umathflattenedaccentbottomshiftdown}}]
% \stopnewprimitive

% \startnewprimitive[title={\prm {Umathflattenedaccenttopshiftup}}]
% \stopnewprimitive

% \startnewprimitive[title={\prm {Umathfractiondelsize}}]
% \stopnewprimitive

% \startnewprimitive[title={\prm {Umathfractiondenomdown}}]
% \stopnewprimitive

% \startnewprimitive[title={\prm {Umathfractiondenomvgap}}]
% \stopnewprimitive

% \startnewprimitive[title={\prm {Umathfractionnumup}}]
% \stopnewprimitive

% \startnewprimitive[title={\prm {Umathfractionnumvgap}}]
% \stopnewprimitive

% \startnewprimitive[title={\prm {Umathfractionrule}}]
% \stopnewprimitive

% \startnewprimitive[title={\prm {Umathfractionvariant}}]
% \stopnewprimitive

% \startnewprimitive[title={\prm {Umathhextensiblevariant}}]
% \stopnewprimitive

% \startnewprimitive[title={\prm {Umathlimitabovebgap}}]
% \stopnewprimitive

% \startnewprimitive[title={\prm {Umathlimitabovekern}}]
% \stopnewprimitive

% \startnewprimitive[title={\prm {Umathlimitabovevgap}}]
% \stopnewprimitive

% \startnewprimitive[title={\prm {Umathlimitbelowbgap}}]
% \stopnewprimitive

% \startnewprimitive[title={\prm {Umathlimitbelowkern}}]
% \stopnewprimitive

% \startnewprimitive[title={\prm {Umathlimitbelowvgap}}]
% \stopnewprimitive

% \startnewprimitive[title={\prm {Umathlimits}}]
% \stopnewprimitive

% \startnewprimitive[title={\prm {Umathnoaxis}}]
% \stopnewprimitive

% \startnewprimitive[title={\prm {Umathnolimits}}]
% \stopnewprimitive

% \startnewprimitive[title={\prm {Umathnolimitsubfactor}}]
% \stopnewprimitive

% \startnewprimitive[title={\prm {Umathnolimitsupfactor}}]
% \stopnewprimitive

% \startnewprimitive[title={\prm {Umathnumeratorvariant}}]
% \stopnewprimitive

% \startnewprimitive[title={\prm {Umathopenupdepth}}]
% \stopnewprimitive

% \startnewprimitive[title={\prm {Umathopenupheight}}]
% \stopnewprimitive

% \startnewprimitive[title={\prm {Umathoperatorsize}}]
% \stopnewprimitive

% \startnewprimitive[title={\prm {Umathoverbarkern}}]
% \stopnewprimitive

% \startnewprimitive[title={\prm {Umathoverbarrule}}]
% \stopnewprimitive

% \startnewprimitive[title={\prm {Umathoverbarvgap}}]
% \stopnewprimitive

% \startnewprimitive[title={\prm {Umathoverdelimiterbgap}}]
% \stopnewprimitive

% \startnewprimitive[title={\prm {Umathoverdelimitervariant}}]
% \stopnewprimitive

% \startnewprimitive[title={\prm {Umathoverdelimitervgap}}]
% \stopnewprimitive

% \startnewprimitive[title={\prm {Umathoverlayaccentvariant}}]
% \stopnewprimitive

% \startnewprimitive[title={\prm {Umathoverlinevariant}}]
% \stopnewprimitive

% \startnewprimitive[title={\prm {Umathphantom}}]
% \stopnewprimitive

% \startnewprimitive[title={\prm {Umathprimeraise}}]
% \stopnewprimitive

% \startnewprimitive[title={\prm {Umathprimeraisecomposed}}]
% \stopnewprimitive

% \startnewprimitive[title={\prm {Umathprimeshiftdrop}}]
% \stopnewprimitive

% \startnewprimitive[title={\prm {Umathprimeshiftup}}]
% \stopnewprimitive

% \startnewprimitive[title={\prm {Umathprimespaceafter}}]
% \stopnewprimitive

% \startnewprimitive[title={\prm {Umathprimevariant}}]
% \stopnewprimitive

% \startnewprimitive[title={\prm {Umathquad}}]
% \stopnewprimitive

% \startnewprimitive[title={\prm {Umathradicaldegreeafter}}]
% \stopnewprimitive

% \startnewprimitive[title={\prm {Umathradicaldegreebefore}}]
% \stopnewprimitive

% \startnewprimitive[title={\prm {Umathradicaldegreeraise}}]
% \stopnewprimitive

% \startnewprimitive[title={\prm {Umathradicalextensibleafter}}]
% \stopnewprimitive

% \startnewprimitive[title={\prm {Umathradicalextensiblebefore}}]
% \stopnewprimitive

% \startnewprimitive[title={\prm {Umathradicalkern}}]
% \stopnewprimitive

% \startnewprimitive[title={\prm {Umathradicalrule}}]
% \stopnewprimitive

% \startnewprimitive[title={\prm {Umathradicalvariant}}]
% \stopnewprimitive

% \startnewprimitive[title={\prm {Umathradicalvgap}}]
% \stopnewprimitive

% \startnewprimitive[title={\prm {Umathruledepth}}]
% \stopnewprimitive

% \startnewprimitive[title={\prm {Umathruleheight}}]
% \stopnewprimitive

% \startnewprimitive[title={\prm {Umathskeweddelimitertolerance}}]
% \stopnewprimitive

% \startnewprimitive[title={\prm {Umathskewedfractionhgap}}]
% \stopnewprimitive

% \startnewprimitive[title={\prm {Umathskewedfractionvgap}}]
% \stopnewprimitive

% \startnewprimitive[title={\prm {Umathsource}}]
% \stopnewprimitive

% \startnewprimitive[title={\prm {Umathspaceafterscript}}]
% \stopnewprimitive

% \startnewprimitive[title={\prm {Umathspacebeforescript}}]
% \stopnewprimitive

% \startnewprimitive[title={\prm {Umathspacebetweenscript}}]
% \stopnewprimitive

% \startnewprimitive[title={\prm {Umathstackdenomdown}}]
% \stopnewprimitive

% \startnewprimitive[title={\prm {Umathstacknumup}}]
% \stopnewprimitive

% \startnewprimitive[title={\prm {Umathstackvariant}}]
% \stopnewprimitive

% \startnewprimitive[title={\prm {Umathstackvgap}}]
% \stopnewprimitive

% \startnewprimitive[title={\prm {Umathsubscriptsnap}}]
% \stopnewprimitive

% \startnewprimitive[title={\prm {Umathsubscriptvariant}}]
% \stopnewprimitive

% \startnewprimitive[title={\prm {Umathsubshiftdown}}]
% \stopnewprimitive

% \startnewprimitive[title={\prm {Umathsubshiftdrop}}]
% \stopnewprimitive

% \startnewprimitive[title={\prm {Umathsubsupshiftdown}}]
% \stopnewprimitive

% \startnewprimitive[title={\prm {Umathsubsupvgap}}]
% \stopnewprimitive

% \startnewprimitive[title={\prm {Umathsubtopmax}}]
% \stopnewprimitive

% \startnewprimitive[title={\prm {Umathsupbottommin}}]
% \stopnewprimitive

% \startnewprimitive[title={\prm {Umathsuperscriptsnap}}]
% \stopnewprimitive

% \startnewprimitive[title={\prm {Umathsuperscriptvariant}}]
% \stopnewprimitive

% \startnewprimitive[title={\prm {Umathsupshiftdrop}}]
% \stopnewprimitive

% \startnewprimitive[title={\prm {Umathsupshiftup}}]
% \stopnewprimitive

% \startnewprimitive[title={\prm {Umathsupsubbottommax}}]
% \stopnewprimitive

% \startnewprimitive[title={\prm {Umathtopaccentvariant}}]
% \stopnewprimitive

% \startnewprimitive[title={\prm {Umathunderbarkern}}]
% \stopnewprimitive

% \startnewprimitive[title={\prm {Umathunderbarrule}}]
% \stopnewprimitive

% \startnewprimitive[title={\prm {Umathunderbarvgap}}]
% \stopnewprimitive

% \startnewprimitive[title={\prm {Umathunderdelimiterbgap}}]
% \stopnewprimitive

% \startnewprimitive[title={\prm {Umathunderdelimitervariant}}]
% \stopnewprimitive

% \startnewprimitive[title={\prm {Umathunderdelimitervgap}}]
% \stopnewprimitive

% \startnewprimitive[title={\prm {Umathunderlinevariant}}]
% \stopnewprimitive

% \startnewprimitive[title={\prm {Umathuseaxis}}]
% \stopnewprimitive

% \startnewprimitive[title={\prm {Umathvextensiblevariant}}]
% \stopnewprimitive

% \startnewprimitive[title={\prm {Umathvoid}}]
% \stopnewprimitive

% \startnewprimitive[title={\prm {Umathxscale}}]
% \stopnewprimitive

% \startnewprimitive[title={\prm {Umathyscale}}]
% \stopnewprimitive

% \startnewprimitive[title={\prm {Umiddle}}]
% \stopnewprimitive

% \startnewprimitive[title={\prm {Uoperator}}]
% \stopnewprimitive

% \startnewprimitive[title={\prm {Uover}}]
% \stopnewprimitive

% \startnewprimitive[title={\prm {Uoverdelimiter}}]
% \stopnewprimitive

% \startnewprimitive[title={\prm {Uoverwithdelims}}]
% \stopnewprimitive

% \startnewprimitive[title={\prm {Uradical}}]
% \stopnewprimitive

% \startnewprimitive[title={\prm {Uright}}]
% \stopnewprimitive

% \startnewprimitive[title={\prm {Uroot}}]
% \stopnewprimitive

% \startnewprimitive[title={\prm {Urooted}}]
% \stopnewprimitive

% \startnewprimitive[title={\prm {Uskewed}}]
% \stopnewprimitive

% \startnewprimitive[title={\prm {Uskewedwithdelims}}]
% \stopnewprimitive

% \startnewprimitive[title={\prm {Ustartdisplaymath}}]
% \stopnewprimitive

% \startnewprimitive[title={\prm {Ustartmath}}]
% \stopnewprimitive

% \startnewprimitive[title={\prm {Ustartmathmode}}]
% \stopnewprimitive

% \startnewprimitive[title={\prm {Ustopdisplaymath}}]
% \stopnewprimitive

% \startnewprimitive[title={\prm {Ustopmath}}]
% \stopnewprimitive

% \startnewprimitive[title={\prm {Ustopmathmode}}]
% \stopnewprimitive

% \startnewprimitive[title={\prm {Ustretched}}]
% \stopnewprimitive

% \startnewprimitive[title={\prm {Ustretchedwithdelims}}]
% \stopnewprimitive

% \startnewprimitive[title={\prm {Uunderdelimiter}}]
% \stopnewprimitive

% \startnewprimitive[title={\prm {Uvextensible}}]
% \stopnewprimitive

\startoldprimitive[title={\prm {above}}][obsolete=yes]

This is a variant of \prm {over} that doesn't put a rule in between.

\stopoldprimitive

\startoldprimitive[title={\prm {abovedisplayshortskip}}]

The glue injected before a display formula when the line above it is not
overlapping with the formula. Watch out for interference with \prm
{baselineskip}. It can be controlled by \prm {displayskipmode}.

\stopoldprimitive

\startoldprimitive[title={\prm {abovedisplayskip}}]

The glue injected before a display formula. Watch out for interference with
\prm {baselineskip}. It can be controlled by \prm {displayskipmode}.

\stopoldprimitive

\startoldprimitive[title={\prm {abovewithdelims}}][obsolete=yes]

This is a variant of \prm {atop} but with delimiters. It has a more advanced
upgrade in \prm {Uabovewithdelims}.

\stopoldprimitive

\startoldprimitive[title={\prm {accent}}][obsolete=yes]

This primitive is kind of obsolete in wide engines and takes two arguments: the
indexes of an accent and a base character.

\stopoldprimitive

\startnewprimitive[title={\prm {additionalpageskip}}]

This quantity will be added to the current page goal, stretch and shrink after
which it will be set to zero.

\stopnewprimitive

\startoldprimitive[title={\prm {adjdemerits}}]

When \TEX\ considers to lines to be incompatible it will add this penalty to its
verdict when considering this breakpoint.

\stopoldprimitive

\startnewprimitive[title={\prm {adjustspacing}}]

This parameter controls expansion (hz). A value~2 expands glyphs and font kerns
and a value of~3 only glyphs. Expansion of kerns can have side effects when they
are used for positioning by \OPENTYPE\ features.

\stopnewprimitive

\startnewprimitive[title={\prm {adjustspacingshrink}}]

When set to a non zero value this overloads the shrink maximum in a font when
expansion is applied. This is then the case for all fonts.

\stopnewprimitive

\startnewprimitive[title={\prm {adjustspacingstep}}]

When set to a non zero value this overloads the expansion step in a font when
expansion is applied. This is then the case for all fonts.

\stopnewprimitive

\startnewprimitive[title={\prm {adjustspacingstretch}}]

When set to a non zero value this overloads the stretch maximum in a font when
expansion is applied. This is then the case for all fonts.

\stopnewprimitive

\startoldprimitive[title={\prm {advance}}]

Advances the given register by an also given value:

\starttyping
\advance\scratchdimen      10pt
\advance\scratchdimen      by 3pt
\advance\scratchcounterone \zerocount
\advance\scratchcounterone \scratchcountertwo
\stoptyping

The \type {by} keyword is optional.

\stopoldprimitive

\startnewprimitive[title={\prm {advanceby}}]

This is slightly more efficient variant of \prm {advance} that doesn't look for
\type {by} and therefore, if one is missing, doesn't need to push back the last
seen token. Using \prm {advance} with \type {by} is nearly as efficient but takes
more tokens.

\stopnewprimitive

\startnewprimitive[title={\prm {afterassigned}}]

The \prm {afterassignment} primitive stores a token to be injected (and thereby
expanded) after an assignment has happened. Unlike \prm {aftergroup}, multiple
calls are not accumulated, and changing that would be too incompatible. This is
why we have \prm {afterassigned}, which can be used to inject a bunch of
tokens. But in order to be consistent this one is also not accumulative.

\startbuffer
\afterassigned{done}%
\afterassigned{{\bf done}}%
\scratchcounter=123
\stopbuffer

\typebuffer

results in: \inlinebuffer\ being typeset.

\stopnewprimitive

\startoldprimitive[title={\prm {afterassignment}}]

The token following \prm {afterassignment}, a traditional \TEX\ primitive, is
saved and gets injected (and then expanded) after a following assignment took
place.

\startbuffer
\afterassignment !\def\MyMacro {}\quad
\afterassignment !\let\MyMacro ?\quad
\afterassignment !\scratchcounter 123\quad
\afterassignment !%
\afterassignment ?\advance\scratchcounter by 1
\stopbuffer

\typebuffer

The \prm {afterassignment}s are not accumulated, the last one wins:

{\getbuffer}

\stopoldprimitive

\startoldprimitive[title={\prm {aftergroup}}]

The traditional \TEX\ \prm {aftergroup} primitive stores the next token and
expands that after the group has been closed.

\startbuffer
before{ ! \aftergroup a\aftergroup f\aftergroup t\aftergroup e\aftergroup r}
\stopbuffer

Multiple \prm {aftergroup}s are combined:

\typebuffer

\getbuffer

\stopoldprimitive

\startnewprimitive[title={\prm {aftergrouped}}]

The in itself powerful \prm {aftergroup} primitives works quite well, even
if you need to do more than one thing: you can either use it multiple times, or
you can define a macro that does multiple things and apply that after the group.
However, you can avoid that by using this primitive which takes a list of tokens.

\startbuffer
regular
\bgroup
\aftergrouped{regular}%
\bf bold
\egroup
\stopbuffer

\typebuffer

Because it happens after the group, we're no longer typesetting in bold.

{\getbuffer}

\stopnewprimitive

\startnewprimitive[title={\prm {aliased}}]

This primitive is part of the overload protection subsystem where control sequences
can be tagged.

\startbuffer
\permanent\def\foo{FOO}
          \let\ofo\foo
\aliased  \let\oof\foo

\meaningasis\foo
\meaningasis\ofo
\meaningasis\oof
\stopbuffer

\typebuffer

gives:

\startlines \tt
\getbuffer
\stoplines

When a something is \prm {let} the \quote {permanent}, \quote {primitive} and
\quote {immutable} flags are removed but the \prm {aliased} prefix retains
them.

\startbuffer
\let\relaxed\relax

\meaningasis\relax
\meaningasis\relaxed
\stopbuffer

\typebuffer

So in this example the \type {\relaxed} alias is not flagged as primitive:

\startlines \tt
\getbuffer
\stoplines

\stopnewprimitive

\startnewprimitive[title={\prm {aligncontent}}]

This is equivalent to a hash in an alignment preamble. Contrary to \prm
{alignmark} there is no need to duplicate inside a macro definition.

\stopnewprimitive

\startnewprimitive[title={\prm {alignmark}}]

When you have the \type {#} not set up as macro parameter character cq.\ align
mark, you can use this primitive instead. The same rules apply with respect to
multiple such tokens in (nested) macros and alignments.

\stopnewprimitive

\startnewprimitive[title={\prm {alignmentcellsource}}]

This sets the source id (a box property) of the current alignment cell.

\stopnewprimitive

\startnewprimitive[title={\prm {alignmentwrapsource}}]

This sets the source id (a box property) of the current alignment row (in a \prm
{halign}) or column (in a \prm {valign}).

\stopnewprimitive

\startnewprimitive[title={\prm {aligntab}}]

When you have the \type {&} not set up as align tab, you can use this primitive
instead. The same rules apply with respect to multiple such tokens in (nested)
macros and alignments.

\stopnewprimitive

\startnewprimitive[title={\prm {allcrampedstyles}}]

A symbolic representation of \prm {crampeddisplaystyle}, \prm {crampedtextstyle},
\prm {crampedscriptstyle} and \prm {crampedscriptscriptstyle}; integer
representation: \the\allcrampedstyles.

\stopnewprimitive

\startnewprimitive[title={\prm {alldisplaystyles}}]

A symbolic representation of \prm {displaystyle} and \prm {crampeddisplaystyle};
integer representation: \the\alldisplaystyles.

\stopnewprimitive

\startnewprimitive[title={\prm {allmainstyles}}]

A symbolic representation of \prm {displaystyle}, \prm {crampeddisplaystyle},
\prm {textstyle} and \prm {crampedtextstyle}; integer representation:
\the\allmainstyles.

\stopnewprimitive

\startnewprimitive[title={\prm {allmathstyles}}]

A symbolic representation of \prm {displaystyle}, \prm {crampeddisplaystyle},
\prm {textstyle}, \prm {crampedtextstyle}, \prm {scriptstyle}, \prm
{crampedscriptstyle}, \prm {scriptscriptstyle} and \prm
{crampedscriptscriptstyle}; integer representation: \the\allmathstyles.

\stopnewprimitive

\startnewprimitive[title={\prm {allscriptscriptstyles}}]

A symbolic representation of \prm {scriptscriptstyle} and \prm
{crampedscriptscriptstyle}; integer representation: \the\allscriptscriptstyles.

\stopnewprimitive

\startnewprimitive[title={\prm {allscriptstyles}}]

A symbolic representation of \prm {scriptstyle} and \prm {crampedscriptstyle};
integer representation: \the\allscriptstyles.

\stopnewprimitive

\startnewprimitive[title={\prm {allsplitstyles}}]

A symbolic representation of \prm {displaystyle} and \prm {textstyle} but not
\prm {scriptstyle} and \prm {scriptscriptstyle}: set versus reset; integer
representation: \the\allsplitstyles.

\stopnewprimitive

\startnewprimitive[title={\prm {alltextstyles}}]

A symbolic representation of \prm {textstyle} and \prm {crampedtextstyle};
integer representation: \the\alltextstyles.

\stopnewprimitive

\startnewprimitive[title={\prm {alluncrampedstyles}}]

A symbolic representation of \prm {displaystyle}, \prm {textstyle}, \prm
{scriptstyle} and \prm {scriptscriptstyle}; integer representation:
\the\alluncrampedstyles.

\stopnewprimitive

\startnewprimitive[title={\prm {allunsplitstyles}}]

A symbolic representation of \prm {scriptstyle} and \prm {scriptscriptstyle}; integer
representation: \the\allunsplitstyles.

\stopnewprimitive

\startnewprimitive[title={\prm {amcode}}]
\stopnewprimitive

\startnewprimitive[title={\prm {associateunit}}]

The \TEX\ engine comes with some build in units, like \type {pt} (fixed) and
\type {em} (adaptive). On top of that a macro package can add additional units, which is
what we do in \CONTEXT. In \in {figure} [fig:units] we show the current repertoire.

\startplacefigure[title=Available units,reference=fig:units]
    \showunitsmap[tight]
    \showunitsmaplegend
\stopplacefigure

When this primitive is used in a context where a number is expected it returns the origin
of the unit (in the color legend running from 1 upto 4). A new unit is defined as:

\starttyping
\newdimen\MyDimenZA  \MyDimenZA=10pt

\protected\def\MyDimenAB{\dimexpr\hsize/2\relax}

\associateunit za \MyDimenZA
\associateunit zb \MyMacroZB
\stoptyping

Possible associations are: macros that expand to a dimension, internal dimension
registers, register dimensions (\prm {dimendef}, direct dimensions (\prm
{dimensiondef}) and \LUA\ functions that return a dimension.

One can run into scanning ahead issues where \TEX\ expects a unit and a user unit
gets expanded. This is why for instance in \CONTEXT\ we define the \type{ma} unit
as:

\starttyping
\protected\def\mathaxisunit{\scaledmathaxis\mathstyle\norelax}

\associateunit ma \mathaxisunit % or \newuserunit \mathaxisunit ma
\stoptyping

So that it can be used in rule specifications that themselves look ahead for
keywords and therefore are normally terminated by a \prm {relax}. Adding the
extra \prm {norelax} will make the scanner see one that doesn't get fed back into
the input. Of course a macro package has to manage extra units in order to avoid
conflicts.

\stopnewprimitive

\startnewprimitive[title={\prm {atendoffile}}]

The \prm {everyeof} primitive is kind of useless because you don't know if a file
(which can be a tokenlist processed as pseudo file) itself includes a file, which
then results in nested application of this token register. One way around this is:

\startbuffer
\atendoffile\SomeCommand
\stopbuffer

\typebuffer

This acts on files the same way as \prm {atendofgroup} does. Multiple calls will
be accumulated and are bound to the current file.

\stopnewprimitive

\startnewprimitive[title={\prm {atendoffiled}}]

This is the multi token variant of \prm {atendoffile}. Multiple invocations are
accumulated and by default prepended to the existing list. As with grouping this
permits proper nesting. You can force an append by the optional keyword {reverse}.

\stopnewprimitive

\startnewprimitive[title={\prm {atendofgroup}}]

The token provided will be injected just before the group ends. Because
these tokens are collected, you need to be aware of possible interference
between them. However, normally this is managed by the macro package.

\startbuffer
\bgroup
\atendofgroup\unskip
\atendofgroup )%
(but it works okay
\egroup
\stopbuffer

\typebuffer

Of course these effects can also be achieved by combining (extra) grouping with
\prm {aftergroup} calls, so this is more a convenience primitives than a real
necessity: {\inlinebuffer}, as proven here.

\stopnewprimitive

\startnewprimitive[title={\prm {atendofgrouped}}]

This is the multi token variant of \prm {atendofgroup}. Of course the next
example is somewhat naive when it comes to spacing and so, but it shows the
purpose.

\startbuffer
\bgroup
\atendofgrouped{\bf QED}%
\atendofgrouped{ (indeed)}%
This sometimes looks nicer.
\egroup
\stopbuffer

\typebuffer

Multiple invocations are accumulated: {\inlinebuffer}.

\stopnewprimitive

\startoldprimitive[title={\prm {atop}}][obsolete=yes]

This one stack two math elements on top of each other, like a fraction but with
no rule. It has a more advanced upgrade in \prm {Uatop}.

\stopoldprimitive

\startoldprimitive[title={\prm {atopwithdelims}}][obsolete=yes]

This is a variant of \prm {atop} but with delimiters. It has a more advanced
upgrade in \prm {Uatopwithdelims}.

\stopoldprimitive

\startnewprimitive[title={\prm {attribute}}]

The following sets an attribute(register) value:

\starttyping
\attribute 999 = 123
\stoptyping

An attribute is unset by assigning \the \attributeunsetvalue\ to it. A user needs
to be aware of attributes being used now and in the future of a macro package and
setting them this way is very likely going to interfere.

\stopnewprimitive

\startnewprimitive[title={\prm {attributedef}}]

This primitive can be used to relate a control sequence to an attribute register
and can be used to implement a mechanism for defining unique ones that won't
interfere. As with other registers: leave management to the macro package in
order to avoid unwanted side effects!

\stopnewprimitive

\startnewprimitive[title={\prm {automaticdiscretionary}}]

This is an alias for the automatic hyphen trigger \type {-}.

\stopnewprimitive

\startnewprimitive[title={\prm {automatichyphenpenalty}}]

The penalty injected after an automatic discretionary \type {-}, when \prm
{hyphenationmode} enables this.

\stopnewprimitive

\startnewprimitive[title={\prm {automigrationmode}}]

This bitset determines what will bubble up to an outer level:

\getbuffer[engine:syntax:automigrationcodes]

The current value is {\tttf 0x\tohexadecimal\automigrationmode}.

\stopnewprimitive

\startnewprimitive[title={\prm {autoparagraphmode}}]

A paragraph can be triggered by an empty line, a \prm {par} token or an
equivalent of it. This parameter controls how \type {\par} is interpreted in
different scenarios:

\getbuffer[engine:syntax:autoparagraphcodes]

The current value is {\tttf 0x\tohexadecimal\autoparagraphmode} and setting it to
a non|-|zero value can have consequences for mechanisms that expect otherwise.
The text option uses the same code as an empty line. The macro option checks a
token in a macro preamble against the frozen \prm {\par} token. The last option
ignores the par token.

\stopnewprimitive

\startoldprimitive[title={\prm {badness}}]

This one returns the last encountered badness value.

\stopoldprimitive

\startoldprimitive[title={\prm {baselineskip}}]

This is the maximum glue put between lines. The depth of the previous and height
of the next line are substracted.

\stopoldprimitive

\startoldprimitive[title={\prm {batchmode}}]

This command disables (error) messages which can safe some runtime in situations
where \TEX's character|-|by|-|character log output impacts runtime. It only makes
sense in automated workflows where one doesn't look at the log anyway.

\stopoldprimitive

\startnewprimitive[title={\prm {begincsname}}]

The next code creates a control sequence token from the given serialized tokens:

\starttyping
\csname mymacro\endcsname
\stoptyping

When \type {\mymacro} is not defined a control sequence will be created with the
meaning \prm {relax}. A side effect is that a test for its existence might fail
because it now exists. The next sequence will {\em not} create an controil
sequence:

\starttyping
\begincsname mymacro\endcsname
\stoptyping

This actually is kind of equivalent to:

\starttyping
\ifcsname mymacro\endcsname
    \csname mymacro\endcsname
\fi
\stoptyping

\stopnewprimitive

\startoldprimitive[title={\prm {begingroup}}]

This primitive starts a group and has to be ended with \prm {endgroup}. See \prm
{beginsimplegroup} for more info.

\stopoldprimitive

\startnewprimitive[title={\prm {beginlocalcontrol}}]

Once \TEX\ is initialized it will enter the main loop. In there certain commands
trigger a function that itself can trigger further scanning and functions. In
\LUAMETATEX\ we can have local main loops and we can either enter it from the
\LUA\ end (which we don't discuss here) or at the \TEX\ end using this primitive.

\startbuffer
\scratchcounter100

\edef\whatever{
    a
    \beginlocalcontrol
        \advance\scratchcounter 10
        b
    \endlocalcontrol
    \beginlocalcontrol
        c
    \endlocalcontrol
    d
    \advance\scratchcounter 10
}

\the\scratchcounter
\whatever
\the\scratchcounter
\stopbuffer

\typebuffer

A bit of close reading probably gives an impression of what happens here:

{\getbuffer}

The local loop can actually result in material being injected in the current node
list. However, where normally assignments are not taking place in an \prm
{edef}, here they are applied just fine. Basically we have a local \TEX\ job, be
it that it shares all variables with the parent loop.

\stopnewprimitive

\startnewprimitive[title={\prm {beginmathgroup}}]

In math mode grouping with \prm {begingroup} and \prm {endgroup} in some cases
works as expected, but because the math input is converted in a list that gets
processed later some settings can become persistent, like changes in style or
family. The engine therefore provides the alternatives \prm {beginmathgroup} and
\prm {endmathgroup} that restore some properties.

\stopnewprimitive

\startnewprimitive[title={\prm {beginsimplegroup}}]

The original \TEX\ engine distinguishes two kind of grouping that at the user end
show up as:

\starttyping
\begingroup \endgroup
\bgroup \egroup { }
\stoptyping

where the last two pairs are equivalent unless the scanner explicitly wants to see a
left and|/|or right brace and not an equivalent. For the sake of simplify we use
the aliases here. It is not possible to mix these pairs, so:

\starttyping
\bgroup xxx\endgroup
\begingroup xxx\egroup
\stoptyping

will in both cases issue an error. This can make it somewhat hard to write generic
grouping macros without somewhat dirty trickery. The way out is to use the generic
group opener \prm {beginsimplegroup}.

Internally \LUAMETATEX\ is aware of  what group it currently is dealing with and
there we distinguish:

\starttabulate[||||]
\NC simple group      \NC \type {\bgroup}           \NC \type {\egroup} \NC \NR
\NC semi simple group \NC \type {\begingroup}       \NC \type {\endgroup} \type {\endsimplegroup} \NC \NR
\NC also simple group \NC \type {\beginsimplegroup} \NC \type {\egroup} \type {\endgroup} \type {\endsimplegroup} \NC \NR
\NC math simple group \NC \type {\beginmathgroup}   \NC \type {\endmathgroup} \NC \NR
\stoptabulate

This means that you can say:

\starttyping
\beginsimplegroup xxx\endsimplegroup
\beginsimplegroup xxx\endgroup
\beginsimplegroup xxx\egroup
\stoptyping

So a group started with \prm {beginsimplegroup} can be finished in three ways
which means that the user (or calling macro) doesn't have take into account what
kind of grouping was used to start with. Normally usage of this primitive is
hidden in macros and not something the user has to be aware of.

\stopnewprimitive

\startoldprimitive[title={\prm {belowdisplayshortskip}}]

The glue injected aftter a display formula when the line above it is not
overlapping with the formula (\TEX\ can't look ahead). Watch out for interference
with \prm {baselineskip}. It can be controlled by \prm {displayskipmode}.

\stopoldprimitive

\startoldprimitive[title={\prm {belowdisplayskip}}]

The glue injected after a display formula. Watch out for interference with \prm
{baselineskip}. It can be controlled by \prm {displayskipmode}.

\stopoldprimitive

\startoldprimitive[title={\prm {binoppenalty}}][obsolete=yes]

This internal quantity is a compatibility feature because normally we will use
the inter atom spacing variables.

\stopoldprimitive

\startoldprimitive[title={\prm {botmark}}][obsolete=yes]

This is a reference to the last mark on the current page, it gives back tokens.

\stopoldprimitive

\startoldprimitive[title={\prm {botmarks}}]

This is a reference to the last mark with the given id (a number) on the current
page, it gives back tokens.

\stopoldprimitive

\startnewprimitive[title={\prm {boundary}}]

Boundaries are signals added to he current list. This primitive injects a user
boundary with the given (integer) value. Such a boundary can be consulted at the
\LUA\ end or with \prm {lastboundary}.

\stopnewprimitive

\startoldprimitive[title={\prm {box}}]

This is the box register accessor. While other registers have one property a box
has many, like \prm {wd}, \prm {ht} and \prm {dp}. This primitive returns the box
and resets the register.

\stopoldprimitive

\startnewprimitive[title={\prm {boxadapt}}]

Adapting will recalculate the dimensions with a scale factor for the glue:

\startbuffer
\setbox 0 \hbox       {test test test}
\setbox 2 \hbox {\red  test test test} \boxadapt 0  200
\setbox 4 \hbox {\blue test test test} \boxadapt 0 -200
\ruledhbox{\box0} \vskip-\lineheight
\ruledhbox{\box0} \vskip-\lineheight
\ruledhbox{\box0}
\stopbuffer

\typebuffer

Like \prm {boxfreeze} and \prm {boxrepack} this primitive has been introduced for
experimental usage, although we do use some in production code.

\getbuffer

\stopnewprimitive

\startnewprimitive[title={\prm {boxanchor}}]

This feature is part of an (experimental) mechanism that relates boxes. The
engine just tags a box and it is up to the macro package to deal with it.

\startbuffer
\setbox0\hbox anchor "01010202 {test}\tohexadecimal\boxanchor0
\stopbuffer

\typebuffer

This gives: \inlinebuffer. Of course this feature is very macro specific and
should not be used across macro packages without coordination. An anchor has
two parts each not exceeding \type {0x0FFF}.

\stopnewprimitive

\startnewprimitive[title={\prm {boxanchors}}]

This feature is part of an (experimental) mechanism that relates boxes. The
engine just tags a box and it is up to the macro package to deal with it.

\startbuffer
\setbox0\hbox anchors "0101 "0202 {test}\tohexadecimal\boxanchors0
\stopbuffer

\typebuffer

This gives: \inlinebuffer. Of course this feature is very macro specific and
should not be used across macro packages without coordination. An anchor has
two parts each not exceeding \type {0x0FFF}.

\stopnewprimitive

\startnewprimitive[title={\prm {boxattribute}}]

Every node, and therefore also every box gets the attributes set that are
active at the moment of creation. Additional attributes can be set too:

\startbuffer
\darkred
\setbox0\hbox attr 9999 1 {whatever}
\the\boxattribute 0 \colorattribute
\the\boxattribute 0 9998
\the\boxattribute 0 9999
\stopbuffer

\typebuffer

A macro package should make provide a way define attributes that don't clash the
ones it needs itself, like, in \CONTEXT, the ones that can set a color

\startlines \getbuffer \stoplines

The number \the\attributeunsetvalue\ (\tohexadecimal\attributeunsetvalue)
indicates an unset attribute.

\stopnewprimitive

\startnewprimitive[title={\prm {boxdirection}}]

The direction of a box defaults to \type {l2r} but can be explicitly set:

\startbuffer
\setbox0\hbox direction 1 {this is a test}\textdirection1
\setbox2\hbox direction 0 {this is a test}\textdirection0
\the\boxdirection0: \box0
\the\boxdirection2: \box2
\stopbuffer

\typebuffer

The \prm {textdirection} does not influence the box direction:

\startlines \getbuffer \stoplines

\stopnewprimitive

\startnewprimitive[title={\prm {boxfinalize}}]

This is special version of \prm {boxfreeze} which we demonstrate
with an example:

\startbuffer[setthem]
\setbox0\ruledvbox to 3cm{\hsize 2cm test\vskip10pt plus 10pt test}
\setbox2\copy0\setbox4\copy0\setbox6\copy0\setbox8\copy0
\stopbuffer

\startbuffer[usethem]
\boxlimitate 0 0   % don't recurse
\boxfreeze   2 0   % don't recurse
\boxfinalize 4 500 % scale glue multiplier by .50
\boxfinalize 6 250 % scale glue multiplier by .25
\boxfinalize 8 100 % scale glue multiplier by .10

\hpack\bgroup
    \copy0\quad\copy2\quad\copy4\quad\copy6\quad\copy8
\egroup
\stopbuffer

\typebuffer[usethem]

where the boxes are populated with:

\typebuffer[setthem]

\startlinecorrection
\getbuffer[setthem,usethem]
\stoplinecorrection

\stopnewprimitive

\startnewprimitive[title={\prm {boxfreeze}}]

Glue in a box has a fixed component that will always be used and stretch and
shrink that kicks in when needed. The effective value (width) of the glue is
driven by some box parameters that are set by the packaging routine. This is why
we can unbox: the original value is kept. It is the backend that calculates the
effective value. Te \prm {boxfreeze} primitive can do the same: turn the flexible
glue into a fixed one.

\startbuffer
\setbox    0 \hbox to 6cm {\hss frost}
\setbox    2 \hbox to 6cm {\hss frost}
\boxfreeze 2 0
\ruledhbox{\unhbox 0}
\ruledhbox{\unhbox 2}
\stopbuffer

\typebuffer

The second parameter to \prm {boxfreeze} determines recursion. We don't recurse
here so just freeze the outer level:

\getbuffer

\stopnewprimitive

\startnewprimitive[title={\prm {boxgeometry}}]

A box can have an orientation, offsets and|/|or anchors. These are stored
independently but for efficiency reasons we register if one or more of these
properties is set. This primitive accesses this state; it is a bitset:

\getbuffer[engine:syntax:listgeometrycodes]

\stopnewprimitive

\startnewprimitive[title={\prm {boxlimit}}]

This primitive will freeze the glue in a box but only when there is glue marked
with the limit option.

\stopnewprimitive

\startnewprimitive[title={\prm {boxlimitate}}]

This primitive will freeze the glue in a box. It takes two arguments, a box
number and an number that when set to non|-|zero will recurse into nested lists.

\stopnewprimitive

\startnewprimitive[title={\prm {boxlimitmode}}]

This variable controls if boxes with glue marked \quote {limit} will be checked
and frozen.

\stopnewprimitive

\startoldprimitive[title={\prm {boxmaxdepth}}]

You can limit the depth of boxes being constructed. It's one of these parameters
that should be used with care because when that box is filled nested boxes can be
influenced.

\stopoldprimitive

\startnewprimitive[title={\prm {boxorientation}}]

The orientation field can take quite some values and is discussed in one of the low level
\CONTEXT\ manuals. Some properties are dealt with in the \TEX\ engine because they
influence dimensions but in the end it is the backend that does the work.

\stopnewprimitive

\startnewprimitive[title={\prm {boxrepack}}]

When a box of to wide or tight we can tweak it a bit with this primitive. The
primitive expects a box register and a dimension, where a positive number adds
and a negatie subtracts from the current box with.

\startbuffer
\setbox 0 \hbox        {test test test}
\setbox 2 \hbox {\red   test test test} \boxrepack0 +.2em
\setbox 4 \hbox {\green test test test} \boxrepack0 -.2em
\ruledhbox{\box0} \vskip-\lineheight
\ruledhbox{\box0} \vskip-\lineheight
\ruledhbox{\box0}
\stopbuffer

\typebuffer

\getbuffer

We can also use this primitive to check the natural dimensions of a box:

\startbuffer
\setbox 0 \hbox spread 10pt {test test test}
\ruledhbox{\box0} (\the\boxrepack0,\the\wd0)
\stopbuffer

\typebuffer

In this context only one argument is expected.

\getbuffer

\stopnewprimitive

\startnewprimitive[title={\prm {boxshift}}]

Returns or sets how much the box is shifted: up or down in horizontally mode, left or
right in vertical mode.

\stopnewprimitive

\startnewprimitive[title={\prm {boxshrink}}]

Returns the amount of shrink found (applied) in a box:

\startbuffer
\setbox0\hbox to 4em {m m m m}
\the\boxshrink0
\stopbuffer

\typebuffer

gives: \inlinebuffer

\stopnewprimitive

\startnewprimitive[title={\prm {boxsource}}]

This feature is part of an (experimental) mechanism that relates boxes. The
engine just tags a box and it is up to the macro package to deal with it.

\startbuffer
\setbox0\hbox source 123 {m m m m}
\the\boxsource0
\stopbuffer

\typebuffer

This gives: \inlinebuffer. Of course this feature is very macro specific and
should not be used across macro packages without coordination.

\stopnewprimitive

\startnewprimitive[title={\prm {boxstretch}}]

Returns the amount of stretch found (applied) in a box:

\startbuffer
\setbox0\hbox to 6em {m m m m}
\the\boxstretch0
\stopbuffer

\typebuffer

gives: \inlinebuffer

\stopnewprimitive

\startnewprimitive[title={\prm {boxtarget}}]

This feature is part of an (experimental) mechanism that relates boxes. The
engine just tags a box and it is up to the macro package to deal with it.

\startbuffer
\setbox0\hbox source 123 {m m m m}
\the\boxsource0
\stopbuffer

\typebuffer

This gives: \inlinebuffer. Of course this feature is very macro specific and
should not be used across macro packages without coordination.

\stopnewprimitive

\startnewprimitive[title={\prm {boxtotal}}]

Returns the total of height and depth of the given box.

\stopnewprimitive

\startnewprimitive[title={\prm {boxvadjust}}]

When used as query this returns a bitset indicating the associated adjust and
migration (marks and inserts) data:

\starttabulate[|T|l|]
\NC 0x1 \NC pre  adjusted \NC \NR
\NC 0x2 \NC post adjusted \NC \NR
\NC 0x4 \NC pre  migrated \NC \NR
\NC 0x8 \NC post migrated \NC \NR
\stoptabulate\

When used as a setter it directly adds adjust data to the box and it accepts the
same keywords as \prm {vadjust}.

\stopnewprimitive

\startnewprimitive[title={\prm {boxxmove}}]

This will set the vertical offset and adapt the dimensions accordingly.

\stopnewprimitive

\startnewprimitive[title={\prm {boxxoffset}}]

Returns or sets the horizontal offset of the given box.

\stopnewprimitive

\startnewprimitive[title={\prm {boxymove}}]

This will set the vertical offset and adapt the dimensions accordingly.

\stopnewprimitive

\startnewprimitive[title={\prm {boxyoffset}}]

Returns or sets the vertical offset of the given box.

\stopnewprimitive

\startnewprimitive[title={\prm {brokenpenalties}}]

Together with \prm {widowpenalties} and \prm {clubpenalties} this one permits
discriminating left- and right page (doublesided) penalties. For this one needs
to also specify \prm {options 4} and provide penalty pairs. Where the others
accept multiple pairs, this primitives expects a count value one.

\stopnewprimitive

\startoldprimitive[title={\prm {brokenpenalty}}]

This penalty is added after a line that ends with a hyphen; it can help to
discourage a page break (or split in a box).

\stopoldprimitive

\startoldprimitive[title={\prm {catcode}}]

Every character can be put in a category, but this is typically
something that the macro package manages because changes can affect
behavior. Also, once passed as an argument, the catcode of a character
is frozen. There are 16 different values:

\starttabulate[|l|c|l|c|]
\NC \type {\escapecatcode     } \NC \the\escapecatcode
\NC \type {\begingroupcatcode } \NC \the\begingroupcatcode  \NC \NR
\NC \type {\endgroupcatcode   } \NC \the\endgroupcatcode
\NC \type {\mathshiftcatcode  } \NC \the\mathshiftcatcode   \NC \NR
\NC \type {\alignmentcatcode  } \NC \the\alignmentcatcode
\NC \type {\endoflinecatcode  } \NC \the\endoflinecatcode   \NC \NR
\NC \type {\parametercatcode  } \NC \the\parametercatcode
\NC \type {\superscriptcatcode} \NC \the\superscriptcatcode \NC \NR
\NC \type {\subscriptcatcode  } \NC \the\subscriptcatcode
\NC \type {\ignorecatcode     } \NC \the\ignorecatcode      \NC \NR
\NC \type {\spacecatcode      } \NC \the\spacecatcode
\NC \type {\lettercatcode     } \NC \the\lettercatcode      \NC \NR
\NC \type {\othercatcode      } \NC \the\othercatcode
\NC \type {\activecatcode     } \NC \the\activecatcode      \NC \NR
\NC \type {\commentcatcode    } \NC \the\commentcatcode
\NC \type {\invalidcatcode    } \NC \the\invalidcatcode     \NC \NR
\stoptabulate

The first column shows the constant that \CONTEXT\ provides and the
name indicates the purpose. Here are two examples:

\starttyping
\catcode123=\begingroupcatcode
\catcode125=\endgroupcatcode
\stoptyping

\stopoldprimitive

\startnewprimitive[title={\prm {catcodetable}}]

The catcode table with the given index will become active.

\stopnewprimitive

\startnewprimitive[title={\prm {cdef}}]

This primitive is like \prm {edef} but in some usage scenarios is slightly
more efficient because (delayed) expansion is ignored which in turn saves
building a temporary token list.

\startbuffer
\edef\FooA{this is foo} \meaningfull\FooA\crlf
\cdef\FooB{this is foo} \meaningfull\FooB\par
\stopbuffer

\typebuffer {\tttf \getbuffer}

\stopnewprimitive

\startnewprimitive[title={\prm {cdefcsname}}]

This primitive is like \prm {edefcsame} but in some usage scenarios is slightly
more efficient because (delayed) expansion is ignored which in turn saves
building a temporary token list.

\startbuffer
\edefcsname FooA\endcsname{this is foo} \meaningasis\FooA\crlf
\cdefcsname FooB\endcsname{this is foo} \meaningasis\FooB\par
\stopbuffer

\typebuffer {\tttf \getbuffer}

\stopnewprimitive

\startnewprimitive[title={\prm {cfcode}}]

This primitive is a companion to \prm {efcode} and sets the compression factor.
It takes three values: font, character code, and factor.

\stopnewprimitive

\startoldprimitive[title={\prm {char}}]

This appends a character with the given index in the current font.

\stopoldprimitive

\startoldprimitive[title={\prm {chardef}}]

The following definition relates a control sequence to a specific character:

\starttyping
\chardef\copyrightsign"A9
\stoptyping

However, because in a context where a number is expected, such a \prm {chardef}
is seen as valid number, there was a time when this primitive was used to define
constants without overflowing the by then limited pool of count registers. In
\ETEX\ aware engines this was less needed, and in \LUAMETATEX\ we have \prm
{integerdef} as a more natural candidate.

\stopoldprimitive

\startoldprimitive[title={\prm {cleaders}}]

See \prm {gleaders} for an explanation.

\stopoldprimitive

\startnewprimitive[title={\prm {clearmarks}}]

This primitive is an addition to the multiple marks mechanism that originates in
\ETEX\ and reset the mark registers of the given category (a number).

\stopnewprimitive

\startoldprimitive[title={\prm {clubpenalties}}]

This is an array of penalty put before the first lines in a paragraph. High values
discourage (or even prevent) a lone line at the end of a page. This
command expects a count value indicating the number of entries that will follow.
The first entry is ends up after the first line.

\stopoldprimitive

\startoldprimitive[title={\prm {clubpenalty}}]

This is the penalty put before a club line in a paragraph. High values discourage
(or even prevent) a lone line at the end of a next page.

\stopoldprimitive

\startnewprimitive[title={\prm {constant}}]

This prefix tags a macro (without arguments) as being constant. The main
consequence is that in some cases expansion gets delayed which gives a little
performance boost and less (temporary) memory usage, for instance in \type
{\csname} like scenarios.

\stopnewprimitive

\startnewprimitive[title={\prm {constrained}}]

See previous section about \prm {retained}.

\stopnewprimitive

\startoldprimitive[title={\prm {copy}}]

This is the box register accessor that returns a copy of the box.

\stopoldprimitive

\startnewprimitive[title={\prm {copymathatomrule}}]

This copies the rule bitset from the parent class (second argument) to the target
class (first argument). The bitset controls the features that apply to atoms.

\stopnewprimitive

\startnewprimitive[title={\prm {copymathparent}}]

This binds the given class (first argument) to another class (second argument) so
that one doesn't need to define all properties.

\stopnewprimitive

\startnewprimitive[title={\prm {copymathspacing}}]

This copies an class spacing specification to another one, so in

\starttyping
\copymathspacing 34 2
\stoptyping

class 34 (a user one) get the spacing from class 2 (binary).

\stopnewprimitive

\startoldprimitive[title={\prm {count}}]

This accesses a count register by index. This is kind of \quote {not done} unless
you do it local and make sure that it doesn't influence macros that you call.

\starttyping
\count4023=10
\stoptyping

In standard \TEX\ the first 10 counters are special because they get reported to
the console, and \type {\count0} is then assumed to be the page counter.

\stopoldprimitive

\startoldprimitive[title={\prm {countdef}}]

This primitive relates a control sequence to a count register. Compare this to
the example in the previous section.

\starttyping
\countdef\MyCounter4023
\MyCounter=10
\stoptyping

However, this is also \quote {not done}. Instead one should use the allocator that
the macro package provides.

\starttyping
\newcount\MyCounter
\MyCounter=10
\stoptyping

In \LUAMETATEX\ we also have integers that don't rely on registers. These are
assigned by the primitive \prm {integerdef}:

\starttyping
\integerdef\MyCounterA 10
\stoptyping

Or better \type {\newinteger}.

\starttyping
\newinteger\MyCounterB
\MyCounterN10
\stoptyping

There is a lowlevel manual on registers.

\stopoldprimitive

\startoldprimitive[title={\prm {cr}}]

This ends a row in an alignment. It also ends an alignment preamble.
\stopoldprimitive

\startnewprimitive[title={\prm {crampeddisplaystyle}}]

A less spacy alternative of \prm {displaystyle}; integer representation:
\the\scriptstyle.

\stopnewprimitive

\startnewprimitive[title={\prm {crampedscriptscriptstyle}}]

A less spacy alternative of \prm {scriptscriptstyle}; integer representation:
\the\scriptscriptstyle.

\stopnewprimitive

\startnewprimitive[title={\prm {crampedscriptstyle}}]

A less spacy alternative of \prm {scriptstyle}; integer representation:
\the\scriptstyle.

\stopnewprimitive

\startnewprimitive[title={\prm {crampedtextstyle}}]

A less spacy alternative of \prm {textstyle}; integer representation:
\the\textstyle.

\stopnewprimitive

\startoldprimitive[title={\prm {crcr}}]

This ends a row in an alignment when it hasn't ended yet.

\stopoldprimitive

\startnewprimitive[title={\prm {csactive}}]

Because \LUATEX\ (and \LUAMETATEX) are \UNICODE\ engines active characters are
implemented a bit differently. They don't occupy a eight bit range of characters
but are stored as control sequence with a special prefix \type {U+FFFF} which
never shows up in documents. The \prm {csstring} primitive injects the name of a
control sequence without leading escape character, the \prm {csactive} injects
the internal name of the following (either of not active) character. As we cannot
display the prefix: \type {\csactive~} will inject the \UTF\ sequences for \type
{U+FFFF} and \type {U+007E}, so here we get the bytes \type {EFBFBF7E}. Basically
the next token is preceded by \prm {string}, so when you don't provide a
character you are in for a surprise.

\stopnewprimitive

\startoldprimitive[title={\prm {csname}}]

This original \TEX\ primitive starts the construction of a control sequence
reference. It does a lookup and when no sequence with than name is found, it will
create a hash entry and defaults its meaning to \prm {relax}.

\starttyping
\csname letters and other characters\endcsname
\stoptyping

\stopoldprimitive

\startnewprimitive[title={\prm {csstring}}]

This primitive returns the name of the control sequence given without the leading
escape character (normally a backslash). Of course you could strip that character
with a simple helper but this is more natural.

\startbuffer
\csstring\mymacro
\stopbuffer

\typebuffer

We get the name, not the meaning: {\tt \inlinebuffer}.

\stopnewprimitive

\startoldprimitive[title={\prm {currentgrouplevel}}]

\startbuffer
[\the\currentgrouplevel] \bgroup
    [\the\currentgrouplevel] \bgroup
        [\the\currentgrouplevel]
    \egroup [\the\currentgrouplevel]
\egroup [\the\currentgrouplevel]
\stopbuffer

The next example gives: \inlinebuffer.

\typebuffer

\stopoldprimitive

\startoldprimitive[title={\prm {currentgrouptype}}]

\startbuffer
[\the\currentgrouptype] \bgroup
    [\the\currentgrouptype] \begingroup
        [\the\currentgrouptype]
    \endgroup [\the\currentgrouptype]
    [\the\currentgrouptype] \beginmathgroup
        [\the\currentgrouptype]
    \endmathgroup [\the\currentgrouptype]
[\the\currentgrouptype] \egroup
\stopbuffer

The next example gives: \inlinebuffer.

\typebuffer

The possible values depend in the engine and for \LUAMETATEX\ they are:

\startluacode
context.startcolumns { n = 4 }
context.starttabulate { "|r|l|" }
for i=0,#tex.groupcodes do
    context.NC() context(i)
    context.NC() context(tex.groupcodes[i])
    context.NC() context.NR()
end
context.stoptabulate()
context.stopcolumns()
\stopluacode

\stopoldprimitive

\startoldprimitive[title={\prm {currentifbranch}}]

\startbuffer
[\the\currentifbranch] \iftrue
    [\the\currentifbranch] \iffalse
        [\the\currentifbranch]
    \else
        [\the\currentifbranch]
    \fi [\the\currentifbranch]
\fi [\the\currentifbranch]
\stopbuffer

The next example gives: \inlinebuffer.

\typebuffer

So when in the \quote {then} branch we get plus one and when in the \quote {else}
branch we end up with a minus one.

\stopoldprimitive

\startoldprimitive[title={\prm {currentiflevel}}]

\startbuffer
[\the\currentiflevel] \iftrue
    [\the\currentiflevel]\iftrue
        [\the\currentiflevel] \iftrue
            [\the\currentiflevel]
        \fi [\the\currentiflevel]
    \fi [\the\currentiflevel]
\fi [\the\currentiflevel]
\stopbuffer

The next example gives: \inlinebuffer.

\typebuffer

\stopoldprimitive

\startoldprimitive[title={\prm {currentiftype}}]

\startbuffer
[\the\currentiftype] \iftrue
    [\the\currentiftype]\iftrue
        [\the\currentiftype] \iftrue
            [\the\currentiftype]
        \fi [\the\currentiftype]
    \fi [\the\currentiftype]
\fi [\the\currentiftype]
\stopbuffer

The next example gives: \inlinebuffer.

\typebuffer

The values are engine dependent:

\startluacode
local t = tex.getiftypes()
context.startcolumns { n = 5 }
context.starttabulate { "|r|l|" }
for i=0,#tex.groupcodes do
    context.NC() context(i)
    context.NC() context(t[i])
    context.NC() context.NR()
end
context.stoptabulate()
context.stopcolumns()
\stopluacode

\stopoldprimitive

\startnewprimitive[title={\prm {currentloopiterator}}]

Here we show the different expanded loop variants:

\startbuffer
\edef\testA{\expandedloop  1 10 1{!}}
\edef\testB{\expandedrepeat  10  {!}}
\edef\testC{\expandedendless     {\ifnum\currentloopiterator>10 \quitloop\else !\fi}}
\edef\testD{\expandedendless     {\ifnum#I>10 \quitloop\else !\fi}}
\stopbuffer

\typebuffer \getbuffer

All these give the same result:

\startlines \tt
\meaningasis\testA
\meaningasis\testB
\meaningasis\testC
\meaningasis\testD
\stoplines

The \type {#I} is a shortcut to the current loop iterator; other shortcuts are
\type {#P} for the parent iterator value and \type {#G} for the grand parent.

\stopnewprimitive

\startnewprimitive[title={\prm {currentloopnesting}}]

This integer reports how many nested loops are currently active. Of course in
practice the value only has meaning when you know at what outer level your nested
loop started.

\stopnewprimitive

% \startnewprimitive[title={\prm {currentlysetmathstyle}}]
% \stopnewprimitive

\startnewprimitive[title={\prm {currentmarks}}]

Marks only get updated when a page is split off or part of a box using \prm
{vsplit} gets wrapped up. This primitive gives access to the current value of a
mark and takes the number of a mark class.

\stopnewprimitive

\startoldprimitive[title={\prm {currentstacksize}}]

This is more diagnostic feature than a useful one but we show it anyway. There is
some basic overhead when we enter a group:

\startbuffer
\bgroup [\the\currentstacksize]
    \bgroup [\the\currentstacksize]
        \bgroup [\the\currentstacksize]
        [\the\currentstacksize] \egroup
    [\the\currentstacksize] \egroup
[\the\currentstacksize] \egroup
\stopbuffer

\typebuffer \getbuffer

As soon as we define something or change a value, the stack gets populated by
information needed for recovery after the group ends.

\startbuffer
\bgroup [\the\currentstacksize]
    \scratchcounter 1
    \bgroup [\the\currentstacksize]
        \scratchdimen 1pt
        \scratchdimen 2pt
        \bgroup [\the\currentstacksize]
            \scratchcounter 2
            \scratchcounter 3
        [\the\currentstacksize] \egroup
    [\the\currentstacksize] \egroup
[\the\currentstacksize] \egroup
\stopbuffer

\typebuffer \getbuffer

The stack also keeps some state information, for instance when a box is being
built. In \LUAMETATEX\ that is is quite a bit more than in other engines but it
is compensated by more efficient save stack handling elsewhere.

\startbuffer
\hbox \bgroup [\the\currentstacksize]
    \hbox \bgroup [\the\currentstacksize]
        \hbox \bgroup [\the\currentstacksize]
        [\the\currentstacksize] \egroup
    [\the\currentstacksize] \egroup
[\the\currentstacksize] \egroup
\stopbuffer

\typebuffer \getbuffer

\stopoldprimitive

\startoldprimitive[title={\prm {day}}]

This internal number starts out with the day that the job started.

\stopoldprimitive

\startnewprimitive[title={\prm {dbox}}]

A \prm {dbox} is just a \prm {vbox} (baseline at the bottom) but it has the
property \quote {dual baseline} which means that is some cases it will behave
like a \prm {vtop} (baseline at the top) too. Like:

\ruledhbox\bgroup \showstruts
    \ruleddbox   {\hsize 3cm \strut dbox    \par \strut dbox    \par\strut dbox   }
    \ruledvbox   {\hsize 3cm \strut vbox    \par \strut vbox    \par\strut vbox   }
    \ruledvtop   {\hsize 3cm \strut vtop    \par \strut vtop    \par\strut vtop   }
    \ruledvcenter{\hsize 3cm \strut vcenter \par \strut vcenter \par\strut vcenter}%
\egroup

A \type {\dbox} behaves like a \type {\vtop} when it's appended to a vertical list which
means that the height of the first box or rule determines the (base)line correction that
gets applied.

% This example is taken from lowlevel-boxes:

\startlinecorrection
\startcombination [nx=3,ny=1,location=top]
    {\vbox \bgroup \hsize .3\textwidth
        \small\small \setupalign[tolerant,stretch] \dontcomplain
        xxxxxxxxxxxxxxxx\par
        \ruledvbox{\samplefile{ward}}\par
        xxxxxxxxxxxxxxxx\par
     \egroup} {\type {\vbox}}
    {\vbox \bgroup \hsize .3\textwidth
        \small\small \setupalign[tolerant,stretch] \dontcomplain
        xxxxxxxxxxxxxxxx\par
        \ruledvtop{\samplefile{ward}}\par
        xxxxxxxxxxxxxxxx\par
     \egroup} {\type {\vtop}}
    {\vbox \bgroup \hsize .3\textwidth
        \small\small \setupalign[tolerant,stretch] \dontcomplain
        xxxxxxxxxxxxxxxx\par
        \ruleddbox{\samplefile{ward}}\par
        xxxxxxxxxxxxxxxx\par
     \egroup} {\type {\dbox}}
\stopcombination
\stoplinecorrection

\stopnewprimitive

\startoldprimitive[title={\prm {deadcycles}}]

This counter is incremented every time the output routine is entered. When \prm
{maxdeadcycles} is reached \TEX\ will issue an error message, so you'd better
reset its value when a page is done.

\stopoldprimitive

\startoldprimitive[title={\prm {def}}]

This is the main definition command, as in:

\starttyping
\def\foo{l me}
\stoptyping

with companions like \prm {gdef}, \prm {edef}, \prm {xdef}, etc. and variants
like:

\starttyping
\def\foo#1{... #1...}
\stoptyping

where the hash is used in the preamble and for referencing. More about that can
be found in the low level manual about macros.

\stopoldprimitive

\startoldprimitive[title={\prm {defaulthyphenchar}}][obsolete=yes]

When a font is loaded its hyphen character is set to this value. It can be
changed afterwards. However, in \LUAMETATEX\ font loading is under \LUA\ control
so these properties can be set otherwise.

\stopoldprimitive

\startoldprimitive[title={\prm {defaultskewchar}}][obsolete=yes]

When a font is loaded its skew character is set to this value. It can be changed
afterwards. However, in \LUAMETATEX\ font loading is under \LUA\ control so these
properties can be set otherwise. Also, \OPENTYPE\ math fonts have top anchor
instead.

\stopoldprimitive

\startnewprimitive[title={\prm {defcsname}}]

We now get a series of log clutter avoidance primitives. It's fine if you argue
that they are not really needed, just don't use them.

\starttyping
\expandafter\def\csname MyMacro:1\endcsname{...}
             \defcsname MyMacro:1\endcsname{...}
\stoptyping

The fact that \TEX\ has three (expanded and global) companions can be seen as a
signal that less verbosity makes sense. It's just that macro packages use plenty
of \prm {csname}'s.

\stopnewprimitive

\startnewprimitive[title={\prm {deferred}}]

This is mostly a compatibility prefix and it can be checked at the \LUA\ end when
there is a \LUA\ based assignment going on. It is the counterpart of \prm
{immediate}. In the traditional engines a \prm {write} is normally deferred
(turned into a node) and can be handled \prm {immediate}, while a \prm {special}
does the opposite.

\stopnewprimitive

\startoldprimitive[title={\prm {delcode}}][obsolete=yes]

This assigns delimiter properties to an eight bit character so it has little use
in an \OPENTYPE\ math setup. WHen the assigned value is hex encoded, the first
byte denotes the small family, then we have two bytes for the small index,
followed by three similar bytes for the large variant.

\stopoldprimitive

\startoldprimitive[title={\prm {delimiter}}][obsolete=yes]

This command inserts a delimiter with the given specification. In \OPENTYPE\ math
we use a different command so it is unlikely that this primitive is used in
\LUAMETATEX. It takes a number that can best be coded hexadecimal: one byte for
the class, one for the small family, two for the small index, one for the large
family and two for the large index. This demonstrates that it can't handle wide
fonts. Also, in \OPENTYPE\ math fonts the larger sizes and extensible come from
the same font as the small symbol. On top of that, in \LUAMETATEX\ we have more
classes than fit in a byte.

\stopoldprimitive

\startoldprimitive[title={\prm {delimiterfactor}}][obsolete=yes]

This is one of the parameters that determines the size of a delimiter: at least
this factor times the formula height divided by 1000. In \OPENTYPE\ math
different properties and strategies are used.

\stopoldprimitive

\startoldprimitive[title={\prm {delimitershortfall}}][obsolete=yes]

This is one of the parameters that determines the size of a delimiter: at least
the formula height minus this parameter. In \OPENTYPE\ math different properties
and strategies are used.

\stopoldprimitive

\startnewprimitive[title={\prm {detokened}}]

The following token will be serialized into characters with category \quote
{other}.

\startbuffer
\toks0{123}
\def\foo{let's be \relax'd}
\def\oof#1{let's see #1}
\detokened\toks0
\detokened\foo
\detokened\oof
\detokened\setbox
\detokened X
\stopbuffer

\typebuffer

Gives:

\startlines\tt
\getbuffer
\stoplines

Macros with arguments are not shown.

\stopnewprimitive

\startoldprimitive[title={\prm {detokenize}}]

This \ETEX\ primitive turns the content of the provides list will become
characters, kind of verbatim.

\startbuffer
\expandafter\let\expandafter\temp\detokenize{1} \meaning\temp
\expandafter\let\expandafter\temp\detokenize{A} \meaning\temp
\stopbuffer

\typebuffer

\startlines \tttf \getbuffer \stoplines

\stopoldprimitive

\startnewprimitive[title={\prm {detokenized}}]

The following (single) token will be serialized into characters with category
\quote {other}.

\startbuffer
\toks0{123}
\def\foo{let's be \relax'd}
\def\oof#1{let's see #1}
\detokenized\toks0
\detokenized\foo
\detokenized\oof
\detokenized\setbox
\detokenized X
\stopbuffer

\typebuffer

Gives:

\startlines\tt
\getbuffer
\stoplines

It is one of these new primitives that complement others like \prm {detokened}
and such, and they are often mostly useful in experiments of some low level
magic, which made them stay.

\stopnewprimitive

\startoldprimitive[title={\prm {dimen}}]

Like \prm {count} this is a register accessor which is described in more
detail in a low level manual.

\starttyping
\dimen0=10pt
\stoptyping

While \TEX\ has some assumptions with respect to the first ten count registers
(as well as the one that holds the output, normally 255), all dimension registers
are treated equal. However, you need to be aware of clashes with other usage.
Therefore you can best use the predefined scratch registers or define dedicate
ones with the \type {\newdimen} macro.

\stopoldprimitive

\startoldprimitive[title={\prm {dimendef}}]

This primitive is used by the \type {\newdimen} macro when it relates a control
sequence with a specific register. Only use it when you know what you're doing.

\stopoldprimitive

\startnewprimitive[title={\prm {dimensiondef}}]

A variant of \prm {integerdef} is:

\starttyping
\dimensiondef\MyDimen = 1234pt
\stoptyping

The properties are comparable to the ones described in the section \prm
{integerdef}.

\stopnewprimitive

\startoldprimitive[title={\prm {dimexpr}}]

This primitive is similar to of \prm {numexpr} but operates on dimensions
instead. Integer quantities are interpreted as dimensions in scaled points.

\startbuffer
\the\dimexpr (1pt + 2pt - 5pt) * 10 / 2 \relax
\stopbuffer

\typebuffer

gives: \inlinebuffer. You can mix in symbolic integers and dimensions. This doesn't work:

\startbuffer
\the\dimexpr 10 * (1pt + 2pt - 5pt) / 2 \relax
\stopbuffer

because the engine scans for a dimension and only for an integer (or equivalent)
after a \type {*} or \type {/}.


\stopoldprimitive

\startnewprimitive[title={\prm {dimexpression}}]

This command is like \prm {numexpression} but results in a dimension instead of
an integer. Where \prm {dimexpr} doesn't like \typ {2 * 10pt} this expression
primitive is quite happy with it.

\stopnewprimitive

\startnewprimitive[title={\prm {directlua}}]

This is the low level interface to \LUA:

\startbuffer
\directlua { tex.print ("Greetings from the lua end!") }
\stopbuffer

Gives: \quotation {\inlinebuffer} as expected. In \LUA\ we have access to all
kind of internals of the engine. In \LUAMETATEX\ the interfaces have been
polished and extended compared to \LUATEX. Although many primitives and
mechanisms were added to the \TEX\ frontend, the main extension interface remains
\LUA. More information can be found in documents that come with \CONTEXT, in
presentations and in articles.

\stopnewprimitive

\startoldprimitive[title={\prm {discretionary}}]

The three snippets given with this command determine the pre, post and replace
component of the injected discretionary node. The \type {penalty} keyword permits
setting a penalty with this node. The \type {postword} keyword indicates that
this discretionary starts a word, and \type {preword} ends it. With \type {break}
the line break algorithm will prefer a pre or post component over a replace, and
with \type {nobreak} replace will win over pre. With \type {class} you can set a
math class that will determine spacing and such for discretionaries used in math
mode.

\stopoldprimitive

\startnewprimitive[title={\prm {discretionaryoptions}}]

Processing of discretionaries is controlled by this bitset:

\getbuffer[engine:syntax:discoptioncodes]

These can also be set on \prm {discretionary} using the \type {options} key.

\stopnewprimitive

\startoldprimitive[title={\prm {displayindent}}]

The \prm {displaywidth}, \prm {displayindent} and \prm {predisplaysize}
parameters are set by the line break routine (but can be adapted by the user), so
that mid|-|par display formula can adapt itself to hanging indentation and par
shapes. I order to calculate thee values and adapt the line break state
afterwards such a display formula is assumed to occupy three lines, so basically
a rather compact formula.

\stopoldprimitive

\startoldprimitive[title={\prm {displaylimits}}]

By default in math display mode limits are place on top while in inline mode they
are placed like scripts, after the operator. Placement can be forced with the
\prm {limits} and \prm {nolimits} modifiers (after the operator). Because there
can be multiple of these in a row there is \prm {displaylimits} that forces the
default placement, so effectively it acts here as a reset modifier.

\stopoldprimitive

\startoldprimitive[title={\prm {displaystyle}}]

One of the main math styles; integer representation: \the\displaystyle.

\stopoldprimitive

\startoldprimitive[title={\prm {displaywidowpenalties}}]

This is a math specific variant of \prm {widowpenalties}.

\stopoldprimitive

\startoldprimitive[title={\prm {displaywidowpenalty}}]

This is a math specific variant of \prm {widowpenalty}.

\stopoldprimitive

\startoldprimitive[title={\prm {displaywidth}}]

This parameter determines the width of the formula and normally defaults to the
\prm {hsize} unless we are in the middle of a paragraph in which case it is
compensated for hanging indentation or the par shape.

\stopoldprimitive

\startoldprimitive[title={\prm {divide}}]

The \prm {divide} operation can be applied to integers, dimensions, float,
attribute and glue quantities. There are subtle rounding differences between
the divisions in expressions and \prm {divide}:

\starttabulate
\NC \type {\scratchcounter1049     \numexpr\scratchcounter/ 10\relax}
\EQ        \scratchcounter1049 \the\numexpr\scratchcounter/ 10\relax
\NC \NR
\NC \type {\scratchcounter1049     \numexpr\scratchcounter: 10\relax}
\EQ        \scratchcounter1049 \the\numexpr\scratchcounter: 10\relax
\NC \NR
\NC \type {\scratchcounter1049 \divide\scratchcounter by 10}
\EQ        \scratchcounter1049 \divide\scratchcounter by 10
                               \the\scratchcounter
\NC \NR
\stoptabulate

The \type {:} divider in \prm {dimexpr} is something that we introduced in
\LUATEX.

\stopoldprimitive

\startnewprimitive[title={\prm {divideby}}]

This is slightly more efficient variant of \prm {divide} that doesn't look for
\type {by}. See previous section.

\stopnewprimitive

\startoldprimitive[title={\prm {doublehyphendemerits}}]

This penalty will be added to the penalty assigned to a breakpoint that results
in two lines ending with a hyphen.

\stopoldprimitive

\startnewprimitive[title={\prm {doublepenaltymode}}]

When set to one this parameter signals the backend to use the alternative (left
side) penalties of the pairs set on \prm {widowpenalties}, \prm {clubpenalties}
and \prm {brokenpenalties}. For more information on this you can consult manuals
(and articles) that come with \CONTEXT.

\stopnewprimitive

\startoldprimitive[title={\prm {dp}}]

Returns the depth of the given box.

\stopoldprimitive

\startnewprimitive[title={\prm {dpack}}]

This does what \prm {dbox} does but without callback overhead.

\stopnewprimitive

\startnewprimitive[title={\prm {dsplit}}]

This is the dual baseline variant of \prm {vsplit} (see \prm {dbox} for what that
means).

\stopnewprimitive

\startoldprimitive[title={\prm {dump}}]

This finishes an (ini) run and dumps a format (basically the current state of the
engine).

\stopoldprimitive

\startoldprimitive[title={\prm {edef}}]

This is the expanded version of \prm {def}.

\startbuffer
\def \foo{foo}      \meaning\foo
\def \ofo{\foo\foo} \meaning\ofo
\edef\oof{\foo\foo} \meaning\oof
\stopbuffer

\typebuffer

Because \type {\foo} is unprotected it will expand inside the body definition:

\startlines \tt
\getbuffer
\stoplines

\stopoldprimitive

\startnewprimitive[title={\prm {edefcsname}}]

This is the companion of \prm {edef}:

\starttyping
\expandafter\edef\csname MyMacro:1\endcsname{...}
             \edefcsname MyMacro:1\endcsname{...}
\stoptyping

\stopnewprimitive

\startnewprimitive[title={\prm {edivide}}]

When expressions were introduced the decision was made to round the divisions which is
incompatible with the way \prm {divide} works. The expression scanners in \LUAMETATEX\
compensates that by providing a \type {:} for integer division. The \prm {edivide} does
the opposite: it rounds the way expressions do.

\startbuffer
\the\dimexpr .4999pt                     : 2 \relax           =.24994pt
\the\dimexpr .4999pt                     / 2 \relax           =.24995pt
\scratchdimen.4999pt \divide \scratchdimen 2 \the\scratchdimen=.24994pt
\scratchdimen.4999pt \edivide\scratchdimen 2 \the\scratchdimen=.24995pt

\the\numexpr   1001                        : 2 \relax             =500
\the\numexpr   1001                        / 2 \relax             =501
\scratchcounter1001  \divide \scratchcounter 2 \the\scratchcounter=500
\scratchcounter1001  \edivide\scratchcounter 2 \the\scratchcounter=501
\stopbuffer

\typebuffer

Keep in mind that with dimensions we have a fractional part so we actually
rounding applies to the fraction. For that reason we also provide \prm {rdivide}.

\startlines
\getbuffer
\stoplines

\stopnewprimitive

\startnewprimitive[title={\prm {edivideby}}]

This the \type {by}|-|less variant of \prm {edivide}.

\stopnewprimitive

\startnewprimitive[title={\prm {efcode}}]

This primitive originates in \PDFTEX\ and can be used to set the expansion factor
of a glyph (characters). This primitive is obsolete because the values can be set
in the font specification that gets passed via \LUA\ to \TEX. Keep in mind that
setting font properties at the \TEX\ end is a global operation and can therefore
influence related fonts. In \LUAMETATEX\ the \prm {cf} code can be used to
specify the compression factor independent from the expansion factor. The
primitive takes three values: font, character code, and factor.

\stopnewprimitive

\startoldprimitive[title={\prm {else}}]

This traditional primitive is part of the condition testing mechanism. When a
condition matches, \TEX\ will continue till it sees an \prm {else} or \prm
{or} or \prm {orelse} (to be discussed later). It will then do a fast skipping
pass till it sees an \prm {fi}.

\stopoldprimitive

\startoldprimitive[title={\prm {emergencyextrastretch}}]

This is one of the extended parbuilder parameters. You can you it so temporary
increase the permitted stretch without knowing or messing with the normal value.

\stopoldprimitive

\startnewprimitive[title={\prm {emergencyleftskip}}]

This is one of the extended parbuilder parameters (playground). It permits going
ragged left in case of a too bad result.

\stopnewprimitive

\startnewprimitive[title={\prm {emergencyrightskip}}]

This is one of the extended parbuilder parameters (playground). It permits going
ragged right in case of a too bad result.

\stopnewprimitive

\startoldprimitive[title={\prm {emergencystretch}}]

When set the par builder will run a third pass in order to fit the set criteria.

\stopoldprimitive

\startoldprimitive[title={\prm {end}}]

This ends a \TEX\ run, unless of course this primitive is redefined.

\stopoldprimitive

\startoldprimitive[title={\prm {endcsname}}]

This primitive is used in combination with \prm {csname}, \prm {ifcsname} and
\prm {begincsname} where its end the scanning for the to be constructed control
sequence token.

\stopoldprimitive

\startoldprimitive[title={\prm {endgroup}}]

This is the companion of the \prm {begingroup} primitive that opens a group. See
\prm {beginsimplegroup} for more info.

\stopoldprimitive

\startoldprimitive[title={\prm {endinput}}]

The engine can be in different input modes: reading from file, reading from a
token list, expanding a macro, processing something that comes back from \LUA,
etc. This primitive quits reading from file:

\startbuffer
this is seen
\endinput
here we're already quit
\stopbuffer

\typebuffer

There is a catch. This is what the above gives:

\getbuffer

but how about this:

\startbuffer
this is seen
before \endinput after
here we're already quit
\stopbuffer

\typebuffer

Here we get:

\getbuffer

Because a token list is one line, the following works okay:

\starttyping
\def\quitrun{\ifsomething \endinput \fi}
\stoptyping

but in a file you'd have to do this when you quit in a conditional:

\starttyping
\ifsomething
    \expandafter \endinput
\fi
\stoptyping

While the one|-|liner works as expected:

\starttyping
\ifsomething \endinput \fi
\stoptyping

\stopoldprimitive

\startoldprimitive[title={\prm {endlinechar}}]

This is an internal integer register. When set to positive value the character
with that code point will be appended to the line. The current value is \the
\endlinechar. Here is an example:

\startbuffer
\endlinechar\hyphenasciicode
line 1
line 2
\stopbuffer

\typebuffer

\start \getbuffer \stop

If the character is active, the property is honored and the command kicks in. The
maximum value is 127 (the maximum character code a single byte \UTF\ character
can carry.)

\stopoldprimitive

\startnewprimitive[title={\prm {endlocalcontrol}}]

See \prm {beginlocalcontrol}.

\stopnewprimitive

\startnewprimitive[title={\prm {endmathgroup}}]

This primitive is the counterpart of \prm {beginmathgroup}.

\stopnewprimitive

\startnewprimitive[title={\prm {endsimplegroup}}]

This one ends a simple group, see \prm {beginsimplegroup} for an explanation
about grouping primitives.

\stopnewprimitive

\startnewprimitive[title={\prm {enforced}}]

The engine can be set up to prevent overloading of primitives and macros defined
as \prm {permanent} or \prm {immutable}. However, a macro package might want
to get around this in controlled situations, which is why we have a \prm
{enforced} prefix. This prefix in interpreted differently in so called \quote
{ini} mode when macro definitions can be dumped in the format. Internally they
get an \type {always} flag as indicator that in these places an overload is
possible.

\starttyping
\permanent\def\foo{original}

\def\oof         {\def\foo{fails}}
\def\oof{\enforced\def\foo{succeeds}}
\stoptyping

Of course this only has an effect when overload protection is enabled.

\stopnewprimitive

\startoldprimitive[title={\prm {eofinput}}]

This is a variant on \prm {input} that takes a token list as first argument. That
list is expanded when the file ends. It has companion primitives \prm
{atendoffile} (single token) and \prm {atendoffiled} (multiple tokens).

\stopoldprimitive

\startoldprimitive[title={\prm {eqno}}]

This primitive stores the (typeset) content (presumably a number) and when the
display formula is wrapped that number will end up right of the formula.

\stopoldprimitive

\startoldprimitive[title={\prm {errhelp}}]

This is additional help information to \prm {errmessage} that triggers an error
and shows a message.

\stopoldprimitive

\startoldprimitive[title={\prm {errmessage}}]

This primitive expects a token list and shows its expansion on the console
and|/|or in the log file, depending on how \TEX\ is configured. After that it
will enter the error state and either goes on or waits for input, again depending
on how \TEX\ is configured. For the record: we don't use this primitive in
\CONTEXT.

\stopoldprimitive

\startoldprimitive[title={\prm {errorcontextlines}}]

This parameter determines the number on lines shown when an error is triggered.

\stopoldprimitive

\startoldprimitive[title={\prm {errorstopmode}}]

This directive stops at every opportunity to interact. In \CONTEXT\ we overload
the actions in a callback and quit the run because we can assume that a
successful outcome is unlikely.

\stopoldprimitive

\startoldprimitive[title={\prm {escapechar}}]

This internal integer has the code point of the character that get prepended to a
control sequence when it is serialized (for instance in tracing or messages).

\stopoldprimitive

\startnewprimitive[title={\prm {etoks}}]

This assigns an expanded token list to a token register:

\starttyping
\def\temp{less stuff}
\etoks\scratchtoks{a bit \temp}
\stoptyping

The orginal value of the register is lost.

\stopnewprimitive

\startnewprimitive[title={\prm {etoksapp}}]

A variant of \prm {toksapp} is the following: it expands the to be appended
content.

\starttyping
\def\temp{more stuff}
\etoksapp\scratchtoks{some \temp}
\stoptyping

\stopnewprimitive

\startnewprimitive[title={\prm {etokspre}}]

A variant of \prm {tokspre} is the following: it expands the to be prepended
content.

\starttyping
\def\temp{less stuff}
\etokspre\scratchtoks{a bit \temp}
\stoptyping

\stopnewprimitive

\startnewprimitive[title={\prm {eufactor}}]

When we introduced the \type {es} (2.5cm) and \type {ts} (2.5mm) units as metric
variants of the \type {in} we also added the \type {eu} factor. One \type {eu}
equals one tenth of a \type {es} times the \prm {eufactor}. The \type {ts} is a
convenient offset in test files, the \type {es} a convenient ones for layouts and
image dimensions and the \type {eu} permits definitions that scale nicely without
the need for dimensions. They also were a prelude to what later became possible
with \prm {associateunit}.

\stopnewprimitive

\startnewprimitive[title={\prm {everybeforepar}}]

This token register is expanded before a paragraph is triggered. The reason for
triggering is available in \prm {lastpartrigger}.

\stopnewprimitive

\startoldprimitive[title={\prm {everycr}}]

This token list gets expanded when a row ends in an alignment. Normally it will
use \prm {noalign} as wrapper

\startbuffer
{\everycr{\noalign{H}}            \halign{#\cr test\cr test\cr}}
{\everycr{\noalign{V}} \hsize 4cm \valign{#\cr test\cr test\cr}}
\stopbuffer

\typebuffer

Watch how the \prm {cr} ending the preamble also get this treatment:

\getbuffer

\stopoldprimitive

\startoldprimitive[title={\prm {everydisplay}}]

This token list gets expanded every time we enter display mode. It is a companion
of \prm {everymath}.

\stopoldprimitive

\startoldprimitive[title={\prm {everyeof}}]

The content of this token list is injected when a file ends but it can only be
used reliably when one is really sure that no other file is loaded in the
process. So in the end it is of no real use in a more complex macro package.

\stopoldprimitive

\startoldprimitive[title={\prm {everyhbox}}]

This token list behaves similar to \prm {everyvbox} so look there for an
explanation.

\stopoldprimitive

\startoldprimitive[title={\prm {everyjob}}]

This token list register is injected at the start of a job, or more precisely,
just before the main control loop starts.

\stopoldprimitive

\startoldprimitive[title={\prm {everymath}}]

Often math needs to be set up independent from the running text and this token
list can be used to do that. There is also \prm {everydisplay}.

\stopoldprimitive

\startnewprimitive[title={\prm {everymathatom}}]

When a math atom is seen this tokenlist is expanded before content is processed
inside the atom body.

\stopnewprimitive

\startoldprimitive[title={\prm {everypar}}]

When a paragraph starts this tokenlist is expanded before content is processed.

\stopoldprimitive

\startnewprimitive[title={\prm {everytab}}]

This token list gets expanded every time we start a table cell in \prm {halign} or
\prm {valign}.

\stopnewprimitive

\startoldprimitive[title={\prm {everyvbox}}]

This token list gets expanded every time we start a vertical box. Like \prm
{everyhbox} this is not that useful unless you are certain that there are no
nested boxes that don't need this treatment. Of course you can wipe this register
in this expansion, like:

\starttyping
\everyvbox{\kern10pt\everyvbox{}}
\stoptyping

\stopoldprimitive

\startnewprimitive[title={\prm {exceptionpenalty}}]

In exceptions we can indicate a penalty by \type {[digit]} in which case a
penalty is injected set by this primitive, multiplied by the digit.

\stopnewprimitive

\startoldprimitive[title={\prm {exhyphenchar}}]

The character that is used as pre component of the related discretionary.

\stopoldprimitive

\startoldprimitive[title={\prm {exhyphenpenalty}}]

The penalty injected after \type {-} or \type {\-} unless \prm {hyphenationmode}
is set to force the dedisated penalties.

\stopoldprimitive

\startnewprimitive[title={\prm {expand}}]

Beware, this is not a prefix but a directive to ignore the protected characters of
the following macro.

\startbuffer
\protected \def \testa{\the\scratchcounter}
           \edef\testb{\testa}
           \edef\testc{\expand\testa}
\stopbuffer

\typebuffer

The meaning of the three macros is:

\startlines \getbuffer \tttf
\meaningfull\testa
\meaningfull\testb
\meaningfull\testc
\stoplines

\stopnewprimitive

\startnewprimitive[title={\prm {expandactive}}]

This a bit of an outlier and mostly there for completeness.

\startbuffer
                          \meaningasis~
\edef\foo{~}              \meaningasis\foo
\edef\foo{\expandactive~} \meaningasis\foo
\stopbuffer

\typebuffer

There seems to be no difference but the real meaning of the first \type {\foo} is
\quote {active character 126} while the second \type {\foo} \quote {protected
call ~} is.

% \showluatokens\foo

\startlines
\getbuffer
\stoplines

Of course the definition of the active tilde is \CONTEXT\ specific and situation
dependent.

\stopnewprimitive

\startoldprimitive[title={\prm {expandafter}}]

This original \TEX\ primitive stores the next token, does a one level expansion
of what follows it, which actually can be an not expandable token, and
reinjects the stored token in the input. Like:

\starttyping
\expandafter\let\csname my weird macro name\endcsname{m w m n}
\stoptyping

Without \prm {expandafter} the \prm {csname} primitive would have been let to
the left brace (effectively then a begin group). Actually in this particular case
the control sequence with the weird name is injected and when it didn't yet exist
it will get the meaning \prm {relax} so we sort of have two assignments in a
row then.

\stopoldprimitive

\startnewprimitive[title={\prm {expandafterpars}}]

Here is another gobbler: the next token is reinjected after following spaces
and par tokens have been read. So:

\startbuffer
[\expandafterpars 1 2]
[\expandafterpars 3
4]
[\expandafterpars 5

6]
\stopbuffer

\typebuffer

gives us: \inlinebuffer, because empty lines are like \prm {par} and therefore
ignored.

\stopnewprimitive

\startnewprimitive[title={\prm {expandafterspaces}}]

This is a gobbler: the next token is reinjected after following spaces have been
read. Here is a simple example:

\startbuffer
[\expandafterspaces 1 2]
[\expandafterspaces 3
4]
[\expandafterspaces 5

6]
\stopbuffer

\typebuffer

We get this typeset: \inlinebuffer, because a newline normally is configured to be
a space (and leading spaces in a line are normally being ingored anyway).

\stopnewprimitive

\startnewprimitive[title={\prm {expandcstoken}}]

The rationale behind this primitive is that when we \prm {let} a single token
like a character it is hard to compare that with something similar, stored in a
macro. This primitive pushes back a single token alias created by \prm {let}
into the input.

\startbuffer
\let\tempA + \meaning\tempA

\let\tempB X \meaning\tempB \crlf
\let\tempC $ \meaning\tempC \par

\edef\temp         {\tempA} \doifelse{\temp}{+}{Y}{N} \meaning\temp \crlf
\edef\temp         {\tempB} \doifelse{\temp}{X}{Y}{N} \meaning\temp \crlf
\edef\temp         {\tempC} \doifelse{\temp}{X}{Y}{N} \meaning\temp \par

\edef\temp{\expandcstoken\tempA} \doifelse{\temp}{+}{Y}{N} \meaning\temp \crlf
\edef\temp{\expandcstoken\tempB} \doifelse{\temp}{X}{Y}{N} \meaning\temp \crlf
\edef\temp{\expandcstoken\tempC} \doifelse{\temp}{$}{Y}{N} \meaning\temp \par

\doifelse{\expandcstoken\tempA}{+}{Y}{N}
\doifelse{\expandcstoken\tempB}{X}{Y}{N}
\doifelse{\expandcstoken\tempC}{$}{Y}{N} \par
\stopbuffer

\typebuffer

The meaning of the \prm {let} macros shows that we have a shortcut to a
character with (in this case) catcode letter, other (here \quote {other
character} gets abbreviated to \quote {character}), math shift etc.

\start \tttf \getbuffer \stop

Here we use the \CONTEXT\ macro \type {\doifelse} which can be implemented in
different ways, but the only property relevant to the user is that the expanded
content of the two arguments is compared.

\stopnewprimitive

\startnewprimitive[title={\prm {expanded}}]

This primitive complements the two expansion related primitives mentioned in the
previous two sections. This time the content will be expanded and then pushed
back into the input. Protected macros will not be expanded, so you can use this
primitive to expand the arguments in a call. In \CONTEXT\ you need to use \type
{\normalexpanded} because we already had a macro with that name. We give some
examples:

\startbuffer
\def\A{!}
          \def\B#1{\string#1}                          \B{\A}
          \def\B#1{\string#1} \normalexpanded{\noexpand\B{\A}}
\protected\def\B#1{\string#1}                          \B{\A}
\stopbuffer

\typebuffer \startlines\getbuffer\stoplines

\stopnewprimitive

\startnewprimitive[title={\prm {expandedafter}}]

The following two lines are equivalent:

\startbuffer
\def\foo{123}
\expandafter[\expandafter[\expandafter\secondofthreearguments\foo]]
\expandedafter{[[\secondofthreearguments}\foo]]
\stopbuffer

\typebuffer

In \CONTEXT\ \MKIV\ the number of times that one has multiple \prm {expandafter}s
is much larger than in \CONTEXT\ \LMTX\ thanks to some of the new features in
\LUAMETATEX, and this primitive is not really used yet in the core code.

\startlines\getbuffer\stoplines

\stopnewprimitive

\startnewprimitive[title={\prm {expandeddetokenize}}]

This is a companion to \prm {detokenize} that expands its argument:

\startbuffer
\def\foo{12#H3}
\def\oof{\foo}
\detokenize        {\foo} \detokenize        {\oof}
\expandeddetokenize{\foo} \expandeddetokenize{\oof}
\edef\ofo{\expandeddetokenize{\foo}} \meaningless\ofo
\edef\ofo{\expandeddetokenize{\oof}} \meaningless\ofo
\stopbuffer

\typebuffer

This is a bit more convenient than

\starttyping
\detokenize \expandafter {\normalexpanded {\foo}}
\stoptyping

kind of solutions. We get:

\startlines
\getbuffer
\stoplines

\stopnewprimitive

\startnewprimitive[title={\prm {expandedendless}}]

This one loops forever but because the loop counter is not set you need to find a
way to quit it.

\stopnewprimitive

\startnewprimitive[title={\prm {expandedloop}}]

This variant of the previously introduced \prm {localcontrolledloop} doesn't
enter a local branch but immediately does its work. This means that it can be
used inside an expansion context like \prm {edef}.

\startbuffer
\edef\whatever
  {\expandedloop 1 10 1
     {\scratchcounter=\the\currentloopiterator\relax}}

\meaningasis\whatever
\stopbuffer

\typebuffer

\start \veryraggedright \tt\tfx \getbuffer \stop \blank

\stopnewprimitive

\startnewprimitive[title={\prm {expandedrepeat}}]

This one takes one instead of three arguments which is sometimes more
convenient.

\stopnewprimitive

\startnewprimitive[title={\prm {expandparameter}}]

This primitive is a predecessor of \prm {parameterdef} so we stick to a simple
example.

\startbuffer
\def\foo#1#2%
  {\integerdef\MyIndexOne\parameterindex\plusone % 1
   \integerdef\MyIndexTwo\parameterindex\plustwo % 2
   \oof{P}\oof{Q}\oof{R}\norelax}

\def\oof#1%
  {<1:\expandparameter\MyIndexOne><1:\expandparameter\MyIndexOne>%
   #1%
   <2:\expandparameter\MyIndexTwo><2:\expandparameter\MyIndexTwo>}

\foo{A}{B}
\stopbuffer

\typebuffer

In principle the whole parameter stack can be accessed but often
one never knows if a specific macro is called nested. The original
idea behind this primitive was tracing but it can also be used to
avoid passing parameters along a chain of calls.

\getbuffer

\stopnewprimitive

\startnewprimitive[title={\prm {expandtoken}}]

This primitive creates a token with a specific combination of catcode and
character code. Because it assumes some knowledge of \TEX\ we can show it
using some \prm {expandafter} magic:

\startbuffer
\expandafter\let\expandafter\temp\expandtoken 11 `X \meaning\temp
\expandafter\let\expandafter\temp\expandtoken 12 `X \meaning\temp
\stopbuffer

\typebuffer

The meanings are:

\startlines \tttf \getbuffer \stoplines

Using other catcodes is possible but the results of injecting them into the input
directly (or here by injecting \type {\temp}) can be unexpected because of what
\TEX\ expects. You can get messages you normally won't get, for instance about
unexpected alignment interference, which is a side effect of \TEX\ using some
catcode|/|character combinations as signals and there is no reason to change
those internals. That said:

\startbuffer
\xdef\tempA{\expandtoken  9 `X} \meaning\tempA
\xdef\tempB{\expandtoken 10 `X} \meaning\tempB
\xdef\tempC{\expandtoken 11 `X} \meaning\tempC
\xdef\tempD{\expandtoken 12 `X} \meaning\tempD
\stopbuffer

\typebuffer

are all valid and from the meaning you cannot really deduce what's in there:

\startlines \tttf \getbuffer \stoplines % grouped therefore xdef

But you can be assured that:

\startbuffer
[AB: \ifx\tempA\tempB Y\else N\fi]
[AC: \ifx\tempA\tempC Y\else N\fi]
[AD: \ifx\tempA\tempD Y\else N\fi]
[BC: \ifx\tempB\tempC Y\else N\fi]
[BD: \ifx\tempB\tempD Y\else N\fi]
[CD: \ifx\tempC\tempD Y\else N\fi]
\stopbuffer

\typebuffer

makes clear that they're different: \inlinebuffer, and in case you wonder, the
characters with catcode 10 are spaces, while those with code 9 are ignored.

\stopnewprimitive

\startnewprimitive[title={\prm {expandtoks}}]

This is a more efficient equivalent of \prm {the} applied to a token register,
so:

\startbuffer
\scratchtoks{just some tokens}
\edef\TestA{[\the       \scratchtoks]}
\edef\TestB{[\expandtoks\scratchtoks]}
[\the       \scratchtoks] [\TestA] \meaning\TestA
[\expandtoks\scratchtoks] [\TestB] \meaning\TestB
\stopbuffer

\typebuffer

does the expected:

\startlines
\getbuffer
\stoplines

The \prm {expandtoken} primitive avoid a copy into the input when there is no
need for it.

\stopnewprimitive

\startnewprimitive[title={\prm {explicitdiscretionary}}]

This is the verbose alias for one of \TEX's single character control sequences:
\type {\-}.

\stopnewprimitive

\startnewprimitive[title={\prm {explicithyphenpenalty}}]

The penalty injected after an automatic discretionary \type {\-}, when \prm
{hyphenationmode} enables this.

\stopnewprimitive

\startnewprimitive[title={\prm {explicititaliccorrection}}]

This is the verbose alias for one of \TEX's single character control sequences:
\type {\/}. Italic correction is a character property specific to \TEX\ and the
concept is not present in modern font technologies. There is a callback that
hooks into this command so that a macro package can provide its own solution
to this (or alternatively it can assign values to the italic correction field.

\stopnewprimitive

\startnewprimitive[title={\prm {explicitspace}}]

This is the verbose alias for one of \TEX's single character control sequences:
\type {\ }. A space is inserted with properties according the space related
variables. There is look|-|back involved in order to deal with space factors.

When \prm {nospaces} is set to~1 no spaces are inserted, when its value is~2 a
zero space is inserted.

\stopnewprimitive

\startoldprimitive[title={\prm {fam}}]

In a numeric context it returns the current family number, otherwise it sets the
given family. The number of families in a traditional engine is 16, in \LUATEX\
it is 256 and in \LUAMETATEX\ we have at most \cldcontext {tex . magicconstants .
max_n_of_math_families} families. A future version can lower that number when we
need more classes.

\stopoldprimitive

\startoldprimitive[title={\prm {fi}}]

This traditional primitive is part of the condition testing mechanism and ends a
test. So, we have:

\starttyping
\ifsomething ... \else ... \fi
\ifsomething ... \or ... \or ... \else ... \fi
\ifsomething ... \orelse \ifsometing  ... \else ... \fi
\ifsomething ... \or ... \orelse \ifsometing  ... \else ... \fi
\stoptyping

The \prm {orelse} is new in \LUAMETATEX\ and a continuation like we find in
other programming languages (see later section).

\stopoldprimitive

\startoldprimitive[title={\prm {finalhyphendemerits}}]

This penalty will be added to the penalty assigned to a breakpoint when that
break results in a pre|-|last line ending with a hyphen.

\stopoldprimitive

\startoldprimitive[title={\prm {firstmark}}][obsolete=yes]

This is a reference to the first mark on the (split off) page, it gives back
tokens.

\stopoldprimitive

\startoldprimitive[title={\prm {firstmarks}}]

This is a reference to the first mark with the given id (a number) on the (split
off) page, it gives back tokens.

\stopoldprimitive

\startnewprimitive[title={\prm {firstvalidlanguage}}]

Language id's start at zero, which makes it the first valid language. You can set
this parameter to indicate the first language id that is actually a language. The
current value is \the \firstvalidlanguage, so lower values will not trigger
hyphenation.

\stopnewprimitive

\startnewprimitive[title={\prm {fitnessdemerits}}]

We can have more fitness classes than traditional \TEX\ that has \quote {very
loose}, \quote {loose}, \quote {decent} and \quote {tight}. In \CONTEXT\ we have
\quote {veryloose}, \quote {loose}, \quote {almostloose}, \quote {barelyloose},
\quote {decent}, \quote {barelytight}, \quote {almosttight}, \quote {tight} and
\quote {verytight}. Although we can go up to 31 this is already more than enough.
The default is the same as in regular \TEX.

The \prm {fitnessdemerits} can be used to set the criteria and like other
specification primitives (like \prm {parshape} and \prm {widowpenalties}, it
expects a count. The criteria come in pairs because we can go up or down in the
chain (getting better or worse). The criterium used when we go from one to
another is the sum of the given values. The rationale behind this approach is
explained in articles, presentations and manuals.

\stopnewprimitive

\startnewprimitive[title={\prm {float}}]

In addition to integers and dimensions, which are fixed 16.16 integer floats we
also have \quote {native} floats, based on 32 bit posit unums.

\startbuffer
\float0 = 123.456           \the\float0
\float2 = 123.456           \the\float0
\advance \float0 by 123.456 \the\float0
\advance \float0 by \float2 \the\float0
\divideby\float0 3          \the\float0
\stopbuffer

\typebuffer

They come with the same kind of support as the other numeric data types:

\startlines \getbuffer \stoplines

\startbuffer
\dimen00 = 123.456pt          \the\dimen0
\dimen02 = 123.456pt          \the\dimen0
\advance \dimen0 by 123.456pt \the\dimen0
\advance \dimen0 by \dimen2   \the\dimen0
\divideby\dimen0 3            \the\dimen0
\stopbuffer

We leave the subtle differences between floats and dimensions to the user to
investigate:

\typebuffer

The nature of posits is that they are more accurate around zero (or smaller
numbers in general).

\startlines \getbuffer \stoplines

This also works:

\startbuffer
\float0=123.456e4
\float2=123.456    \multiply\float2 by 10000
\the\float0
\the\float2
\stopbuffer

\typebuffer

The values are (as expected) the same:

\startlines \getbuffer \stoplines

\stopnewprimitive

\startnewprimitive[title={\prm {floatdef}}]

This primitive defines a symbolic (macro) alias to a float register, just like
\prm {countdef} and friends do.

\stopnewprimitive

\startnewprimitive[title={\prm {floatexpr}}]

This is the companion of \prm {numexpr}, \prm {dimexpr} etc.

\startbuffer
\scratchcounter 200
\the   \floatexpr 123.456/456.123     \relax
\the   \floatexpr 1.2*\scratchcounter \relax
\the   \floatexpr \scratchcounter/3   \relax
\number\floatexpr \scratchcounter/3   \relax
\stopbuffer

\typebuffer

Watch the difference between \prm {the} and \prm {number}:

\startlines \getbuffer \stoplines

\stopnewprimitive

\startoldprimitive[title={\prm {floatingpenalty}}]

When an insertion is split (across pages) this one is added to to accumulated
\prm {insertpenalties}. In \LUAMETATEX\ this penalty can be stored per insertion
class.

\stopoldprimitive

\startnewprimitive[title={\prm {flushmarks}}]

This primitive is an addition to the multiple marks mechanism that originates in
\ETEX\ and inserts a reset signal for the mark given category that will perform a
clear operation (like \prm {clearmarks} which operates immediately).

\stopnewprimitive

\startoldprimitive[title={\prm {font}}]

This primitive is either a symbolic reference to the current font or in the
perspective of an assignment is used to trigger a font definitions with a given
name (\type {cs}) and specification. In \LUAMETATEX\ the assignment will trigger
a callback that then handles the definition; in addition to the filename an
optional size specifier is checked (\type {at} or \type {scaled}).

In \LUAMETATEX\ {\em all} font loading is delegated to \LUA, and there is no
loading code built in the engine. Also, instead of \prm {font} in \CONTEXT\ one
uses dedicated and more advanced font definition commands.

\stopoldprimitive

\startnewprimitive[title={\prm {fontcharba}}]

Fetches the bottom anchor of a character in the given font, so:

\startbuffer
\the\fontcharba\textfont0 "222B
\stopbuffer

results in: \inlinebuffer. However, this anchor is only available when it is set
and it is not part of \OPENTYPE; it is something that \CONTEXT\ provides for math
fonts.

\stopnewprimitive

\startoldprimitive[title={\prm {fontchardp}}]

Fetches the depth of a character in the given font, so:

\startbuffer
\the\fontchardp\font`g
\stopbuffer

results in: \inlinebuffer.

\stopoldprimitive

\startoldprimitive[title={\prm {fontcharht}}]

Fetches the width of a character in the given font, so:

\startbuffer
\the\fontcharht\font`g
\stopbuffer

results in: \inlinebuffer.

\stopoldprimitive

\startoldprimitive[title={\prm {fontcharic}}]

Fetches the italic correction of a character in the given font, but because it is
not an \OPENTYPE\ property it is unlikely to return something useful. Although
math fonts have such a property in \CONTEXT\ we deal with it differently.

\stopoldprimitive

\startnewprimitive[title={\prm {fontcharta}}]

Fetches the top anchor of a character in the given font, so:

\startbuffer
\the\fontcharba\textfont0 "222B
\stopbuffer

results in: \inlinebuffer. This is a specific property of math characters because
in text mark anchoring is driven by a feature.

\stopnewprimitive

\startoldprimitive[title={\prm {fontcharwd}}]

Fetches the width of a character in the given font, so:

\startbuffer
\the\fontcharwd\font`g
\stopbuffer

results in: \inlinebuffer.

\stopoldprimitive

\startoldprimitive[title={\prm {fontdimen}}][obsolete=yes]

A traditional \TEX\ font has a couple of font specific dimensions, we only
mention the seven that come with text fonts:

\startitemize[n,packed]
\startitem
    The slant (slope) is an indication that we have an italic shape. The value
    divided by 65.536 is a fraction that can be compared with for instance the
    \type {slanted} operator in \METAPOST. It is used for positioning accents, so
    actually not limited to oblique fonts (just like italic correction can be a
    property of any character). It is not relevant in the perspective of
    \OPENTYPE\ fonts where we have glyph specific top and bottom anchors.
\stopitem
\startitem
    Unless is it overloaded by \prm {spaceskip} this determines the space between
    words (or actually anything separated by a space).
\stopitem
\startitem
    This is the stretch component of \prm {fontdimen}~2(space).
\startitem
    This is the shrink component of \prm {fontdimen}~2(space).
\stopitem
\startitem
    The so called ex|-|height is normally the height of the \quote {x} and is
    also accessible as \type {em} unit.
\stopitem
\startitem
    The so called em|-|width or in \TEX\ speak quad width is about the with of an
    \quote {M} but in many fonts just matches the font size. It is also
    accessible as \type {em} unit.
\stopitem
\startitem
    This is a very \TEX\ specific property also known as extra space. It gets
    {\em added} to the regular space after punctuation when \prm {spacefactor} is
    2000 or more. It can be overloaded by \prm {xspaceskip}.
\stopitem
\stopitemize

This primitive expects a a number and a font identifier. Setting a font dimension
is a global operation as it directly pushes the value in the font resource.

\stopoldprimitive

\startnewprimitive[title={\prm {fontid}}]

Returns the (internal) number associated with the given font:

\startbuffer
{\bf \xdef\MyFontA{\the\fontid\font}}
{\sl \xdef\MyFontB{\setfontid\the\fontid\font}}
\stopbuffer

\typebuffer \getbuffer

with:

\startbuffer
test {\setfontid\MyFontA test} test {\MyFontB test} test
\stopbuffer

\typebuffer

gives: \inlinebuffer.

\stopnewprimitive

\startnewprimitive[title={\prm {fontmathcontrol}}]

The \prm {mathfontcontrol} parameter controls how the engine deals with specific
font related properties and possibilities. It is set at the \TEX\ end. It makes
it possible to fine tune behavior in this mixed traditional and not perfect
\OPENTYPE\ math font arena. One can also set this bitset when initializing
(loading) the font (at the \LUA\ end) and the value set there is available in
\prm {fontmathcontrol}. The bits set in the font win over those in \prm
{mathfontcontrol}. There are a few cases where we set these options in the (so
called) goodie files. For instance we ignore font kerns in Libertinus, Antykwa
and some more.

\starttabulate[|l|rT|]
\NC modern     \NC \bgroup\switchtobodyfont    [modern]\normalexpanded{\egroup 0x\tohexadecimal\fontmathcontrol\textfont\fam} \NC \NR
\NC pagella    \NC \bgroup\switchtobodyfont   [pagella]\normalexpanded{\egroup 0x\tohexadecimal\fontmathcontrol\textfont\fam} \NC \NR
\NC antykwa    \NC \bgroup\switchtobodyfont   [antykwa]\normalexpanded{\egroup 0x\tohexadecimal\fontmathcontrol\textfont\fam} \NC \NR
\NC libertinus \NC \bgroup\switchtobodyfont[libertinus]\normalexpanded{\egroup 0x\tohexadecimal\fontmathcontrol\textfont\fam} \NC \NR
\stoptabulate

\stopnewprimitive

\startoldprimitive[title={\prm {fontname}}]

Depending on how the font subsystem is implemented this gives some information
about the used font:

\startbuffer
{\tf \fontname\font}
{\bf \fontname\font}
{\sl \fontname\font}
\stopbuffer

\typebuffer

\startlines
\getbuffer
\stoplines

\stopoldprimitive

\startnewprimitive[title={\prm {fontspecdef}}]

This primitive creates a reference to a specification that when triggered will
change multiple parameters in one go.

\starttyping
\fontspecdef\MyFontSpec
    \fontid\font
    scale  1200
    xscale 1100
    yscale  800
    weight  200
    slant   500
\relax
\stoptyping

is equivalent to:

\starttyping
\fontspecdef\MyFontSpec
    \fontid\font
    all 1200 1100 800 200 500
\relax
\stoptyping

while

\starttyping
\fontspecdef\MyFontSpec
    \fontid\font
    all \glyphscale \glyphxscale \glyphyscale \glyphslant \glyphweight
\relax
\stoptyping

is the same as

\starttyping
\fontspecdef\MyFontSpec
    \fontid\font
\relax
\stoptyping

The engine adapts itself to these glyph parameters but when you access certain
quantities you have to make sure that you use the scaled ones. The same is true
at the \LUA\ end. This is somewhat fundamental in the sense that when one uses
these sort of dynamic features one also need to keep an eye on code that uses
font specific dimensions.

\stopnewprimitive

\startnewprimitive[title={\prm {fontspecid}}]

Internally a font reference is a number and this primitive returns the number of
the font bound to the specification.

\stopnewprimitive

\startnewprimitive[title={\prm {fontspecifiedname}}]

Depending on how the font subsystem is implemented this gives some information
about the (original) definition of the used font:

\startbuffer
{\tf \fontspecifiedname\font}
{\bf \fontspecifiedname\font}
{\sl \fontspecifiedname\font}
\stopbuffer

\typebuffer

\startlines
\getbuffer
\stoplines

\stopnewprimitive

\startnewprimitive[title={\prm {fontspecifiedsize}}]

Depending on how the font subsystem is implemented this gives some information
about the (original) size of the used font:

\startbuffer
{\tf  \the\fontspecifiedsize\font : \the\glyphscale}
{\bfa \the\fontspecifiedsize\font : \the\glyphscale}
{\slx \the\fontspecifiedsize\font : \the\glyphscale}
\stopbuffer

\typebuffer

Depending on how the font system is setup, this is not the real value that is
used in the text because we can use for instance \prm {glyphscale}. So the next
lines depend on what font mode this document is typeset.

\startlines
\getbuffer
\stoplines

\stopnewprimitive

\startnewprimitive[title={\prm {fontspecscale}}]

This returns the scale factor of a fontspec where as usual 1000 means scaling
by~1.

\stopnewprimitive

\startnewprimitive[title={\prm {fontspecslant}}]

This returns the slant factor of a font specification, usually between zero and
1000 with 1000 being maximum slant.

\stopnewprimitive

\startnewprimitive[title={\prm {fontspecweight}}]

This returns the weight of the font specification. Reasonable values are between
zero and 500.

\stopnewprimitive

\startnewprimitive[title={\prm {fontspecxscale}}]

This returns the scale factor of a font specification where as usual 1000 means
scaling by~1.

\stopnewprimitive

\startnewprimitive[title={\prm {fontspecyscale}}]

This returns the scale factor of a font specification where as usual 1000 means
scaling by~1.

\stopnewprimitive

\startnewprimitive[title={\prm {fonttextcontrol}}]

This returns the text control flags that are set on the given font, here {\tttf
0x\tohexadecimal \fonttextcontrol \font}. Bits that can be set are:

\getbuffer[engine:syntax:textcontrolcodes]

\stopnewprimitive

\startnewprimitive[title={\prm {forcedleftcorrection}}]

This is a callback driven left correction signal similar to italic corrections.

\stopnewprimitive

\startnewprimitive[title={\prm {forcedrightcorrection}}]

This is a callback driven right correction signal similar to italic corrections.

\stopnewprimitive

\startnewprimitive[title={\prm {formatname}}]

It is in the name: {\tttf \filenameonly {\formatname}}, but we cheat here by only
showing the filename and not the full path, which in a \CONTEXT\ setup can span
more than a line in this paragraph.

\stopnewprimitive

\startnewprimitive[title={\prm {frozen}}]

You can define a macro as being frozen:

\starttyping
\frozen\def\MyMacro{...}
\stoptyping

When you redefine this macro you get an error:

\starttyping
! You can't redefine a frozen macro.
\stoptyping

This is a prefix like \prm {global} and it can be combined with other prefixes.
\footnote {The \prm {outer} and \prm {long} prefixes are no|-|ops in
\LUAMETATEX\ and \LUATEX\ can be configured to ignore them.}

\stopnewprimitive

\startnewprimitive[title={\prm {futurecsname}}]

In order to make the repertoire of \type {def}, \type {let} and \type {futurelet}
primitives complete we also have:

\starttyping
\futurecsname MyMacro:1\endcsname\MyAction
\stoptyping

\stopnewprimitive

\startnewprimitive[title={\prm {futuredef}}]

We elaborate on the example of using \prm {futurelet} in the previous section.
Compare that one with the next:

\startbuffer
\def\MySpecialToken{[}
\def\DoWhatever{\ifx\NextToken\MySpecialToken YES\else NOP\fi : }
\futurelet\NextToken\DoWhatever [A]\crlf
\futurelet\NextToken\DoWhatever (A)\par
\stopbuffer

\typebuffer

This time we get:

{\getbuffer}

It is for that reason that we now also have \prm {futuredef}:

\startbuffer
\def\MySpecialToken{[}
\def\DoWhatever{\ifx\NextToken\MySpecialToken YES\else NOP\fi : }
\futuredef\NextToken\DoWhatever [A]\crlf
\futuredef\NextToken\DoWhatever (A)\par
\stopbuffer

\typebuffer

So we're back to what we want:

{\getbuffer}

\stopnewprimitive

\startnewprimitive[title={\prm {futureexpand}}]

This primitive can be used as an alternative to a \prm {futurelet} approach,
which is where the name comes from. \footnote {In the engine primitives
that have similar behavior are grouped in commands that are then dealt with
together, code wise.}

\startbuffer
\def\variantone<#1>{(#1)}
\def\varianttwo#1{[#1]}
\futureexpand<\variantone\varianttwo<one>
\futureexpand<\variantone\varianttwo{two}
\stopbuffer

\typebuffer

So, the next token determines which of the two variants is taken:

{\getbuffer}

Because we look ahead there is some magic involved: spaces are ignored but when
we have no match they are pushed back into the input. The next variant
demonstrates this:

\startbuffer
\def\variantone<#1>{(#1)}
\def\varianttwo{}
\def\temp{\futureexpand<\variantone\varianttwo}
[\temp <one>]
[\temp {two}]
[\expandafter\temp\space <one>]
[\expandafter\temp\space {two}]
\stopbuffer

\typebuffer

This gives us:

{\getbuffer}

\stopnewprimitive

\startnewprimitive[title={\prm {futureexpandis}}]

We assume that the previous section is read. This variant will not push back
spaces, which permits a consistent approach i.e.\ the user can assume that macro
always gobbles the spaces.

\startbuffer
\def\variantone<#1>{(#1)}
\def\varianttwo{}
\def\temp{\futureexpandis<\variantone\varianttwo}
[\temp <one>]
[\temp {two}]
[\expandafter\temp\space <one>]
[\expandafter\temp\space {two}]
\stopbuffer

\typebuffer

So, here no spaces are pushed back. This \type {is} in the name of this primitive
means \quote {ignore spaces}, but having that added to the name would have made
the primitive even more verbose (after all, we also don't have \type
{\expandeddef} but \prm {edef} and no \type {\globalexpandeddef} but \prm
{xdef}.

{\getbuffer}

\stopnewprimitive

\startnewprimitive[title={\prm {futureexpandisap}}]

This primitive is like the one in the previous section but also ignores par
tokens, so \type {isap} means \quote {ignore spaces and paragraphs}.

\stopnewprimitive

\startoldprimitive[title={\prm {futurelet}}]

The original \TEX\ primitive \prm {futurelet} can be used to create an alias to a next token,
push it back into the input and then expand a given token.

\startbuffer
\let\MySpecialTokenL[
\let\MySpecialTokenR] % nicer for checker
\def\DoWhatever{\ifx\NextToken\MySpecialTokenL YES\else NOP\fi : }
\futurelet\NextToken\DoWhatever [A]\crlf
\futurelet\NextToken\DoWhatever (A)\par
\stopbuffer

\typebuffer

This is typically the kind of primitive that most users will never use because it
expects a sane follow up handler (here \type {\DoWhatever}) and therefore is
related to user interfacing.

{\getbuffer}

\stopoldprimitive

\startoldprimitive[title={\prm {gdef}}]

The is the global companion of \prm {def}.

\stopoldprimitive

\startnewprimitive[title={\prm {gdefcsname}}]

As with standard \TEX\ we also define global ones:

\starttyping
\expandafter\gdef\csname MyMacro:1\endcsname{...}
             \gdefcsname MyMacro:1\endcsname{...}
\stoptyping

\stopnewprimitive

\startnewprimitive[title={\prm {givenmathstyle}}]

This primitive expects a math style and returns it when valid or otherwise issues
an error.

\stopnewprimitive

\startnewprimitive[title={\prm {gleaders}}]

Leaders are glue with special property: a box, rule of (in \LUAMETATEX) glyph, like:

\startlines
x\leaders   \glyph `M \hfill  x
xx\leaders  \glyph `M \hfill xx

x\cleaders  \glyph `M \hfill x
xx\cleaders \glyph `M \hfill xx

x\xleaders  \glyph `M \hfill x
xx\xleaders \glyph `M \hfill xx

x\gleaders  \glyph `M \hfill x
xx\gleaders \glyph `M \hfill xx
\stoplines

Leaders fill the available space. The \prm {leaders} command starts at the left
edge and stops when there is no more space. The blobs get centered when we use
\prm {cleaders}: excess space is distributed before and after a blob while \prm
{xleaders} also puts space between the blobs.

When a rule is given the advance (width or height and depth) is ignored, so these
are equivalent.

\starttyping
x\leaders \hrule           \hfill x
x\leaders \hrule width 1cm \hfill x
\stoptyping

When a box is used one will normally have some alignment in that box.

\starttyping
x\leaders \hbox {\hss.\hss} \hfill            x
x\leaders \hbox {\hss.\hss} \hskip 6cm \relax x
\stoptyping

The reference point is the left edge of the current (outer) box and the effective
glue (when it has stretch or shrink) depends on that box. The \prm {gleaders}
variant takes the page as reference. That makes it possible to \quote {align}
across boxes.

\stopnewprimitive

\startnewprimitive[title={\prm {glet}}]

This is the global companion of \prm {let}. The fact that it is not an original
primitive is probably due to the expectation for it not it not being used (as)
often (as in \CONTEXT).

\stopnewprimitive

\startnewprimitive[title={\prm {gletcsname}}]

Naturally \LUAMETATEX\ also provides a global variant:

\starttyping
\expandafter\global\expandafter\let\csname MyMacro:1\endcsname\relax
\expandafter                  \glet\csname MyMacro:1\endcsname\relax
                               \gletcsname MyMacro:1\endcsname\relax
\stoptyping

So, here we save even more.

\stopnewprimitive

\startnewprimitive[title={\prm {glettonothing}}]

This is the global companion of \prm {lettonothing}.

\stopnewprimitive

\startoldprimitive[title={\prm {global}}]

This is one of the original prefixes that can be used when we define a macro of
change some register.

\starttyping
\bgroup
       \def\MyMacroA{a}
\global\def\MyMacroB{a}
      \gdef\MyMacroC{a}
\egroup
\stoptyping

The macro defined in the first line is forgotten when the groups is left. The
second and third definition are both global and these definitions are retained.

\stopoldprimitive

\startoldprimitive[title={\prm {globaldefs}}]

When set to a positive value, this internal integer will force all definitions to
be global, and in a complex macro package that is not something a user will do
unless it is very controlled.

\stopoldprimitive

\startoldprimitive[title={\prm {glueexpr}}]

This is a more extensive variant of \prm {dimexpr} that also handles the optional
stretch and shrink components.

\stopoldprimitive

\startoldprimitive[title={\prm {glueshrink}}]

This returns the shrink component of a glue quantity. The result is a dimension
so you need to apply \prm {the} when applicable.

\stopoldprimitive

\startoldprimitive[title={\prm {glueshrinkorder}}]

This returns the shrink order of a glue quantity. The result is a integer so you
need to apply \prm {the} when applicable.

\stopoldprimitive

\startnewprimitive[title={\prm {gluespecdef}}]

A variant of \prm {integerdef} and \prm {dimensiondef} is:

\starttyping
\gluespecdef\MyGlue = 3pt plus 2pt minus 1pt
\stoptyping

The properties are comparable to the ones described in the previous sections.

\stopnewprimitive

\startoldprimitive[title={\prm {gluestretch}}]

This returns the stretch component of a glue quantity. The result is a dimension
so you need to apply \prm {the} when applicable.

\stopoldprimitive

\startoldprimitive[title={\prm {gluestretchorder}}]

This returns the stretch order of a glue quantity. The result is a integer so you
need to apply \prm {the} when applicable.

\stopoldprimitive

\startoldprimitive[title={\prm {gluetomu}}]

The sequence \typ {\the \gluetomu 20pt plus 10pt minus 5pt} gives \the \gluetomu
20pt plus 10pt minus 5pt.

\stopoldprimitive

\startnewprimitive[title={\prm {glyph}}]

This is a more extensive variant of \prm {char} that permits setting some
properties if the injected character node.

\startbuffer
\ruledhbox{\glyph
    scale 2000 xscale 9000 yscale 1200
    slant 700 weight 200
    xoffset 10pt yoffset -5pt left 10pt right 20pt
    123}
\quad
\ruledhbox{\glyph
    scale 2000 xscale 9000 yscale 1200
    slant 700 weight 200
    125}
\stopbuffer

\typebuffer

In addition one can specify \type {font} (symbol), \type {id} (valid font id
number), an \type {options} (bit set) and \type {raise}.

\startlinecorrection
\dontleavehmode\getbuffer
\stoplinecorrection

When no parameters are set, the current ones are used. More details and examples
of usage can be found in the \CONTEXT\ distribution.

\stopnewprimitive

\startnewprimitive[title={\prm {glyphdatafield}}]

The value of this parameter is assigned to data field in glyph nodes that get
injected. It has no meaning in itself but can be used at the \LUA\ end.

\stopnewprimitive

\startnewprimitive[title={\prm {glyphoptions}}]

The value of this parameter is assigned to the options field in glyph nodes that
get injected.

\startcolumns
\getbuffer[engine:syntax:glyphoptions]
\stopcolumns

\stopnewprimitive

\startnewprimitive[title={\prm {glyphscale}}]

An integer parameter defining the current glyph scale, assigned to glyphs
(characters) inserted into the current list.

\stopnewprimitive

\startnewprimitive[title={\prm {glyphscriptfield}}]

The value of this parameter is assigned to script field in glyph nodes that get
injected. It has no meaning in itself but can be used at the \LUA\ end.

\stopnewprimitive

\startnewprimitive[title={\prm {glyphscriptscale}}]

This multiplier is applied to text font and glyph dimension properties when script
style is used.

\stopnewprimitive

\startnewprimitive[title={\prm {glyphscriptscriptscale}}]

This multiplier is applied to text font and glyph dimension properties when
script script style is used.

\stopnewprimitive

\startnewprimitive[title={\prm {glyphslant}}]

An integer parameter defining the current glyph slant, assigned to glyphs
(characters) inserted into the current list.

\stopnewprimitive

\startnewprimitive[title={\prm {glyphstatefield}}]

The value of this parameter is assigned to script state in glyph nodes that get
injected. It has no meaning in itself but can be used at the \LUA\ end.

\stopnewprimitive

\startnewprimitive[title={\prm {glyphtextscale}}]

This multiplier is applied to text font and glyph dimension properties when text
style is used.

\stopnewprimitive

\startnewprimitive[title={\prm {glyphweight}}]

An integer parameter defining the current glyph weight, assigned to glyphs
(characters) inserted into the current list.

\stopnewprimitive

\startnewprimitive[title={\prm {glyphxoffset}}]

An integer parameter defining the current glyph x offset, assigned to glyphs
(characters) inserted into the current list. Normally this will only be set when
one explicitly works with glyphs and defines a specific sequence.

\stopnewprimitive

\startnewprimitive[title={\prm {glyphxscale}}]

An integer parameter defining the current glyph x scale, assigned to glyphs
(characters) inserted into the current list.

\stopnewprimitive

\startnewprimitive[title={\prm {glyphxscaled}}]

This primitive returns the given dimension scaled by the \prm {glyphscale} and
\prm {glyphxscale}.

\stopnewprimitive

\startnewprimitive[title={\prm {glyphyoffset}}]

An integer parameter defining the current glyph x offset, assigned to glyphs
(characters) inserted into the current list. Normally this will only be set when
one explicitly works with glyphs and defines a specific sequence.

\stopnewprimitive

\startnewprimitive[title={\prm {glyphyscale}}]

An integer parameter defining the current glyph y scale, assigned to glyphs
(characters) inserted into the current list.

\stopnewprimitive

\startnewprimitive[title={\prm {glyphyscaled}}]

This primitive returns the given dimension scaled by the \prm {glyphscale} and
\prm {glyphyscale}.

\stopnewprimitive

\startnewprimitive[title={\prm {gtoksapp}}]

This is the global variant of \prm {toksapp}.

\stopnewprimitive

\startnewprimitive[title={\prm {gtokspre}}]

This is the global variant of \prm {tokspre}.

\stopnewprimitive

\startoldprimitive[title={\prm {halign}}]

This command starts horizontally aligned material. Macro packages use this
command in table mechanisms and math alignments. It starts with a preamble
followed by entries (rows and columns).

\stopoldprimitive

\startoldprimitive[title={\prm {hangafter}}]

This parameter tells the par builder when indentation specified with \prm
{hangindent} starts. A negative value does the opposite and starts indenting
immediately. So, a value of $-2$ will make the first two lines indent.

\stopoldprimitive

\startoldprimitive[title={\prm {hangindent}}]

This parameter relates to \prm {hangafter} and sets the amount of indentation.
When larger than zero indentation happens left, otherwise it starts at the right
edge.

\stopoldprimitive

\startoldprimitive[title={\prm {hbadness}}]

This sets the threshold for reporting a horizontal badness value, its current
value is \the \badness.

\stopoldprimitive

\startoldprimitive[title={\prm {hbox}}]

This constructs a horizontal box. There are a lot of optional parameters so more
details can be found in dedicated manuals. When the content is packed a callback
can kick in that can be used to apply for instance font features.

\stopoldprimitive

\startnewprimitive[title={\prm {hccode}}]

The \TEX\ engine is good at hyphenating but traditionally that has been limited
to hyphens. Some languages however use different characters. You can set up a
different \prm {hyphenchar} as well as pre and post characters, but there's also
a dedicated code for controlling this.

\startbuffer
\hccode"2013 "2013

\hsize 50mm test\char"2013test\par
\hsize  1mm test\char"2013test\par

\hccode"2013 `!

\hsize 50mm test\char"2013test\par
\hsize  1mm test\char"2013test\par
\stopbuffer

\typebuffer

This example shows that we can mark a character as hyphen|-|like but also can
remap it to something else:

\startpacked \getbuffer \stoppacked

\stopnewprimitive

\startoldprimitive[title={\prm {hfil}}]

This is a shortcut for \typ {\hskip plus 1 fil} (first order filler).

\stopoldprimitive

\startoldprimitive[title={\prm {hfill}}]

This is a shortcut for \typ {\hskip plus 1 fill} (second order filler).

\stopoldprimitive

\startoldprimitive[title={\prm {hfilneg}}]

This is a shortcut for \typ {\hskip plus - 1 fil} so it can compensate \prm
{hfil}.

\stopoldprimitive

\startoldprimitive[title={\prm {hfuzz}}]

This dimension sets the threshold for reporting horizontal boxes that are under-
or overfull. The current value is \the \hfuzz.

\stopoldprimitive

\startnewprimitive[title={\prm {hjcode}}]

The so called lowercase code determines if a character is part of a
to|-|be|-|hyphenated word. In \LUATEX\ we introduced the \quote {hyphenation
justification} code as replacement. When a language is saved and no \prm {hjcode}
is set the \prm {lccode} is used instead. This code serves a second purpose. When
the assigned value is greater than 0 but less than 32 it indicated the to be used
length when checking for left- and righthyphenmin. For instance it make sense to
set the code to~2 for characters like œ.

\stopnewprimitive

\startoldprimitive[title={\prm {hkern}}]

This primitive is like \prm {kern} but will force the engine into horizontal mode
if it isn't yet.

\stopoldprimitive

\startnewprimitive[title={\prm {hmcode}}]

The \type {hm} stands for \quote {hyphenation math}. When bit~1 is set the
characters will be repeated on the next line after a break. The second bit
concerns italic correction but is of little relevance now that we moved to a
different model in \CONTEXT. Here are some examples, we also show an example of
\prm {mathdiscretionary} because that is what this code triggers:

\startbuffer
test $ \dorecurse {50} {
    a \discretionary class 2 {$\darkred +$}{$\darkgreen +$}{$\darkblue +$}
} b$

test $ a \mathdiscretionary class 1 {-}{-}{-} b$

\bgroup
    \hmcode"002B=1 % +
    \hmcode"002D=1 % -
    \hmcode"2212=1 % -
    test $ \dorecurse{50}{a + b - } c$
\egroup
\stopbuffer

\typebuffer

{\setuptolerance[verytolerant,stretch]\getbuffer}

\stopnewprimitive

\startoldprimitive[title={\prm {holdinginserts}}]

When set to a positive value inserts will be kept in the stream and not moved to
the insert registers.

\stopoldprimitive

\startnewprimitive[title={\prm {holdingmigrations}}]

When set to a positive value marks (and adjusts) will be kept in the stream and
not moved to the outer level or related registers.

\stopnewprimitive

\startnewprimitive[title={\prm {hpack}}]

This primitive is like \prm {hbox} but without the callback overhead.

\stopnewprimitive

\startnewprimitive[title={\prm {hpenalty}}]

This primitive is like \prm {penalty} but will force the engine into horizontal
mode if it isn't yet.

\stopnewprimitive

\startoldprimitive[title={\prm {hrule}}]

This creates a horizontal rule. Unless the width is set it will stretch to fix
the available width. In addition to the traditional \type {width}, \type {height}
and \type {depth} specifiers some more are accepted. These are discussed in other
manuals. To give an idea:

\startbuffer
h\hrule width 10mm height 2mm depth 1mm \relax rule
h\hrule width 10mm height 2mm depth 1mm xoffset 30mm yoffset -10mm \relax rule
v\vrule width 10mm height 2mm depth 1mm \relax rule
v\vrule width 10mm height 2mm depth 1mm xoffset 30mm yoffset  10mm \relax rule
\stopbuffer

\typebuffer

The \prm {relax} stops scanning and because we have more keywords we get a
different error report than in traditional \TEX\ when a lookahead confuses the
engine. On separate lines we get the following.

\startlines
\getbuffer
\stoplines

\stopoldprimitive

\startoldprimitive[title={\prm {hsize}}]

This sets (or gets) the current horizontal size.

\startbuffer
\hsize 40pt \setbox0\vbox{x} hsize: \the\wd0
\setbox0\vbox{\hsize 40pt x} hsize: \the\wd0
\stopbuffer

\typebuffer

In both cases we get the same size reported but the first one will also influence
the current paragraph when used ungrouped.

\startlines
\getbuffer
\stoplines

\stopoldprimitive

\startoldprimitive[title={\prm {hskip}}]

The given glue is injected in the horizontal list. If possible horizontal mode is
entered.

\stopoldprimitive

\startoldprimitive[title={\prm {hss}}]

\startbuffer
x\hbox to 0pt{\hskip 0pt plus 1 fil minus 1 fil\relax test}x
x\hbox to 0pt{\hss test}x
x\hbox to 0pt{test\hskip 0pt plus 1 fil minus 1 fil\relax}x
x\hbox to 0pt{test\hss}x
\stopbuffer

In traditional \TEX\ glue specifiers are shared. This makes a lot of sense when
memory has to be saved. For instance spaces in a paragraph of text are often the
same and a glue specification has at least an amount, stretch, shrink, stretch
order and shrink order field plus a leader pointer; in \LUAMETATEX\ we have even
more fields. In \LUATEX\ these shared (and therefore referenced) glue spec nodes
became just copies.

\typebuffer

The \prm {hss} primitives injects a glue node with one order stretch and one
order shrink. In traditional \TEX\ this is a reference to a shared specification,
and in \LUATEX\ just a copy of a predefined specifier. The only gain is now in
tokens because one could just be explicit or use a glue register with that value
because we have plenty glue registers.

\startlines
\getbuffer
\stoplines

We could have this:

\starttyping
\permanent\protected\untraced\def\hss
  {\hskip0pt plus 1 fil minus 1 fil\relax}
\stoptyping

or this:

\starttyping
\gluespecdef\hssglue 0pt plus 1 fil minus 1 fil

\permanent\protected\untraced\def\hss
  {\hskip\hssglue}
\stoptyping

but we just keep the originals around.

\stopoldprimitive

\startoldprimitive[title={\prm {ht}}]

Returns the height of the given box.

\stopoldprimitive

\startoldprimitive[title={\prm {hyphenation}}]

The list passed to this primitive contains hyphenation exceptions that get bound
to the current language. In \LUAMETATEX\ this can be managed at the \LUA\ end.
Exceptions are not stored in the format file.

\stopoldprimitive

\startnewprimitive[title={\prm {hyphenationmin}}]

This property (that also gets bond to the current language) sets the minimum
length of a word that gets hyphenated.

\stopnewprimitive

% \startnewprimitive[title={\prm {hyphenationmode}}]
% \stopnewprimitive

\startoldprimitive[title={\prm {hyphenchar}}]

This is one of the font related primitives: it returns the number of the hyphen
set in the given font.

\stopoldprimitive

\startoldprimitive[title={\prm {hyphenpenalty}}]

Discretionary nodes have a related default penalty. The \prm {hyphenpenalty} is
injected after a regular discretionary, and \prm {exhyphenpenalty} after \type
{\-} or \type {-}. The later case is called an automatic discretionary. In
\LUAMETATEX\ we have two extra penalties: \prm {explicithyphenpenalty} and \prm
{automatichyphenpenalty} and these are used when the related bits are set in \prm
{hyphenationmode}.

\stopoldprimitive

\startoldprimitive[title={\prm {if}}]

This traditional \TEX\ conditional checks if two character codes are the same. In
order to understand unexpanded results it is good to know that internally \TEX\
groups primitives in a way that serves the implementation. Each primitive has a
command code and a character code, but only for real characters the name
character code makes sense. This condition only really tests for character codes
when we have a character, in all other cases, the result is true.

\startbuffer
\def\A{A}\def\B{B} \chardef\C=`C \chardef\D=`D \def\AA{AA}

[\if AA   YES \else NOP \fi] [\if AB   YES \else NOP \fi]
[\if \A\B YES \else NOP \fi] [\if \A\A YES \else NOP \fi]
[\if \C\D YES \else NOP \fi] [\if \C\C YES \else NOP \fi]
[\if \count\dimen YES \else NOP \fi] [\if \AA\A YES \else NOP \fi]

\stopbuffer

\typebuffer

The last example demonstrates that the tokens get expanded, which is why
we get the extra \type {A}:

{\getbuffer}

\stopoldprimitive

\startnewprimitive[title={\prm {ifabsdim}}]

This test will negate negative dimensions before comparison, as in:

\startbuffer
\def\TestA#1{\ifdim   #1<2pt too small\orelse\ifdim   #1>4pt too large\else okay\fi}
\def\TestB#1{\ifabsdim#1<2pt too small\orelse\ifabsdim#1>4pt too large\else okay\fi}

\TestA {1pt}\quad\TestA {3pt}\quad\TestA {5pt}\crlf
\TestB {1pt}\quad\TestB {3pt}\quad\TestB {5pt}\crlf
\TestB{-1pt}\quad\TestB{-3pt}\quad\TestB{-5pt}\par
\stopbuffer

\typebuffer

So we get this:

{\getbuffer}

\stopnewprimitive

\startnewprimitive[title={\prm {ifabsfloat}}]

This test will negate negative floats before comparison, as in:

\startbuffer
\def\TestA#1{\iffloat   #1<2.46 small\orelse\iffloat   #1>4.68 large\else medium\fi}
\def\TestB#1{\ifabsfloat#1<2.46 small\orelse\ifabsfloat#1>4.68 large\else medium\fi}

\TestA {1.23}\quad\TestA {3.45}\quad\TestA {5.67}\crlf
\TestB {1.23}\quad\TestB {3.45}\quad\TestB {5.67}\crlf
\TestB{-1.23}\quad\TestB{-3.45}\quad\TestB{-5.67}\par
\stopbuffer

\typebuffer

So we get this:

{\getbuffer}

\stopnewprimitive

\startnewprimitive[title={\prm {ifabsnum}}]

This test will negate negative numbers before comparison, as in:

\startbuffer
\def\TestA#1{\ifnum   #1<100 too small\orelse\ifnum   #1>200 too large\else okay\fi}
\def\TestB#1{\ifabsnum#1<100 too small\orelse\ifabsnum#1>200 too large\else okay\fi}

\TestA {10}\quad\TestA {150}\quad\TestA {210}\crlf
\TestB {10}\quad\TestB {150}\quad\TestB {210}\crlf
\TestB{-10}\quad\TestB{-150}\quad\TestB{-210}\par
\stopbuffer

\typebuffer

Here we get the same result each time:

{\getbuffer}

\stopnewprimitive

\startnewprimitive[title={\prm {ifarguments}}]

This is a variant of \prm {ifcase} were the selector is the number of arguments
picked up. For example:

\startbuffer
\def\MyMacro#1#2#3{\ifarguments\0\or1\or2\or3\else ?\fi} \MyMacro{A}{B}{C}
\def\MyMacro#1#0#3{\ifarguments\0\or1\or2\or3\else ?\fi} \MyMacro{A}{B}{C}
\def\MyMacro#1#-#2{\ifarguments\0\or1\or2\or3\else ?\fi} \MyMacro{A}{B}{C}\par
\stopbuffer

\typebuffer

Watch the non counted, ignored, argument in the last case. Normally this test will
be used in combination with \prm {ignorearguments}.

{\getbuffer}

\stopnewprimitive

\startnewprimitive[title={\prm {ifboolean}}]

This tests a number (register or equivalent) and any nonzero value represents
\type {true}, which is nicer than using an \type {\unless \ifcase}.

\stopnewprimitive

\startoldprimitive[title={\prm {ifcase}}]

This numeric \TEX\ conditional takes a counter (literal, register, shortcut to a
character, internal quantity) and goes to the branch that matches.

\startbuffer
\ifcase 3 zero\or one\or two\or three\or four\else five or more\fi
\stopbuffer

\typebuffer

Indeed: \inlinebuffer\ equals three. In later sections we will see some
\LUAMETATEX\ primitives that behave like an \prm {ifcase}.

\stopoldprimitive

\startoldprimitive[title={\prm {ifcat}}]

Another traditional \TEX\ primitive: what happens with what gets read in depends
on the catcode of a character, think of characters marked to start math mode, or
alphabetic characters (letters) versus other characters (like punctuation).

\startbuffer
\def\A{A}\def\B{,} \chardef\C=`C \chardef\D=`, \def\AA{AA}

[\ifcat $!   YES \else NOP \fi] [\ifcat ()   YES \else NOP \fi]
[\ifcat AA   YES \else NOP \fi] [\ifcat AB   YES \else NOP \fi]
[\ifcat \A\B YES \else NOP \fi] [\ifcat \A\A YES \else NOP \fi]
[\ifcat \C\D YES \else NOP \fi] [\ifcat \C\C YES \else NOP \fi]
[\ifcat \count\dimen YES \else NOP \fi] [\ifcat \AA\A YES \else NOP \fi]
\stopbuffer

\typebuffer

Close reading is needed here:

{\getbuffer}

This traditional \TEX\ condition as a well as the one in the previous section are
hardly used in \CONTEXT, if only because they expand what follows and we seldom
need to compare characters.

\stopoldprimitive

\startnewprimitive[title={\prm {ifchkdim}}]

A variant on the checker in the previous section is a dimension checker:

\startbuffer
\ifchkdim oeps        \or okay\else error\fi\quad
\ifchkdim 12          \or okay\else error\fi\quad
\ifchkdim 12pt        \or okay\else error\fi\quad
\ifchkdim 12pt or more\or okay\else error\fi
\stopbuffer

\typebuffer

We get:

{\getbuffer}

\stopnewprimitive

\startnewprimitive[title={\prm {ifchkdimension}}]

COntrary to \prm {ifchkdim} this test doesn't accept trailing crap:

\startbuffer
\ifchkdimension oeps        \or okay\else error\fi\quad
\ifchkdimension 12          \or okay\else error\fi\quad
\ifchkdimension 12pt        \or okay\else error\fi\quad
\ifchkdimension 12pt or more\or okay\else error\fi
\stopbuffer

\typebuffer

reports:

{\getbuffer}

\stopnewprimitive

\startnewprimitive[title={\prm {ifchknum}}]

In \CONTEXT\ there are quite some cases where a variable can have a number or a
keyword indicating a symbolic name of a number or maybe even some special
treatment. Checking if a valid number is given is possible to some extend, but a
native checker makes much sense too. So here is one:

\startbuffer
\ifchknum oeps        \or okay\else error\fi\quad
\ifchknum 12          \or okay\else error\fi\quad
\ifchknum 12pt        \or okay\else error\fi\quad
\ifchknum 12pt or more\or okay\else error\fi
\stopbuffer

\typebuffer

The result is as expected:

{\getbuffer}

\stopnewprimitive

\startnewprimitive[title={\prm {ifchknumber}}]

This check is more restrictive than \prm {ifchknum} discussed in the previous
section:

\startbuffer
\ifchknumber oeps        \or okay\else error\fi\quad
\ifchknumber 12          \or okay\else error\fi\quad
\ifchknumber 12pt        \or okay\else error\fi\quad
\ifchknumber 12pt or more\or okay\else error\fi
\stopbuffer

\typebuffer

Here we get:

{\getbuffer}

\stopnewprimitive

\startnewprimitive[title={\prm {ifcmpdim}}]

This conditional compares two dimensions and the resulting \prm {ifcase}
reflects their relation:

\startbuffer
[1pt 2pt : \ifcmpdim 1pt 2pt less\or equal\or more\fi]\quad
[1pt 1pt : \ifcmpdim 1pt 1pt less\or equal\or more\fi]\quad
[2pt 1pt : \ifcmpdim 2pt 1pt less\or equal\or more\fi]
\stopbuffer

\typebuffer

This gives:

{\getbuffer}

\stopnewprimitive

\startnewprimitive[title={\prm {ifcmpnum}}]

This conditional compares two numbers and the resulting \prm {ifcase} reflects
their relation:

\startbuffer
[1 2 : \ifcmpnum 1 2 less\or equal\or more\fi]\quad
[1 1 : \ifcmpnum 1 1 less\or equal\or more\fi]\quad
[2 1 : \ifcmpnum 2 1 less\or equal\or more\fi]
\stopbuffer

\typebuffer

This gives:

{\getbuffer}

\stopnewprimitive

\startnewprimitive[title={\prm {ifcondition}}]

The conditionals in \TEX\ are hard coded as primitives and although it might
look like \type {\newif} creates one, it actually just defined three macros.

\startbuffer
\newif\ifMyTest
\meaning\MyTesttrue  \crlf
\meaning\MyTestfalse \crlf
\meaning\ifMyTest    \crlf \MyTesttrue
\meaning\ifMyTest    \par
\stopbuffer

\typebuffer {\tttf \getbuffer}

This means that when you say:

\starttyping
\ifMytest ... \else ... \fi
\stoptyping

You actually have one of:

\starttyping
\iftrue  ... \else ... \fi
\iffalse ... \else ... \fi
\stoptyping

and because these are proper conditions nesting them like:

\starttyping
\ifnum\scratchcounter > 0 \ifMyTest A\else B\fi \fi
\stoptyping

will work out well too. This is not true for macros, so for instance:

\starttyping
\scratchcounter = 1
\unexpanded\def\ifMyTest{\iftrue}
\ifnum\scratchcounter > 0 \ifMyTest A\else B\fi \fi
\stoptyping

will make a run fail with an error (or simply loop forever, depending on your
code). This is where \prm {ifcondition} enters the picture:

\starttyping
\def\MyTest{\iftrue} \scratchcounter0
\ifnum\scratchcounter > 0
    \ifcondition\MyTest A\else B\fi
\else
    x
\fi
\stoptyping

This primitive is seen as a proper condition when \TEX\ is in \quotation {fast
skipping unused branches} mode but when it is expanding a branch, it checks if
the next expanded token is a proper tests and if so, it deals with that test,
otherwise it fails. The main condition here is that the \type {\MyTest} macro
expands to a proper true or false test, so, a definition like:

\starttyping
\def\MyTest{\ifnum\scratchcounter<10 }
\stoptyping

is also okay. Now, is that neat or not?

\stopnewprimitive

\startnewprimitive[title={\prm {ifcramped}}]

Depending on the given math style this returns true of false:

\startbuffer
\ifcramped\mathstyle        no  \fi
\ifcramped\crampedtextstyle yes \fi
\ifcramped\textstyle        no  \fi
\ifcramped\displaystyle     yes \fi
\stopbuffer

\typebuffer

gives: \inlinebuffer.

\stopnewprimitive

\startoldprimitive[title={\prm {ifcsname}}]

This is an \ETEX\ conditional that complements the one on the previous section:

\starttyping
\expandafter\ifx\csname MyMacro\endcsname\relax ... \else ... \fi
            \ifcsname   MyMacro\endcsname       ... \else ... \fi
\stoptyping

Here the first one has the side effect of defining the macro and defaulting it to
\prm {relax}, while the second one doesn't do that. Just think of checking a
few million different names: the first one will deplete the hash table and
probably string space too.

In \LUAMETATEX\ the construction stops when there is no letter or other character
seen (\TEX\ expands on the go so expandable macros are dealt with). Instead of an
error message, the match is simply false and all tokens till the \prm
{endcsname} are gobbled.

\stopoldprimitive

\startnewprimitive[title={\prm {ifcstok}}]

A variant on the primitive mentioned in the previous section is one that
operates on lists and macros:

\startbuffer[a]
\def\a{a} \def\b{b} \def\c{a}
\stopbuffer

\typebuffer[a]

\startbuffer[b]
\ifcstok\a\b   Y\else N\fi\space
\ifcstok\a\c   Y\else N\fi\space
\ifcstok{\a}\c Y\else N\fi\space
\ifcstok{a}\c  Y\else N\fi
\stopbuffer

This:

\typebuffer[b]

{\getbuffer[a]will give us: \inlinebuffer[b].}

\stopnewprimitive

\startoldprimitive[title={\prm {ifdefined}}]

In traditional \TEX\ checking for a macro to exist was a bit tricky and therefore
\ETEX\ introduced a convenient conditional. We can do this:

\starttyping
\ifx\MyMacro\undefined ... \else ... \fi
\stoptyping

but that assumes that \type {\undefined} is indeed undefined. Another test often
seen was this:

\starttyping
\expandafter\ifx\csname MyMacro\endcsname\relax ... \else ... \fi
\stoptyping

Instead of comparing with \type {\undefined} we need to check with \prm {relax}
because the control sequence is defined when not yet present and defaults to
\prm {relax}. This is not pretty.

\stopoldprimitive

\startoldprimitive[title={\prm {ifdim}}]

Dimensions can be compared with this traditional \TEX\ primitive.

\startbuffer
\scratchdimen=1pt \scratchcounter=65536

\ifdim\scratchdimen=\scratchcounter sp YES \else NOP\fi
\ifdim\scratchdimen=1               pt YES \else NOP\fi
\stopbuffer

\typebuffer

The units are mandate:

{\getbuffer}

\stopoldprimitive

\startnewprimitive[title={\prm {ifdimexpression}}]

The companion of the previous primitive is:

\startbuffer
\ifdimexpression 10pt > 10bp \relax
    do-something
\fi
\stopbuffer

This matches when the result is non zero, and you can mix calculations and tests
as with normal expressions. Contrary to the number variant units can be used and
precision kicks in.

\stopnewprimitive

\startnewprimitive[title={\prm {ifdimval}}]

This conditional is a variant on \prm {ifchkdim} and provides some more
detailed information about the value:

\startbuffer
[-12pt : \ifdimval-12pt\or negative\or zero\or positive\else error\fi]\quad
[0pt   : \ifdimval  0pt\or negative\or zero\or positive\else error\fi]\quad
[12pt  : \ifdimval 12pt\or negative\or zero\or positive\else error\fi]\quad
[oeps  : \ifdimval oeps\or negative\or zero\or positive\else error\fi]
\stopbuffer

\typebuffer

This gives:

{\getbuffer}

\stopnewprimitive

\startnewprimitive[title={\prm {ifempty}}]

This conditional checks if a control sequence is empty:

\starttyping
is \ifempty\MyMacro \else not \fi empty
\stoptyping

It is basically a shortcut of:

\starttyping
is \ifx\MyMacro\empty \else not \fi empty
\stoptyping

with:

\starttyping
\def\empty{}
\stoptyping

Of course this is not empty at all:

\starttyping
\def\notempty#1{}
\stoptyping

\stopnewprimitive

\startoldprimitive[title={\prm {iffalse}}]

Here we have a traditional \TEX\ conditional that is always false (therefore the
same is true for any macro that is \prm {let} to this primitive).

\stopoldprimitive

\startnewprimitive[title={\prm {ifflags}}]

This test primitive relates to the various flags that one can set on a control
sequence in the perspective of overload protection and classification.

\startbuffer
\protected\untraced\tolerant\def\foo[#1]{...#1...}
\permanent\constant         \def\oof{okay}
\stopbuffer

\typebuffer

\start \getbuffer
\starttabulate[|l|c|c|l|c|c|]
\FL
\NC flag          \NC \type  {\foo}                          \NC \type  {\oof}
\NC flag          \NC \type  {\foo}                          \NC \type  {\oof}                          \NC \NR
\ML
\NC frozen        \NC \ifflags\foo\frozen        Y\else N\fi \NC \ifflags\oof\frozen        Y\else N\fi
\NC permanent     \NC \ifflags\foo\permanent     Y\else N\fi \NC \ifflags\oof\permanent     Y\else N\fi \NC \NR
\NC immutable     \NC \ifflags\foo\immutable     Y\else N\fi \NC \ifflags\oof\immutable     Y\else N\fi
\NC mutable       \NC \ifflags\foo\mutable       Y\else N\fi \NC \ifflags\oof\mutable       Y\else N\fi \NC \NR
\NC noaligned     \NC \ifflags\foo\noaligned     Y\else N\fi \NC \ifflags\oof\noaligned     Y\else N\fi
\NC instance      \NC \ifflags\foo\instance      Y\else N\fi \NC \ifflags\oof\instance      Y\else N\fi \NC \NR
\NC untraced      \NC \ifflags\foo\untraced      Y\else N\fi \NC \ifflags\oof\untraced      Y\else N\fi
\NC global        \NC \ifflags\foo\global        Y\else N\fi \NC \ifflags\oof\global        Y\else N\fi \NC \NR
\NC tolerant      \NC \ifflags\foo\tolerant      Y\else N\fi \NC \ifflags\oof\tolerant      Y\else N\fi
\NC constant      \NC \ifflags\foo\constant      Y\else N\fi \NC \ifflags\oof\constant      Y\else N\fi \NC \NR
\NC protected     \NC \ifflags\foo\protected     Y\else N\fi \NC \ifflags\oof\protected     Y\else N\fi
\NC semiprotected \NC \ifflags\foo\semiprotected Y\else N\fi \NC \ifflags\oof\semiprotected Y\else N\fi \NC \NR
\LL
\stoptabulate
\stop

Instead of checking against a prefix you can test against a bitset made from:

\startluacode
context.starttabulate { "|r|l|r|l|r|l|r|l|" }
local n = 4
for k, v in table.sortedhash(tex.flagcodes) do
    if tonumber(k) then
        n = n - 1
        context.NC() context("0x%X",k)
        context.NC() context(v)
        if n == 0 then
            context.NC()
            context.NR()
            n = 4
        end
    end
end
context.stoptabulate()
\stopluacode

\stopnewprimitive

\startnewprimitive[title={\prm {iffloat}}]

This test does for floats what \prm {ifnum}, \prm {ifdim} do for numbers and
dimensions: comparing two of them.

\stopnewprimitive

\startoldprimitive[title={\prm {iffontchar}}]

This is an \ETEX\ conditional. It takes a font identifier and a character number.
In modern fonts simply checking could not be enough because complex font features
can swap in other ones and their index can be anything. Also, a font mechanism
can provide fallback fonts and characters, so don't rely on this one too much. It
just reports true when the font passed to the frontend has a slot filled.

\stopoldprimitive

\startnewprimitive[title={\prm {ifhaschar}}]

This one is a simplified variant of the above:

\startbuffer
\ifhaschar !{this ! works} yes \else no \fi
\stopbuffer

\typebuffer

and indeed we get: \inlinebuffer ! Of course the spaces in this this example
code are normally not present in such a test.

\stopnewprimitive

\startnewprimitive[title={\prm {ifhastok}}]

This conditional looks for occurrences in token lists where each argument has to
be a proper list.

\startbuffer
\def\scratchtoks{x}

\ifhastoks{yz}         {xyz} Y\else N\fi\quad
\ifhastoks\scratchtoks {xyz} Y\else N\fi
\stopbuffer

\typebuffer

We get:

{\getbuffer}

\stopnewprimitive

\startnewprimitive[title={\prm {ifhastoks}}]

This test compares two token lists. When a macro is passed it's meaning
gets used.

\startbuffer
\def\x  {x}
\def\xyz{xyz}

(\ifhastoks  {x}  {xyz}Y\else N\fi)\quad
(\ifhastoks {\x}  {xyz}Y\else N\fi)\quad
(\ifhastoks  \x   {xyz}Y\else N\fi)\quad
(\ifhastoks  {y}  {xyz}Y\else N\fi)\quad
(\ifhastoks {yz}  {xyz}Y\else N\fi)\quad
(\ifhastoks {yz} {\xyz}Y\else N\fi)
\stopbuffer

\typebuffer {\getbuffer}

\stopnewprimitive

\startnewprimitive[title={\prm {ifhasxtoks}}]

This primitive is like the one in the previous section but this time the
given lists are expanded.

\startbuffer
\def\x  {x}
\def\xyz{\x yz}

(\ifhasxtoks  {x}  {xyz}Y\else N\fi)\quad
(\ifhasxtoks {\x}  {xyz}Y\else N\fi)\quad
(\ifhastoks   \x   {xyz}Y\else N\fi)\quad
(\ifhasxtoks  {y}  {xyz}Y\else N\fi)\quad
(\ifhasxtoks {yz}  {xyz}Y\else N\fi)\quad
(\ifhasxtoks {yz} {\xyz}Y\else N\fi)
\stopbuffer

\typebuffer {\getbuffer}

This primitive has some special properties.

\startbuffer
\edef\+{\expandtoken 9 `+}

\ifhasxtoks {xy}   {xyz}Y\else N\fi\quad
\ifhasxtoks {x\+y} {xyz}Y\else N\fi
\stopbuffer

\typebuffer

Here the first argument has a token that has category code \quote {ignore} which
means that such a character will be skipped when seen. So the result is:

{\getbuffer}

This permits checks like these:

\startbuffer
\edef\,{\expandtoken 9 `,}

\ifhasxtoks{\,x\,} {,x,y,z,}Y\else N\fi\quad
\ifhasxtoks{\,y\,} {,x,y,z,}Y\else N\fi\quad
\ifhasxtoks{\,z\,} {,x,y,z,}Y\else N\fi\quad
\ifhasxtoks{\,x\,}  {,xy,z,}Y\else N\fi
\stopbuffer

\typebuffer

I admit that it needs a bit of a twisted mind to come up with this, but it works
ok:

{\getbuffer}

\stopnewprimitive

\startoldprimitive[title={\prm {ifhbox}}]

This traditional conditional checks if a given box register or internal box
variable represents a horizontal box,

\stopoldprimitive

\startoldprimitive[title={\prm {ifhmode}}]

This traditional conditional checks we are in (restricted) horizontal mode.

\stopoldprimitive

\startnewprimitive[title={\prm {ifinalignment}}]

As the name indicates, this primitive tests for being in an alignment. Roughly
spoken, the engine is either in a state of align, handling text or dealing with
math.

\stopnewprimitive

\startnewprimitive[title={\prm {ifincsname}}]

This conditional is sort of obsolete and can be used to check if we're inside a
\prm {csname} or \prm {ifcsname} construction. It's not used in \CONTEXT.

\stopnewprimitive

\startoldprimitive[title={\prm {ifinner}}]

This traditional one can be confusing. It is true when we are in restricted
horizontal mode (a box), internal vertical mode (a box), or inline math mode.

\startbuffer
test \ifhmode \ifinner INNER\fi HMODE\fi\crlf
\hbox{test \ifhmode \ifinner INNER \fi HMODE\fi} \par

\ifvmode \ifinner INNER\fi VMODE \fi\crlf
\vbox{\ifvmode \ifinner INNER \fi VMODE\fi} \crlf
\vbox{\ifinner INNER \ifvmode VMODE \fi \fi} \par
\stopbuffer

\typebuffer

Watch the last line: because we typeset \type {INNER} we enter horizontal mode:

{\getbuffer}

\stopoldprimitive

\startnewprimitive[title={\prm {ifinsert}}]

This is the equivalent of \prm {ifvoid} for a given insert class.

\stopnewprimitive

\startnewprimitive[title={\prm {ifintervaldim}}]

This conditional is true when the intervals around the values of two dimensions
overlap. The first dimension determines the interval.

\startbuffer
[\ifintervaldim1pt 20pt 21pt \else no \fi overlap]
[\ifintervaldim1pt 18pt 20pt \else no \fi overlap]
\stopbuffer

\typebuffer

So here: \inlinebuffer

\stopnewprimitive

\startnewprimitive[title={\prm {ifintervalfloat}}]

This one does with floats what we described under \prm {ifintervaldim}.

\stopnewprimitive

\startnewprimitive[title={\prm {ifintervalnum}}]

This one does with integers what we described under \prm {ifintervaldim}.

\stopnewprimitive

\startnewprimitive[title={\prm {iflastnamedcs}}]

When a \prm {csname} is constructed and succeeds the last one is remembered and
can be accessed with \prm {lastnamedcs}. It can however be an undefined one. That
state can be checked with this primitive. Of course it also works with the \prm
{ifcsname} and \prm {begincsname} variants.

\stopnewprimitive

\startnewprimitive[title={\prm {ifmathparameter}}]

This is an \prm {ifcase} where the value depends on if the given math parameter
is zero, (\type {0}), set (\type {1}), or unset (\type {2}).

\starttyping
\ifmathparameter\Umathpunctclosespacing\displaystyle
    zero    \or
    nonzero \or
    unset   \fi
\stoptyping

\stopnewprimitive

\startnewprimitive[title={\prm {ifmathstyle}}]

This is a variant of \prm {ifcase} were the number is one of the seven possible
styles: display, text, cramped text, script, cramped script, script script,
cramped script script.

\starttyping
\ifmathstyle
  display
\or
  text
\or
  cramped text
\else
  normally smaller than text
\fi
\stoptyping

\stopnewprimitive

\startoldprimitive[title={\prm {ifmmode}}]

This traditional conditional checks we are in (inline or display) math mode mode.

\stopoldprimitive

\startoldprimitive[title={\prm {ifnum}}]

This is a frequently used conditional: it compares two numbers where a number is
anything that can be seen as such.

\startbuffer
\scratchcounter=65 \chardef\A=65

\ifnum65=`A              YES \else NOP\fi
\ifnum\scratchcounter=65 YES \else NOP\fi
\ifnum\scratchcounter=\A YES \else NOP\fi
\stopbuffer

\typebuffer

Unless a number is an unexpandable token it ends with a space or \prm {relax},
so when you end up in the true branch, you'd better check if \TEX\ could
determine where the number ends.

{\getbuffer}

% When comparing integers, definitions (for instance characters) that can be seen
% as such, or any converter that produces a number (like the \type {`} or \prm
% {number} the usual \type {=}, \type {<} or \type {>} can be used. However, in
% \LUAMETATEX\ you can negate such a comparison by \type {!}: \type {!=}, \type
% {!<} or \type {!>}. Successive \type {!} toggle the negation state.

On top of these \ASCII\ combinations, the engine also accepts some \UNICODE\
characters. This brings the full repertoire to:

\starttabulate[|l|cT|cT|l|]
\FL
\BC character      \BC               \BC    \BC operation         \NC \NR
\ML
\NC \type {0x003C} \NC $\Uchar"003C$ \NC    \NC less              \NC \NR
\NC \type {0x003D} \NC $\Uchar"003D$ \NC    \NC equal             \NC \NR
\NC \type {0x003E} \NC $\Uchar"003E$ \NC    \NC more              \NC \NR
\NC \type {0x2208} \NC $\Uchar"2208$ \NC    \NC element of        \NC \NR
\NC \type {0x2209} \NC $\Uchar"2209$ \NC    \NC not element of    \NC \NR
\NC \type {0x2260} \NC $\Uchar"2260$ \NC != \NC not equal         \NC \NR
\NC \type {0x2264} \NC $\Uchar"2264$ \NC !> \NC less equal        \NC \NR
\NC \type {0x2265} \NC $\Uchar"2265$ \NC !< \NC greater equal     \NC \NR
\NC \type {0x2270} \NC $\Uchar"2270$ \NC    \NC not less equal    \NC \NR
\NC \type {0x2271} \NC $\Uchar"2271$ \NC    \NC not greater equal \NC \NR
\LL
\stoptabulate

This also applied to \prm {ifdim} although in the case of element we discard the
fractional part (read: divide the numeric representation by 65536).

\stopoldprimitive

\startnewprimitive[title={\prm {ifnumexpression}}]

Here is an example of a conditional using expressions:

\startbuffer
\ifnumexpression (\scratchcounterone > 5) and (\scratchcountertwo > 5) \relax
    do-something
\fi
\stopbuffer

This matches when the result is non zero, and you can mix calculations and tests
as with normal expressions.

\stopnewprimitive

\startnewprimitive[title={\prm {ifnumval}}]

This conditional is a variant on \prm {ifchknum}. This time we get
some more detail about the value:

\startbuffer
[-12  : \ifnumval  -12\or negative\or zero\or positive\else error\fi]\quad
[0    : \ifnumval    0\or negative\or zero\or positive\else error\fi]\quad
[12   : \ifnumval   12\or negative\or zero\or positive\else error\fi]\quad
[oeps : \ifnumval oeps\or negative\or zero\or positive\else error\fi]
\stopbuffer

\typebuffer

This gives:

{\getbuffer}

\stopnewprimitive

\startoldprimitive[title={\prm {ifodd}}]

One reason for this condition to be around is that in a double sided layout we
need test for being on an odd or even page. It scans for a number the same was
as other primitives,

\startbuffer
\ifodd65 YES \else NO\fi &
\ifodd`B YES \else NO\fi .
\stopbuffer

\typebuffer

So: {\inlinebuffer}

\stopoldprimitive

\startnewprimitive[title={\prm {ifparameter}}]

In a macro body \type {#1} is a reference to a parameter. You can check if one is
set using a dedicated parameter condition:

\startbuffer
\tolerant\def\foo[#1]#*[#2]%
  {\ifparameter#1\or one\else no one\fi\enspace
   \ifparameter#2\or two\else no two\fi\emspace}

\foo
\foo[1]
\foo[1][2]
\stopbuffer

\typebuffer

We get:

\getbuffer

\stopnewprimitive

\startnewprimitive[title={\prm {ifparameters}}]

This is equivalent to an \prm {ifcase} with as value the number of parameters
passed to the current macro.

\stopnewprimitive

\startnewprimitive[title={\prm {ifrelax}}]

This is a convenient shortcut for \typ {\ifx\relax} and the motivation for adding
this one is (as with some others) to get less tracing.

\stopnewprimitive

\startnewprimitive[title={\prm {iftok}}]

When you want to compare two arguments, the usual way to do this is the
following:

\starttyping
\edef\tempA{#1}
\edef\tempb{#2}
\ifx\tempA\tempB
    the same
\else
    different
\fi
\stoptyping

This works quite well but the fact that we need to define two macros can be
considered a bit of a nuisance. It also makes macros that use this method to be
not so called \quote {fully expandable}. The next one avoids both issues:

\starttyping
\iftok{#1}{#2}
    the same
\else
    different
\fi
\stoptyping

Instead of direct list you can also pass registers, so given:

\startbuffer[a]
\scratchtoks{a}%
\toks0{a}%
\stopbuffer

\typebuffer[a]

This:

\startbuffer[b]
\iftok 0 \scratchtoks          Y\else N\fi\space
\iftok{a}\scratchtoks          Y\else N\fi\space
\iftok\scratchtoks\scratchtoks Y\else N\fi
\stopbuffer

\typebuffer[b]

{\getbuffer[a]gives: \inlinebuffer[b].}

\stopnewprimitive

\startoldprimitive[title={\prm {iftrue}}]

Here we have a traditional \TEX\ conditional that is always true (therefore the
same is true for any macro that is \prm {let} to this primitive).

\stopoldprimitive

\startoldprimitive[title={\prm {ifvbox}}]

This traditional conditional checks if a given box register or internal box
variable represents a vertical box,

\stopoldprimitive

\startoldprimitive[title={\prm {ifvmode}}]

This traditional conditional checks we are in (internal) vertical mode.

\stopoldprimitive

\startoldprimitive[title={\prm {ifvoid}}]

This traditional conditional checks if a given box register or internal box
variable has any content.

\stopoldprimitive

\startoldprimitive[title={\prm {ifx}}]

We use this traditional \TEX\ conditional a lot in \CONTEXT. Contrary to \prm {if}
the two tokens that are compared are not expanded. This makes it possible to compare
the meaning of two macros. Depending on the need, these macros can have their content
expanded or not. A different number of parameters results in false.

Control sequences are identical when they have the same command code and
character code. Because a \prm {let} macro is just a reference, both let macros
are the same and equal to \prm {relax}:

\starttyping
\let\one\relax \let\two\relax
\stoptyping

The same is true for other definitions that result in the same (primitive) or
meaning encoded in the character field (think of \prm {chardef}s and so).

\stopoldprimitive

\startnewprimitive[title={\prm {ifzerodim}}]

This tests for a dimen (dimension) being zero so we have:

\starttyping
\ifdim<dimension>=0pt
\ifzerodim<dimension>
\ifcase<dimension register>
\stoptyping

\stopnewprimitive

\startnewprimitive[title={\prm {ifzerofloat}}]

As the name indicated, this tests for a zero float value.

\startbuffer
[\scratchfloat\zerofloat \ifzerofloat\scratchfloat \else not \fi zero]
[\scratchfloat\plusone   \ifzerofloat\scratchfloat \else not \fi zero]
[\scratchfloat 0.01      \ifzerofloat\scratchfloat \else not \fi zero]
[\scratchfloat 0.0e0     \ifzerofloat\scratchfloat \else not \fi zero]
[\scratchfloat \zeropoint\ifzerofloat\scratchfloat \else not \fi zero]
\stopbuffer

\typebuffer

So: \inlinebuffer

\stopnewprimitive

\startnewprimitive[title={\prm {ifzeronum}}]

This tests for a number (integer) being zero so we have these variants now:

\starttyping
\ifnum<integer or equivalent>=0
\ifzeronum<integer or equivalent>
\ifcase<integer or equivalent>
\stoptyping

\stopnewprimitive

\startnewprimitive[title={\prm {ignorearguments}}]

This primitive will quit argument scanning and start expansion of the body of a
macro. The number of grabbed arguments can be tested as follows:

\startbuffer
\def\MyMacro[#1][#2][#3]%
 {\ifarguments zero\or one\or two\or three \else hm\fi}

\MyMacro          \ignorearguments \quad
\MyMacro       [1]\ignorearguments \quad
\MyMacro    [1][2]\ignorearguments \quad
\MyMacro [1][2][3]\ignorearguments \par
\stopbuffer

\typebuffer

{\getbuffer}

{\em Todo: explain optional delimiters.}

\stopnewprimitive

\startnewprimitive[title={\prm {ignoredepthcriterion}}]

When setting the \prm {prevdepth} (either by \TEX\ or by the current user) of the
current vertical list the value 1000pt is a signal for special treatment of the
skip between \quote {lines}. There is an article on that in the distribution. It
also demonstrates that \prm {ignoredepthcriterion} can be used to change this
special signal, just in case it is needed.

\stopnewprimitive

\startnewprimitive[title={\prm {ignorenestedupto}}]

This primitive gobbles following tokens and can deal with nested \quote
{environments}, for example:

\startbuffer
\def\startfoo{\ignorenestedupto\startfoo\stopfoo}

(before
\startfoo
    test \startfoo test \stopfoo
   {test \startfoo test \stopfoo}
\stopfoo
after)
\stopbuffer

\typebuffer

delivers:

\getbuffer

\stopnewprimitive

\startnewprimitive[title={\prm {ignorepars}}]

This is a variant of \prm {ignorespaces}: following spaces {\em and} \type
{\par} equivalent tokens are ignored, so for instance:

\startbuffer
one + \ignorepars

two = \ignorepars \par
three
\stopbuffer

\typebuffer

renders as: \inlinebuffer. Traditionally \TEX\ has been sensitive to \prm {par}
tokens in some of its building blocks. This has to do with the fact that it could
indicate a runaway argument which in the times of slower machines and terminals
was best to catch early. In \LUAMETATEX\ we no longer have long macros and the
mechanisms that are sensitive can be told to accept \prm {par} tokens (and
\CONTEXT\ set them such that this is the case).

\stopnewprimitive

\startnewprimitive[title={\prm {ignorerest}}]

An example shows what this primitive does:

\startbuffer
\tolerant\def\foo[#1]#*[#2]%
  {1234
   \ifparameter#1\or\else
     \expandafter\ignorerest
   \fi
   /#1/
   \ifparameter#2\or\else
     \expandafter\ignorerest
   \fi
   /#2/ }

\foo test \foo[456] test \foo[456][789] test
\stopbuffer

\typebuffer

As this likely makes most sense in conditionals you need to make sure the current
state is properly finished. Because \prm {expandafter} bumps the input state,
here we actually quit two levels; this is because so called \quote {backed up
text} is intercepted by this primitive.

\getbuffer

\stopnewprimitive

\startoldprimitive[title={\prm {ignorespaces}}]

This traditional \TEX\ primitive signals the scanner to ignore the following
spaces, if any. We mention it because we show a companion in the next section.

\stopoldprimitive

\startnewprimitive[title={\prm {ignoreupto}}]

This ignores everything upto the given token, so

\startbuffer
\ignoreupto \foo not this but\foo only this
\stopbuffer

\typebuffer

will give: \inlinebuffer .

\stopnewprimitive

\startoldprimitive[title={\prm {immediate}}]

This one has no effect unless you intercept it at the \LUA\ end and act upon it.
In original \TEX\ immediate is used in combination with read from and write to
file operations. So, this is an old primitive with a new meaning.

\stopoldprimitive

\startnewprimitive[title={\prm {immutable}}]

This prefix flags what follows as being frozen and is usually applied to for
instance \prm {integerdef}'d control sequences. In that respect is is like \prm
{permanent} but it makes it possible to distinguish quantities from macros.

\stopnewprimitive

\startoldprimitive[title={\prm {indent}}]

In engines other than \LUAMETATEX\ a paragraph starts with an indentation box.
The width of that (empty) box is determined by \prm {parindent}. In \LUAMETATEX\
we can use a dedicated indentation skip instead (as part of paragraph
normalization). An indentation can be zero'd with \prm {undent}.

\stopoldprimitive

\startnewprimitive[title={\prm {indexedsubprescript}}]

This primitive (or \type {____}) puts a flag on the script but renders
the same:

\startbuffer
$
    x \indexedsuperprescript{2} \subprescript       {2} +
    x \superprescript       {2} \indexedsubprescript{2} +
    x \superprescript       {2} ____                {2} =
    x \superprescript       {2} \subprescript       {2}
$
\stopbuffer

\typebuffer

Gives: \inlinebuffer.

\stopnewprimitive

\startnewprimitive[title={\prm {indexedsubscript}}]

This primitive (or \type {__}) puts a flag on the script but renders
the same:

\startbuffer
$
    x \indexedsuperscript{2} \subscript       {2} +
    x \superscript       {2} \indexedsubscript{2} +
    x \superscript       {2} __               {2} =
    x \superscript       {2} \subscript       {2}
$
\stopbuffer

\typebuffer

Gives: \inlinebuffer.

\stopnewprimitive

\startnewprimitive[title={\prm {indexedsuperprescript}}]

This primitive (or \type {^^^^}) puts a flag on the script but renders
the same:

\startbuffer
$
    x \indexedsuperprescript{2} \subprescript       {2} +
    x ^^^^                  {2} \subprescript       {2} +
    x \superprescript       {2} \indexedsubprescript{2} =
    x \superprescript       {2} \subprescript       {2}
$
\stopbuffer

\typebuffer

Gives: \inlinebuffer.

\stopnewprimitive

\startnewprimitive[title={\prm {indexedsuperscript}}]

This primitive (or \type {^^}) puts a flag on the script but renders
the same:

\startbuffer
$
    x \indexedsuperscript{2} \subscript       {2} +
    x ^^                 {2} \subscript       {2} +
    x \superscript       {2} \indexedsubscript{2} =
    x \superscript       {2} \subscript       {2}
$
\stopbuffer

\typebuffer

Gives: \inlinebuffer.

\stopnewprimitive

\startnewprimitive[title={\prm {indexofcharacter}}]

This primitive is more versatile variant of the backward quote operator, so
instead of:

\starttyping
\number`|
\number`~
\number`\a
\number`\q
\stoptyping

you can say:

\starttyping
\the\indexofcharacter |
\the\indexofcharacter ~
\the\indexofcharacter \a
\the\indexofcharacter \q
\stoptyping

In both cases active characters and unknown single character control sequences
are valid. In addition this also works:

\starttyping
\chardef    \foo 128
\mathchardef\oof 130

\the\indexofcharacter \foo
\the\indexofcharacter \oof
\stoptyping

An important difference is that \prm {indexofcharacter} returns an integer and
not a serialized number. A negative value indicates no valid character.

\stopnewprimitive

\startnewprimitive[title={\prm {indexofregister}}]

You can use this instead of \prm {number} for determining the index of a register
but it also returns a number when a register value is seen. The result is an
integer, not a serialized number.

\stopnewprimitive

\startnewprimitive[title={\prm {inherited}}]

When this prefix is used in a definition using \prm {let} the target will inherit
all the properties of the source.

\stopnewprimitive

\startnewprimitive[title={\prm {initcatcodetable}}]

This initializes the catcode table with the given index.

\stopnewprimitive

\startnewprimitive[title={\prm {initialpageskip}}]

When a page starts the value of this register are used to initialize \prm
{pagetotal}, \prm {pagestretch} and \prm {pageshrink}. This make nicer code than
using a \prm {topskip} with weird values.

\stopnewprimitive

\startnewprimitive[title={\prm {initialtopskip}}]

When set this one will be used instead of \prm {topskip}. The rationale is that
the \prm {topskip} is often also used for side effects and compensation.

\stopnewprimitive

\startoldprimitive[title={\prm {input}}]

There are several ways to use this primitive:

\starttyping
\input  test
\input {test}
\input "test"
\input 'test'
\stoptyping

When no suffix is given, \TEX\ will assume the suffix is \type {.tex}. The second
one is normally used.

\stopoldprimitive

\startoldprimitive[title={\prm {inputlineno}}]

This integer holds the current linenumber but it is not always reliable.

\stopoldprimitive

\startoldprimitive[title={\prm {insert}}]

This stores content in the insert container with the given index. In \LUAMETATEX\
inserts bubble up to outer boxes so we don't have the \quote {deeply buried
insert issue}.

\stopoldprimitive

\startnewprimitive[title={\prm {insertbox}}]

This is the accessor for the box (with results) of an insert with the given
index. This is equivalent to the \prm {box} in the traditional method.

\stopnewprimitive

\startnewprimitive[title={\prm {insertcopy}}]

This is the accessor for the box (with results) of an insert with the given
index. It makes a copy so the original is kept. This is equivalent to a \prm
{copy} in the traditional method.

\stopnewprimitive

\startnewprimitive[title={\prm {insertdepth}}]

This is the (current) depth of the inserted material with the given index. It is
comparable to the \prm {dp} in the traditional method.

\stopnewprimitive

\startnewprimitive[title={\prm {insertdistance}}]

This is the space before the inserted material with the given index. This is
equivalent to \prm {glue} in the traditional method.

\stopnewprimitive

\startnewprimitive[title={\prm {insertheight}}]

This is the (current) depth of the inserted material with the given index. It is
comparable to the \prm {ht} in the traditional method.

\stopnewprimitive

\startnewprimitive[title={\prm {insertheights}}]

This is the combined height of the inserted material.

\stopnewprimitive

\startnewprimitive[title={\prm {insertlimit}}]

This is the maximum height that the inserted material with the given index can
get. This is equivalent to \prm {dimen} in the traditional method.

\stopnewprimitive

\startnewprimitive[title={\prm {insertmaxdepth}}]

This is the maximum depth that the inserted material with the given index can
get.

\stopnewprimitive

\startnewprimitive[title={\prm {insertmode}}]

In traditional \TEX\ inserts are controlled by a \prm {box}, \prm {dimen}, \prm
{glue} and \prm {count} register with the same index. The allocators have to take
this into account. When this primitive is set to one a different model is
followed with its own namespace. There are more abstract accessors to interface
to this. \footnote {The old model might be removed at some point.}

\stopnewprimitive

\startnewprimitive[title={\prm {insertmultiplier}}]

This is the height (contribution) multiplier for the inserted material with the
given index. This is equivalent to \prm {count} in the traditional method.

\stopnewprimitive

\startoldprimitive[title={\prm {insertpenalties}}]

This dual purpose internal counter holds the sum of penalties for insertions that
got split. When we're the output routine in reports the number of insertions that
is kept in store.

\stopoldprimitive

\startnewprimitive[title={\prm {insertpenalty}}]

This is the insert penalty associated with the inserted material with the given
index.

\stopnewprimitive

\startnewprimitive[title={\prm {insertprogress}}]

This returns the current accumulated insert height of the insert with the given
index.

\stopnewprimitive

\startnewprimitive[title={\prm {insertstorage}}]

The value passed will enable (one) or disable (zero) the insert with the given
index.

\stopnewprimitive

\startnewprimitive[title={\prm {insertstoring}}]

The value passed will enable (one) or disable (zero) inserts.

\stopnewprimitive

\startnewprimitive[title={\prm {insertunbox}}]

This is the accessor for the box (with results) of an insert with the given
index. It makes a copy so the original is kept. The content is unpacked and
injected. This is equivalent to an \prm {unvbox} in the traditional method.

\stopnewprimitive

\startnewprimitive[title={\prm {insertuncopy}}]

This is the accessor for the box (with results) of an insert with the given
index. It makes a copy so the original is kept. The content is unpacked and
injected. This is equivalent to the \prm {unvcopy} in the traditional method.

\stopnewprimitive

\startnewprimitive[title={\prm {insertwidth}}]

This is the (current) width of the inserted material with the given index. It is
comparable to the \prm {wd} in the traditional method.

\stopnewprimitive

\startnewprimitive[title={\prm {instance}}]

This prefix flags a macro as an instance which is mostly relevant when a macro
package want to categorize macros.

\stopnewprimitive

\startnewprimitive[title={\prm {integerdef}}]

You can alias to a count (integer) register with \prm {countdef}:

\starttyping
\countdef\MyCount134
\stoptyping

Afterwards the next two are equivalent:

\starttyping
\MyCount   = 99
\count1234 = 99
\stoptyping

where \type {\MyCount} can be a bit more efficient because no index needs to be
scanned. However, in terms of storage the value (here 99) is always in the register
so \type {\MyCount} has to get there. This indirectness has the benefit that directly
setting the value is reflected in the indirect accessor.

\starttyping
\integerdef\MyCount = 99
\stoptyping

This primitive also defines a numeric equivalent but this time the number is stored
with the equivalent. This means that:

\starttyping
\let\MyCopyOfCount = \MyCount
\stoptyping

will store the {\em current} value of \type {\MyCount} in \type {\MyCopyOfCount} and
changing either of them is not reflected in the other.

The usual \prm {advance}, \prm {multiply} and \prm {divide} can be used with these
integers and they behave like any number. But compared to registers they are actually
more a constant.

\stopnewprimitive

\startoldprimitive[title={\prm {interactionmode}}]

This internal integer can be used to set or query the current interaction mode:

\starttabulate[||||]
\NC \type {\batchmode    } \NC \the\batchmodecode     \NC omits all stops and terminal output \NC \NR
\NC \type {\nonstopmode  } \NC \the\nonstopmodecode   \NC omits all stops \NC \NR
\NC \type {\scrollmode   } \NC \the\scrollmodecode    \NC omits error stops \NC \NR
\NC \type {\errorstopmode} \NC \the\errorstopmodecode \NC stops at every opportunity to interact \NC \NR
\stoptabulate

% In \LUAMETATEX, for consistency, we have enabled these four as integers after
% \prm {the} but we can also decide to remove them and do this. So we leave this
% as an undocumented feature. It could have been an \ETEX\ way of abstracting the
% numeric values.
%
% \untraced\permanent\protected\def\batchmode    {\interactionmode\batchmodecode}
% \untraced\permanent\protected\def\nonstopmode  {\interactionmode\nonstopmodecode}
% \untraced\permanent\protected\def\scrollmode   {\interactionmode\scrollmodecode}
% \untraced\permanent\protected\def\errorstopmode{\interactionmode\errorstopmodecode}

\stopoldprimitive

\startoldprimitive[title={\prm {interlinepenalties}}]

This is a more granular variant of \prm {interlinepenalty}: an array of penalties
to be put between successive line from the start of a paragraph. The list starts
with the number of penalties that gets passed.

\stopoldprimitive

\startoldprimitive[title={\prm {interlinepenalty}}]

This is the penalty that is put between lines.

\stopoldprimitive

\startoldprimitive[title={\prm {jobname}}]

This gives the current job name without suffix: {\tttf \jobname}.

\stopoldprimitive

\startoldprimitive[title={\prm {kern}}]

A kern is injected with the given dimension. For variants that switch to a mode
we have \prm {hkern} and \prm {vkern}.

\stopoldprimitive

\startoldprimitive[title={\prm {language}}]

Sets (or returns) the current language, a number. In \LUATEX\ and \LUAMETATEX\
the current language is stored in the glyph nodes.

\stopoldprimitive

\startnewprimitive[title={\prm {lastarguments}}]

\startbuffer
\def\MyMacro    #1{\the\lastarguments (#1) }          \MyMacro{1}       \crlf
\def\MyMacro  #1#2{\the\lastarguments (#1) (#2)}      \MyMacro{1}{2}    \crlf
\def\MyMacro#1#2#3{\the\lastarguments (#1) (#2) (#3)} \MyMacro{1}{2}{3} \par

\def\MyMacro    #1{(#1)           \the\lastarguments} \MyMacro{1}       \crlf
\def\MyMacro  #1#2{(#1) (#2)      \the\lastarguments} \MyMacro{1}{2}    \crlf
\def\MyMacro#1#2#3{(#1) (#2) (#3) \the\lastarguments} \MyMacro{1}{2}{3} \par
\stopbuffer

\typebuffer

The value of \prm {lastarguments} can only be trusted in the expansion until
another macro is seen and expanded. For instance in these examples, as soon as a
character (like the left parenthesis) is seen, horizontal mode is entered and
\prm {everypar} is expanded which in turn can involve macros. You can see that
in the second block (that is: unless we changed \prm {everypar} in the
meantime).

{\getbuffer}

\stopnewprimitive

\startnewprimitive[title={\prm {lastatomclass}}]

This returns the class number of the last atom seen in the math input parser.

\stopnewprimitive

\startnewprimitive[title={\prm {lastboundary}}]

This primitive looks back in the list for a user boundary injected with \prm
{boundary} and when seen it returns that value or otherwise zero.

\stopnewprimitive

\startoldprimitive[title={\prm {lastbox}}]

When issued this primitive will, if possible, pull the last box from the current
list.

\stopoldprimitive

\startnewprimitive[title={\prm {lastchkdimension}}]

When the last check for a dimension with \prm {ifchkdimension} was successful
this primitive returns the value.

\stopnewprimitive

\startnewprimitive[title={\prm {lastchknumber}}]

When the last check for an integer with \prm {ifchknumber} was successful this
primitive returns the value.

\stopnewprimitive

\startoldprimitive[title={\prm {lastkern}}]

This returns the last kern seen in the list (if possible).

\stopoldprimitive

\startnewprimitive[title={\prm {lastleftclass}}]

This variable registers the first applied math class in a formula.

\stopnewprimitive

\startoldprimitive[title={\prm {lastlinefit}}]

The \ETEX\ manuals explains this parameter in detail but in practice it is enough
to know that when set to 1000 spaces in the last line might match those in the
previous line. Basically it counters the strong push of a \prm {parfillskip}.

\stopoldprimitive

\startnewprimitive[title={\prm {lastloopiterator}}]

In addition to \prm {currentloopiterator} we have a variant that stores the value
in case an unexpanded loop is used:

\startbuffer
\localcontrolledrepeat 8 { [\the\currentloopiterator\eq\the\lastloopiterator] }
\expandedrepeat        8 { [\the\currentloopiterator\eq\the\lastloopiterator] }
\unexpandedrepeat      8 { [\the\currentloopiterator\ne\the\lastloopiterator] }
\stopbuffer

\typebuffer

\startlines
\getbuffer
\stoplines

\stopnewprimitive

\startnewprimitive[title={\prm {lastnamedcs}}]

The example code in the previous section has some redundancy, in the sense that
there to be looked up control sequence name \type {mymacro} is assembled twice.
This is no big deal in a traditional eight bit \TEX\ but in a \UNICODE\ engine
multi|-|byte sequences demand some more processing (although it is unlikely that
control sequences have many multi|-|byte \UTF8\ characters).

\starttyping
\ifcsname mymacro\endcsname
    \csname mymacro\endcsname
\fi
\stoptyping

Instead we can say:

\starttyping
\ifcsname mymacro\endcsname
    \lastnamedcs
\fi
\stoptyping

Although there can be some performance benefits another advantage is that it uses
less tokens and parsing. It might even look nicer.

\stopnewprimitive

\startnewprimitive[title={\prm {lastnodesubtype}}]

When possible this returns the subtype of the last node in the current node list.
Possible values can be queried (for each node type) via \LUA\ helpers.

\stopnewprimitive

\startoldprimitive[title={\prm {lastnodetype}}]

When possible this returns the type of the last node in the current node list.
Possible values can be queried via \LUA\ helpers.

\stopoldprimitive

\startnewprimitive[title={\prm {lastpageextra}}]

This reports the last applied (permitted) overshoot.

\stopnewprimitive

\startnewprimitive[title={\prm {lastparcontext}}]

When a paragraph is wrapped up the reason is reported by this state variable.
Possible values are:

\startcolumns[n=4]
\getbuffer[engine:syntax:parcontextcodes]
\stopcolumns

\stopnewprimitive

\startnewprimitive[title={\prm {lastpartrigger}}]

There are several reasons for entering a paragraphs and some are automatic and
triggered by other commands that force \TEX\ into horizontal mode.

\startcolumns[n=4]
\getbuffer[engine:syntax:partriggercodes]
\stopcolumns

\stopnewprimitive

\startoldprimitive[title={\prm {lastpenalty}}]

This returns the last penalty seen in the list (if possible).

\stopoldprimitive

\startnewprimitive[title={\prm {lastrightclass}}]

This variable registers the last applied math class in a formula.

\stopnewprimitive

\startoldprimitive[title={\prm {lastskip}}]

This returns the last glue seen in the list (if possible).

\stopoldprimitive

\startoldprimitive[title={\prm {lccode}}]

When the \prm {lowercase} operation is applied the lowercase code of a character
is used for the replacement. This primitive is used to set that code, so it
expects two character number. The code is also used to determine what characters
make a word suitable for hyphenation, although in \LUATEX\ we introduced the \prm
{hj} code for that.

\stopoldprimitive

\startoldprimitive[title={\prm {leaders}}]

See \prm {gleaders} for an explanation.

\stopoldprimitive

\startoldprimitive[title={\prm {left}}]

Inserts the given delimiter as left fence in a math formula.

\stopoldprimitive

\startoldprimitive[title={\prm {lefthyphenmin}}]

This is the minimum number of characters after the last hyphen in a hyphenated
word.

\stopoldprimitive

\startnewprimitive[title={\prm {leftmarginkern}}]

The dimension returned is the protrusion kern that has been added (if at all)
to the left of the content in the given box.

\stopnewprimitive

\startoldprimitive[title={\prm {leftskip}}]

This skip will be inserted at the left of every line.

\stopoldprimitive

\startoldprimitive[title={\prm {leqno}}]

This primitive stores the (typeset) content (presumably a number) and when the
display formula is wrapped that number will end up left of the formula.

\stopoldprimitive

\startoldprimitive[title={\prm {let}}]

Where a \prm {def} creates a new macro, either or not with argument, a \prm {let}
creates an alias. You are not limited to aliasing macros, basically everything
can be aliased.

\stopoldprimitive

\startnewprimitive[title={\prm {letcharcode}}]

Assigning a meaning to an active character can sometimes be a bit cumbersome;
think of using some documented uppercase magic that one tends to forget as it's
used only a few times and then never looked at again. So we have this:

\startbuffer
{\letcharcode 65 1 \catcode 65 13 A : \meaning A}\crlf
{\letcharcode 65 2 \catcode 65 13 A : \meaning A}\par
\stopbuffer

\typebuffer

here we define \type {A} as an active charcter with meaning \type {1} in the
first line and \type {2} in the second.

{\tttf \getbuffer}

Normally one will assign a control sequence:

\startbuffer
{\letcharcode 66 \bf \catcode 66 13 {B   bold}: \meaning B}\crlf
{\letcharcode 73 \it \catcode 73 13 {I italic}: \meaning I}\par
\stopbuffer

\typebuffer

Of course \type {\bf} and \type {\it} are \CONTEXT\ specific commands:

{\tttf \getbuffer}

\stopnewprimitive

\startnewprimitive[title={\prm {letcsname}}]

It is easy to see that we save two tokens when we use this primitive. As with the
\type {..defcs..} variants it also saves a push back of the composed macro name.

\starttyping
\expandafter\let\csname MyMacro:1\endcsname\relax
             \letcsname MyMacro:1\endcsname\relax
\stoptyping

\stopnewprimitive

\startnewprimitive[title={\prm {letfrozen}}]

You can explicitly freeze an unfrozen macro:

\starttyping
\def\MyMacro{...}
\letfrozen\MyMacro
\stoptyping

A redefinition will now give:

\starttyping
! You can't redefine a frozen macro.
\stoptyping

\stopnewprimitive

\startnewprimitive[title={\prm {letmathatomrule}}]

You can change the class for a specific style. This probably only makes sense
for user classes. It's one of those features that we used when experimenting
with more control.

\starttyping
\letmathatomrule 4 = 4 4 0 0
\letmathatomrule 5 = 5 5 0 0
\stoptyping

This changes the classes~4 and~5 into class~ 0 in the two script styles and keeps
them the same in display and text. We leave it to the reader to ponder how useful
this is.

\stopnewprimitive

\startnewprimitive[title={\prm {letmathparent}}]

This primitive takes five arguments: the target class, and four classes that
determine the pre penalty class, post penalty class, options class and a dummy
class for future use.

\stopnewprimitive

\startnewprimitive[title={\prm {letmathspacing}}]

By default inter|-|class spacing inherits from the ordinary class but you can
remap specific combinations is you want:

\starttyping
\letmathspacing \mathfunctioncode
    \mathordinarycode \mathordinarycode
    \mathordinarycode \mathordinarycode
\stoptyping

The first value is the target class, and the nest four tell how it behaves in
display, text, script and script script style. Here \typ {\mathfunctioncode} is a
\CONTEXT\ specific class (\the\mathfunctioncode), one of the many.

\stopnewprimitive

\startnewprimitive[title={\prm {letprotected}}]

Say that you have these definitions:

\startbuffer
             \def  \MyMacroA{alpha}
\protected   \def  \MyMacroB{beta}
             \edef \MyMacroC{\MyMacroA\MyMacroB}
\letprotected      \MyMacroA
             \edef \MyMacroD{\MyMacroA\MyMacroB}
\meaning           \MyMacroC\crlf
\meaning           \MyMacroD\par
\stopbuffer

\typebuffer

The typeset meaning in this example is:

{\tttf \getbuffer}

\stopnewprimitive

\startnewprimitive[title={\prm {lettolastnamedcs}}]

The \prm {lastnamedcs} primitive is somewhat special as it is a (possible)
reference to  a control sequence which is why we have a dedicated variant of
\prm {let}.

\startbuffer
\csname relax\endcsname\let                         \foo\lastnamedcs \meaning\foo
\csname relax\endcsname\expandafter\let\expandafter \oof\lastnamedcs \meaning\oof
\csname relax\endcsname\lettolastnamedcs            \ofo             \meaning\ofo
\stopbuffer

\typebuffer % we need oneliners because intermediate csnames kick in

These give the following where the first one obviously is not doing what we want
and the second one is kind of cumbersome.

\startlines
\getbuffer
\stoplines

\stopnewprimitive

\startnewprimitive[title={\prm {lettonothing}}]

This one let's a control sequence to nothing. Assuming that \type {\empty}
is indeed empty, these two lines are equivalent.

\starttyping
\let         \foo\empty
\lettonothing\oof
\stoptyping

\stopnewprimitive

\startoldprimitive[title={\prm {limits}}]

This is a modifier: it flags the previous math atom to have its scripts above and
below the (summation, product, integral etc.) symbol. In \LUAMETATEX\ this can be
any atom (that is: any class). In display mode the location defaults to above and
below.

Like any modifier it looks back for a math specific element. This means that the
following will work well:

\starttyping
\sum \limits ^2 _3
\sum ^2 \limits _3
\sum ^2 _3 \limits
\sum ^2 _3 \limits \nolimits \limits
\stoptyping

because scripts are bound to these elements so looking back just sees the element.

\stopoldprimitive

\startnewprimitive[title={\prm {linebreakoptional}}]

This selects the optional text range that is to be used. Optional content is
marked with {optionalboundary} nodes.

\stopnewprimitive

\startnewprimitive[title={\prm {linebreakpasses}}]

When set to a positive value it will apply additional line break runs defined
with \prm {parpasses} until the criteria set in there are met. When set to~$-1$
it will signal a final pass

\stopnewprimitive

\startnewprimitive[title={\prm {linedirection}}]

This sets the text direction (1 for \type {r2l}) to the given value but keeps
preceding glue into the range.

\stopnewprimitive

\startoldprimitive[title={\prm {linepenalty}}]

Every line gets this penalty attached, so normally it is a small value, like
here: \the \linepenalty.

\stopoldprimitive

\startoldprimitive[title={\prm {lineskip}}]

This is the amount of glue that gets added when the distance between lines falls
below \prm {lineskiplimit}.

\stopoldprimitive

\startoldprimitive[title={\prm {lineskiplimit}}]

When the distance between two lines becomes less than \prm {lineskiplimit} a \prm
{lineskip} glue item is added.

\startbuffer
\ruledvbox{
    \lineskiplimit 0pt \lineskip3pt \baselineskip0pt
    \ruledhbox{line 1}
    \ruledhbox{line 2}
    \ruledhbox{\tx line 3}
}
\stopbuffer

\typebuffer

Normally the \prm {baselineskip} kicks in first but here we've set that to zero,
so we get two times a 3pt glue injected.

\startlocallinecorrection
\getbuffer
\stoplocallinecorrection

\stopoldprimitive

% \startnewprimitive[title={\prm {localbrokenpenalty}}]
% \stopnewprimitive

\startnewprimitive[title={\prm {localcontrol}}]

This primitive takes a single token:

\startbuffer
\edef\testa{\scratchcounter123 \the\scratchcounter}
\edef\testc{\testa \the\scratchcounter}
\edef\testd{\localcontrol\testa \the\scratchcounter}
\stopbuffer

\typebuffer

The three meanings are:

\start \getbuffer
\starttabulate[|T|T|]
\NC \string\testa \NC \meaning\testa \NC \NR
\NC \string\testc \NC \meaning\testc \NC \NR
\NC \string\testd \NC \meaning\testd \NC \NR
\stoptabulate
\stop

The \prm {localcontrol} makes that the following token gets expanded so we don't
see the yet to be expanded assignment show up in the macro body.

\stopnewprimitive

\startnewprimitive[title={\prm {localcontrolled}}]

The previously described local control feature comes with two extra helpers. The
\prm {localcontrolled} primitive takes a token list and wraps this into a local
control sidetrack. For example:

\startbuffer
\edef\testa{\scratchcounter123 \the\scratchcounter}
\edef\testb{\localcontrolled{\scratchcounter123}\the\scratchcounter}
\stopbuffer

\typebuffer

The two meanings are:

\start \getbuffer
\starttabulate[|T|T|]
\NC \string\testa \NC \meaningfull\testa \NC \NR
\NC \string\testb \NC \meaningfull\testb \NC \NR
\stoptabulate
\stop

The assignment is applied immediately in the expanded definition.

\stopnewprimitive

\startnewprimitive[title={\prm {localcontrolledendless}}]

As the name indicates this will loop forever. You need to explicitly quit the
loop with \prm {quitloop} or \prm {quitloopnow}. The first quitter aborts the
loop at the start of a next iteration, the second one tries to exit immediately,
but is sensitive for interference with for instance nested conditionals.

\stopnewprimitive

\startnewprimitive[title={\prm {localcontrolledloop}}]

As with more of the primitives discussed here, there is a manual in the \quote
{lowlevel} subset that goes into more detail. So, here a simple example has to
do:

\startbuffer
\localcontrolledloop 1 100 1 {%
    \ifnum\currentloopiterator>6\relax
        \quitloop
    \else
        [\number\currentloopnesting:\number\currentloopiterator]
        \localcontrolledloop 1 8 1 {%
            (\number\currentloopnesting:\number\currentloopiterator)
        }\par
    \fi
}
\stopbuffer

\typebuffer

Here we see the main loop primitive being used nested. The code shows how we can
\prm {quitloop} and have access to the \prm {currentloopiterator} as well as the
nesting depth \prm {currentloopnesting}.

\startpacked \getbuffer \stoppacked

Be aware of the fact that \prm {quitloop} will end the loop at the {\em next}
iteration so any content after it will show up. Normally this one will be issued
in a condition and we want to end that properly. Also keep in mind that because
we use local control (a nested \TEX\ expansion loop) anything you feed back can
be injected out of order.

The three numbers can be separated by an equal sign which is a trick to avoid
look ahead issues that can result from multiple serialized numbers without spaces
that indicate the end of sequence of digits.

\stopnewprimitive

\startnewprimitive[title={\prm {localcontrolledrepeat}}]

This one takes one instead three arguments which looks a bit better
in simple looping.

\stopnewprimitive

% \startnewprimitive[title={\prm {localinterlinepenalty}}]
% \stopnewprimitive

\startnewprimitive[title={\prm {localleftbox}}]

This sets the box that gets injected at the left of every line.

\stopnewprimitive

\startnewprimitive[title={\prm {localleftboxbox}}]

This returns the box set with \prm {localleftbox}.

\stopnewprimitive

\startnewprimitive[title={\prm {localmiddlebox}}]

This sets the box that gets injected at the left of every line but its width
is ignored.

\stopnewprimitive

\startnewprimitive[title={\prm {localmiddleboxbox}}]

This returns the box set with \prm {localmiddlebox}.

\stopnewprimitive

% \startnewprimitive[title={\prm {localpretolerance}}]
% \stopnewprimitive

\startnewprimitive[title={\prm {localrightbox}}]

This sets the box that gets injected at the right of every line.

\stopnewprimitive

\startnewprimitive[title={\prm {localrightboxbox}}]

This returns the box set with \prm {localrightbox}.

\stopnewprimitive

% \startnewprimitive[title={\prm {localtolerance}}]
% \stopnewprimitive

\startoldprimitive[title={\prm {long}}][obsolete=yes]

This original prefix gave the macro being defined the property that it could not
have \prm {par} (or the often equivalent empty lines) in its arguments. It was
mostly a protection against a forgotten right curly brace, resulting in a so called
run|-|away argument. That mattered on a paper terminal or slow system where such a
situation should be catched early. In \LUATEX\ it was already optional, and in
\LUAMETATEX\ we dropped this feature completely (so that we could introduce others).

\stopoldprimitive

\startoldprimitive[title={\prm {looseness}}]

The number fo lines in the current paragraph will be increased by given number of
lines. For this to succeed there need to be enough stretch in the spacing to make
that happen. There is some wishful thinking involved.

\stopoldprimitive

\startoldprimitive[title={\prm {lower}}]

This primitive takes two arguments, a dimension and a box. The box is moved down.
The operation only succeeds in horizontal mode.

\stopoldprimitive

\startoldprimitive[title={\prm {lowercase}}]

This token processor converts character tokens to their lowercase counterparts as
defined per \prm {lccode}. In order to permit dirty tricks active characters are
also processed. We don't really use this primitive in \CONTEXT, but for
consistency we let it respond to \prm {expand}: \footnote {Instead of providing
\type {\lowercased} and \type {\uppercased} primitives that would clash with
macros anyway.}

\startbuffer
\edef           \foo       {\lowercase{tex TeX \TEX}} \meaningless\foo
\lowercase{\edef\foo                  {tex TeX \TEX}} \meaningless\foo
\edef           \foo{\expand\lowercase{tex TeX \TEX}} \meaningless\foo
\stopbuffer

\typebuffer

Watch how \prm {lowercase} is not expandable but can be forced to. Of course, as
the logo macro is protected the \TEX\ logo remains mixed case.

\startlines
\getbuffer
\stoplines

\stopoldprimitive

\startnewprimitive[title={\prm {lpcode}}]

This one can be used to set the left protrusion factor of a glyph in a font and
takes three arguments: font, character code and factor. It is kind of obsolete
because we can set up vectors at definition time and tweaking from \TEX\ can have
side effects because it globally adapts the font.

\stopnewprimitive

\startnewprimitive[title={\prm {luaboundary}}]

This primive inserts a boundary that takes two integer values. Some mechanisms
(like math constructors) can trigger a callback when preceded by such a boundary.
As we go more mechanisms might do such a check but we don't want a performance
hit on \CONTEXT\ as we do so (nor unwanted interference).

\stopnewprimitive

\startnewprimitive[title={\prm {luabytecode}}]

This behaves like \prm {luafunction} but here the number is a byte code register.
These bytecodes are in the \typ {lua.bytecode} array.

\stopnewprimitive

\startnewprimitive[title={\prm {luabytecodecall}}]

This behaves like \prm {luafunctioncall} but here the number is a byte code
register. These bytecodes are in the \typ {lua.bytecode} array.

\stopnewprimitive

\startnewprimitive[title={\prm {luacopyinputnodes}}]

When set to a positive value this will ensure that when nodes are printed from
\LUA\ to \TEX\ copies are used.

\stopnewprimitive

\startnewprimitive[title={\prm {luadef}}]

% \edef\foocode{\ctxlua{
%     context(
%         context.functions.register(
%             function() context("!") end
%         )
%     )
% }}

This command relates a (user) command to a \LUA\ function registered in the \typ
{lua.lualib_get_functions_table()}, so after:

\starttyping
\luadef\foo123
\stoptyping

the \type {\foo} command will trigger the function at index 123. Of course a
macro package has to make sure that these definitions are unique. \footnote
{Plain \TEX\ established a norm for allocating registers, like \typ {\newdimen}
but there is no such convention for \LUA\ functions.}

This command is accompanied by \prm {luafunctioncall} and
\prm {luafunction}. When we have funciton 123 defined as

\starttyping
function() tex.sprint("!") end
\stoptyping

the following:

\starttyping
(\luafunctioncall  \foocode ?)
(\normalluafunction\foocode ?)
(\foo                       ?)
\stoptyping

gives three times \type {(!?)}. But this:

\starttyping
\edef\oof{\foo                      } \meaning\oof  % protected
\edef\oof{\luafunctioncall  \foocode} \meaning\oof  % protected
\edef\oof{\normalluafunction\foocode} \meaning\oof  % expands
\stoptyping

returns:

\starttyping
macro:!
macro:\luafunctioncall 1740
macro:!
\stoptyping

Because the definition command is like any other

\starttyping
\permanent\protected\luadef\foo123
\stoptyping

boils down to:

\starttyping
permanent protected luacall 123
\stoptyping

\stopnewprimitive

\startnewprimitive[title={\prm {luaescapestring}}]

This command converts the given (token) list into something that is acceptable
for \LUA. It is inherited from \LUATEX\ and not used in \CONTEXT.

\startbuffer
\directlua { tex.print ("\luaescapestring {{\tt This is a "test".}}") }
\stopbuffer

\typebuffer

Results in: \inlinebuffer\space (Watch the grouping.)

\stopnewprimitive

\startnewprimitive[title={\prm {luafunction}}]

The integer passed to this primitive is the index in the table returned by \typ
{lua.lualib_get_functions_table()}. Of course a macro package has to provide
reliable management for this. This is a so called convert command so it expands
in an expansion context (like an \prm {edef}).

\stopnewprimitive

\startnewprimitive[title={\prm {luafunctioncall}}]

The integer passed to this primitive is the index in the table returned by \typ
{lua.lualib_get_functions_table()}. Of course a macro package has to provide
reliable management for this. This primitive doesn't expand in an expansion
context (like an \prm {edef}).

\stopnewprimitive

\startnewprimitive[title={\prm {luatexbanner}}]

This gives: {\tttf \luatexbanner}.

\stopnewprimitive

\startnewprimitive[title={\prm {luatexrevision}}]

This is an integer. The current value is: {\tttf \number\luatexrevision}.

\stopnewprimitive

\startnewprimitive[title={\prm {luatexversion}}]

This is an integer. The current value is: {\tttf \number\luatexversion}.

\stopnewprimitive

\startoldprimitive[title={\prm {mark}}][obsolete=yes]

The given token list is stored in a node in the current list and might become
content of \prm {topmark}, \prm {botmark} or \prm {firstmark} when a page split
off, or in the case of a box split in \prm {splitbotmark} or \prm
{splitfirstmark}. In \LUAMETATEX\ deeply burried marks bubbly up to an outer box
level.

\stopoldprimitive

\startoldprimitive[title={\prm {marks}}]

This command is similar to \prm {mark} but first expects a number of a mark
register. Multiple marks were introduced in \ETEX.

\stopoldprimitive

\startoldprimitive[title={\prm {mathaccent}}][obsolete=yes]

This takes a number and a math object to put the accent on. The four byte number
has a dummy class byte, a family byte and two index bytes. It is replaced by \prm
{Umathaccent} that handles wide fonts.

\stopoldprimitive

\startnewprimitive[title={\prm {mathatom}}]

This operation wraps following content in a atom with the given class. It is part
of \LUAMETATEX's extended math support. There are three class related
key|/|values: \type {class}, \typ {leftclass} and \typ {rightclass} (or \type
{all} for all of them). When none is given this command expects a class number
before scanning the content. The \type {options} key expects a bitset but there
are also direct option keys, like \type {limits}, \typ {nolimits}, \type
{unpack}, \type {unroll}, \type {single}, \type {nooverflow}, \type {void} and
\type {phantom}. A \type {source} id can be set, one or more \type {attr}
assigned, and for specific purposes \typ {textfont} and \typ {mathfont}
directives are accepted. Features like this are discussed in dedicated manuals.

\stopnewprimitive

\startnewprimitive[title={\prm {mathatomglue}}]

This returns the glue that will be inserted between two atoms of a given class
for a specific style.

\startbuffer
\the\mathatomglue \textstyle   1 1
\the\mathatomglue \textstyle   0 2
\the\mathatomglue \scriptstyle 1 1
\the\mathatomglue \scriptstyle 0 2
\stopbuffer

\typebuffer

\startlines \tttf
\getbuffer
\stoplines

\stopnewprimitive

\startnewprimitive[title={\prm {mathatomskip}}]

\startbuffer
$x x$
$x \mathatomskip \textstyle   1 1 x$
$x \mathatomskip \textstyle   0 2 x$
$x \mathatomskip \scriptstyle 1 1 x$
$x \mathatomskip \scriptstyle 0 2 x$
\stopbuffer

This injects a glue with the given style and class pair specification:
\inlinebuffer.

\typebuffer


\stopnewprimitive

\startnewprimitive[title={\prm {mathbackwardpenalties}}]

See \prm {mathforwardpenalties} for an explanation.

\stopnewprimitive

\startnewprimitive[title={\prm {mathbeginclass}}]

This variable can be set to signal the class that starts the formula (think of an
imaginary leading atom).

\stopnewprimitive

\startoldprimitive[title={\prm {mathbin}}]

This operation wraps following content in a atom with class \quote {binary}.

\stopoldprimitive

\startnewprimitive[title={\prm {mathboundary}}]

This primitive is part of an experiment with granular penalties in math. When set
nested fences will use the \prm {mathdisplaypenaltyfactor} or \prm
{mathinlinepenaltyfactor} to increase nested penalties. A bit more control is
possible with \prm {mathboundary}:

\starttabulate[||||]
\NC 0 \NC begin \NC factor 1000  \NC \NR
\NC 1 \NC end   \NC factor 1000  \NC \NR
\NC 2 \NC begin \NC given factor \NC \NR
\NC 3 \NC end   \NC given factor \NC \NR
\stoptabulate

These will be used when the mentioned factors are zero. The last two variants
expect factor to be given.

\stopnewprimitive

\startoldprimitive[title={\prm {mathchar}}][obsolete=yes]

Replaced by \prm {Umathchar} this old one takes a four byte number: one byte for
the class, one for the family an two for the index. The specified character is
appended to to the list.

\stopoldprimitive

\startnewprimitive[title={\prm {mathcharclass}}]

Returns the slot (in the font) of the given math character.

\startbuffer
\the\mathcharclass\Umathchar 4 2 123
\stopbuffer

\typebuffer

The first passed number is the class, so we get: \inlinebuffer.

\stopnewprimitive

\startoldprimitive[title={\prm {mathchardef}}][obsolete=yes]

Replaced by \prm {Umathchardef} this primitive relates a control sequence with a
four byte number: one byte for the class, one for the family an two for the
index. The defined command will insert that character.

\stopoldprimitive

\startnewprimitive[title={\prm {mathcharfam}}]

Returns the family number of the given math character.

\startbuffer
\the\mathcharfam\Umathchar 4 2 123
\stopbuffer

\typebuffer

The second passed number is the family, so we get: \inlinebuffer.

\stopnewprimitive

\startnewprimitive[title={\prm {mathcharslot}}]

Returns the slot (or index in the font) of the given math character.

\startbuffer
\the\mathcharslot\Umathchar 4 2 123
\stopbuffer

\typebuffer

The third passed number is the slot, so we get: \inlinebuffer.

\stopnewprimitive

\startnewprimitive[title={\prm {mathcheckfencesmode}}]

When set to a positive value there will be no warning if a right fence (\prm
{right} or \prm {Uright}) is missing.

\stopnewprimitive

\startoldprimitive[title={\prm {mathchoice}}]

This command expects four subformulas, for display, text, script and scriptscript
and it will eventually use one of them depending on circumstances later on. Keep
in mind that a formula is first scanned and when that is finished the analysis
and typesetting happens.

\stopoldprimitive

\startnewprimitive[title={\prm {mathclass}}]

There are build in classes and user classes. The first possible user class is
\cldcontext {tex . magicconstants . mathfirstuserclass} and the last one is
\cldcontext {tex . magicconstants . mathlastuserclass}. You can better not touch
the special classes \quote {all} (\number \mathallcode), \quote {begin} (\number
\mathbegincode) and \quote {end} (\number \mathendcode). The basic 8 classes that
original \TEX\ provides are of course also present in \LUAMETATEX. In addition we
have some that relate to constructs that the engine builds.

\starttabulate[|l|l|T|l|]
\FL
\NC ordinary        \NC ord    \NC \the\mathordinarycode    \NC the default \NR
\NC operator        \NC op     \NC \the\mathoperatorcode    \NC small and large operators \NC \NR
\NC binary          \NC bin    \NC \the\mathbinarycode      \NC \NC \NR
\NC relation        \NC rel    \NC \the\mathrelationcode    \NC \NC \NR
\NC open            \NC        \NC \the\mathopencode        \NC \NC \NR
\NC close           \NC        \NC \the\mathclosecode       \NC \NC \NR
\NC punctuation     \NC punct  \NC \the\mathpunctuationcode \NC \NC \NR
\NC variable        \NC        \NC \the\mathvariablecode    \NC adapts to the current family \NC \NR
\NC active          \NC        \NC \the\mathactivecode      \NC character marked as such becomes active \NR
\NC inner           \NC        \NC \the\mathinnercode       \NC this class is not possible for characters \NR
\ML
\NC under           \NC        \NC \the\mathundercode       \NC \NC \NR
\NC over            \NC        \NC \the\mathovercode        \NC \NC \NR
\NC fraction        \NC        \NC \the\mathfractioncode    \NC \NC \NR
\NC radical         \NC        \NC \the\mathradicalcode     \NC \NC \NR
\NC middle          \NC        \NC \the\mathmiddlecode      \NC \NC \NR
\NC accent          \NC        \NC \the\mathaccentcode      \NC \NC \NR
\NC fenced          \NC        \NC \the\mathfencedcode      \NC \NC \NR
\NC ghost           \NC        \NC \the\mathghostcode       \NC \NC \NR
\NC vcenter         \NC        \NC \the\mathvcentercode     \NC \NC \NR
\LL
\stoptabulate

There is no standard for user classes but \CONTEXT\ users should be aware of
quite some additional ones that are set up. The engine initialized the default
properties of classes (spacing, penalties, etc.) the same as original \TEX.

Normally characters have class bound to them but you can (temporarily) overload
that one. The \prm {mathclass} primitive expects a class number and a valid
character number or math character and inserts the symbol as if it were of the
given class; so the original class is replaced.

\startbuffer
\ruledhbox{$(x)$} and \ruledhbox{$\mathclass 1 `(x\mathclass 1 `)$}
\stopbuffer

\typebuffer

Changing the class is likely to change the spacing, compare \inlinebuffer.

\stopnewprimitive

\startoldprimitive[title={\prm {mathclose}}]

This operation wraps following content in a atom with class \quote {close}.

\stopoldprimitive

\startoldprimitive[title={\prm {mathcode}}][obsolete=yes]

This maps a character to one in a family: the assigned value has one byte for the
class, one for the family and two for the index. It has little use in an
\OPENTYPE\ math setup.

\stopoldprimitive

\startnewprimitive[title={\prm {mathdictgroup}}]

This is an experimental feature that in due time will be explored in \CONTEXT. It
currently has no consequences for rendering.

\stopnewprimitive

\startnewprimitive[title={\prm {mathdictionary}}]

This is an experimental feature that in due time will be explored in \CONTEXT. It
currently has no consequences for rendering.

\stopnewprimitive

\startnewprimitive[title={\prm {mathdictproperties}}]

This is an experimental feature that in due time will be explored in \CONTEXT. It
currently has no consequences for rendering.

\stopnewprimitive

\startnewprimitive[title={\prm {mathdirection}}]

When set to 1 this will result in \type {r2l} typeset math formulas but of course
you then also need to set up math accordingly (which is the case in \CONTEXT).

\stopnewprimitive

% \startnewprimitive[title={\prm {mathdiscretionary}}]
% \stopnewprimitive

\startnewprimitive[title={\prm {mathdisplaymode}}]

Display mode is entered with two dollars (other characters can be used but the
dollars are a convention). Mid paragraph display formulas get a different
treatment with respect to the width and indentation than stand alone. When \prm
{mathdisplaymode} is larger than zero the double dollars (or equivalents) will
behave as inline formulas starting out in \prm {displaystyle} and with \prm
{everydisplay} expanded.

\stopnewprimitive

\startnewprimitive[title={\prm {mathdisplaypenaltyfactor}}]

This one is simular to \prm {mathinlinepenaltyfactor} but is used when we're in
display style.

\stopnewprimitive

\startnewprimitive[title={\prm {mathdisplayskipmode}}]

A display formula is preceded and followed by vertical glue specified by
\prm{abovedisplayskip} and \prm {belowdisplayskip} or \prm
{abovedisplayshortskip} and \prm {belowdisplayshortskip}. Spacing \quote {above}
is always inserted, even when zero, but the spacing \quote {below} is only
inserted when it is non|-|zero. There's also \prm {baselineskip} involved. The
way spacing is handled can be influenced with \prm {mathdisplayskipmode}, which
takes the following values:

\starttabulate[|c|l|]
\NC 0 \NC does the same as any \TEX\ engine       \NC \NR
\NC 1 \NC idem                                    \NC \NR
\NC 2 \NC only insert spacing when it is not zero \NC \NR
\NC 3 \NC never insert spacing                    \NC \NR
\stoptabulate

\stopnewprimitive

\startnewprimitive[title={\prm {mathdoublescriptmode}}]

When this parameter has a negative value double scripts trigger an error, so with
\prm {superscript}, \prm {nosuperscript}, \prm {indexedsuperscript}, \prm
{superprescript}, \prm {nosuperprescript}, \prm {indexedsuperprescript}, \prm
{subscript}, \prm {nosubscript}, \prm {indexedsubscript}, \prm {subprescript},
\prm {nosubprescript}, \prm {indexedsubprescript} and \prm {primescript}, as well
as their (multiple) \type {_} and \type {^} aliases.

A value of zero does the normal and inserts a dummy atom (basically a \type {{}})
but a positive value is more interesting. Compare these:

\startbuffer
{\mathdoublescriptmode 0      $x_x_x$}
{\mathdoublescriptmode"000000 $x_x_x$}
{\mathdoublescriptmode"030303 $x_x_x$}
{$x_x_x$}
\stopbuffer

\typebuffer

The three pairs of bytes indicate the main class, left side class and right side
class of the inserted atom, so we get this: \inlinebuffer. The last line gives
what \CONTEXT\ is configured for.

\stopnewprimitive

\startnewprimitive[title={\prm {mathendclass}}]

This variable can be set to signal the class that ends the formula (think of an
imaginary trailing atom).

\stopnewprimitive

\startnewprimitive[title={\prm {matheqnogapstep}}]

The display formula number placement heuristic puts the number on the same line
when there is place and then separates it by a quad. In \LUATEX\ we decided to
keep that quantity as it can be tight into the math font metrics but introduce
a multiplier \prm {matheqnogapstep} that defaults to 1000.

\stopnewprimitive

\startnewprimitive[title={\prm {mathfontcontrol}}]

This bitset controls how the math engine deals with fonts, and provides a way
around dealing with inconsistencies in the way they are set up. The \prm
{fontmathcontrol} makes it possible to bind options ot a specific math font. In
practice, we just set up the general approach which ii possible because we
normalize the math fonts and \quote {fix} issues at runtime.

\getbuffer[engine:syntax:mathcontrolcodes]

\stopnewprimitive

\startnewprimitive[title={\prm {mathforwardpenalties}}]

Inline math can have multiple atoms and constructs and one can configure the
penalties between then bases on classes. In addition it is possible to configure
additional penalties starting from the beginning or end using \prm
{mathforwardpenalties} and \prm {mathbackwardpenalties}. This is one the features
that we added in the perspective of breaking paragraphs heavy on math into lines.
It not that easy to come up with useable values.

\stopnewprimitive

\startnewprimitive[title={\prm {mathgluemode}}]

We can influence the way math glue is handled. By default stretch and shrink is applied but
this variable can be used to change that. The limit option ensures that the stretch and shrink
doesn't go beyond their natural values.

\getbuffer[engine:syntax:mathgluecodes]

\stopnewprimitive

\startnewprimitive[title={\prm {mathgroupingmode}}]

Normally a \type {{}} or \type {\bgroup}|-|\type {\egroup} pair in math create a
math list. However, users are accustomed to using it also for grouping and then a
list being created might not be what a user wants. As an alternative to the more
verbose \prm {begingroup}|-|\prm {endgroup} or even less sensitive \prm
{beginmathgroup}|-|\prm {endmathgroup} you can set the math grouping mode to a
non zero value which makes curly braces (and the aliases) behave as expected.

\stopnewprimitive

\startnewprimitive[title={\prm {mathinlinepenaltyfactor}}]

A math formula can have nested (sub)formulas and one might want to discourage a
line break inside those. If this value is non zero it becomes a mulitiplier, so a
value of 1000 will make an inter class penalty of 100 into 200 when at nesting
level 2 and 500 when at level 5.

\stopnewprimitive

\startoldprimitive[title={\prm {mathinner}}]

This operation wraps following content in a atom with class \quote {inner}. In
\LUAMETATEX\ we have more classes and this general wrapper one is therefore kind
of redundant.

\stopoldprimitive

\startnewprimitive[title={\prm {mathleftclass}}]

When set this class will be used when a formula starts.

\stopnewprimitive

\startnewprimitive[title={\prm {mathlimitsmode}}]

When this parameter is set to a value larger than zero real dimensions are used
and longer limits will not stick out, which is a traditional \TEX\ feature. We
could have more advanced control but this will do.

\startbuffer
\dontleavehmode
\ruledhbox{\dm{\left|\integral^{!!!!!!!!!!!}\right|}}
\ruledhbox{\dm{\left|\integral^{!!!!!!!!!!} \right|}}
\ruledhbox{\dm{\left|\integral^{!!!!!!!!!}  \right|}}
\ruledhbox{\dm{\left|\integral^{!!!!!!!!}   \right|}}
\ruledhbox{\dm{\left|\integral^{!!!!!!!}    \right|}}
\ruledhbox{\dm{\left|\integral^{!!!!!!}     \right|}}
\ruledhbox{\dm{\left|\integral^{!!!!!}      \right|}}
\ruledhbox{\dm{\left|\integral^{!!!!}       \right|}}
\ruledhbox{\dm{\left|\integral^{!!!}        \right|}}
\ruledhbox{\dm{\left|\integral^{!!}         \right|}}
\ruledhbox{\dm{\left|\integral^{!}          \right|}}
\blank
\dontleavehmode
\ruledhbox{\dm{\left|\integral_{!!!!!!!!!!!}\right|}}
\ruledhbox{\dm{\left|\integral_{!!!!!!!!!!} \right|}}
\ruledhbox{\dm{\left|\integral_{!!!!!!!!!}  \right|}}
\ruledhbox{\dm{\left|\integral_{!!!!!!!!}   \right|}}
\ruledhbox{\dm{\left|\integral_{!!!!!!!}    \right|}}
\ruledhbox{\dm{\left|\integral_{!!!!!!}     \right|}}
\ruledhbox{\dm{\left|\integral_{!!!!!}      \right|}}
\ruledhbox{\dm{\left|\integral_{!!!!}       \right|}}
\ruledhbox{\dm{\left|\integral_{!!!}        \right|}}
\ruledhbox{\dm{\left|\integral_{!!}         \right|}}
\ruledhbox{\dm{\left|\integral_{!}          \right|}}
\blank
\dontleavehmode
\ruledhbox{\dm{\left|\integral_{!}      ^{!}              \right|}}
\ruledhbox{\dm{\left|\integral_{!!!!!!!}^{!}              \right|}}
\ruledhbox{\dm{\left|\integral_{!}      ^{!!!!!!!}        \right|}}
\ruledhbox{\dm{\left|\integral_{!}      ^{!!!!!!!!!!}     \right|}}
\ruledhbox{\dm{\left|\integral_{!!!!!!!}^{!!!!!!!!!}      \right|}}
\ruledhbox{\dm{\left|\integral_{!!!!!!!}^{!!!!!!!!!!!!!!!}\right|}}
\ruledhbox{\dm{\left|\integral^{\mtext{for demanding}}_{\mtext{integral freaks}}\right|}}
\stopbuffer

Compare the zero setting:

{\switchtobodyfont[modern]\setupmathoperators[integral][method=limits]\mathlimitsmode 0 \getbuffer}

with the positive variant:

{\switchtobodyfont[modern]\setupmathoperators[integral][method=limits]\mathlimitsmode 1 \getbuffer}

Here we switched to Latin Modern because it's font dependent how serious this
issue is. In Pagella all is fine in both modes.

\stopnewprimitive

\startnewprimitive[title={\prm {mathmainstyle}}]

This inspector returns the outermost math style (contrary to \prm {mathstyle}),
as we can see in the next examples where use these snippets:

\startbuffer
\def\foo{(\the\mathmainstyle,\the\mathstyle)}
\def\oof{\sqrt[\foo]{\foo}}
\def\ofo{\frac{\foo}{\foo}}
\def\fof{\mathchoice{\foo}{\foo}{\foo}{\foo}}
\stopbuffer

\typebuffer \getbuffer

When we use the regular math triggers we get this:

\startbuffer
$\displaystyle     \foo + \oof + \ofo$
$\textstyle        \foo + \oof + \ofo$
$\displaystyle     \foo + \fof$
$\textstyle        \foo + \fof$
$\scriptstyle      \foo + \fof$
$\scriptscriptstyle\foo + \fof$
\stopbuffer

\typebuffer

\startlines
\getbuffer
\stoplines

But we can also do this:

\startbuffer
\Ustartmathmode \displaystyle     \foo + \oof + \ofo \Ustopmathmode
\Ustartmathmode \textstyle        \foo + \oof + \ofo \Ustopmathmode
\Ustartmathmode \displaystyle     \foo + \fof \Ustopmathmode
\Ustartmathmode \textstyle        \foo + \fof \Ustopmathmode
\Ustartmathmode \scriptstyle      \foo + \fof \Ustopmathmode
\Ustartmathmode \scriptscriptstyle\foo + \fof \Ustopmathmode
\stopbuffer

\typebuffer

\startlines
\getbuffer
\stoplines

\stopnewprimitive

\startnewprimitive[title={\prm {mathnolimitsmode}}]

This parameter influences the placement of scripts after an operator. The reason
we have this lays in the fact that traditional \TEX\ uses italic correction and
\OPENTYPE\ math does the same but fonts are not consistent in how they set this
up. Actually, in \OPENTYPE\ math it's the only reason that there is italic
correction. Say that we have a shift $\delta$ determined by the italic correction:

\starttabulate[|c|c|c|]
\BC mode  \BC top                \BC bottom                \NC \NR
\NC $0$   \NC $0$                \NC $-\delta$             \NC \NR
\NC $1$   \NC $\delta\times f_t$ \NC $\delta\times f_b$    \NC \NR
\NC $2$   \NC $0$                \NC $0$                   \NC \NR
\NC $3$   \NC $0$                \NC $-\delta/2$           \NC \NR
\NC $4$   \NC $\delta/2$         \NC $-\delta/2$           \NC \NR
\NC $>15$ \NC $0$                \NC $-n\times\delta/1000$ \NC \NR
\stoptabulate

Mode~1 uses two font parameters: $f_b$: \prm {Umathnolimitsubfactor} and $f_t$:
\prm {Umathnolimitsupfactor}.

\stopnewprimitive

\startoldprimitive[title={\prm {mathop}}]

This operation wraps following content in a atom with class \quote {operator}.

\stopoldprimitive

\startoldprimitive[title={\prm {mathopen}}]

This operation wraps following content in a atom with class \quote {open}.

\stopoldprimitive

\startoldprimitive[title={\prm {mathord}}]

This operation wraps following content in a atom with class \quote {ordinary}.

\stopoldprimitive

\startnewprimitive[title={\prm {mathparentstyle}}]

This inspector returns the math style used in a construct, so is is either
equivalent to \prm {mathmainstyle} or a nested \prm {mathstyle}. For instance
in a nested fraction we get this (in \CONTEXT) in display formulas:

\def\foo{(\the\mathmainstyle,\the\mathparentstyle,\the\mathstyle)}

\dm{
    \frac{\frac{\foo}{\foo}}{\frac{\foo}{\foo}} + \foo
}

but this in inline formulas:

\im{
    \frac{\frac{\foo}{\foo}}{\frac{\foo}{\foo}} + \foo
}

where the first element in a nested fraction.

\stopnewprimitive

\startnewprimitive[title={\prm {mathpenaltiesmode}}]

Normally the \TEX\ math engine only inserts penalties when in textstyle. You can
force penalties in displaystyle with this parameter. In inline math we always
honor penalties, with mode~0 and mode~1 we get this:

\pushoverloadmode

\startlines
{\mathpenaltiesmode=0 \let\mathpenaltiesmode\scratchcounter $\showmakeup[hpenalty]x+2x=0$}
{\mathpenaltiesmode=1 \let\mathpenaltiesmode\scratchcounter $\showmakeup[hpenalty]x+2x=1$}
\stoplines

However in \CONTEXT, where all is done in inline math mode, we set this this
parameter to~1, otherwise we wouldn't get these penalties, as shown next:

{\mathpenaltiesmode=0 \let\mathpenaltiesmode\scratchcounter \startformula \showmakeup[hpenalty]x+2x=0 \stopformula}
{\mathpenaltiesmode=1 \let\mathpenaltiesmode\scratchcounter \startformula \showmakeup[hpenalty]x+2x=1 \stopformula}

\popoverloadmode

If one uses a callback it is possible to force penalties from there too.

\stopnewprimitive

\startnewprimitive[title={\prm {mathpretolerance}}]

This is used instead of \prm {pretolerance} when a breakpoint is calculated when
a math formula starts.

\stopnewprimitive

\startoldprimitive[title={\prm {mathpunct}}]

This operation wraps following content in a atom with class \quote {punctuation}.

\stopoldprimitive

\startoldprimitive[title={\prm {mathrel}}]

This operation wraps following content in a atom with class \quote {relation}.

\stopoldprimitive

\startnewprimitive[title={\prm {mathrightclass}}]

When set this class will be used when a formula ends.

\stopnewprimitive

\startnewprimitive[title={\prm {mathrulesfam}}]

When set, this family will be used for setting rule properties in fractions,
under and over.

\stopnewprimitive

\startnewprimitive[title={\prm {mathrulesmode}}]

When set to a non zero value rules (as in fractions and radicals) will be based
on the font parameters in the current family.

\stopnewprimitive

\startnewprimitive[title={\prm {mathscale}}]

In \LUAMETATEX\ we can either have a family of three (text, script and
scriptscript) fonts or we can use one font that we scale and where we also pass
information about alternative shapes for the smaller sizes. When we use this
more compact mode this primitive reflects the scale factor used.

\startbuffer
\im {
    \textstyle        \the\mathscale\textfont        \fam\enspace
    \scriptstyle      \the\mathscale\scriptfont      \fam\enspace
    \scriptscriptstyle\the\mathscale\scriptscriptfont\fam\enspace
    \textstyle        \the\mathscale\textfont        \fam\enspace
    \scriptstyle      \the\mathscale\textfont        \fam\enspace
    \scriptscriptstyle\the\mathscale\textfont        \fam

}
\stopbuffer

What gets reported depends on how math is implemented, where in \CONTEXT\ we can
have either normal or compact mode: \inlinebuffer. In compact mode we have the
same font three times so then it doesn't matter which of the three is passed.

\stopnewprimitive

\startnewprimitive[title={\prm {mathscriptsmode}}]

There are situations where you don't want \TEX\ to be clever and optimize the
position of super- and subscripts by shifting. This parameter can be used to
influence this.

\startlinecorrection
\startcombination[3*2]
  {\startoverlay
     {\ruledhbox     {\mathscriptsmode 0 0: $x^2_2 + y^x_x + z_2 + w^2$}}
     {\ruledhbox{\red \mathscriptsmode 1 1: $x^2_2 + y^x_x + z_2 + w^2$}}
   \stopoverlay} {}
  {\startoverlay
     {\ruledhbox     {\mathscriptsmode 0 0: $x^2_2 + y^x_x + z_2 + w^2$}}
     {\ruledhbox{\red \mathscriptsmode 2 2: $x^2_2 + y^x_x + z_2 + w^2$}}
   \stopoverlay} {}
  {\startoverlay
     {\ruledhbox     {\mathscriptsmode 1 1: $x^2_2 + y^x_x + z_2 + w^2$}}
     {\ruledhbox{\red \mathscriptsmode 2 2: $x^2_2 + y^x_x + z_2 + w^2$}}
   \stopoverlay} {}
  {\startoverlay
     {\ruledhbox     {\mathscriptsmode 0 0: $x^f_f + y^x_x + z_f + w^f$}}
     {\ruledhbox{\red \mathscriptsmode 1 1: $x^f_f + y^x_x + z_f + w^f$}}
   \stopoverlay} {1 over 0}
  {\startoverlay
     {\ruledhbox     {\mathscriptsmode 0 0: $x^f_f + y^x_x + z_f + w^f$}}
     {\ruledhbox{\red \mathscriptsmode 2 2: $x^f_f + y^x_x + z_f + w^f$}}
   \stopoverlay} {2 over 0}
  {\startoverlay
     {\ruledhbox     {\mathscriptsmode 1 1: $x^f_f + y^x_x + z_f + w^f$}}
     {\ruledhbox{\red \mathscriptsmode 2 2: $x^f_f + y^x_x + z_f + w^f$}}
   \stopoverlay} {2 over 1}
\stopcombination
\stoplinecorrection

The next table shows what parameters kick in when:

\starttabulate[|l|l|l|p|]
\NC       \BC or (1)         \BC and (2)            \BC otherwise       \NC \NR
\BC super \NC sup shift up   \NC sup shift up       \NC sup shift up,
                                                        sup bot min     \NC \NR
\BC sub   \NC sub shift down \NC sub sup shift down \NC sub shift down,
                                                        sub top max     \NC \NR
\BC both  \NC sub shift down \NC sub sup shift down \NC sub sup shift down,
                                                        sub sup vgap,
                                                        sup sub bot max \NC \NR
\stoptabulate

\stopnewprimitive

\startnewprimitive[title={\prm {mathslackmode}}]

When positive this parameter will make sure that script spacing is discarded when
there is no reason to add it.

\startlinecorrection
\startcombination[nx=3,ny=1]
    {\mathslackmode0
     \ruledhbox{$\setmscale{3}x^2 + x^2$}\space
     \ruledhbox{$\setmscale{3}x^2$}}
    {disabled (0)}
    {\mathslackmode1
     \ruledhbox{$\setmscale{3}x^2 + x^2$}\space
     \ruledhbox{$\setmscale{3}x^2$}}
    {enabled (1)}
    {\startoverlay
       {\mathslackmode0     \ruledhbox{$\setmscale{3}x^2 + x^2$}}
       {\mathslackmode1 \red\ruledhbox{$\setmscale{3}x^2 + x^2$}}
     \stopoverlay
     \space
     \startoverlay
       {\mathslackmode0     \ruledhbox{$\setmscale{3}x^2$}}
       {\mathslackmode1 \red\ruledhbox{$\setmscale{3}x^2$}}
     \stopoverlay}
   {{\red enabled} over disabled}
\stopcombination
\stoplinecorrection

\stopnewprimitive

\startnewprimitive[title={\prm {mathspacingmode}}]

Zero inter|-|class glue is not injected but setting this parameter to a positive
value bypasses that check. This can be handy when checking (tracing) how (and
what) spacing is applied. Keep in mind that glue in math is special in the sense
that it is not a valid breakpoint. Line breaks in (inline) math are driven by
penalties.

\stopnewprimitive

\startnewprimitive[title={\prm {mathstack}}]

There are a few commands in \TEX\ that can behave confusing due to the way they
are scanned. Compare these:

\starttyping
$ 1 \over 2 $
$ 1 + x \over 2 + x$
$ {1 + x} \over {2 + x}$
$ {{1 + x} \over {2 + x}}$
\stoptyping

A single $1$ is an atom as is the curly braced ${1 + x}$. The two arguments to
\prm {over} eventually will get typeset in the style that this fraction
constructor uses for the numerator and denominator but on might actually also
like to relate that to the circumstances. It is comparable to using a \prm
{mathchoice}. In order not to waste runtime on four variants, which itself can have
side effects, for instance when counters are involved, \LUATEX\ introduced
\prm {mathstack}, used like:

\starttyping
$\mathstack {1 \over 2}$
\stoptyping

This \prm {mathstack} command will scan the next brace and opens a new math group
with the correct (in this case numerator) math style. The \prm {mathstackstyle}
primitive relates to this feature that defaults to \quote {smaller unless already
scriptscript}.

\stopnewprimitive

\startnewprimitive[title={\prm {mathstackstyle}}]

This returns the (normally) numerator style but the engine can be configured to
default to another style. Although all these in the original \TEX\ engines hard
coded style values can be changed in \LUAMETATEX\ it is unlikely to happen. So
this primitive will normally return the (current) style \quote {smaller unless
already scriptscript}.

\stopnewprimitive

\startnewprimitive[title={\prm {mathstyle}}]

This returns the current math style, so \type {$\the \mathstyle$} gives $\the
\mathstyle$.

\stopnewprimitive

\startnewprimitive[title={\prm {mathstylefontid}}]

This returns the font id (a number) of a style|/|family combination. What you get
back depends on how a macro package implements math fonts.

\startbuffer
(\the\mathstylefontid\textstyle        \fam)
(\the\mathstylefontid\scriptstyle      \fam)
(\the\mathstylefontid\scriptscriptstyle\fam)
\stopbuffer

\typebuffer

In \CONTEXT\ gives: \inlinebuffer.

\stopnewprimitive

\startoldprimitive[title={\prm {mathsurround}}]

The kern injected before and after an inline math formula. In practice it will be
set to zero, if only because otherwise nested math will also get that space
added. We also have \prm {mathsurroundskip} which, when set, takes precedence.
Spacing is controlled by \prm {mathsurroundmode}.

\stopoldprimitive

\startnewprimitive[title={\prm {mathsurroundmode}}]

The possible ways to control spacing around inline math formulas in other manuals
and mostly serve as playground.

\stopnewprimitive

\startnewprimitive[title={\prm {mathsurroundskip}}]

When set this one wins over \prm {mathsurround}.

\stopnewprimitive

\startnewprimitive[title={\prm {maththreshold}}]

This is a glue parameter. The amount determines what happens: when it is non zero
and the inline formula is less than that value it will become a special kind of
box that can stretch and|/| or shrink within the given specification. The par
builder will use these stretch and|/| or shrink components but it is up to one of
the \LUA\ callbacks to deal with the content eventually (if at all). As this is
somewhat specialized, more details can be found on \CONTEXT\ documentation.

\stopnewprimitive

\startnewprimitive[title={\prm {mathtolerance}}]

This is used instead of \prm {tolerance} when a breakpoint is calculated when a
math formula starts.

\stopnewprimitive

\startoldprimitive[title={\prm {maxdeadcycles}}]

When the output routine is called this many times and no page is shipped out an
error will be triggered. You therefore need to reset its companion counter \prm
{deadcycles} if needed. Keep in mind that \LUAMETATEX\ has no real \prm {shipout}
because providing a backend is up to the macro package.

\stopoldprimitive

\startoldprimitive[title={\prm {maxdepth}}]

The depth of the page is limited to this value.

\stopoldprimitive

\startoldprimitive[title={\prm {meaning}}]

We start with a primitive that will be used in the following sections. The
reported meaning can look a bit different than the one reported by other engines
which is a side effect of additional properties and more extensive argument
parsing.

\startbuffer
\tolerant\permanent\protected\gdef\foo[#1]#*[#2]{(#1)(#2)} \meaning\foo
\stopbuffer

\typebuffer \getbuffer

\stopoldprimitive

\startnewprimitive[title={\prm {meaningasis}}]

Although it is not really round trip with the original due to information
being lost this primitive tries to return an equivalent definition.

\startbuffer
\tolerant\permanent\protected\gdef\foo[#1]#*[#2]{(#1)(#2)} \meaningasis\foo
\stopbuffer

\typebuffer \getbuffer

\stopnewprimitive

\startnewprimitive[title={\prm {meaningful}}]

This one reports a bit less than \prm {meaningful}.

\startbuffer
\tolerant\permanent\protected\gdef\foo[#1]#*[#2]{(#1)(#2)} \meaningful\foo
\stopbuffer

\typebuffer \getbuffer

\stopnewprimitive

\startnewprimitive[title={\prm {meaningfull}}]

This one reports a bit more than \prm {meaning}.

\startbuffer
\tolerant\permanent\protected\gdef\foo[#1]#*[#2]{(#1)(#2)} \meaningfull\foo
\stopbuffer

\typebuffer \getbuffer

\stopnewprimitive

\startnewprimitive[title={\prm {meaningles}}]

This one reports a bit less than \prm {meaningless}.

\startbuffer
\tolerant\permanent\protected\gdef\foo[#1]#*[#2]{(#1)(#2)} \meaningles\foo
\stopbuffer

\typebuffer \getbuffer

\stopnewprimitive

\startnewprimitive[title={\prm {meaningless}}]

This one reports a bit less than \prm {meaning}.

\startbuffer
\tolerant\permanent\protected\gdef\foo[#1]#*[#2]{(#1)(#2)} \meaningless\foo
\stopbuffer

\typebuffer \getbuffer

\stopnewprimitive

\startoldprimitive[title={\prm {medmuskip}}]

A predefined mu skip register that can be used in math (inter atom) spacing. The
current value is {\tt \the\medmuskip}. In traditional \TEX\ most inter atom
spacing is hard coded using the predefined registers.

\stopoldprimitive

\startoldprimitive[title={\prm {message}}]

Prints the serialization of the (tokenized) argument to the log file and|/|or
console.

\stopoldprimitive

\startoldprimitive[title={\prm {middle}}]

Inserts the given delimiter as middle fence in a math formula. In \LUAMETATEX\ it
is a full blown fence and not (as in \ETEX) variation of \prm {open}.

\stopoldprimitive

\startoldprimitive[title={\prm {mkern}}]

This one injects a kern node in the current (math) list and expects a value in so
called mu units.

\stopoldprimitive

\startoldprimitive[title={\prm {month}}]

This internal number starts out with the month that the job started.

\stopoldprimitive

\startoldprimitive[title={\prm {moveleft}}]

This primitive takes two arguments, a dimension and a box. The box is moved to
the left. The operation only succeeds in vertical mode.

\stopoldprimitive

\startoldprimitive[title={\prm {moveright}}]

This primitive takes two arguments, a dimension and a box. The box is moved to
the right. The operation only succeeds in vertical mode.

\stopoldprimitive

\startoldprimitive[title={\prm {mskip}}]

The given math glue (in \type {mu} units) is injected in the horizontal list. For
this to succeed we need to be in math mode.

\stopoldprimitive

\startoldprimitive[title={\prm {muexpr}}]

This is a companion of \prm {glueexpr} so it handles the optional stretch and
shrink components. Here math units (\type {mu}) are expected.

\stopoldprimitive

\startnewprimitive[title={\prm {mugluespecdef}}]

A variant of \prm {gluespecdef} that expects \type {mu} units is:

\starttyping
\mugluespecdef\MyGlue = 3mu plus 2mu minus 1mu
\stoptyping

The properties are comparable to the ones described in the previous sections.

\stopnewprimitive

\startoldprimitive[title={\prm {multiply}}]

The given quantity is multiplied by the given integer (that can be preceded by
the keyword \quote {by}, like:

\starttyping
\scratchdimen=10pt \multiply\scratchdimen by 3
\stoptyping

\stopoldprimitive

\startnewprimitive[title={\prm {multiplyby}}]

This is slightly more efficient variant of \prm {multiply} that doesn't look for
\type {by}. See previous section.

\stopnewprimitive

\startoldprimitive[title={\prm {muskip}}]

This is the accessor for an indexed muskip (muglue) register.

\stopoldprimitive

\startoldprimitive[title={\prm {muskipdef}}]

This command associates a control sequence with a muskip (math skip) register
(accessed by number).

\stopoldprimitive

\startnewprimitive[title={\prm {mutable}}]

This prefix flags what follows can be adapted and is not subjected to overload
protection.

\stopnewprimitive

\startoldprimitive[title={\prm {mutoglue}}]

The sequence \typ {\the \mutoglue 20mu plus 10mu minus 5mu} gives \the \mutoglue
20mu plus 10mu minus 5mu.

\stopoldprimitive

\startnewprimitive[title={\prm {nestedloopiterator}}]

This is one of the accessors of loop iterators:

\startbuffer
\expandedrepeat 2 {%
    \expandedrepeat 3 {%
        (n=\the\nestedloopiterator  1,
         p=\the\previousloopiterator1,
         c=\the\currentloopiterator)
    }%
}%
\stopbuffer

\typebuffer

Gives:

\getbuffer

Where a nested iterator starts relative to innermost loop, the previous one is
relative to the outer loop (which is less predictable because we can already be
in a loop).

\stopnewprimitive

\startoldprimitive[title={\prm {newlinechar}}]

When something is printed to one of the log channels the character with this code
will trigger a linebreak. That also resets some counters that deal with
suppressing redundant ones and possible indentation. Contrary to other engines
\LUAMETATEX\ doesn't bother about the length of lines.

\stopoldprimitive

\startoldprimitive[title={\prm {noalign}}]

The token list passed to this primitive signals that we don't enter a table row
yet but for instance in a \prm {halign} do something between the lines: some
calculation or injecting inter-row material. In \LUAMETATEX\ this primitive can
be used nested.

\stopoldprimitive

\startnewprimitive[title={\prm {noaligned}}]

The alignment mechanism is kind of special when it comes to expansion because it
has to look ahead for a \prm {noalign}. This interferes with for instance
protected macros, but using this prefix we get around that. Among the reasons to
use protected macros inside an alignment is that they behave better inside for
instance \prm {expanded}.

\stopnewprimitive

\startnewprimitive[title={\prm {noatomruling}}]

Spacing in math is based on classes and this primitive inserts a signal that
there is no ruling in place here. Basically we have a zero skip glue tagged as
non breakable because in math mode glue is not a valid breakpoint unless we have
configured inter|-|class penalties.

\stopnewprimitive

\startnewprimitive[title={\prm {noboundary}}]

This inserts a boundary node with no specific property. It can still serve as
boundary but is not interpreted in special ways, like the others.

\stopnewprimitive

\startoldprimitive[title={\prm {noexpand}}]

This prefix prevents expansion in a context where expansion happens. Another way
to prevent expansion is to define a macro as \prm {protected}.

\startbuffer
          \def\foo{foo} \edef\oof{we expanded      \foo} \meaning\oof
          \def\foo{foo} \edef\oof{we keep \noexpand\foo} \meaning\oof
\protected\def\foo{foo} \edef\oof{we keep          \foo} \meaning\oof
\stopbuffer

\typebuffer

\startlines
\getbuffer
\stoplines

\stopoldprimitive

\startnewprimitive[title={\prm {nohrule}}]

This is a rule but flagged as empty which means that the dimensions kick in as
for a normal rule but the backend can decide not to show it.

\stopnewprimitive

\startoldprimitive[title={\prm {noindent}}]

This starts a paragraph. In \LUATEX\ (and \LUAMETATEX) a paragraph starts with a
so called par node (see \prm {indent} on how control that. After that comes
either \prm {parindent} glue or a horizontal box. The \prm {indent} makes gives
them some width, while \prm {noindent} keeps that zero.

\stopoldprimitive

\startoldprimitive[title={\prm {nolimits}}]

This is a modifier: it flags the previous math atom to have its scripts after the
the atom (contrary to \prm {limits}. In \LUAMETATEX\ this can be any atom (that
is: any class). In display mode the location defaults to above and below.

\stopoldprimitive

% \startnewprimitive[title={\prm {nomathchar}}]
% \stopnewprimitive

\startoldprimitive[title={\prm {nonscript}}]

This prevents \TEX\ from adding inter|-|atom glue at this spot in script or
scriptscript mode. It actually is a special glue itself that serves as signal.

\stopoldprimitive

\startoldprimitive[title={\prm {nonstopmode}}]

This directive omits all stops.

\stopoldprimitive

\startnewprimitive[title={\prm {norelax}}]

The rationale for this command can be shown by a few examples:

\startbuffer
\dimen0 1pt \dimen2 1pt \dimen4 2pt
\edef\testa{\ifdim\dimen0=\dimen2\norelax N\else Y\fi}
\edef\testb{\ifdim\dimen0=\dimen2\relax   N\else Y\fi}
\edef\testc{\ifdim\dimen0=\dimen4\norelax N\else Y\fi}
\edef\testd{\ifdim\dimen0=\dimen4\relax   N\else Y\fi}
\edef\teste{\norelax}
\stopbuffer

\typebuffer

The five meanings are:

\start \getbuffer \starttabulate[|T|T|]
\NC \string\testa \NC \meaning\testa \NC \NR
\NC \string\testb \NC \meaning\testb \NC \NR
\NC \string\testc \NC \meaning\testc \NC \NR
\NC \string\testd \NC \meaning\testd \NC \NR
\NC \string\teste \NC \meaning\teste \NC \NR
\stoptabulate \stop

So, the \prm {norelax} acts like \prm {relax} but is not pushed back as
usual (in some cases).

\stopnewprimitive

\startnewprimitive[title={\prm {normalizelinemode}}]

The \TEX\ engine was not designed to be opened up, and therefore the result of
the linebreak effort can differ depending on the conditions. For instance not
every line gets the left- or rightskip. The first and last lines have some unique
components too. When \LUATEX\ made it possible too get the (intermediate) result
manipulating the result also involved checking what one encountered, for instance
glue and its origin. In \LUAMETATEX\ we can normalize lines so that they have for
instance balanced skips.

\startcolumns[n=2]
\getbuffer[engine:syntax:normalizelinecodes]
\stopcolumns

The order in which the skips get inserted when we normalize is as follows:

\starttabulate
\NC \prm {lefthangskip}     \NC the hanging indentation (or zero) \NC \NR
\NC \prm {leftskip}         \NC the value even when zero \NC \NR
\NC \prm {parfillleftskip}  \NC only on the last line \NC \NR
\NC \prm {parinitleftskip}  \NC only on the first line \NC \NR
\NC \prm {indentskip}       \NC the amount of indentation \NC \NR
\NC \unknown                \NC the (optional) content \NC \NR
\NC \prm {parinitrightskip} \NC only on the first line \NC \NR
\NC \prm {parfillrightskip} \NC only on the last line \NC \NR
\NC \prm {correctionskip}   \NC the correction needed to stay within the \prm {hsize} \NC \NR
\NC \prm {rightskip}        \NC the value even when zero \NC \NR
\NC \prm {righthangskip}    \NC the hanging indentation (or zero) \NC \NR
\stoptabulate

The init and fill skips can both show up when we have a single line. The
correction skip replaces the traditional juggling with the right skip and shift
of the boxed line.

For now we leave the other options to your imagination. Some of these can be
achieved by callbacks (as we did in older versions of \CONTEXT) but having the
engine do the work we get a better performance.

\stopnewprimitive

\startnewprimitive[title={\prm {normalizeparmode}}]

For now we just mention the few options available. It is also worth mentioning that
\LUAMETATEX\ tries to balance the direction nodes.

\startcolumns[n=2]
\getbuffer[engine:syntax:normalizeparcodes]
\stopcolumns

\stopnewprimitive

\startnewprimitive[title={\prm {noscript}}]

In math we can have multiple pre- and postscript. These get typeset in pairs and
this primitive can be used to skip one. More about multiple scripts (and indices)
can be found in the \CONTEXT\ math manual.

\stopnewprimitive

\startnewprimitive[title={\prm {nospaces}}]

When \prm {nospaces} is set to~1 no spaces are inserted, when its value is~2 a
zero space is inserted. The default value is~0 which means that spaces become
glue with properties depending on the font, specific parameters and|/|or space
factors determined preceding characters. A value of~3 will inject a glyph node
with code \prm {spacechar}.

\stopnewprimitive

\startnewprimitive[title={\prm {nosubprescript}}]

This processes the given script in the current style, so:

\startbuffer
$ x___2 + x\nosubprescript{2} + x\subprescript{2} $
\stopbuffer

comes out as: \inlinebuffer.

\stopnewprimitive

\startnewprimitive[title={\prm {nosubscript}}]

This processes the given script in the current style, so:

\startbuffer
$ x_2 + x\nosubscript{2} + x\subscript{2} $
\stopbuffer

comes out as: \inlinebuffer.

\stopnewprimitive

\startnewprimitive[title={\prm {nosuperprescript}}]

This processes the given script in the current style, so:

\startbuffer
$ x^^^2 + x\nosuperprescript{2} + x\superprescript{2} $
\stopbuffer

comes out as: \inlinebuffer.

\stopnewprimitive

\startnewprimitive[title={\prm {nosuperscript}}]

This processes the given script in the current style, so:

\startbuffer
$ x^2 + x\nosuperprescript{2} + x\superprescript{2} $
\stopbuffer

comes out as: \inlinebuffer.

\stopnewprimitive

\startnewprimitive[title={\prm {novrule}}]

This is a rule but flagged as empty which means that the dimensions kick in as
for a normal rule but the backend can decide not to show it.

\stopnewprimitive

\startoldprimitive[title={\prm {nulldelimiterspace}}]

In fenced math delimiters can be invisible in which case this parameter
determines the amount of space (width) that ghost delimiter takes.

\stopoldprimitive

\startoldprimitive[title={\prm {nullfont}}]

This a symbolic reference to a font with no glyphs and a minimal set of font
dimensions.

\stopoldprimitive

\startoldprimitive[title={\prm {number}}]

This \TEX\ primitive serializes the next token into a number, assuming that it
is indeed a number, like

\starttyping
\number`A
\number65
\number\scratchcounter
\stoptyping

For counters and such the \prm {the} primitive does the same, but when you're
not sure if what follows is a verbose number or (for instance) a counter the
\prm {number} primitive is a safer bet, because \type {\the 65} will not work.

\stopoldprimitive

\startnewprimitive[title={\prm {numericscale}}]

This primitive can best be explained by a few examples:

\startbuffer
\the\numericscale 1323
\the\numericscale 1323.0
\the\numericscale 1.323
\the\numericscale 13.23
\stopbuffer

\typebuffer

In several places \TEX\ uses a scale but due to the lack of floats it then uses
1000 as 1.0 replacement. This primitive can be used for \quote {real} scales:

\startlines \getbuffer \stoplines

\stopnewprimitive

\startnewprimitive[title={\prm {numericscaled}}]

This is a variant if \prm {numericscale}:

\startbuffer
\scratchcounter 1000
\the\numericscaled 1323   \scratchcounter
\the\numericscaled 1323.0 \scratchcounter
\the\numericscaled 1.323  \scratchcounter
\the\numericscaled 13.23  \scratchcounter
\stopbuffer

\typebuffer

The second number gets multiplied by the first fraction:

\startlines \getbuffer \stoplines

\stopnewprimitive

\startoldprimitive[title={\prm {numexpr}}]

This primitive was introduced by \ETEX\ and supports a simple expression syntax:

\startbuffer
\the\numexpr 10 * (1 + 2 - 5) / 2 \relax
\stopbuffer

\typebuffer

gives: \inlinebuffer. You can mix in symbolic integers and dimensions.

\stopoldprimitive

\startnewprimitive[title={\prm {numexpression}}]

The normal \prm {numexpr} primitive understands the \type {+}, \type {-}, \type
{*} and \type {/} operators but in \LUAMETATEX\ we also can use \type {:} for a
non rounded integer division (think of \LUA's \type {//}). if you want more than
that, you can use the new expression primitive where you can use the following
operators.

\starttabulate[||cT|cT|]
\BC add       \NC +                    \NC        \NC \NR
\BC subtract  \NC -                    \NC        \NC \NR
\BC multiply  \NC *                    \NC        \NC \NR
\BC divide    \NC / :                  \NC        \NC \NR
\BC mod       \NC \letterpercent       \NC mod    \NC \NR
\BC band      \NC &                    \NC band   \NC \NR
\BC bxor      \NC ^                    \NC bxor   \NC \NR
\BC bor       \NC \letterbar \space v  \NC bor    \NC \NR
\BC and       \NC &&                   \NC and    \NC \NR
\BC or        \NC \letterbar\letterbar \NC or     \NC \NR
\BC setbit    \NC <undecided>          \NC bset   \NC \NR
\BC resetbit  \NC <undecided>          \NC breset \NC \NR
\BC left      \NC <<                   \NC        \NC \NR
\BC right     \NC >>                   \NC        \NC \NR
\BC less      \NC <                    \NC        \NC \NR
\BC lessequal \NC <=                   \NC        \NC \NR
\BC equal     \NC = ==                 \NC        \NC \NR
\BC moreequal \NC >=                   \NC        \NC \NR
\BC more      \NC >                    \NC        \NC \NR
\BC unequal   \NC <> != \lettertilde = \NC        \NC \NR
\BC not       \NC ! \lettertilde       \NC not    \NC \NR
\stoptabulate

An example of the verbose bitwise operators is:

\starttyping
\scratchcounter = \numexpression
    "00000 bor "00001 bor "00020 bor "00400 bor "08000 bor "F0000
\relax
\stoptyping

In the table you might have notices that some operators have equivalents. This
makes the scanner a bit less sensitive for catcode regimes.

When \prm {tracingexpressions} is set to one or higher the intermediate \quote
{reverse polish notation} stack that is used for the calculation is shown, for
instance:

\starttyping
4:8: {numexpression rpn: 2 5 > 4 5 > and}
\stoptyping

When you want the output on your console, you need to say:

\starttyping
\tracingexpressions 1
\tracingonline      1
\stoptyping

\stopnewprimitive

\startoldprimitive[title={\prm {omit}}]

This primitive cancels the template set for the upcoming cell. Often it is used
in combination with \prm {span}.

\stopoldprimitive

\startnewprimitive[title={\prm {optionalboundary}}]

This boundary is used to mark optional content. An positive \prm
{optionalboundary} starts a range and a zero one ends it. Nesting is not
supported. Optional content is considered when an additional paragraph pass
enables it as part of its recipe.

\stopnewprimitive

\startoldprimitive[title={\prm {or}}]

This traditional primitive is part of the condition testing mechanism and relates
to an \prm {ifcase} test (or a similar test to be introduced in later
sections). Depending on the value, \TEX\ will do a fast scanning till the right
\prm {or} is seen, then it will continue expanding till it sees a \prm {or}
or \prm {else} or \prm {orelse} (to be discussed later). It will then do a
fast skipping pass till it sees an \prm {fi}.

\stopoldprimitive

\startnewprimitive[title={\prm {orelse}}]

This primitive provides a convenient way to flatten your conditional tests. So
instead of

\starttyping
\ifnum\scratchcounter<-10
    too small
\else\ifnum\scratchcounter>10
    too large
\else
    just right
\fi\fi
\stoptyping

You can say this:

\starttyping
\ifnum\scratchcounter<-10
    too small
\orelse\ifnum\scratchcounter>10
    too large
\else
    just right
\fi
\stoptyping

You can mix tests and even the case variants will work in most cases \footnote {I
just play safe because there are corner cases that might not work yet.}

\starttyping
\ifcase\scratchcounter          zero
\or                             one
\or                             two
\orelse\ifnum\scratchcounter<10 less than ten
\else                           ten or more
\fi
\stoptyping

Performance wise there are no real benefits although in principle there is a bit
less housekeeping involved than with nested checks. However you might like this:

\starttyping
\ifnum\scratchcounter<-10
    \expandafter\toosmall
\orelse\ifnum\scratchcounter>10
    \expandafter\toolarge
\else
    \expandafter\justright
\fi
\stoptyping

over:

\starttyping
\ifnum\scratchcounter<-10
    \expandafter\toosmall
\else\ifnum\scratchcounter>10
    \expandafter\expandafter\expandafter\toolarge
\else
    \expandafter\expandafter\expandafter\justright
\fi\fi
\stoptyping

or the more \CONTEXT\ specific:

\starttyping
\ifnum\scratchcounter<-10
    \expandafter\toosmall
\else\ifnum\scratchcounter>10
    \doubleexpandafter\toolarge
\else
    \doubleexpandafter\justright
\fi\fi
\stoptyping

But then, some \TEX ies like complex and obscure code and throwing away working
old code that took ages to perfect and get working and also showed that one
masters \TEX\ might hurt.

\stopnewprimitive

\startnewprimitive[title={\prm {orphanpenalties}}]

This an (single entry) array parameter: first the size is given followed by that
amount of penalties. These penalties are injected before spaces, going backward
from the end of a paragraph. When we see a math node with a penalty set then we
take the max and jump over a (preceding) skip.

\stopnewprimitive

\startnewprimitive[title={\prm {orphanpenalty}}]

This penalty is inserted before the last space in a paragraph, unless \prm
{orphanpenalties} mandates otherwise.

\stopnewprimitive

\startnewprimitive[title={\prm {orunless}}]

This is the negated variant of \prm {orelse} (prefixing that one with \tex
{unless} doesn't work well.

\stopnewprimitive

\startoldprimitive[title={\prm {outer}}][obsolete=yes]

An outer macro is one that can only be used at the outer level. This property is
no longer supported. Like \prm {long}, the \prm {outer} prefix is now an
no|-|op (and we don't expect this to have unfortunate side effects).

\stopoldprimitive

\startoldprimitive[title={\prm {output}}]

This token list register holds the code that will be expanded when \TEX\ enters
the output routine. That code is supposed to do something with the content in
the box with number \prm {outputbox}. By default this is box 255 but that can be
changed with \prm {outputbox}.

\stopoldprimitive

\startnewprimitive[title={\prm {outputbox}}]

This is where the split off page contend ends up when the output routine is
triggered.

\stopnewprimitive

\startoldprimitive[title={\prm {outputpenalty}}]

This is the penalty that triggered the output routine.

\stopoldprimitive

\startoldprimitive[title={\prm {over}}][obsolete=yes]

This math primitive is actually a bit of a spoiler for the parser as it is one of
the few that looks back. The \prm {Uover} variant is different and takes two
arguments. We leave it to the user to predicts the results of:

\starttyping
$    {1} \over {x}    $
$     1  \over  x     $
$    12  \over  x / y $
$ a + 1  \over {x}    $
\stoptyping

and:

\starttyping
$  \textstyle 1  \over x  $
$ {\textstyle 1} \over x  $
$  \textstyle {1 \over x} $
\stoptyping

It's one of the reasons why macro packages provide \type {\frac}.

\stopoldprimitive

\startoldprimitive[title={\prm {overfullrule}}]

When an overfull box is encountered a rule can be shown in the margin and this
parameter sets its width. For the record: \CONTEXT\ does it different.

\stopoldprimitive

\startoldprimitive[title={\prm {overline}}]

This is a math specific primitive that draws a line over the given content. It is
a poor mans replacement for a delimiter. The thickness is set with \prm
{Umathoverbarrule}, the distance between content and rule is set by \prm
{Umathoverbarvgap} and \prm {Umathoverbarkern} is added above the rule. The style
used for the content under the rule can be set with \prm {Umathoverlinevariant}.

Because \CONTEXT\ set up math in a special way, the following example:

\startbuffer[demo]
\normaloverline {
    \blackrule[color=red, height=1ex,depth=0ex,width=2cm]%
    \kern-2cm
    \blackrule[color=blue,height=0ex,depth=.5ex,width=2cm]
    x + x
}
\stopbuffer

\typebuffer[demo]

gives: \ruledhbox {$\getbuffer[demo]$}, while:

\startbuffer[setup]
\mathfontcontrol\zerocount
\Umathoverbarkern\allmathstyles10pt
\Umathoverbarvgap\allmathstyles5pt
\Umathoverbarrule\allmathstyles2.5pt
\Umathoverlinevariant\textstyle\scriptstyle
\stopbuffer

\typebuffer[setup]

gives this: \ruledhbox {$\getbuffer[setup,demo]$}. We have to disable the related
\prm {mathfontcontrol} bits because otherwise the thickness is taken from the font. The
variant is just there to overload the (in traditional \TEX\ engines) default.

\stopoldprimitive

\startnewprimitive[title={\prm {overloaded}}]

This prefix can be used to overload a frozen macro.

\stopnewprimitive

\startnewprimitive[title={\prm {overloadmode}}]

The overload protection mechanism can be used to prevent users from redefining
a control sequence. The mode can have several values, the higher the more strict
we are:

\starttabulate[||||||||]
    \NC   \NC         \NC immutable \NC permanent \NC primitive \NC frozen \NC instance \NC \NR
    \NC 1 \NC warning \NC +         \NC +         \NC +         \NC        \NC          \NC \NR
    \NC 2 \NC error   \NC +         \NC +         \NC +         \NC        \NC          \NC \NR
    \NC 3 \NC warning \NC +         \NC +         \NC +         \NC +      \NC          \NC \NR
    \NC 4 \NC error   \NC +         \NC +         \NC +         \NC +      \NC          \NC \NR
    \NC 5 \NC warning \NC +         \NC +         \NC +         \NC +      \NC +        \NC \NR
    \NC 6 \NC error   \NC +         \NC +         \NC +         \NC +      \NC +        \NC \NR
\stoptabulate

When you set a high error value, you can of course temporary lower or even zero
the mode. In \CONTEXT\ all macros and quantities are tagged so there setting the
mode to~6 gives a proper protection against overloading. We need to zero the mode
when we load for instance tikz, so when you use that generic package, you loose
some.

\stopnewprimitive

\startnewprimitive[title={\prm {overshoot}}]

This primitive is a companion to \prm {badness} and reports how much a box
overflows.

\startbuffer
\setbox0\hbox to 1em {mmm} \the\badness\quad\the\overshoot
\setbox0\hbox         {mm} \the\badness\quad\the\overshoot
\setbox0\hbox to 3em   {m} \the\badness\quad\the\overshoot
\stopbuffer

\typebuffer

This reports:

\startlines
\getbuffer
\stoplines

When traditional \TEX\ wraps up the lines in a paragraph it uses a mix of shift
(a box property) to position the content suiting the hanging indentation and|/|or
paragraph shape, and fills up the line using right skip glue, also in order to
silence complaints in packaging. In \LUAMETATEX\ the lines can be normalized so
that they all have all possible skips to the left and right (even if they're
zero). The \prm {overshoot} primitive fits into this picture and is present as a
compensation glue. This all fits better in a situation where the internals are
opened up via \LUA.

\stopnewprimitive

\startoldprimitive[title={\prm {overwithdelims}}][obsolete=yes]

This is a variant of \prm {over} but with delimiters. It has a more advanced
upgrade in \prm {Uoverwithdelims}.

\stopoldprimitive

\startnewprimitive[title={\prm {pageboundary}}]

In order to avoid side effects of triggering the page builder with a specific
penalty we can use this primitive which expects a value that actually gets
inserted as zero penalty before triggering the page builder callback. Think of
adding a no|-|op to the contribution list. We fake a zero penalty so that all
gets processed. The main rationale is that we get a better indication of what we
do. Of course a callback can remove this node so that it is never seen.
Triggering from the callback is not doable. Consider this experimental code
(which is actually used in \CONTEXT\ anyway).

\stopnewprimitive

\startnewprimitive[title={\prm {pagedepth}}]

This page property holds the depth of the page.

\stopnewprimitive

\startoldprimitive[title={\prm {pagediscards}}]

The left|-|overs after a page is split of the main vertical list when glue and
penalties are normally discarded. The discards can be pushed back in (for
instance) trial runs.

\stopoldprimitive

\startnewprimitive[title={\prm {pageexcess}}]

This page property hold the amount of overflow when a page break occurs.

\stopnewprimitive

\startnewprimitive[title={\prm {pageextragoal}}]

This (experimental) dimension will be used when the page overflows but a bit of
overshoot is considered okay.

\stopnewprimitive

\startoldprimitive[title={\prm {pagefilllstretch}}]

The accumulated amount of third order stretch on the current page.

\stopoldprimitive

\startoldprimitive[title={\prm {pagefillstretch}}]

The accumulated amount of second order stretch on the current page.

\stopoldprimitive

\startoldprimitive[title={\prm {pagefilstretch}}]

The accumulated amount of first order stretch on the current page.

\stopoldprimitive

\startnewprimitive[title={\prm {pagefistretch}}]

The accumulated amount of zero order stretch on the current page.

\stopnewprimitive

\startoldprimitive[title={\prm {pagegoal}}]

The target height of a page (the running text). This value will be decreased by
the height of inserts something to keep into mind when messing around with this
and other (pseudo) page related parameters like \prm {pagetotal}.

\stopoldprimitive

\startnewprimitive[title={\prm {pagelastdepth}}]

The accumulated depth of the current page.

\stopnewprimitive

\startnewprimitive[title={\prm {pagelastfilllstretch}}]

The accumulated amount of third order stretch on the current page. Contrary to
\prm {pagefilllstretch} this is the really contributed amount, not the upcoming.

\stopnewprimitive

\startnewprimitive[title={\prm {pagelastfillstretch}}]

The accumulated amount of second order stretch on the current page. Contrary to
\prm {pagefillstretch} this is the really contributed amount, not the upcoming.

\stopnewprimitive

\startnewprimitive[title={\prm {pagelastfilstretch}}]

The accumulated amount of first order stretch on the current page. Contrary to
\prm {pagefilstretch} this is the really contributed amount, not the upcoming.

\stopnewprimitive

\startnewprimitive[title={\prm {pagelastfistretch}}]

The accumulated amount of zero order stretch on the current page. Contrary to
\prm {pagefistretch} this is the really contributed amount, not the upcoming.

\stopnewprimitive

\startnewprimitive[title={\prm {pagelastheight}}]

The accumulated height of the current page.

\stopnewprimitive

\startnewprimitive[title={\prm {pagelastshrink}}]

The accumulated amount of shrink on the current page. Contrary to \prm
{pageshrink} this is the really contributed amount, not the upcoming.

\stopnewprimitive

\startnewprimitive[title={\prm {pagelaststretch}}]

The accumulated amount of stretch on the current page. Contrary to \prm
{pagestretch} this is the really contributed amount, not the upcoming.

\stopnewprimitive

\startoldprimitive[title={\prm {pageshrink}}]

The accumulated amount of shrink on the current page.

\stopoldprimitive

\startoldprimitive[title={\prm {pagestretch}}]

The accumulated amount of stretch on the current page.

\stopoldprimitive

\startoldprimitive[title={\prm {pagetotal}}]

The accumulated page total (height) of the current page.

\stopoldprimitive

\startnewprimitive[title={\prm {pagevsize}}]

This parameter, when set, is used as the target page height. This lessens the
change of \prm {vsize} interfering.

\stopnewprimitive

\startoldprimitive[title={\prm {par}}]

This is the explicit \quote {finish paragraph} command. Internally we distinguish
a par triggered by a new line, as side effect of another primitive or this \prm
{par} command.

\stopoldprimitive

\startnewprimitive[title={\prm {parametercount}}]

The number of parameters passed to the current macro.

\stopnewprimitive

\startnewprimitive[title={\prm {parameterdef}}]

Here is an example of binding a variable to a parameter. The alternative is of
course to use an \prm {edef}.

\startbuffer
\def\foo#1#2%
  {\parameterdef\MyIndexOne\plusone % 1
   \parameterdef\MyIndexTwo\plustwo % 2
   \oof{P}\oof{Q}\oof{R}\norelax}

\def\oof#1%
  {<1:\MyIndexOne><1:\MyIndexOne>%
   #1%
   <2:\MyIndexTwo><2:\MyIndexTwo>}

\foo{A}{B}
\stopbuffer

\typebuffer

The outcome is:

\getbuffer

\stopnewprimitive

\startnewprimitive[title={\prm {parameterindex}}]

This gives the zero based position on the parameter stack. One reason for
introducing \prm {parameterdef} is that the position remains abstract so there we
don't need to use \prm {parameterindex}.

\stopnewprimitive

\startnewprimitive[title={\prm {parametermark}}]

This is an equivalent for \type {#}.

\stopnewprimitive

\startnewprimitive[title={\prm {parametermode}}]

Setting this internal integer to a positive value (best use~1 because future
versions might use bit set) will enable the usage of \type {#} for escaped in the
main text and body of macros.

\stopnewprimitive

\startnewprimitive[title={\prm {parattribute}}]

This primitive takes an attribute index and value and sets that attribute on the
current paragraph.

\stopnewprimitive

\startnewprimitive[title={\prm {pardirection}}]

This set the text direction for the whole paragraph which in the case of \type
{r2l} (1) makes the right edge the starting point.

\stopnewprimitive

\startnewprimitive[title={\prm {parfillleftskip}}]

The glue inserted at the start of the last line.

\stopnewprimitive

\startnewprimitive[title={\prm {parfillrightskip}}]

The glue inserted at the end of the last line (aka \prm {parfillskip}).

\stopnewprimitive

\startoldprimitive[title={\prm {parfillskip}}]

The glue inserted at the end of the last line.

\stopoldprimitive

\startoldprimitive[title={\prm {parindent}}]

The amount of space inserted at the start of the first line. When bit \tobit
\parindentskipnormalizecode\ is set in \prm {normalizelinemode} a glue is
inserted, otherwise an empty \prm {hbox} with the given width is inserted.

\stopoldprimitive

\startnewprimitive[title={\prm {parinitleftskip}}]

The glue inserted at the start of the first line.

\stopnewprimitive

\startnewprimitive[title={\prm {parinitrightskip}}]

The glue inserted at the end of the first line.

\stopnewprimitive

\startnewprimitive[title={\prm {parpasses}}]

Specifies one or more recipes for additional second linebreak passes. Examples
can be found in the \CONTEXT\ distribution.

\stopnewprimitive

\startoldprimitive[title={\prm {parshape}}]

Stores a shape specification. The first argument is the length of the list,
followed by that amount of indentation|-|width pairs (two dimensions).

\stopoldprimitive

\startoldprimitive[title={\prm {parshapedimen}}]

This oddly named (\ETEX) primitive returns the width component (dimension) of the
given entry (an integer). Obsoleted by \prm {parshapewidth}.

\stopoldprimitive

\startoldprimitive[title={\prm {parshapeindent}}]

Returns the indentation component (dimension) of the given entry (an integer).

\stopoldprimitive

\startoldprimitive[title={\prm {parshapelength}}]

Returns the number of entries (an integer).

\stopoldprimitive

\startnewprimitive[title={\prm {parshapewidth}}]

Returns the width component (dimension) of the given entry (an integer).

\stopnewprimitive

\startoldprimitive[title={\prm {parskip}}]

This is the amount of glue inserted before a new paragraph starts.

\stopoldprimitive

\startoldprimitive[title={\prm {patterns}}]

The argument to this primitive contains hyphenation patterns that are bound to
the current language. In \LUATEX\ and \LUAMETATEX\ we can also manage this at the
\LUA\ end. In \LUAMETATEX\ we don't store patterns in te format file

\stopoldprimitive

\startoldprimitive[title={\prm {pausing}}][obsolete=yes]

In \LUAMETATEX\ this variable is ignored but in other engines it can be used to
single step thought the input file by setting it to a positive value.

\stopoldprimitive

\startoldprimitive[title={\prm {penalty}}]

The given penalty (a number) is inserted at the current spot in the horizontal or
vertical list. We also have \prm {vpenalty} and \prm {hpenalty} that first change
modes.

\stopoldprimitive

\startnewprimitive[title={\prm {permanent}}]

This is one of the prefixes that is part of the overload protection mechanism. It
is normally used to flag a macro as being at the same level as a primitive: don't
touch it. primitives are flagged as such but that property cannot be set on
regular macros. The similar \prm {immutable} flag is normally used for variables.

\stopnewprimitive

\startnewprimitive[title={\prm {pettymuskip}}]

A predefined mu skip register that can be used in math (inter atom) spacing. The
current value is {\tt \the\pettymuskip}. This one complements \prm {thinmuskip},
\prm {medmuskip}, \prm {thickmuskip} and the new \prm {tinymuskip}.

\stopnewprimitive

\startnewprimitive[title={\prm {positdef}}]

The engine uses 32 bit integers for various purposes and has no (real) concept of
a floating point quantity. We get around this by providing a floating point data
type based on 32 bit unums (posits). These have the advantage over native floats
of more precision in the lower ranges but at the cost of a software
implementation.

The \prm {positdef} primitive is the floating point variant of \prm {integerdef}
and \prm {dimensiondef}: an efficient way to implement named quantities other
than registers.

\startbuffer
\positdef     \MyFloatA 5.678
\positdef     \MyFloatB 567.8
[\the\MyFloatA] [\todimension\MyFloatA] [\tointeger\MyFloatA]
[\the\MyFloatB] [\todimension\MyFloatB] [\tointeger\MyFloatB]
\stopbuffer

\typebuffer

For practical reasons we can map posit (or float) onto an integer or dimension:

\startlines
\getbuffer
\stoplines

% {\em I might eventually decide to go for 32 bit floats but it all depends on how
% unums evolve cq. become native to \CCODE.}

\stopnewprimitive

\startoldprimitive[title={\prm {postdisplaypenalty}}]

This is the penalty injected after a display formula.

\stopoldprimitive

\startnewprimitive[title={\prm {postexhyphenchar}}]

This primitive expects a language number and a character code. A negative
character code is equivalent to ignore. In case of an explicit discretionary the
character is injected at the beginning of a new line.

\stopnewprimitive

\startnewprimitive[title={\prm {posthyphenchar}}]

This primitive expects a language number and a character code. A negative
character code is equivalent to ignore. In case of an automatic discretionary the
character is injected at the beginning of a new line.

\stopnewprimitive

\startnewprimitive[title={\prm {postinlinepenalty}}]

When set this penalty is inserted after an inline formula unless we have a short
formula and \prm {postshortinlinepenalty} is set.

\stopnewprimitive

\startnewprimitive[title={\prm {postshortinlinepenalty}}]

When set this penalty is inserted after a short inline formula. The criterium is
set by \prm {shortinlinemaththreshold} but only applied when it is enabled for
the class involved.

\stopnewprimitive

\startnewprimitive[title={\prm {prebinoppenalty}}]

This internal quantity is a compatibility feature because normally we will use
the inter atom spacing variables.

\stopnewprimitive

\startoldprimitive[title={\prm {predisplaydirection}}]

This is the direction that the math sub engine will take into account when
dealing with right to left typesetting.

\stopoldprimitive

\startnewprimitive[title={\prm {predisplaygapfactor}}]

The heuristics related to determine if the previous line in a formula overlaps
with a (display) formula are hard coded but in \LUATEX\ to be two times the quad
of the current font. This parameter is a multiplier set to 2000 and permits you
to change the overshoot in this heuristic.

\stopnewprimitive

\startoldprimitive[title={\prm {predisplaypenalty}}]

This is the penalty injected before a display formula.

\stopoldprimitive

\startoldprimitive[title={\prm {predisplaysize}}]

This parameter holds the length of the last line in a paragraph when a display
formula is part of it.

\stopoldprimitive

\startnewprimitive[title={\prm {preexhyphenchar}}]

This primitive expects a language number and a character code. A negative
character code is equivalent to ignore. In case of an explicit discretionary the
character is injected at the end of the line.

\stopnewprimitive

\startnewprimitive[title={\prm {prehyphenchar}}]

This primitive expects a language number and a character code. A negative
character code is equivalent to ignore. In case of an automatic discretionary the
character is injected at the end of the line.

\stopnewprimitive

\startnewprimitive[title={\prm {preinlinepenalty}}]

When set this penalty is inserted before an inline formula unless we have a short
formula and \prm {preshortinlinepenalty} is set.

\stopnewprimitive

\startnewprimitive[title={\prm {prerelpenalty}}]

This internal quantity is a compatibility feature because normally we will use
the inter atom spacing variables.

\stopnewprimitive

\startnewprimitive[title={\prm {preshortinlinepenalty}}]

When set this penalty is inserted before a short inline formula. The criterium is
set by \prm {shortinlinemaththreshold} but only applied when it is enabled for
the class involved.

\stopnewprimitive

\startoldprimitive[title={\prm {pretolerance}}]

When the badness of a line in a paragraph exceeds this value a second linebreak
pass will be enabled.

\stopoldprimitive

\startoldprimitive[title={\prm {prevdepth}}]

The depth of current list. It can also be set to special (signal) values in order
to inhibit line corrections. It is not an internal dimension but a (current) list
property.

\stopoldprimitive

\startoldprimitive[title={\prm {prevgraf}}]

The number of lines in a previous paragraph.

\stopoldprimitive

\startnewprimitive[title={\prm {previousloopiterator}}]

\startbuffer
\edef\testA{
    \expandedrepeat 2 {%
        \expandedrepeat 3 {%
            (\the\previousloopiterator1:\the\currentloopiterator)
        }%
    }%
}
\edef\testB{
    \expandedrepeat 2 {%
        \expandedrepeat 3 {%
            (#P:#I) % #G is two levels up
        }%
    }%
}
\stopbuffer

\typebuffer \getbuffer

These give the same result:

\startlines \tt
\meaningasis\testA
\meaningasis\testB
\stoplines

The number indicates the number of levels we go up the loop chain.

\stopnewprimitive

\startnewprimitive[title={\prm {primescript}}]

This is a math script primitive dedicated to primes (which are somewhat
troublesome on math). It complements the six script primitives (like \prm
{subscript} and \prm {presuperscript}).

\stopnewprimitive

\startoldprimitive[title={\prm {protected}}]

A protected macro is one that doesn't get expanded unless it is time to do so.
For instance, inside an \prm {edef} it just stays what it is. It often makes
sense to pass macros as|-|is to (multi|-|pass) file (for tables of contents).

In \CONTEXT\ we use either \prm {protected} or \prm {unexpanded} because the
later was the command we used to achieve the same results before \ETEX\
introduced this protection primitive. Originally the \prm {protected} macro was
also defined but it has been dropped.

\stopoldprimitive

\startnewprimitive[title={\prm {protecteddetokenize}}]

This is a variant of \prm {protecteddetokenize} that uses some escapes encoded as
body parameters, like \type {#H} for a hash.

\stopnewprimitive

\startnewprimitive[title={\prm {protectedexpandeddetokenize}}]

This is a variant of \prm {expandeddetokenize} that uses some escapes encoded as
body parameters, like \type {#H} for a hash.

\stopnewprimitive

\startnewprimitive[title={\prm {protrudechars}}]

This variable controls protrusion (into the margin). A value~2 is comparable with
other engines, while a value of~3 does a bit more checking when we're doing
right|-|to|-|left typesetting.

\stopnewprimitive

\startnewprimitive[title={\prm {protrusionboundary}}]

This injects a boundary with the given value:

\getbuffer[engine:syntax:protrusionboundarycodes]

This signal makes the protrusion checker skip over a node.

\stopnewprimitive

\startnewprimitive[title={\prm {pxdimen}}]

The current numeric value of this dimension is \tointeger \pxdimen, \todimension
\pxdimen: one \type {bp}. We kept it around because it was introduced in \PDFTEX\
and made it into \LUATEX, where it relates to the resolution of included images.
In \CONTEXT\ it is not used.

\stopnewprimitive

\startnewprimitive[title={\prm {quitloop}}]

There are several loop primitives and they can be quit with \prm {quitloop} at
the next the {\em next} iteration. An immediate quit is possible with \prm
{quitloopnow}. An example is given with \prm {localcontrolledloop}.

\stopnewprimitive

\startnewprimitive[title={\prm {quitloopnow}}]

There are several loop primitives and they can be quit with \prm {quitloopnow}
at the spot.

\stopnewprimitive

\startnewprimitive[title={\prm {quitvmode}}]

This primitive forces horizontal mode but has no side effects when we're already
in that mode.

\stopnewprimitive

\startoldprimitive[title={\prm {radical}}][obsolete=yes]

This old school radical constructor is replaced by \prm {Uradical}. It takes a
number where the first byte is the small family, the next two index of this
symbol from that family, and the next three the family and index of the first
larger variant.

\stopoldprimitive

\startoldprimitive[title={\prm {raise}}]

This primitive takes two arguments, a dimension and a box. The box is moved up.
The operation only succeeds in horizontal mode.

\stopoldprimitive

\startnewprimitive[title={\prm {rdivide}}]

This is variant of \prm {divide} that rounds the result. For integers the result
is the same as \prm {edivide}.

\startbuffer
\the\dimexpr .4999pt                     : 2 \relax            =.24994pt
\the\dimexpr .4999pt                     / 2 \relax            =.24995pt
\the\dimexpr .4999pt                     ; 2 \relax            =.00002pt
\scratchdimen.4999pt \divide \scratchdimen 2 \the\scratchdimen =.24994pt
\scratchdimen.4999pt \edivide\scratchdimen 2 \the\scratchdimen =.24995pt
\scratchdimen 4999pt \rdivide\scratchdimen 2 \the\scratchdimen =2500.0pt
\scratchdimen 5000pt \rdivide\scratchdimen 2 \the\scratchdimen =2500.0pt

\the\numexpr   1001                       : 2 \relax             =500
\the\numexpr   1001                       / 2 \relax             =501
\the\numexpr   1001                       ; 2 \relax             =1
\scratchcounter1001 \divide \scratchcounter 2 \the\scratchcounter=500
\scratchcounter1001 \edivide\scratchcounter 2 \the\scratchcounter=501
\scratchcounter1001 \rdivide\scratchcounter 2 \the\scratchcounter=501
\stopbuffer

\typebuffer

\startlines
\getbuffer
\stoplines

The integer division \type{:} and modulo \type {;} are an addition to the \ETEX\
compatible expressions.

\stopnewprimitive

\startnewprimitive[title={\prm {rdivideby}}]

This is the \type {by}|-|less companion to \prm {rdivide}.

\stopnewprimitive

\startnewprimitive[title={\prm {realign}}]

Where \prm {omit} suspends a preamble template, this one overloads is for the
current table cell. It expects two token lists as arguments.

\stopnewprimitive

\startoldprimitive[title={\prm {relax}}]

This primitive does nothing and is often used to end a verbose number or
dimension in a comparison, for example:

\starttyping
\ifnum \scratchcounter = 123\relax
\stoptyping

which prevents a lookahead. A variant would be:

\starttyping
\ifnum \scratchcounter = 123 %
\stoptyping

assuming that spaces are not ignored. Another application is finishing
an expression like \prm {numexpr} or \prm {dimexpr}. I is also used
to prevent lookahead in cases like:

\starttyping
\vrule height 3pt depth 2pt width 5pt\relax
\hskip 5pt plus 3pt minus 2pt\relax
\stoptyping

Because \prm {relax} is not expandable the following:

\startbuffer
\edef\foo{\relax}   \meaningfull\foo
\edef\oof{\norelax} \meaningfull\oof
\stopbuffer

\typebuffer

gives this:

\startlines
\getbuffer
\stoplines

A \prm {norelax} disappears here but in the previously mentioned scenarios
it has the same function as \prm {relax}. It will not be pushed back either
in cases where a lookahead demands that.

\stopoldprimitive

\startoldprimitive[title={\prm {relpenalty}}]

This internal quantity is a compatibility feature because normally we will use
the inter atom spacing variables.

\stopoldprimitive

\startnewprimitive[title={\prm {resetlocalboxes}}]

Its purpose should be clear from the name.

\stopnewprimitive

\startnewprimitive[title={\prm {resetmathspacing}}]

This initializes all parameters to their initial values.

\stopnewprimitive

\startnewprimitive[title={\prm {restorecatcodetable}}]

This is an experimental feature that should be used with care. The next example
shows usage. It was added when debugging and exploring a side effect.

\starttyping
\tracingonline1

\bgroup

    \catcode`6 = 11 \catcode`7 = 11

    \bgroup

    \tracingonline1

    current: \the\catcodetable

    original: \the\catcode`6\quad \the\catcode`7

    \catcode`6 = 11 \catcode`7 = 11

    \showcodestack\catcode

    assigned: \the\catcode`6\quad \the\catcode`7

    \showcodestack\catcode

    \catcodetable\ctxcatcodes switched: \the\catcodetable

    stored: \the\catcode`6\quad \the\catcode`7

    \showcodestack\catcode

    \restorecatcodetable\ctxcatcodes

    \showcodestack\catcode

    restored: \the\catcode`6\quad \the\catcode`7

    \showcodestack\catcode

    \egroup

    \catcodetable\ctxcatcodes

    inner: \the\catcode`6\quad\the\catcode`7

\egroup

outer: \the\catcode`6\quad\the\catcode`7
\stoptyping

In \CONTEXT\ this typesets:

\starttyping
current: 9
original: 11 11
assigned: 11 11
switched: 9
stored: 11 11
restored: 12 12
inner: 11 11
outer; 12 12
\stoptyping

and on the console we see:

\starttyping
3:3: [codestack 1, size 3]
3:3: [1: level 2, code 54, value 12]
3:3: [2: level 2, code 55, value 12]
3:3: [3: level 3, code 54, value 11]
3:3: [4: level 3, code 55, value 11]
3:3: [codestack 1 bottom]
3:3: [codestack 1, size 3]
3:3: [1: level 2, code 54, value 12]
3:3: [2: level 2, code 55, value 12]
3:3: [3: level 3, code 54, value 11]
3:3: [4: level 3, code 55, value 11]
3:3: [codestack 1 bottom]
3:3: [codestack 1, size 3]
3:3: [1: level 2, code 54, value 12]
3:3: [2: level 2, code 55, value 12]
3:3: [3: level 3, code 54, value 11]
3:3: [4: level 3, code 55, value 11]
3:3: [codestack 1 bottom]
3:3: [codestack 1, size 7]
3:3: [1: level 2, code 54, value 12]
3:3: [2: level 2, code 55, value 12]
3:3: [3: level 3, code 54, value 11]
3:3: [4: level 3, code 55, value 11]
3:3: [5: level 3, code 55, value 11]
3:3: [6: level 3, code 54, value 11]
3:3: [7: level 3, code 55, value 11]
3:3: [8: level 3, code 54, value 11]
3:3: [codestack 1 bottom]
3:3: [codestack 1, size 7]
3:3: [1: level 2, code 54, value 12]
3:3: [2: level 2, code 55, value 12]
3:3: [3: level 3, code 54, value 11]
3:3: [4: level 3, code 55, value 11]
3:3: [5: level 3, code 55, value 11]
3:3: [6: level 3, code 54, value 11]
3:3: [7: level 3, code 55, value 11]
3:3: [8: level 3, code 54, value 11]
3:3: [codestack 1 bottom]
\stoptyping

So basically \prm {restorecatcodetable} brings us (temporarily) back to the
global settings.

\stopnewprimitive

\startnewprimitive[title={\prm {retained}}]

When a value is assigned inside a group \TEX\ pushes the current value on the save
stack in order to be able to restore the original value after the group has ended. You
can reach over a group by using the \prm {global} prefix. A mix between local and
global assignments can be achieved with the \prm {retained} primitive.

\newdimension\MyDim

\startbuffer[one]
\MyDim 15pt \bgroup \the\MyDim \space
\bgroup
    \bgroup
        \bgroup \advance\MyDim10pt \the\MyDim \egroup\space
        \bgroup \advance\MyDim10pt \the\MyDim \egroup\space
    \egroup
    \bgroup
        \bgroup \advance\MyDim10pt \the\MyDim \egroup\space
        \bgroup \advance\MyDim10pt \the\MyDim \egroup\space
    \egroup
\egroup
\egroup \the\MyDim
\stopbuffer

\startbuffer[two]
\MyDim 15pt \bgroup \the\MyDim \space
\bgroup
    \bgroup
        \bgroup \global\advance\MyDim10pt \the\MyDim \egroup\space
        \bgroup \global\advance\MyDim10pt \the\MyDim \egroup\space
    \egroup
    \bgroup
        \bgroup \global\advance\MyDim10pt \the\MyDim \egroup\space
        \bgroup \global\advance\MyDim10pt \the\MyDim \egroup\space
    \egroup
\egroup
\egroup \the\MyDim
\stopbuffer

\startbuffer[three]
\MyDim 15pt \bgroup \the\MyDim \space
    \constrained\MyDim\zeropoint
    \bgroup
        \bgroup \retained\advance\MyDim10pt \the\MyDim \egroup\space
        \bgroup \retained\advance\MyDim10pt \the\MyDim \egroup\space
    \egroup
    \bgroup
        \bgroup \retained\advance\MyDim10pt \the\MyDim \egroup\space
        \bgroup \retained\advance\MyDim10pt \the\MyDim \egroup\space
    \egroup
\egroup \the\MyDim
\stopbuffer

\typebuffer[one,two,three]

These lines result in:

\startlines
\hbox{\getbuffer[one]}
\hbox{\getbuffer[two]}
\hbox{\getbuffer[three]}
\stoplines

Because \LUAMETATEX\ avoids redundant stack entries and reassignments this
mechanism is a bit fragile but the \prm {constrained} prefix makes sure that we
do have a stack entry. If it is needed depends on the usage pattern.

\stopnewprimitive

\startnewprimitive[title={\prm {retokenized}}]

This is a companion of \prm {tokenized} that accepts a catcode table, so the
whole repertoire is:

\startbuffer
\tokenized                             {test $x$ test: current}
\tokenized   catcodetable \ctxcatcodes {test $x$ test: context}
\tokenized   catcodetable \vrbcatcodes {test $x$ test: verbatim}
\retokenized              \ctxcatcodes {test $x$ test: context}
\retokenized              \vrbcatcodes {test $x$ test: verbatim}
\stopbuffer

\typebuffer

Here we pass the numbers known to \CONTEXT\ and get:

\startlines
\getbuffer
\stoplines

\stopnewprimitive

\startoldprimitive[title={\prm {right}}]

Inserts the given delimiter as right fence in a math formula.

\stopoldprimitive

\startoldprimitive[title={\prm {righthyphenmin}}]

This is the minimum number of characters before the first hyphen in a hyphenated
word.

\stopoldprimitive

\startnewprimitive[title={\prm {rightmarginkern}}]

The dimension returned is the protrusion kern that has been added (if at all) to
the left of the content in the given box.

\stopnewprimitive

\startoldprimitive[title={\prm {rightskip}}]

This skip will be inserted at the right of every line.

\stopoldprimitive

\startoldprimitive[title={\prm {romannumeral}}]

This converts a number into a sequence of characters representing a roman
numeral. Because the Romans had no zero, a zero will give no output, a fact that
is sometimes used for hacks and showing off ones macro coding capabilities. A
large number will for sure result in a long string because after thousand we
start duplicating.

\stopoldprimitive

\startnewprimitive[title={\prm {rpcode}}]

This is the companion of \prm {lpcode} (see there) and also takes three
arguments: font, character code and factor.

\stopnewprimitive

\startnewprimitive[title={\prm {savecatcodetable}}]

This primitive stores the currently set catcodes in the current table.

\stopnewprimitive

\startoldprimitive[title={\prm {savinghyphcodes}}]

When set to non|-|zero, this will trigger the setting of \prm {hjcode}s from \prm
{lccode}s for the current font. These codes determine what characters are taken
into account when hyphenating words.

\stopoldprimitive

\startoldprimitive[title={\prm {savingvdiscards}}]

When set to a positive value the page builder will store the discarded items
(like glues) so that they can later be retrieved and pushed back if needed with
\prm {pagediscards} or \prm {splitdiscards}.

\stopoldprimitive

\startnewprimitive[title={\prm {scaledemwidth}}]

Returns the current (font specific) emwidth scaled according to \prm {glyphscale}
and \prm {glyphxscale}.

\stopnewprimitive

\startnewprimitive[title={\prm {scaledexheight}}]

Returns the current (font specific) exheight scaled according to \prm {glyphscale}
and \prm {glyphyscale}.

\stopnewprimitive

\startnewprimitive[title={\prm {scaledextraspace}}]

Returns the current (font specific) extra space value scaled according to \prm
{glyphscale} and \prm {glyphxscale}.

\stopnewprimitive

\startnewprimitive[title={\prm {scaledfontcharba}}]

Returns the bottom accent position of the given font|-|character pair scaled
according to \prm {glyphscale} and \prm {glyphyscale}.

\stopnewprimitive

\startnewprimitive[title={\prm {scaledfontchardp}}]

Returns the depth of the given font|-|character pair scaled according to \prm
{glyphscale} and \prm {glyphyscale}.

\stopnewprimitive

\startnewprimitive[title={\prm {scaledfontcharht}}]

Returns the height of the given font|-|character pair scaled according to \prm
{glyphscale} and \prm {glyphyscale}.

\stopnewprimitive

\startnewprimitive[title={\prm {scaledfontcharic}}]

Returns the italic correction of the given font|-|character pair scaled according
to \prm {glyphscale} and \prm {glyphxscale}. This property is only real for
traditional fonts.

\stopnewprimitive

\startnewprimitive[title={\prm {scaledfontcharta}}]

Returns the top accent position of the given font|-|character pair scaled
according to \prm {glyphscale} and \prm {glyphxscale}.

\stopnewprimitive

\startnewprimitive[title={\prm {scaledfontcharwd}}]

Returns width of the given font|-|character pair scaled according to \prm
{glyphscale} and \prm {glyphxscale}.

\stopnewprimitive

\startnewprimitive[title={\prm {scaledfontdimen}}]

Returns value of a (numeric) font dimension of the given font|-|character pair
scaled according to \prm {glyphscale} and \prm {glyphxscale} and|/|or \prm
{glyphyscale}.

\stopnewprimitive

\startnewprimitive[title={\prm {scaledinterwordshrink}}]

Returns the current (font specific) shrink of a space value scaled according to
\prm {glyphscale} and \prm {glyphxscale}.

\stopnewprimitive

\startnewprimitive[title={\prm {scaledinterwordspace}}]

Returns the current (font specific) space value scaled according to \prm
{glyphscale} and \prm {glyphxscale}.

\stopnewprimitive

\startnewprimitive[title={\prm {scaledinterwordstretch}}]

Returns the current (font specific) stretch of a space value scaled according to
\prm {glyphscale} and \prm {glyphxscale}.

\stopnewprimitive

\startnewprimitive[title={\prm {scaledmathaxis}}]

This primitive returns the math axis of the given math style. It's a dimension.

\stopnewprimitive

\startnewprimitive[title={\prm {scaledmathemwidth}}]

Returns the emwidth of the given style scaled according to \prm {glyphscale} and
\prm {glyphxscale}.

\stopnewprimitive

\startnewprimitive[title={\prm {scaledmathexheight}}]

Returns the exheight of the given style scaled according to \prm {glyphscale} and
\prm {glyphyscale}.

\stopnewprimitive

\startnewprimitive[title={\prm {scaledmathstyle}}]

This command inserts a signal in the math list that tells how to scale the (upcoming)
part of the formula.

\startbuffer
$ x + {\scaledmathstyle900 x} + x$
\stopbuffer

\typebuffer

We get: \inlinebuffer. Of course using this properly demands integration in the macro
packages font system.

\stopnewprimitive

\startnewprimitive[title={\prm {scaledslantperpoint}}]

This primitive is equivalent to \typ {\scaledfontdimen1\font} where \quote
{scaled} means that we multiply by the glyph scales.

\stopnewprimitive

\startnewprimitive[title={\prm {scantextokens}}]

This primitive scans the input as if it comes from a file. In the next examples
the \prm {detokenize} primitive turns tokenized code into verbatim code that is
similar to what is read from a file.

\startbuffer
\edef\whatever{\detokenize{This is {\bf bold} and this is not.}}
\detokenize   {This is {\bf bold} and this is not.}\crlf
\scantextokens{This is {\bf bold} and this is not.}\crlf
\scantextokens{\whatever}\crlf
\scantextokens\expandafter{\whatever}\par
\stopbuffer

\typebuffer

This primitive does not have the end|-|of|-|file side effects of its precursor
\prm {scantokens}.

{\getbuffer}

\stopnewprimitive

\startoldprimitive[title={\prm {scantokens}}]

Just forget about this \ETEX\ primitive, just take the one in the next section.

\stopoldprimitive

\startoldprimitive[title={\prm {scriptfont}}]

This primitive is like \prm {font} but with a family number as (first) argument
so it is specific for math. It is the middle one of the three family members; its
relatives are \prm {textfont} and \prm {scriptscriptfont}.

\stopoldprimitive

\startoldprimitive[title={\prm {scriptscriptfont}}]

This primitive is like \prm {font} but with a family number as (first) argument
so it is specific for math. It is the smallest of the three family members; its
relatives are \prm {textfont} and \prm {scriptfont}.

\stopoldprimitive

\startoldprimitive[title={\prm {scriptscriptstyle}}]

One of the main math styles, normally one size smaller than \prm {scriptstyle}:
integer representation: \the\scriptscriptstyle.

\stopoldprimitive

\startoldprimitive[title={\prm {scriptspace}}][obsolete=yes]

The math engine will add this amount of space after subscripts and superscripts.
It can be seen as compensation for the often too small widths of characters (in
the traditional engine italic correction is used too). It prevents scripts from
running into what follows.

\stopoldprimitive

\startnewprimitive[title={\prm {scriptspaceafterfactor}}]

This is a (1000 based) multiplier for \prm {Umathspaceafterscript}.

\stopnewprimitive

\startnewprimitive[title={\prm {scriptspacebeforefactor}}]

This is a (1000 based) multiplier for \prm {Umathspacebeforescript}.

\stopnewprimitive

\startnewprimitive[title={\prm {scriptspacebetweenfactor}}]

This is a (1000 based) multiplier for \prm {Umathspacebetweenscript}.

\stopnewprimitive

\startoldprimitive[title={\prm {scriptstyle}}]

One of the main math styles, normally one size smaller than \prm {displaystyle}
and \prm {textstyle}; integer representation: \the\scriptstyle.

\stopoldprimitive

\startoldprimitive[title={\prm {scrollmode}}]

    This directive omits error stops.

\stopoldprimitive

\startnewprimitive[title={\prm {semiexpand}}]

This command expands the next macro when it is protected with \prm
{semprotected}. See that primitive there for an example.

\stopnewprimitive

\startnewprimitive[title={\prm {semiexpanded}}]

This command expands the tokens in the given list including the macros protected
by with \prm {semprotected}. See that primitive there for an example.

\stopnewprimitive

\startnewprimitive[title={\prm {semiprotected}}]

\startbuffer
              \def\TestA{A}
\semiprotected\def\TestB{B}
    \protected\def\TestC{C}

\edef\TestD{\TestA           \TestB           \TestC}
\edef\TestE{\TestA\semiexpand\TestB\semiexpand\TestC}
\edef\TestF{\TestA\expand    \TestB\expand    \TestC}

\edef\TestG{\normalexpanded    {\TestA\TestB\TestC}}
\edef\TestH{\normalsemiexpanded{\TestA\TestB\TestC}}
\stopbuffer

The working of this prefix can best be explained with an example. We define a few
macros first:

\typebuffer \getbuffer

The meaning of the macros that are made from the other three are:

\startbuffer
\meaningless\TestD
\meaningless\TestE
\meaningless\TestF
\meaningless\TestG
\meaningless\TestH
\stopbuffer

Here we use the \type {\normal..} variants because (currently) we still have the
macro with the \type {\expanded} in the \CONTEXT\ core.

\startlines \tttf \getbuffer \stoplines

\stopnewprimitive

\startoldprimitive[title={\prm {setbox}}]

This important primitive is used to set a box register. It expects a number and a
box, like \prm {hbox} or \prm {box}. There is no \type {\boxdef} primitive
(analogue to other registers) because it makes no sense but numeric registers or
equivalents are okay as register value.

\stopoldprimitive

\startnewprimitive[title={\prm {setdefaultmathcodes}}]

This sets the math codes of upper- and lowercase alphabet and digits and the
delimiter code of the period. It's not so much a useful feature but more just an
accessor to the internal initializer.

\stopnewprimitive

\startnewprimitive[title={\prm {setfontid}}]

Internally a font instance has a number and this number is what gets assigned to
a glyph node. You can get the number with \prm {fontid} an set it with \prm
{setfontid}.

\starttyping
\setfontid\fontid\font
\stoptyping

The code above shows both primitives and effectively does nothing useful but
shows the idea.

\stopnewprimitive

\startoldprimitive[title={\prm {setlanguage}}]

In \LUATEX\ and \LUAMETATEX\ this is equivalent to \prm {language} because we
carry the language in glyph nodes instead of putting triggers in the list.

\stopoldprimitive

\startnewprimitive[title={\prm {setmathatomrule}}]

The math engine has some built in logic with respect to neighboring atoms that
change the class. The following combinations are intercepted and remapped:

\starttabulate[|c|c|c|c|]
\BC old first   \BC old second  \NC new first   \NC new second  \NC \NR
\ML
\NC begin       \NC binary      \NC ordinary    \NC ordinary    \NC \NR
\NC             \NC             \NC             \NC             \NC \NR
\NC operator    \NC binary      \NC operator    \NC ordinary    \NC \NR
\NC open        \NC binary      \NC open        \NC ordinary    \NC \NR
\NC punctuation \NC binary      \NC punctuation \NC ordinary    \NC \NR
\NC             \NC             \NC             \NC             \NC \NR
\NC binary      \NC end         \NC ordinary    \NC ordinary    \NC \NR
\NC binary      \NC binary      \NC binary      \NC ordinary    \NC \NR
\NC binary      \NC close       \NC ordinary    \NC close       \NC \NR
\NC binary      \NC punctuation \NC ordinary    \NC punctuation \NC \NR
\NC binary      \NC relation    \NC ordinary    \NC relation    \NC \NR
\NC             \NC             \NC             \NC             \NC \NR
\NC relation    \NC binary      \NC relation    \NC ordinary    \NC \NR
\NC relation    \NC close       \NC ordinary    \NC close       \NC \NR
\NC relation    \NC punctuation \NC ordinary    \NC punctuation \NC \NR
\stoptabulate

You can change this logic if needed, for instance:

\starttyping
\setmathatomrule 1 2 \allmathstyles 1 1
\stoptyping

Keep in mind that the defaults are what users expect. You might set them up for
additional classes that you define but even then you probably clone an existing
class and patch its properties. Most extra classes behave like ordinary anyway.

\stopnewprimitive

\startnewprimitive[title={\prm {setmathdisplaypostpenalty}}]

This penalty is inserted after an item of a given class but only in inline math
when display style is used, for instance:

\starttyping
\setmathdisplayprepenalty 2 750
\stoptyping

\stopnewprimitive

\startnewprimitive[title={\prm {setmathdisplayprepenalty}}]

This penalty is inserted before an item of a given class but only in inline math
when display style is used, for instance:

\starttyping
\setmathdisplayprepenalty 2 750
\stoptyping

\stopnewprimitive

\startnewprimitive[title={\prm {setmathignore}}]

You can flag a math parameter to be ignored, like:

\starttyping
\setmathignore \Umathxscale             2
\setmathignore \Umathyscale             2
\setmathignore \Umathspacebeforescript  1
\setmathignore \Umathspacebetweenscript 1
\setmathignore \Umathspaceafterscript   1
\stoptyping

A value of two will not initialize the variable, so its old value (when set) is
kept. This is somewhat experimental and more options might show up.

\stopnewprimitive

\startnewprimitive[title={\prm {setmathoptions}}]

This primitive expects a class (number) and a bitset.

\testpage[2]

\startcolumns
\getbuffer[engine:syntax:classoptioncodes] % weird: one liner at end of page
\stopcolumns

\stopnewprimitive

\startnewprimitive[title={\prm {setmathpostpenalty}}]

This penalty is inserted after an item of a given class but only in inline math
when text, script or scriptscript style is used, for instance:

\starttyping
\setmathpostpenalty 2 250
\stoptyping

\stopnewprimitive

\startnewprimitive[title={\prm {setmathprepenalty}}]

This penalty is inserted before an item of a given class but only in inline math
when text, script or scriptscript style is used, for instance:

\starttyping
\setmathprepenalty 2 250
\stoptyping

\stopnewprimitive

\startnewprimitive[title={\prm {setmathspacing}}]

More details about this feature can be found in \CONTEXT\ but it boils down to
registering what spacing gets inserted between a pair of classes. It can be
defined per style or for a set of styles, like:

\starttyping
\inherited\setmathspacing
  \mathimplicationcode \mathbinarycode
  \alldisplaystyles \thickermuskip
\inherited\setmathspacing
  \mathradicalcode \mathmiddlecode
  \allunsplitstyles \pettymuskip
\stoptyping

Here the \prm {inherited} prefix signals that a change in for instance \prm
{pettymuskip} is reflected in this spacing pair. In \CONTEXT\ there is a lot of
granularity with respect to spacing and it took years of experimenting (and
playing with examples) to get at the current stage. In general users are not
invited to mess around too much with these values, although changing the bound
registers (here \prm {pettymuskip} and \type {thickermuskip}) is no problem as it
consistently makes related spacing pairs follow.

\stopnewprimitive

\startoldprimitive[title={\prm {sfcode}}]

You can set a space factor on a character. That factor is used when a space
factor is applied (as part of spacing). It is (mostly) used for adding a
different space (glue) after punctuation. In some languages different punctuation
has different factors.

\stopoldprimitive

\startnewprimitive[title={\prm {shapingpenaltiesmode}}]

Shaping penalties are inserted after the lines of a \prm {parshape} and
accumulate according to this mode, a bitset of:

\getbuffer[engine:syntax:shapingpenaltiescodes]

\stopnewprimitive

\startnewprimitive[title={\prm {shapingpenalty}}]

In order to prevent a \prm {parshape} to break in unexpected ways we can add a
dedicated penalty, specified by this parameter.

\stopnewprimitive

\startoldprimitive[title={\prm {shipout}}][obsolete=yes]

Because there is no backend, this is not supposed to be used. As in traditional
\TEX\ a box is grabbed but instead of it being processed it gets shown and then
wiped. There is no real benefit of turning it into a callback.

\stopoldprimitive

\startnewprimitive[title={\prm {shortinlinemaththreshold}}]

This parameter determines when an inline formula is considered to be short. This
criterium is used for for \prm {preshortinlinepenalty} and \prm
{postshortinlinepenalty}.

\stopnewprimitive

\startnewprimitive[title={\prm {shortinlineorphanpenalty}}]

Short formulas at the end of a line are normally not followed by something other
than punctuation. This penalty will discourage a break before a short inline
formula. In practice one can set this penalty to e.g. a relatively low 200 to get
the desired effect.

\stopnewprimitive

\startoldprimitive[title={\prm {show}}]

Prints to the console (and/or log) what the token after
it represents.

\stopoldprimitive

\startoldprimitive[title={\prm {showbox}}]

The given box register is shown in the log and on te console (depending on \prm
{tracingonline}. How much is shown depends on \prm {showboxdepth} and \prm
{showboxbreadth}. In \LUAMETATEX\ we show more detailed information than in the
other engines; some specific information is provided via callbacks.

\stopoldprimitive

\startoldprimitive[title={\prm {showboxbreadth}}]

This primitives determine how much of a box is shown when asked for or when
tracing demands it.

\stopoldprimitive

\startoldprimitive[title={\prm {showboxdepth}}]

This primitives determine how deep tracing a box goes into the box. Some boxes,
like the ones that has the assembled page.

\stopoldprimitive

\startnewprimitive[title={\prm {showcodestack}}]

This inspector is only useful for low level debugging and reports the current
state of for instance the current catcode table: \typ {\showcodestack \catcode}.
See \prm {restorecatcodes} for an example.

\stopnewprimitive

\startoldprimitive[title={\prm {showgroups}}]

This primitive reports the group nesting. At this spot we have a not so
impressive nesting:

\starttyping
2:3: simple group entered at line 9375:
1:3: semisimple group: \begingroup
0:3: bottomlevel
\stoptyping

\stopoldprimitive

\startoldprimitive[title={\prm {showifs}}]

This primitive will show the conditional stack in the log file or on the console
(assuming \prm {tracingonline} being non|-|zero). The shown data is different
from other engines because we have more conditionals and also support a more flat
nesting model

\stopoldprimitive

\startoldprimitive[title={\prm {showlists}}]

This shows the currently built list.

\stopoldprimitive

\startoldprimitive[title={\prm {shownodedetails}}]

When set to a positive value more details will be shown of nodes when applicable.
Values larger than one will also report attributes. What gets shown depends on
related callbacks being set.

\stopoldprimitive

\startoldprimitive[title={\prm {showstack}}]

This tracer is only useful for low level debugging of macros, for instance when
you run out of save space or when you encounter a performance hit.

\starttyping
  test\scratchcounter0 \showstack
 {test\scratchcounter1 \showstack}
{{test\scratchcounter1 \showstack}}
\stoptyping

reports

\starttyping
1:3: [savestack size 0]
1:3: [savestack bottom]

2:3: [savestack size 2]
2:3: [1: restore, level 1, cs \scratchcounter=integer 1]
2:3: [0: boundary, group 'bottomlevel', boundary 0, attrlist 3600, line 0]
2:3: [savestack bottom]

3:3: [savestack size 3]
3:3: [2: restore, level 1, cs \scratchcounter=integer 1]
3:3: [1: boundary, group 'simple', boundary 0, attrlist 3600, line 12]
3:3: [0: boundary, group 'bottomlevel', boundary 0, attrlist 3600, line 0]
3:3: [savestack bottom]
\stoptyping

while

\starttyping
  test\scratchcounter1 \showstack
 {test\scratchcounter1 \showstack}
{{test\scratchcounter1 \showstack}}
\stoptyping

shows this:

\starttyping
1:3: [savestack size 0]
1:3: [savestack bottom]

2:3: [savestack size 1]
2:3: [0: boundary, group 'bottomlevel', boundary 0, attrlist 3600, line 0]
2:3: [savestack bottom]

3:3: [savestack size 2]
3:3: [1: boundary, group 'simple', boundary 0, attrlist 3600, line 16]
3:3: [0: boundary, group 'bottomlevel', boundary 0, attrlist 3600, line 0]
3:3: [savestack bottom]
\stoptyping

Because in the second example the value of \type {\scratchcounter} doesn't really
change inside the group there is no need for a restore entry on the stack. In
\LUAMETATEX\ there are checks for that so that we consume less stack space. We
also store some states (like the line number and current attribute list pointer)
in a stack boundary.

\stopoldprimitive

\startoldprimitive[title={\prm {showthe}}]

Prints to the console (and/or log) the value of token after it.

\stopoldprimitive

\startoldprimitive[title={\prm {showtokens}}]

This command expects a (balanced) token list, like

\starttyping
\showtokens{a few tokens}
\stoptyping

Depending on what you want to see you need to expand:

\starttyping
\showtokens\expandafter{\the\everypar}
\stoptyping

which is equivalent to \typ {\showthe \everypar}. It is an \ETEX\ extension.

\stopoldprimitive

\startnewprimitive[title={\prm {singlelinepenalty}}]

This is a penalty that gets injected before a paragraph that has only one line.
It is a one|-|shot parameter, so like \prm {looseness} it only applies to the
upcoming (or current) paragraph.

\stopnewprimitive

\startoldprimitive[title={\prm {skewchar}}][obsolete=yes]

This is an (imaginary) character that is used in math fonts. The kerning pair
between this character and the current one determines the top anchor of a
possible accent. In \OPENTYPE\ there is a dedicated character property for this
(but for some reason not for the bottom anchor).

\stopoldprimitive

\startoldprimitive[title={\prm {skip}}]

This is the accessor for an indexed skip (glue) register.

\stopoldprimitive

\startoldprimitive[title={\prm {skipdef}}]

This command associates a control sequence with a skip register (accessed by number).

\stopoldprimitive

\startnewprimitive[title={\prm {snapshotpar}}]

There are many parameters involved in typesetting a paragraph. One complication
is that parameters set in the middle might have unpredictable consequences due to
grouping, think of:

\starttyping
text  text <some setting> text   text \par
text {text <some setting> text } text \par
\stoptyping

This makes in traditional \TEX\ because there is no state related to the current
paragraph. But in \LUATEX\ we have the initial so called par node that remembers
the direction as well as local boxes. In \LUAMETATEX\ we store way more when this
node is created. That means that later settings no longer replace the stored ones.

The \prm {snapshotpar} takes a bitset that determine what stored parameters get
updated to the current values.

\startcolumns[n=3]
\getbuffer[engine:syntax:frozenparcodes]
\stopcolumns

One such value covers multiple values, so for instance \type {skip} is good for
storing the current \prm {leftskip} and \prm {rightskip} values. More about this
feature can be found in the \CONTEXT\ documentation.

The list of parameters that gets reset after a paragraph is longer than for
\PDFTEX\ and \LUAMETATEX: \prm {emergencyleftskip}, \prm {emergencyrightskip}, \prm
{hangafter}, \prm {hangindent}, \prm {interlinepenalties}, \prm
{localbrokenpenalty}, \prm {localinterlinepenalty}, \prm {localpretolerance},
\prm {localtolerance}, \prm {looseness}, \prm {parshape} and \prm
{singlelinepenalty}.

\stopnewprimitive

\startnewprimitive[title={\prm {spacechar}}]

When \prm {nospaces} is set to~3 a glyph node with the character value of
this parameter is injected.

\stopnewprimitive

\startoldprimitive[title={\prm {spacefactor}}]

The space factor is a somewhat complex feature. When during scanning a character
is appended that has a \prm {sfcode} other than 1000, that value is saved. When
the time comes to insert a space triggered glue, and that factor is 2000 or more,
and when \prm {xspaceskip} is nonzero, that value is used and we're done.

If these criteria are not met, and \prm {spaceskip} is nonzero, that value is
used, otherwise the space value from the font is used. Now, it if the space factor
is larger than 2000 the extra space value from the font is added to the set value.
Next the engine is going to tweak the stretch and shrink if that value and in
\LUAMETATEX\ that can be done in different ways, depending on \prm {spacefactormode},
\prm {spacefactorstretchlimit} and \prm {spacefactorshrinklimit}.

First the stretch. When the set limit is 1000 or more and the saved space factor
is also 1000 or more, we multiply the stretch by the limit, otherwise the saved
space factor is used.

Shrink is done differently. When the shrink limit and space factor are both 1000
or more, we will scale the shrink component by the limit, otherwise we multiply
by the saved space factor but here we have three variants, determined by the
value of \prm {spacefactormode}.

In the first case, when the limit kicks in, a mode value~1 will multiply by limit
and divides by 1000. A value of~2 multiplies by 2000 and divides by the limit.
Other mode values multiply by 1000 and divide by the limit. When the limit is not
used, the same happens but with the saved space factor.

If this sounds complicated, here is what regular \TEX\ does: stretch is
multiplied by the factor and divided by 1000 while shrink is multiplied by 1000
and divided by the saved factor. The (new) mode driven alternatives are the
result of extensive experiments done in the perspective of enhancing the
rendering of inline math as well as additional par builder passes. For sure
alternative strategies are possible and we can always add more modes.

A better explanation of the default strategy around spaces can be found in (of
course) The \TEX book and \TEX\ by Topic.

\stopoldprimitive

\startnewprimitive[title={\prm {spacefactormode}}]

Its setting determines the way the glue components (currently only shrink) adapts
itself to the current space factor (determined by by the character preceding a
space).

\stopnewprimitive

\startnewprimitive[title={\prm {spacefactoroverload}}]

When set to value between zero and thousand, this value will be used when \TEX\
encounters a below thousand space factor situation (usually used to suppress
additional space after a period following an uppercase character which then gets
(often) a 999 space factor. This feature only kicks in when the overload flag is
set in the glyph options, so it can be applied selectively.

\stopnewprimitive

\startnewprimitive[title={\prm {spacefactorshrinklimit}}]

This limit is used when \prm {spacefactormode} is set. See \prm {spacefactor} for a
bit more explanation.

\stopnewprimitive

\startnewprimitive[title={\prm {spacefactorstretchlimit}}]

This limit is used when \prm {spacefactormode} is set. See \prm {spacefactor} for a
bit more explanation.

\stopnewprimitive

\startoldprimitive[title={\prm {spaceskip}}]

Normally the glue inserted when a space is encountered is taken from the font but
this parameter can overrule that.

\stopoldprimitive

\startoldprimitive[title={\prm {span}}]

This primitive combined two upcoming cells into one. Often it is used in
combination with \prm {omit}. However, in the preamble it forces the next token
to be expanded, which means that nested \prm {tabskips} and align content markers
are seen.

\stopoldprimitive

\startoldprimitive[title={\prm {splitbotmark}}][obsolete=yes]

This is a reference to the last mark on the currently split off box, it gives
back tokens.

\stopoldprimitive

\startoldprimitive[title={\prm {splitbotmarks}}]

This is a reference to the last mark with the given id (a number) on the
currently split off box, it gives back tokens.

\stopoldprimitive

\startoldprimitive[title={\prm {splitdiscards}}]

When a box is split off, items like glue are discarded. This internal register
keeps the that list so that it can be pushed back if needed.

\stopoldprimitive

\startoldprimitive[title={\prm {splitfirstmark}}][obsolete=yes]

This is a reference to the first mark on the currently split off box, it gives
back tokens.

\stopoldprimitive

\startoldprimitive[title={\prm {splitfirstmarks}}]

This is a reference to the first mark with the given id (a number) on the
currently split off box, it gives back tokens.

\stopoldprimitive

\startoldprimitive[title={\prm {splitmaxdepth}}]

The depth of the box that results from a \prm {vsplit}.

\stopoldprimitive

\startoldprimitive[title={\prm {splittopskip}}]

This is the amount of glue that is added to the top of a (new) split of part of a
box when \prm {vsplit} is applied.

\stopoldprimitive

\startnewprimitive[title={\prm {srule}}]

This inserts a rule with no width. When a \type {font} and a \type {char} are
given the height and depth of that character are taken. Instead of a font \type
{fam} is also accepted so that we can use it in math mode.

\stopnewprimitive

\startoldprimitive[title={\prm {string}}]

We mention this original primitive because of the one in the next section. It
expands the next token or control sequence as if it was just entered, so normally
a control sequence becomes a backslash followed by characters and a space.

\stopoldprimitive

\startnewprimitive[title={\prm {subprescript}}]

Instead of three or four characters with catcode \the\subscriptcatcode\ (\type
{__} or \type {____}) this primitive can be used. It will add the following
argument as lower left script to the nucleus.

\stopnewprimitive

\startnewprimitive[title={\prm {subscript}}]

Instead of one or two characters with catcode \the\superscriptcatcode\ (\type {_}
or \type {__}) this primitive can be used. It will add the following argument as
upper left script to the nucleus.

\stopnewprimitive

\startnewprimitive[title={\prm {superprescript}}]

Instead of three or four characters with catcode \the\superscriptcatcode\ (\type
{^^^} or \type {^^^^}) this primitive can be used. It will add the following
argument as upper left script to the nucleus.

\stopnewprimitive

\startnewprimitive[title={\prm {superscript}}]

Instead of one or two character with catcode \the\superscriptcatcode\ (\type {^}
or \type {^^})this primitive can be used. It will add the following argument as
upper right script to the nucleus.

\stopnewprimitive

\startnewprimitive[title={\prm {supmarkmode}}]

As in other languages, \TEX\ has ways to escape characters and get whatever
character needed into the input. By default multiple \type {^} are used for this.
The dual \type {^^} variant is a bit weird as it is not continuous but \type
{^^^^} and \type {^^^^^^} provide four or six byte hexadecimal references ot
characters. The single \type {^} is also used for superscripts but because we
support prescripts and indices we get into conflicts with the escapes.

When this internal quantity is set to zero, multiple \type {^}'s are interpreted
in the input and produce characters. Other values disable the multiple parsing in
text and|/|or math mode:

\startbuffer
\normalsupmarkmode0 $ X^58 \quad X^^58 $ ^^58
\normalsupmarkmode1 $ X^58 \quad X^^58 $ ^^58
\normalsupmarkmode2 $ X^58 \quad X^^58 $ % ^^58 : error
\stopbuffer

\typebuffer

In \CONTEXT\ we default to one but also have the \prm {catcode} set to \the
\catcode`^ and the \prm {amcode} to \the \amcode `^.

\startlines
\catcode`^=\superscriptcatcode % to make sure we handle it in math
\amcode `^=\superscriptcatcode
\getbuffer
\stoplines

\stopnewprimitive

\startnewprimitive[title={\prm {swapcsvalues}}]

Because we mention some \type {def} and \type {let} primitives here, it makes
sense to also mention a primitive that will swap two values (meanings). This one
has to be used with care. Of course that what gets swapped has to be of the same
type (or at least similar enough not to cause issues). Registers for instance
store their values in the token, but as soon as we are dealing with token lists
we also need to keep an eye on reference counting. So, to some extend this is
an experimental feature.

\stopnewprimitive

\startnewprimitive[title={\prm {tabsize}}]

This primitive can be used in the preamble of an alignment and sets the size of
a column, as in:

\startbuffer
\halign{%
    \aligncontent             \aligntab
    \aligncontent\tabsize 3cm \aligntab
    \aligncontent             \aligntab
    \aligncontent\tabsize 0cm \cr
    1  \aligntab 111\aligntab 1111\aligntab 11\cr
    222\aligntab 2  \aligntab 2222\aligntab 22\cr
}
\stopbuffer

\typebuffer

As with \prm {tabskip} you need to reset the value explicitly, so that is why we
get two wide columns:

\blank {\showboxes \getbuffer} \blank

\stopnewprimitive

\startoldprimitive[title={\prm {tabskip}}]

This traditional primitive can be used in the preamble of an alignment and sets the
space added between columns, for example:

\startbuffer
\halign{%
    \aligncontent             \aligntab
    \aligncontent\tabskip 3cm \aligntab
    \aligncontent             \aligntab
    \aligncontent\tabskip 0cm \cr
    1  \aligntab 111\aligntab 1111\aligntab 11\cr
    222\aligntab 2  \aligntab 2222\aligntab 22\cr
}
\stopbuffer

\typebuffer

You need to reset the skip explicitly, which is why we get it applied twice here:

\blank {\showboxes \getbuffer} \blank

\stopoldprimitive

\startnewprimitive[title={\prm {textdirection}}]

This set the text direction to \type {l2r} (0) or \type {r2l} (1). It also
triggers additional checking for balanced flipping in node lists.

\stopnewprimitive

\startoldprimitive[title={\prm {textfont}}]

This primitive is like \prm {font} but with a family number as (first) argument
so it is specific for math. It is the largest one of the three family members; its
relatives are \prm {scriptfont} and \prm {scriptscriptfont}.

\stopoldprimitive

\startoldprimitive[title={\prm {textstyle}}]

One of the main math styles; integer representation: \the\textstyle.

\stopoldprimitive

\startoldprimitive[title={\prm {the}}]

The \prm {the} primitive serializes the following token, when applicable:
integers, dimensions, token registers, special quantities, etc. The catcodes of
the result will be according to the current settings, so in \type {\the \dimen0},
the \type {pt} will have catcode \quote {letter} and the number and period will
become \quote {other}.

\stopoldprimitive

\startnewprimitive[title={\prm {thewithoutunit}}]

The \prm{the} primitive, when applied to a dimension variable, adds a \type {pt}
unit. because dimensions are the only traditional unit with a fractional part
they are sometimes used as pseudo floats in which case \prm {thewithoutunit} can
be used to avoid the unit. This is more convenient than stripping it off
afterwards (via an expandable macro).

\stopnewprimitive

\startoldprimitive[title={\prm {thickmuskip}}]

A predefined mu skip register that can be used in math (inter atom) spacing. The
current value is {\tt \the\thickmuskip}. In traditional \TEX\ most inter atom
spacing is hard coded using the predefined registers.

\stopoldprimitive

\startoldprimitive[title={\prm {thinmuskip}}]

A predefined mu skip register that can be used in math (inter atom) spacing. The
current value is {\tt \the\thinmuskip}. In traditional \TEX\ most inter atom
spacing is hard coded using the predefined registers.

\stopoldprimitive

\startoldprimitive[title={\prm {time}}]

This internal number starts out with minute (starting at midnight) that the job
started.

\stopoldprimitive

\startnewprimitive[title={\prm {tinymuskip}}]

A predefined mu skip register that can be used in math (inter atom) spacing. The
current value is {\tt \the\tinymuskip}. This one complements \prm {thinmuskip},
\prm {medmuskip}, \prm {thickmuskip} and the new \prm {pettymuskip}

\stopnewprimitive

\startnewprimitive[title={\prm {tocharacter}}]

The given number is converted into an \UTF-8 sequence. In \LUATEX\ this one is
named \type {\Uchar}.

\stopnewprimitive

\startnewprimitive[title={\prm {toddlerpenalty}}]

This penalty controls line breaks after a single glyph. A high value prevents
single character at the end of a line.

\stopnewprimitive

\startnewprimitive[title={\prm {todimension}}]

\startbuffer
\scratchdimen = 1234pt \todimension\scratchdimen
\stopbuffer

The following code gives this: {\nospacing\inlinebuffer} and like its numeric
counterparts accepts anything that resembles a number this one goes beyond
(user, internal or pseudo) registers values too.

\typebuffer

\stopnewprimitive

\startnewprimitive[title={\prm {tohexadecimal}}]

\startbuffer
\scratchcounter = 1234 \tohexadecimal\scratchcounter
\stopbuffer

The following code gives this: {\nospacing\inlinebuffer} with uppercase letters.

\typebuffer

\stopnewprimitive

\startnewprimitive[title={\prm {tointeger}}]

\startbuffer
\scratchcounter = 1234 \tointeger\scratchcounter
\stopbuffer

The following code gives this: {\nospacing\inlinebuffer} and is equivalent to
\prm {number}.

\typebuffer

\stopnewprimitive

\startnewprimitive[title={\prm {tokenized}}]

Just as \prm {expanded} has a counterpart \prm {unexpanded}, it makes sense to give
\prm {detokenize} a companion:

\startbuffer
\edef\foo{\detokenize{\inframed{foo}}}
\edef\oof{\detokenize{\inframed{oof}}}

\meaning\foo \crlf \dontleavehmode\foo

\edef\foo{\tokenized{\foo\foo}}

\meaning\foo \crlf \dontleavehmode\foo

\dontleavehmode\tokenized{\foo\oof}
\stopbuffer

\typebuffer {\tttf \getbuffer}

This primitive is similar to:

\starttyping
\def\tokenized#1{\scantextokens\expandafter{\normalexpanded{#1}}}
\stoptyping

and should be more efficient, not that it matters much as we don't use it that
much (if at all).

\stopnewprimitive

\startoldprimitive[title={\prm {toks}}]

This is the accessor of a token register so it expects a number or \prm
{toksdef}'d macro.

\stopoldprimitive

\startnewprimitive[title={\prm {toksapp}}]

One way to append something to a token list is the following:

\starttyping
\scratchtoks\expandafter{\the\scratchtoks more stuff}
\stoptyping

This works all right, but it involves a copy of what is already in \type
{\scratchtoks}. This is seldom a real issue unless we have large token lists and
many appends. This is why \LUATEX\ introduced:

\starttyping
\toksapp\scratchtoks{more stuff}
\toksapp\scratchtoksone\scratchtokstwo
\stoptyping

At some point, when working on \LUAMETATEX, I realized that primitives like this
one and the next appenders and prependers to be discussed were always on the
radar of Taco and me. Some were even implemented in what we called \type {eetex}:
extended \ETEX, and we even found back the prototypes, dating from pre|-|\PDFTEX\
times.

\stopnewprimitive

\startoldprimitive[title={\prm {toksdef}}]

The given name (control sequence) will be bound to the given token register (a
number). Often this primitive is hidden in a high level macro that manages
allocation.

\stopoldprimitive

\startnewprimitive[title={\prm {tokspre}}]

Where appending something is easy because of the possible \prm {expandafter}
trickery a prepend would involve more work, either using temporary token
registers and|/|or using a mixture of the (no)expansion added by \ETEX, but all
are kind of inefficient and cumbersome.

\starttyping
\tokspre\scratchtoks{less stuff}
\tokspre\scratchtoksone\scratchtokstwo
\stoptyping

This prepends the token list that is provided.

\stopnewprimitive

\startoldprimitive[title={\prm {tolerance}}]

When the par builder runs into a line with a badness larger than this value and
when \prm {emergencystretch} is set a third pass is enabled. In \LUAMETATEX\ we
can have more than one second pass and there are more parameters that influence
the process.

\stopoldprimitive

\startnewprimitive[title={\prm {tolerant}}]

This prefix tags the following macro as being tolerant with respect to the
expected arguments. It only makes sense when delimited arguments are used or when
braces are mandate.

\startbuffer
\tolerant\def\foo[#1]#*[#2]{(#1)(#2)}
\stopbuffer

\typebuffer \getbuffer

This definition makes \type {\foo} tolerant for various calls:

\startbuffer
\foo \foo[1] \foo [1] \foo[1] [2] \foo [1] [2]
\stopbuffer

\typebuffer

these give: \inlinebuffer. The spaces after the first call disappear because the
macro name parser gobbles it, while in the second case the \type {#*} gobbles
them. Here is a variant:

\startbuffer
\tolerant\def\foo[#1]#,[#2]{!#1!#2!}

\foo[?] x
\foo[?] [?] x

\tolerant\def\foo[#1]#*[#2]{!#1!#2!}

\foo[?] x
\foo[?] [?] x
\stopbuffer

\typebuffer

We now get the following:

\getbuffer

Here the \type {#,} remembers that spaces were gobbles and they will be put back
when there is no further match. These are just a few examples of this tolerant
feature. More details can be found in the lowlevel manuals.

\stopnewprimitive

\startnewprimitive[title={\prm {tomathstyle}}]

Internally math styles are numbers, where \prm {displaystyle} is \tomathstyle
\displaystyle \space and \prm {crampedscriptscriptstyle} is \tomathstyle
\crampedscriptscriptstyle. You can convert the verbose style to a number with
\prm {tomathstyle}.

\stopnewprimitive

\startoldprimitive[title={\prm {topmark}}][obsolete=yes]

This is a reference to the last mark on the previous (split off) page, it gives
back tokens.

\stopoldprimitive

\startoldprimitive[title={\prm {topmarks}}]

This is a reference to the last mark with the given id (a number) on the previous
page, it gives back tokens.

\stopoldprimitive

\startoldprimitive[title={\prm {topskip}}]

This is the amount of glue that is added to the top of a (new) page.

\stopoldprimitive

\startnewprimitive[title={\prm {toscaled}}]

\startbuffer
\scratchdimen = 1234pt \toscaled\scratchdimen
\stopbuffer

The following code gives this: {\nospacing\inlinebuffer} is similar to \prm
{todimension} but omits the \type {pt} so that we don't need to revert to some
nasty stripping code.

\typebuffer

\stopnewprimitive

\startnewprimitive[title={\prm {tosparsedimension}}]

\startbuffer
\scratchdimen = 1234pt \tosparsedimension\scratchdimen
\stopbuffer

The following code gives this: {\nospacing\inlinebuffer} where \quote {sparse}
indicates that redundant trailing zeros are not shown.

\typebuffer

\stopnewprimitive

\startnewprimitive[title={\prm {tosparsescaled}}]

\startbuffer
\scratchdimen = 1234pt \tosparsescaled\scratchdimen
\stopbuffer

The following code gives this: {\nospacing\inlinebuffer} where \quote {sparse}
means that redundant trailing zeros are omitted.

\typebuffer

\stopnewprimitive

\startnewprimitive[title={\prm {tpack}}]

This primitive is like \prm {vtop} but without the callback overhead.

\stopnewprimitive

\startnewprimitive[title={\prm {tracingadjusts}}]

In \LUAMETATEX\ the adjust feature has more functionality and also is carried
over. When set to a positive values \prm {vadjust} processing reports details.
The higher the number, the more you'll get.

\stopnewprimitive

\startnewprimitive[title={\prm {tracingalignments}}]

When set to a positive value the alignment mechanism will keep you informed about
what is done in various stages. Higher values unleash more information, including
what callbacks kick in.

\stopnewprimitive

\startoldprimitive[title={\prm {tracingassigns}}]

When set to a positive values assignments to parameters and variables are
reported on the console and|/|or in the log file. Because \LUAMETATEX\ avoids
redundant assignments these don't get reported.

\stopoldprimitive

\startoldprimitive[title={\prm {tracingcommands}}]

When set to a positive values the commands (primitives) are reported on the console
and|/|or in the log file.

\stopoldprimitive

\startnewprimitive[title={\prm {tracingexpressions}}]

The extended expression commands like \prm {numexpression} and \prm
{dimexpression} can be traced by setting this parameter to a positive value.

\stopnewprimitive

\startnewprimitive[title={\prm {tracingfitness}}]

Because we have more fitness classes we also have (need) a (bit) more detailed
tracing.

\stopnewprimitive

\startnewprimitive[title={\prm {tracingfullboxes}}]

When set to a positive value the box will be shown in case of an overfull box.
When a quality callback is set this will not happen as all reporting is then
delegated.

\stopnewprimitive

\startoldprimitive[title={\prm {tracinggroups}}]

When set to a positive values grouping is reported on the console and|/|or in the
log file.

\stopoldprimitive

\startnewprimitive[title={\prm {tracinghyphenation}}]

When set to a positive values the hyphenation process is reported on the console
and|/|or in the log file.

\stopnewprimitive

\startoldprimitive[title={\prm {tracingifs}}]

When set some details of what gets tested and what results are seen is reported.

\stopoldprimitive

\startnewprimitive[title={\prm {tracinginserts}}]

A positive value enables tracing where values larger than~1 will report more
details.

\stopnewprimitive

\startnewprimitive[title={\prm {tracinglevels}}]

The lines in a log file can be prefixed with some details, depending on the bits
set:

\starttabulate[|T|l|]
\NC 0x1 \NC current group \NC \NR
\NC 0x2 \NC current input \NC \NR
\NC 0x4 \NC catcode table \NC \NR
\stoptabulate

\stopnewprimitive

\startnewprimitive[title={\prm {tracinglists}}]

At various stages the lists being processed can be shown. This is mostly an
option for developers.

\stopnewprimitive

\startnewprimitive[title={\prm {tracingloners}}]

With loners we mean \quote {widow} and \quote {club} lines. This tracer can be
handy when \prm {doublepenaltymode} is set and facing pages have different
penalty values.

\stopnewprimitive

\startoldprimitive[title={\prm {tracinglostchars}}]

When set to one characters not present in a font will be reported in the log
file, a value of two will also report this on the console.

\stopoldprimitive

\startoldprimitive[title={\prm {tracingmacros}}]

This parameter controls reporting of what macros are seen and expanded.

\stopoldprimitive

\startnewprimitive[title={\prm {tracingmarks}}]

Marks are information blobs that track states that can be queried when a page is
handled over to the shipout routine. They travel through the system in a bit
different than traditionally: like like adjusts and inserts deeply buried ones
bubble up to outer level boxes. This parameters controls what progress gets
reported.

\stopnewprimitive

\startnewprimitive[title={\prm {tracingmath}}]

The higher the value, the more information you will get about the various stages
in rendering math. Because tracing of nodes is rather verbose you need to know a
bit what this engine does. Conceptually there are differences between the
\LUAMETATEX\ and traditional engine, like more passes, inter-atom spacing,
different low level mechanisms. This feature is mostly meant for developers who
tweak the many available parameters.

\stopnewprimitive

\startoldprimitive[title={\prm {tracingnesting}}]

A positive value triggers log messages about the current level.

\stopoldprimitive

\startnewprimitive[title={\prm {tracingnodes}}]

When set to a positive value more details about nodes (in boxes) will be
reported. Because this is also controlled by callbacks what gets reported is
macro package dependent.

\stopnewprimitive

\startoldprimitive[title={\prm {tracingonline}}]

The engine has two output channels: the log file and the console and by default
most tracing (when enabled) goes to the log file. When this parameter is set to a
positive value tracing will also happen in the console. Messages from the \LUA\
end can be channeled independently.

\stopoldprimitive

\startoldprimitive[title={\prm {tracingoutput}}]

Values larger than one result in some information about what gets passed to the
output routine.

\stopoldprimitive

\startoldprimitive[title={\prm {tracingpages}}]

Values larger than one result in some information about the page building
process. In \LUAMETATEX\ there is more info for higher values.

\stopoldprimitive

\startoldprimitive[title={\prm {tracingparagraphs}}]

Values larger than one result in some information about the par building process.
In \LUAMETATEX\ there is more info for higher values.

\stopoldprimitive

\startnewprimitive[title={\prm {tracingpasses}}]

In \LUAMETATEX\ you can configure additional second stage par builder passes and
this parameter controls what gets reported on the console and|/|or in the log
file.

\stopnewprimitive

\startnewprimitive[title={\prm {tracingpenalties}}]

This setting triggers reporting of actions due to special penalties in the
page builder.

\stopnewprimitive

\startoldprimitive[title={\prm {tracingrestores}}]

When set to a positive values (re)assignments after grouping to parameters and
variables are reported on the console and|/|or in the log file. Because
\LUAMETATEX\ avoids redundant assignments these don't get reported.

\stopoldprimitive

\startoldprimitive[title={\prm {tracingstats}}]

This parameter is a dummy in \LUAMETATEX. There are anyway some statistic
reported when the format is made but for a regular run it is up to the macro
package to come up with useful information.

\stopoldprimitive

\startnewprimitive[title={\prm {tsplit}}]

This splits like \prm {vsplit} but it returns a \prm {vtop} box instead.

\stopnewprimitive

\startoldprimitive[title={\prm {uccode}}]

When the \prm {uppercase} operation is applied the uppercase code of a character
is used for the replacement. This primitive is used to set that code, so it
expects two character number.

\stopoldprimitive

\startoldprimitive[title={\prm {uchyph}}]

When set to a positive number words that start with a capital will be hyphenated.

\stopoldprimitive

\startnewprimitive[title={\prm {uleaders}}]

This leader adapts itself after a paragraph has been typeset. Here are a few
examples:

\startbuffer
test \leaders  \hbox      {x}\hfill\              test
test \uleaders \hbox{x x x x}\hfill\              test
test           \hbox{x x x x}\hskip 3cm plus 1cm\ test
test \uleaders \hbox{x x x x}\hskip 3cm plus 1cm\ test
\stopbuffer

\typebuffer

When an \prm {uleaders} is used the glue in the given box will be adapted to the
available space.

\startlines \getbuffer \stoplines

\startsetups adaptive:test
    \setbox\usedadaptivebox\hbox to \usedadaptivewidth yoffset -\usedadaptivedepth \bgroup
        \externalfigure
          [cow.pdf]
          [width=\usedadaptivewidth,
           height=\dimexpr\usedadaptiveheight+\usedadaptivedepth\relax]%
   \egroup
\stopsetups

Optionally the \type {callback} followed by a number can be given, in which case
a callback kicks in that gets that the node, a group identifier, and the number
passed. It permits (for instance) adaptive graphics: \dostepwiserecurse {1} {100} {5}
{\hbox {#1=\romannumerals{#1}} {\adaptivebox [strut=yes, setups=adaptive:test]{}} }.

\stopnewprimitive

\startoldprimitive[title={\prm {unboundary}}]

When possible a preceding boundary node will be removed.

\stopoldprimitive

\startnewprimitive[title={\prm {undent}}]

When possible the already added indentation will be removed.

\stopnewprimitive

\startoldprimitive[title={\prm {underline}}]

This is a math specific primitive that draws a line under the given content. It
is a poor mans replacement for a delimiter. The thickness is set with \prm
{Umathunderbarrule}, the distance between content and rule is set by \prm
{Umathunderbarvgap} and \prm {Umathunderbarkern} is added above the rule. The
style used for the content under the rule can be set with \prm
{Umathunderlinevariant}. See \prm {overline} for what these parameters do.

\stopoldprimitive

\startoldprimitive[title={\prm {unexpanded}}]

This is an \ETEX\ enhancement. The content will not be expanded in a context
where expansion is happening, like in an \prm {edef}. In \CONTEXT\ you need to
use \prm {normalunexpanded} because we already had a macro with that name.

\startbuffer
\def \A{!}                       \meaning\A
\def \B{?}                       \meaning\B
\edef\C{\A\B}                    \meaning\C
\edef\C{\normalunexpanded{\A}\B} \meaning\C
\stopbuffer

\typebuffer

\startlines \tttf \getbuffer \stoplines

\stopoldprimitive

\startnewprimitive[title={\prm {unexpandedendless}}]

This one loops forever so you need to quit explicitly.

\stopnewprimitive

\startnewprimitive[title={\prm {unexpandedloop}}]

As follow up on \prm {expandedloop} we now show its counterpart:

\startbuffer
\edef\whatever
  {\unexpandedloop 1 10 1
     {\scratchcounter=\the\currentloopiterator\relax}}

\meaningasis\whatever
\stopbuffer

\typebuffer

\start \veryraggedright \tt\tfx \getbuffer \stop \blank

The difference between the (un)expanded loops and a local controlled
one is shown here. Watch the out of order injection of \type {A}'s.

\startbuffer
\edef\TestA{\localcontrolledloop 1 5 1 {A}} % out of order
\edef\TestB{\expandedloop        1 5 1 {B}}
\edef\TestC{\unexpandedloop      1 5 1 {C\relax}}
\stopbuffer

\typebuffer \getbuffer

We show the effective definition as well as the outcome of using them

\startbuffer
\meaningasis\TestA
\meaningasis\TestB
\meaningasis\TestC

A: \TestA
B: \TestB
C: \TestC
\stopbuffer

\typebuffer \startlines \tttf \getbuffer \stoplines

Watch how because it is empty \type {\TestA} has become a constant macro because
that's what deep down empty boils down to.

\stopnewprimitive

\startnewprimitive[title={\prm {unexpandedrepeat}}]

This one takes one instead of three arguments which looks better in simple loops.

\stopnewprimitive

\startoldprimitive[title={\prm {unhbox}}]

A box is a packaged list and once packed travels through the system as a single
object with properties, like dimensions. This primitive injects the original list
and discards the wrapper.

\stopoldprimitive

\startoldprimitive[title={\prm {unhcopy}}]

This is like \prm {unhbox} but keeps the original. It is one of the more costly
operations.

\stopoldprimitive

\startnewprimitive[title={\prm {unhpack}}]

This primitive is like \prm {unhbox} but without the callback overhead.

\stopnewprimitive

\startoldprimitive[title={\prm {unkern}}]

This removes the last kern, if possible.

\stopoldprimitive

\startoldprimitive[title={\prm {unless}}]

This \ETEX\ prefix will negate the test (when applicable).

\starttyping
       \ifx\one\two YES\else NO\fi
\unless\ifx\one\two NO\else YES\fi
\stoptyping

This primitive is hardly used in \CONTEXT\ and we probably could get rid of these
few cases.

\stopoldprimitive

\startnewprimitive[title={\prm {unletfrozen}}]

A frozen macro cannot be redefined: you get an error. But as nothing in \TEX\ is set
in stone, you can do this:

\starttyping
\frozen\def\MyMacro{...}
\unletfrozen\MyMacro
\stoptyping

and \type {\MyMacro} is no longer protected from overloading. It is still
undecided to what extend \CONTEXT\ will use this feature.

\stopnewprimitive

\startnewprimitive[title={\prm {unletprotected}}]

The complementary operation of \prm {letprotected} can be used to unprotect
a macro, so that it gets expandable.

\startbuffer
               \def  \MyMacroA{alpha}
\protected     \def  \MyMacroB{beta}
               \edef \MyMacroC{\MyMacroA\MyMacroB}
\unletprotected      \MyMacroB
               \edef \MyMacroD{\MyMacroA\MyMacroB}
\meaning             \MyMacroC\crlf
\meaning             \MyMacroD\par
\stopbuffer

\typebuffer

Compare this with the example in the previous section:

{\tttf \getbuffer}

\stopnewprimitive

\startoldprimitive[title={\prm {unpenalty}}]

This removes the last penalty, if possible.

\stopoldprimitive

\startoldprimitive[title={\prm {unskip}}]

This removes the last glue, if possible.

\stopoldprimitive

\startnewprimitive[title={\prm {untraced}}]

Related to the meaning providers is the \prm {untraced} prefix. It marks a macro
as to be reported by name only. It makes the macro look like a primitive.

\starttyping
         \def\foo{}
\untraced\def\oof{}

\scratchtoks{\foo\foo\oof\oof}

\tracingall \the\scratchtoks \tracingnone
\stoptyping

This will show up in the log as follows:

\starttyping
1:4: {\the}
1:5: \foo ->
1:5: \foo ->
1:5: \oof
1:5: \oof
\stoptyping

This is again a trick to avoid too much clutter in a log. Often it doesn't matter
to users what the meaning of a macro is (if they trace at all). \footnote {An
earlier variant could also hide the expansion completely but that was just
confusing.}

\stopnewprimitive

\startoldprimitive[title={\prm {unvbox}}]

A box is a packaged list and once packed travels through the system as a single
object with properties, like dimensions. This primitive injects the original list
and discards the wrapper.

\stopoldprimitive

\startoldprimitive[title={\prm {unvcopy}}]

This is like \prm {unvbox} but keeps the original. It is one of the more costly
operations.

\stopoldprimitive

\startnewprimitive[title={\prm {unvpack}}]

This primitive is like \prm {unvbox} but without the callback overhead.

\stopnewprimitive

\startoldprimitive[title={\prm {uppercase}}]

See its counterpart \prm {lowercase} for an explanation.

\stopoldprimitive

\startoldprimitive[title={\prm {vadjust}}]

This injects a node that stores material that will injected before or after the
line where it has become part of. In \LUAMETATEX\ there are more features, driven
by keywords.

\stopoldprimitive

\startoldprimitive[title={\prm {valign}}]

This command starts vertically aligned material. Its counterpart \prm {halign} is
used more frequently. Most macro packages provide wrappers around these commands.
First one specifies a preamble which is then followed by entries (rows and
columns).

\stopoldprimitive

\startnewprimitive[title={\prm {variablefam}}]

In traditional \TEX\ sets the family of what are considered variables (class 7)
to the current family (which often means that they adapt to the current alphabet)
and then injects a math character of class ordinary. This parameter can be used
to obey the given class when the family set for a character is the same as this
parameter. So we then use the given class with the current family. It is mostly
there for compatibility with \LUATEX\ and experimenting (outside \CONTEXT).

\stopnewprimitive

\startoldprimitive[title={\prm {vbadness}}]

This sets the threshold for reporting a (vertical) badness value, its current
value is \the \badness.

\stopoldprimitive

\startoldprimitive[title={\prm {vbox}}]

This creates a vertical box. In the process callbacks can be triggered that can
preprocess the content, influence line breaking as well as assembling the
resulting paragraph. More can be found in dedicated manuals. The baseline is
at the bottom.

\stopoldprimitive

\startoldprimitive[title={\prm {vcenter}}]

In traditional \TEX\ this box packer is only permitted in math mode but in
\LUAMETATEX\ it also works in text mode. The content is centered in the vertical
box.

\stopoldprimitive

\startoldprimitive[title={\prm {vfil}}]

This is a shortcut for \typ {\vskip plus 1 fil} (first order filler).

\stopoldprimitive

\startoldprimitive[title={\prm {vfill}}]

This is a shortcut for \typ {\vskip plus 1 fill} (second order filler).

\stopoldprimitive

\startoldprimitive[title={\prm {vfilneg}}]

This is a shortcut for \typ {\vskip plus - 1 fil} so it can compensate \prm
{vfil}.

\stopoldprimitive

\startoldprimitive[title={\prm {vfuzz}}]

This dimension sets the threshold for reporting vertical boxes that are under- or
overfull. The current value is \the \vfuzz.

\stopoldprimitive

\startnewprimitive[title={\prm {virtualhrule}}]

This is a horizontal rule with zero dimensions from the perspective of the
frontend but the backend can access them as set.

\stopnewprimitive

\startnewprimitive[title={\prm {virtualvrule}}]

This is a vertical rule with zero dimensions from the perspective of the frontend
but the backend can access them as set.

\stopnewprimitive

\startoldprimitive[title={\prm {vkern}}]

This primitive is like \prm {kern} but will force the engine into vertical mode
if it isn't yet.

\stopoldprimitive

\startnewprimitive[title={\prm {vpack}}]

This primitive is like \prm {vbox} but without the callback overhead.

\stopnewprimitive

\startnewprimitive[title={\prm {vpenalty}}]

This primitive is like \prm {penalty} but will force the engine into vertical
mode if it isn't yet.

\stopnewprimitive

\startoldprimitive[title={\prm {vrule}}]

This creates a vertical rule. Unless the height and depth are set they will
stretch to fix the available space. In addition to the traditional \type {width},
\type {height} and \type {depth} specifiers some more are accepted. These are
discussed in other manuals. See \prm {hrule} for a simple example.

\stopoldprimitive

\startoldprimitive[title={\prm {vsize}}]

This sets (or gets) the current vertical size. While setting the \prm {hsize}
inside a \prm {vbox} has consequences, setting the \prm {vsize} mostly makes
sense at the outer level (the page).

\stopoldprimitive

\startoldprimitive[title={\prm {vskip}}]

The given glue is injected in the vertical list. If possible vertical mode is
entered.

\stopoldprimitive

\startoldprimitive[title={\prm {vsplit}}]

This operator splits a given amount from a vertical box. In \LUAMETATEX\ we can
split \type {to} but also \type {upto}, so that we don't have to repack the
result in order to see how much is actually in there.

\stopoldprimitive

\startoldprimitive[title={\prm {vss}}]

This is the vertical variant of \prm {hss}. See there for what it means.

\stopoldprimitive

\startoldprimitive[title={\prm {vtop}}]

This creates a vertical box. In the process callbacks can be triggered that can
preprocess the content, influence line breaking as well as assembling the
resulting paragraph. More can be found in dedicated manuals. The baseline is
at the top.

\stopoldprimitive

\startoldprimitive[title={\prm {wd}}]

Returns the width of the given box.

\stopoldprimitive

\startoldprimitive[title={\prm {widowpenalties}}]

This is an array of penalty put before the last lines in a paragraph. High values
discourage (or even prevent) a lone line at the beginning of a next page. This
command expects a count value indicating the number of entries that will follow.
The first entry is ends up before the last line.

\stopoldprimitive

\startoldprimitive[title={\prm {widowpenalty}}]

This is the penalty put before a widow line in a paragraph. High values
discourage (or even prevent) a lone line at the beginning of a next page.

\stopoldprimitive

\startnewprimitive[title={\prm {wordboundary}}]

The hypenation routine has to decide where a word begins and ends. If you want to
make sure that there is a proper begin or end of a word you can inject this
boundary.

\stopnewprimitive

\startnewprimitive[title={\prm {wrapuppar}}]

What this primitive does can best be shown with an example:

\startbuffer
some text\wrapuppar{one} and some\wrapuppar{two} more
\stopbuffer

\typebuffer

We get:

\blank \getbuffer \blank

So, it is a complementary command to \prm {everypar}. It can only be issued
inside a paragraph.

\stopnewprimitive

\startoldprimitive[title={\prm {xdef}}]

This is an alternative for \type {\global \edef}:

\starttyping
\xdef\MyMacro{...}
\stoptyping

\stopoldprimitive

\startnewprimitive[title={\prm {xdefcsname}}]

This is the companion of \prm {xdef}:

\starttyping
\expandafter\xdef\csname MyMacro:1\endcsname{...}
             \xdefcsname MyMacro:1\endcsname{...}
\stoptyping

\stopnewprimitive

\startoldprimitive[title={\prm {xleaders}}]

See \prm {gleaders} for an explanation.

\stopoldprimitive

\startoldprimitive[title={\prm {xspaceskip}}]

Normally the glue inserted when a space is encountered after a character with a
space factor other than 1000 is taken from the font (fontdimen 7) unless this
parameter is set in which case its value is added.

\stopoldprimitive

\startnewprimitive[title={\prm {xtoks}}]

This is the global variant of \prm {etoks}.

\stopnewprimitive

\startnewprimitive[title={\prm {xtoksapp}}]

This is the global variant of \prm {etoksapp}.

\stopnewprimitive

\startnewprimitive[title={\prm {xtokspre}}]

This is the global variant of \prm {etokspre}.

\stopnewprimitive

\startoldprimitive[title={\prm {year}}]

This internal number starts out with the year that the job started.

\stopoldprimitive

\stopsubject

\startsubject[title=Obsolete]

The \LUAMETATEX\ engine has more than its \LUATEX\ ancestor but it also has less.
Because in the end the local control mechanism performed quite okay I decided to
drop the \prm {immediateassignment} and \prm {immediateassigned} variants. They
sort of used the same trick so there isn't much to gain and it was less generic
(read: error prone).

% \startnewprimitive[title={\prm {immediateassignment}}]
%
% Assignments are not expandable which means that you cannot define fully
% expandable macros that have assignments. But, there is a way out of this:
%
% \startbuffer
% \scratchcounter = 10
% \edef\whatever{%
%     (\the\scratchcounter)
%     \immediateassignment\scratchcounter\numexpr\scratchcounter+10\relax
%     \immediateassignment\advance\scratchcounter -5
%     (\the\scratchcounter)
% }
% \meaning\whatever
% \stopbuffer
%
% \typebuffer
%
% Don't expect miracles: you can't mix|-|in content or unexpandable tokens as they
% will either show up or quit the scanning.
%
% {\getbuffer}
%
% \stopnewprimitive
%
% \startnewprimitive[title={\prm {immediateassigned}}]
%
% This is the multi|-|token variant of the primitive mentioned in the previous
% section.
%
% \startbuffer
% \scratchcounter = 10
% \edef\whatever{%
%     (\the\scratchcounter)
%     \immediateassigned{
%         \scratchcounter\numexpr\scratchcounter+10\relax
%         \advance\scratchcounter -5
%     }%
%     (\the\scratchcounter)
% }
% \meaning\whatever
% \stopbuffer
%
% \typebuffer
%
% The results are the same as in the previous section:
%
% {\getbuffer}
%
% \stopnewprimitive

\stopsubject

\page


\definehead
  [Syntax]
  [subsection]
%   [style=\tta\bf,
%    numberwidth=2fs,
%    after=\blank\startpacked,
%    aftersection=\stoppacked]

\startsubject[title=Syntax]

\startpagecolumns[page=no]
    \startluacode
        moduledata.engine.allspecifications()
    \stopluacode
\stoppagecolumns

\stopsubject

\page

% % It doesn't make sense to typeset this, also because it makes me feel old.
%
% \startsubject[title=A few notes on extensions] % ,placeholder=todo]
%
% This is a companion to the regular \LUAMETATEX\ reference manual, which is mostly
% a concise summary of the program and its features. They don't replace each other,
% and none of them claims completeness. There might be more manuals that discuss
% specific kind of extensions in the future. First some comments on extensions.
%
% The starting point of all \TEX\ engines is \TEX. The first follow up was \TEX\
% with support for 8 bit and languages. After that it took some time, but then two
% projects started that extended \TEX: \ETEX\ and \OMEGA. In the end the first
% brought some extensions to the macro machinery, more registers, a simple right to
% left typesetting feature, some more tracing, etc. The second was more ambitious
% and has input translation mechanisms, larger fonts, and multi directional
% typesetting. By the time \ETEX\ became stable, the \PDFTEX\ engine had showed up
% and at some point it integrated \ETEX. But \PDFTEX\ itself also extended the
% several components that make up \TEX. The \NTS\ project was started as follow up
% on \ETEX\ but although an engine written in \JAVA\ was the result it never was
% used for extensions; this project was fully funded by the german language user
% group.
%
% The \CONTEXT\ macro package was an early adopter of the \ETEX\ and \PDFTEX\
% extensions. Probably the most significant effect was that we got more registers
% (some other features were already kind of present in macro form). In between
% these engines we played with \type {eetex} (extended \ETEX) because we had some
% wishes of our own. We also explored extensions to the \DVI\ format but in the end
% \PDF\ won that race. An example of a new mechanism that we introduced in
% \CONTEXT\ was position tracking: marking positions that can be saved when the
% output is created and used in a second run. This started as a \DVI\ postprocessor
% in \PERL\ written by me, later turned into a \CLANGUAGE\ program by Taco, and
% eventually integrated in \PDFTEX\ by Thanh (\PDFTEX\ was a phd project). At some
% point \XETEX\ was developed, funded and driven by an organization that did high
% end multi lingual typesetting; it was based on \ETEX\ and uses a \DVI\ to \PDF\
% backend processor. Both \PDFTEX\ and \XETEX\ are supported by \CONTEXT\ \MKII,
% and both engines are basically stable and frozen.
%
% At some point Hartmut and I started playing with \LUA\ in \PDFTEX\ but soon Taco,
% Hartmut and I decided to start a follow up project. All the work on \LUATEX\ (and
% later \LUAMETATEX) is done whenever there is time and without financial
% compensation, so we have a slow but steady development track. Early in the
% \LUATEX\ development there has been some funding for the initial transition (by
% Taco) from \PDFTEX\ to what became the early versions of \LUATEX . You can read
% more about the oriental \TEX\ project in other documents and articles in user
% group journals. There has been some funded development of a library subsystem
% (which for some reason never took off) as well as \LUAJIT\ integration (by
% Luigi). The initial \METAPOST\ library (also done by Taco) was funded by a couple
% of user groups. Then there are the (ongoing) font projects by GUST that got
% funded by user groups as these were much needed for the \UNICODE\ engines.
%
% After the jump start, most work was and is still done in the usual \TEX\ spirit,
% on a voluntary basis, by folks from the \CONTEXT\ community, and after a decades
% we reached the stable version 1.00. It's one of the engines in \TEXLIVE\ and
% Luigi makes sure it integrates well in there. We did continue and around 1.10 the
% more of less final version was reached and \LUAMETATEX\ took off. In \LUATEX\
% only bugs get fixed, occasionally some helpers can get added, and we might port
% some of \LUAMETATEX\ back to its parent, when it doesn't harm compatibility.
%
% Already early in development some primitives were added that enhance the macro
% language. More were added later. It's these extensions that are discussed in this
% document. There are several documents in the \CONTEXT\ distribution that discuss
% the (ongoing) development, right from the start, and these often contain
% examples. For instance some of the new primitives have been introduced there,
% complete with a rationale and examples of usage.
%
% Just for the record: the \CONTEXT\ group runs the build farm that is used to
% generate binaries for all sorts of platforms. We make sure that there are always
% versions that can be used for real production jobs. You can expect regular
% updates as long as there are developments (of course, eventually we're done).
%
% \stopsubject

\startsubject[title=Rationale] % ,placeholder=todo]

Some words about the why and how it came. One of the early adopters of \CONTEXT\
was Taco Hoekwater and we spent numerous trips to \TEX\ meetings all over the
globe. He was also the only one I knew who had read the \TEX\ sources. Because
\CONTEXT\ has always been on the edge of what is possible and at that time we
both used it for rather advanced rendering, we also ran into the limitations. I'm
not talking of \TEX\ features here. Naturally old school \TEX\ is not really
geared for dealing with images of all kind, colors in all kind of color spaces,
highly interactive documents, input methods like \XML, etc. The nice thing is
that it offers some escapes, like specials and writes and later execution of
programs that opened up lots of possibilities, so in practice there were no real
limitations to what one could do. But coming up with a consistent and extensible
(multi lingual) user interface was non trivial, because it had an impact in
memory usage and performance. A lot could be done given some programming, as
\CONTEXT\ \MKII\ proves, but it was not always pretty under the hood. The move to
\LUATEX\ and \MKIV\ transferred some action to \LUA, and because \LUATEX\
effectively was a \CONTEXT\ related project, we could easily keep them in sync.

Our traveling together, meeting several times per year, and eventually email and
intense \LUATEX\ developments (lots of Skype sessions) for a couple of years,
gave us enough opportunity to discuss all kind of nice features not present in
the engine. The previous century we discussed lots of them, rejected some, stayed
with others, and I admit that forgot about most of the arguments already. Some
that we did was already explored in \type {eetex}, some of those ended up in
\LUATEX, and eventually what we have in \LUAMETATEX\ can been seen as the result
of years of programming in \TEX, improving macros, getting more performance and
efficiency out of existing \CONTEXT\ code and inspiration that we got out of the
\CONTEXT\ community, a demanding lot, always willing to experiment with us.

Once I decided to work on \LUAMETATEX\ and bind its source to the \CONTEXT\
distribution so that we can be sure that it won't get messed up and might
interfere with the \CONTEXT\ expectations, some more primitives saw their way
into it. It is very easy to come up with all kind of bells and whistles but it is
equally easy to hurt performance of an engine and what might go unnoticed in
simple tests can really affect a macro package that depends on stability. So, what
I did was mostly looking at the \CONTEXT\ code and wondering how to make some of
the low level macros look more natural, also because I know that there are users
who look into these sources. We spend a lot of time making them look consistent
and nice and the nicer the better. Getting a better performance was seldom an
argument because much is already as fast as can be so there is not that much to
gain, but less clutter in tracing was an argument for some new primitives. Also,
the fact that we soon might need to fall back on our phones to use \TEX\ a
smaller memory footprint and less byte shuffling also was a consideration. The
\LUAMETATEX\ memory footprint is somewhat smaller than the \LUATEX\ footprint.
By binding \LUAMETATEX\ to \CONTEXT\ we can also guarantee that the combinations
works as expected.

I'm aware of the fact that \CONTEXT\ is in a somewhat unique position. First of
all it has always been kind of cutting edge so its users are willing to
experiment. There are users who immediately update and run tests, so bugs can and
will be fixed fast. Already for a long time the community has an convenient
infrastructure for updating and the build farm for generating binaries (also for
other engines) is running smoothly.

Then there is the \CONTEXT\ user interface that is quite consistent and permits
extensions with staying backward compatible. Sometimes users run into old manuals
or examples and then complain that \CONTEXT\ is not compatible but that then
involves obsolete technology: we no longer need font and input encodings and font
definitions are different for \OPENTYPE\ fonts. We always had an abstract backend
model, but nowadays \PDF\ is kind of dominant and drives a lot of expectations.
So, some of the \MKII\ commands are gone and \MKIV\ has some more. Also, as
\METAPOST\ evolved that department in \CONTEXT\ also evolved. Think of it like
cars: soon all are electric so one cannot expect a hole to poor in some fluid but
gets a (often incompatible) plug instead. And buttons became touch panels. There
is no need to use much force to steer or brake. Navigation is different, as are
many controls. And do we need to steer ourselves a decade from now?

So, just look at \TEX\ and \CONTEXT\ in the same way. A system from the nineties
in the previous century differs from one three decades later. Demands differ,
input differs, resources change, editing and processing moves on, and so on.
Manuals, although still being written are seldom read from cover to cover because
online searching replaced them. And who buys books about programming? So
\LUAMETATEX, while still being \TEX\ also moves on, as do the way we do our low
level coding. This makes sense because the original \TEX\ ecosystem was not made
with a huge and complex macro package in mind, that just happened. An author was
supposed to make a style for each document. An often used argument for using
another macro package over \CONTEXT\ was that the later evolved and other macro
packages would work the same forever and not change from the perspective of the
user. In retrospect those arguments were somewhat strange because the world,
computers, users etc.\ do change. Standards come and go, as do software politics
and preferences. In many aspects the \TEX\ community is not different from other
large software projects, operating system wars, library devotees, programming
language addicts, paradigm shifts. But, don't worry, if you don't like
\LUAMETATEX\ and its new primitives, just forget about them. The other engines
will be there forever and are a safe bet, although \LUATEX\ already stirred up
the pot I guess. But keep in mind that new features in the latest greatest
\CONTEXT\ version will more and more rely on \LUAMETATEX\ being used; after all
that is where it's made for. And this manual might help understand its users why,
where and how the low level code differs between \MKII, \MKIV\ and \LMTX.

Can we expect more new primitives than the ones introduced here? Given the amount
of time I spent on experimenting and considering what made sense and what not,
the answer probably is \quotation {no}, or at least \quotation {not that much}.
As in the past no user ever requested the kind of primitives that were added, I
don't expect users to come up with requests in the future either. Of course,
those more closely related to \CONTEXT\ development look at it from the other
end. Because it's there where the low level action really is, demands might still
evolve.

Basically there are wo areas where the engine can evolve: the programming part
and the rendering. In this manual we focus on the programming and writing the
manual sort of influences how details get filled in. Rendering in more complex
because there heuristics and usage plays a more dominant role. Good examples are
the math, par and page builder. They were extended and features were added over
time but improved rendering came later. Not all extensions are critical, some are
there (and got added) in order to write more readable code but there is only so
much one can do in that area. Occasionally a feature pops up that is a side
effect of a challenge. No matter what gets added it might not affect complexity
too much and definitely not impact performance significantly!

Hans Hagen \crlf Hasselt NL

\stopsubject

\popoverloadmode

\startluacode
    local match = string.match
    local find  = string.match

    function document.CheckCompleteness()
        local primitives = token.getprimitives()
        local luametatex = { }
        local indexed    = { }
        local everything = { }

        for i=1,#primitives do
            local prim = primitives[i]
            local name = prim[3]
            if prim[4] == 4 then
                if find(name,"U") or find(name,"math") then
                    -- ignore
                    luametatex[name] = nil
                    everything[name] = false
                else
                    luametatex[name] = false
                end
            else
                everything[name] = true
            end
        end

        local function collect(index)
            if index then
                local data = index.entries
                for i=1,#data do
                    local name = match(data[i].list[1][1],"\\tex%s*{(.-)}") or ""
                    if luametatex[name] == false then
                        luametatex[name] = true
                    end
                    indexed[name] = true
                    everything[name] = nil
                end
            end
        end

        collect(structures.registers.collected and structures.registers.collected.index)

        context.startsubject { title = "To be checked primitives (new)" }

        context.blank()
        context.startcolumns { n = 2 }
        for k, v in table.sortedhash(luametatex) do
            if not v then
                context.dontleavehmode()
                context.type(k)
                context.crlf()
            end
        end
        context.stopcolumns()

        context.page()

        context.stopsubject()

        everything[""]  = nil
        everything[" "] = nil

        context.startsubject { title = "To be checked primitives (math)" }

        context.blank()
        context.startcolumns { n = 2 }
        for k, v in table.sortedhash(everything) do
            if not v then
                context.dontleavehmode()
                context.type(k)
                context.crlf()
            end
        end
        context.stopcolumns()

        context("Many primitives starting with \\type{Umath} are math parameters that are discussed elsewhere, if at all.")

        context.page()

        context.stopsubject()

        context.startsubject { title = "To be checked primitives (old)" }

        context.blank()
        context.startcolumns { n = 2 }
        for k, v in table.sortedhash(everything) do
            if v then
                context.dontleavehmode()
                context.type(k)
                context.crlf()
            end
        end
        context.stopcolumns()

        context.page()

        context.stopsubject()

        context.startsubject { title = "Indexed primitives" }

        context.blank()
        context.startcolumns { n = 2 }
        for k, v in table.sortedhash(indexed) do
            context.dontleavehmode()
            if luametatex[k] == true then
                context("\\color[darkgreen]{\\tttf %s}",k)
            elseif luametatex[k] == false then
                context("\\color[darkred]{\\tttf %s}",k)
            else
                context("{\\tttf %s}",k)
            end
            context.crlf()
        end
        context.stopcolumns()

        context.page()

        context.stopsubject()
    end
\stopluacode

\startmode[atpragma]
    \startluacode
        context.page()
        document.CheckCompleteness()
    \stopluacode

%     Run \type {s-system-syntax.mkxl} for a complete overview of the \LUAMETATEX\
%     primitives.
\stopmode

\stopbodymatter

\stoptext

% disk and math options: orphaned
