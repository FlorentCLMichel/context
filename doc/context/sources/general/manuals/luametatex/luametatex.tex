% language=us runpath=texruns:manuals/luametatex

\unprotect \pushoverloadmode
\protect

%     \definefontfeature
%       [default]
%       [default]
%       [expansion=quality]

%    %     \setupalignpass[none]
%     \setupalignpass[decent]
%    %     \setupalignpass[quality]

% \pushoverloadmode \let\cdef\edef \let\cdefcsname\edefcsname \popoverloadmode

% \enabledirectives[backend.pdf.inmemory]

% \setupalign[profile]
% \enabletrackers[profiling.lines.show]

% ------------------------   ------ ------------------   ------------------------
% 2019-12-17  32bit  64bit   2020-01-10  32bit  64bit   2020-11-30  32bit  64bit
% ------------------------   ------------------------   ------------------------
% freebsd     2270k  2662k   freebsd     2186k  2558k   freebsd     2108k  2436k
% openbsd6.6  2569k  2824k   openbsd6.6  2472k  2722k   openbsd6.8  2411k  2782k
% linux-armhf 2134k          linux-armhf 2063k          linux-armhf 2138k  2860k
% linux       2927k  2728k   linux       2804k  2613k   linux   (?) 3314k  2762k
%                                                       linux-musl  2532k  2686k
% osx                2821k   osx                2732k   osx                2711k
% ms mingw    2562k  2555k   ms mingw    2481k  2471k   ms mingw    2754k  2760k
%                                                       ms intel           2448k
%                                                       ms arm             3894k
%                                                       ms clang           2159k
% ------------------------   ------------------------   ------------------------
%
% performance of mingw and native are getting close (small win for mingw) but
% clang bins are still slower .. quite inconsistent differences between 32 and
% 64 bit (not all compilers are the same version)

% \enableautoglyphscaling % saves only a few instances ... no gain .. a few pages more

% \enabletexdirective{vspacing.experimental}

% \nopdfcompression

% 20200509 : 258 pages
%
% my 2013 i7 laptop with windows : 11.8 sec mingw64
% raspberry pi 64 bit with ssd   : 39.5 sec gcc 9.2
% idem with native bin           : 38.5 sec
% idem overclocked f=2000/v=6    : 31.5 sec
%
% 20200526 : 258 pages
% mingw all in: 10.9 / rpi 32 bit: 33.1

% 20200610 : 258 pages
% mingw all in: 10.6

% 20201204 : 260 pages
% mingw all in: 10.0

% 20201204 : 262 pages
% mingw all in: 9.9

% 20200720      : 258 pages / all in
%
% mingw         : 10.6 (sometimes less)
% rpi 32        : 32.3
% rpi 64        : 26.0 (overclocked)
% amd 10 fitlet : 28.2

% 20210812      : 298 pages
% mingw         :  9.7

% 20210903      : 300 pages
% mingw         :  9.6

% 20210916      : 300 pages
% mingw         :  9.4 (9.25 with LTO, 9.75 native)

% But speed can differ a bit per compilation (upto .5 seconds maybe due to how
% compiled code is organized which might influence caching. Who knows ... (or
% cares). For instance at 20200407 I measured 10.9 seconds after some new low level
% metapost magic was added but who knows if that was the reason, because mp
% processing is already fast. A week later, at 20200415, a by then 254 page file
% took 10.5 seconds, that is, we were at exactly 24 pages per second but after
% switching to gcc9 it dropped again. In december 2020, with IPO enabled I crossed
% the 26 pps barrier and went below 10 seconds but that was also after some further
% cleanup in \LMTX. Mid August 2021 I measured over 30 pps but not all is due to
% the engine I guess. End august we were are over 31 pps so it's not getting worse.

% (This means that on a modern desktop we probably can get around 100 pps on a
% manual like this but, always using laptops, I don't have access to machines like
% that. A modern laptop probably could perform over 50 pps.)

% msvc           1899k (2% slower than mingw)
% msvc /GL       2297k (similar to mingw)
% msvc /GL /Ob3  2847k (not faster than /GL)
% msvc /Ob3      2052k (slower than /Ob2)
% msvc /Ob1      1763k (slower than /Ob2)
% clang          2460k (15% slower than mingw)

% Thanks to sebastian.miele@gmail.com for close reading the manual and sending
% fixes.

% \listcallbackmode0

% 290 pages, 10.8 sec, 292M lua, 99M tex, 158 instances
% 290 pages,  9.5 sec, 149M lua, 35M tex,  30 instances

% On the 2018 Dell 7520 Precission Xeon laptop runtime for 322 pages is now exactly
% 8 seconds (just below 40 pages per second), we talking July 16, 2022.
%
% At the first of August 2022 the (slightly reorganized and updated) manual counted
% 350 pages and took 8.6 seconds to process (some 41 pages per second), so we're
% rather stable. End of 2022 processing that amount of pages dropped somewhat
% (occasionally under 8.3 seconds for 356 pages with little other system load). In
% means that 43 pages per seconds is now the new normal but that might of course
% become less again as we evolve.

% End Feburari 2023 I observed 8.2 seconds for 360 pages and making a format needed
% 1.9 seconds instead if the usual 2.1 but that can be a side effect of the terminal
% because the amount of output which is sensitive for refresh delays set. The new
% target is now 50 pages per second for this manual but on this laptop that is
% unlikely to happen any time soon. With tabulateusesize and tabulatesparseskips
% experiments enabled we needed 8.1 second and 44.3 pps.% 368 pages may/june 2023:
%
% no constant definitions : 8.55
% constant \*!whatever    : 8.47
% also constant \current* : 8.36

% \enableexperiments [tabulateusesize]
% \enableexperiments [tabulatesparseskips]

% 2023-07-10: context --make > temp.log (after a few times so with cached files)
%
% 1.90 sec including mtxrun
% 1.65 sec tex only
% 1.45 sec without saving format

% 20230714      : 368 pages
%
% mingw         :  8.4 (we gained more that for the amd ... more cache and faster mem)
% rpi 64        :      (hdd, not overclocked)
% amd 10 fitlet : 30.9 12pps (we gained over 25% compared to the 268 page manual)


% \ifdefined\linebuffering \linebuffering \fi

\enableexperiments[fonts.compact]
%disableexperiments[fonts.accurate]

% \enabledirectives[fonts.injections.method=advance] % tricky ... not all xoffsets are advance robust

% This is only for testing, use the command line (or banner line) instead.
%
% \startluacode
%     lpdf.setencryption {
%         ownerpassword = "password",
%         userpassword  = " ",
%         permissions   = "print,quality"
%     }
% \stopluacode

\pushoverloadmode \unprotect
    % test code
\protect \popoverloadmode

\enabletrackers[system.usage=summary]

\environment luametatex-style
\environment luametatex-private

\startdocument
  [manual=LuaMeta\TeX,
  %status=experimental,
   version=\cldcontext{status.luatex_verbose}]

\component luametatex-titlepage
\component luametatex-firstpage

\startfrontmatter
    \component luametatex-contents
    \component luametatex-introduction
\stopfrontmatter

\startbodymatter
    \component luametatex-modifications
    \component luametatex-lua
    \component luametatex-enhancements
    \component luametatex-fonts
    \component luametatex-languages
    \component luametatex-math
    \component luametatex-building
    \component luametatex-nodes
    \component luametatex-callbacks
    \component luametatex-tex
    \component luametatex-metapost
    \component luametatex-pdf
    \component luametatex-libraries
    \component luametatex-primitives % this generates a list
  % \component luametatex-rejected % local file
\stopbodymatter

\startbackmatter
    \component luametatex-codes
    \component luametatex-registers
    \component luametatex-statistics
    \component luametatex-remarks
\stopbackmatter

\stopdocument

