% language=us runpath=texruns:manuals/luametatex

\environment luametatex-style

\startdocument[title=Libraries]

\startsection[title={Introduction}]

The engines has quite some libraries built in of which some are discussed in
dedicated chapters. Not all libraries will be detailed here, for instance, so
called optional libraries depend on system libraries and usage is wrapped in
modules because we delegate as much as possible to \LUA.

\stopsection

\startsection[title={Third party}]

There is not much to tell here other than it depends on the \LUA\ symbols being
visible and the \LUA\ version matching. We don't use this in \CONTEXT\ and have a
different mechanism instead: optional libraries.

\stopsection

\startsection[title={Core}]

The core libraries are those that interface with \TEX\ and \METAPOST, these are
discussed in dedicated chapters:

\starttabulate[|l|Tl|]
\FL
\BC chapter   \BC library          \NC \NR
\TL
\NC \LUA      \NC lua luac         \NC \NR
\NC \TEX      \NC status tex texio \NC \NR
\NC \METAPOST \NC mp               \NC \NR
\NC Nodes     \NC node             \NC \NR
\NC Tokens    \NC token            \NC \NR
\NC Callbacks \NC callback         \NC \NR
\NC Fonts     \NC font             \NC \NR
\NC Languages \NC language         \NC \NR
\NC Libraries \NC library          \NC \NR
\LL
\stoptabulate

Some, like \type {node}, \type {token} and \type {tex} provide a lot of functions
but most are used in more higher level \CONTEXT\ specific functions and
interfaces. This means that in the code you will more often font \type {nodes}
and \type {tokens} being used as well as functions that the macro package adds to
the various built-in libraries.

\stopsection

\startsection[title={Auxiliary}]

\startsubsection[title=Extensions]

These are the libraries that are needed to implement various subsystems, like for
instance the backend and image inclusion. Although much can be done in pure \LUA\
for performance reasons helpers make sense. However, we try to minimize this,
which means that for instance the \type {zip} library provides what we need for
(de)compressing for instance \PDF\ streams but that unzipping files is done with
\LUA\ code wrapped around the core zip routines. The same is true for \PNG\
inclusion: all that was done in pure \LUA\ but a few critical helpers were
translated to \CCODE.

Some libraries extend existing ones, like for instance \type {file}, \type {io}
and \type {os} and \type {string}.

\stopsubsection

\startsubsection[title=Extra file helpers]

The original \type {lfs} module has been adapted a bit to our needs but for
practical reasons we kept the namespace. In \LUAMETATEX\ we operate in \UTF8
so for \MSWINDOWS\ system interfaces we convert from and to \UNICODE16.

The \type {attributes} checker returns a table with details.

\starttyping[option=LUA]
function lfs.attributes ( <t:string> name )
    return <t:table> -- details
end
\stoptyping

The table has the following fields:

\starttabulate[|T|||]
\FL
\BC field                \BC type    \BC meaning \NC \NR
\TL
\NC \type {mode}         \NC string  \NC \type{file} \type{directory} \type{link} \type{other} \NC \NR
\NC \type {size}         \NC integer \NC bytes \NC \NR
\NC \type {modification} \NC integer \NC time \NC \NR
\NC \type {access}       \NC integer \NC time \NC \NR
\NC \type {change}       \NC integer \NC time \NC \NR
\NC \type {permissions}  \NC string  \NC \type {rwxrwxrwx} \NC \NR
\NC \type {nlink}        \NC integer \NC number of links \NC \NR
\LL
\stoptabulate

If you're not interested in details, then the next calls are more efficient:

\starttyping[option=LUA]
function lfs.isdir           ( <t:string> name ) return <t:boolean> end
function lfs.isfile          ( <t:string> name ) return <t:boolean> end
function lfs.iswriteabledir  ( <t:string> name ) return <t:boolean> end
function lfs.iswriteablefile ( <t:string> name ) return <t:boolean> end
function lfs.isreadabledir   ( <t:string> name ) return <t:boolean> end
function lfs.isreadablefile  ( <t:string> name ) return <t:boolean> end
\stoptyping

The current (working) directory is fetch with:

\starttyping[option=LUA]
function lfs.currentdir ( )
    return <t:string> -- directory
end
\stoptyping

These three return \type {true} is the action was a success:

\starttyping[option=LUA]
function lfs.chdir ( <t:string> name ) return <t:boolean> end
function lfs.mkdir ( <t:string> name ) return <t:boolean> end
function lfs.rmdir ( <t:string> name ) return <t:boolean> end
\stoptyping

Here the second and third argument are optional:

\starttyping[option=LUA]
function lfs.touch (
    <t:string>  name,
    <t:integer> accesstime,
    <t:integer> modificationtime
)
    return <t:boolean> -- success
end
\stoptyping

The \type {dir} function is a traverser which in addition to the name returns
some more properties. Keep in mind that the traverser loops over a directory and
that it doesn't run well when used nested. This is a side effect of the operating
system. It is also the reason why we return some properties because querying them
via \type {attributes} would interfere badly. The directory iterator has two
variants:

\starttyping[option=LUA]
for
    <t:string> name,
    <t:string> mode
in lfs.dir (
    <t:string> name
)
    -- actions
end
\stoptyping

This one provides more details:

\starttyping[option=LUA]
for
    <t:string>  name,
    <t:string>  mode,
    <t:integer> size,
    <t:integer> mtime
in lfs.dir (
    <t:string>  name,
    <t:true>
)
    -- actions
end
\stoptyping

Here the boolean indicates if we want a symlink (\type {true}) or hard link
(\type {false}).

\starttyping[option=LUA]
function lfs.link (
    <t:string>  source,
    <t:string>  target,
    <t:boolean> symlink
)
    return <t:boolean> -- success
end
\stoptyping

The next one is sort of redundant but explicit:

\starttyping[option=LUA]
function lfs.symlink (
    <t:string>  source,
    <t:string>  target,
)
    return <t:boolean> -- success
end
\stoptyping

Helpers like these are a bit operating system and user permission dependent:

\starttyping[option=LUA]
function lfs.setexecutable ( <t:string> name )
    return <t:boolean> -- success
end
\stoptyping

\starttyping[option=LUA]
function lfs.symlinktarget ( <t:string> name )
    return <t:string> -- target
end
\stoptyping

\stopsubsection

\startsubsection[title=Reading from a file]

Because we load fonts in \LUA\ and because these are binary files we have some
helpers that can read integers of various kind and some more. Originally we did
this in pure \LUA, which actually didn't perform that bad but this is of course
more efficient.

We have readers for signed and unsigned, little and big endian. All return a (64
bit) \LUA\ integer.

\starttyping[option=LUA]
function fio.readcardinal1 ( <t:file> handle ) return <t:integer> end
function fio.readcardinal2 ( <t:file> handle ) return <t:integer> end
function fio.readcardinal3 ( <t:file> handle ) return <t:integer> end
function fio.readcardinal4 ( <t:file> handle ) return <t:integer> end

function fio.readcardinal1le ( <t:file> handle ) return <t:integer> end
function fio.readcardinal2le ( <t:file> handle ) return <t:integer> end
function fio.readcardinal3le ( <t:file> handle ) return <t:integer> end
function fio.readcardinal4le ( <t:file> handle ) return <t:integer> end

function fio.readinteger1 ( <t:file> handle ) return <t:integer> end
function fio.readinteger2 ( <t:file> handle ) return <t:integer> end
function fio.readinteger3 ( <t:file> handle ) return <t:integer> end
function fio.readinteger4 ( <t:file> handle ) return <t:integer> end

function fio.readinteger1le ( <t:file> handle ) return <t:integer> end
function fio.readinteger2le ( <t:file> handle ) return <t:integer> end
function fio.readinteger3le ( <t:file> handle ) return <t:integer> end
function fio.readinteger4le ( <t:file> handle ) return <t:integer> end
\stoptyping

These float readers are rather specific for fonts:

\starttyping[option=LUA]
function fio.readfixed2 ( <t:file> handle ) return <t:number>  end
function fio.readfixed4 ( <t:file> handle ) return <t:number>  end
function fio.read2dot14 ( <t:file> handle ) return <t:number>  end
\stoptyping

Of these two the first reads a line and the second a string the \CCODE\ way, so
ending with a newline and null character:

\starttyping[option=LUA]
function fio.readcline   ( <t:file> handle ) return <t:string> end
function fio.readcstring ( <t:file> handle ) return <t:string> end
\stoptyping

The next set of readers reads multiple integers in one call:

\starttyping[option=LUA]
function fio.readbytes (
    <t:file> handle
)
    return <t:integer> -- one or more
end
\stoptyping

\starttyping[option=LUA]
function fio.readintegertable (
    <t:file>    handle,
    <t:integer> size,
    <t:integer> bytes
)
    return <t:table>
end

function fio.readcardinaltable (
    <t:file>    handle,
    <t:integer> size,
    <t:integer> bytes
)
    return <t:table>
end

function fio.readbytetable (
    <t:file> handle
)
    return <t:table>
end
\stoptyping

In case we need a random access the following have to be used:

\starttyping[option=LUA]
function fio.setposition  ( <t:file> handle, <t:integer> ) return <t:integer> end
function fio.getposition  ( <t:file> handle              ) return <t:integer> end
function fio.skipposition ( <t:file> handle, <t:integer> ) return <t:integer> end
\stoptyping

The library also provide a few writers:

\starttyping[option=LUA]
function fio.writecardinal1 ( <t:file> handle, <t:integer> value ) end
function fio.writecardinal2 ( <t:file> handle, <t:integer> value ) end
function fio.writecardinal3 ( <t:file> handle, <t:integer> value ) end
function fio.writecardinal4 ( <t:file> handle, <t:integer> value ) end

function fio.writecardinal1le ( <t:file> handle, <t:integer> value ) end
function fio.writecardinal2le ( <t:file> handle, <t:integer> value ) end
function fio.writecardinal3le ( <t:file> handle, <t:integer> value ) end
function fio.writecardinal4le ( <t:file> handle, <t:integer> value ) end
\stoptyping

\stopsubsection

\startsubsection[title=Reading from a string]

These readers take a string and position. We could have used a userdata approach
but it saves little. (Nowadays we can more easily store the position with the
userdata so maybe some day \unknown).

\starttyping[option=LUA]
function sio.readcardinal1 ( <t:string> s, <t:integer> p ) return <t:integer> end
function sio.readcardinal2 ( <t:string> s, <t:integer> p ) return <t:integer> end
function sio.readcardinal3 ( <t:string> s, <t:integer> p ) return <t:integer> end
function sio.readcardinal4 ( <t:string> s, <t:integer> p ) return <t:integer> end

function sio.readcardinal1le ( <t:string> s, <t:integer> p ) return <t:integer> end
function sio.readcardinal2le ( <t:string> s, <t:integer> p ) return <t:integer> end
function sio.readcardinal3le ( <t:string> s, <t:integer> p ) return <t:integer> end
function sio.readcardinal4le ( <t:string> s, <t:integer> p ) return <t:integer> end

function sio.readinteger1 ( <t:string> s, <t:integer> p ) return <t:integer> end
function sio.readinteger2 ( <t:string> s, <t:integer> p ) return <t:integer> end
function sio.readinteger3 ( <t:string> s, <t:integer> p ) return <t:integer> end
function sio.readinteger4 ( <t:string> s, <t:integer> p ) return <t:integer> end

function sio.readinteger1le ( <t:string> s, <t:integer> p ) return <t:integer> end
function sio.readinteger2le ( <t:string> s, <t:integer> p ) return <t:integer> end
function sio.readinteger3le ( <t:string> s, <t:integer> p ) return <t:integer> end
function sio.readinteger4le ( <t:string> s, <t:integer> p ) return <t:integer> end
\stoptyping

Here are the (handy for fonts) float readers:

\starttyping[option=LUA]
function sio.readfixed2 ( <t:string> s, <t:integer> p ) return <t:number> end
function sio.readfixed4 ( <t:string> s, <t:integer> p ) return <t:number> end
function sio.read2dot14 ( <t:string> s, <t:integer> p ) return <t:number> end
\stoptyping

A \CCODE\ line (terminated by a newline) and string (terminated by null) are read
by:

\starttyping[option=LUA]
function sio.readcline   ( <t:string> s, <t:integer> p ) return <t:string> end
function sio.readcstring ( <t:string> s, <t:integer> p ) return <t:string> end
\stoptyping

\starttyping[option=LUA]
function sio.readbytes (
    <t:string>  str,
    <t:integer> pos
)
    return <t:integer> -- one or more
end
\stoptyping

\starttyping[option=LUA]
function sio.readintegertable (
    <t:string>  str,
    <t:integer> pos,
    <t:integer> size,
    <t:integer> bytes
)
    return <t:table>
end

function sio.readcardinaltable (
    <t:string>  str,
    <t:integer> pos,
    <t:integer> size,
    <t:integer> bytes
)
    return <t:table>
end

function sio.readbytetable (
    <t:string>  str,
    <t:integer> pos
)
    return <t:table>
end
\stoptyping

Here are a few straightforward converters:

\starttyping[option=LUA]
function sio.tocardinal1 ( <t:string> ) return <t:integer> end
function sio.tocardinal2 ( <t:string> ) return <t:integer> end
function sio.tocardinal3 ( <t:string> ) return <t:integer> end
function sio.tocardinal4 ( <t:string> ) return <t:integer> end

function sio.tocardinal1le ( <t:string> ) return <t:integer> end
function sio.tocardinal2le ( <t:string> ) return <t:integer> end
function sio.tocardinal3le ( <t:string> ) return <t:integer> end
function sio.tocardinal4le ( <t:string> ) return <t:integer> end
\stoptyping

\stopsubsection

\startsubsection[title=Extra file helpers]

This function gobble characters upto a newline. When characters are gobbled.
\type {true} is returned when we end up at a newline or when something is gobbled
before the file ends, other wise we get \type {false}. A \type {nil} return value
indicates a bad handle.

\starttyping[option=LUA]
function io.gobble( <t:file> )
    return <t:boolean> | <t:nil>
end
\stoptyping

Function like type {io.open} \type {io.popen} are patched to support files on
\MSWINDOWS\ that use wide \UNICODE.

\stopsubsection

\startsubsection[title=Extra operating system helpers]

The \type {os} library has a few extra functions and variables so for complete
overview you need to look in the \LUA\ manual.

We can sleep for the given number of seconds. When the optional \type {units}
arguments is (for instance) 1000 we assume milliseconds.

\starttyping[option=LUA]
function os.sleep (
    <t:integer>  seconds,
    <t:integers> units
)
    -- no return values
end
\stoptyping

The \type {os.uname} function returns a table with specific operating system
information acquired at runtime. The fields in the returned table are:
\showenginekeylist {os.getunamefields()}.

\starttyping[option=LUA]
function os.uname ( )
    return <t:table>
\stoptyping

The \type {os.gettimeofday} function returns the current \quote {\UNIX\ time},
but as a float. Keep in mind that there might be platforms where this function is
not available.

\starttyping[option=LUA]
function os.gettimeofday ( )
    return <t:number>
end
\stoptyping

When we execute a command the return code is returned. Interpretation is up to
the caller.

\starttyping[option=LUA]
function os.execute ( <t:string> )
    return <t:integer> -- return code
end
\stoptyping

This one enable interpreting \ANSI\ escape sequences in the console. It is only
implemented for \MSWINDOWS. In \CONTEXT\ you can run with \type {--ansi}.

\starttyping[option=LUA]
function os.enableansi ( )
    return <t:boolean>
end
\stoptyping

This one only returns something useful for \MSWINDOWS. One can of course just set
your the system for \UTF8. It's just a reporter meant for debugging issues.

\starttyping[option=LUA]
function os.getcodepage ( )
    return
        <t:integer> oemcodepage,
        <t:integer> applicationcodepage
end
\stoptyping

The \type {os.setenv} function sets a variable in the environment. Passing
\type {nil} instead of a value string will remove the variable.

\starttyping[option=LUA]
function os.setenv (
    <t:string> key,
    <t:string> value
)
    -- no return values
end
\stoptyping

The possible values of \type {os.type} are: \showenginekeylist
{os.gettypevalues()}.

\starttyping[option=LUA]
local currenttype = os.type
\stoptyping

The \type {os.name} string gives a more precise indication of the operating
system. The possible values are: \showenginekeylist {os.getnamevalues()}.

\starttyping[option=LUA]
local currentname = os.name
\stoptyping

On \MSWINDOWS\ the original \type {os.rename}, \type {os.remove} and \type
{os.getenv} functions are replaced by variants that interface to and convert from
\UNICODE16\ to \UTF8.

\stopsubsection

\startsubsection[title=Extra string helpers]

The \type {string} library has gotten a couple of extra functions too, some of
which are iterators. There are some \UNICODE\ related helpers too. When we
started \LUA\ had no \UTF8 function, now it has a few, but we keep using our own,
if only because they were there before. We also add plenty extra functions in the
\type {string} name space at the \LUA\ end.

This first function runs over a string and pick sup single characters:

\starttyping[option=LUA]
for <t:string> c in string.characters ( <t:string> s ) do
    -- some action
end
\stoptyping

\startbuffer
\startluacode
for c in string.characters("τεχ") do
    context("[%02X]",string.byte(c))
end
\stopluacode
\stopbuffer

\typebuffer

gives: \inlinebuffer.

\starttyping[option=LUA]
for <t:string> l, <t:string> r in string.characterpairs ( <t:string> s ) do
    -- some action
end
\stoptyping

\startbuffer
\startluacode
for l, r in string.characterpairs("τεχ") do
    context("[%02X %02X]",string.byte(l),string.byte(r))
end
\stopluacode
\stopbuffer

\typebuffer

gives: \inlinebuffer.

\starttyping[option=LUA]
for <t:string> c in string.utfcharacters( <t:string> s ) do
    -- some action
end
\stoptyping

\startbuffer
\startluacode
for c in string.utfcharacters("τεχ") do
    context("[%s]",c)
end
\stopluacode
\stopbuffer

\typebuffer

gives: \inlinebuffer.

Instead of getting strings back we can also get integers.

\starttyping[option=LUA]
for <t:integer> c in string.bytes ( <t:string> s ) do
    -- some action
end
\stoptyping

\startbuffer
\startluacode
for b in string.bytes("τεχ") do
    context("[%02X]",b)
end
\stopluacode
\stopbuffer

\typebuffer

gives: \inlinebuffer.

\starttyping[option=LUA]
for <t:integer> l, <t:integer> r in string.bytepairs ( <t:string> s ) do
    -- some action
end
\stoptyping

\startbuffer
\startluacode
for l, r in string.bytepairs("τεχ") do
    context("[%02X %02X]",l,r)
end
\stopluacode
\stopbuffer

\typebuffer

gives: \inlinebuffer.

\starttyping[option=LUA]
for <t:integer> u in string.utfvalues( <t:string> s ) do
    -- some action
end
\stoptyping

\startbuffer
\startluacode
for c in string.utfvalues("τεχ") do
    context("[%U]",c)
end
\stopluacode
\stopbuffer

\typebuffer

gives: \inlinebuffer.

The \type {bytetable} function splits a string in bytes.

\starttyping[option=LUA]
function string.bytetable ( <s:string> s) do
    return <t:table> -- with bytes
end
\stoptyping

Here is a line splitter:

\starttyping[option=LUA]
function string.linetable  ( <s:string> s) do
    return <t:table> -- with lines
end
\stoptyping

This one converts an integer (code point) into an \UTF\ string:

\starttyping[option=LUA]
function string.utfcharacter ( <t:string> s )
    return <t:string>
end
\stoptyping

We also have a variant that takes a table. The table can have integers, strings,
and subtables.

\starttyping[option=LUA]
function string.utfcharacter ( <t:table> s )
    return <t:string>
end
\stoptyping

This an \UTF8\ variant of \type {string.byte} and it returns the code points of
the split on the stack.

\starttyping[option=LUA]
function string.utfvalue ( <t:string> s )
    return <t:integer> -- zero or more
end
\stoptyping

Instead of a list on the stack you can get a table:

\starttyping[option=LUA]
function string.utfvaluetable ( <t:string> s )
    return <t:table> -- indexed
end
\stoptyping

The name says it all:

\starttyping[option=LUA]
function string.utflength ( <tr:string> s )
    return <t:integer>
end
\stoptyping

Here we split a string in characters that are collected in an indexed table:

\starttyping[option=LUA]
function string.utfcharactertable ( <t:string> s )
    return <t:table> -- indexed
end
\stoptyping

In \CONTEXT\ we mostly use \type {string.formatters} which is often more
efficient then \type {string.format} and also has additional formatting options,
one being for instance \type {N} which is like \type {f} but strips trailing zero
and returns efficient zeros and ones. Here is a similar low level formatter:

\starttyping[option=LUA]
function string.f6 ( <t:number> n )
    return <t:string>
end

function string.f6 ( <t:number> n,  <t:string> f )
    return <t:string>
end
\stoptyping

In the first case it returns a string with at most 6 digits while the second one
uses given format but tail strips the result.

\starttyping[option=LUA]
function string.tounicode16 ( <t:integer> code  ) return <t:string> end
\stoptyping

\starttyping[option=LUA]
function string.toutf8  ( <t:table>   codes ) return <t:string> end
-------- string.toutf16 ( <t:table>   codes ) return <t:string> end
function string.toutf32 ( <t:table>   codes ) return <t:string> end
\stoptyping

The next one has quite some variation in calling:

\starttyping[option=LUA]
function string.utf16toutf8 ( <t:string> str, <t:true> )
    return <t:string>  -- big endian
end
\stoptyping

\starttyping[option=LUA]
function string.utf16toutf8 ( <t:string> str, <t:false> )
    return <t:string> -- little endian
end
\stoptyping

\starttyping[option=LUA]
function string.utf16toutf8 ( <t:string> str, <t:nil>, <t:true> )
    return <t:string> -- check bom, default to big endian
end
\stoptyping

\starttyping[option=LUA]
function string.utf16toutf8 ( <t:string> str, <t:nil>, <t:false> )
    return <t:string> end -- check bom, default to little endian
end
\stoptyping

\starttyping[option=LUA]
function string.utf16toutf8 ( <t:string> str, <t:nil>, <t:nil> )
    return <t:string> end -- check bom, default to little endian
end
\stoptyping

The next packer is used for creating bitmaps:

\starttyping[option=LUA]
function string.packrowscolumns ( <t:table> data )
    return <t:string>
end
\stoptyping

\startbuffer
\startluacode
local t = {
    { 65, 66, 67 },
    { 68, 69, 70 },
}
context(string.packrowscolumns(t))
\stopluacode
\stopbuffer

For example:

\typebuffer

gives: \inlinebuffer

\startbuffer
\startluacode
local t = {
    { { 114, 103, 98 }, { 114, 103, 98 } },
    { { 114, 103, 98 }, { 114, 103, 98 } },
}

context(string.packrowscolumns(t))
\stopluacode
\stopbuffer

While:

\typebuffer

gives: \inlinebuffer

A string with hexadecimals can be converted with the following. Spaces are
ignored. We use this for instance in the \METAPOST\ potrace interface to
permits nice input.

\starttyping[option=LUA]
function string.hextocharacters ( <t:string> data )
    return <t:string>
end
\stoptyping

\startbuffer
\startluacode
local t = [[
  414243 44 4546 47
  414243 44 4546 47
]]

context(string.hextocharacters(t))
\stopluacode
\stopbuffer

So:

\typebuffer

gives: \inlinebuffer

These take strings and return integers:

\starttyping[option=LUA]
function string.octtointeger ( <t:string> octstr ) return <t:integer> end
function string.dectointeger ( <t:string> decstr ) return <t:integer> end
function string.hextointeger ( <t:string> hexstr ) return <t:integer> end
function string.chrtointeger ( <t:string> chrstr ) return <t:integer> end
\stoptyping

\stopsubsection

\startsubsection[title=Extra table helpers]

This returns the keys of the given table:

\starttyping[option=LUA]
function table.getkeys ( < t:table> )
    return <t:table>
end
\stoptyping

\stopsubsection

\startsubsection[title=Byte encoding and decoding]

We use some helpers from \type {pplib}.

\starttyping[option=LUA]
function basexx.encode16  ( <t:string> str, <t:boolean> newline )
    return <t:string>
end
function basexx.encode64  ( <t:string> str, <t:boolean> newline )
    return <t:string>
end
function basexx.encode85  ( <t:string> str, <t:boolean> newline )
    return <t:string>
end

function basexx.decode16  ( <t:string> str ) return <t:string> end
function basexx.decode64  ( <t:string> str ) return <t:string> end
function basexx.decode85  ( <t:string> str ) return <t:string> end

function basexx.encodeRL  ( <t:string> str ) return <t:string> end
function basexx.decodeRL  ( <t:string> str ) return <t:string> end

function basexx.encodeLZW ( <t:string> str ) return <t:string> end
function basexx.decodeLZW ( <t:string> str ) return <t:string> end
\stoptyping

The last two functions accept an optional bitset with coder flags that we leave
for the user to ponder about. The \type {newline} directive in the first three is
optional.

\stopsubsection

\startsubsection[title=\PNG\ decoding]

These function started out as pure \LUA\ functions (extrapolated from the
descriptions in the standard) but eventually became library helpers. It is
worth noticing that \PDF\ supports \JPEG\ directly so there we can just use
\LUA\ to interpret the file and pass relevant data. Support for \PNG\ is
actually just support for \PNG\ compression, so there we need to do more
work and filter the content:

\starttyping[option=LUA]
function decode.applyfilter (
    <t:string>  data,
    <t:integer> nx,
    <t:integer> ny,
    <t:integer> slice
)
    return <t:string>
end
\stoptyping

We also need to split off the mask as ie becomes a separate object:

\starttyping[option=LUA]
function decode.splitmask (
    <t:string>  data,
    <t:integer> nx,
    <t:integer> ny,
    <t:integer> bpp,
    <t:integer> bytes
)
    return
        <t:string>, -- bitmap
        <t:string>  -- mask
end
\stoptyping

If present we have to deinterlace:

\starttyping[option=LUA]
function decode.interlace (
    <t:string>  data,
    <t:integer> nx,
    <t:integer> ny,
    <t:integer> slice,
    <t:integer> pass
)
    return <t:string>
end
\stoptyping

And maybe expand compressed:

\starttyping[option=LUA]
function decode.expand (
    <t:string>  data,
    <t:integer> nx,
    <t:integer> ny,
    <t:integer> parts,
    <t:integer> xline,
    <t:integer> factor
)
    return <t:string>
end
\stoptyping

These are just helpers that permit integration in the \CONTEXT\ graphic
ecosystem (including \METAPOST):

\starttyping[option=LUA]
function decode.tocmyk ( <t:string data )
    return <t:string>
end
\stoptyping

For usage see the \CONTEXT\ sources.

\starttyping[option=LUA]
function decode.tomask (
    <t:string>  content,
    <t:string>  mapping,
    <t:integer> xsize,
    <t:integer> ysize,
    <t:integer> colordepth
)
    return <t:string>
end
\stoptyping

There are to variants:

\starttyping[option=LUA]
function decode.makemask (
    <t:string>  content,
    <t:integer> mapping
)
    return <t:string>
end

function decode.makemask (
    <t:string> content,
    <t:table>  mapping
)
    return <t:string>
end
\stoptyping

\stopsubsection

\startsubsection[title=MD5 hashing]

In the meantime we use some helpers from \type {pplib} because we have that
anyway. These are useful when we need a reasonable unique hash of limited length:

\starttyping[option=LUA]
function md5.sum ( <t:string> ) return <t:string> end
function md5.hex ( <t:string> ) return <t:string> end
function md5.HEX ( <t:string> ) return <t:string> end
\stoptyping

\startbuffer
\startluacode
context.type(md5.HEX("normally this is unique enough"))
\stopluacode
\stopbuffer

Using a hexadecimal representation of the 16 byte calculated checksum is less
sensitive for escaping. This:

\typebuffer

gives: \inlinebuffer.

\stopsubsection

\startsubsection[title=SHA2 hashing]

Because \type {pplib} comes with some SHA2 support we can borrow its helpers
instead of the \LUA\ code we used before (which was anyway fun to write).

\starttyping[option=LUA]
function sha2.digest256 ( <t:string> data ) return <t:string> end
function sha2.digest384 ( <t:string> data ) return <t:string> end
function sha2.digest512 ( <t:string> data ) return <t:string> end
function sha2.sum256    ( <t:string> data ) return <t:string> end
function sha2.sum384    ( <t:string> data ) return <t:string> end
function sha2.sum512    ( <t:string> data ) return <t:string> end
function sha2.hex256    ( <t:string> data ) return <t:string> end
function sha2.hex384    ( <t:string> data ) return <t:string> end
function sha2.hex512    ( <t:string> data ) return <t:string> end
function sha2.HEX256    ( <t:string> data ) return <t:string> end
function sha2.HEX384    ( <t:string> data ) return <t:string> end
function sha2.HEX512    ( <t:string> data ) return <t:string> end
\stoptyping

\startbuffer
\startluacode
context.type(sha2.HEX256("normally this is unique enough"))
\stopluacode
\stopbuffer

The number refers to bytes, so with 256 we get a 32 byte hash that we show in
hexadecimal because that is less sensitive for escaping:

\typebuffer

gives: \inlinebuffer.

\stopsubsection

\startsubsection[title=AES encryption]

In the next encryption functions the key should be 16, 24 or 32 bytes long.

\starttyping[option=LUA]
function aes.encode (
    <t:string> data,
    <t:string> key
)
    return <t:string>
end

function aes.decode (
    <t:string> data,
    <t:string> key
)
    return <t:string>
end
\stoptyping

This returns a string. The default length is 16; the optional length is limited
to 32.

\starttyping[option=LUA]
function aes.random ( <t:integer> length )
    return <t:string>
end
\stoptyping

Here is an example:

\startbuffer
\startluacode
context.type ( basexx.encode16 ( aes.encode (
    "normally this is unique enough",
    "The key of live!"
) ) )
\stopluacode
\stopbuffer

\typebuffer

This gives: \inlinebuffer, where we hexed the result because
it is unlikely to be valid \UTF8.

\stopsubsection

\startsubsection[title=ZIP (de)compression]

We provide the minimum needed to support compression in the backend but even this
limited set makes it possible to implement a zip file compression utility which
is indeed what we do in \CONTEXT. We use \type {minizip} as codebase, without the
zip utility code. The meaning and application of the various arguments can be
found (and are better explained) on the internet.

\starttyping[option=LUA]
function xzip.compress (
    <t:string>  data,
    <t:integer> compresslevel,
    <t:integer> method,
    <t:integer> window,
    <t:integer> memory,
    <t:integer> strategy
)
    return <t:string>
end

function xzip.compresssize (
    <t:string>  data,
    <t:integer> buffersize,
    <t:integer> compresslevel,
    <t:integer> window
)
    return <t:string>
end

function xzip.decompress (
    <t:string>  data,
    <t:integer> window
)
    return <t:string>
end

function xzip.decompresssize (
    <t:string>  data,
    <t:integer> targetsize,
    <t:integer> window
)
    return <t:string>
end

function xzip.adler32 (
    <t:string>  buffer,
    <t:integer> checksum
)
    return <t:integer>
end

function xzip.crc32 (
    <t:string>  buffer,
    <t:integer> checksum
)
    return <t:integer>
end
\stoptyping

\stopsubsection

\startsubsection[title=Potrace]

The excellent potrace manual explains everything about this library therefore
here we just show the interface. Possible fields in specification are:
\showenginekeylist {table.sortedkeys (potrace.getnewfields ())}

\starttyping[option=LUA]
function potrace.new ( <t:table> specification )
    return <t:userdata> -- instance
end
\stoptyping


\starttyping[option=LUA]
function potrace.free ( <t:userdata> instance)
    -- no return values
end
\stoptyping

The process is controlled by the specification: \showenginekeylist {table.sortedkeys
(potrace.getprocessfields ())}, where permitted policy values are \showenginekeylist
{potrace.getpolicyvalues()}.

\starttyping[option=LUA]
function potrace.process ( <t:userdata> instance, <t:table> specification )
    return <t:boolean> -- success
end
\stoptyping

Results are collected in a table that we can feed into \METAPOST, The table has
subtables per traced shape and these contain indexed tables with two (pair) or
six (curve) entries. There is a boolean \type {sign} field and an integer \type
{index} field. In the next function only the first argument is mandate.

\starttyping[option=LUA]
function potrace.totable  (
    <t:userdata> instance,
    <t:boolean>  debug,
    <t:integer>  first,
    <t:integer>  last
)
    return <t:table>
end
\stoptyping

\stopsubsection

\startsubsection[title=Sparse hashes]

The sparse library is just there because we use similar code to store all these
character related codes that way (\type {\lccode}) and such). The entries can be
1 (\type {0xFF}), 2 (\type {0xFFFF}) or 4 (\type {0xFFFFFFFF}) bytes wide. When 0
is used as width then nibbles (\type {0xF}) are assumed.

\starttyping[option=LUA]
function sparse.new (
    <t:integer> bytes,
    <t:integer> default
)
    return <t:userdata>
end
\stoptyping

You set a value by index. Optionally there can be the \type {"global"} keyword
before the second argument.

\starttyping[option=LUA]
function sparse.set (
    <t:userdata> instance,
    <t:integer>  index,
    <t:integer>  value
)
    return <t:integer>
end
\stoptyping

We get back integers as that is what we store:

\starttyping[option=LUA]
function sparse.get ( <t:userdata> instance ) return <t:integer> end
function sparse.min ( <t:userdata> instance ) return <t:integer> end
function sparse.max ( <t:userdata> instance ) return <t:integer> end
\stoptyping

The range is fetched with:

\starttyping[option=LUA]
function sparse.range ( <t:userdata> instance )
    return
        <t:integer>, -- min
        <t:integer>  -- max
end
\stoptyping

We can iterate over the hash:

\starttyping[option=LUA]
for
    <t:integer> index,
    <t:integer> value
in sparse.traverse (
    <t:userdata> instance
) do
    -- actions
end
\stoptyping

This is a somewhat strange one but it permits packing all values in a string.
It's another way to create bitmaps.

\starttyping[option=LUA]
function sparse.concat (
    <t:userdata> instance
    <t:integer>  min,
    <t:integer>  max,
    <t:integer>  how  -- 0=byte, 1=lsb 2=msb
)
    return <t:string>
end
\stoptyping

Setting values obeys grouping in \TEX, but we can restore any time:

\starttyping[option=LUA]
function sparse.restore ( <t:userdata> instance )
    -- nothing to return
end
\stoptyping

We can also wipe all values:

\starttyping[option=LUA]
function sparse.wipe (<t:userdata> instance )
    -- nothing to return
end
\stoptyping

\stopsubsection

\startsubsection[title=Posits]

We implement posits as userdata . We use the library from the posit team,
although it is not complete so we might roll out our own variant (as we need less
anyway). The advance of userdata is that we can use the binary and relation
operators.

Here are the housekeeping functions. Some are more tolerant with respect to
arguments, take the allocator:

\starttyping[option=LUA]
function posit.new (              ) return <t:posit> end
function posit.new ( <t:string> s ) return <t:posit> end
function posit.new ( <t:number> n ) return <t:posit> end
\stoptyping

When a posit is expected a number or string is also accepted which is then
converted to a posit.

\starttyping[option=LUA]
function posit.copy      ( <t:posit>  p ) return <t:posit>   end
function posit.tostring  ( <t:posit>  p ) return <t:string>  end
function posit.tonumber  ( <t:posit>  p ) return <t:number>  end
function posit.integer   ( <t:posit>  p ) return <t:integer  end
function posit.rounded   ( <t:posit>  p ) return <t:integer> end
function posit.toposit   ( <t:number> n ) return <t:posit>   end
function posit.fromposit ( <t:posit>  p ) return <t:number>  end
\stoptyping

\starttyping[option=LUA]
function posit.NaN ( <t:posit> p ) return <t:boolean> end
function posit.NaR ( <t:posit> p ) return <t:boolean> end
\stoptyping

Here are the logical operators:

\starttyping[option=LUA]
function posit.bor  ( <t:posit> p1, <t:posit> p2 ) return <t:posit> end
function posit.bxor ( <t:posit> p1, <t:posit> p2 ) return <t:posit> end
function posit.band ( <t:posit> p1, <t:posit> p2 ) return <t:posit> end
\stoptyping

Ans shifters:

\starttyping[option=LUA]
function posit.shift  ( <t:posit> p1, <t:integer> n ) return <t:posit> end
function posit.rotate ( <t:posit> p,  <t:integer> n ) return <t:posit> end
\stoptyping

There is a limited repertoire of math functions (basically what we needed for
\METAPOST):

\starttyping[option=LUA]
function posit.abs   ( <t:posit> p ) return <t:posit> end
function posit.conj  ( <t:posit> p ) return <t:posit> end
function posit.acos  ( <t:posit> p ) return <t:posit> end
function posit.asin  ( <t:posit> p ) return <t:posit> end
function posit.atan  ( <t:posit> p ) return <t:posit> end
function posit.ceil  ( <t:posit> p ) return <t:posit> end
function posit.cos   ( <t:posit> p ) return <t:posit> end
function posit.exp   ( <t:posit> p ) return <t:posit> end
function posit.exp2  ( <t:posit> p ) return <t:posit> end
function posit.floor ( <t:posit> p ) return <t:posit> end
function posit.log   ( <t:posit> p ) return <t:posit> end
function posit.log10 ( <t:posit> p ) return <t:posit> end
function posit.log1p ( <t:posit> p ) return <t:posit> end
function posit.log2  ( <t:posit> p ) return <t:posit> end
function posit.logb  ( <t:posit> p ) return <t:posit> end
function posit.round ( <t:posit> p ) return <t:posit> end
function posit.sin   ( <t:posit> p ) return <t:posit> end
function posit.sqrt  ( <t:posit> p ) return <t:posit> end
function posit.tan   ( <t:posit> p ) return <t:posit> end

function posit.modf ( <t:posit> p )
    return
        <t:posit>,
        <t:posit>
end

function posit.min ( <t:posit> p1, <t:posit> p2 ) return <t:posit> end
function posit.max ( <t:posit> p1, <t:posit> p2 ) return <t:posit> end
function posit.pow ( <t:posit> p1, <t:posit> p2 ) return <t:posit> end
\stoptyping

% unsupported : acosh asinh atan2 atanh cbrt copysign cosh deg erf erfc expm1
%               fabs fdim fma fmax fmin fmod frexp gamma hypot isfinite isinf
%               isnan isnormal j0 j1 jn ldexp lgamma modf nearbyint nextafter rad
%               remainder remquo scalbn sinh tanh tgamma trunc y0 y1 yn
% supported   : add sub mul div unm pow / eq le lt / bor bxor band shl shr
% unsupported : idiv mod

\stopsubsection

\startsubsection[title=Complex numbers]

\starttyping[option=LUA]
function xcomplex.new ( )
    return <t:complex>
end

function xcomplex.new (
    <t:number> re,
    <t:number> im
)
    return <t:complex>
end
\stoptyping

\starttyping[option=LUA]
function xcomplex.tostring ( <t:complex> z )
    return <t:string>
end
\stoptyping

\starttyping[option=LUA]
function xcomplex.topair ( <t:complex> z )
    return
        <t:number>, -- re
        <t:number>  -- im
end
\stoptyping

There is a bunch of functions that take a complex number:

\starttyping[option=LUA]
function xcomplex.abs   ( <t:complex> z ) return <t:complex> end
function xcomplex.arg   ( <t:complex> z ) return <t:complex> end
function xcomplex.imag  ( <t:complex> z ) return <t:complex> end
function xcomplex.real  ( <t:complex> z ) return <t:complex> end
function xcomplex.onj   ( <t:complex> z ) return <t:complex> end
function xcomplex.proj  ( <t:complex> z ) return <t:complex> end
function xcomplex.exp   ( <t:complex> z ) return <t:complex> end
function xcomplex.log   ( <t:complex> z ) return <t:complex> end
function xcomplex.sqrt  ( <t:complex> z ) return <t:complex> end
function xcomplex.sin   ( <t:complex> z ) return <t:complex> end
function xcomplex.cos   ( <t:complex> z ) return <t:complex> end
function xcomplex.tan   ( <t:complex> z ) return <t:complex> end
function xcomplex.asin  ( <t:complex> z ) return <t:complex> end
function xcomplex.acos  ( <t:complex> z ) return <t:complex> end
function xcomplex.atan  ( <t:complex> z ) return <t:complex> end
function xcomplex.sinh  ( <t:complex> z ) return <t:complex> end
function xcomplex.cosh  ( <t:complex> z ) return <t:complex> end
function xcomplex.tanh  ( <t:complex> z ) return <t:complex> end
function xcomplex.asinh ( <t:complex> z ) return <t:complex> end
function xcomplex.acosh ( <t:complex> z ) return <t:complex> end
function xcomplex.atanh ( <t:complex> z ) return <t:complex> end

function xcomplex.pow   ( <t:complex> z1, <t:complex> z2 ) return <t:complex> end
\stoptyping

We added the \type {cerf} functions but none can wonder if we should carry that
burden around (instead of just assuming a library to be used).

The complex error function \type {erf(z)}:

\starttyping[option=LUA]
function cerf.erf ( <t:complex> z )
    return <t:complex>
end
\stoptyping

The complex complementary error function \type {erfc(z) = 1 - erf(z)}:

\starttyping[option=LUA]
function cerf.erfc ( <t:complex> z )
    return <t:complex>
end
\stoptyping

The underflow-compensating function \type {erfcx(z) = exp(z^2) erfc(z)}:

\starttyping[option=LUA]
function cerf.erfcx ( <t:complex> z )
    return <t:complex>
end
\stoptyping

The imaginary error function \type {erfi(z) = -i erf(iz)}:

\starttyping[option=LUA]
function cerf.erfi ( <t:complex> z )
    return <t:complex>
end
\stoptyping

Dawson's integral \type {D(z) = sqrt(pi)/2 * exp(-z^2) * erfi(z)}:

\starttyping[option=LUA]
function cerf.dawson ( <t:complex> z )
    return <t:complex>
end
\stoptyping

The convolution of a Gaussian and a Lorentzian:

\starttyping[option=LUA]
function cerf.voigt (
    <t:number> n1,
    <t:number> n2,
    <t:number> n3
)
    return <t:number>
end
\stoptyping

The half width at half maximum of the Voigt profile:

\starttyping[option=LUA]
function cerf.voigt_hwhm (
    <t:number> n1,
    <t:number> n2
)
    return <t:number>
end
\stoptyping

\stopsubsection

\startsubsection[title=Decimal numbers]

Because in \METAPOST\ we support the decimal number system, we also provide this
at the \TEX\ end Apart from the usual support for operators there are some
functions available.

\starttyping[option=LUA]
function xdecimal.new ( )              return <t:decimal> end
function xdecimal.new ( <t:number> n ) return <t:decimal> end
function xdecimal.new ( <t:string> s ) return <t:decimal> end
\stoptyping

\starttyping[option=LUA]
function xdecimal.copy     ( <t:decimal> a ) return <t:decimal> end
function xdecimal.tostring ( <t:decimal> a ) return <t:string>  end
function xdecimal.tonumber ( <t:decimal> a ) return <t:number>  end
\stoptyping

\starttyping[option=LUA]
function xdecimal.setprecision ( <t:integer> digits )
    --nothing to return
end

function xdecimal.getprecision ( )
    return <t:integer>
end
\stoptyping

\starttyping[option=LUA]
function xdecimal.bor  ( <t:decimal> a, <t:decimal> b ) return <t:decimal> end
function xdecimal.bxor ( <t:decimal> a, <t:decimal> b ) return <t:decimal> end
function xdecimal.band ( <t:decimal> a, <t:decimal> b ) return <t:decimal> end
\stoptyping

\starttyping[option=LUA]
function xdecimal.shift  ( <t:decimal> a, <t:integer> n ) return <t:decimal> end
function xdecimal.rotate ( <t:decimal> a, <t:integer> n ) return <t:decimal> end
\stoptyping

\starttyping[option=LUA]
function xdecimal.abs   ( <t:decimal> a ) return <t:decimal> end
function xdecimal.trim  ( <t:decimal> a ) return <t:decimal> end
function xdecimal.conj  ( <t:decimal> a ) return <t:decimal> end
function xdecimal.abs   ( <t:decimal> a ) return <t:decimal> end
function xdecimal.sqrt  ( <t:decimal> a ) return <t:decimal> end
function xdecimal.ln    ( <t:decimal> a ) return <t:decimal> end
function xdecimal.log   ( <t:decimal> a ) return <t:decimal> end
function xdecimal.exp   ( <t:decimal> a ) return <t:decimal> end
function xdecimal.minus ( <t:decimal> a ) return <t:decimal> end
function xdecimal.plus  ( <t:decimal> a ) return <t:decimal> end

function xdecimal.min ( <t:decimal> a, <t:decimal> b ) return <t:decimal> end
function xdecimal.max ( <t:decimal> a, <t:decimal> b ) return <t:decimal> end
function xdecimal.pow ( <t:decimal> a, <t:decimal> b ) return <t:decimal> end
\stoptyping

\stopsubsection

\startsubsection[title=Math helpers]

The \type {xmath} library provides function and a few constants:

\starttyping[option=LUA]
local infinty    = xmath.inf
local notanumber = xmath.nan
local pi         = xmath.pi
\stoptyping

There are more helpers than the average used needs. We also use these
to extend the \METAPOST\ repertoire.

\starttyping[option=LUA]
function xmath.acos       ( <t:number> a )               return <t:number> end
function xmath.acosh      ( <t:number> a )               return <t:number> end
function xmath.asin       ( <t:number> a )               return <t:number> end
function xmath.asinh      ( <t:number> a )               return <t:number> end
function xmath.atan       ( <t:number> a )               return <t:number> end
function xmath.atan       ( <t:number> a, <t:number> b ) return <t:number> end
function xmath.atan2      ( <t:number> a )               return <t:number> end
function xmath.atan2      ( <t:number> a, <t:number> b ) return <t:number> end
function xmath.atanh      ( <t:number> a )               return <t:number> end
function xmath.cbrt       ( <t:number> a )               return <t:number> end
function xmath.ceil       ( <t:number> a )               return <t:number> end
function xmath.copysign   ( <t:number> a, <t:number> b ) return <t:number> end
function xmath.cos        ( <t:number> a )               return <t:number> end
function xmath.cosh       ( <t:number> a )               return <t:number> end
function xmath.deg        ( <t:number> a )               return <t:number> end
function xmath.erf        ( <t:number> a )               return <t:number> end
function xmath.erfc       ( <t:number> a )               return <t:number> end
function xmath.exp        ( <t:number> a )               return <t:number> end
function xmath.exp2       ( <t:number> a )               return <t:number> end
function xmath.expm1      ( <t:number> a )               return <t:number> end
function xmath.fabs       ( <t:number> a )               return <t:number> end
function xmath.fdim       ( <t:number> a, <t:number> b ) return <t:number> end
function xmath.floor      ( <t:number> a )               return <t:number> end
function xmath.fmax       ( ... )                        return <t:number> end
function xmath.fmin       ( ... )                        return <t:number> end
function xmath.fmod       ( <t:number> a, <t:number> b ) return <t:number> end
function xmath.frexp      ( <t:number> a, <t:number> b ) return <t:number> end
function xmath.gamma      ( <t:number> a )               return <t:number> end
function xmath.hypot      ( <t:number> a, <t:number> b ) return <t:number> end
function xmath.isfinite   ( <t:number> a )               return <t:number> end
function xmath.isinf      ( <t:number> a )               return <t:number> end
function xmath.isnan      ( <t:number> a )               return <t:number> end
function xmath.isnormal   ( <t:number> a )               return <t:number> end
function xmath.j0         ( <t:number> a )               return <t:number> end
function xmath.j1         ( <t:number> a )               return <t:number> end
function xmath.jn         ( <t:number> a, <t:number> b ) return <t:number> end
function xmath.ldexp      ( <t:number> a, <t:number> b ) return <t:number> end
function xmath.lgamma     ( <t:number> a )               return <t:number> end
function xmath.l0         ( <t:number> a )               return <t:number> end
function xmath.l1         ( <t:number> a )               return <t:number> end
function xmath.ln         ( <t:number> a, <t:number> b ) return <t:number> end
function xmath.log        ( <t:number> a [,b])           return <t:number> end
function xmath.log10      ( <t:number> a )               return <t:number> end
function xmath.log1p      ( <t:number> a )               return <t:number> end
function xmath.log2       ( <t:number> a )               return <t:number> end
function xmath.logb       ( <t:number> a )               return <t:number> end
function xmath.modf       ( <t:number> a, <t:number> b ) return <t:number> end
function xmath.nearbyint  ( <t:number> a )               return <t:number> end
function xmath.nextafter  ( <t:number> a, <t:number> b ) return <t:number> end
function xmath.pow        ( <t:number> a, <t:number> b ) return <t:number> end
function xmath.rad        ( <t:number> a )               return <t:number> end
function xmath.remainder  ( <t:number> a, <t:number> b ) return <t:number> end
function xmath.remquo     ( <t:number> a, <t:number> b ) return <t:number> end
function xmath.round      ( <t:number> a )               return <t:number> end
function xmath.scalbn     ( <t:number> a, <t:number> b ) return <t:number> end
function xmath.sin        ( <t:number> a )               return <t:number> end
function xmath.sinh       ( <t:number> a )               return <t:number> end
function xmath.sqrt       ( <t:number> a )               return <t:number> end
function xmath.tan        ( <t:number> a )               return <t:number> end
function xmath.tanh       ( <t:number> a )               return <t:number> end
function xmath.tgamma     ( <t:number> a )               return <t:number> end
function xmath.trunc      ( <t:number> a )               return <t:number> end
function xmath.y0         ( <t:number> a )               return <t:number> end
function xmath.y1         ( <t:number> a )               return <t:number> end
function xmath.yn         ( <t:number> a )               return <t:number> end

function xmath.fma (
    <t:number> a,
    <t:number> b,
    <t:number> c
)
    return <t:number>
end
\stoptyping

\stopsubsection

\stopsection

\startsection[title={Optional}]

\startsubsection[title=Loading]

The optional libraries are (indeed) optional. Compilation of \LUAMETATEX\ doesn't
depend on them being present. Loading (and binding) is delayed. In practice we
only see a few being of interest and used, like \type {zint} for barcodes, \type
{mysql} for database processing and \type {graphicmagick} for an occasional
runtime conversion. Some are just there to show the principles and were used to
test the interfaces and loading.

A library can be loaded, and thereby registered in the \quote {optional}
namespace, assuming that \type {--permitloadlib} is given with:

\starttyping[option=LUA]
function library.load (
    <t:string> filename,
    <t:string> openname,
)
    return
        <t:function>, -- target
        <t:string>    -- foundname
end
\stoptyping

but there are no guarantees that it will work.

\stopsubsection

\startsubsection[title=Management]

{\em Todo: something about how optionals are implemented and are supposed to work.}

\stopsubsection

\startsubsection[title=TDS (kpse)]

The optional \type {kpse} library deals with lookups in the \TEX\ Directory
Structure and before it can be used it has to be initialized:

\starttyping[option=LUA]
function optional.kpse.initialize ( <t:string> filename )
    return <t:boolean>
end
\stoptyping

By setting the program name the library knows in what namespace to resolve
filenames and variables.

\starttyping[option=LUA]
function optional.kpse.set_program_name (
    <t:string> binaryname,
    <t:string> programname
)
    -- no return values
end
\stoptyping

The main finder has one or more arguments. When the second and later arguments
can be a boolean, string or number. The boolean indicates if the file must exist.
A string sets the file type and a number does the same.

\starttyping[option=LUA]
function optional.kpse.find_file(
    <t:string>  filename,
    <t:string>  filetype.
    <t:boolean> mustexist
)
    return <t:string>
end
\stoptyping

You can also ask for a list of found files:

\starttyping[option=LUA]
function optional.kpse.find_files (
    <t:string> userpath,
    <t:string> filename
)
    return <t:table>
end
\stoptyping

These return variables, values and paths:

\starttyping[option=LUA]
function optional.kpse.expand_path   ( <t:string> name ) return <t:string> end
function optional.kpse.expand_var    ( <t:string> name ) return <t:string> end
function optional.kpse.expand_braces ( <t:string> name ) return <t:string> end
function optional.kpse.var_value     ( <t:string> name ) return <t:string> end
\stoptyping

If possible this returns the (first found) filename that is readable:

\starttyping[option=LUA]
function optional.kpse.readable_file ( <t:string> filename )
    return <t:string>
end
\stoptyping

The list of supported file types can be fetched with:

\starttyping[option=LUA]
function optional.kpse.get_file_types ( )
    return <t:table>
end
\stoptyping

\stopsubsection

\startsubsection[title=Graphics]

\startsubsubsubject[title=ghostscript]

The \type {ghostscript} library has to be initialized:

\starttyping[option=LUA]
function optional.ghostscript.initialize ( <t:string> filename )
    return <t:boolean>
end
\stoptyping

A conversion is executed with the following command. Here the table is a mixed
list of strings and numbers that represent the otherwise command like arguments.

\starttyping[option=LUA]
function optional.ghostscript.execute ( <t:table> )
    return
        <t:boolean>, -- success
        <t:string>,  -- result
        <t:string>   -- message
end
\stoptyping

\stopsubsubsubject

\startsubsubsubject[title=graphicsmagick]

The \type {graphicsmagick} library has to be initialized:

\starttyping[option=LUA]
function optional.graphicsmagick.initialize ( <t:string> filename )
    return <t:boolean>
end
\stoptyping

A conversion is executed with the following command.

\starttyping[option=LUA]
function optional.graphicsmagick.execute (
    {
        inputfilename  = <t:string>,
        outputfilename = <t:string>,
        blur           = {
            radius = <t:number>,
            sigma  = <t:number>,
        },
        noise          - {
            type   = <t:integer>,
        },
    }
)
    return <t:boolean>
end
\stoptyping

The noise types can be fetched with:

\starttyping[option=LUA]
function optional.graphicsmagick.noisetypes ( )
    return <t:table>
end
\stoptyping

\stopsubsubsubject

\startsubsubsubject[title=imagemagick]

The \type {imagemagick} library is initialized with:

\starttyping[option=LUA]
function optional.imagemagick.initialize ( <t:string> filename )
    return <t:boolean>
end
\stoptyping

After that you can execute convert commands. The options table is a sequence of
strings, numbers and booleans that gets passes, in the same order, but where a
boolean becomes one of the strings \type {true} or \type {false}.

\starttyping[option=LUA]
function optional.imagemagick.execute (
    {
        inputfilename  = <t:string>,
        outputfilename = <t:string>,
        options        = <t:table>,
    }
)
    return <t:boolean>
end
\stoptyping

\stopsubsubsubject

\startsubsubsubject[title=zint]

The \type {zint} library is initialized with:

\starttyping[option=LUA]
function optional.zint.initialize ( <t:string> filename )
    return <t:boolean>
end
\stoptyping

As with the other graphic libraries we execute a command but here we implement a
converter a bit more specific because we want back a result that we can handle in
a combination of \TEX\ and \METAPOST.

\starttyping[option=LUA]
function optional.zint.execute (
    {
        code   = <t:integer>,
        text   = <t:string>,
        option = <t:string>, -- "square"
    }
)
    return <t:table>
end
\stoptyping

We get back a table that has graphic components, where each components table can
zero or more subtables.

\starttyping[option=LUA]
result = {
    rectangles = {
        { <t:integer> x, <t:integer> y, <t:integer> w, <t:integer> h }, ...
    },
    hexagons = {
        { <t:integer> x, <t:integer> y, <t:integer> d }, ...
    },
    circles = {
        { <t:integer> x, <t:integer> y, <t:nteger> d }, ...
    },
    strings = {
        { <t:integer> x, <t:integer> y, <t:integer> s, <t:string> t }, ...
    }
}
\stoptyping

\stopsubsubsubject

\stopsubsection

\startsubsection[title=Compression]

\startsubsubsubject[title=lz4]

The library is initialized with:

\starttyping[option=LUA]
function optional.lz4.initialize ( )
    return <t:boolean> -- success
end
\stoptyping

There are compressors and decompressors. If you want the more efficient
decompressor, make sure to save the size with the compressed stream and pass that
when decompressing.

\starttyping[option=LUA]
function optional.lz4.compress (
    <t:string>  data,
    <t:integer> acceleration -- default 1
)
    return <t:string>
end

function optional.lz4.decompresssize (
    <t:string>  data,
    <t:integer> size
)
    return <t:string>
end
\stoptyping

These are the frame based variants:

\starttyping[option=LUA]
function optional.lz4.framecompress ( <t:string> data )
    return <t:string>
end

function optional.lz4.framedecompress ( return <t:string>  )
    return <t:string>
end
\stoptyping

\stopsubsubsubject

\startsubsubsubject[title=lzma]

The library is initialized with:

\starttyping[option=LUA]
function optional.lzma.initialize ( )
    return <t:boolean> -- success
end
\stoptyping

The compressor can take an estimated size which makes it possible to preallocate
a buffer.

\starttyping[option=LUA]
function optional.lzma.compress (
    <t:string>  data,
    <t:integer> level,
    <t:integer> size   -- estimated
)
    return <t:string>
end
\stoptyping

The decompressor can be told what the final size is which is more efficient.

\starttyping[option=LUA]
function optional.lzma.decompress (
    <t:string>  data,
    <t:integer> size   -- estimated
)
    return <t:string>
end
\stoptyping

\stopsubsubsubject

\startsubsubsubject[title=lzo]

The library is initialized with:

\starttyping[option=LUA]
function optional.lzo.initialize ( )
    return <t:boolean> -- success
end
\stoptyping

There is not much to tell about:

\starttyping[option=LUA]
function optional.lzo.compress ( <t:string> data )
    return <t:string>
end
\stoptyping

and

\starttyping[option=LUA]
function optional.lzo.decompresssize (
    <t:string>  data,
    <t:integer> size
)
    return <t:string>
end
\stoptyping

\stopsubsubsubject

\startsubsubsubject[title=zstd]

The library is initialized with:

\starttyping[option=LUA]
function optional.zstd.initialize ( )
    return <t:boolean> -- success
end
\stoptyping

The compressor:

\starttyping[option=LUA]
function optional.zstd.compress (
    <t:string>  data,
    <t:integer> level
)
    return <t:string>
end
\stoptyping

The decompressor:

\starttyping[option=LUA]
function optional.zstd.decompress ( <t:string> data )
    return <t:string>
end
\stoptyping

\stopsubsubsubject

\stopsubsection

\startsubsection[title=Databases]

\startsubsubsubject[title=mysql]

We start with the usual initializer:

\starttyping[option=LUA]
function optional.mysql.initialize ( )
    return <t:boolean> -- success
end
\stoptyping

Opening the database is done with:

\starttyping[option=LUA]
function optional.mysql.open (
    <t:string>  database,
    <t:string>  username,
    <t:string>  password,
    <t:string>  host,
    <t:integer> port -- optional
)
    return <t:userdata> -- instance
end
\stoptyping

The database is kept \quote {open} but can be closed with:

\starttyping[option=LUA]
function optional.mysql.close ( <t:userdata> instance )
    -- no return values
end
\stoptyping

A query is executed with:

\starttyping[option=LUA]
function optional.mysql.execute (
    <t:userdata> instance,
    <t:string>   query,
    <t:function> callback
)
    return <t:boolean> -- success
end
\stoptyping

The callback is a \LUA\ function that looks like this:

\starttyping[option=LUA]
function callback(nofcolumns,values,fields)
    ...
end
\stoptyping

It gets called for every row of the result. The fields table is only filled the
first time, if at all.

This interface is rather minimalistic but in \CONTEXT\ we wrap all in a more
advanced setup. It's among the oldest \LUA\ code in the distribution and evolved
with the possibilities (client as well as external libraries) and is quite
performing also due to the use of templates, caching, built-in conversions etc.

If there is an error we can fetch the message with:

\starttyping[option=LUA]
function optional.mysql.getmessage ( <t:userdata> instance )
    return <t:string> | <t:nil> -- last error message
end
\stoptyping

\stopsubsubsubject

\startsubsubsubject[title=postgress]

This library has the same interface as the \type {mysql} interface, so it can be
used instead.

\stopsubsubsubject

\startsubsubsubject[title=sqlite]

This library has the same interface as the \type {mysql} interface, so it can be
used instead. The only function that differs is the opener:

\starttyping[option=LUA]
function optional.sqlite.open ( <t:string> filename )
    return <t:userdata> -- instance
end
\stoptyping

\stopsubsubsubject

\stopsubsection

\startsubsection[title=Whatever]

\startsubsubsubject[title=cerf]

This library is plugged in the \type {xcomplex} so there is no need to discuss it
here unless we decide to move it to an optional loaded library, which might
happen eventually (depends on need).

\stopsubsubsubject

\startsubsubsubject[title=curl]

The library is initialized with:

\starttyping[option=LUA]
function optional.curl.initialize ( )
    return <t:boolean> -- success
end
\stoptyping

The fetcher stays kind of close to how the library wants it so we have no fancy
interface. We have pairs where the first member is an integer indicating the
option. The library only has string and integer options so booleans are effective
zeros or ones. A \LUA\ boolean therefore becomes an integer.

\starttyping[option=LUA]
function optional.curl.fetch (
    {
        <t:integer>, <t:string> | <t:integer> | <t:boolean>,
        ...
    }
)
end
\stoptyping

A \URL\ can be (un)escaped:

\starttyping[option=LUA]
function optional.curl.escape ( <t:string> data )
    return <t:string>
end

function optional.curl.unescape ( <t:string> data )
    return <t:string>
end
\stoptyping

The current version of the library:

\starttyping[option=LUA]
function optional.curl.getversion ( )
    return <t:string>
end
\stoptyping

\stopsubsubsubject

\startsubsubsubject[title=hb]

This module is mostly there to help Idris Hamid (The Oriental \TEX\ Project
develop his fonts in such away that they work with other libraries (also
uniscribe). We need to initialize this library with the following function. Best
have the library in the \TEX\ tree because either more are present or the
operating system updates them. As we don't use this in \CONTEXT\ we're also not
sure of things work ok but we can assume stable interfaces anyway. See the plugin
module for more info.

\starttyping[option=LUA]
function optional.hb.initialize ( )
    return <t:boolean> -- success
end
\stoptyping

It probably makes sense to check for the version because (in the \TEXLIVE\ code
base) it is one of the most frequently updated code bases and for \TEX\ stability
and predictability (when working on a specific project) is important. When you
initialize

\starttyping[option=LUA]
function optional.hb.getversion ( )
    return <t:string>
end
\stoptyping

\starttyping[option=LUA]
function optional.hb.getshapers ( )
    return <t:table> -- strings
end
\stoptyping

\starttyping[option=LUA]
function optional.hb.loadfont (
    <t:integer> id,
    <t:string>  name
)
    return <t:userdata> -- instance
end
\stoptyping

A run over characters happens with the next one. You get back a list of tables
that specify to be handled glyphs. The interface is pretty much the same as what
Kai Eigner came up with at the time he wanted to compare the results with the
regular font loader, for which the \LUATEX\ and \LUAJITTEX) \FFI\ interfaces were
used.

\starttyping[option=LUA]
function optional.hb.shapestring (
    <t:userdata> font,
    <t:string>   script,
    <t:string>   language,
    <t:string>   direction,
    <t:table>    shapers,
    <t:table>    features,
    <t:string>   text
    <t:boolean>  reverse
    <t:integer>  utfbits, -- default 8
)
    return {
        {
            <t:integer>, -- codepoint
            <t:integer>, -- cluster
            <t:integer>, -- xoffset
            <t:integer>, -- yoffset
            <t:integer>, -- xadvance
            <t:integer>, -- uadvance
        },
        ...
    }
end
\stoptyping

\stopsubsubsubject

\startsubsubsubject[title=mujs]

This is just a fun experiment that permits \JAVASCRIPT\ code to be used instead of
\LUA. It was actually one of the first optional libraries I played with and as
with the other optionals there is a module that wraps it. The library is
initialized with:

\starttyping[option=LUA]
function optional.mujs.initialize ( )
    return <t:boolean> -- success
end
\stoptyping

There are a few \quote {mandate} callbacks than need to be implemented:

\starttyping[option=LUA]
function optional.mujs.setfindfile (
    function ( <t:string> name )
        return <t:string>
    end
)
    -- no return values
end

function optional.mujs.setopenfile  (
    function ( <t:string> name )
        return <t:integer> id
    end
)
    -- no return values
end

function optional.mujs.setclosefile (
    function ( <t:integer> id )
        -- no return values
    end
)
    -- no return values
end

function optional.mujs.setreadfile (
    function ( <t:integer> id )
        return <t:string> | <t:nil>
    end
)
    -- no return values
end

function optional.mujs.setseekfile (
    function ( <t:integer> id, <t:integer> position )
        return <t:integer>
    end
)
    -- no return values
end

function optional.mujs.setconsole ( )
    function ( <t:string> category, <t:string> message )
        -- no return values
    end
)
    -- no return values
end
\stoptyping

The library implements a few \JAVASCRIPT\ functions, like the ones
printing to \TEX, they take an optional catcodes reference:

\starttyping
texprint (catcodes, ...)
texsprint(catcodes, ...)
\stoptyping

and a reporter:

\starttyping
console  (category, message)
\stoptyping

The next function resets the interpreter:

\starttyping[option=LUA]
function optional.mujs.reset ( )
    -- no return value
end
\stoptyping

A snippet of \JAVASCRIPT\ can be executed with:

\starttyping[option=LUA]
function optional.mujs.execute ( <t:string> filename )
    -- no return value
end
\stoptyping

This loads a \JAVASCRIPT\ file:

\starttyping[option=LUA]
function optional.mujs.dofile ( <t:string> filename )
    -- no return value
end
\stoptyping

\stopsubsubsubject

\startsubsubsubject[title=openssl]

We use this module for some \PDF\ features. Given the frequent updates to the
(external) code base, it's for sure not something one wants in the engine. We use
only a small subset of functionality. The library is initialized with:

\starttyping[option=LUA]
function optional.openssl.initialize ( )
    return <t:boolean> -- success
end
\stoptyping

When signing succeeds the first return value is \type {true} and possibly there
is a string as second return value. When \type {false} is returned the second
argument is an error code.

\starttyping[option=LUA]
function optional.openssl.sign (
    {
        certfile   = <t:string>,
        datafile   = <t:string>,
        data       = <t:string>,
        password   = <t:string>,
        resultfile = <t:string>,
    }
)
    return
        <t:boolean>, -- success
        <t:string> | <t:integer> | <t:nil>
end
\stoptyping

Verifying needs similar data:

\starttyping[option=LUA]
function optional.openssl.verify (
    {
        certfile - <t:string>,
        datafile - <t:string>,
        data     - <t:string>,
        signature- <t:string>,
        password - <t:string>,
    }
)
    return
        <t:boolean>, -- success
        <t:integer> | <t:nil>
end
\stoptyping

This needs no explanation:

\starttyping[option=LUA]
function optional.openssl.getversion ( )
    return <t:integer>
end
\stoptyping

\stopsubsubsubject

\stopsubsection

\startsubsection[title=Foreign]

{\em Todo: something about how the foreign interface can be used (inspired by
alien). Also see \typ {libs-imp-foreign.mkxl}.}

\stopsubsection

\usemodule[matrix]

\startsubsection[title=Vector]

% This chapter was written with Archive's Voleuses film music in the background. I
% couldn't find the lyrics of \quote {A world that can exist} yet. And one day I
% need to see that movie.

This module has been added for the sake of \METAPOST\ graphics. Usually we talk
in terms of matrices. The code evolved from a module in \CONTEXT. There are some
helpers in the \METAPOST\ library that are \quote {vector aware}. We give some
examples of usage and render the matrices involved.

% use lua buffers

\startbuffer
\startluacode
document.v44 = vector.new {
    { 11, 12, 13, 14 },
    { 21, 22, 23, 24 },
    { 31, 32, 33, 34 },
    { 41, 42, 43, 44 }
}

document.v14 = vector.new {
    { 1, 2, 3, 4 }
}

document.v41 = vector.new {
    { 1 }, { 2 }, { 3 }, { 4 }
}

document.v22 = vector.new {
    { 1.1 , 2.2 },
    { 8.8 , 9.9 }
}

document.v33 = vector.new {
    { 1 , 2, 0.5 },
    { 3 , 4, 1.0 },
    { 5 , 6, 2.0 }
}
\stopluacode
\stopbuffer

\typebuffer[option=TEX] \getbuffer

These matrices have different dimensions (rows and columns):

\startlinecorrection
\startcombination[nx=5,ny=1,distance=2em]
    {\ctxlua{vector.typeset(document.v44)}} {\hbox{\tt 4 × 4}}
    {\ctxlua{vector.typeset(document.v14)}} {\hbox{\tt 1 × 4}}
    {\ctxlua{vector.typeset(document.v41)}} {\hbox{\tt 4 × 1}}
    {\ctxlua{vector.typeset(document.v22)}} {\hbox{\tt 2 × 2}}
    {\ctxlua{vector.typeset(document.v33)}} {\hbox{\tt 3 × 3}}
\stopcombination
\stoplinecorrection

If performance is important, you can avoid the tables and specify the dimensions
with (optionally) a series of values.

\starttyping
local a = vector.new(2,3, 1,2,3, 4,5,6)
local b = vector.new(3,2, 1,2, 3,4, 5,6)
local a = vector.new(2,2, 1,2 ,3,4)
local b = vector.new(2,1, 5,6)
\stoptyping

A vector is a user data object that supports addition, subtraction,
multiplication and division. Here we show multiplication:

\startbuffer
\startluacode
document.vdemo1 = 4 * document.v41
document.vdemo2 = document.v41 * 4
document.vdemo3 = document.v44 * document.v41
\stopluacode
\stopbuffer

\typebuffer[option=TEX] \getbuffer

When two matrices are multiplied, you get a matrix product, with numbers the
elements are multiplied.

\startlinecorrection
\startcombination[nx=4,ny=1,distance=2em]
    {\ctxlua{vector.typeset(document.v44   )}} {}
    {\ctxlua{vector.typeset(document.vdemo1)}} {\hbox{\tt 4 ×}}
    {\ctxlua{vector.typeset(document.vdemo2)}} {\hbox{\tt × 4}}
    {\ctxlua{vector.typeset(document.vdemo3)}} {\hbox{\tt ×  }}
\stopcombination
\stoplinecorrection

In addition to \type {+} (\type {add}), \type {-} (\type {sub}), \type {*} (\type
{mul}) and \type {/} (\type {div}) you can use the unary minus \type {-} (\type
{negate}) and test for equality with \type {=} (\type {equal}). The concat
operator \type {..} (\type {concat}) is also supported.

The functions applied to a vector return a new vector. In most cases they also
accept a table and then convert that on the fly to a vector. The \type {copy}
function can be used to make a copy but is not really needed.

\startbuffer
\startluacode
document.v1 = vector.new { { 1, 2 }, { 3, 4 } }
document.v2 = document.v1
document.v3 = vector.copy(document.v1)
\stopluacode
\stopbuffer

\typebuffer[option=TEX] \getbuffer

Here the two assignments to \type {v2} and \type {v3} are apparently
equivalent but this is what we really have:

\startlines
\cldcontext{tostring(document.v1)}
\cldcontext{tostring(document.v2)}
\cldcontext{tostring(document.v3)}
\stoplines

\startbuffer
\startluacode
vector.set  (document.v1, 4, 100)
vector.set  (document.v2, 4, 200)
vector.set  (document.v3, 4, 300)
vector.setrc(document.v1, 1, 1, 400)
vector.setrc(document.v2, 1, 1, 500)
vector.setrc(document.v3, 1, 1, 600)
\stopluacode
\stopbuffer

\typebuffer[option=TEX] \getbuffer

The copy actually is a copy while the assignment is an alias, as with tables. The
only time that this shows is when we do in place assignments as here.

\startlinecorrection
\startcombination[nx=2,ny=3,distance=2em]
  {\ctxlua{vector.typeset(document.v1)}} {} {\hbox{\tt \cldcontext{tostring(document.v1)}}} {}
  {\ctxlua{vector.typeset(document.v2)}} {} {\hbox{\tt \cldcontext{tostring(document.v2)}}} {}
  {\ctxlua{vector.typeset(document.v3)}} {} {\hbox{\tt \cldcontext{tostring(document.v3)}}} {}
\stopcombination
\stoplinecorrection

There are also \type {get} and \type {getrc}. The reason why we have \type {set}
and \type {get} is that they are equivalent to the direct accessors (using \type
{[]}).

\startbuffer
\startluacode
document.v1[3] = 2 * document.v1[3]
document.v2[3] = 2 * document.v2[3]
document.v3[3] = 2 * document.v3[3]
\stopluacode
\stopbuffer

\typebuffer[option=TEX] \getbuffer

Watch how we multiply the shared vector object twice:

\startlinecorrection % idem as above
\startcombination[nx=2,ny=3,distance=2em]
  {\ctxlua{vector.typeset(document.v1)}} {} {\hbox{\tt \cldcontext{tostring(document.v1)}}} {}
  {\ctxlua{vector.typeset(document.v2)}} {} {\hbox{\tt \cldcontext{tostring(document.v2)}}} {}
  {\ctxlua{vector.typeset(document.v3)}} {} {\hbox{\tt \cldcontext{tostring(document.v3)}}} {}
\stopcombination
\stoplinecorrection

\startbuffer
\startluacode
document.vdemo1 = vector.round  (document.v22)
document.vdemo2 = vector.floor  (document.v22)
document.vdemo3 = vector.ceiling(document.v22)
\stopluacode
\stopbuffer

\typebuffer[option=TEX] \getbuffer

There are various operations on vectors and you get back a new one; we don't
replace in the passed vectors.

\startlinecorrection
\startcombination[nx=4,ny=1,distance=2em]
  {\ctxlua{vector.typeset(document.v22   )}} {}
  {\ctxlua{vector.typeset(document.vdemo1)}} {\hbox{\type {round}}}
  {\ctxlua{vector.typeset(document.vdemo2)}} {\hbox{\type {floor}}}
  {\ctxlua{vector.typeset(document.vdemo3)}} {\hbox{\type {ceiling}}}
\stopcombination
\stoplinecorrection

\startbuffer
\startluacode
document.vdemo1 = vector.identity(3)
document.vdemo2 = vector.identity(4)
document.vdemo3 = vector.identity(2)
\stopluacode
\stopbuffer

The identity matrix can be created with:

\typebuffer[option=TEX] \getbuffer

\startlinecorrection
\startcombination[nx=3,ny=1,distance=2em]
  {\ctxlua{vector.typeset(document.vdemo1)}} {\hbox{\type {3 × 3}}}
  {\ctxlua{vector.typeset(document.vdemo2)}} {\hbox{\type {4 × 4}}}
  {\ctxlua{vector.typeset(document.vdemo3)}} {\hbox{\type {2 × 2}}}
\stopcombination
\stoplinecorrection

\startbuffer
\startluacode
document.vdemo1 = vector.transpose(document.v44)
document.vdemo2 = vector.swap(document.v41)
document.vdemo3 = vector.swap(document.v14)
\stopluacode
\stopbuffer

\typebuffer[option=TEX] \getbuffer

Transposing has a simple companion that makes a vector with a single row into one
with a single column and vise versa.

\startlinecorrection
\startcombination[nx=6,ny=1,distance=2em]
    {\ctxlua{vector.typeset(document.v44   )}} {}
    {\ctxlua{vector.typeset(document.vdemo1)}} {\hbox{\tt transpose}}
    {\ctxlua{vector.typeset(document.v41   )}} {}
    {\ctxlua{vector.typeset(document.vdemo2)}} {\hbox{\tt swap}}
    {\ctxlua{vector.typeset(document.v14   )}} {}
    {\ctxlua{vector.typeset(document.vdemo3)}} {\hbox{\tt swap}}
\stopcombination
\stoplinecorrection

\startbuffer
\startluacode
document.vdemo1 = vector.inverse   (document.v33)
document.vdemo2 = vector.rowechelon(document.v33,true)
document.vdemo3 = vector.rowechelon(document.v33)
\stopluacode
\stopbuffer

\typebuffer[option=TEX] \getbuffer

\startlinecorrection
\startcombination[nx=5,ny=1,distance=2em]
    {\ctxlua{vector.typeset(document.v33   )}}              {\hbox {v33}}
    {\ctxlua{vector.typeset(document.vdemo1)}}              {\hbox {inverse}}
    {\ctxlua{vector.typeset(document.vdemo1*document.v33)}} {\hbox {verified}}
    {\ctxlua{vector.typeset(document.vdemo2)}}              {\hbox {row echelon}}
    {\ctxlua{vector.typeset(document.vdemo3)}}              {\hbox {reduced echelon}}
\stopcombination
\stoplinecorrection

The first one normalize each column vector in a matrix. The other one homogenize,
dividing every entry in a row by the last element in the row, given that it is
non-zero.

\startbuffer
\startluacode
document.vdemo1 = vector.normalize (document.v33)
document.vdemo2 = vector.homogenize(document.v33)
\stopluacode
\stopbuffer

\typebuffer[option=TEX] \getbuffer

\startlinecorrection
\startcombination[nx=3,ny=1,distance=2em]
    {\ctxlua{vector.typeset(document.v33   )}} {}
    {\ctxlua{vector.typeset(document.vdemo1)}} {}
    {\ctxlua{vector.typeset(document.vdemo2)}} {}
\stopcombination
\stoplinecorrection

Internally we use a criterium for determining if we have a zero entry:
\cldcontext {vector.getepsilon()} that can be queried by \type {getepsilon}. The
\type {truncate} function will reduce numbers outside this interval to zero. The
\type {iszero} function takes a number and returns \type {true} when its within
these bounds.

\startbuffer
\startluacode
document.vdemo0 = vector.new {
    { 0.00000005, 0.001 },
    { 0.001, 0.00000001 },
}

document.vdemo1 = vector.truncate(document.vdemo0)
\stopluacode
\stopbuffer

\typebuffer[option=TEX] \getbuffer

\startlinecorrection
\startcombination[nx=2,ny=1,distance=2em]
    {\ctxlua{vector.typeset(document.vdemo0, { template = 8 } )}} {}
    {\ctxlua{vector.typeset(document.vdemo1, { template = 3 } )}} {}
\stopcombination
\stoplinecorrection

The \type {determinant} function return a number so we don't show this here but
we do mention the various products. These are optimized for the dimensions.

We feed \typ {inner} with two vectors (row or column) and get the inner product
of them in return.

\startbuffer
\startluacode
document.num0 = vector.inner(document.v41,document.v41)
document.num1 = vector.inner(document.v14,document.v14)
\stopluacode
\stopbuffer

\typebuffer[option=TEX] \getbuffer

\startlinecorrection
\startcombination[nx=2,ny=1,distance=2em]
    {\ctxlua{vector.typeset(document.num0)}} {}
    {\ctxlua{vector.typeset(document.num1)}} {}
\stopcombination
\stoplinecorrection

The \typ {crossproduct} needs two 3-vectors, and returns one 3-vector. They can
either be row or column vectors, and we get the same type in return.

\startbuffer
\startluacode
document.cvec0 = vector.crossproduct(vector.new {{1,2,3}},     vector.new {{4,5,6}})
document.cvec1 = vector.crossproduct(vector.new {{1},{2},{3}}, vector.new {{4},{5},{6}})
\stopluacode
\stopbuffer

\typebuffer[option=TEX] \getbuffer

\startlinecorrection
\startcombination[nx=2,ny=1,distance=2em]
    {\ctxlua{vector.typeset(document.cvec0)}} {}
    {\ctxlua{vector.typeset(document.cvec1)}} {}
\stopcombination
\stoplinecorrection

% \startlines \em
% MS todo: inner
% MS todo: product
% MS todo: crossproduct
% \stoplines

\startbuffer
\startluacode
document.vdemo0 = vector.new {
    { 11, 12, 13, 14 },
    { 21, 22, 23, 24 },
    { 31, 32, 33, 34 },
    { 41, 42, 43, 44 },
}

document.vdemo1 = vector.slice(document.vdemo0,2,2,1,1)
document.vdemo2 = vector.slice(document.vdemo0,2,2,2,2)
document.vdemo3 = vector.slice(document.vdemo0,2,2,3,3)
\stopluacode
\stopbuffer

\typebuffer[option=TEX] \getbuffer

\startlinecorrection
\startcombination[nx=4,ny=1,distance=2em]
    {\ctxlua{vector.typeset(document.vdemo0)}} {}
    {\ctxlua{vector.typeset(document.vdemo1)}} {}
    {\ctxlua{vector.typeset(document.vdemo2)}} {}
    {\ctxlua{vector.typeset(document.vdemo3)}} {}
\stopcombination
\stoplinecorrection

\startbuffer
\startluacode
document.vdemo1 = vector.delete(document.vdemo0,2,2)
document.vdemo2 = vector.delete(document.vdemo0,2,1)
document.vdemo3 = vector.delete(document.vdemo0,2,2,true)
document.vdemo4 = vector.delete(document.vdemo0,2,1,true)
\stopluacode
\stopbuffer

\typebuffer[option=TEX] \getbuffer

\startlinecorrection
\startcombination[nx=7,ny=1,distance=2em]
    {\ctxlua{vector.typeset(document.vdemo0)}} {}
    {\ctxlua{vector.typeset(document.vdemo1)}} {}
    {\ctxlua{vector.typeset(document.vdemo2)}} {}
    {\ctxlua{vector.typeset(document.vdemo3)}} {}
    {\ctxlua{vector.typeset(document.vdemo4)}} {}
\stopcombination
\stoplinecorrection

\startbuffer
\startluacode
document.vdemo1 = vector.remove(document.vdemo0,2,2)
document.vdemo2 = vector.remove(document.vdemo0,2,1)
document.vdemo3 = vector.remove(document.vdemo0,2,2,true)
document.vdemo4 = vector.remove(document.vdemo0,2,1,true)
\stopluacode
\stopbuffer

\typebuffer[option=TEX] \getbuffer

\startlinecorrection
\startcombination[nx=5,ny=1,distance=2em]
    {\ctxlua{vector.typeset(document.vdemo0)}} {}
    {\ctxlua{vector.typeset(document.vdemo1)}} {}
    {\ctxlua{vector.typeset(document.vdemo2)}} {}
    {\ctxlua{vector.typeset(document.vdemo3)}} {}
    {\ctxlua{vector.typeset(document.vdemo4)}} {}
\stopcombination
\stoplinecorrection

\startbuffer
\startluacode
document.vdemo0 = vector.new {
    { 11, 12, 13, 14 },
    { 21, 22, 23, 24 },
    { 31, 32, 33, 34 },
    { 41, 42, 43, 44 },
}

document.vdemo0r = vector.new {
    { 101, 201, 301, 401 },
    { 102, 202, 302, 402 },
}

document.vdemo0c = vector.new {
    { 101, 201 },
    { 102, 202 },
    { 103, 203 },
    { 104, 204 },
}

document.vdemo0rc = vector.new {
    { 101, 201 },
    { 102, 202 },
}
\stopluacode
\stopbuffer

\typebuffer[option=TEX] \getbuffer

\startlinecorrection
\startcombination[nx=4,ny=1,distance=2em]
    {\ctxlua{vector.typeset(document.vdemo0)}}   {}
    {\ctxlua{vector.typeset(document.vdemo0r)}}  {}
    {\ctxlua{vector.typeset(document.vdemo0c)}}  {}
    {\ctxlua{vector.typeset(document.vdemo0rc)}} {}
\stopcombination
\stoplinecorrection

\startbuffer
\startluacode
document.vdemo1 = vector.insert(document.vdemo0,document.vdemo0c,0)
document.vdemo2 = vector.insert(document.vdemo0,document.vdemo0c,1)
document.vdemo3 = vector.insert(document.vdemo0,document.vdemo0c,2)
document.vdemo4 = vector.insert(document.vdemo0,document.vdemo0c,3)
document.vdemo5 = vector.insert(document.vdemo0,document.vdemo0c,4)
\stopluacode
\stopbuffer

\typebuffer[option=TEX] \getbuffer

\startlinecorrection
\startcombination[nx=5,ny=1,distance=2em]
    {\ctxlua{vector.typeset(document.vdemo1)}} {}
    {\ctxlua{vector.typeset(document.vdemo2)}} {}
    {\ctxlua{vector.typeset(document.vdemo3)}} {}
    {\ctxlua{vector.typeset(document.vdemo4)}} {}
    {\ctxlua{vector.typeset(document.vdemo5)}} {}
\stopcombination
\stoplinecorrection

\startbuffer
\startluacode
document.vdemo1 = vector.insert(document.vdemo0,document.vdemo0r,0,true)
document.vdemo2 = vector.insert(document.vdemo0,document.vdemo0r,1,true)
document.vdemo3 = vector.insert(document.vdemo0,document.vdemo0r,2,true)
document.vdemo4 = vector.insert(document.vdemo0,document.vdemo0r,3,true)
document.vdemo5 = vector.insert(document.vdemo0,document.vdemo0r,4,true)
\stopluacode
\stopbuffer

\typebuffer[option=TEX] \getbuffer

\startlinecorrection
\startcombination[nx=5,ny=1,distance=2em]
    {\ctxlua{vector.typeset(document.vdemo1)}} {}
    {\ctxlua{vector.typeset(document.vdemo2)}} {}
    {\ctxlua{vector.typeset(document.vdemo3)}} {}
    {\ctxlua{vector.typeset(document.vdemo4)}} {}
    {\ctxlua{vector.typeset(document.vdemo5)}} {}
\stopcombination
\stoplinecorrection

\startbuffer
\startluacode
document.vdemo1 = vector.replace(document.vdemo0,document.vdemo0rc,1,1)
document.vdemo2 = vector.replace(document.vdemo0,document.vdemo0rc,2,2)
document.vdemo3 = vector.replace(document.vdemo0,document.vdemo0rc,3,3)
\stopluacode
\stopbuffer

\typebuffer[option=TEX] \getbuffer

\startlinecorrection
\startcombination[nx=3,ny=1,distance=2em]
    {\ctxlua{vector.typeset(document.vdemo1)}} {}
    {\ctxlua{vector.typeset(document.vdemo2)}} {}
    {\ctxlua{vector.typeset(document.vdemo3)}} {}
\stopcombination
\stoplinecorrection

\startbuffer
\startluacode
document.vdemo1 = vector.exchange(document.vdemo0,1,1)
document.vdemo2 = vector.exchange(document.vdemo0,1,2,true)
\stopluacode
\stopbuffer

\typebuffer[option=TEX] \getbuffer

\startlinecorrection
\startcombination[nx=3,ny=1,distance=2em]
    {\ctxlua{vector.typeset(document.vdemo1)}} {}
    {\ctxlua{vector.typeset(document.vdemo2)}} {}
\stopcombination
\stoplinecorrection

This leaves a couple of auxiliary function that we just mention:

\starttabulate[|Tl|Tl|l|]
\NC isvector      \NC (v)       \NC return \type{true} when the argument is a vector \NC \NR
\NC onerow        \NC (n,...,m) \NC a fast way to create a single row vector \NC \NR
\NC onecolumn     \NC (n,...,m) \NC a fast way to create a single column vector \NC \NR
\NC type          \NC (v)       \NC return the string \type {vector} if we pass one \NC \NR
\NC tostring      \NC (v)       \NC returns the dimensions and pointer value \NC \NR
\NC totable       \NC (v)       \NC returns a regular table representation \NC \NR
\NC determinant   \NC (v)       \NC returns the determinant \NC \NR
\NC issingular    \NC (v,[n])   \NC returns true if the determinant is \m{< 0.001}\NC \NR
%NC gettype       \NC (v)       \NC returns the type of the vector (experimental) \NC \NR
\NC setstacking   \NC (v,n)     \NC set the stacking property (\METAPOST) \NC \NR
\NC getstacking   \NC (v)       \NC get the stacking property (\METAPOST) \NC \NR
\NC getdimensions \NC (v)       \NC returns the number of rows and columns \NC \NR
\stoptabulate

We already mentioned that often you can also provide a table instead
of a vector:

\startbuffer
\startluacode
document.t1 = {
    { 1, 2, 3 },
    { 1, 2, 3 },
    { 1, 2, 3 },
}

document.t2 = {
    { 11, 2, 3 },
    { 1, 12, 3 },
    { 1, 2, 13 },
}

document.v1 = vector.new(document.t1)
document.v2 = vector.new(document.t2)

document.v1t1 = vector.product(document.v1,document.t2)
document.t1v1 = vector.product(document.t1,document.v1)
document.v1v2 = vector.product(document.v1,document.v2)
document.t1t2 = vector.product(document.t1,document.t2)
\stopluacode
\stopbuffer

\typebuffer[option=TEX] \getbuffer

\startlinecorrection
\startcombination[nx=4,ny=1,distance=2em]
    {\ctxlua{vector.typeset(document.v1t1)}} {\type{v t}}
    {\ctxlua{vector.typeset(document.t1v1)}} {\type{t v}}
    {\ctxlua{vector.typeset(document.v1v2)}} {\type{v v}}
    {\ctxlua{vector.typeset(document.t1t2)}} {\type{t t}}
\stopcombination
\stoplinecorrection

When you define and fill a vector, you can use the setters mentioned before. Here
we show a few ways, assuming that \type {ns} and \type {nt} are known and \type
{calulate} does some useful magic.

\starttyping[option=LUA]
local t = { }  -- we could preallocate

for s = 0, ns do
    for t = 0, nt do
        local x, y, z = calculate(s,t)
        t[#t+1] = { x, y, z, 1 }
   end
end

local v = newvector(t)
\stoptyping

We can avoid the fourth entry which is always~0 which gives smaller
intermediate tables and just append these ones later using
the \type {append} function.

\starttyping[option=LUA]
local t = { }  -- we could preallocate

for s = 0, ns do
    for t = 0, nt do
        local x, y, z = calculate(s,t)
        t[#t+1] = { x, y, z }
   end
end

local v = newvector(t)
v = vector.append(v,1)
\stoptyping

We can avoid the \LUA\\ table completely and just populate the
vector as we go. The \type {setnext} does the trick:

\starttyping[option=LUA]
local setnext = vector.setnext

local v = newvector((ns + 1) * (nt + 1), 4)
for s = 0, ns do
    for t = 0, nt do
        local x, y, z = calculate(s,t)
        setnext(v, x, y, z, 1)
   end
end
\stoptyping

That last variant is the most natural and also the most efficient approach but
taste can differ.

\stopsubsection

\stopsection

\stopdocument

% Initial chapter documenting moment, musical timstamp: The Warning - You Oughta
% Know - (Alanis Morissette) Cover, 2024, plus some energetic live shows on screen ...
%
% Coincidentally, at that time (October 30, 2024), after finishing this chapter we
% had exactly 600 pages, but still plenty to go.
