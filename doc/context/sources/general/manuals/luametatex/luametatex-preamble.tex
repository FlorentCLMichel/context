% language=uk

\environment luametatex-style

\startcomponent luametatex-preamble

\startchapter[reference=preamble,title={The internals}]

\topicindex{nodes}
\topicindex{boxes}
\topicindex{\LUA}

This is a reference manual, not a tutorial. This means that we discuss changes
relative to traditional \TEX\ and also present new functionality. As a
consequence we will refer to concepts that we assume to be known or that might be
explained later. Because the \LUATEX\ and \LUAMETATEX\ engines open up \TEX\
there's suddenly quite some more to explain, especially about the way a (to be)
typeset stream moves through the machinery. However, discussing all that in
detail makes not much sense, because deep knowledge is only relevant for those
who write code not possible with regular \TEX\ and who are already familiar with
these internals (or willing to spend time on figuring it out).

So, the average user doesn't need to know much about what is in this manual. For
instance fonts and languages are normally dealt with in the macro package that
you use. Messing around with node lists is also often not really needed at the
user level. If you do mess around, you'd better know what you're dealing with.
Reading \quotation {The \TEX\ Book} by Donald Knuth is a good investment of time
then also because it's good to know where it all started. A more summarizing
overview is given by \quotation {\TEX\ by Topic} by Victor Eijkhout. You might
want to peek in \quotation {The \ETEX\ manual} too.

But \unknown\ if you're here because of \LUA, then all you need to know is that
you can call it from within a run. If you want to learn the language, just read
the well written \LUA\ book. The macro package that you use probably will provide
a few wrapper mechanisms but the basic \lpr {directlua} command that does the job
is:

\starttyping
\directlua{tex.print("Hi there")}
\stoptyping

You can put code between curly braces but if it's a lot you can also put it in a
file and load that file with the usual \LUA\ commands. If you don't know what
this means, you definitely need to have a look at the \LUA\ book first.

If you still decide to read on, then it's good to know what nodes are, so we do a
quick introduction here. If you input this text:

\starttyping
Hi There ...
\stoptyping

eventually we will get a linked lists of nodes, which in \ASCII\ art looks like:

\starttyping
H <=> i <=> [glue] <=> T <=> h <=> e <=> r <=> e ...
\stoptyping

When we have a paragraph, we actually get something like this, where a \type
{localpar} node stores some metadata and is followed by a \type {hlist} flagged
as indent box:

\starttyping
[localpar] <=> [hlist] <=> H <=> i <=> [glue] <=> T <=> h <=> e <=> r <=> e ...
\stoptyping

Each character becomes a so called glyph node, a record with properties like the
current font, the character code and the current language. Spaces become glue
nodes. There are many node types that we will discuss later. Each node points
back to a previous node or next node, given that these exist. Sometimes
multiple characters are represented by one glyphs, so one can also get:

\starttyping
[localpar] <=> [hlist] <=> H <=> i <=> [glue] <=> Th <=> e <=> r <=> e ...
\stoptyping

And maybe some characters get positioned relative to each other, so we might
see:

\starttyping
[localpar] <=> [hlist] <=> H <=> [kern] <=> i <=> [glue] <=> Th <=> e <=> r <=> e ...
\stoptyping

It's also good to know beforehand that \TEX\ is basically centered around
creating paragraphs and pages. The par builder takes a list and breaks it into
lines. At some point horizontal blobs are wrapped into vertical ones. Lines are
so called boxes and can be separated by glue, penalties and more. The page
builder accumulates lines and when feasible triggers an output routine that will
take the list so far. Constructing the actual page is not part of \TEX\ but done
using primitives that permit manipulation of boxes. The result is handled back to
\TEX\ and flushed to a (often \PDF) file.

The \LUATEX\ engine provides hooks for \LUA\ code at nearly every reasonable
point in the process: collecting content, hyphenating, applying font features,
breaking into lines, etc. This means that you can overload \TEX's natural
behaviour, which still is the benchmark. When we refer to \quote {callbacks} we
means these hooks. The \TEX\ engine itself is pretty well optimized but when you
kick in much \LUA\ code, you will notices that performance drops. Don't blame and
bother the authors with performance issues. In \CONTEXT\ over 50\% of the time
can be spent in \LUA, but so far we didn't get many complaints about efficiency.
Adding more callbacks makes no sense, also because at some point the performance
hit gets too large. There are plenty ways to achieve one goals.

Where plain \TEX\ is basically a basic framework for writing a specific style,
macro packages like \CONTEXT\ and \LATEX\ provide the user a whole lot of
additional tools to make documents look good. They hide the dirty details of font
management, language demands, turning structure into typeset results, wrapping
pages, including images, and so on. You should be aware of the fact that when you
hook in your own code to manipulate lists, this can interfere with the macro
package that you use. Each successive step expects a certain result and if you
mess around to much, the engine eventually might bark and quit. It can even
crash, because testing everywhere for what users can do wrong is no real option.

When you read about nodes in the following chapters it's good to keep in mind
what commands relate to them. Here are a few:

\starttabulate[|l|l|p|]
\DB command              \BC node          \BC explanation \NC \NR
\TB
\NC \prm {hbox}          \NC \nod {hlist} \NC horizontal box \NC \NR
\NC \prm {vbox}          \NC \nod {vlist} \NC vertical box with the baseline at the bottom \NC \NR
\NC \prm {vtop}          \NC \nod {vlist} \NC vertical box with the baseline at the top \NC \NR
\NC \prm {hskip}         \NC \nod {glue}  \NC horizontal skip with optional stretch and shrink \NC \NR
\NC \prm {vskip}         \NC \nod {glue}  \NC vertical skip with optional stretch and shrink \NC \NR
\NC \prm {kern}          \NC \nod {kern}  \NC horizontal or vertical fixed skip \NC \NR
\NC \prm {discretionary} \NC \nod {disc}  \NC hyphenation point (pre, post, replace) \NC \NR
\NC \prm {char}          \NC \nod {glyph} \NC a character \NC \NR
\NC \prm {hrule}         \NC \nod {rule}  \NC a horizontal rule \NC \NR
\NC \prm {vrule}         \NC \nod {rule}  \NC a vertical rule \NC \NR
\NC \prm {textdirection} \NC \nod {dir}   \NC a change in text direction \NC \NR
\LL
\stoptabulate

Text (interspersed with macros) comes from an input medium. This can be a file,
token list, macro body cq.\ arguments, \ some internal quantity (like a number),
\LUA, etc. Macros get expanded. In the process \TEX\ can enter a group. Inside
the group, changes to registers get saved on a stack, and restored after leaving
the group. When conditionals are encountered, another kind of nesting happens,
and again there is a stack involved. Tokens, expansion, stacks, input levels are
all terms used in the next chapters. Don't worry, they loose their magic once you
use \TEX\ a lot. You have access to most of the internals and when not, at least
it is possible to query some state we're in or level we're at.

When we talk about pack(ag)ing it can mean two things. When \TEX\ has consumed
some tokens that represent text. When the text is put into a so called \type
{\hbox} it (normally) first gets hyphenated (even in an horizontal list), next
ligatures are build, and finally kerns are added. Each of these stages can be
overloaded using \LUA\ code. When these three stages are finished, the dimension
of the content is calculated and the box gets its width, height and depth. What
happens with the box depends on what macros do with it.

The other thing that can happen is that the text starts a new paragraph. In that
case some information is stored in a leading \type {localpar} node. Then
indentation is appended and the paragraph ends with some glue. Again the three
stages are applied but this time, afterwards, the long line is broken into lines
and the result is either added to the content of a box or to the main vertical
list (the running text so to say). This is called par building. At some point
\TEX\ decides that enough is enough and it will trigger the page builder. So,
building is another concept we will encounter. Another example of a builder is
the one that turns an intermediate math list into something typeset.

Wrapping something in a box is called packing. Adding something to a list is
described in terms of contributing. The more complicated processes are wrapped
into builders. For now this should be enough to enable you to understand the next
chapters. The text is not as enlightening and entertaining as Don Knuths books,
sorry.

\stopchapter

\stopcomponent
