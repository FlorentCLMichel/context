% language=us runpath=texruns:manuals/luametatex

\environment luametatex-style

\startdocument[title=Math]

\startsection[title={Introduction}]

There is a lot to tell about math typesetting in \LUAMETATEX\ but plenty is
covered in articles, progress reports and manuals. Here we limit ourselves to
some basics. This chapter mostly contains information that is not presented
elsewhere. Because math in regular \TEX\ is basically frozen and other macro
packaged depend on that, the extensions we have in \LUAMETATEX\ are mainly useful
for \CONTEXT. Even there we don't use all features, because completely opening up
and providing ways to control every aspect also served the purpose of testing: it
just comes with the package.

This chapter is a variant on the one in the old \LUAMETATEX\ manual and it might
evolve a bit. We will not discuss the many options that the engine provides, at
least not now. There is an extensive \quotation {Math in \CONTEXT} that shows the
state of the art and serves as reference. In due time we might write some more
about what happens deep down in the engine, although already plenty has been
published during the upgrade, about dealing with math fonts as well as
experimenting with new features. Because all gets wrapped in high level
interfaces there is not that much need (nor audience) for endless explanations
anyway. There are also examples given in the chapter that discusses all
primitives. Most \CONTEXT\ users will never see these low level math commands!

\stopsection

\startsection[title={Traditional alongside \OPENTYPE}]

Because we started in 2019 from \LUATEX, by the end of 2021 this chapter started
with this, even if we already reworked the engine:

\startquotation
At this point there is no difference between \LUAMETATEX\ and \LUATEX\ with
respect to math. Well, this might no longer be true because we have more control
options that define default behavior and also have a more extensive scaling
model. Anyway, it should not look worse, and maybe even a bit better. The
handling of mathematics in \LUATEX\ differs quite a bit from how \TEX82 (and
therefore \PDFTEX) handles math. First, \LUATEX\ adds primitives and extends some
others so that \UNICODE\ input can be used easily. Second, all of \TEX82's
internal special values (for example for operator spacing) have been made
accessible and changeable via control sequences. Third, there are extensions that
make it easier to use \OPENTYPE\ math fonts. And finally, there are some
extensions that have been proposed or considered in the past that are now added
to the engine.

You might be surprised that we don't use all these new control features in
\CONTEXT\ \LMTX, but who knows what might happen because users drive it. The main
reason for adding so much is that I decided it made more sense to be complete now
than gradually add more and more. At some point we should be able to say \quote
{This is it}. Also, when looking at these features, you need to keep in mind that
when it comes to math, \LATEX\ is the dominant macro package and it never needed
these engine features, so most are probably just here for exploration purposes.
\stopquotation

Although we still process math as \TEX\ does, there have been some fundamental
changes to the machinery. Most of that is discussed in documents that come with
\CONTEXT\ and in Mikael Sundqvist math manual. Together we explored some new ways
to deal with math spacing, penalties, fencing, operators, fractions, atoms and
other features of the \TEX\ engine. We started from the way \CONTEXT\ used the
already present functionality combine with sometimes somewhat dirty (but on the
average working well) tricks.

Much in \LUAMETATEX\ math handling is about micro|-|typography and for us the
results are quite visible. But, as far as we know, there have never been
complaints or demands in the direction of the features discussed here. Also,
\TEX\ math usage outside \CONTEXT\ is rather chiseled in stone (already for
nearly three decades) so we don't expect other macro packages to use the new
features anyway. Anyway, after spending a real lot of time on this we both
decided that we're mostly feature complete.

\stopsection

\startsection[title=Intermezzo]

It is important to understand a bit how \TEX\ handles math. The math engine
is a large subsystem and basically can be divided in two parts: convert
sequential input into a list of nodes where math related ones actually are
sort of intermediate and therefore called noads.

In text mode entering \type {abc} results in three glyph nodes and \type {a b
c} in three glyph nodes separated by (spacing) glue. Successive glyphs can be
transformed in the font engine later on, just as hyphenation directive can
be added. Eventually one (normally) gets a mix of glyphs, font kerns from
a sequence of glyphs

In math mode \type {abc} results in three simple ordinary noads and \type {a b c}
is equivalent to that: three noads. But \type {a bc} results in two ordinary
noads where the second one has a sublist of two ordinary noads. Because
characters have class properties, \type {( a + b = c )} results in a simple open
noad, a simple ordinary, a simple binary, a simple ordinary, a simple relation, a
simple ordinary and simple close noad. The next samples show a bit of this; in
order to see the effects of spacing between ordinary atoms set it to \type {9mu}.

\startbuffer
\typebuffer[sample]
% \tracingmath  \plusone
% \tracingonline\plusone
\startlinecorrection
\setmathspacing\mathordinarycode\mathordinarycode\allmathstyles9mu
\mathgroupingmode\zerocount
\scale[scale=2000]{\showmakeup[mathglue]\showboxes\mathspacingmode\plusone\getbuffer[sample]}
\stoplinecorrection
\blank[2*line]
\stopbuffer

\startbuffer[sample]
$a b c$ \quad $a bc$ \quad $abc$
\stopbuffer

\getbuffer

With \type {\tracingmath 1} we get this logged:

\starttyping
> \inlinemath=
\noad[ord][...]
.\nucleus
..\mathchar[ord][...], family "0, character "61
\noad[ord][...]
.\nucleus
..\mathchar[ord][...], family "0, character "62
\noad[ord][...]
.\nucleus
..\mathchar[ord][...], family "0, character "63

> \inlinemath=
\noad[ord][...]
.\nucleus
..\mathchar[ord][...], family "0, character "61
\noad[ord][...]
.\nucleus
..\mathchar[ord][...], family "0, character "62
\noad[ord][...]
.\nucleus
..\mathchar[ord][...], family "0, character "63

> \inlinemath=
\noad[ord][...]
.\nucleus
..\mathchar[ord][...], family "0, character "61
\noad[ord][...]
.\nucleus
..\mathchar[ord][...], family "0, character "62
\noad[ord][...]
.\nucleus
..\mathchar[ord][...], family "0, character "63
\stoptyping

\startbuffer[sample]
${a} {b} {c}$ \quad ${a} {bc}$ \quad ${abc}$
\stopbuffer

\getbuffer

If the previous log surprises you, that might be because in \CONTEXT\ we set up
the engine differently: curly braces don't create ordinary atoms. However, when
we set \type {\mathgroupingmode 0} we return to what the engine normally does.

\starttyping
> \inlinemath=
\noad[ord][...]
.\nucleus
..\mathchar[ord][...], family "0, character "61
\noad[ord][...]
.\nucleus
..\mathchar[ord][...], family "0, character "62
\noad[ord][...]
.\nucleus
..\mathchar[ord][...], family "0, character "63

> \inlinemath=
\noad[ord][...]
.\nucleus
..\mathchar[ord][...], family "0, character "61
\noad[ord][...]
.\nucleus
..\submlist[0][...][tracing depth 5 reached]

> \inlinemath=
\noad[ord][...]
.\nucleus
..\submlist[0][...][tracing depth 5 reached]
\stoptyping

A warning is in place: tracing in \LUAMETATEX\ gets extended when we feel the
need to gat more feedback from the engine. But it will only be more.

From the first example you can imagine what these sub lists look like: a list of
ordinary atoms. The final list that is mix of nodes and yet unprocessed noads get
fed into the math|-|to|-|hlist function and eventually the noads become glyphs,
boxes, kerns, glue and whatever makes sense. A lot goes on there: think scripts,
fractions, fences, accents, radicals, spacing, break control.

An example of more tricky scanning is shown here:

\starttyping
a +   1 \over 2   + b
a +  {1}\over{2}  + b
a + {{1}\over{2}} + b
\stoptyping

In this case the \type {\over} makes \TEX\ reconsider the last noad, remove if
from the current list and save it for later, then scan for a following atom a
single character turned atom or a braced sequence that then is an ordinary noad.
In the end a fraction noad is made. When that gets processed later specific
numerator and denominator styles get applied (explicitly entered style nodes of
course overload this for the content). The fact that this construct is all about
(implicit) ordinary noads, themselves captured in noads, combined with the wish
for enforced consistent positioning of numerator and denominator, plus style
overload, color support and whatever comes to mind means that in practice one
will use a \type {\frac} macro that provides all that control. \footnote {There
are now a \type {\Uover} primitives that look ahead and then of course still
treat curly braces as math lists to be picked up.}

A similar tricky case is this:

\starttyping
      ( a +       ( b - c        ) + d        )
\left ( a + \left ( b - c \right ) + d \right )
\stoptyping

Here the first line creates a list of noads but the second line create a fenced
structure that is handled as a whole in order to make the fences match. \footnote
{Actually instead of such a structure there could have been delimiters with
backlinks but one never knows what happens with these links when processing
passes are made so that fragility is avoided.} A fence noad will not break across
lines as it is boxed and that is the reason why macro packages have these \type
{\bigg} macros: they explicitly force a size using some trickery. In \LUAMETATEX\
a fence object can actually be unpacked when the class is configured as such. It
is one of the many extensions we have.

There are some peculiar cases that one can run into but that actually are
mentioned in the \TEX\ book. Often these reasons for intentional side effects
become clear when one thinks of the average usage but unless one is willing to
spend time on the \quote {fine points of math} they can also interfere with
intentions. The next bits of code are just for the reader to look at. Try to
predict the outcome. Watch out: in \LMTX\ the outcome is not what one gets by
default in \LUATEX, \PDFTEX\ or regular \TEX. \footnote {One can set \typ
{\mathgroupingmode = 0} to get close.}

\starttyping
$ 1 {\red +} 2$\par
$ 1 \color[red]{+} 2$\par
$ 1 \mathbin{\red +} 2$\par
$ a + - b + {- b} $
$ a \pm - b - {+ b} $
$ - b $
$ {- b} $
\stoptyping

The message here is that when a user is coding the mindset with respect to
grouping using curly braces has to be switched to math mode too. And how many
users really read the relevant chapters of the \TEX\ book a couple of times (as
much makes only sense after playing with math in \TEX)? Even if one doesn't grasp
everything it's a worthwhile read. Also consider this: did you really ask for an
ordinary atom when you uses curly braces where no lists were expected? And what
would have happened when ordinary related spacing had been set to non|-|zero?

All the above (and plenty more) is why in \CONTEXT\ \LMTX\ we make extensive use
of some \LUAMETATEX\ features, like: additional atom classes, configurable inter
atom spacing and penalties, pairwise atom rules that can change classes, class
based rendering options, more font parameters, configurable style instead of hard
coded ones in constructs, more granular spacing, etc. That way we get quite
predictable results but also drop some older (un)expected behavior and side
effects. It is also why we cannot show many examples in the \LUAMETATEX\ manual:
it uses \CONTEXT\ and we see no reason to complicate out lives (and spend energy
on) turning off all the nicely cooperating features (and then for sure forgetting
one) just for the sake of demos. It also gave us the opportunity to improve
existing mechanisms and|/|or at least simplify their sometimes complex code.

One last word here about sequences of ordinary atoms: the traditional code path
feeds ordinary atoms into a ligature and kerning routine and does that when it
encounters one. However, in \OPENTYPE\ we don't have ligatures not (single) kerns
so there that doesn't apply. As we're not aware of traditional math fonts with
ligatures and no one is likely to use these fonts with \LUAMETATEX\ the ligature
code has been disabled. \footnote {It might show up in a different way if we feel
the need in which case it's more related to runtime patches to fonts and class
bases ligature building.} The kerning has been redone a bit so that it permits us
to fine tune spacing (which in \CONTEXT\ we control with goodie files). The
mentioned routine can also add italic correction, but that happens selectively
because it is driven by specifications and circumstances. It is one of the places
where the approach differs from the original, if only for practical reasons.

In addition to what we explained above, we mention the \typ {\beginmathgroup} and
\typ {\endmathgroup} primitives behave like \typ {\begingroup} and \typ {\endgroup}
but restore a style change inside the group. Style changes are actually injecting
s special style noad which makes them sort of persistent till the next explicit
change which can be confusing. This additional grouping model compensates for
that.

\stopsection

\startsection[title={Unicode math characters}]

For various reasons we need to encode a math character in a 32 bit number and
because we often also need to keep track of families and classes the range of
characters is limited to 20 bits. There are upto 64 classes (which is a lot more
than in \LUATEX) and 64 families (less than in \LUATEX). The upper limit of
characters is less that what \UNICODE\ offers but for math we're okay. If needed
we can provide less families.

The math primitives from \TEX\ are kept as they are, except for the ones that
convert from input to math commands: \type {mathcode}, and \type {delcode}. These
two now allow for the larger character codes argument on the left hand side of
the equals sign. The number variants of some primitives might be dropped in favor
of the primitives that read more than one separate value (class, family and
code). All relevant primitives are explained in the primitives chapter.

A delimiter in traditional \TEX\ combines two definitions: the regular character
and the way it can become a larger (extensible) one. The small character is just
like a math character but the larger one can come from a different font (family).
However, in \OPENTYPE\ math fonts the larger sizes (variants) and extensibles
(parts) come from the same font. For that reason \LUAMETATEX\ also accepts a math
character when a delimited specifier is expected. It basically means that we
could remove delimiters as such from the engine. After all, when we let \LUA\
load a traditional font we can as well use virtual fonts to handle the variants
and extensibles, which is indeed the case when we support the jmh fonts.

\stopsection

\startsection[title={Math classes}]

Most characters belong to a so called math class which can be set for each
character if needed. There are upto 64 classes of which at this moment about 20
are predefined so, taking some future usage by the engine into account,you can
assume 32 upto 60 to be available for any purpose. The number of families has
been reduced from 256 to 64 which is plenty for daily use in an \OPENTYPE\ setup.
If we ever need to expand the \UNICODE\ range there will be less families or we
just go for a larger internal record. The values of begin and end classes and the
number of classes can be fetched from the \LUA\ status table. There are callbacks
that makes it possible to report user classes when there is the need.

\stopsection

\startsection[title=Setting up the engine]

Rendering math has long been dominated by \TEX\ but that changed when \MICROSOFT\
came with \OPENTYPE\ math: an implementation as well as a font. Some of that was
modeled after \TEX\ and some was dictated (we think) by the way word processors
deal with math. For instance, traditional \TEX\ math has a limited set of glyph
properties and therefore has a somewhat complex interplay between width and
italic correction. There are no kerns, contrary to \OPENTYPE\ math fonts that
provides staircase kerns. Interestingly \TEX\ does have some ligature building
going on in the engine.

In traditional \TEX\ italic correction gets added to the width and selectively
removed later (or compensated by some shift and|/|or cheating with the box
width). When we started with \LUATEX\ we had to gamble quite a bit about how to
apply parameters and glyph properties which resulted in different code paths,
heuristics, etc. That worked on the average but fonts are often not perfect and
when served as an example for another one the bad bits can be inherited. That
said, over time the descriptions improved and this is what the \OPENTYPE\
specification has to say about italic correction now \footnote {\type
{https://docs.microsoft.com/en-us/typography/opentype/spec/math}}:

\startitemize [n]
    \startitem
        When a run of slanted characters is followed by a straight character
        (such as an operator or a delimiter), the italics correction of the last
        glyph is added to its advance width.
    \stopitem
    \startitem
        When positioning limits on an N-ary operator (e.g., integral sign), the
        horizontal position of the upper limit is moved to the right by half the
        italics correction, while the position of the lower limit is moved to the
        left by the same distance.
    \stopitem
    \startitem
        When positioning superscripts and subscripts, their default horizontal
        positions are also different by the amount of the italics correction of
        the preceding glyph.
    \stopitem
\stopitemize

The first rule is complicated by the fact that \quote {followed} is vague: in
\TEX\ the sequence \typ {$ a b c def $} results in six separate atoms, separated
by inter atom spacing. The characters in these atoms are the nucleus and there
can be a super- and|/|or subscript attached and in \LUAMETATEX\ also a prime,
super-prescript and/or sub-prescript.

The second rule comes from \TEX\ and one can wonder why the available top accent
anchor is not used. Maybe because bottom accent anchors are missing? Anyway,
we're stuck with this now.

The third rule also seems to come from \TEX. Take the \quote {\it f} character:
in \TEX\ fonts that one has a narrow width and part sticks out (in some even at
the left edge). That means that when the subscript gets attached it will move
inwards relative to the real dimensions. Before the superscript an italic
correction is added so what that correction is non|-|zero the scripts are
horizontally shifted relative to each other.

Now look at this specification of staircase kerns \footnote {Idem.}:

\startnarrower
    The \type {MathKernInfo} table provides mathematical kerning values used for
    kerning of subscript and superscript glyphs relative to a base glyph. Its
    purpose is to improve spacing in situations such as omega with superscript f
    or capital V with subscript capital A.

    Mathematical kerning is height dependent; that is, different kerning amounts
    can be specified for different heights within a glyph’s vertical extent. For
    any given glyph, different values can be specified for four corner positions,
    top|-|right, to|-|left, etc., allowing for different kerning adjustments
    according to whether the glyph occurs as a subscript, a superscript, a base
    being kerned with a subscript, or a base being kerned with a superscript.
\stopnarrower

Again we're talking super- and subscripts and should we now look at the italic
correction or assume that the kerns do the job? This is a mixed bag because
scripts are not always (single) characters. We have to guess a bit here. After
years of experimenting we came to the conclusion that it will never be okay so
that's why we settled on controls and runtime fixes to fonts.

This means that processing math is controlled by \typ {\mathfontcontrol}, a
numeric bitset parameter. The recommended bits are marked with a star but it
really depends on the macro package to set up the machinery well. Of course one
can just enable all and see what happens. \footnote {This model was more granular
and could even be font (and character) specific but that was dropped because
fonts are too inconsistent and an occasional fit is more robust that a generally
applied rule.} A list of possible control bits can be found in the primitives
chapter where we discuss this parameter.

So, to summarize: the reason for this approach is that traditional and \OPENTYPE\
fonts have different approaches (especially when it comes to dealing with the
width and italic corrections) and is even more complicated by the fact that the
fonts are often inconsistent (within and between). In \CONTEXT\ we deal with this
by runtime fixes to fonts. In any case the Cambria font is taken as reference.

{\em It is important to notice that in \CONTEXT\ we no longer use italic
correction at all. After many experiments Mikael Sundvist and I settled on a
different approach where we use true widths, proper anchors, a new set of corner
kerns, additional parameters and more. We tweak the fonts to match this model
which in our opinion gives better results and less interference. We could
actually simplify the engine and kick italics out of math but for the moment we
keep it around so that we can show improvements in manuals and articles.}

\stopsection

\startsection[title={Math styles}]

It is possible to discover the math style that will be used for a formula in an
expandable fashion (while the math list is still being read). To make this
possible, \LUATEX\ adds the new primitive: \typ {\mathstyle}. This is a \quote
{convert command} like e.g. \typ {\romannumeral}: its value can only be read,
not set. Beware that contrary to \LUATEX\ this is now a proper number so you need
to use \type {\number} or \type {\the} in order to serialize it.

The returned value is between 0 and 7 (in math mode), or $-1$ (all other modes).
For easy testing, the eight math style commands have been altered so that they can
be used as numeric values, so you can write code like this:

\starttyping
\ifnum\mathstyle=\textstyle
    \message{normal text style}
\else \ifnum\mathstyle=\crampedtextstyle
    \message{cramped text style}
\fi \fi
\stoptyping

Sometimes you won't get what you expect so a bit of explanation might help to
understand what happens. When math is parsed and expanded it gets turned into a
linked list. In a second pass the formula will be build. This has to do with the
fact that in order to determine the automatically chosen sizes (in for instance
fractions) following content can influence preceding sizes. A side effect of this
is for instance that one cannot change the definition of a font family (and
thereby reusing numbers) because the number that got used is stored and used in
the second pass (so changing \type {\fam 12} mid|-|formula spoils over to
preceding use of that family).

The style switching primitives like \type {\textstyle} are turned into nodes so the
styles set there are frozen. The \type {\mathchoice} primitive results in four
lists being constructed of which one is used in the second pass. The fact that
some automatic styles are not yet known also means that the \type {\mathstyle}
primitive expands to the current style which can of course be different from the
one really used. It's a snapshot of the first pass state. As a consequence in the
following example you get a style number (first pass) typeset that can actually
differ from the used style (second pass). In the case of a math choice used
ungrouped, the chosen style is used after the choice too, unless you group.

\startbuffer[1]
    [a:\number\mathstyle]\quad
    \bgroup
    \mathchoice
        {\bf \scriptstyle       (x:d :\number\mathstyle)}
        {\bf \scriptscriptstyle (x:t :\number\mathstyle)}
        {\bf \scriptscriptstyle (x:s :\number\mathstyle)}
        {\bf \scriptscriptstyle (x:ss:\number\mathstyle)}
    \egroup
    \quad[b:\number\mathstyle]\quad
    \mathchoice
        {\bf \scriptstyle       (y:d :\number\mathstyle)}
        {\bf \scriptscriptstyle (y:t :\number\mathstyle)}
        {\bf \scriptscriptstyle (y:s :\number\mathstyle)}
        {\bf \scriptscriptstyle (y:ss:\number\mathstyle)}
    \quad[c:\number\mathstyle]\quad
    \bgroup
    \mathchoice
        {\bf \scriptstyle       (z:d :\number\mathstyle)}
        {\bf \scriptscriptstyle (z:t :\number\mathstyle)}
        {\bf \scriptscriptstyle (z:s :\number\mathstyle)}
        {\bf \scriptscriptstyle (z:ss:\number\mathstyle)}
    \egroup
    \quad[d:\number\mathstyle]
\stopbuffer

\startbuffer[2]
    [a:\number\mathstyle]\quad
    \begingroup
    \mathchoice
        {\bf \scriptstyle       (x:d :\number\mathstyle)}
        {\bf \scriptscriptstyle (x:t :\number\mathstyle)}
        {\bf \scriptscriptstyle (x:s :\number\mathstyle)}
        {\bf \scriptscriptstyle (x:ss:\number\mathstyle)}
    \endgroup
    \quad[b:\number\mathstyle]\quad
    \mathchoice
        {\bf \scriptstyle       (y:d :\number\mathstyle)}
        {\bf \scriptscriptstyle (y:t :\number\mathstyle)}
        {\bf \scriptscriptstyle (y:s :\number\mathstyle)}
        {\bf \scriptscriptstyle (y:ss:\number\mathstyle)}
    \quad[c:\number\mathstyle]\quad
    \begingroup
    \mathchoice
        {\bf \scriptstyle       (z:d :\number\mathstyle)}
        {\bf \scriptscriptstyle (z:t :\number\mathstyle)}
        {\bf \scriptscriptstyle (z:s :\number\mathstyle)}
        {\bf \scriptscriptstyle (z:ss:\number\mathstyle)}
    \endgroup
    \quad[d:\number\mathstyle]
\stopbuffer

\typebuffer[1]

This gives:

\blank $\displaystyle \getbuffer[1]$ \blank
\blank $\textstyle    \getbuffer[1]$ \blank

Using \type {\begingroup} \unknown\ \type {\endgroup} instead gives:

\blank $\displaystyle \getbuffer[2]$ \blank
\blank $\textstyle    \getbuffer[2]$ \blank

This might look wrong but it's just a side effect of \type {\mathstyle} expanding
to the current (first pass) style and the number being injected in the list that
gets converted in the second pass. It all makes sense and it illustrates the
importance of grouping. In fact, the math choice style being effective afterwards
has advantages. It would be hard to get it otherwise.

So far for the more \LUATEX ish approach. One problem with \type {\mathstyle} is
that when you got it, and want to act upon it, you need to remap it onto say \type
{\scriptstyle} which can be done with an eight branched \type {\ifcase}. This is
why we also have a more efficient alternative that you can use in macros:

\starttyping
\normalexpand{ ... \givenmathstyle\the\mathstyle      ...}
\normalexpand{ ... \givenmathstyle\the\mathstackstyle ...}
\stoptyping

This new primitive \typ {\givenmathstyle} accepts a numeric value. The \typ
{\mathstackstyle} primitive is just a bonus (it complements \type {\mathstack}).

The styles that the different math components and their sub components start out
with are no longer hard coded but can be set at runtime:

\starttabulate
\FL
\BC primitive name                        \BC default \NC \NR
\TL
\NC \type {\Umathoverlinevariant}           \NC cramped           \NC \NR
\NC \type {\Umathunderlinevariant}          \NC normal            \NC \NR
\NC \type {\Umathoverdelimitervariant}      \NC small             \NC \NR
\NC \type {\Umathunderdelimitervariant}     \NC small             \NC \NR
\NC \type {\Umathdelimiterovervariant}      \NC normal            \NC \NR
\NC \type {\Umathdelimiterundervariant}     \NC normal            \NC \NR
\NC \type {\Umathhextensiblevariant}        \NC normal            \NC \NR
\NC \type {\Umathvextensiblevariant}        \NC normal            \NC \NR
\NC \type {\Umathfractionvariant}           \NC cramped           \NC \NR
\NC \type {\Umathradicalvariant}            \NC cramped           \NC \NR
\NC \type {\Umathdegreevariant}             \NC doublesuperscript \NC \NR
\NC \type {\Umathaccentvariant}             \NC cramped           \NC \NR
\NC \type {\Umathtopaccentvariant}          \NC cramped           \NC \NR
\NC \type {\Umathbottomaccentvariant}       \NC cramped           \NC \NR
\NC \type {\Umathoverlayaccentvariant}      \NC cramped           \NC \NR
\NC \type {\Umathnumeratorvariant}          \NC numerator         \NC \NR
\NC \type {\Umathdenominatorvariant}        \NC denominator       \NC \NR
\NC \type {\Umathsuperscriptvariant}        \NC superscript       \NC \NR
\NC \type {\Umathsubscriptvariant}          \NC subscript         \NC \NR
\NC \type {\Umathprimevariant}              \NC superscript       \NC \NR
\NC \type {\Umathstackvariant}              \NC numerator         \NC \NR
\LL
\stoptabulate

These defaults remap styles are as follows:

\starttabulate[|Tl|l|l|]
\FL
\BC default           \BC result                    \BC mapping \NC \NR
\TB
\NC cramped           \NC cramp the style           \NC D' D' T' T' S'  S'  SS' SS' \NC \NR
\NC subscript         \NC smaller and cramped       \NC S' S' S' S' SS' SS' SS' SS' \NC \NR
\NC small             \NC smaller                   \NC S  S  S  S  SS  SS  SS  SS  \NC \NR
\NC superscript       \NC smaller                   \NC S  S  S  S  SS  SS  SS  SS  \NC \NR
\NC smaller           \NC smaller unless already SS \NC S  S' S  S' SS  SS' SS  SS' \NC \NR
\NC numerator         \NC smaller unless already SS \NC S  S' S  S' SS  SS' SS  SS' \NC \NR
\NC denominator       \NC smaller, all cramped      \NC S' S' S' S' SS' SS' SS' SS' \NC \NR
\NC doublesuperscript \NC smaller, keep cramped     \NC S  S' S  S' SS  SS' SS  SS' \NC \NR
\LL
\stoptabulate

The main reason for opening this up was that it permits experiments and removed
hard coded internal values. But as these defaults served well for decades there
are no real reasons to change them.

There are a few math commands in \TEX\ where the style that will be used is not
known straight from the start. These commands (\type {\over}, \type {\atop},
\typ {\overwithdelims}, \typ {\atopwithdelims}) would therefore normally return
wrong values for \type {\mathstyle}. To fix this, \LUATEX\ introduces a special
prefix command: \type {\mathstack}:

\starttyping
$\mathstack {a \over b}$
\stoptyping

The \type {\mathstack} command will scan the next brace and start a new math group
with the correct (numerator) math style. The \typ {\mathstackstyle} primitive
relates to this feature.

\LUATEX\ has four new primitives to set the cramped math styles directly:

\starttyping
\crampeddisplaystyle
\crampedtextstyle
\crampedscriptstyle
\crampedscriptscriptstyle
\stoptyping

These additional commands are not all that valuable on their own, but they come
in handy as arguments to the math parameter settings that will be added shortly.

Because internally the eight styles are represented as numbers some of the new
primitives that relate to them also work with numbers and often you can use them
mixed. The \typ {\tomathstyle} prefix converts a symbolic style into a number so
\typ {\number \tomathstyle \crampedscriptstyle} gives~\number \tomathstyle
\crampedscriptstyle.

In Eijkhouts \quotation {\TEX\ by Topic} the rules for handling styles in scripts
are described as follows:

\startitemize
\startitem
    In any style superscripts and subscripts are taken from the next smaller style.
    Exception: in display style they are in script style.
\stopitem
\startitem
    Subscripts are always in the cramped variant of the style; superscripts are only
    cramped if the original style was cramped.
\stopitem
\startitem
    In an \type {..\over..} formula in any style the numerator and denominator are
    taken from the next smaller style.
\stopitem
\startitem
    The denominator is always in cramped style; the numerator is only in cramped
    style if the original style was cramped.
\stopitem
\startitem
    Formulas under a \type {\sqrt} or \type {\overline} are in cramped style.
\stopitem
\stopitemize

In \LUATEX\ one can set the styles in more detail which means that you sometimes
have to set both normal and cramped styles to get the effect you want. (Even) if
we force styles in the script using \typ {\scriptstyle} and \typ
{\crampedscriptstyle} we get this:

\startbuffer[demo]
\starttabulate
\FL
\BC style         \BC example \NC \NR
\TL
\NC default       \NC $b_{x=xx}^{x=xx}$ \NC \NR
\NC script        \NC $b_{\scriptstyle x=xx}^{\scriptstyle x=xx}$ \NC \NR
\NC crampedscript \NC $b_{\crampedscriptstyle x=xx}^{\crampedscriptstyle x=xx}$ \NC \NR
\LL
\stoptabulate
\stopbuffer

\getbuffer[demo]

Now we set the following parameters using \type {\setmathspacing} that accepts two
class identifier, a style and a value.

\startbuffer[setup]
\setmathspacing 0 3 \scriptstyle = 30mu
\setmathspacing 0 3 \scriptstyle = 30mu
\stopbuffer

\typebuffer[setup]

This gives a different result:

\start\getbuffer[setup,demo]\stop

But, as this is not what is expected (visually) we should say:

\startbuffer[setup]
\setmathspacing 0 3 \scriptstyle        = 30mu
\setmathspacing 0 3 \scriptstyle        = 30mu
\setmathspacing 0 3 \crampedscriptstyle = 30mu
\setmathspacing 0 3 \crampedscriptstyle = 30mu
\stopbuffer

\typebuffer[setup]

Now we get:

\start\getbuffer[setup,demo]\stop

\stopsection

\startsection[title={Math parameters}]

In \LUATEX, the font dimension parameters that \TEX\ used in math typesetting are
now accessible via primitive commands. In fact, refactoring of the math engine
has resulted in turning some hard codes properties into parameters.

{\em The next needs checking ...}

\starttabulate
\FL
\BC primitive name                 \BC description \NC \NR
\TL
\NC \type {\Umathquad}               \NC the width of 18 mu's \NC \NR
\NC \type {\Umathaxis}               \NC height of the vertical center axis of
                                       the math formula above the baseline \NC \NR
\NC \type {\Umathoperatorsize}       \NC minimum size of large operators in display mode \NC \NR
\NC \type {\Umathoverbarkern}        \NC vertical clearance above the rule \NC \NR
\NC \type {\Umathoverbarrule}        \NC the width of the rule \NC \NR
\NC \type {\Umathoverbarvgap}        \NC vertical clearance below the rule \NC \NR
\NC \type {\Umathunderbarkern}       \NC vertical clearance below the rule \NC \NR
\NC \type {\Umathunderbarrule}       \NC the width of the rule \NC \NR
\NC \type {\Umathunderbarvgap}       \NC vertical clearance above the rule \NC \NR
\NC \type {\Umathradicalkern}        \NC vertical clearance above the rule \NC \NR
\NC \type {\Umathradicalrule}        \NC the width of the rule \NC \NR
\NC \type {\Umathradicalvgap}        \NC vertical clearance below the rule \NC \NR
\NC \type {\Umathradicaldegreebefore}\NC the forward kern that takes place before placement of
                                       the radical degree \NC \NR
\NC \type {\Umathradicaldegreeafter} \NC the backward kern that takes place after placement of
                                       the radical degree \NC \NR
\NC \type {\Umathradicaldegreeraise} \NC this is the percentage of the total height and depth of
                                       the radical sign that the degree is raised by; it is
                                       expressed in \type {percents}, so 60\% is expressed as the
                                       integer $60$ \NC \NR
\NC \type {\Umathstackvgap}          \NC vertical clearance between the two
                                       elements in an \type {\atop} stack \NC \NR
\NC \type {\Umathstacknumup}         \NC numerator shift upward in \type {\atop} stack \NC \NR
\NC \type {\Umathstackdenomdown}     \NC denominator shift downward in \type {\atop} stack \NC \NR
\NC \type {\Umathfractionrule}       \NC the width of the rule in a \type {\over} \NC \NR
\NC \type {\Umathfractionnumvgap}    \NC vertical clearance between the numerator and the rule \NC \NR
\NC \type {\Umathfractionnumup}      \NC numerator shift upward in \type {\over} \NC \NR
\NC \type {\Umathfractiondenomvgap}  \NC vertical clearance between the denominator and the rule \NC \NR
\NC \type {\Umathfractiondenomdown}  \NC denominator shift downward in \type {\over} \NC \NR
\NC \type {\Umathfractiondelsize}    \NC minimum delimiter size for \type {\...withdelims} \NC \NR
\NC \type {\Umathlimitabovevgap}     \NC vertical clearance for limits above operators \NC \NR
\NC \type {\Umathlimitabovebgap}     \NC vertical baseline clearance for limits above operators \NC \NR
\NC \type {\Umathlimitabovekern}     \NC space reserved at the top of the limit \NC \NR
\NC \type {\Umathlimitbelowvgap}     \NC vertical clearance for limits below operators \NC \NR
\NC \type {\Umathlimitbelowbgap}     \NC vertical baseline clearance for limits below operators \NC \NR
\NC \type {\Umathlimitbelowkern}     \NC space reserved at the bottom of the limit \NC \NR
\NC \type {\Umathoverdelimitervgap}  \NC vertical clearance for limits above delimiters \NC \NR
\NC \type {\Umathoverdelimiterbgap}  \NC vertical baseline clearance for limits above delimiters \NC \NR
\NC \type {\Umathunderdelimitervgap} \NC vertical clearance for limits below delimiters \NC \NR
\NC \type {\Umathunderdelimiterbgap} \NC vertical baseline clearance for limits below delimiters \NC \NR
\NC \type {\Umathsubshiftdrop}       \NC subscript drop for boxes and subformulas \NC \NR
\NC \type {\Umathsubshiftdown}       \NC subscript drop for characters \NC \NR
\NC \type {\Umathsupshiftdrop}       \NC superscript drop (raise, actually) for boxes and subformulas \NC \NR
\NC \type {\Umathsupshiftup}         \NC superscript raise for characters \NC \NR
\NC \type {\Umathsubsupshiftdown}    \NC subscript drop in the presence of a superscript \NC \NR
\NC \type {\Umathsubtopmax}          \NC the top of standalone subscripts cannot be higher than this
                                       above the baseline \NC \NR
\NC \type {\Umathsupbottommin}       \NC the bottom of standalone superscripts cannot be less than
                                       this above the baseline \NC \NR
\NC \type {\Umathsupsubbottommax}    \NC the bottom of the superscript of a combined super- and subscript
                                       be at least as high as this above the baseline \NC \NR
\NC \type {\Umathsubsupvgap}         \NC vertical clearance between super- and subscript \NC \NR
\NC \type {\Umathspaceafterscript}   \NC additional space added after a super- or subscript \NC \NR
\NC \type {\Umathconnectoroverlapmin}\NC minimum overlap between parts in an extensible recipe \NC \NR
\LL
\stoptabulate

In addition to the above official \OPENTYPE\ font parameters we have these (the
undefined will get presets, quite likely zero):

\starttabulate
\FL
\BC primitive name                             \BC description \NC \NR
\TL
\NC \type {\Umathconnectoroverlapmin}            \NC \NC \NR
\NC \type {\Umathsubsupshiftdown}                \NC \NC \NR
\NC \type {\Umathfractiondelsize}                \NC \NC \NR
\NC \type {\Umathnolimitsupfactor}               \NC a multiplier for the way limits are shifted up and down \NC \NR
\NC \type {\Umathnolimitsubfactor}               \NC a multiplier for the way limits are shifted up and down \NC \NR
\NC \type {\Umathaccentbasedepth}                \NC the complement of \type {\Umathaccentbaseheight} \NC \NR
\NC \type {\Umathflattenedaccentbasedepth}       \NC the complement of \type {\Umathflattenedaccentbaseheight} \NC \NR
\NC \type {\Umathspacebeforescript}              \NC \NC \NR
\NC \type {\Umathprimeraise}                     \NC \NC \NR
\NC \type {\Umathprimeraisecomposed}             \NC \NC \NR
\NC \type {\Umathprimeshiftup}                   \NC the prime variant of \type {\Umathsupshiftup} \NC \NR
\NC \type {\Umathprimespaceafter}                \NC the prescript variant of \type {\Umathspaceafterscript} \NC \NR
\NC \type {\Umathprimeshiftdrop}                 \NC the prime variant of \type {\Umathsupshiftdrop} \NC \NR
\NC \type {\Umathskeweddelimitertolerance}       \NC \NC \NR
\NC \type {\Umathaccenttopshiftup}               \NC the amount that a top accent is shifted up \NC \NR
\NC \type {\Umathaccentbottomshiftdown}          \NC the amount that a bottom accent is shifted down \NC \NR
\NC \type {\Umathaccenttopovershoot}             \NC \NC \NR
\NC \type {\Umathaccentbottomovershoot}          \NC \NC \NR
\NC \type {\Umathaccentsuperscriptdrop}          \NC \NC \NR
\NC \type {\Umathaccentsuperscriptpercent}       \NC \NC \NR
\NC \type {\Umathaccentextendmargin}             \NC margins added to automatically extended accents \NC \NR
\NC \type {\Umathflattenedaccenttopshiftup}      \NC the amount that a wide top accent is shifted up \NC \NR
\NC \type {\Umathflattenedaccentbottomshiftdown} \NC the amount that a wide bottom accent is shifted down \NC \NR
\NC \type {\Umathdelimiterpercent}               \NC \NC \NR
\NC \type {\Umathdelimitershortfall}             \NC \NC \NR
\NC \type {\Umathradicalextensiblebefore}        \NC \NC \NR
\NC \type {\Umathradicalextensibleafter}         \NC \NC \NR
\LL
\stoptabulate

These relate to the font parameters and in \CONTEXT\ we assign some different
defaults and tweak them in the goodie files:

\starttabulate[|T|T|c|]
\FL
\BC font parameter                    \BC primitive name                             \BC default   \NC \NR
\TL
\NC MinConnectorOverlap               \NC \type {\Umathconnectoroverlapmin}            \NC 0         \NC \NR
\NC SubscriptShiftDownWithSuperscript \NC \type {\Umathsubsupshiftdown}                \NC inherited \NC \NR
\NC FractionDelimiterSize             \NC \type {\Umathfractiondelsize}                \NC undefined \NC \NR
\NC FractionDelimiterDisplayStyleSize \NC \type {\Umathfractiondelsize}                \NC undefined \NC \NR
\NC NoLimitSubFactor                  \NC \type {\Umathnolimitsupfactor}               \NC 0         \NC \NR
\NC NoLimitSupFactor                  \NC \type {\Umathnolimitsubfactor}               \NC 0         \NC \NR
\NC AccentBaseDepth                   \NC \type {\Umathaccentbasedepth}                \NC reserved  \NC \NR
\NC FlattenedAccentBaseDepth          \NC \type {\Umathflattenedaccentbasedepth}       \NC reserved  \NC \NR
\NC SpaceBeforeScript                 \NC \type {\Umathspacebeforescript}              \NC 0         \NC \NR
\NC PrimeRaisePercent                 \NC \type {\Umathprimeraise}                     \NC 0         \NC \NR
\NC PrimeRaiseComposedPercent         \NC \type {\Umathprimeraisecomposed}             \NC 0         \NC \NR
\NC PrimeShiftUp                      \NC \type {\Umathprimeshiftup}                   \NC 0         \NC \NR
\NC PrimeShiftUpCramped               \NC \type {\Umathprimeshiftup}                   \NC 0         \NC \NR
\NC PrimeSpaceAfter                   \NC \type {\Umathprimespaceafter}                \NC 0         \NC \NR
\NC PrimeBaselineDropMax              \NC \type {\Umathprimeshiftdrop}                 \NC 0         \NC \NR
\NC SkewedDelimiterTolerance          \NC \type {\Umathskeweddelimitertolerance}       \NC 0         \NC \NR
\NC AccentTopShiftUp                  \NC \type {\Umathaccenttopshiftup}               \NC undefined \NC \NR
\NC AccentBottomShiftDown             \NC \type {\Umathaccentbottomshiftdown}          \NC undefined \NC \NR
\NC AccentTopOvershoot                \NC \type {\Umathaccenttopovershoot}             \NC 0         \NC \NR
\NC AccentBottomOvershoot             \NC \type {\Umathaccentbottomovershoot}          \NC 0         \NC \NR
\NC AccentSuperscriptDrop             \NC \type {\Umathaccentsuperscriptdrop}          \NC 0         \NC \NR
\NC AccentSuperscriptPercent          \NC \type {\Umathaccentsuperscriptpercent}       \NC 0         \NC \NR
\NC AccentExtendMargin                \NC \type {\Umathaccentextendmargin}             \NC 0         \NC \NR
\NC FlattenedAccentTopShiftUp         \NC \type {\Umathflattenedaccenttopshiftup}      \NC undefined \NC \NR
\NC FlattenedAccentBottomShiftDown    \NC \type {\Umathflattenedaccentbottomshiftdown} \NC undefined \NC \NR
\NC DelimiterPercent                  \NC \type {\Umathdelimiterpercent}               \NC 0         \NC \NR
\NC DelimiterShortfall                \NC \type {\Umathdelimitershortfall}             \NC 0         \NC \NR
\LL
\stoptabulate

These parameters not only provide a bit more control over rendering, they also
can be used in compensating issues in font, because no font is perfect. Some are
the side effects of experiments and they have CamelCase companions in the \type
{MathConstants} table. For historical reasons the names are a bit inconsistent as
some originate in \TEX\ so we prefer to keep those names. Not many users will
mess around with these font parameters anyway. \footnote {I wonder if some names
should change, so that decision is pending.}

Each of the parameters in this section can be set by a command like this:

\starttyping
\Umathquad\displaystyle=1em
\stoptyping

they obey grouping, and you can use \type {\the\Umathquad\displaystyle} if
needed.

There are quite some parameters that can be set and there are eight styles, which means a lot
of keying in. For that reason is is possible to set parameters groupwise:

\starttabulate[|l|c|c|c|c|c|c|c|c|]
\FL
\BC primitive name               \BC D \BC D' \BC T \BC T' \BC S \BC S' \BC SS \BC SS' \NC \NR
\TL
\NC \type {\alldisplaystyles}      \NC$+$\NC$+ $\NC   \NC    \NC   \NC    \NC    \NC     \NC \NR
\NC \type {\alltextstyles}         \NC   \NC    \NC$+$\NC$+ $\NC   \NC    \NC    \NC     \NC \NR
\NC \type {\allscriptstyles}       \NC   \NC    \NC   \NC    \NC$+$\NC$+ $\NC    \NC     \NC \NR
\NC \type {\allscriptscriptstyles} \NC   \NC    \NC   \NC    \NC   \NC    \NC$+ $\NC$+  $\NC \NR
\NC \type {\allmathstyles}         \NC$+$\NC$+ $\NC$+$\NC$+ $\NC$+$\NC$+ $\NC$+ $\NC$+  $\NC \NR
\NC \type {\allmainstyles}         \NC   \NC    \NC   \NC    \NC   \NC    \NC    \NC     \NC \NR
\NC \type {\allsplitstyles}        \NC$+$\NC$+ $\NC$+$\NC$+ $\NC$-$\NC$- $\NC$- $\NC$-  $\NC \NR
\NC \type {\allunsplitstyles}      \NC   \NC    \NC   \NC    \NC$+$\NC$+ $\NC$+ $\NC$+  $\NC \NR
\NC \type {\alluncrampedstyles}    \NC$+$\NC    \NC$+$\NC    \NC$+$\NC    \NC$+ $\NC     \NC \NR
\NC \type {\allcrampedstyles}      \NC   \NC$+ $\NC   \NC$+ $\NC   \NC$+ $\NC    \NC$+  $\NC \NR
\LL
\stoptabulate

These groups are especially handy when you set up inter atom spacing, pre- and
post atom penalties and atom rules.

We already introduced the font specific math parameters but we tell abit more
about them and how they relate to the original \TEX\ font dimensions.

While it is nice to have these math parameters available for tweaking, it would
be tedious to have to set each of them by hand. For this reason, \LUATEX\
initializes a bunch of these parameters whenever you assign a font identifier to
a math family based on either the traditional math font dimensions in the font
(for assignments to math family~2 and~3 using \TFM|-|based fonts like \type
{cmsy} and \type {cmex}), or based on the named values in a potential \type
{MathConstants} table when the font is loaded via Lua. If there is a \type
{MathConstants} table, this takes precedence over font dimensions, and in that
case no attention is paid to which family is being assigned to: the \type
{MathConstants} tables in the last assigned family sets all parameters.

In the table below, the one|-|letter style abbreviations and symbolic tfm font
dimension names match those used in the \TeX book. Assignments to \typ
{\textfont} set the values for the cramped and uncramped display and text styles,
\typ {\scriptfont} sets the script styles, and \typ {\scriptscriptfont} sets the
scriptscript styles, so we have eight parameters for three font sizes. In the
\TFM\ case, assignments only happen in family~2 and family~3 (and of course only
for the parameters for which there are font dimensions).

Besides the parameters below, \LUATEX\ also looks at the \quote {space} font
dimension parameter. For math fonts, this should be set to zero.

\def\MathLine#1#2#3#4#5%
  {\TB
   \NC \llap{\high{\tx #2\enspace}}\tt \type {#1} \NC \tt #5 \NC \NR
   \NC \tx #3 \NC \tt #4 \NC \NR}

\starttabulate[|l|l|]
\FL
\BC variable / style \BC tfm / opentype \NC \NR
\TL
\MathLine{\Umathaxis}                     {}   {}                     {AxisHeight}                              {axis_height}
\MathLine{\Umathaccentbaseheight}         {}   {}                     {AccentBaseHeight}                        {xheight}
\MathLine{\Umathflattenedaccentbaseheight}{}   {}                     {FlattenedAccentBaseHeight}               {xheight}
\MathLine{\Umathoperatorsize}             {6}  {D, D'}                {DisplayOperatorMinHeight}                {\emdash}
\MathLine{\Umathfractiondelsize}          {9}  {D, D'}                {FractionDelimiterDisplayStyleSize}       {delim1}
\MathLine{\Umathfractiondelsize}          {9}  {T, T', S, S', SS, SS'}{FractionDelimiterSize}                   {delim2}
\MathLine{\Umathfractiondenomdown}        {}   {D, D'}                {FractionDenominatorDisplayStyleShiftDown}{denom1}
\MathLine{\Umathfractiondenomdown}        {}   {T, T', S, S', SS, SS'}{FractionDenominatorShiftDown}            {denom2}
\MathLine{\Umathfractiondenomvgap}        {}   {D, D'}                {FractionDenominatorDisplayStyleGapMin}   {3*default_rule_thickness}
\MathLine{\Umathfractiondenomvgap}        {}   {T, T', S, S', SS, SS'}{FractionDenominatorGapMin}               {default_rule_thickness}
\MathLine{\Umathfractionnumup}            {}   {D, D'}                {FractionNumeratorDisplayStyleShiftUp}    {num1}
\MathLine{\Umathfractionnumup}            {}   {T, T', S, S', SS, SS'}{FractionNumeratorShiftUp}                {num2}
\MathLine{\Umathfractionnumvgap}          {}   {D, D'}                {FractionNumeratorDisplayStyleGapMin}     {3*default_rule_thickness}
\MathLine{\Umathfractionnumvgap}          {}   {T, T', S, S', SS, SS'}{FractionNumeratorGapMin}                 {default_rule_thickness}
\MathLine{\Umathfractionrule}             {}   {}                     {FractionRuleThickness}                   {default_rule_thickness}
\MathLine{\Umathskewedfractionhgap}       {}   {}                     {SkewedFractionHorizontalGap}             {math_quad/2}
\MathLine{\Umathskewedfractionvgap}       {}   {}                     {SkewedFractionVerticalGap}               {math_x_height}
\MathLine{\Umathlimitabovebgap}           {}   {}                     {UpperLimitBaselineRiseMin}               {big_op_spacing3}
\MathLine{\Umathlimitabovekern}           {1}  {}                     {0}                                       {big_op_spacing5}
\MathLine{\Umathlimitabovevgap}           {}   {}                     {UpperLimitGapMin}                        {big_op_spacing1}
\MathLine{\Umathlimitbelowbgap}           {}   {}                     {LowerLimitBaselineDropMin}               {big_op_spacing4}
\MathLine{\Umathlimitbelowkern}           {1}  {}                     {0}                                       {big_op_spacing5}
\MathLine{\Umathlimitbelowvgap}           {}   {}                     {LowerLimitGapMin}                        {big_op_spacing2}
\MathLine{\Umathoverdelimitervgap}        {}   {}                     {StretchStackGapBelowMin}                 {big_op_spacing1}
\MathLine{\Umathoverdelimiterbgap}        {}   {}                     {StretchStackTopShiftUp}                  {big_op_spacing3}
\MathLine{\Umathunderdelimitervgap}       {}   {}                     {StretchStackGapAboveMin}                 {big_op_spacing2}
\MathLine{\Umathunderdelimiterbgap}       {}   {}                     {StretchStackBottomShiftDown}             {big_op_spacing4}
\MathLine{\Umathoverbarkern}              {}   {}                     {OverbarExtraAscender}                    {default_rule_thickness}
\MathLine{\Umathoverbarrule}              {}   {}                     {OverbarRuleThickness}                    {default_rule_thickness}
\MathLine{\Umathoverbarvgap}              {}   {}                     {OverbarVerticalGap}                      {3*default_rule_thickness}
\MathLine{\Umathquad}                     {1}  {}                     {<font_size(f)>}                          {math_quad}
\MathLine{\Umathradicalkern}              {}   {}                     {RadicalExtraAscender}                    {default_rule_thickness}
\MathLine{\Umathradicalrule}              {2}  {}                     {RadicalRuleThickness}                    {<not set>}
\MathLine{\Umathradicalvgap}              {3}  {D, D'}                {RadicalDisplayStyleVerticalGap}          {default_rule_thickness+abs(math_x_height)/4}
\MathLine{\Umathradicalvgap}              {3}  {T, T', S, S', SS, SS'}{RadicalVerticalGap}                      {default_rule_thickness+abs(default_rule_thickness)/4}
\MathLine{\Umathradicaldegreebefore}      {2}  {}                     {RadicalKernBeforeDegree}                 {<not set>}
\MathLine{\Umathradicaldegreeafter}       {2}  {}                     {RadicalKernAfterDegree}                  {<not set>}
\MathLine{\Umathradicaldegreeraise}       {2,7}{}                     {RadicalDegreeBottomRaisePercent}         {<not set>}
\MathLine{\Umathspaceafterscript}         {4}  {}                     {SpaceAfterScript}                        {script_space}
\MathLine{\Umathstackdenomdown}           {}   {D, D'}                {StackBottomDisplayStyleShiftDown}        {denom1}
\MathLine{\Umathstackdenomdown}           {}   {T, T', S, S', SS, SS'}{StackBottomShiftDown}                    {denom2}
\MathLine{\Umathstacknumup}               {}   {D, D'}                {StackTopDisplayStyleShiftUp}             {num1}
\MathLine{\Umathstacknumup}               {}   {T, T', S, S', SS, SS'}{StackTopShiftUp}                         {num3}
\MathLine{\Umathstackvgap}                {}   {D, D'}                {StackDisplayStyleGapMin}                 {7*default_rule_thickness}
\MathLine{\Umathstackvgap}                {}   {T, T', S, S', SS, SS'}{StackGapMin}                             {3*default_rule_thickness}
\MathLine{\Umathsubshiftdown}             {}   {}                     {SubscriptShiftDown}                      {sub1}
\MathLine{\Umathsubshiftdrop}             {}   {}                     {SubscriptBaselineDropMin}                {sub_drop}
\MathLine{\Umathsubsupshiftdown}          {8}  {}                     {SubscriptShiftDownWithSuperscript}       {\emdash}
\MathLine{\Umathsubtopmax}                {}   {}                     {SubscriptTopMax}                         {abs(math_x_height*4)/5}
\MathLine{\Umathsubsupvgap}               {}   {}                     {SubSuperscriptGapMin}                    {4*default_rule_thickness}
\MathLine{\Umathsupbottommin}             {}   {}                     {SuperscriptBottomMin}                    {abs(math_x_height/4)}
\MathLine{\Umathsupshiftdrop}             {}   {}                     {SuperscriptBaselineDropMax}              {sup_drop}
\MathLine{\Umathsupshiftup}               {}   {D}                    {SuperscriptShiftUp}                      {sup1}
\MathLine{\Umathsupshiftup}               {}   {T, S, SS,}            {SuperscriptShiftUp}                      {sup2}
\MathLine{\Umathsupshiftup}               {}   {D', T', S', SS'}      {SuperscriptShiftUpCramped}               {sup3}
\MathLine{\Umathsupsubbottommax}          {}   {}                     {SuperscriptBottomMaxWithSubscript}       {abs(math_x_height*4)/5}
\MathLine{\Umathunderbarkern}             {}   {}                     {UnderbarExtraDescender}                  {default_rule_thickness}
\MathLine{\Umathunderbarrule}             {}   {}                     {UnderbarRuleThickness}                   {default_rule_thickness}
\MathLine{\Umathunderbarvgap}             {}   {}                     {UnderbarVerticalGap}                     {3*default_rule_thickness}
\MathLine{\Umathconnectoroverlapmin}      {5}  {}                     {MinConnectorOverlap}                     {0}
\LL
\stoptabulate

A few notes:

\startitemize[n]

\startitem
    \OPENTYPE\ fonts set \typ {\Umathlimitabovekern} and \typ
    {\Umathlimitbelowkern} to zero and set \typ {\Umathquad} to the font size of
    the used font, because these are not supported in the \type {MATH} table.
\stopitem

\startitem
    Traditional \TFM\ fonts do not set \typ {\Umathradicalrule} because \TEX82\
    uses the height of the radical instead. When this parameter is indeed not set
    when \LUATEX\ has to typeset a radical, a backward compatibility mode will
    kick in that assumes that an oldstyle \TEX\ font is used. Also, they do not
    set \typ {\Umathradicaldegreebefore}, \typ {\Umathradicaldegreeafter}, and \typ
    {\Umathradicaldegreeraise}. These are then automatically initialized to
    $5/18$quad, $-10/18$quad, and 60.
\stopitem

\startitem
    If \TFM\ fonts are used, then the \typ {\Umathradicalvgap} is not set until
    the first time \LUATEX\ has to typeset a formula because this needs
    parameters from both family~2 and family~3. This provides a partial backward
    compatibility with \TEX82, but that compatibility is only partial: once the
    \typ {\Umathradicalvgap} is set, it will not be recalculated any more.
\stopitem

\startitem
    When \TFM\ fonts are used a similar situation arises with respect to \typ
    {\Umathspaceafterscript}: it is not set until the first time \LUATEX\ has to
    typeset a formula. This provides some backward compatibility with \TEX82. But
    once the \typ {\Umathspaceafterscript} is set, \typ {\scriptspace} will never
    be looked at again.
\stopitem

\startitem
    Traditional \TFM\ fonts set \typ {\Umathconnectoroverlapmin} to zero because
    \TEX82\ always stacks extensibles without any overlap.
\stopitem

\startitem
    The \typ {\Umathoperatorsize} is only used in \typ {\displaystyle}, and is only
    set in \OPENTYPE\ fonts. In \TFM\ font mode, it is artificially set to one
    scaled point more than the initial attempt's size, so that always the \quote
    {first next} will be tried, just like in \TEX82.
\stopitem

\startitem
    The \typ {\Umathradicaldegreeraise} is a special case because it is the only
    parameter that is expressed in a percentage instead of a number of scaled
    points.
\stopitem

\startitem
    \typ {SubscriptShiftDownWithSuperscript} does not actually exist in the
    \quote {standard} \OPENTYPE\ math font Cambria, but it is useful enough to be
    added.
\stopitem

\startitem
    \typ {FractionDelimiterDisplayStyleSize} and \typ {FractionDelimiterSize}
    do not actually exist in the \quote {standard} \OPENTYPE\ math font Cambria,
    but were useful enough to be added.
\stopitem

\stopitemize

As this mostly refers to \LUATEX\ there is more to tell about how \LUAMETATEX\
deals with it. However, it is enough to know that much more behavior is
configurable.

You can let the engine ignore a parameter with \typ {\setmathignore}, like:

\starttyping
\setmathignore \Umathspacebeforescript 1
\setmathignore \Umathspaceafterscript  1
\stoptyping

Be aware of the fact that a global setting can get unnoticed by users because
there is no warning that some parameter is ignored.

There are a couple of parameters that don't relate to the font but are more generally
influencing the appearances. Some were added for experimenting.

{\em This is not complete}

\starttabulate[|l|l|]
\FL
\BC primitive \BC meaning \NC \NR
\TL
\type {\Umathextrasubpreshift} \NC \NR
\type {\Umathextrasubprespace} \NC \NR
\type {\Umathextrasubshift}    \NC \NR
\type {\Umathextrasubspace}    \NC \NR
\type {\Umathextrasuppreshift} \NC \NR
\type {\Umathextrasupprespace} \NC \NR
\type {\Umathextrasupshift}    \NC \NR
\type {\Umathextrasupspace}    \NC \NR
\type {\Umathprimeshiftdrop}   \NC \NR
\LL
\stoptabulate

\stopsection

\startsection[title={Math spacing}]

Besides the parameters mentioned in the previous sections, there are also
primitives to control the math spacing table (as explained in Chapter~18 of the
\TEX book). This happens per class pair. Because we have many possible classes,
we no longer have the many primitives that \LUATEX\ has but you can define then
using the generic \type {\setmathspacing} primitive:

\starttyping
\def\Umathordordspacing     {\setmathspacing 0 0 }
\def\Umathordordopenspacing {\setmathspacing 0 4 }
\stoptyping

These parameters are (normally) of type \type {\muskip}, so setting a parameter can
be done like this:

\starttyping
\setmathspacing 1 0 \displaystyle=4mu plus 2mu % op ord Umathopordspacing
\stoptyping

The atom pairs known by the engine are all initialized by \type {initex} to the
values mentioned in the table in Chapter~18 of the \TEX book.

For ease of use as well as for backward compatibility, \typ {\thinmuskip}, \typ
{\medmuskip} and \typ {\thickmuskip} are treated specially. In their case a pointer
to the corresponding internal parameter is saved, not the actual \type {\muskip}
value. This means that any later changes to one of these three parameters will be
taken into account. As a bonus we also introduced the \typ {\tinymuskip} and \typ
{\pettymuskip} primitives, just because we consider these fundamental, but they
are not assigned internally to atom spacing combinations.

In \LUAMETATEX\ we go a bit further. Any named dimension, glue and mu glue
register as well as the constants with these properties can be bound to a pair by
prefixing \typ {\setmathspacing} by \type {\inherited}.

Careful readers will realize that there are also primitives for the items marked
\type {*} in the \TEX book. These will actually be used because we pose no
restrictions. However, you can enforce the remapping rules to conform to the
rules of \TEX\ (or yourself).

Every class has a set of spacing parameters and the more classes you define the more
pairwise spacing you need to define. However, you can default to an existing class.
By default all spacing is zero and you can get rid of the defaults inherited from
good old \TEX\ with \typ {\resetmathspacing}. You can alias class spacing to an exiting
class with \type {\letmathspacing}:

\starttyping
\letmathspacing class displayclass textclass scriptclass scriptscriptclass
\stoptyping

Instead you can copy spacing with \type {\copymathspacing}:

\starttyping
\copymathspacing class parentclass
\stoptyping

Specific paring happens with \type {\setmathspacing}:

\starttyping
\setmathspacing leftclass rightclass style value
\stoptyping

Unless we have a frozen parameter, the prefix \type {\inherited} makes it possible
to have a more dynamic relationship: the used value resolves to the current value
of the given register. Possible values are the usual mu skip register, a regular
skip or dimension register, or just some mu skip value.

A similar set of primitives deals with rules. These remap pairs onto other pairs, so
\typ {\setmathatomrule} looks like:

\starttyping
\setmathatomrule oldleftclass oldrightclass newleftclass newrightclass
\stoptyping

The \type {\letmathatomrule} and \type {\copymathatomrule} primitives take two
classes where the second is the parent.

% Some primitives are still experimental and might evolve, like \type
% {\letmathparent} and \type {\copymathparent} that take numbers as in:
%
% \starttyping
% \letmathatomrule class spacingclass prepenaltyclass postpenaltyclass options reserved
% \stoptyping
%
% Primitives like this were used when experimenting and when re use them in \CONTEXT\
% eventually they will become stable.

The \type {\setmathprepenalty} and \type {\setmathpostpenalty} primitives take a
class and penalty (integer) value. These are injected before and after atoms with
the given class where a penalty of 10000 is a signal to ignore it.

The engine control options for a class can be set with \type {\setmathoptions}. The
possible options are discussed elsewhere. This primitive takes a class number and
an integer (bitset). For all these setters the \CONTEXT\ math setup gives examples.

Math is processed in two passes. The first pass is needed to intercept for
instance \type {\over}, one of the few \TEX\ commands that actually has a
preceding argument. There are often lots of curly braces used in math and these
can result in a nested run of the math sub engine. However, you need to be aware
of the fact that some properties are kind of global to a formula and the last
setting (for instance a family switch) wins. This also means that a change (or
again, the last one) in math parameters affects the whole formula. In
\LUAMETATEX\ we have changed this model a bit. One can argue that this introduces
an incompatibility but it's hard to imagine a reason for setting the parameters
at the end of a formula run and assume that they also influence what goes in
front.

\startbuffer
$
                                               x \subscript  {-}
     \frozen\Umathsubshiftdown\textstyle  0pt  x \subscript  {0}
    {\frozen\Umathsubshiftdown\textstyle  5pt  x \subscript  {5}}
                                               x \subscript  {0}
    {\frozen\Umathsubshiftdown\textstyle 15pt  x \subscript {15}}
                                               x \subscript  {0}
    {\frozen\Umathsubshiftdown\textstyle 20pt  x \subscript {20}}
                                               x \subscript  {0}
     \frozen\Umathsubshiftdown\textstyle 10pt  x \subscript {10}
                                               x \subscript  {0}
$
\stopbuffer

\typebuffer

The \type {\frozen} prefix does the magic: it injects information in the
math list about the set parameter.

In \LUATEX\ 1.10+ the last setting, the \type {10pt} drop wins, but in
\LUAMETATEX\ you will see each local setting taking effect. The implementation
uses a new node type, parameters nodes, so you might encounter these in an
unprocessed math list. The result looks as follows:

\blank \getbuffer \blank

The \type {\mathatom} primitive is the generic one and it accepts a couple of
keywords:

{\em to be checked}

\starttabulate[|lT|l|l|]
\FL
\BC keyword    \BC argument \NC meaning \NC \NR
\TL
\NC attr       \NC int int \NC attributes to be applied to this atom \NC \NR
\NC leftclass  \NC class   \NC the left edge class that determines spacing etc \NC \NR
\NC rightclass \NC class   \NC the right edge class that determines spacing etc \NC \NR
\NC class      \NC class   \NC the general class \NC \NR
\NC unpack     \NC         \NC unpack this atom in inline math \NC \NR
\NC source     \NC int     \NC a symbolic index of the resulting box \NC \NR
\NC textfont   \NC         \NC use the current text font \NC \NR
\NC mathfont   \NC         \NC use the current math font \NC \NR
\NC limits     \NC         \NC put scripts on top and below \NC \NR
\NC nolimits   \NC         \NC force scripts to be postscripts \NC \NR
\NC nooverflow \NC         \NC keep (extensible) within target dimensions \NC \NR
\NC options    \NC int     \NC bitset with options \NC \NR
\NC void       \NC         \NC discard content and ignore dimensions \NC \NR
\NC phantom    \NC         \NC discard content but retain dimensions \NC \NR
\LL
\stoptabulate

To what extend the options kick in depends on the class as well where and how the
atom is used.

The original \TEX\ engines has three atom modifiers: \type {\displaylimits}, \type
{\limits}, and \type {\nolimits}. These look back to the last atom and set a limit
related signal. Just to be consistent we have some more of that: \typ
{\Umathadapttoleft}, \typ {\Umathadapttoright}, \typ {\Umathuseaxis}, \typ
{\Umathnoaxis}, \typ {\Umathphantom}, \typ {\Umathvoid}, \typ {\Umathsource}, \typ
{\Umathopenupheight}, \typ {\Umathopenupdepth}, \typ {\Umathlimits}, \typ
{\Umathnolimits}. The last two are equivalent to the lowercase ones with the
similar names. Al these modifiers are cheap primitives and one can wonder if they
are needed but that also now also applies to the original three. We could stick
to one modifier that takes an integer but let's not diverge too much from the
original concept.

The \type {\nonscript} primitive injects a glue node that signals that the next
glue is to be ignored when we are in script or scriptscript mode. The \typ
{\noatomruling} does the same but this time the signal is that no inter|-|atom
rules need to be applied.

\stopsection

\startsection[title={Fonts}]

When we started with \LUATEX\ there was only Cambria as \OPENTYPE\ math font.
However, as soon as we could load a wide font, and basic math handling was
adapted to handle a fonts passed via \LUA, in \CONTEXT\ we switched to \OPENTYPE\
math exclusively. This was possible because at the same time virtual fonts were
integrated in the engine. Because the way \TEX\ approaches math differs from
\OPENTYPE\ we had code paths that could handle both and were somewhat complex.
Later these code paths were split more visible and detailed control over specific
features was introduced. The reason for this came from the fact that the Latin
Modern Math as well as additional fonts were a mix of \OPENTYPE\ and traditional
(metric wise). Inconsistencies were handles by \CONTEXT\ when loading and passing
fonts and runtime patching was our way out. There is also some juggling of math
lists in \LUA\ involved.

In \LUAMETATEX\ much more control was added alongside many new features in
rendering math. Although by making decisions with respect to fonts in the end we
could potentially use a much simpler code base. However we keep what we have
because we need to write articles, manuals, presentations etc.\ that show the
differences. We settled on the fact that fonts are what they are and won't
change. Font specific tweaks are dealt with in a math font goodie file: most
tweaks are generic and applied to all fonts, some are optional, and many can be
tuned by parameters. In the end one can argue that we render math a bit different
due to different font and character properties; for instance we got rid of italic
correction and often deal with kerning, variants and extensibles a bit different.

A consequence of this is that we will not describe in detail what happens in the
math engine, first of all because we don't expect other macro packages to follow
\CONTEXT\ in the way it deals with rendering math and the \LUATEX\ kind of hybrid
approach is likely good enough because after all there was never demand for more
advanced math rendering nor attempts to extend the engines in that area. This is
why \LUAMETATEX\ tries to be \LUATEX\ compatible when it comes to the basics
required by potential other usage than \CONTEXT. However, we might eventually
drop some eight bit font related features, simply because one can pass them
wrapped in a \UNICODE\ and \OPENTYPE\ math disguise. This is to be decided.

The process of upgrading math is described in manuals, articles and presentations
by the authors. There one can find a discussion about decisions made.

\stopsection

\startsection[title={Scripts}]

\startbuffer[show:3]
\startlinecorrection[blank]
    \scale[s=3]\bgroup
        \switchtobodyfont[modern]%
        \showmakeup[glyph,continuation]%
        \getbuffer
    \egroup
\stoplinecorrection
\stopbuffer

The \LUAMETATEX\ engine has native support for prescripts and primes. Here we
dive a bit into the former. We start with a regular sub and superscript example:

\startbuffer
\im { F _ {a} ^ {b} }
\stopbuffer

\typebuffer \getbuffer[show:3]

Depending on how the font is set up, a subscript might get a (negative) kern.
Kerning at the top left of the nucleus is ignored, because one never sees it in
for instance chemistry:

\startbuffer
\im { F _ {a} ^ {b} ___ {c} ^^^ {d} }
\stopbuffer

\typebuffer \getbuffer[show:3]

There can be multiple pre- and postscripts. In traditional \TEX\ one sometimes
has to inject fake nuclei but in \LUAMETATEX\ this is done automatically. These
are called continuation atoms.

\startbuffer
\im { F
    _ {a} ^ {b}
    _ {a} ^ {b}
} \quad
\im { F
    _ {a} ^ {b} ___ {c} ^^^ {d}
    _ {a} ^ {b} ___ {c} ^^^ {d}
}
\stopbuffer

\typebuffer \getbuffer[show:3]

You will notice that the subscript no longer aligns, a feature that deals with
rendering tensors. these features are controlled by the (four byte) \prm
{mathdoublescriptmode} parameter. In \CONTEXT\ this one is set up as follows:

\starttyping
\mathdoublescriptmode
  "\tohexadecimal\numexpr
     \inheritclassdoublescriptmodecode
   + \discardshapekerndoublescriptmodecode
   + \realignscriptsdoublescriptmodecode
   + \reorderprescriptsdoublescriptmodecode
   \relax
   \tohexadecimal\mathcontinuationcode % 2 bytes
   \tohexadecimal\mathcontinuationcode % 2 bytes
   \tohexadecimal\mathcontinuationcode % 2 bytes
\stoptyping

The first byte set the options, the second the subtype of the continuation node
and the last two set the left and right class values. In \CONTEXT\ we have a
dedicated continuation class (\mono{0x\tohexadecimal\mathcontinuationcode}). So,
current value of this parameter is \mono{0x\tohexadecimal\mathdoublescriptmode},
but we can do this:

\startbuffer[temp]
\advance\mathdoublescriptmode
   -"\tohexadecimal\discardshapekerndoublescriptmodecode 000000
\stopbuffer

\typebuffer[temp]

and get:

{\getbuffer[temp,show:3]}

\startbuffer
\im { F
    _ {a} ^ {b}
    ___ {c}
    ___ {c}
}
\stopbuffer

\typebuffer \getbuffer[show:3]

The next set of examples demonstrates that the \type {\noscript} injects a bogus
atom that breaks the alignment chain. You probably never needs this feature which
is kind of a left-over that we keep around.

\startbuffer
\im { F ^ {a}
    ___ {a} ___ {a} ___ {a}
} \quad
\im { F ^ {a}
    \noscript ___ {a} ___ {a} ___ {a}
} \quad
\im { F ^ {a}
    ___ {a} \noscript ___ {a} ___ {a}
} \quad
\im { F ^ {a}
    ___ {a}  ___ {a} \noscript ___ {a} ___ {a}
}
\stopbuffer

\typebuffer \getbuffer[show:3]

\startbuffer
\im { F ^ {a} _{a}
    ___ {a} ___ {a} ___ {a}
} \quad
\im { F ^ {a} _{a}
    \noscript ___ {a} ___ {a} ___ {a}
} \quad
\im { F ^ {a} _{a}
    ___ {a} \noscript ___ {a} ___ {a}
} \quad
\im { F ^ {a} _{a}
    ___ {a}  ___ {a} \noscript ___ {a} ___ {a}
}
\stopbuffer

\typebuffer \getbuffer[show:3]

% \startbuffer
% \m { F _ {c} }
% \m { F _ {c} _ {d} }
% \m { F _ {c} \noscript _ {d} \noscript _ {e} }
% \stopbuffer
%
% \typebuffer \getbuffer[show:3]
%
% \startbuffer
% \m { F ^ {c} }
% \m { F ^ {c} ^ {d} }
% \m { F ^ {c} \noscript ^ {d} \noscript ^ {e} }
% \stopbuffer
%
% \typebuffer \getbuffer[show:3]
%
% \startbuffer
% \m { F _c ^ {c} }
% \m { F _c ^ {c} _ {d} ^ {d} }
% \m { F _c ^ {c} \noscript _ {d} ^ {d} \noscript ^ {e} _ {e} }
% \stopbuffer
%
% \typebuffer \getbuffer[show:3]
%
% \startbuffer
% \m { F ___ {c} }
% \m { F ___ {c} ___ {d} }
% \m { F ___ {c} \noscript ___ {d} \noscript ___ {e} }
% \stopbuffer
%
% \typebuffer \getbuffer[show:3]
%
% \startbuffer
% \m { F ^^^ {c} }
% \m { F ^^^ {c} ^^^ {d} }
% \m { F ^^^ {c} \noscript ^^^ {d} \noscript ^^^ {e} }
% \stopbuffer
%
% \typebuffer \getbuffer[show:3]
%
% \startbuffer
% \m { F _ {c} ^ {c} }
% \m { F _ {c} ^ {c} _ {d} ^ {d} }
% \m { F _ {c} ^ {c} \noscript _ {d} ^ {d} \noscript ^ {e} _ {e} }
% \stopbuffer
%
% \typebuffer \getbuffer[show:3]
%
% \startbuffer
% \m { F
%     ^{a} _{d} ^^^{d} ___ {d} \noscript
%     ^{b} _{e} ^^^{e} ___ {e} \noscript
%     ^{c} _{f} ^^^{f} ___ {f}
% }
% \stopbuffer
%
% \typebuffer \getbuffer[show:3]
%
% \startbuffer
% \m { F
%     ^^^{d} ___ {d}
%     ^  {a} _   {d} \noscript
%     ^  {a} _   {d} \noscript
%     ^  {c} _   {f}
% }
% \stopbuffer
%
% \typebuffer \getbuffer[show:3]
%
% \startbuffer
% \m { F
%     ___ {d}
%     _   {f}
% }
% \m { F
%     ___ {d} \noscript
%     _   {f}
% }
% \stopbuffer
%
% \typebuffer \getbuffer[show:3]
%
% \startbuffer
% \m { F
%     ___ {a}
%     _   {b} \noscript
%     _   {c} \noscript
%     _   {d} \noscript
%     _   {e}
% }
% \stopbuffer
%
% \typebuffer \getbuffer[show:3]

\stopsection

\stopdocument
