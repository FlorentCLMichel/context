% language=us runpath=texruns:manuals/luametatex

\environment luametatex-style

\startdocument[title=Engines]

\startsection[title={Introduction}]

There are good reasons why we started the \LUATEX\ and later \LUAMETATEX\
projects. Here I will go into some of them. It is just short wrap up of how it
started, how other engines influenced the process and how we see usage. There are
plenty of documents out there that go into more detail. The main objective of
this section is to put documentation into perspective.

\stopsection

\startsection[title={How it started}]

When we started with \CONTEXT, hardware was rather limited compared to what we
have today. A personal computer had some 640kB memory, possibly bumped to 1MB
with help from a memory extender. This put some restrictions to how macro
packages could be defined, also because that memory had to be shared with the
baseline operating system. However, over time memory and runtime became less of
an issue and the \TEX\ engine could be configured to use whatever was available.
Extending the program other than increasing the available memory became more
feasible.

As with any program, there is always something to wish for which is why the
\ETEX\ variant came into view. Before those extensions could be used, \PDFTEX\
showed up. That variant simplified the \quote {\TEX\ plus separate backend
driver} model to a one-step process. Eventually \ETEX\ was merged into \PDFTEX,
and that became the de facto standard engine. There was never a follow up on
\ETEX, and more drastic deviations like \OMEGA\ were never ready for production.
At some point \XETEX\ came around but that was mostly a font specific extension.
We were kind of stuck with a wish-list that never would be fulfilled but we
occasionally pondered a follow up. We drafted an extended \ETEX\ proposal, played
with some features related to \PDF, improved a few things but that was it.

Having some experiences with \LUA\ as extension language in \SCITE, I wondered
what something like that would bring to \TEX\ and after discussing this with
Hartmut he made variant of \PDFTEX\ that has some basic interfaces: we could
access properties of registers and print something to \TEX\ as if it came from
file. As is common with some variant, a new name was coined and \LUATEX\ came
into existence. We're talking 2005.

Because Idris wanted to typeset high quality scholar manuscripts mixing Arabic
and Latin we discussed how to do that in \CONTEXT\ and his experiences with
\OMEGA\ were such that alternatives had to be considered: the Oriental \TEX\
project was started and \LUATEX\ was the starting point. Taco merged some parts
of \ALEPH\ (a somewhat stable variant of \OMEGA) into the code base and stepwise
some primitives were added. It was overall a rather large and serious project
that took a lot of our time. It was not commercially driven, mostly for \CONTEXT\
users and therefore also a lot of fun to do. As often with such projects, early
adapters keep things going.

It took a while before \LUATEX\ was stable in the sense that nothing more was
added. Because the engine was developed alongside what is called \CONTEXT\ \MKIV,
we could easily adapt both to each other. Even better: users could use both in
production. However, in order for other macro packages to use \LUATEX\ (per
request) it had to be frozen, and that happened around 2015, some 10 years after
we started. However, we were not done yet and in order not to violate this stability
principle the follow up was called \LUAMETATEX. Because it was a more drastic
extension project, and also a somewhat drastic separation of the code base from
the complex \LUATEX\ one, the related \CONTEXT\ code was also separated, this time
tagged \MKXL, or \LMTX\ when we talk about the combination. The project started
around 2019 and soon again entered a state of combined development and use in
production and most users switched to this variant.

There are more complete wrap ups of these developments and we systematically
reported on them in various documents that are available in the distribution
and|/|or published in user group journals.

\stopsection

\startsection[title={The engines}]

Of course all starts with original \TEX. We want to be compatible so we keep that
functionality. However, for practical reasons \LUAMETATEX\ omits two core
components. Font loading is not present in the frontend and there is no backend.
Both are supposed to be provided via \LUA\ plugins. This makes sense because in
the meantime font technologies have changed and keep changing and backend also
are a moving target. In \CONTEXT\ we already did all that in \LUA, so there was
no need to keep that font and \PDF\ generation code around in the engine. There
are a few more deviations, like dropping some system specific features (terminal
related) and in former times practical features like outer and long macros that
no longer made sense and complicated integrating new features unnecessarily.

As mentioned in the introduction \PDFTEX\ is the basis for \LUATEX\ and \LUATEX\ is
where we started with \LUAMETATEX. If we compare \PDFTEX\ with traditional \TEX\
the main additions are:

\startitemize[packed]
\startitem
    There is an integrated \PDF\ backend that also supports for instance hyperlinks and
    various annotations.
\stopitem
\startitem
    Expansion of glyphs (aka hz) has been added to the engine and integrated in the
    par builder. The same is true for character protrusion (in the margin).
\stopitem
\startitem
    There is, to some extend, support for inter-character kerning.
\stopitem
\startitem
    There are some handy helpers, for instance for calculating hashes,
    randomization, etc.
\stopitem
\startitem
    There is an extension to injection between lines (adjust).
\stopitem
\startitem
    We have few more conditionals (like testing for a csname and absolute values).
\stopitem
\startitem
    A few helpers like \type {\quitvmode} (that we liked to have in \CONTEXT)
    were added.
\stopitem
\stopitemize

Because \PDFTEX\ was actively developed and maintained over many years,
extensions showed up stepwise, also depending on usage and needs. That is also
why the \ETEX\ extensions were included:

\startitemize[packed]
\startitem
    More that 256 registers, including marks.
\stopitem
\startitem
    Access to discarded material in the vertical splitting code.
\stopitem
\startitem
    Protection against expansion of macros (the \type {\protected} prefix).
\stopitem
\startitem
    A simple right to left typesetting mechanism.
\stopitem
\startitem
    Access to some states, a limited set of last nodes, etc.
\stopitem
\startitem
    There are some additional tracing features.
\stopitem
\startitem
    One can reprocess tokens and produce detokenized lists.
\stopitem
\stopitemize

In \LUATEX\ we also looked at what \OMEGA\ could bring:

\startitemize[packed]
\startitem
    More that 256 registers.
\stopitem
\startitem
    Multi-directional typesetting.
\stopitem
\startitem
    Local boxes (in lines).
\stopitem
\startitem
    Input processing.
\stopitem
\stopitemize

If we combine these lists, we see font expansion and protrusion coming back in
\LUAMETATEX. However, already in \LUATEX\ expansion and protrusion were dealt
with a bit differently and even more so in \LUAMETATEX, while protection in
\LUAMETATEX\ is implemented differently. We also kept injection of vertical
material but in \LUAMETATEX\ that done quite differently. Most if \ETEX\ is there
but not right to left typesetting and the register approach. Of course we kept
the additional conditionals but implemented them a bit different.

In \LUATEX\ we took the \OMEGA\ enlarged register approach and directional
typesetting although that has been stripped down and redone to right to left
only. Local boxes are there but redone in \LUAMETATEX. There was no need for
input processing because we have \LUA. In the end there is little that we kept
from the other engines which also means that one cannot take the manuals that
come with these engines and simply assume that it is there.

We should of course mention \METAPOST. That graphical subsystem was integrated in
\LUATEX\ and on the one hand stripped down (less backend) and extended (remove
bottlenecks and add some functionality) in \LUAMETATEX. With respect to \LUA\
we moved to more recent versions and dropped support for just in time compilation.

There is of course a lot in \LUATEX\ that can be found also in \LUAMETATEX\ but
the later one goes way beyond its predecessor. It actually provides what we always
wanted (as \CONTEXT\ developers) but never showed up. And this brings us to a next
topic.

\stopsection

\startsection[title=Usage]

Why is it that there has been little fundamental development around \TEX\
engines? One of the reasons is that macro packages have to be stable. New
features can be added but if they are only available in one engine (and there are
a few more around now, like \XETEX\ and the \CJK\ specific ones) a macro package
has to provide ways around them when they are not available. Risking some
criticism I dare to say that in order to use \LUATEX\ to its full potential,
macro package has to be set up such that this is possible and \CONTEXT\ does just
that. When we talk backends it's relatively easy, and when we talk fonts it's
doable. But if you are not willing to adapt the core of your code dramatically
(and conceptually) all you get from \LUATEX\ is a built-in scripting language and
some occasional messing around with node lists. In \CONTEXT\ we could transition
rather well because the user interfaces permitted to do so without users
noticing. Of course there were changes, for instance because encodings matter
less, which is also true for e.g.\ \XETEX, and font technologies changed. But for
macro packages other than \CONTEXT\ just the availability of \LUA\ might be
enough reasons to use that engine. That also means that documentation of the more
intricate features is less important: one can just learn by example and \CONTEXT\
is that example.

With \LUAMETATEX\ we go further because here one really has to make some
fundamental choices. Again this could be done within the existing user
interfaces, but here we are not only talking of fundamental improvements, like
rendering math or breaking paragraphs into lines, but also of more flexible
handling of alignments, inserts, adjusts, marks, par and page building, etc.\
Basically all mechanism got extended and opened up. In order to profit from this
you have to be able to throw away existing solution and use these extensions to
come up with better ones. If one can put sentiments aside, this also takes quite
some time.

A very important aspect (at least for me) is that I want the macro code to look
nice and in that respect stick to the \TEX\ syntax as much as possible. That
means that we have more programming related primitives, enhanced macro argument
parsing, more (flexible) conditionals, additional registers, extra expansion
related features and so on. Instead of some intermediate layer (like the helpers
in \CONTEXT) we can stick closer to the language itself. Of course this is not
something that most users will notice. What users might notice, is that on the
average \CONTEXT\ with \LUAMETATEX\ performs better than with \LUATEX\ or even
\LUAMETATEX. Even with more performance critical components delegated to \LUA\
(like the backend \PDF\ generation) we gain and often can compete performance
wise well with the faster eight bit \PDFTEX\ engine.

The fact that one has to make (and cannot make) drastic choices has a consequence
for documentation. Most of what is new and interesting is discussed in articles
and low level manuals. However, it is often discussed in the perspective of
\CONTEXT. Although we do discuss and show generic solutions it makes little sense
to go into details there simply because in the end only \CONTEXT\ will use them
as intended. It's just a waste of time to implement variants that are more
generic because they will never be used elsewhere, especially in situations where
the solutions are considered \quote {standard} and will not change. In \CONTEXT\
we always followed the principle that if we can do better, we will do better, and
interfaces are such that this can be done.

Of course that brings up the question \quotation {How do you know that these are
the best solutions} and the answer is that we don't. However, we're not talking
of quick and dirty solutions. For instance it took years to enhance math support:
experiments, discussion, reconsideration, documenting, writing articles, looking
at usage, fonts, etc. A wider discussion would not have brought better solutions,
if at all. If that were the case, there would already have been successors. The
same is true for most extensions: there was little need for them outside the
\CONTEXT\ community. So in the end that's what those interested should look at:
how is \LUAMETATEX\ used in \CONTEXT. It is the combined development together with
acceptance by users that makes this possible,

\stopsection

\stopdocument

