\enableexperiments[fonts.compact]
\enableexperiments[fonts.accurate]

\startenvironment luametatex-style

\usemodule[system-syntax]
\usemodule[system-units]
\usemodule[abbreviations-logos]
\usemodule[scite]

\definemeasure [document:margin]  [\paperheight/20]

\setuplayout
  [topspace=\measure{document:margin},
   bottomspace=\measure{document:margin},
   backspace=\measure{document:margin},
   header=\measure{document:margin},
   footer=0pt,
   width=middle,
   height=middle]

\setupbodyfont
  [dejavu,10pt] % 12pt is just to large and we use this for all kind of demos

\setupwhitespace
  [big]

\setuplist
  [interaction=all]

\setupalign
  [tolerant,stretch]

\setupdocument
  [title=No Title,
   subtitle={in luametatex},
   author=No Author,
   coauthor=,
   color=NoColor]

\definecolor[maincolor][darkblue]
\definecolor[primcolor][darkblue]
\definecolor[nonecolor][darkgray]

\startMPdefinitions
    vardef MyText (expr txt) =
        image (
            picture p ; p := lmt_outline [
                text   = txt,
                kind   = "path",
                anchor = "lft",
            ] ;
            for i within p :
                draw pathpart i
                    withpen pencircle scaled 2/5
                    withcolor "white" ;
                draw pathpart i
                    withpen pencircle scaled 1/3
                    withcolor "darkgray" ;
                drawcontrollines pathpart i
                    withpen pencircle scaled 1/3
                    withcolor "middlegray" ;
                drawcontrolpoints pathpart i
                    withpen pencircle scaled 1/3
                    withcolor "white" ;
                drawpoints pathpart i
                    withpen pencircle scaled 1/3
                    withcolor "darkyellow" ;
            endfor ;
        )
    enddef ;
\stopMPdefinitions

\startsetups document:contents

    \page

    \determinelistcharacteristics[chapter,section] [criterium=previous]

    \ifcase\listlength\else
        \startsubjectlevel[title=Contents]
            \placelist[chapter,section] [criterium=previous]
        \stopsubjectlevel
    \fi

\stopsetups

\startrawsetups document:start

    % raw keeps the lines

    \doifelse {\documentvariable{alternative}} {manual} {


        \setuppagenumbering
            [alternative=doublesided]

        \doifelse {\documentvariable{title}} {luametatex} {

            \startMPpage
                StartPage;
                    picture p[] ;
                    p[1] := image (
                      % draw MyText("LuaMeta\TEX") xsized (0.8PaperWidth) ;
                        draw MyText("luametatex") xsized (0.8PaperWidth) ;
                    ) ;
                    draw p[1] shifted (.1PaperWidth,.8PaperHeight) ;
                    p[2] := image (
                        draw MyText("the manual") ysized 15mm ;
                    ) ;
                    draw p[2] shifted urcorner Page shifted (-bbwidth p[2] - .1PaperWidth, - bbheight(p[1]) - bbheight(p[2]) - .2PaperHeight);
                    p[3] := image (
                        draw MyText("version \luametatexverboseversion") ysized 10mm ;
                    ) ;
%                     draw p[3] shifted lrcorner Page shifted (-bbwidth p[3] - .1PaperWidth, 15mm);
                    draw p[3] shifted lrcorner Page shifted (-bbwidth p[3] - .1PaperWidth, 25mm);
                    p[4] := image (
                        draw MyText("dev id \luametatexfunctionality") ysized 5mm ;
                    ) ;
%                     draw p[4] shifted lrcorner Page shifted (-bbwidth p[4] - .1PaperWidth, 5mm);
                    draw p[4] shifted lrcorner Page shifted (-bbwidth p[4] - .1PaperWidth, 15mm);
                    addbackground withcolor "darkblue" ;
                StopPage;
            \stopMPpage

            \page[right]

            \setups[document:contents]

        } {

            \startMPpage
                StartPage;
                    picture p[] ;
                    p[3] := image (
                        draw MyText("\word{\documentvariable{title}}") scaled 6 ;
                    ) ;
                    draw p[3] shifted lrcorner Page shifted (-bbwidth p[3] - .1PaperWidth, 15mm);
                    addbackground withcolor "darkblue" ;
                StopPage;
            \stopMPpage

            \page[right]

            \normalexpanded{\startchapter[title={\documentvariable{title}}]}
        }

    } {

        \startMPpage
            fill Page
                withcolor "darkgray" ;
%             draw textext("\doifelsedocumentvariable{xtitle}{\sstf \enspace \documentvariable{xtitle}}{\word{\sstf {\white the} \documentvariable{title}}}")
            draw textext("\sstf\enspace\doifelsedocumentvariable{xtitle}{\documentvariable{xtitle}}{\word{\documentvariable{title}}}")
                xysized (.9bbwidth(Page),bbheight(Page)-2cm)
                xsized (.9bbwidth(Page))
                shifted center Page
                withcolor "maincolor" ;
            draw textext.ulft("\word{\sstf  \documentvariable{subtitle}}")
                xysized (.9bbwidth(Page)/3,(bbheight(Page)-2cm)/6)
                shifted center lrcorner Page
                shifted (-.1bbwidth(Page),.05bbwidth(Page))
                withcolor "white" ;
            setbounds currentpicture to Page ;
        \stopMPpage

        \page[right]

        \setups[document:contents]

    }

    \setupdocument
      [title=No Title,
       subtitle={in luametatex}]

\stoprawsetups

\startrawsetups document:stop

    \doif {\documentvariable{alternative}} {manual} {

        \doifnot {\documentvariable{title}} {luametatex} {

            \stopchapter

        }

    }

\stoprawsetups

\definehead
  [newprimitive]
  [subsubsubsection]

\definehead
  [oldprimitive]
  [subsubsubsection]

\setuphead
  [chapter,title]
  [style=\bfc,
   color=maincolor,
   headerstate=high,
   interaction=all]

\setuphead
  [section]
  [style=\bfb,
   color=maincolor]

\setuphead
  [subsection]
  [style=\bfa,
   color=maincolor]

\setuphead
  [subsubsection]
  [style=\bf,
   color=maincolor,
   after=]

\setuphead
  [Syntax]
  [numberwidth=fit,
   style=\tta\bf]

\pushoverloadmode

    \definehead
      [Syntax]
      [subsection]
      [style=\ttbfb,
       numberwidth=0pt,
       after=\blank\startpacked,
       aftersection=\stoppacked]

\popoverloadmode

\setupexternalfigures
  [location={default,global}]

\setupinteraction
  [state=start,
   color=,
   contrastcolor=,
   style=,
   contraststyle=]

% \setuptabulate
%   [blank={small,samepage},
%    margin=.25em,
%    headstyle=bold,
%    rulecolor=maincolor,
%    rulethickness=1pt,
%    foregroundcolor=white,
%    foregroundstyle=\ss\bfx\WORD,
%    backgroundcolor=maincolor]

\setuptabulate
  [%margin=.25em,
   headstyle=normal,
   headcolor=maincolor,
   rulecolor=maincolor,
   rulethickness=1pt]

%D These are mostly

\setuptyping
  [option=tex]

\setuptype
  [option=tex]

\setuphead
  [newprimitive]
  [color=maincolor]

\setuphead
  [oldprimitive]
  [%color=nonecolor,
   color=\ifcstok{\structureuservariable{obsolete}}{yes}darkred\else nonecolor\fi]

\setuplist
  [newprimitive]
  [textcolor=maincolor]

\setuplist
  [oldprimitive]
  [textcolor=nonecolor]

\setuplist
  [chapter]
  [color=maincolor,
   style=bold,
   before={\testpage[3]\blank}]

% We use the next one because we want to check what has been done. In a document
% like this using \type {\foo} makes more sense.

\protected\def\prm#1%
% {\doifmode{*bodypart}{\index{\tex{#1}}}\tex{#1}}
% {\ifmode{*bodypart}\index{\tex{#1!\string#1!}}\fi\tex{#1}}
  {%ctxlua{if string.find("#1","^U") then context.writestatus("!!!!!!","#1") end}%
   \ifmode{*bodypart}\index{\tex{#1}}\fi\tex{#1}}

\protected\def\stx#1%
  {\ctxlua{moduledata.engine.specification("#1")}} % only for checking

% This is why we need to tag bodymatter.

% for nodes and tokens and such

\definehead
  [enginesubject]
  [subsubsubject]
  [before=\testpage[4]\blank,
   after=\blank,
   style=bold]

\startluacode
 -- no_usage_target       = 0,
 -- direct_usage_target   = 1, /* only available for direct nodes */
 -- userdata_usage_target = 2, /* only available for userdata nodes */
 -- separate_usage_target = 3, /* dedicated versions for direct and userdata nodes */
 -- hybrid_usage_target   = 4, /* shared version for direct and userdata nodes */
 -- general_usage_target  = 5, /* no direct or userdata arguments */
    local getusage  = node.direct.getusage
    local hasusage  = node.direct.hasusage
    local nodecodes = nodes.nodecodes
    function moduledata.showrepertoire(name, target)
        local t = getusage(name, target) -- functionname -> id list
        if t == true then
            context("all nodes")
        elseif t and #t > 0 then
            for i=1,#t do
                t[i] = nodecodes[t[i]]
            end
            context("% t",t)
        else
            context("(yet) unknown")
        end
    end
    function moduledata.showusagepernode(name, target)
        local t = getusage(nodecodes[name], target) -- id : list per node
        if t and #t > 0 then
            table.sort(t)
            context("% t",t)
        else
            context("nothing (yet)")
        end
    end

    local report = logs.reporter("manual","todo")

    for name, value in table.sortedhash(node.direct, target) do
        if type(value) == "function" and not hasusage(name, target) then
            report("helper %s",name)
        end
    end

    local ignore = { id = true }

    function moduledata.showfieldsupport(name,what)
        local id = tonumber(name) or nodes.nodecodes[name]
        if id then
            node.setfielderror(1)
            local fields = node.fields(id)
            local n = node.new(id )
            local t = { }
            for k, v in table.sortedhash(fields) do
                if not ignore[v] then
                    node[what](n,v)
                    if node.getfielderror() ~= 0 then
                        t[#t+1] = v
                    end
                end
            end
            node.free(n)
            if next(t) then
                report("")
                report("%s %s : % t",name,what,t)
                report("")
                context("% t",t)
            else
                context("-")
            end
            node.setfielderror(0)
        end
    end
\stopluacode

\tolerant\protected\def\showenginerepertoire[#1]#*#:#2%
  {\startenginesubject[title={targets}]
     \ctxlua{moduledata.showrepertoire("#2")}
   \stopenginesubject}

\tolerant\protected\def\showenginesetget[#1]#*#:#2%
  {\startenginesubject[title={#2}]
     no get: \start \tttf \veryraggedright \ctxlua{moduledata.showfieldsupport("#2","getfield")} \stop \par
     no set: \start \tttf \veryraggedright \ctxlua{moduledata.showfieldsupport("#2","setfield")} \stop \par
   \stopenginesubject}

\enablemode[checkfield]

\tolerant\protected\def\showengineusagepernode[#1]#*#:#2%
  {\startenginesubject[title={direct helpers}]
     \start \tttf \veryraggedright
     \ctxlua{moduledata.showusagepernode("#2", 1)}
     \stop
   \stopenginesubject
   \startenginesubject[title={userdata helpers}]
     \start \tttf \veryraggedright
     \ctxlua{moduledata.showusagepernode("#2", 2)}
     \stop
   \stopenginesubject
   \startmode[checkfield]
   \startenginesubject[title={userdata helpers}]
     \startlines
     no get: \start \tttf \veryraggedright \ctxlua{moduledata.showfieldsupport("#2","getfield")} \stop \par
     no set: \start \tttf \veryraggedright \ctxlua{moduledata.showfieldsupport("#2","setfield")} \stop \par
     \stoplines
   \stopenginesubject
   \stopmode}

\tolerant\protected\def\showenginevalues[#1]#*#:#2%
  {\startenginesubject[title={#2}]
     \startrows[n=\ifparameter#1\or#1\else2\fi,before=\blank,after=\blank]
       \ctxlua{moduledata.engine.codes("get#2", 1)}
     \stoprows
   \stopenginesubject}

\tolerant\protected\def\showenginefields[#1]#*#:#2%
  {\startenginesubject[title=fields]
     \startrows[n=\ifparameter#1\or#1\else3\fi,before=\blank,after=\blank]
       \ctxlua{moduledata.node.codes("fields","#2",true)}
     \stoprows
   \stopenginesubject}

\tolerant\protected\def\showenginesubtypes[#1]#*#:#2%
  {\startenginesubject[title=subtypes]
     \startrows[n=\ifparameter#1\or#1\else3\fi,before=\blank,after=\blank]
       \ctxlua{moduledata.node.codes("subtypes","#2",true)}
     \stoprows
  \stopenginesubject}

\tolerant\protected\def\showengineprimitives[#1]%
  {%\startenginesubject[title=primitives]
     \startrows[n=\ifparameter#1\or#1\else2\fi,before=\blank,after=\blank]
     \startluacode
        local p = token.getprimitives()
        table.sort(p,function(a,b)
            if a[1] == b[1] then
                return a[2] < b[2]
            else
                return a[1] < b[1]
            end
        end)
        local skiplist = { [106] = true }
        local skipping = false
        context.starttabulate { "|r|r|l|" }
        for i=1,\letterhash p do
            local pi = p[i]
            local pc = pi[1]
            if skiplist[pc] then
                if not skipping then
                    context.NC() context(pc)
                    context.NC()
                    context.NC() context.tex("Umath...")
                    context.NC() context.NR()
                    skipping = true
                end
            else
                context.NC() context(pc)
                context.NC() context(pi[2])
                context.NC() context.tex(pi[3])
                context.NC() context.NR()
                skipping = false
            end
        end
        context.stoptabulate()
     \stopluacode
     \stoprows
  }%\stopenginesubject}

\tolerant\protected\def\showenginekeylist#1%
  {\startluacode
     local t = #1
     table.sort(t)
     context.typ(t[1])
     for i=2,\letterhash t do
       context(", ")
       context.typ(t[i])
     end
   \stopluacode}

% \setuptyping
%   [align=hangright,
%    keeptogether=yes,
%    keeptogether=paragraph,
%    numbering=]

\definerows[two]  [n=2,before=\blank,after=\blank]
\definerows[three][n=3,before=\blank,after=\blank]
\definerows[four] [n=4,before=\blank,after=\blank]

\definerows[tworows]  [define=yes,n=2,before=\blank,after=\blank]
\definerows[threerows][define=yes,n=3,before=\blank,after=\blank]
\definerows[fourrows] [define=yes,n=4,before=\blank,after=\blank]

\setuplinenumbering
  [margin=0em,
   style=\ttx,
   width=\margindistance,
   distance=-\margindistance]

\stopenvironment
