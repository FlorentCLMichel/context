% language=uk

\environment luametatex-style

\startcomponent luametatex-nodes

\startchapter[reference=nodes,title={Nodes}]

\startsection[title={\LUA\ node representation}][library=node]

\topicindex {nodes}

\libindex {fields}
\libindex {subtypes}
\libindex {values}

\TEX's nodes are represented in \LUA\ as userdata objects with a variable set of
fields or by a numeric identifier when requested. When you print a node userdata
object you will see these numbers. In the following syntax tables the type of
such a userdata object is represented as \syntax {<node>}.

\blank
\dontleavehmode {\bf The return values of \type {node.types} are:} \showtypes
\blank

In \ETEX\ the \prm {lastnodetype} primitive has been introduced. With this
primitive the valid range of numbers is still $[-1,15]$ and glyph nodes (formerly
known as char nodes) have number~0. That way macro packages can use the same
symbolic names as in traditional \ETEX. But you need to keep in mind that these
\ETEX\ node numbers are different from the real internal ones. When you set \prm
{internalcodesmode} to a non|-|zero value, the internal codes will be used in
the \ETEX\ introspection commands \prm {lastnodetype} and \prm {currentiftype}.

You can ask for a list of fields with \type {node.fields} and for valid subtypes
with \type {node.subtypes}. The \type {node.values} function reports some used
values. Valid arguments are \type {glue}, \type {style} and \type {math}. Keep in
mind that the setters normally expect a number, but this helper gives you a list
of what numbers matter. For practical reason the \type {pagestate} values are
also reported with this helper, but they are backend specific.

\def\ShowValues#1{
    \blank
    \dontleavehmode
    {\bf The return values of \type {node.values("#1")} are:}
    \showvalues{#1}
    \blank
}

\ShowValues{glue} \ShowValues{style} \ShowValues{math} \ShowValues{pagestate}

\stopsection

\startsection[title={Main text nodes}]

\topicindex {nodes+text}

These are the nodes that comprise actual typesetting commands. A few fields are
present in all nodes regardless of their type, these are:

\starttabulate[|l|l|p|]
\DB field          \BC type   \BC explanation \NC \NR
\TB
\NC \type{next}    \NC node   \NC the next node in a list, or nil \NC \NR
\NC \type{id}      \NC number \NC the node's type (\type {id}) number \NC \NR
\NC \type{subtype} \NC number \NC the node \type {subtype} identifier \NC \NR
\LL
\stoptabulate

The \type {subtype} is sometimes just a dummy entry because not all nodes
actually use the \type {subtype}, but this way you can be sure that all nodes
accept it as a valid field name, and that is often handy in node list traversal.
In the following tables \type {next} and \type {id} are not explicitly mentioned.

Besides these three fields, almost all nodes also have an \type {attr} field, and
there is a also a field called \type {prev}. That last field is always present,
but only initialized on explicit request: when the function \type {node.slide}
is called, it will set up the \type {prev} fields to be a backwards pointer in
the argument node list. By now most of \TEX's node processing makes sure that the
\type {prev} nodes are valid but there can be exceptions, especially when the
internal magic uses a leading \nod {temp} nodes to temporarily store a state.

The \LUAMETATEX\ engine provides a lot of freedom and it is up to the user to
make sure that the node lists remain sane. There are some safeguards but there
can be cases where the engine just quits out of frustration. And, of course you
can make the engine crash.

\startsubsection[title={\nod {hlist} and \nod {vlist} nodes}]

\topicindex {nodes+lists}
\topicindex {lists}

These lists share fields and subtypes although some subtypes can only occur in
horizontal lists while others are unique for vertical lists. The possible
fields are \showfields {hlist}.

\starttabulate[|l|l|p|]
\DB field              \BC type   \BC explanation \NC \NR
\TB
\NC \type{subtype}     \NC number \NC \showsubtypes{list} \NC \NR
\NC \type{attr}        \NC node   \NC list of attributes \NC \NR
\NC \type{width}       \NC number \NC the width of the box \NC \NR
\NC \type{height}      \NC number \NC the height of the box \NC \NR
\NC \type{depth}       \NC number \NC the depth of the box \NC \NR
\NC \type{direction}   \NC number \NC the direction of this box, see~\in [dirnodes] \NC \NR
\NC \type{shift}       \NC number \NC a displacement perpendicular to the character
                                      (hlist) or line (vlist) progression direction \NC \NR
\NC \type{glue_order}  \NC number \NC a number in the range $[0,4]$, indicating the
                                      glue order \NC \NR
\NC \type{glue_set}    \NC number \NC the calculated glue ratio \NC \NR
\NC \type{glue_sign}   \NC number \NC 0 = \type {normal}, 1 = \type {stretching}, 2 =
                                      \type {shrinking} \NC \NR
\NC \type{list}        \NC node   \NC the first node of the body of this list \NC \NR
\LL
\stoptabulate

The \type {orientation}, \type {woffset}, \type {hoffset}, \type {doffset},
\type {xoffset} and \type {yoffset} fields are special. They can be used to make
the backend rotate and shift boxes which can be handy in for instance vertical
typesetting. Because they relate to (and depend on the) the backend they are not
discussed here (yet).

A warning: never assign a node list to the \type {list} field unless you are sure
its internal link structure is correct, otherwise an error may result.

Note: the field name \type {head} and \type {list} are both valid. Sometimes it
makes more sense to refer to a list by \type {head}, sometimes \type {list} makes
more sense.

\stopsubsection

\startsubsection[title={\nod {rule} nodes}]

\topicindex {nodes+rules}
\topicindex {rules}

Contrary to traditional \TEX, \LUATEX\ has more \prm {rule} subtypes because we
also use rules to store reuseable objects and images. User nodes are invisible
and can be intercepted by a callback. The supported fields are \showfields {rule}.

\starttabulate[|l|l|p|]
\DB field            \BC type   \BC explanation \NC \NR
\TB
\NC \type{subtype}   \NC number \NC \showsubtypes {rule} \NC \NR
\NC \type{attr}      \NC node   \NC list of attributes \NC \NR
\NC \type{width}     \NC number \NC the width of the rule where the special value
                                    $-1073741824$ is used for \quote {running} glue dimensions \NC \NR
\NC \type{height}    \NC number \NC the height of the rule (can be negative) \NC \NR
\NC \type{depth}     \NC number \NC the depth of the rule (can be negative) \NC \NR
\NC \type{left}      \NC number \NC shift at the left end (also subtracted from width) \NC \NR
\NC \type{right}     \NC number \NC (subtracted from width) \NC \NR
\NC \type{dir}       \NC string \NC the direction of this rule, see~\in[dirnodes] \NC \NR
\NC \type{index}     \NC number \NC an optional index that can be referred to \NC \NR
\NC \type{transform} \NC number \NC an private variable (also used to specify outline width) \NC \NR
\LL
\stoptabulate

The \type {left} and type {right} keys are somewhat special (and experimental).
When rules are auto adapting to the surrounding box width you can enforce a shift
to the right by setting \type {left}. The value is also subtracted from the width
which can be a value set by the engine itself and is not entirely under user
control. The \type {right} is also subtracted from the width. It all happens in
the backend so these are not affecting the calculations in the frontend (actually
the auto settings also happen in the backend). For a vertical rule \type {left}
affects the height and \type {right} affects the depth. There is no matching
interface at the \TEX\ end (although we can have more keywords for rules it would
complicate matters and introduce a speed penalty.) However, you can just
construct a rule node with \LUA\ and write it to the \TEX\ input. The \type
{outline} subtype is just a convenient variant and the \type {transform} field
specifies the width of the outline.

The \type {xoffset} and \type {yoffset} fields are special. They can be used to
shift rules. Because they relate to (and depend on the) the backend they are not
discussed here (yet).

\stopsubsection

\startsubsection[title={\nod {ins} nodes}]

\topicindex {nodes+insertions}
\topicindex {insertions}

This node relates to the \prm {insert} primitive and support the fields: \showfields{ins}.

\starttabulate[|l|l|p|]
\DB field          \BC type   \BC explanation \NC \NR
\TB
\NC \type{subtype} \NC number \NC the insertion class \NC \NR
\NC \type{attr}    \NC node   \NC list of attributes \NC \NR
\NC \type{cost}    \NC number \NC the penalty associated with this insert \NC \NR
\NC \type{height}  \NC number \NC height of the insert \NC \NR
\NC \type{depth}   \NC number \NC depth of the insert \NC \NR
\NC \type{list}    \NC node   \NC the first node of the body of this insert \NC \NR
\LL
\stoptabulate

There is a set of extra fields that concern the associated glue: \type {width},
\type {stretch}, \type {stretch_order}, \type {shrink} and \type {shrink_order}.
These are all numbers.

A warning: never assign a node list to the \type {head} field unless you are sure
its internal link structure is correct, otherwise an error may result. You can use
\type {list} instead (often in functions you want to use local variable with similar
names and both names are equally sensible).

\stopsubsection

\startsubsection[title={\nod {mark} nodes}]

\topicindex {nodes+marks}
\topicindex {marks}

This one relates to the \prm {mark} primitive and only has a few fields:
\showfields {mark}.

\starttabulate[|l|l|p|]
\DB field          \BC type   \BC explanation \NC \NR
\TB
\NC \type{subtype} \NC number \NC unused \NC \NR
\NC \type{attr}    \NC node   \NC list of attributes \NC \NR
\NC \type{class}   \NC number \NC the mark class \NC \NR
\NC \type{mark}    \NC table  \NC a table representing a token list \NC \NR
\LL
\stoptabulate

\stopsubsection

\startsubsection[title={\nod {adjust} nodes}]

\topicindex {nodes+adjust}
\topicindex {adjust}

This node comes from \prm {vadjust} primitive and has fields: \showfields {adjust}.

\starttabulate[|l|l|p|]
\DB field          \BC type   \BC explanation \NC \NR
\TB
\NC \type{subtype} \NC number \NC \showsubtypes{adjust} \NC \NR
\NC \type{attr}    \NC node   \NC list of attributes \NC \NR
\NC \type{list}    \NC node   \NC adjusted material \NC \NR
\LL
\stoptabulate

A warning: never assign a node list to the \type {head} field unless you are sure
its internal link structure is correct, otherwise an error may be the result.

\stopsubsection

\startsubsection[title={\nod {disc} nodes}]

\topicindex {nodes+discretionaries}
\topicindex {discretionaries}

The \prm {discretionary} and \prm {-}, the \type {-} character but also the
hyphenation mechanism produces these nodes. The available fields are: \showfields
{disc}.

\starttabulate[|l|l|p|]
\DB field          \BC type   \BC explanation \NC \NR
\TB
\NC \type{subtype} \NC number \NC \showsubtypes{disc} \NC \NR
\NC \type{attr}    \NC node   \NC list of attributes \NC \NR
\NC \type{pre}     \NC node   \NC pointer to the pre|-|break text \NC \NR
\NC \type{post}    \NC node   \NC pointer to the post|-|break text \NC \NR
\NC \type{replace} \NC node   \NC pointer to the no|-|break text \NC \NR
\NC \type{penalty} \NC number \NC the penalty associated with the break, normally
                                  \prm {hyphenpenalty} or \prm {exhyphenpenalty} \NC \NR
\LL
\stoptabulate

The subtype numbers~4 and~5 belong to the \quote {of-f-ice} explanation given
elsewhere. These disc nodes are kind of special as at some point they also keep
information about breakpoints and nested ligatures.

The \type {pre}, \type {post} and \type {replace} fields at the \LUA\ end are in
fact indirectly accessed and have a \type {prev} pointer that is not \type {nil}.
This means that when you mess around with the head of these (three) lists, you
also need to reassign them because that will restore the proper \type {prev}
pointer, so:

\starttyping
pre = d.pre
-- change the list starting with pre
d.pre = pre
\stoptyping

Otherwise you can end up with an invalid internal perception of reality and
\LUAMETATEX\ might even decide to crash on you. It also means that running forward
over for instance \type {pre} is ok but backward you need to stop at \type {pre}.
And you definitely must not mess with the node that \type {prev} points to, if
only because it is not really a node but part of the disc data structure (so
freeing it again might crash \LUAMETATEX).

\stopsubsection

\startsubsection[title={\nod {math} nodes}]

\topicindex {nodes+math}
\topicindex {math+nodes}

Math nodes represent the boundaries of a math formula, normally wrapped into
\type {$} signs. The following fields are available: \showfields {math}.

\starttabulate[|l|l|p|]
\DB field                \BC type   \BC explanation \NC \NR
\TB
\NC \type{subtype}       \NC number \NC \showsubtypes{math} \NC \NR
\NC \type{attr}          \NC node   \NC list of attributes \NC \NR
\NC \type{surround}      \NC number \NC width of the \prm {mathsurround} kern \NC \NR
\NC \type{width}         \NC number \NC the horizontal or vertical displacement \NC \NR
\NC \type{stretch}       \NC number \NC extra (positive) displacement or stretch amount \NC \NR
\NC \type{stretch_order} \NC number \NC factor applied to stretch amount \NC \NR
\NC \type{shrink}        \NC number \NC extra (negative) displacement or shrink amount\NC \NR
\NC \type{shrink_order}  \NC number \NC factor applied to shrink amount \NC \NR
\LL
\stoptabulate

The glue fields only kick in when the \type {surround} fields is zero.

\stopsubsection

\startsubsection[title={\nod {glue} nodes}]

\topicindex {nodes+glue}
\topicindex {glue}

Skips are about the only type of data objects in traditional \TEX\ that are not a
simple value. They are inserted when \TEX\ sees a space in the text flow but also
by \prm {hskip} and \prm {vskip}. The structure that represents the glue
components of a skip internally is called a \nod {glue_spec}. In \LUAMETATEX\ we
don't use the spec itself but just its values. A glue node has the fields:
\showfields {glue}.

\starttabulate[|l|l|pA{flushleft,tolerant}|]
\DB field                \BC type   \BC explanation \NC \NR
\TB
\NC \type{subtype}       \NC number \NC \showsubtypes{glue} \NC \NR
\NC \type{attr}          \NC node   \NC list of attributes \NC \NR
\NC \type{leader}        \NC node   \NC pointer to a box or rule for leaders \NC \NR
\NC \type{width}         \NC number \NC the horizontal or vertical displacement \NC \NR
\NC \type{stretch}       \NC number \NC extra (positive) displacement or stretch amount \NC \NR
\NC \type{stretch_order} \NC number \NC factor applied to stretch amount \NC \NR
\NC \type{shrink}        \NC number \NC extra (negative) displacement or shrink amount\NC \NR
\NC \type{shrink_order}  \NC number \NC factor applied to shrink amount \NC \NR
\LL
\stoptabulate

Note that we use the key \type {width} in both horizontal and vertical glue. This
suits the \TEX\ internals well so we decided to stick to that naming.

The effective width of some glue subtypes depends on the stretch or shrink needed
to make the encapsulating box fit its dimensions. For instance, in a paragraph
lines normally have glue representing spaces and these stretch or shrink to make
the content fit in the available space. The \type {effective_glue} function that
takes a glue node and a parent (hlist or vlist) returns the effective width of
that glue item. When you pass \type {true} as third argument the value will be
rounded.

\stopsubsection

\startsubsection[title={\nod {glue_spec} nodes}]

\topicindex {nodes+glue}
\topicindex {gluespec}

Internally \LUAMETATEX\ (like its ancestors) also uses nodes to store data that
is not seen in node lists. For instance the state of expression scanning (\type
{\dimexpr} etc.) and conditionals (\type {\ifcase} etc.) is also kept in lists of
nodes. A glue, which has five components, is stored in a node as well, so, where
most registers store just a number, a skip register (of internal quantity) uses a
pointer to a glue spec node. It has similar fields as glue nodes: \showfields
{glue_spec}, which is not surprising because in the past (and other engines than
\LUATEX) a glue node also has its values stored in a glue spec. This has some
advantages because often the values are the same, so for instance spacing related
skips were not resolved immediately but pointed to the current value of a space
related internal register (like \type {\spaceskip}). But, in \LUATEX\ we do
resolve these quantities immediately and we put the current values in the glue
nodes.

\starttabulate[|l|l|pA{flushleft,tolerant}|]
\DB field                \BC type   \BC explanation \NC \NR
\TB
\NC \type{width}         \NC number \NC the horizontal or vertical displacement \NC \NR
\NC \type{stretch}       \NC number \NC extra (positive) displacement or stretch amount \NC \NR
\NC \type{stretch_order} \NC number \NC factor applied to stretch amount \NC \NR
\NC \type{shrink}        \NC number \NC extra (negative) displacement or shrink amount\NC \NR
\NC \type{shrink_order}  \NC number \NC factor applied to shrink amount \NC \NR
\LL
\stoptabulate

You will only find these nodes in a few places, for instance when you query an
internal quantity. In principle we could do without them as we have interfaces
that use the five numbers instead. For compatibility reasons we keep glue spec
nodes exposed but this might change in the future.

\stopsubsection

\startsubsection[title={\nod {kern} nodes}]

\topicindex {nodes+kerns}
\topicindex {kerns}

The \prm {kern} command creates such nodes but for instance the font and math
machinery can also add them. There are not that many fields: \showfields {kern}.

\starttabulate[|l|l|p|]
\DB field                   \BC type   \BC explanation \NC \NR
\TB
\NC \type{subtype}          \NC number \NC \showsubtypes{kern} \NC \NR
\NC \type{attr}             \NC node   \NC list of attributes \NC \NR
\NC \type{kern}             \NC number \NC fixed horizontal or vertical advance \NC \NR
\NC \type{expansion_factor} \NC number \NC multiplier related to hz for font kerns \NC \NR
\LL
\stoptabulate

\stopsubsection

\startsubsection[title={\nod {penalty} nodes}]

\topicindex {nodes+penalty}
\topicindex {penalty}

The \prm {penalty} command is one that generates these nodes. It is one of the
type of nodes often found in vertical lists. It has the fields: \showfields
{penalty}.

\starttabulate[|l|l|p|]
\DB field          \BC type   \BC explanation \NC \NR
\TB
\NC \type{subtype} \NC number \NC \showsubtypes{penalty} \NC \NR
\NC \type{attr}    \NC node   \NC list of attributes \NC \NR
\NC \type{penalty} \NC number \NC the penalty value \NC \NR
\LL
\stoptabulate

The subtypes are just informative and \TEX\ itself doesn't use them. When you
run into an \type {linebreakpenalty} you need to keep in mind that it's a
accumulation of \type {club}, \type{widow} and other relevant penalties.

\stopsubsection

\startsubsection[title={\nod {glyph} nodes},reference=glyphnodes]

\topicindex {nodes+glyph}
\topicindex {glyphs}

These are probably the mostly used nodes and although you can push them in the
current list with for instance \prm {char} \TEX\ will normally do it for you when
it considers some input to be text. Glyph nodes are relatively large and have many
fields: \showfields {glyph}.

\starttabulate[|l|l|p|]
\DB field                   \BC type    \BC explanation \NC \NR
\TB
\NC \type{subtype}          \NC number  \NC bit field \NC \NR
\NC \type{attr}             \NC node    \NC list of attributes \NC \NR
\NC \type{char}             \NC number  \NC the character index in the font \NC \NR
\NC \type{font}             \NC number  \NC the font identifier \NC \NR
\NC \type{lang}             \NC number  \NC the language identifier \NC \NR
\NC \type{left}             \NC number  \NC the frozen \type {\lefthyphenmnin} value \NC \NR
\NC \type{right}            \NC number  \NC the frozen \type {\righthyphenmnin} value \NC \NR
\NC \type{uchyph}           \NC boolean \NC the frozen \prm {uchyph} value \NC \NR
\NC \type{components}       \NC node    \NC pointer to ligature components \NC \NR
\NC \type{xoffset}          \NC number  \NC a virtual displacement in horizontal direction \NC \NR
\NC \type{yoffset}          \NC number  \NC a virtual displacement in vertical direction \NC \NR
\NC \type{width}            \NC number  \NC the (original) width of the character \NC \NR
\NC \type{height}           \NC number  \NC the (original) height of the character\NC \NR
\NC \type{depth}            \NC number  \NC the (original) depth of the character\NC \NR
\NC \type{expansion_factor} \NC number  \NC the to be applied expansion_factor \NC \NR
\NC \type{data}             \NC number  \NC a general purpose field for users (we had room for it) \NC \NR
\LL
\stoptabulate

The \type {width}, \type {height} and \type {depth} values are read|-|only. The
\type {expansion_factor} is assigned in the par builder and used in the backend.

A warning: never assign a node list to the components field unless you are sure
its internal link structure is correct, otherwise an error may be result. Valid
bits for the \type {subtype} field are:

\starttabulate[|c|l|]
\DB bit \BC meaning   \NC \NR
\TB
\NC 0   \NC character \NC \NR
\NC 1   \NC ligature  \NC \NR
\NC 2   \NC ghost     \NC \NR
\NC 3   \NC left      \NC \NR
\NC 4   \NC right     \NC \NR
\LL
\stoptabulate

The \type {expansion_factor} has been introduced as part of the separation
between front- and backend. It is the result of extensive experiments with a more
efficient implementation of expansion. Early versions of \LUATEX\ already
replaced multiple instances of fonts in the backend by scaling but contrary to
\PDFTEX\ in \LUATEX\ we now also got rid of font copies in the frontend and
replaced them by expansion factors that travel with glyph nodes. Apart from a
cleaner approach this is also a step towards a better separation between front-
and backend.

The \type {is_char} function checks if a node is a glyph node with a subtype still
less than 256. This function can be used to determine if applying font logic to a
glyph node makes sense. The value \type {nil} gets returned when the node is not
a glyph, a character number is returned if the node is still tagged as character
and \type {false} gets returned otherwise. When nil is returned, the id is also
returned. The \type {is_glyph} variant doesn't check for a subtype being less
than 256, so it returns either the character value or nil plus the id. These
helpers are not always faster than separate calls but they sometimes permit
making more readable tests. The \type {uses_font} helpers takes a node
and font id and returns true when a glyph or disc node references that font.

\stopsubsection

\startsubsection[title={\nod {boundary} nodes}]

\topicindex {nodes+boundary}
\topicindex {boundary}

This node relates to the \prm {noboundary}, \prm {boundary}, \prm
{protrusionboundary} and \prm {wordboundary} primitives. These are small
nodes: \showfields {boundary} are the only fields.

\starttabulate[|l|l|p|]
\DB field          \BC type   \BC explanation \NC \NR
\TB
\NC \type{subtype} \NC number \NC \showsubtypes{boundary} \NC \NR
\NC \type{attr}    \NC node   \NC list of attributes \NC \NR
\NC \type{value}   \NC number \NC values 0--255 are reserved \NC \NR
\LL
\stoptabulate

\stopsubsection

\startsubsection[title={\nod {local_par} nodes}]

\topicindex {nodes+paragraphs}
\topicindex {paragraphs}

This node is inserted at the start of a paragraph. You should not mess
too much with this one. Valid fields are: \showfields {local_par}.

\starttabulate[|l|l|p|]
\DB field                  \BC type   \BC explanation \NC \NR
\TB
\NC \type{attr}            \NC node   \NC list of attributes \NC \NR
\NC \type{pen_inter}       \NC number \NC local interline penalty (from \lpr {localinterlinepenalty}) \NC \NR
\NC \type{pen_broken}      \NC number \NC local broken penalty (from \lpr {localbrokenpenalty}) \NC \NR
\NC \type{dir}             \NC string \NC the direction of this par. see~\in [dirnodes] \NC \NR
\NC \type{box_left}        \NC node   \NC the \lpr {localleftbox} \NC \NR
\NC \type{box_left_width}  \NC number \NC width of the \lpr {localleftbox} \NC \NR
\NC \type{box_right}       \NC node   \NC the \lpr {localrightbox} \NC \NR
\NC \type{box_right_width} \NC number \NC width of the \lpr {localrightbox} \NC \NR
\LL
\stoptabulate

A warning: never assign a node list to the \type {box_left} or \type {box_right}
field unless you are sure its internal link structure is correct, otherwise an
error may result.

\stopsubsection

\startsubsection[title={\nod {dir} nodes},reference=dirnodes]

\topicindex {nodes+direction}
\topicindex {directions}

Direction nodes mark parts of the running text that need a change of direction
and the \prm {textdir} command generates them. Again this is a small node, we
just have \showfields {dir}.

\starttabulate[|l|l|p|]
\DB field          \BC type   \BC explanation \NC \NR
\TB
\NC \type{subtype} \NC number \NC \showsubtypes{dir} \NC \NR
\NC \type{attr}    \NC node   \NC list of attributes \NC \NR
\NC \type{dir}     \NC string \NC the direction (\type {0} = l2r, \type {1} = r2l) \NC \NR
\NC \type{level}   \NC number \NC nesting level of this direction \NC \NR
\LL
\stoptabulate

There are only two directions: left|-|to|-|right (\type {0}) and
right|-|to|-|left (\type {1}). This is different from \LUATEX\ that has four
directions.

\stopsubsection

\startsubsection[title={\nod {marginkern} nodes}]

\topicindex {nodes+paragraphs}
\topicindex {paragraphs}
\topicindex {protrusion}

Margin kerns result from protrusion and have: \showfields {margin_kern}.

\starttabulate[|l|l|p|]
\DB field          \BC type   \BC explanation \NC \NR
\TB
\NC \type{subtype} \NC number \NC \showsubtypes{marginkern} \NC \NR
\NC \type{attr}    \NC node   \NC list of attributes \NC \NR
\NC \type{width}   \NC number \NC the advance of the kern \NC \NR
\NC \type{glyph}   \NC node   \NC the glyph to be used \NC \NR
\LL
\stoptabulate

\stopsubsection

\startsubsection[title={Whatsits}]

A whatsit node is a real simple one and it only has a subtype. It is even less
than a user node (which it actually could be) and uses hardly any memory. What
you do with it it entirely up to you: it's is real minimalistic. You can assign a
subtype and it has attributes. It is all up to the user how they are handled.

\stopsubsection

\startsubsection[title={Math noads}]

\topicindex {nodes+math}
\topicindex {math+nodes}

\startsubsubsection[title=The concept]

These are the so||called \quote {noad}s and the nodes that are specifically
associated with math processing. When you enter a formula, \TEX\ creates a node
list with regular nodes and noads. Then it hands over the list the math
processing engine. The result of that is a nodelist without noads. Most of the
noads contain subnodes so that the list of possible fields is actually quite
small. Math formulas are both a linked list and a tree. For instance in $e =
mc^2$ there is a linked list \type {e = m c} but the \type {c} has a superscript
branch that itself can be a list with branches.

First, there are the objects (the \TEX book calls them \quote {atoms}) that are
associated with the simple math objects: ord, op, bin, rel, open, close, punct,
inner, over, under, vcenter. These all have the same fields, and they are combined
into a single node type with separate subtypes for differentiation: \showfields
{noad}.

Many object fields in math mode are either simple characters in a specific family
or math lists or node lists: \type {math_char}, \type {math_text_char}, {sub_box}
and \type {sub_mlist} and \type {delim}. These are endpoints and therefore the
\type {next} and \type {prev} fields of these these subnodes are unused.

Some of the more elaborate noads have an option field. The values in this bitset
are common:

\starttabulate[|l|r|]
\DB meaning        \BC bits                      \NC \NR
\TB
\NC set            \NC               \type{0x08} \NC \NR
\NC internal       \NC \type{0x00} + \type{0x08} \NC \NR
\NC internal       \NC \type{0x01} + \type{0x08} \NC \NR
\NC axis           \NC \type{0x02} + \type{0x08} \NC \NR
\NC no axis        \NC \type{0x04} + \type{0x08} \NC \NR
\NC exact          \NC \type{0x10} + \type{0x08} \NC \NR
\NC left           \NC \type{0x11} + \type{0x08} \NC \NR
\NC middle         \NC \type{0x12} + \type{0x08} \NC \NR
\NC right          \NC \type{0x14} + \type{0x08} \NC \NR
\NC no subscript   \NC \type{0x21} + \type{0x08} \NC \NR
\NC no superscript \NC \type{0x22} + \type{0x08} \NC \NR
\NC no script      \NC \type{0x23} + \type{0x08} \NC \NR
\LL
\stoptabulate

\stopsubsubsection

\startsubsubsection[title={\nod {math_char} and \nod {math_text_char} subnodes}]

These are the most common ones, as they represent characters, and they both have
the same fields: \showfields {math_char}.

\starttabulate[|l|l|p|]
\DB field       \BC type   \BC explanation \NC \NR
\TB
\NC \type{attr} \NC node   \NC list of attributes \NC \NR
\NC \type{char} \NC number \NC the character index \NC \NR
\NC \type{fam}  \NC number \NC the family number \NC \NR
\LL
\stoptabulate

The \nod {math_char} is the simplest subnode field, it contains the character and
family for a single glyph object. The family eventually resolves on a reference
to a font. The \nod {math_text_char} is a special case that you will not normally
encounter, it arises temporarily during math list conversion (its sole function
is to suppress a following italic correction).

\stopsubsubsection

\startsubsubsection[title={\nod {sub_box} and \nod {sub_mlist} subnodes}]

These two subnode types are used for subsidiary list items. For \nod {sub_box},
the \type {list} points to a \quote {normal} vbox or hbox. For \nod {sub_mlist},
the \type {list} points to a math list that is yet to be converted. Their fields
are: \showfields {sub_box}.

\starttabulate[|l|l|p|]
\DB field       \BC type \BC explanation \NC \NR
\TB
\NC \type{attr} \NC node \NC list of attributes \NC \NR
\NC \type{list} \NC node \NC list of nodes \NC \NR
\LL
\stoptabulate

A warning: never assign a node list to the \type {list} field unless you are sure
its internal link structure is correct, otherwise an error is triggered.

\stopsubsubsection

\startsubsubsection[title={\nod {delim} subnodes}]

There is a fifth subnode type that is used exclusively for delimiter fields. As
before, the \type {next} and \type {prev} fields are unused, but we do have:
\showfields {delim}.

\starttabulate[|l|l|p|]
\DB field             \BC type   \BC explanation \NC \NR
\TB
\NC \type{attr}       \NC node   \NC list of attributes \NC \NR
\NC \type{small_char} \NC number \NC character index of base character \NC \NR
\NC \type{small_fam}  \NC number \NC family number of base character \NC \NR
\NC \type{large_char} \NC number \NC character index of next larger character \NC \NR
\NC \type{large_fam}  \NC number \NC family number of next larger character \NC \NR
\LL
\stoptabulate

The fields \type {large_char} and \type {large_fam} can be zero, in that case the
font that is set for the \type {small_fam} is expected to provide the large
version as an extension to the \type {small_char}.

\stopsubsubsection

\startsubsubsection[title={simple \nod {noad} nodes}]

In these noads, the \type {nucleus}, \type {sub} and \type {sup} fields can
branch of. Its fields are: \showfields {noad}.

\starttabulate[|l|l|p|]
\DB field          \BC type        \BC explanation \NC \NR
\TB
\NC \type{subtype} \NC number      \NC \showsubtypes{noad} \NC \NR
\NC \type{attr}    \NC node        \NC list of attributes \NC \NR
\NC \type{nucleus} \NC kernel node \NC base \NC \NR
\NC \type{sub}     \NC kernel node \NC subscript \NC \NR
\NC \type{sup}     \NC kernel node \NC superscript \NC \NR
\NC \type{options} \NC number      \NC bitset of rendering options \NC \NR
\LL
\stoptabulate

\stopsubsubsection

\startsubsubsection[title={\nod {accent} nodes}]

Accent nodes deal with stuff on top or below a math constructs. They support:
\showfields {accent}.

\starttabulate[|l|l|p|]
\DB field             \BC type        \BC explanation \NC \NR
\TB
\NC \type{subtype}    \NC number      \NC \showsubtypes{accent} \NC \NR
\NC \type{nucleus}    \NC kernel node \NC base \NC \NR
\NC \type{sub}        \NC kernel node \NC subscript \NC \NR
\NC \type{sup}        \NC kernel node \NC superscript \NC \NR
\NC \type{accent}     \NC kernel node \NC top accent \NC \NR
\NC \type{bot_accent} \NC kernel node \NC bottom accent \NC \NR
\NC \type{fraction}   \NC number      \NC larger step criterium (divided by 1000) \NC \NR
\LL
\stoptabulate

\stopsubsubsection

\startsubsubsection[title={\nod {style} nodes}]

These nodes are signals to switch to another math style. They are quite simple:
\showfields {style}. Currently the subtype is actually used to store the style
but don't rely on that for the future. Fields are: \showfields {style}.

\starttabulate[|l|l|p|]
\DB field        \BC type   \BC explanation    \NC \NR
\TB
\NC \type{style} \NC string \NC contains the style \NC \NR
\LL
\stoptabulate

Valid styles are: \showvalues{style}.

\stopsubsubsection

\startsubsubsection[title={\nod {parameter} nodes}]

These nodes are used to (locally) set math parameters: \showfields {parameter}.
Fields are: \showfields {parameter}.

\starttabulate[|l|l|p|]
\DB field        \BC type   \BC explanation    \NC \NR
\TB
\NC \type{style} \NC string \NC contains the style \NC \NR
\NC \type{name}  \NC string \NC defines the parameter \NC \NR
\NC \type{value} \NC number \NC holds the value, in case of a muglue multiple \NC \NR
\LL
\stoptabulate

\stopsubsubsection

\startsubsubsection[title={\nod {choice} nodes}]

Of its fields \showfields {choice} most are lists. Warning: never assign a node
list unless you are sure its internal link structure is correct, otherwise an
error can occur.

\starttabulate[|l|l|p|]
\DB field               \BC type \BC explanation \NC \NR
\TB
\NC \type{attr}         \NC node \NC list of attributes \NC \NR
\NC \type{display}      \NC node \NC list of display size alternatives \NC \NR
\NC \type{text}         \NC node \NC list of text size alternatives \NC \NR
\NC \type{script}       \NC node \NC list of scriptsize alternatives \NC \NR
\NC \type{scriptscript} \NC node \NC list of scriptscriptsize alternatives \NC \NR
\LL
\stoptabulate

\stopsubsubsection

\startsubsubsection[title={\nod {radical} nodes}]

Radical nodes are the most complex as they deal with scripts as well as
constructed large symbols. Many fields: \showfields {radical}. Warning: never
assign a node list to the \type {nucleus}, \type {sub}, \type {sup}, \type
{left}, or \type {degree} field unless you are sure its internal link structure
is correct, otherwise an error can be triggered.

\starttabulate[|l|l|p|]
\DB field          \BC type           \BC explanation \NC \NR
\TB
\NC \type{subtype} \NC number         \NC \showsubtypes{radical} \NC \NR
\NC \type{attr}    \NC node           \NC list of attributes \NC \NR
\NC \type{nucleus} \NC kernel node    \NC base \NC \NR
\NC \type{sub}     \NC kernel node    \NC subscript \NC \NR
\NC \type{sup}     \NC kernel node    \NC superscript \NC \NR
\NC \type{left}    \NC delimiter node \NC \NC \NR
\NC \type{degree}  \NC kernel node    \NC only set by \lpr {Uroot} \NC \NR
\NC \type{width}   \NC number         \NC required width \NC \NR
\NC \type{options} \NC number         \NC bitset of rendering options \NC \NR
\LL
\stoptabulate

\stopsubsubsection

\startsubsubsection[title={\nod {fraction} nodes}]

Fraction nodes are also used for delimited cases, hence the \type {left} and
\type {right} fields among: \showfields {fraction}.

\starttabulate[|l|l|p|]
\DB field          \BC type           \BC explanation \NC \NR
\TB
\NC \type{attr}    \NC node           \NC list of attributes \NC \NR
\NC \type{width}   \NC number         \NC (optional) width of the fraction \NC \NR
\NC \type{num}     \NC kernel node    \NC numerator \NC \NR
\NC \type{denom}   \NC kernel node    \NC denominator \NC \NR
\NC \type{left}    \NC delimiter node \NC left side symbol \NC \NR
\NC \type{right}   \NC delimiter node \NC right side symbol \NC \NR
\NC \type{middle}  \NC delimiter node \NC middle symbol \NC \NR
\NC \type{options} \NC number         \NC bitset of rendering options \NC \NR
\LL
\stoptabulate

Warning: never assign a node list to the \type {num}, or \type {denom} field
unless you are sure its internal link structure is correct, otherwise an error
can result.

\stopsubsubsection

\startsubsubsection[title={\nod {fence} nodes}]

Fence nodes come in pairs but either one can be a dummy (this period driven empty
fence). Fields are: \showfields {fence}. Some of these fields are used by the
renderer and might get adapted in the process.

\starttabulate[|l|l|p|]
\DB field          \BC type           \BC explanation \NC \NR
\TB
\NC \type{subtype} \NC number         \NC \showsubtypes{fence} \NC \NR
\NC \type{attr}    \NC node           \NC list of attributes \NC \NR
\NC \type{delim}   \NC delimiter node \NC delimiter specification \NC \NR
\NC \type{italic}  \NC number         \NC italic correction \NC \NR
\NC \type{height}  \NC number         \NC required height \NC \NR
\NC \type{depth}   \NC number         \NC required depth \NC \NR
\NC \type{options} \NC number         \NC bitset of rendering options \NC \NR
\NC \type{class}   \NC number         \NC spacing related class \NC \NR
\LL
\stoptabulate

\stopsubsubsection

\stopsubsection

\stopsection

\startsection[title={The \type {node} library}][library=node]

\startsubsection[title={Introduction}]

The \type {node} library provides methods that facilitate dealing with (lists of)
nodes and their values. They allow you to create, alter, copy, delete, and insert
node, the core objects within the typesetter. Nodes are represented in \LUA\ as
userdata. The various parts within a node can be accessed using named fields.

Each node has at least the three fields \type {next}, \type {id}, and \type
{subtype}. The other available fields depend on the \type {id}.

\startitemize[intro]

\startitem
    The \type {next} field returns the userdata object for the next node in a
    linked list of nodes, or \type {nil}, if there is no next node.
\stopitem

\startitem
    The \type {id} indicates \TEX's \quote{node type}. The field \type {id} has a
    numeric value for efficiency reasons, but some of the library functions also
    accept a string value instead of \type {id}.
\stopitem

\startitem
    The \type {subtype} is another number. It often gives further information
    about a node of a particular \type {id}.
\stopitem

\stopitemize

% Support for \nod {unset} (alignment) nodes is partial: they can be queried and
% modified from \LUA\ code, but not created.

Nodes can be compared to each other, but: you are actually comparing indices into
the node memory. This means that equality tests can only be trusted under very
limited conditions. It will not work correctly in any situation where one of the
two nodes has been freed and|/|or reallocated: in that case, there will be false
positives. The general approach to a node related callback is as follows:

\startitemize

\startitem
    Assume that the node list that you get is okay and properly double linked.
    If for some reason the links are not right, you can apply \type {node.slide}
    to the list.
\stopitem

\startitem
    When you insert a node, make sure you use a previously removed one, a new one
    or a copy. Don't simply inject the same node twice.
\stopitem

\startitem
    When you remove a node, make sure that when this is permanent, you also free
    the node or list. When you free a node its components are checked and when
    they are nodes themselves they are also freed.
\stopitem

\startitem
    Although you can fool the system, normally you will trigger an error when you
    try to copy a nonexisting node, or free an already freed node. There is some
    overhead involved in this checking but the current compromise is acceptable.
\stopitem

\startitem
    When you're done, pass back (if needed) the result. It's your responsibility
    to make sure that the list is properly linked (you can play safe and again
    apply \type {node.slide}. In principle you can put nodes in a list that are
    not acceptable in the following up actions. Some nodes get ignored, others
    will trigger an error, and sometimes the engine will just crash.
\stopitem

\stopitemize

So, from the above it will be clear then memory management of nodes has to be
done explicitly by the user. Nodes are not \quote {seen} by the \LUA\ garbage
collector, so you have to call the node freeing functions yourself when you are
no longer in need of a node (list). Nodes form linked lists without reference
counting, so you have to be careful that when control returns back to \LUATEX\
itself, you have not deleted nodes that are still referenced from a \type {next}
pointer elsewhere, and that you did not create nodes that are referenced more
than once. Normally the setters and getters handle this for you.

A good example are discretionary nodes that themselves have three sublists.
Internally they use special pointers, but the user never sees them because when
you query them or set fields, this property is hidden and taken care of. You just
see a list. But, when you mess with these sub lists it is your responsibility
that it only contains nodes that are permitted in a discretionary.

There are statistics available with regards to the allocated node memory, which
can be handy for tracing. Normally the amount of used nodes is not that large.
Typesetting a page can involve thousands of them but most are freed when the page
has been shipped out. Compared to other programs, node memory usage is not that
excessive. So, if for some reason your application leaks nodes, if at the end of
your run you lost as few hundred it's not a real problem. In fact, if you created
boxes and made copies but not flushed them for good reason, your run will for
sure end with used nodes and the statistics will mention that. The same is true
for attributes and skips (glue spec nodes): keeping the current state involves
using nodes.

\stopsubsection

\startsubsection[title={Housekeeping}]

\startsubsubsection[title={\type {types}}]

\libindex {types}

This function returns an array that maps node id numbers to node type strings,
providing an overview of the possible top|-|level \type {id} types.

\startfunctioncall
<table> t = node.types()
\stopfunctioncall

When we issue this command, we get a table. The currently visible types are
\inlineluavalue { node.types() } where the numbers are the internal identifiers.
Only those nodes are reported that make sense to users so there can be gaps in
the range of numbers.

\stopsubsubsection

\startsubsubsection[title={\type {id} and \type {type}}]

\libindex{id}
\libindex{type}

This converts a single type name to its internal numeric representation.

\startfunctioncall
<number> id = node.id(<string> type)
\stopfunctioncall

The \type {node.id("glyph")} command returns the number \inlineluavalue { node.id
("glyph") } and \type {node.id("hlist")} returns \inlineluavalue { node.id
("hlist") } where the numbers don't relate to importance or some ordering; they
just appear in the order that is handy for the engine. Commands like this are
rather optimized so performance should be ok but you can of course always store
the id in a \LUA\ number.

The reverse operation is: \type {node.type} If the argument is a number, then the
next function converts an internal numeric representation to an external string
representation. Otherwise, it will return the string \type {node} if the object
represents a node, and \type {nil} otherwise.

\startfunctioncall
<string> type = node.type(<any> n)
\stopfunctioncall

The \type {node.type(4)} command returns the string \inlineluavalue { node.type
(4) } and \type {node.id(99)} returns \inlineluavalue { node.id (99) } because
there is no node with that id.

\stopsubsubsection

\startsubsubsection[title={\type {fields} and \type {has_field}}]

\libindex {fields}
\libindex {has_field}

This function returns an indexed table with valid field names for a particular
type of node.

\startfunctioncall
<table> t = node.fields(<number|string> id)
\stopfunctioncall

The function accepts a string or number, so \typ {node.fields ("glyph")} returns
\inlineluavalue { node.fields ("glyph") } and \typ {node.fields (12)} gives
\inlineluavalue { node.fields (12) }.

The \type {has_field} function returns a boolean that is only true if \type {n}
is actually a node, and it has the field.

\startfunctioncall
<boolean> t = node.has_field(<node> n, <string> field)
\stopfunctioncall

This function probably is not that useful but some nodes don't have a \type
{subtype}, \type {attr} or \type {prev} field and this is a way to test for that.

\stopsubsubsection

\startsubsubsection[title={\type {is_node}}]

\topicindex {nodes+functions}

\libindex {is_node}

\startfunctioncall
<boolean|integer> t = node.is_node(<any> item)
\stopfunctioncall

This function returns a number (the internal index of the node) if the argument
is a userdata object of type \type {<node>} and false when no node is passed.

\stopsubsubsection

\startsubsubsection[title={\type {new}}]

\libindex{new}

The \type {new} function creates a new node. All its fields are initialized to
either zero or \type {nil} except for \type {id} and \type {subtype}. Instead of
numbers you can also use strings (names). If you pass a second argument the
subtype will be set too.

\startfunctioncall
<node> n = node.new(<number|string> id)
<node> n = node.new(<number|string> id, <number|string> subtype)
\stopfunctioncall

As already has been mentioned, you are responsible for making sure that nodes
created this way are used only once, and are freed when you don't pass them
back somehow.

\stopsubsubsection

\startsubsubsection[title={\type {free}, \type {flush_node} and \type {flush_list}}]

\libindex{free}
\libindex{flush_node}
\libindex{flush_list}

The next one frees node \type {n} from \TEX's memory. Be careful: no checks are
done on whether this node is still pointed to from a register or some \type
{next} field: it is up to you to make sure that the internal data structures
remain correct. Fields that point to nodes or lists are flushed too. So, when
you used their content for something else you need to set them to nil first.

\startfunctioncall
<node> next = node.free(<node> n)
flush_node(<node> n)
\stopfunctioncall

The \type {free} function returns the next field of the freed node, while the
\type {flush_node} alternative returns nothing.

A list starting with node \type {n} can be flushed from \TEX's memory too. Be
careful: no checks are done on whether any of these nodes is still pointed to
from a register or some \type {next} field: it is up to you to make sure that the
internal data structures remain correct.

\startfunctioncall
node.flush_list(<node> n)
\stopfunctioncall

When you free for instance a discretionary node, \type {flush_list} is applied to
the \type {pre}, \type {post}, \type {replace} so you don't need to do that
yourself. Assigning them \type {nil} won't free those lists!

\stopsubsubsection

\startsubsubsection[title={\type {copy} and \type {copy_list}}]

\libindex{copy}
\libindex{copy_list}

This creates a deep copy of node \type {n}, including all nested lists as in the case
of a hlist or vlist node. Only the \type {next} field is not copied.

\startfunctioncall
<node> m = node.copy(<node> n)
\stopfunctioncall

A deep copy of the node list that starts at \type {n} can be created too. If
\type {m} is also given, the copy stops just before node \type {m}.

\startfunctioncall
<node> m = node.copy_list(<node> n)
<node> m = node.copy_list(<node> n, <node> m)
\stopfunctioncall

Note that you cannot copy attribute lists this way. However, there is normally no
need to copy attribute lists as when you do assignments to the \type {attr} field
or make changes to specific attributes, the needed copying and freeing takes
place automatically. When you change a value of an attribute {\em in} a list, it will
affect all the nodes that share that list.

\stopsubsubsection

\startsubsubsection[title={\type {write}}]

\libindex {write}

\startfunctioncall
node.write(<node> n)
\stopfunctioncall

This function will append a node list to \TEX's \quote {current list}. The node
list is not deep|-|copied! There is no error checking either! You might need to
enforce horizontal mode in order for this to work as expected.

\stopsubsubsection

\stopsubsection

\startsubsection[title={Manipulating lists}]

\startsubsubsection[title={\type {slide}}]

\libindex {slide}

This helper makes sure that the node list is double linked and returns the found
tail node.

\startfunctioncall
<node> tail = node.slide(<node> n)
\stopfunctioncall

After some callbacks automatic sliding takes place. This feature can be turned
off with \type {node.fix_node_lists(false)} but you better make sure then that
you don't mess up lists. In most cases \TEX\ itself only uses \type {next}
pointers but your other callbacks might expect proper \type {prev} pointers too.
Future versions of \LUATEX\ can add more checking but this will not influence
usage.

\stopsubsubsection

\startsubsubsection[title={\type {tail}}]

\libindex {tail}

\startfunctioncall
<node> m = node.tail(<node> n)
\stopfunctioncall

Returns the last node of the node list that starts at \type {n}.

\stopsubsubsection

\startsubsubsection[title={\type {length} and \type {count}}]

\libindex {length}
\libindex {count}

\startfunctioncall
<number> i = node.length(<node> n)
<number> i = node.length(<node> n, <node> m)
\stopfunctioncall

Returns the number of nodes contained in the node list that starts at \type {n}.
If \type {m} is also supplied it stops at \type {m} instead of at the end of the
list. The node \type {m} is not counted.

\startfunctioncall
<number> i = node.count(<number> id, <node> n)
<number> i = node.count(<number> id, <node> n, <node> m)
\stopfunctioncall

Returns the number of nodes contained in the node list that starts at \type {n}
that have a matching \type {id} field. If \type {m} is also supplied, counting
stops at \type {m} instead of at the end of the list. The node \type {m} is not
counted. This function also accept string \type {id}'s.

\stopsubsubsection

\startsubsubsection[title={\type {remove}}]

\libindex {remove}

\startfunctioncall
<node> head, current, removed =
    node.remove(<node> head, <node> current)
<node> head, current =
    node.remove(<node> head, <node> current, <boolean> true)
\stopfunctioncall

This function removes the node \type {current} from the list following \type
{head}. It is your responsibility to make sure it is really part of that list.
The return values are the new \type {head} and \type {current} nodes. The
returned \type {current} is the node following the \type {current} in the calling
argument, and is only passed back as a convenience (or \type {nil}, if there is
no such node). The returned \type {head} is more important, because if the
function is called with \type {current} equal to \type {head}, it will be
changed. When the third argument is passed, the node is freed.

\stopsubsubsection

\startsubsubsection[title={\type {insert_before}}]

\libindex {insert_before}

\startfunctioncall
<node> head, new = node.insert_before(<node> head, <node> current, <node> new)
\stopfunctioncall

This function inserts the node \type {new} before \type {current} into the list
following \type {head}. It is your responsibility to make sure that \type
{current} is really part of that list. The return values are the (potentially
mutated) \type {head} and the node \type {new}, set up to be part of the list
(with correct \type {next} field). If \type {head} is initially \type {nil}, it
will become \type {new}.

\stopsubsubsection

\startsubsubsection[title={\type {insert_after}}]

\libindex {insert_after}

\startfunctioncall
<node> head, new = node.insert_after(<node> head, <node> current, <node> new)
\stopfunctioncall

This function inserts the node \type {new} after \type {current} into the list
following \type {head}. It is your responsibility to make sure that \type
{current} is really part of that list. The return values are the \type {head} and
the node \type {new}, set up to be part of the list (with correct \type {next}
field). If \type {head} is initially \type {nil}, it will become \type {new}.

\stopsubsubsection

\startsubsubsection[title={\type {last_node}}]

\libindex {last_node}

\startfunctioncall
<node> n = node.last_node()
\stopfunctioncall

This function pops the last node from \TEX's \quote{current list}. It returns
that node, or \type {nil} if the current list is empty.

\stopsubsubsection

\startsubsubsection[title={\type {traverse}}]

\libindex {traverse}

\startfunctioncall
<node> t, id, subtype = node.traverse(<node> n)
\stopfunctioncall

This is a \LUA\ iterator that loops over the node list that starts at \type {n}.
Typically code looks like this:

\starttyping
for n in node.traverse(head) do
   ...
end
\stoptyping

is functionally equivalent to:

\starttyping
do
  local n
  local function f (head,var)
    local t
    if var == nil then
       t = head
    else
       t = var.next
    end
    return t
  end
  while true do
    n = f (head, n)
    if n == nil then break end
    ...
  end
end
\stoptyping

It should be clear from the definition of the function \type {f} that even though
it is possible to add or remove nodes from the node list while traversing, you
have to take great care to make sure all the \type {next} (and \type {prev})
pointers remain valid.

If the above is unclear to you, see the section \quote {For Statement} in the
\LUA\ Reference Manual.

\stopsubsubsection

\startsubsubsection[title={\type {traverse_id}}]

\libindex {traverse_id}

\startfunctioncall
<node> t, subtype = node.traverse_id(<number> id, <node> n)
\stopfunctioncall

This is an iterator that loops over all the nodes in the list that starts at
\type {n} that have a matching \type {id} field.

See the previous section for details. The change is in the local function \type
{f}, which now does an extra while loop checking against the upvalue \type {id}:

\starttyping
 local function f(head,var)
   local t
   if var == nil then
      t = head
   else
      t = var.next
   end
   while not t.id == id do
      t = t.next
   end
   return t
 end
\stoptyping

\stopsubsubsection

\startsubsubsection[title={\type {traverse_char} and \type {traverse_glyph}}]

\libindex {traverse_char}
\libindex {traverse_glyph}

The \type{traverse_char} iterator loops over the \nod {glyph} nodes in a list.
Only nodes with a subtype less than 256 are seen.

\startfunctioncall
<node> n, font, char = node.traverse_char(<node> n)
\stopfunctioncall

The \type{traverse_glyph} iterator loops over a list and returns the list and
filters all glyphs:

\startfunctioncall
<node> n, font, char = node.traverse_glyph(<node> n)
\stopfunctioncall

\stopsubsubsection

\startsubsubsection[title={\type {traverse_list}}]

\libindex {traverse_list}

This iterator loops over the \nod {hlist} and \nod {vlist} nodes in a list.

\startfunctioncall
<node> n, id, subtype, list = node.traverse_list(<node> n)
\stopfunctioncall

The four return values can save some time compared to fetching these fields but
in practice you seldom need them all. So consider it a (side effect of
experimental) convenience.

\stopsubsubsection

\startsubsubsection[title={\type {find_node}}]

\libindex {find_node}

This helper returns the location of the first match at or after node \type {n}:

\startfunctioncall
<node> n = node.find_node(<node> n, <integer> subtype)
<node> n, subtype = node.find_node(<node> n)
\stopfunctioncall

\stopsubsubsection


\stopsubsection

\startsubsection[title={Glue handling}][library=node]

\startsubsubsection[title={\type {setglue}}]

\libindex {setglue}

You can set the properties of a glue in one go. If you pass no values, the glue
will become a zero glue.

\startfunctioncall
node.setglue(<node> n)
node.setglue(<node> n,width,stretch,shrink,stretch_order,shrink_order)
\stopfunctioncall

When you pass values, only arguments that are numbers are assigned so

\starttyping
node.setglue(n,655360,false,65536)
\stoptyping

will only adapt the width and shrink.

When a list node is passed, you set the glue, order and sign instead.

\stopsubsubsection

\startsubsubsection[title={\type {getglue}}]

\libindex {getglue}

The next call will return 5 values or nothing when no glue is passed.

\startfunctioncall
<integer> width, <integer> stretch, <integer> shrink, <integer> stretch_order,
    <integer> shrink_order = node.getglue(<node> n)
\stopfunctioncall

When the second argument is false, only the width is returned (this is consistent
with \type {tex.get}).

When a list node is passed, you get back the glue that is set, the order of that
glue and the sign.

\stopsubsubsection

\startsubsubsection[title={\type {is_zero_glue}}]

\libindex {is_zero_glue}

This function returns \type {true} when the width, stretch and shrink properties
are zero.

\startfunctioncall
<boolean> isglue = node.is_zero_glue(<node> n)
\stopfunctioncall

\stopsubsubsection

\stopsubsection

\startsubsection[title={Attribute handling}][library=node]

\startsubsubsection[title={Attributes}]

\topicindex {attributes}

Assignments to attributes registers result in assigning lists with set attributes
to nodes and the implementation is non|-|trivial because the value that is
attached to a node is essentially a (sorted) sparse array of key|-|value pairs.
It is generally easiest to deal with attribute lists and attributes by using the
dedicated functions in the \type {node} library.

\stopsubsubsection

\startsubsubsection[title={\nod {attribute_list} nodes}]

\topicindex {nodes+attributes}

An \type {attribute_list} item is used as a head pointer for a list of attribute
items. It has only one user|-|visible field:

\starttabulate[|l|l|p|]
\DB field       \BC type \BC explanation \NC \NR
\TB
\NC \type{next} \NC node \NC pointer to the first attribute \NC \NR
\LL
\stoptabulate

\stopsubsubsection

\startsubsubsection[title={\nod {attr} nodes}]

A normal node's attribute field will point to an item of type \nod
{attribute_list}, and the \type {next} field in that item will point to the first
defined \quote {attribute} item, whose \type {next} will point to the second
\quote {attribute} item, etc.

\starttabulate[|l|l|p|]
\DB field         \BC type   \BC explanation \NC \NR
\TB
\NC \type{next}   \NC node   \NC pointer to the next attribute \NC \NR
\NC \type{number} \NC number \NC the attribute type id \NC \NR
\NC \type{value}  \NC number \NC the attribute value \NC \NR
\LL
\stoptabulate

As mentioned it's better to use the official helpers rather than edit these
fields directly. For instance the \type {prev} field is used for other purposes
and there is no double linked list.

\stopsubsubsection

\startsubsubsection[title={\type {current_attr}}]

\libindex{current_attr}

This returns the currently active list of attributes, if there is one.

\startfunctioncall
<node> m = node.current_attr()
\stopfunctioncall

The intended usage of \type {current_attr} is as follows:

\starttyping
local x1 = node.new("glyph")
x1.attr = node.current_attr()
local x2 = node.new("glyph")
x2.attr = node.current_attr()
\stoptyping

or:

\starttyping
local x1 = node.new("glyph")
local x2 = node.new("glyph")
local ca = node.current_attr()
x1.attr = ca
x2.attr = ca
\stoptyping

The attribute lists are ref counted and the assignment takes care of incrementing
the refcount. You cannot expect the value \type {ca} to be valid any more when
you assign attributes (using \type {tex.setattribute}) or when control has been
passed back to \TEX.

\stopsubsubsection

\startsubsubsection[title={\type {has_attribute}}]

\libindex {has_attribute}

\startfunctioncall
<number> v = node.has_attribute(<node> n, <number> id)
<number> v = node.has_attribute(<node> n, <number> id, <number> val)
\stopfunctioncall

Tests if a node has the attribute with number \type {id} set. If \type {val} is
also supplied, also tests if the value matches \type {val}. It returns the value,
or, if no match is found, \type {nil}.

\stopsubsubsection

\startsubsubsection[title={\type {get_attribute}}]

\libindex {get_attribute}

\startfunctioncall
<number> v = node.get_attribute(<node> n, <number> id)
\stopfunctioncall

Tests if a node has an attribute with number \type {id} set. It returns the
value, or, if no match is found, \type {nil}. If no \type {id} is given then the
zero attributes is assumed.

\stopsubsubsection

\startsubsubsection[title={\type {find_attribute}}]

\libindex {find_attribute}

\startfunctioncall
<number> v, <node> n = node.find_attribute(<node> n, <number> id)
\stopfunctioncall

Finds the first node that has attribute with number \type {id} set. It returns
the value and the node if there is a match and otherwise nothing.

\stopsubsubsection

\startsubsubsection[title={\type {set_attribute}}]

\libindex {set_attribute}

\startfunctioncall
node.set_attribute(<node> n, <number> id, <number> val)
\stopfunctioncall

Sets the attribute with number \type {id} to the value \type {val}. Duplicate
assignments are ignored.

\stopsubsubsection

\startsubsubsection[title={\type {unset_attribute}}]

\libindex {unset_attribute}

\startfunctioncall
<number> v =
    node.unset_attribute(<node> n, <number> id)
<number> v =
    node.unset_attribute(<node> n, <number> id, <number> val)
\stopfunctioncall

Unsets the attribute with number \type {id}. If \type {val} is also supplied, it
will only perform this operation if the value matches \type {val}. Missing
attributes or attribute|-|value pairs are ignored.

If the attribute was actually deleted, returns its old value. Otherwise, returns
\type {nil}.

\stopsubsubsection

\stopsubsection

\startsubsection[title={Glyph handling}][library=node]

\startsubsubsection[title={\type {first_glyph}}]

\libindex {first_glyph}

\startfunctioncall
<node> n = node.first_glyph(<node> n)
<node> n = node.first_glyph(<node> n, <node> m)
\stopfunctioncall

Returns the first node in the list starting at \type {n} that is a glyph node
with a subtype indicating it is a glyph, or \type {nil}. If \type {m} is given,
processing stops at (but including) that node, otherwise processing stops at the
end of the list.

\stopsubsubsection

\startsubsubsection[title={\type {is_char} and \type {is_glyph}}]

\libindex {is_char}
\libindex {is_glyph}

The subtype of a glyph node signals if the glyph is already turned into a character reference
or not.

\startfunctioncall
<boolean> b = node.is_char(<node> n)
<boolean> b = node.is_glyph(<node> n)
\stopfunctioncall

\stopsubsubsection

\startsubsubsection[title={\type {has_glyph}}]

\libindex {has_glyph}

This function returns the first glyph or disc node in the given list:

\startfunctioncall
<node> n = node.has_glyph(<node> n)
\stopfunctioncall

\stopsubsubsection

\startsubsubsection[title={\type {ligaturing}}]

\libindex {ligaturing}

\startfunctioncall
<node> h, <node> t, <boolean> success = node.ligaturing(<node> n)
<node> h, <node> t, <boolean> success = node.ligaturing(<node> n, <node> m)
\stopfunctioncall

Apply \TEX-style ligaturing to the specified nodelist. The tail node \type {m} is
optional. The two returned nodes \type {h} and \type {t} are the new head and
tail (both \type {n} and \type {m} can change into a new ligature).

\stopsubsubsection

\startsubsubsection[title={\type {kerning}}]

\libindex {kerning}

\startfunctioncall
<node> h, <node> t, <boolean> success = node.kerning(<node> n)
<node> h, <node> t, <boolean> success = node.kerning(<node> n, <node> m)
\stopfunctioncall

Apply \TEX|-|style kerning to the specified node list. The tail node \type {m} is
optional. The two returned nodes \type {h} and \type {t} are the head and tail
(either one of these can be an inserted kern node, because special kernings with
word boundaries are possible).

\stopsubsubsection

\startsubsubsection[title={\type {unprotect_glyph[s]}}]

\libindex {unprotect_glyphs}
\libindex {unprotect_glyph}

\startfunctioncall
node.unprotect_glyph(<node> n)
node.unprotect_glyphs(<node> n,[<node> n])
\stopfunctioncall

Subtracts 256 from all glyph node subtypes. This and the next function are
helpers to convert from \type {characters} to \type {glyphs} during node
processing. The second argument is optional and indicates the end of a range.

\stopsubsubsection

\startsubsubsection[title={\type {protect_glyph[s]}}]

\libindex {protect_glyphs}
\libindex {protect_glyph}

\startfunctioncall
node.protect_glyph(<node> n)
node.protect_glyphs(<node> n,[<node> n])
\stopfunctioncall

Adds 256 to all glyph node subtypes in the node list starting at \type {n},
except that if the value is 1, it adds only 255. The special handling of 1 means
that \type {characters} will become \type {glyphs} after subtraction of 256. A
single character can be marked by the singular call. The second argument is
optional and indicates the end of a range.

\stopsubsubsection

\startsubsubsection[title={\type {protrusion_skippable}}]

\libindex {protrusion_skippable}

\startfunctioncall
<boolean> skippable = node.protrusion_skippable(<node> n)
\stopfunctioncall

Returns \type {true} if, for the purpose of line boundary discovery when
character protrusion is active, this node can be skipped.

\stopsubsubsection

\startsubsubsection[title={\type {check_discretionary}, \type {check_discretionaries}}]

\libindex{check_discretionary}
\libindex{check_discretionaries}

When you fool around with disc nodes you need to be aware of the fact that they
have a special internal data structure. As long as you reassign the fields when
you have extended the lists it's ok because then the tail pointers get updated,
but when you add to list without reassigning you might end up in trouble when
the linebreak routine kicks in. You can call this function to check the list for
issues with disc nodes.

\startfunctioncall
node.check_discretionary(<node> n)
node.check_discretionaries(<node> head)
\stopfunctioncall

The plural variant runs over all disc nodes in a list, the singular variant
checks one node only (it also checks if the node is a disc node).

\stopsubsubsection

\startsubsubsection[title={\type {flatten_discretionaries}}]

\libindex {flatten_discretionaries}

This function will remove the discretionaries in the list and inject the replace
field when set.

\startfunctioncall
<node> head, count = node.flatten_discretionaries(<node> n)
\stopfunctioncall

\stopsubsubsection

\stopsubsection

\startsubsection[title={Packaging}][library=node]

\startsubsubsection[title={\type {hpack}}]

\libindex {hpack}

This function creates a new hlist by packaging the list that begins at node \type
{n} into a horizontal box. With only a single argument, this box is created using
the natural width of its components. In the three argument form, \type {info}
must be either \type {additional} or \type {exactly}, and \type {w} is the
additional (\type {\hbox spread}) or exact (\type {\hbox to}) width to be used.
The second return value is the badness of the generated box.

\startfunctioncall
<node> h, <number> b =
    node.hpack(<node> n)
<node> h, <number> b =
    node.hpack(<node> n, <number> w, <string> info)
<node> h, <number> b =
    node.hpack(<node> n, <number> w, <string> info, <string> dir)
\stopfunctioncall

Caveat: there can be unexpected side|-|effects to this function, like updating
some of the \prm {marks} and \type {\inserts}. Also note that the content of
\type {h} is the original node list \type {n}: if you call \type {node.free(h)}
you will also free the node list itself, unless you explicitly set the \type
{list} field to \type {nil} beforehand. And in a similar way, calling \type
{node.free(n)} will invalidate \type {h} as well!

\stopsubsubsection

\startsubsubsection[title={\type {vpack}}]

\libindex {vpack}

This function creates a new vlist by packaging the list that begins at node \type
{n} into a vertical box. With only a single argument, this box is created using
the natural height of its components. In the three argument form, \type {info}
must be either \type {additional} or \type {exactly}, and \type {w} is the
additional (\type {\vbox spread}) or exact (\type {\vbox to}) height to be used.

\startfunctioncall
<node> h, <number> b =
    node.vpack(<node> n)
<node> h, <number> b =
    node.vpack(<node> n, <number> w, <string> info)
<node> h, <number> b =
    node.vpack(<node> n, <number> w, <string> info, <string> dir)
\stopfunctioncall

The second return value is the badness of the generated box. See the description
of \type {hpack} for a few memory allocation caveats.

\stopsubsubsection

\startsubsubsection[title={\type {prepend_prevdepth}}]

\libindex {prepend_prevdepth}

This function is somewhat special in the sense that it is an experimental helper
that adds the interlinespace to a line keeping the baselineskip and lineskip into
account.

\startfunctioncall
<node> n, <number> delta =
    node.prepend_prevdepth(<node> n,<number> prevdepth)
\stopfunctioncall

\stopsubsubsection

\startsubsubsection[title={\type {dimensions}, \type {rangedimensions}, \type {naturalwidth}}]

\libindex{dimensions}
\libindex{rangedimensions}

\startfunctioncall
<number> w, <number> h, <number> d  =
    node.dimensions(<node> n)
<number> w, <number> h, <number> d  =
    node.dimensions(<node> n, <node> t)
\stopfunctioncall

This function calculates the natural in|-|line dimensions of the node list starting
at node \type {n} and terminating just before node \type {t} (or the end of the
list, if there is no second argument). The return values are scaled points. An
alternative format that starts with glue parameters as the first three arguments
is also possible:

\startfunctioncall
<number> w, <number> h, <number> d  =
    node.dimensions(<number> glue_set, <number> glue_sign, <number> glue_order,
        <node> n)
<number> w, <number> h, <number> d  =
    node.dimensions(<number> glue_set, <number> glue_sign, <number> glue_order,
        <node> n, <node> t)
\stopfunctioncall

This calling method takes glue settings into account and is especially useful for
finding the actual width of a sublist of nodes that are already boxed, for
example in code like this, which prints the width of the space in between the
\type {a} and \type {b} as it would be if \type {\box0} was used as-is:

\starttyping
\setbox0 = \hbox to 20pt {a b}

\directlua{print (node.dimensions(
    tex.box[0].glue_set,
    tex.box[0].glue_sign,
    tex.box[0].glue_order,
    tex.box[0].head.next,
    node.tail(tex.box[0].head)
)) }
\stoptyping

You need to keep in mind that this is one of the few places in \TEX\ where floats
are used, which means that you can get small differences in rounding when you
compare the width reported by \type {hpack} with \type {dimensions}.

The second alternative saves a few lookups and can be more convenient in some
cases:

\startfunctioncall
<number> w, <number> h, <number> d  =
    node.rangedimensions(<node> parent, <node> first)
<number> w, <number> h, <number> d  =
    node.rangedimensions(<node> parent, <node> first, <node> last)
\stopfunctioncall

A simple and somewhat more efficient variant is this:

\startfunctioncall
<number> w =
    node.naturalwidth(<node> start, <node> stop)
\stopfunctioncall

\stopsubsubsection

\stopsubsection

\startsubsection[title={Math}][library=node]

\startsubsubsection[title={\type {mlist_to_hlist}}]

\libindex {mlist_to_hlist}

\startfunctioncall
<node> h =
    node.mlist_to_hlist(<node> n, <string> display_type, <boolean> penalties)
\stopfunctioncall

This runs the internal mlist to hlist conversion, converting the math list in
\type {n} into the horizontal list \type {h}. The interface is exactly the same
as for the callback \cbk {mlist_to_hlist}.

\stopsubsubsection

\startsubsubsection[title={\type {end_of_math}}]

\libindex {end_of_math}

\startfunctioncall
<node> t = node.end_of_math(<node> start)
\stopfunctioncall

Looks for and returns the next \type {math_node} following the \type {start}. If
the given node is a math end node this helper returns that node, else it follows
the list and returns the next math endnote. If no such node is found nil is
returned.

\stopsubsubsection

\stopsubsection

\stopsection

\startsection[title={Two access models}][library=node]

\topicindex{nodes+direct}
\topicindex{direct nodes}

\libindex {todirect}
\libindex {tonode}
\libindex {tostring}

Deep down in \TEX\ a node has a number which is a numeric entry in a memory
table. In fact, this model, where \TEX\ manages memory is real fast and one of
the reasons why plugging in callbacks that operate on nodes is quite fast too.
Each node gets a number that is in fact an index in the memory table and that
number often is reported when you print node related information. You go from
userdata nodes and there numeric references and back with:

\startfunctioncall
<integer> d = node.todirect(<node> n))
<node> n = node.tonode(<integer> d))
\stopfunctioncall

The userdata model is rather robust as it is a virtual interface with some
additional checking while the more direct access which uses the node numbers
directly. However, even with userdata you can get into troubles when you free
nodes that are no longer allocated or mess up lists. if you apply \type
{tostring} to a node you see its internal (direct) number and id.

The first model provides key based access while the second always accesses fields
via functions:

\starttyping
nodeobject.char
getfield(nodenumber,"char")
\stoptyping

If you use the direct model, even if you know that you deal with numbers, you
should not depend on that property but treat it as an abstraction just like
traditional nodes. In fact, the fact that we use a simple basic datatype has the
penalty that less checking can be done, but less checking is also the reason why
it's somewhat faster. An important aspect is that one cannot mix both methods,
but you can cast both models. So, multiplying a node number makes no sense.

So our advice is: use the indexed (table) approach when possible and investigate
the direct one when speed might be a real issue. For that reason \LUATEX\ also
provide the \type {get*} and \type {set*} functions in the top level node
namespace. There is a limited set of getters. When implementing this direct
approach the regular index by key variant was also optimized, so direct access
only makes sense when nodes are accessed millions of times (which happens in some
font processing for instance).

We're talking mostly of getters because setters are less important. Documents
have not that many content related nodes and setting many thousands of properties
is hardly a burden contrary to millions of consultations.

Normally you will access nodes like this:

\starttyping
local next = current.next
if next then
    -- do something
end
\stoptyping

Here \type {next} is not a real field, but a virtual one. Accessing it results in
a metatable method being called. In practice it boils down to looking up the node
type and based on the node type checking for the field name. In a worst case you
have a node type that sits at the end of the lookup list and a field that is last
in the lookup chain. However, in successive versions of \LUATEX\ these lookups
have been optimized and the most frequently accessed nodes and fields have a
higher priority.

Because in practice the \type {next} accessor results in a function call, there
is some overhead involved. The next code does the same and performs a tiny bit
faster (but not that much because it is still a function call but one that knows
what to look up).

\starttyping
local next = node.next(current)
if next then
    -- do something
end
\stoptyping

In the direct namespace there are more helpers and most of them are accompanied
by setters. The getters and setters are clever enough to see what node is meant.
We don't deal with whatsit nodes: their fields are always accessed by name. It
doesn't make sense to add getters for all fields, we just identifier the most
likely candidates. In complex documents, many node and fields types never get
seen, or seen only a few times, but for instance glyphs are candidates for such
optimization. The \type {node.direct} interface has some more helpers. \footnote
{We can define the helpers in the node namespace with \type {getfield} which is
about as efficient, so at some point we might provide that as module.}

The \type {setdisc} helper takes three (optional) arguments plus an optional
fourth indicating the subtype. Its \type {getdisc} takes an optional boolean;
when its value is \type {true} the tail nodes will also be returned. The \type
{setfont} helper takes an optional second argument, it being the character. The
directmode setter \type {setlink} takes a list of nodes and will link them,
thereby ignoring \type {nil} entries. The first valid node is returned (beware:
for good reason it assumes single nodes). For rarely used fields no helpers are
provided and there are a few that probably are used seldom too but were added for
consistency. You can of course always define additional accessors using \type
{getfield} and \type {setfield} with little overhead. When the second argument of
\type {setattributelist} is \type {true} the current attribute list is assumed.

In \CONTEXT\ some of the not performance|-|critical userdata variants are
emulated in \LUA\ and not in the engine, so we retain downward compatibility.

\def\yes{$+$} \def\nop{$-$}

\def\supported#1#2#3#4%
 {\NC \type{#1}
  \NC \ifx#2\yes\lix{node}       {#1}\fi #2
  \NC \ifx#3\yes\lix{node.direct}{#1}\fi #3 \NC
  \NC                                    #4 \NC
  \NR}

\starttabulate[|l|c|c|]
\DB function \BC node \BC direct \NC emulated \NC \NR
\TB
\supported {check_discretionaries}   \nop \yes \yes
\supported {check_discretionary}     \nop \yes \yes
\supported {copy}                    \yes \yes \relax
\supported {copy_list}               \yes \yes \relax
%supported {copy_only}               \nop \yes \relax
\supported {count}                   \nop \yes \yes
\supported {current_attr}            \yes \yes \relax
\supported {dimensions}              \nop \yes \yes
\supported {effective_glue}          \nop \yes \yes
\supported {end_of_math}             \nop \yes \yes
\supported {find_attribute}          \nop \yes \yes
%supported {find_attribute_range}    \nop \yes \relax
%supported {find_node}               \nop \yes \relax
\supported {first_glyph}             \nop \yes \yes
\supported {flatten_discretionaries} \nop \yes \yes
%supported {flush_components}        \nop \yes \relax
\supported {flush_list}              \yes \yes \relax
\supported {flush_node}              \yes \yes \relax
\supported {free}                    \yes \yes \relax
\supported {get_attribute}           \yes \yes \relax
\supported {get_properties_table}    \yes \yes \relax
\supported {get_synctex_fields}      \nop \yes \relax
\supported {getattributelist}        \nop \yes \relax
\supported {getboth}                 \nop \yes \relax
\supported {getbox}                  \nop \yes \relax
\supported {getchar}                 \nop \yes \relax
\supported {getcomponents}           \nop \yes \relax
\supported {getdata}                 \nop \yes \relax
\supported {getdepth}                \nop \yes \relax
\supported {getdirection}            \nop \yes \relax
\supported {getdisc}                 \nop \yes \relax
\supported {getexpansion}            \nop \yes \relax
\supported {getfam}                  \nop \yes \relax
\supported {getfield}                \yes \yes \relax
\supported {getfont}                 \nop \yes \relax
\supported {getglue}                 \nop \yes \yes
%supported {getglyphdata}            \nop \yes \relax % experiment
\supported {getheight}               \nop \yes \relax
\supported {getid}                   \nop \yes \relax
\supported {getkern}                 \nop \yes \relax
\supported {getlang}                 \nop \yes \relax
\supported {getleader}               \nop \yes \relax
\supported {getlist}                 \nop \yes \relax
\supported {getnext}                 \nop \yes \relax
\supported {getnormalizedline}       \nop \yes \relax
\supported {getnucleus}              \nop \yes \relax
\supported {getoffsets}              \nop \yes \relax
\supported {getorientation}          \nop \yes \relax
\supported {getpenalty}              \nop \yes \relax
\supported {getpost}                 \nop \yes \relax
\supported {getpre}                  \nop \yes \relax
\supported {getprev}                 \nop \yes \relax
\supported {getproperty}             \yes \yes \relax
\supported {getreplace}              \nop \yes \relax
\supported {getshift}                \nop \yes \relax
\supported {getsub}                  \nop \yes \relax
\supported {getsubtype}              \nop \yes \relax
\supported {getsup}                  \nop \yes \relax
\supported {getwhd}                  \nop \yes \relax
\supported {getwidth}                \nop \yes \relax
\supported {has_attribute}           \yes \yes \relax
\supported {has_dimensions}          \nop \yes \relax
\supported {has_field}               \yes \yes \relax
\supported {has_glyph}               \nop \yes \yes
\supported {hpack}                   \nop \yes \yes
%supported {ignore_math_skip}        \nop \yes \relax
\supported {insert_after}            \yes \yes \relax
\supported {insert_before}           \yes \yes \relax
\supported {is_char}                 \nop \yes \relax
\supported {is_direct}               \nop \yes \relax
\supported {is_glyph}                \nop \yes \relax
\supported {is_node}                 \yes \yes \relax
\supported {is_valid}                \nop \yes \relax
\supported {is_zero_glue}            \nop \yes \yes
\supported {kerning}                 \nop \yes \yes
\supported {last_node}               \nop \yes \yes
\supported {length}                  \nop \yes \yes
\supported {ligaturing}              \nop \yes \yes
\supported {make_extensible}         \nop \yes \yes
\supported {mlist_to_hlist}          \nop \yes \yes
\supported {naturalwidth}            \nop \yes \yes
\supported {new}                     \yes \yes \relax
\supported {prepend_prevdepth}       \nop \yes \yes
\supported {protect_glyphs}          \nop \yes \yes
\supported {protect_glyph}           \nop \yes \yes
\supported {protrusion_skippable}    \nop \yes \yes
\supported {rangedimensions}         \nop \yes \yes
\supported {remove}                  \yes \yes \relax
\supported {set_attribute}           \yes \yes \relax
\supported {set_synctex_fields}      \nop \yes \relax
\supported {setattributelist}        \nop \yes \relax
\supported {setboth}                 \nop \yes \relax
\supported {setbox}                  \nop \yes \relax
\supported {setchar}                 \nop \yes \relax
\supported {setcomponents}           \nop \yes \relax
\supported {setdata}                 \nop \yes \relax
\supported {setdepth}                \nop \yes \relax
\supported {setdirection}            \nop \yes \relax
\supported {setdisc}                 \nop \yes \relax
\supported {setexpansion}            \nop \yes \relax
\supported {setfam}                  \nop \yes \relax
\supported {setfield}                \yes \yes \relax
\supported {setfont}                 \nop \yes \relax
\supported {setglue}                 \yes \yes \relax
%supported {setglyphdata}            \nop \yes \relax % experiment
\supported {setheight}               \nop \yes \relax
\supported {setkern}                 \nop \yes \relax
\supported {setlang}                 \nop \yes \relax
\supported {setleader}               \nop \yes \relax
\supported {setlink}                 \nop \yes \relax
\supported {setlist}                 \nop \yes \relax
\supported {setnext}                 \nop \yes \relax
\supported {setnucleus}              \nop \yes \relax
\supported {setoffsets}              \nop \yes \relax
\supported {setorientation}          \nop \yes \relax
\supported {setpenalty}              \nop \yes \relax
\supported {setprev}                 \nop \yes \relax
\supported {setproperty}             \yes \yes \relax
\supported {setshift}                \nop \yes \relax
\supported {setsplit}                \nop \yes \relax
\supported {setsub}                  \nop \yes \relax
\supported {setsubtype}              \nop \yes \relax
\supported {setsup}                  \nop \yes \relax
\supported {setwhd}                  \nop \yes \relax
\supported {setwidth}                \nop \yes \relax
\supported {slide}                   \nop \yes \yes
\supported {start_of_par}            \nop \yes \relax
\supported {subtype}                 \nop \nop \relax
\supported {tail}                    \yes \yes \relax
\supported {todirect}                \nop \yes \relax
\supported {tonode}                  \nop \yes \relax
\supported {tostring}                \yes \nop \relax
\supported {traverse}                \yes \yes \relax
\supported {traverse_char}           \yes \yes \relax
\supported {traverse_glyph}          \yes \yes \relax
\supported {traverse_id}             \yes \yes \relax
\supported {traverse_list}           \yes \yes \relax
\supported {type}                    \yes \nop \relax
\supported {unprotect_glyphs}        \nop \yes \yes
\supported {unprotect_glyph}         \nop \yes \yes
\supported {unset_attribute}         \yes \yes \relax
\supported {usedlist}                \nop \yes \yes
\supported {uses_font}               \nop \yes \yes
\supported {vpack}                   \nop \yes \yes
\supported {write}                   \yes \yes \relax
\LL
\stoptabulate

The \type {node.next} and \type {node.prev} functions will stay but for
consistency there are variants called \type {getnext} and \type {getprev}. We had
to use \type {get} because \type {node.id} and \type {node.subtype} are already
taken for providing meta information about nodes. Note: The getters do only basic
checking for valid keys. You should just stick to the keys mentioned in the
sections that describe node properties.

Some of the getters and setters handle multiple node types, given that the field
is relevant. In that case, some field names are considered similar (like \type
{kern} and \type {width}, or \type {data} and \type {value}. In retrospect we
could have normalized field names better but we decided to stick to the original
(internal) names as much as possible. After all, at the \LUA\ end one can easily
create synonyms.

Some nodes have indirect references. For instance a math character refers to a
family instead of a font. In that case we provide a virtual font field as
accessor. So, \type {getfont} and \type {.font} can be used on them. The same is
true for the \type {width}, \type {height} and \type {depth} of glue nodes. These
actually access the spec node properties, and here we can set as well as get the
values.

In some places \LUATEX\ can do a bit of extra checking for valid node lists and
you can enable that with:

\startfunctioncall
node.fix_node_lists(<boolean> b)
\stopfunctioncall

You can set and query the \SYNCTEX\ fields, a file number aka tag and a line
number, for a glue, kern, hlist, vlist, rule and math nodes as well as glyph
nodes (although this last one is not used in native \SYNCTEX).

\startfunctioncall
node.set_synctex_fields(<integer> f, <integer> l)
<integer> f, <integer> l =
    node.get_synctex_fields(<node> n)
\stopfunctioncall

Of course you need to know what you're doing as no checking on sane values takes
place. Also, the synctex interpreter used in editors is rather peculiar and has
some assumptions (heuristics).

\stopsection

\startsection[title={Normalization}][library=node]

As an experiment the lines resulting from paragraph construction can be normalized.
There are several modes, that can be set and queried with:

\startfunctioncall
node.direct.setnormalize(<integer> n)
<integer> n = node.direct.getnormalize()
\stopfunctioncall

The state of a line (a hlist) can be queried with:

\startfunctioncall
<integer> leftskip, <integer> rightskip,
    <integer> lefthangskip, <integer> righthangskip,
    <node> head, <node> tail,
    <integer> parindent, <integer> parfillskip = node.direct.getnormalized()
\stopfunctioncall

The modes accumulate, so mode \type {4} includes \type {1} upto \type {3}:

\starttabulate[|l|p|]
\DB value    \BC explanation \NC \NR
\TB
\NC \type{1} \NC left and right skips and directions \NC \NR
\NC \type{2} \NC indentation and parfill skip \NC \NR
\NC \type{3} \NC hanging indentation and par shapes \NC \NR
\NC \type{4} \NC idem but before left and right skips \NC \NR
\NC \type{5} \NC inject compensation for overflow \NC \NR
\LL
\stoptabulate

This is experimental code and might take a while to become frozen.

\stopsection

\startsection[title={Properties}][library=node]

\topicindex {nodes+properties}
\topicindex {properties}

\libindex{flush_properties_table}
\libindex{get_properties_table}
\libindex{set_properties_mode}

Attributes are a convenient way to relate extra information to a node. You can
assign them at the \TEX\ end as well as at the \LUA\ end and and consult them at
the \LUA\ end. One big advantage is that they obey grouping. They are linked
lists and normally checking for them is pretty efficient, even if you use a lot
of them. A macro package has to provide some way to manage these attributes at the
\TEX\ end because otherwise clashes in their usage can occur.

Each node also can have a properties table and you can assign values to this
table using the \type {setproperty} function and get properties using the \type
{getproperty} function. Managing properties is way more demanding than managing
attributes.

Take the following example:

\starttyping
\directlua {
    local n = node.new("glyph")

    node.setproperty(n,"foo")
    print(node.getproperty(n))

    node.setproperty(n,"bar")
    print(node.getproperty(n))

    node.free(n)
}
\stoptyping

This will print \type {foo} and \type {bar} which in itself is not that useful
when multiple mechanisms want to use this feature. A variant is:

\starttyping
\directlua {
    local n = node.new("glyph")

    node.setproperty(n,{ one = "foo", two = "bar" })
    print(node.getproperty(n).one)
    print(node.getproperty(n).two)

    node.free(n)
}
\stoptyping

This time we store two properties with the node. It really makes sense to have a
table as property because that way we can store more. But in order for that to
work well you need to do it this way:

\starttyping
\directlua {
    local n = node.new("glyph")

    local t = node.getproperty(n)

    if not t then
        t = { }
        node.setproperty(n,t)
    end
    t.one = "foo"
    t.two = "bar"

    print(node.getproperty(n).one)
    print(node.getproperty(n).two)

    node.free(n)
}
\stoptyping

Here our own properties will not overwrite other users properties unless of
course they use the same keys. So, eventually you will end up with something:

\starttyping
\directlua {
    local n = node.new("glyph")

    local t = node.getproperty(n)

    if not t then
        t = { }
        node.setproperty(n,t)
    end
    t.myself = { one = "foo", two = "bar" }

    print(node.getproperty(n).myself.one)
    print(node.getproperty(n).myself.two)

    node.free(n)
}
\stoptyping

This assumes that only you use \type {myself} as subtable. The possibilities are
endless but care is needed. For instance, the generic font handler that ships
with \CONTEXT\ uses the \type {injections} subtable and you should not mess with
that one!

There are a few helper functions that you normally should not touch as user: \typ
{flush_properties_table} will wipe the table (normally a bad idea), \typ
{get_properties_table} and will give the table that stores properties (using
direct entries) and you can best not mess too much with that one either because
\LUATEX\ itself will make sure that entries related to nodes will get wiped when
nodes get freed, so that the \LUA\ garbage collector can do its job. In fact, the
main reason why we have this mechanism is that it saves the user (or macro
package) some work. One can easily write a property mechanism in \LUA\ where
after a shipout properties gets cleaned up but it's not entirely trivial to make
sure that with each freed node also its properties get freed, due to the fact
that there can be nodes left over for a next page. And having a callback bound to
the node deallocator would add way to much overhead.

When we copy a node list that has a table as property, there are several
possibilities: we do the same as a new node, we copy the entry to the table in
properties (a reference), we do a deep copy of a table in the properties, we
create a new table and give it the original one as a metatable. After some
experiments (that also included timing) with these scenarios we decided that a
deep copy made no sense, nor did nilling. In the end both the shallow copy and
the metatable variant were both ok, although the second one is slower. The most
important aspect to keep in mind is that references to other nodes in properties
no longer can be valid for that copy. We could use two tables (one unique and one
shared) or metatables but that only complicates matters.

When defining a new node, we could already allocate a table but it is rather easy
to do that at the lua end e.g.\ using a metatable \type {__index} method. That
way it is under macro package control. When deleting a node, we could keep the
slot (e.g. setting it to false) but it could make memory consumption raise
unneeded when we have temporary large node lists and after that only small lists.
Both are not done.

So in the end this is what happens now: when a node is copied, and it has a table
as property, the new node will share that table. If the second argument of \typ
{set_properties_mode} is \type {true} then a metatable approach is chosen: the
copy gets its own table with the original table as metatable. If you use the
generic font loader the mode is enabled that way.

A few more xperiments were done. For instance: copy attributes to the properties
so that we have fast access at the \LUA\ end. In the end the overhead is not
compensated by speed and convenience, in fact, attributes are not that slow when
it comes to accessing them. So this was rejected.

Another experiment concerned a bitset in the node but again the gain compared to
attributes was neglectable and given the small amount of available bits it also
demands a pretty strong agreement over what bit represents what, and this is
unlikely to succeed in the \TEX\ community. It doesn't pay off.

Just in case one wonders why properties make sense: it is not so much speed that
we gain, but more convenience: storing all kinds of (temporary) data in attributes
is no fun and this mechanism makes sure that properties are cleaned up when a
node is freed. Also, the advantage of a more or less global properties table is
that we stay at the \LUA\ end. An alternative is to store a reference in the node
itself but that is complicated by the fact that the register has some limitations
(no numeric keys) and we also don't want to mess with it too much.

\stopsection

\stopchapter

\stopcomponent
