% language=uk

\environment luametatex-style

\startcomponent luametatex-lua

\startchapter[reference=lua,title={Using \LUAMETATEX}]

\startsection[title={Initialization},reference=init]

\startsubsection[title={\LUAMETATEX\ as a \LUA\ interpreter}]

\topicindex {initialization}
\topicindex {\LUA+interpreter}

Although \LUAMETATEX\ is primarily meant as a \TEX\ engine, it can also serve as
a stand alone \LUA\ interpreter. There are two ways to make \LUAMETATEX\ behave
like a standalone \LUA\ interpreter:

\startitemize[packed]
\startitem
    if a \type {--luaonly} option is given on the commandline, or
\stopitem
\startitem
    if the only non|-|option argument (file) on the commandline has the extension
    \type {lua} or \type {luc}.
\stopitem
\stopitemize

In this mode, it will set \LUA's \type {arg[0]} to the found script name, pushing
preceding options in negative values and the rest of the command line in the
positive values, just like the \LUA\ interpreter does.

\LUAMETATEX\ will exit immediately after executing the specified \LUA\ script and is,
in effect, a somewhat bulky stand alone \LUA\ interpreter with a bunch of extra
preloaded libraries.

When no argument is given, \LUAMETATEX\ will look for a \LUA\ file with the same
name as the binary and run that one when present. This makes it possible to use
the engine as a stub. For instance, in \CONTEXT\ a symlink from \type {mtxrun} to
type {luametatex} will run the \type {mtxrun.lua} script when present in the same
path as the binary itself

\stopsubsection

\startsubsection[title={Other commandline processing}]

\topicindex {command line}

When the \LUAMETATEX\ executable starts, it looks for the \type {--lua} command line
option. If there is no \type {--lua} option, the command line is interpreted in a
similar fashion as the other \TEX\ engines. All options are accepted but only some
have are understood by \LUAMETATEX\ itself:

\starttabulate[|l|p|]
\DB commandline argument    \BC explanation \NC \NR
\TB
\NC \type{--credits}        \NC display credits and exit \NC \NR
\NC \type{--fmt=FORMAT}     \NC load the format file \type {FORMAT} \NC\NR
\NC \type{--help}           \NC display help and exit \NC\NR
\NC \type{--ini}            \NC be \type {iniluatex}, for dumping formats \NC\NR
\NC \type{--jobname=STRING} \NC set the job name to \type {STRING} \NC \NR
\NC \type{--lua=FILE}       \NC load and execute a \LUA\ initialization script \NC\NR
\NC \type{--version}        \NC display version and exit \NC \NR
\LL
\stoptabulate

There are less options than with \LUATEX, because one has to deal with them in
\LUA\ anyway. There are no options to enter a safer mode or control executing
programs. This can easily be achieved with a startup \LUA\ script.

The value to use for \prm {jobname} is decided as follows:

\startitemize
\startitem
    If \type {--jobname} is given on the command line, its argument will be the
    value for \prm {jobname}, without any changes. The argument will not be
    used for actual input so it need not exist. The \type {--jobname} switch only
    controls the \prm {jobname} setting.
\stopitem
\startitem
    Otherwise, \prm {jobname} will be the name of the first file that is read
    from the file system, with any path components and the last extension (the
    part following the last \type {.}) stripped off.
\stopitem
\startitem
    There is an exception to the previous point: if the command line goes into
    interactive mode (by starting with a command) and there are no files input
    via \prm {everyjob} either, then the \prm {jobname} is set to \type
    {texput} as a last resort.
\stopitem
\stopitemize

Next the initialization script is loaded and executed. From within the script,
the entire command line is available in the \LUA\ table \type {arg}, beginning
with \type {arg[0]}, containing the name of the executable. As consequence
warnings about unrecognized options are suppressed.

Command line processing happens very early on. So early, in fact, that none of
\TEX's initializations have taken place yet. The \LUA\ libraries that don't deal
with \TEX\ are initialized early.

\LUAMETATEX\ allows some of the command line options to be overridden by reading
values from the \type {texconfig} table at the end of script execution (see the
description of the \type {texconfig} table later on in this document for more
details on which ones exactly).

So let's summarize this. The handling of when is called jobname is a bit complex.
There can be explicit names set on the command line but when not set they can be
taken from the \type {texconfig} table.

\starttabulate[|l|T|T|T|]
\NC startup filename \NC --lua     \NC a \LUA\ file  \NC                      \NC \NR
\NC startup jobname  \NC --jobname \NC a \TEX\ tex   \NC texconfig.jobname    \NC \NR
\NC startup dumpname \NC --fmt     \NC a format file \NC texconfig.formatname \NC \NR
\stoptabulate

These names are initialized according to \type {--luaonly} or the first filename
seen in the list of options. Special treatment of \type {&} and \type {*} as well
as interactive startup is gone.

When we are in \TEX\ mode at some point the engine needs a filename, for instance
for opening a log file. At that moment the set jobname becomes the internal one
and when it has not been set which internalized to jobname but when not set
becomes \type {texput}. When you see a \type {texput.log} file someplace on your
system it normally indicates a bad run.

When running on \MSWINDOWS\ the command line, filenames, environment variable
access etc.\ internally uses the current code page but to the user is exposed as
\UTF8. Normally users won't notice this.

% fileio_state     .jobname         : a tex string (set when a (log) file is opened)
% engine_state     .startup_jobname : handles by option parser
% environment_state.input_name      : temporary interceptor

\stopsubsection

\stopsection

\startsection[title={\LUA\ behaviour}]

\startsubsection[title={The \LUA\ version}]

\topicindex {\LUA+libraries}
\topicindex {\LUA+extensions}

We currently use \LUA\ 5.4 and will follow developments of the language but
normally with some delay. Therefore the user needs to keep an eye on (subtle)
differences in successive versions of the language. Here is an example of one
aspect.

\LUA s \type {tostring} function (and \type {string.format} may return values in
scientific notation, thereby confusing the \TEX\ end of things when it is used as
the right|-|hand side of an assignment to a \prm {dimen} or \prm {count}. The
output of these serializers also depend on the \LUA\ version, so in \LUA\ 5.3 you
can get different output than from 5.2. It is best not to depend the automatic
cast from string to number and vise versa as this can change in future versions.

\stopsubsection

\startsubsection[title={Locales}]

\index {locales}

In stock \LUA, many things depend on the current locale. In \LUAMETATEX, we can't
do that, because it makes documents unportable. While \LUAMETATEX\ is running if
forces the following locale settings:

\starttyping
LC_CTYPE=C
LC_COLLATE=C
LC_NUMERIC=C
\stoptyping

There is no way to change that as it would interfere badly with the often
language specific conversions needed at the \TEX\ end.

\stopsubsection

\stopsection

\startsection[title={\LUA\ modules}]

\topicindex {\LUA+libraries}
\topicindex {\LUA+modules}

Of course the regular \LUA\ modules are present. In addition we provide the \type
{lpeg} library by Roberto Ierusalimschy, This library is not \UNICODE|-|aware,
but interprets strings on a byte|-|per|-|byte basis. This mainly means that \type
{lpeg.S} cannot be used with \UTF8 characters encoded in more than two bytes, and
thus \type {lpeg.S} will look for one of those two bytes when matching, not the
combination of the two. The same is true for \type {lpeg.R}, although the latter
will display an error message if used with multibyte characters. Therefore \type
{lpeg.R('aä')} results in the message \type {bad argument #1 to 'R' (range must
have two characters)}, since to \type {lpeg}, \type {ä} is two 'characters'
(bytes), so \type {aä} totals three. In practice this is no real issue and with
some care you can deal with \UNICODE\ just fine.

There are some more libraries present. These are discussed on a later chapter.
For instance we embed \type {luasocket} but contrary to \LUATEX\ don't embed the
related \LUA\ code. An adapted version of \type {luafilesystem} is also included.
There is a more extensive math library and there are libraries that deal with
encryption and compression.

\stopsection

\startsection[title={Testing}]

\topicindex {testing}

For development reasons you can influence the used startup date and time. By
setting the \type {start_time} variable in the \type {texconfig} table; as with
other variables we use the internal name there. When Universal Time is needed,
set the entry \type {use_utc_time} in the \type {texconfig} table.

In \CONTEXT\ we provide the command line argument \type {--nodates} that does
a bit more than disabling dates; it avoids time dependent information in the
output file for instance.

\stopsection

\stopchapter

\stopcomponent
