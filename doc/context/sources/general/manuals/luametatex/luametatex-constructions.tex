% language=us runpath=texruns:manuals/luametatex

\environment luametatex-style

\startdocument[title=Constructions]

\startluacode
local styles     = tex.getmathstylevalues    () styles     = table.swapped(styles,styles)
local parameters = tex.getmathparametervalues() parameters = table.swapped(parameters,parameters)
local variants   = tex.getmathvariantvalues  () variants   = table.swapped(variants,variants)
local presets    = tex.getmathvariantpresets () presets    = table.swapped(presets,presets)

-- inspect(presets)
-- inspect(presets)

function document.printmathpresets()
    context.starttabulate { "||||" }
    context.FL()
    context.BC() context("construct")
    context.BC() context("value")
    context.BC() context("preset name")
    context.BC() context.NR()
    context.TL()
    for parameter=parameters.overlinevariant,parameters.stackvariant do
        local value  = tex.getmath(parameter,"display")
        local preset = variants[presets[value]]
        context.NC() context.type("\\Umath"..parameters[parameter])
        context.NC() context("0x%08X",value)
        context.NC() context(preset)
        context.NC() context.NR()
    end
    context.LL()
    context.stoptabulate()
end

function document.printmathvariants()
    for parameter=parameters.overlinevariant,parameters.stackvariant do
        for style=styles.display,styles.crampedscriptscript do
            local p, s = tex.getmathstylevariant(style,parameter)
            if p and s then
                 logs.report("mathvariants","%-24s %-20s %-18s %-20s",parameters[parameter],styles[style],variants[p],styles[s])
            end
        end
    end
end

function document.showmathvariant(variant)
    local parameter = type(variant) == "string" and parameters[variant] or variant
    context.startsubsubsubject { title = variant }
    context.starttabulate { "||T||" }
    context.BC() context("current style")
    context.BC() context("mapping")
    context.BC() context("used style")
    context.BC() context.NR()
    context.ML()
    for style=styles.display,styles.crampedscriptscript do
        local p, s = tex.getmathstylevariant(style,parameter)
        if p and s then
            context.NC() context(styles[style])
            context.NC() context("0x%08X",p)
            context.NC() context(styles[s])
            context.NC() context.NR()
        end
    end
    context.stoptabulate()
    context.stopsubsubsubject()
end
\stopluacode

\startsection[title={Introduction}]

This is more a discussion of the way some constructs in for instance math work.
It will never be exhausting and mostly is for our own usage. We don't discuss all
the options bit many are interfaced in higher level macros in \CONTEXT. This
chapter will gradually grow, depending on time and mood.

\stopsection

\startsection[title=Boxes]

Boxes are very important in \TEX. We have horizontal boxes and vertical boxes.
When you look at a page of text, the page itself is a vertical box, and among
other things it packs lines that are themselves horizontal boxes. The lines that
make a paragraph are the result of breaking a long horizontal box in smaller
pieces.

\startlinecorrection
\ruledvbox \bgroup \hsize 20em \showmakeup[line] \showboxes
    This is a vertical box. It has a few lines of text that started out as
    one long line but has been broken in pieces. Doing this as good as possible
    is one of \TEX's virtues.
\egroup
\stoplinecorrection

There is a low level manual on boxes so here we can limit the discussion to
basics. A box is in \TEX\ speak a node. In traditional \TEX\ it has a
width, height, depth and shift.

\startlinecorrection
\startMPcode
    numeric wd ; wd := 24 ;
    numeric ht ; ht :=  7 ;
    numeric dp ; dp :=  3 ;

    path body ; body := unitsquare xyscaled (wd, ht+dp) shifted (0,-dp) ;
    path line ; line := (0,0) -- (wd,0) ;

    draw line withcolor "middlegray" ;
    draw body withcolor "darkred" ;

    currentpicture := currentpicture scaled 5 ;
\stopMPcode
\stoplinecorrection

Here we see a box and the gray line is called the baseline, the height goes up
and the depth goes down. Normally the height and depth are determined by what
goes in the box but they can be changed as we like.

\startbuffer
\setbox\scratchboxone\ruledhpack{SHIFT 1}
\setbox\scratchboxtwo\ruledhpack{SHIFT 2}

\boxshift\scratchboxtwo 1ex \dontleavehmode \box\scratchboxone\box\scratchboxtwo

\setbox\scratchboxone\ruledvpack{SHIFT 3}
\setbox\scratchboxtwo\ruledvpack{SHIFT 4}

\boxshift\scratchboxtwo 1ex \box\scratchboxone\box\scratchboxtwo
\stopbuffer

\typebuffer

In this example you'll notice that the shift depends on the box being horizontal
or vertical. The primitives \type {\raise}, \type {\lower}, \type {\moveleft} and
\type {\moveright} can be used to shift a box.

\startlinecorrection
\getbuffer
\stoplinecorrection

The reason why we have the shift property is that it is more efficient than
wrapping a box in another box and shifting with kerns. In that case we also have
to go via a box register so that we can manipulate the final dimensions. Another
advantage is that the engine can use shifts to position for instance elements in
a math formula and even the par builder used shifts to deal with positioning the
lines according to shape and margin. In \LUAMETATEX\ the later is no longer the
case.

Inside a box there can be mark (think running headers), insert (think footnotes)
and adjust (think injecting something before or after the current line) nodes. The
par builder will move this from inside the box to between the lines but when boxes
are nested too deeply this won't happen and they get lost. In \LUAMETATEX\ these
objects do bubble up because we make them box properties. So, in addition to
the dimensions and shift a box also has migration fields.

In the low level manuals you can find examples of accessing various properties of
boxes so here we stick to a short description. The reason for mentioning them is
that it gives you an idea of what goes on in the engine.

\starttabulate
\FL
\BC field                 \BC usage \NC \NR
\TL
\NC \type {width}         \NC the (used) width \NC \NR
\NC \type {height}        \NC the (used) height \NC \NR
\NC \type {depth}         \NC the (used) depth \NC \NR
\NC \type {shift_amount}  \NC the shift (right or down) \NC \NR
\ML
\NC \type {list}          \NC pointer to the content \NC \NR
\ML
\NC \type {glue_order}    \NC the calculated order of glue stretch of shrink \NC \NR
\NC \type {glue_sign}     \NC the determined sign of glue stretch of shrink \NC \NR
\NC \type {glue_set}      \NC the calculated multiplier for glue stretch or shrink \NC \NR
\ML
\NC \type {geometry}      \NC a bit set registering manipulations \NC \NR
\NC \type {orientation}   \NC positional manipulations \NC \NR
\NC \type {w_offset}      \NC used in horizontal movement calculations \NC \NR
\NC \type {h_offset}      \NC used in vertical movement calculations \NC \NR
\NC \type {d_offset}      \NC used in vertical movement calculations \NC \NR
\NC \type {x_offset}      \NC a horizontal shift independent of dimensions \NC \NR
\NC \type {y_offset}      \NC a vertical shift independent of dimensions  \NC \NR
\NC \type {axis}          \NC the math axis \NC \NR
\ML
\NC \type {dir}           \NC the direction the box goes to (l2r or r2l) \NC \NR
\NC \type {package_state} \NC a bitset indicating how the box came to be as it is \NC \NR
\NC \type {index}         \NC a (system dependent) identifier \NC \NR
\ML
\NC \type {pre_migrated}  \NC content bound to the box that eventually will be injected \NC \NR
\NC \type {post_migrated} \NC idem \NC \NR
\NC \type {pre_adjusted}  \NC idem \NC \NR
\NC \type {post_adjusted} \NC idem \NC \NR
\ML
\NC \type {source_anchor} \NC an identifier bound to the box \NC \NR
\NC \type {target_anchor} \NC idem \NC \NR
\NC \type {anchor}        \NC a bitset indicating where and how to anchor \NC \NR
\ML
\NC \type {except}        \NC carried information about additional virtual depth \NC \NR
\NC \type {exdepth}       \NC additional virtual depth taken into account in the page builder \NC \NR
\LL
\stoptabulate

We have the usual dimension but also extra ones that relate to \typ {\boxxoffset}
and \typ {\boxyoffset} (these are virtual) as well as \typ {\boxxmove} and \typ
{\boxymove} (these influence dimensions). The \typ {\boxorientation} also gets
registered. The state fields carry information that is used in various places,
the pre and post fields relate to the mentioned embedded content. Anchors are
just there so that a macro package can play with this and excepts refer to an
additional dimensions that is looked at in the page builder, for instance in
order to prevent a page break at an unlucky spot. It all gives an indication of
what we are dealing with.

\stopsection

\startsection[title=Math style variants]

The \LUAMETATEX\ math engine is a follow up on the one in \LUATEX. That one
gradually became more configurable in order to deal with both traditional fonts
and \OPENTYPE\ fonts. In \LUAMETATEX\ much has been redone, opened up and
extended. New mechanisms and constructs have been added. In the process hard
coded heuristics with regards to math styles inside constructions were made
configurable, a feature that is probably not used much, apart from experimenting.
A side effect is that we can show how the engine is set up, so we do that when
applicable.

% \starttabulate
% \NC normal            \NC \tohexa8 \mathnormalstylepreset            \NC \NR
% \NC cramped           \NC \tohexa8 \mathcrampedstylepreset           \NC \NR
% \NC subscript         \NC \tohexa8 \mathsubscriptstylepreset         \NC \NR
% \NC superscript       \NC \tohexa8 \mathsuperscriptstylepreset       \NC \NR
% \NC small             \NC \tohexa8 \mathsmallstylepreset             \NC \NR
% \NC smaller           \NC \tohexa8 \mathsmallerstylepreset           \NC \NR
% \NC numerator         \NC \tohexa8 \mathnumeratorstylepreset         \NC \NR
% \NC denominator       \NC \tohexa8 \mathdenominatorstylepreset       \NC \NR
% \NC doublesuperscript \NC \tohexa8 \mathdoublesuperscriptstylepreset \NC \NR
% \stoptabulate

\ctxlua{document.printmathpresets()}

\stopsection

\startsection[title=Math scripts]

The basic components in a math formula are characters, accents, fractions,
radicals and fences. They are represented in the to be processed node list as
noads and eventually are converted in glyph, kern, glue and list nodes. Each noad
carries similar but also specific information about its purpose and intended
rendering. In \LUAMETATEX\ that is quite a bit more than in traditional \TEX.

These noads are often called atoms. The center piece in a noad is called the
nucleus. The fact that these noads also can have scripts attached makes them more
like molecules. Scripts can be attached to the left and right, high or low. That
makes fours of them: pre|/|post super|/|sub scripts. In \LUAMETATEX\ we also have
a prime script, which comes on its own, above a post subscript or after the post
superscript, if given.

\startlinecorrection
\startMPcode
    numeric wd ; wd := 10 ;
    numeric ht ; ht :=  8 ;
    numeric dp ; dp :=  2 ;

    path nucleus         ; nucleus         := fullsquare xyscaled (2wd,  ht  +dp  ) shifted (0,-dp  ) ;
    path follower        ; follower        := fullsquare xyscaled ( wd/3,ht/6+dp/4) shifted (0,-dp  ) ;
    path presubscript    ; presubscript    := fullsquare xyscaled ( wd,  ht/2+dp/2) shifted (0,-dp/2) ;
    path presuperscript  ; presuperscript  := fullsquare xyscaled ( wd,  ht/2+dp/2) shifted (0,-dp/2) ;
    path postsubscript   ; postsubscript   := fullsquare xyscaled ( wd,  ht/2+dp/2) shifted (0,-dp/2) ;
    path postsuperscript ; postsuperscript := fullsquare xyscaled ( wd,  ht/2+dp/2) shifted (0,-dp/2) ;
    path primescript     ; primescript     := fullsquare xyscaled ( wd/2,ht/2+dp/2) shifted (0,-dp/2) ;

    presubscript    := anchored.lrt  (presubscript,    llcorner nucleus) shifted (-1,0) ;
    presuperscript  := anchored.lrt  (presuperscript,  ulcorner nucleus) shifted (-1,0) ;
    postsubscript   := anchored.llft (postsubscript,   lrcorner nucleus) shifted ( 1,0) ;
    postsuperscript := anchored.llft (postsuperscript, urcorner nucleus) shifted ( 1,0) ;
    primescript     := anchored.llft (primescript,     urcorner nucleus) shifted (bbwidth postsuperscript + 2,0) ;

    draw nucleus         withcolor "darkgray"   ;
    draw presubscript    withcolor "darkred"    ;
    draw presuperscript  withcolor "darkgreen"  ;
    draw postsubscript   withcolor "darkblue"   ;
    draw postsuperscript withcolor "darkyellow" ;
    draw primescript     withcolor "darkorange" ;

    draw follower        shifted (  .5bbwidth nucleus + bbwidth postsuperscript + bbwidth primescript + 3,0) withcolor "darkgray"   ;
    draw follower        shifted (- .5bbwidth nucleus - bbwidth presuperscript                        - 1,0) withcolor "darkgray"   ;
    draw presubscript    shifted (                    - bbwidth presubscript                          - 1,0) withcolor "darkred"    ;
    draw presuperscript  shifted (                    - bbwidth presuperscript                        - 1,0) withcolor "darkgreen"  ;
    draw postsubscript   shifted (                      bbwidth postsubscript   + bbwidth primescript + 2,0) withcolor "darkblue"   ;
    draw postsuperscript shifted (                      bbwidth postsuperscript + bbwidth primescript + 2,0) withcolor "darkyellow" ;

    currentpicture := currentpicture scaled 5 ;
\stopMPcode
\stoplinecorrection

Here the raised rectangle represents the prime. The large center piece is the
nucleus. Four scripts are attached to the nucleus. The two smaller center pieces
indicate follow up atoms. They make it possible to have multiple pre- and
postscripts. For single scripts we get combinations like these:

\startlinecorrection
\startcombination[nx=3,ny=2,distance=4em]
    {\scale [s=3] {\showglyphs\im { X   ^{\tt a} _{\tt b} ^^^{\tt c} ___{\tt d} \mathaxisbelow }}} {}
    {\scale [s=3] {\showglyphs\im { X            _{\tt b}            ___{\tt d} \mathaxisbelow }}} {}
    {\scale [s=3] {\showglyphs\im { X   ^{\tt a}          ^^^{\tt c}            \mathaxisbelow }}} {}
    {\scale [s=3] {\showglyphs\im { X ' ^{\tt a} _{\tt b} ^^^{\tt c} ___{\tt d} \mathaxisbelow }}} {}
    {\scale [s=3] {\showglyphs\im { X '          _{\tt b}            ___{\tt d} \mathaxisbelow }}} {}
    {\scale [s=3] {\showglyphs\im { X ' ^{\tt a}          ^^^{\tt c}            \mathaxisbelow }}} {}
\stopcombination
\stoplinecorrection

And for multiple (there can be more that two) we get this assembly:

\startlinecorrection
\startcombination[nx=3,ny=2,distance=4em]
    {\scale [s=3] {\showglyphs\im { X   ^{\tt a} _{\tt b} ^^^{\tt c} ___{\tt d}  ^{\tt A} _{\tt B} ^^^{\tt C} ___{\tt D} \mathaxisbelow }}} {}
    {\scale [s=3] {\showglyphs\im { X            _{\tt b}            ___{\tt d}           _{\tt B}            ___{\tt D} \mathaxisbelow }}} {}
    {\scale [s=3] {\showglyphs\im { X   ^{\tt a}          ^^^{\tt c}             ^{\tt A}          ^^^{\tt C}            \mathaxisbelow }}} {}
    {\scale [s=3] {\showglyphs\im { X ' ^{\tt a} _{\tt b} ^^^{\tt c} ___{\tt d}  ^{\tt A} _{\tt B} ^^^{\tt C} ___{\tt D} \mathaxisbelow }}} {}
    {\scale [s=3] {\showglyphs\im { X '          _{\tt b}            ___{\tt d}           _{\tt B}            ___{\tt D} \mathaxisbelow }}} {}
    {\scale [s=3] {\showglyphs\im { X ' ^{\tt a}          ^^^{\tt c}             ^{\tt A}          ^^^{\tt C}            \mathaxisbelow }}} {}
\stopcombination
\stoplinecorrection

It will be clear that there is quite a bit of code involved in dealing with this
because these scripts are not only to be anchored relative to the nucleus but
also to each other. The dimensions of the scripts determine for instance how
close a combined super and subscript are positioned.

\startlinecorrection
\startcombination[nx=3,ny=1,distance=2em]
    {\scale [s=5] {\showglyphs\im { X          _{\tt p} \mathaxisbelow }}} {}
    {\scale [s=5] {\showglyphs\im { X ^{\tt p} _{\tt p} \mathaxisbelow }}} {}
    {\scale [s=5] {\showglyphs\im { X ^{\tt p}          \mathaxisbelow }}} {}
\stopcombination
\stoplinecorrection

The rendering of scripts involves several parameters, of which some relate to
font parameters. In \LUAMETATEX\ we have a few more variables and we also
overload font parameters, if only because only a few make sense and it looks like
font designers just copy values from the first available fonts so in the end we
can as well use our own preferred values.

The following parameters play a role in rendering the shown assembly, The
traditional \TEX\ engine expects a math font to set quite some parameters for
positioning the scripts but has no concept of prescripts and neither has
\OPENTYPE. This is why we have extra parameters (and for completeness we also
have them for the post scripts). One can wonder of font parameters make sense
here because in the end we can decide for a better visual result with different
ones. After all, assembling scripts is not really what fonts are about.

\starttabulate[||||||]
\FL
\BC engine parameter                                \BC target \BC open type font                      \BC tex font            \BC \NR
\TL
\NC                   subscriptshiftdrop            \NC post   \NC SubscriptBaselineDropMin            \NC subdrop             \NC \NR
\NC                   subscriptshiftdown            \NC post   \NC SubscriptShiftDown                  \NC sub1                \NC \NR
\NC                   subscriptsuperscriptshiftdown \NC post   \NC SubscriptShiftDown[WithSuperscript] \NC sub2                \NC \NR
\NC                   subscriptsuperscriptvgap      \NC post   \NC SubSuperscriptGapMin                \NC 4   rulethickness   \NC \NR
\NC                   subscripttopmax               \NC post   \NC SubscriptTopMax                     \NC 4/5 xheight         \NC \NR

\NC                   superscriptshiftdrop          \NC post   \NC SuperscriptBaselineDropMax          \NC supdrop             \NC \NR
\NC                   superscriptbottommin          \NC post   \NC SuperscriptBottomMin                \NC 1/4 xheight         \NC \NR
\NC                   superscriptshiftup            \NC post   \NC SuperscriptShiftUp[Cramped]         \NC sup1 sup2 sup3      \NC \NR
\NC                   superscriptsubscriptbottommax \NC post   \NC SuperscriptBottomMaxWithSubscript   \NC 4/5 xheight         \NC \NR

\NC \llap{\star\space}primeraise                    \NC prime  \NC PrimeRaisePercent                   \NC                     \NC \NR
\NC \llap{\star\space}primeraisecomposed            \NC prime  \NC PrimeRaiseComposedPercent           \NC                     \NC \NR
\NC \llap{\star\space}primeshiftup                  \NC prime  \NC PrimeShiftUp[Cramped]               \NC                     \NC \NR
\NC \llap{\star\space}primeshiftdrop                \NC prime  \NC PrimeBaselineDropMax                \NC                     \NC \NR
\NC \llap{\star\space}primespaceafter               \NC prime  \NC PrimeSpaceAfter                     \NC                     \NC \NR

\NC                   spaceafterscript              \NC post   \NC SpaceAfterScript                    \NC \typ {\scriptspace} \NC \NR
\NC \llap{\star\space}spacebeforescript             \NC post   \NC SpaceBeforeScript                   \NC                     \NC \NR
\NC \llap{\star\space}spacebetweenscript            \NC multi  \NC SpaceBetweenScript                  \NC                     \NC \NR

\NC \llap{\star\space}extrasuperscriptshift         \NC pre    \NC                                     \NC                     \NC \NR
\NC \llap{\star\space}extrasuperprescriptshift      \NC pre    \NC                                     \NC                     \NC \NR
\NC \llap{\star\space}extrasubscriptshift           \NC pre    \NC                                     \NC                     \NC \NR
\NC \llap{\star\space}extrasubprescriptshift        \NC pre    \NC                                     \NC                     \NC \NR

\NC \llap{\star\space}extrasuperscriptspace         \NC post   \NC                                     \NC                     \NC \NR
\NC \llap{\star\space}extrasubscriptspace           \NC post   \NC                                     \NC                     \NC \NR
\NC \llap{\star\space}extrasuperprescriptspace      \NC pre    \NC                                     \NC                     \NC \NR
\NC \llap{\star\space}extrasubprescriptspace        \NC pre    \NC                                     \NC                     \NC \NR
\LL
\stoptabulate

The parameters marked by a \star\ are \LUAMETATEX\ specific. Some have an
associated font parameter but that is not official \OPENTYPE. For a very long
time we had only a few math fonts but even today most of these fonts seem to
use values that are similar to the ones \TEX\ uses. In that respect one can
as well turn them into rendering specific ones. After all, changes are slim that
a formula rendered by \TEX\ or e.g.\ \MSWORD\ are metric compatible and with
the advanced spacing options in \LUAMETATEX\ we're even further off. Also keep
in mind that the \TEX\ font parameters could be overloaded at the \TEX\ end.

The spacing after a (combination of) postscript(s) is determined by \quote {space
after script} and the spacing before a (combination of) prescript(s) by \quote
{space before script}. If we have multi-scripts the \quote {space between script}
kicks in and the space after the script is subtracted from it. The given space
between is scaled with the \type {\scriptspacebetweenfactor} parameter.

The default style mapping that we use are the same as those (hard coded) in
regular \TEX\ and those for prime scripts are the same as for superscripts.

\startluacode
document.showmathvariant("subscriptvariant")
document.showmathvariant("superscriptvariant")
document.showmathvariant("primevariant")
\stopluacode

\stopsection

\startsection[title=Skewed fractions]

Skewed fractions are native in \LUAMETATEX. Such a fraction is a horizontal
construct with the numerator and denominator shifted up and down a bit. It looks
like this:

\startlinecorrection
\startMPcode
    vardef baseline expr p =
        (xpart llcorner p, 0) -- (xpart lrcorner p,0)
    enddef ;

    numeric wd ; wd := 12 ;
    numeric ht ; ht :=  7 ;
    numeric dp ; dp :=  3 ;

    numeric ax ; ax :=  4 ;
    numeric ws ; ws :=  5 ;
    numeric hs ; hs :=  6 ;
    numeric ds ; ds :=  3 ;

    path slash       ; slash       := (-ws,-(hs+ds)) -- (ws,(hs+ds))  ;
    path numerator   ; numerator   := unitsquare xyscaled ( wd, ht+dp) shifted (0,-dp) && ((0,0) -- (wd,0)) ;
    path denominator ; denominator := numerator ;

    numerator   := numerator   shifted - center numerator ;
    denominator := denominator shifted - center denominator ;

    numerator   := anchored.lft (numerator,   center slash) shifted (-3.5ws, ax/2) ;
    denominator := anchored.rt  (denominator, center slash) shifted ( 3.5ws,-ax/2) ;

    draw slash       withcolor "darkgray" withpen pencircle scaled 1 ;
    draw numerator   withcolor "darkred"  ;
    draw denominator withcolor "darkblue" ;

    path line ; line := (baseline currentpicture) leftenlarged 1 rightenlarged 1 ;

    draw line                withstacking -1 withcolor "middlegray" ;
    draw line shifted (0,ax) withstacking -1 withcolor "darkgreen"  ; % axis

    currentpicture := currentpicture scaled 5 ;
\stopMPcode
\stoplinecorrection

The rendering is driven by some parameters that determine the horizontal and
vertically shifts but we found that the ones given by the font make no sense (and
are not that well defined in the standard either). The horizontal shift relates
to the width (and angle) of the slash and the vertical relates to the math axis.
We don't listen to \quote {skewed fraction hgap} nor to \quote {skewed fraction
vgap} but use the width of the middle character, normally a slash, that can grow
on demand and multiply that with a \type {hfactor} that can be passed with the
fraction command. A \type {vfactor} is used a multiplier for the vertical shift
over the axis. Examples of (more)) control can be found in the \CONTEXT\ math
manual. Here we just show a few examples that use \type {\vfrac} with its
default values.

\startlinecorrection
\startcombination[nx=3,ny=2,distance=4em]
    {\scale [s=3] {\showglyphs\im { \vfrac{1}    {2}     \mathaxisbelow }}} {}
    {\scale [s=3] {\showglyphs\im { \vfrac{a}    {b}     \mathaxisbelow }}} {}
    {\scale [s=3] {\showglyphs\im { \vfrac{b}    {a}     \mathaxisbelow }}} {}
    {\scale [s=3] {\showglyphs\im { \vfrac{x^2}  {x^3}   \mathaxisbelow }}} {}
    {\scale [s=3] {\showglyphs\im { \vfrac{(x+1)}{(x+2)} \mathaxisbelow }}} {}
    {\scale [s=3] {\showglyphs\im { \vfrac{x+1}  {x+2}   \mathaxisbelow }}} {}
\stopcombination
\stoplinecorrection

The quality of the slashes differs per font, some lack granularity in sizes,
others have inconsistent angles between the base character and larger variants.

The following commands are used:

\starttyping
\Uskewed
\Uskewedwithdelims
\stoptyping

There are some parameter involved:

\starttyping
\Umathskeweddelimitertolerance
\Umathskewedfractionhgap
\Umathskewedfractionvgap
\stoptyping

\stopsection

\startsection[title=Math fractions]

todo

\stopsection

\startsection[title=Math radicals]

Radicals indeed look like roots. But the radical mechanism basically is a
wrapping construct: there's something at the left that in traditional \TEX\ gets
a rule appended. The left piece is an extensible, so it first grow with variant
glyphs and when we run out if these we get an upward extensible with a repeated
upward rule like symbol that then connect with the horizontal rule. In
\LUAMETATEX\ the horizontal rule can be an extensible (repeated symbol) and we
can also have a symbol at the right, which indeed can be a vertical extensible
too.

\startlinecorrection
\startMPcode
    vardef baseline expr p =
        (xpart llcorner p, 0) -- (xpart lrcorner p,0)
    enddef ;

    numeric wd ; wd := 24 ;
    numeric ht ; ht :=  7 ;
    numeric dp ; dp :=  3 ;

    numeric dw ; dw :=  3 ;
    numeric dh ; dh :=  3 ;
    numeric dd ; dd :=  1 ;

    numeric rw ; rw :=  2 ;
    numeric rh ; rh :=  2 + ht ;
    numeric rd ; rd :=  2 + dp ;

    path rleft   ; rleft   := (-3rw,rh/3) -- (-2rw,-rd) -- (-rw,rh) ;
    path rright  ; rright  := (wd+2,rh) -- (wd+2,rh-2);
    path rmiddle ; rmiddle := (-rw,rh) -- (wd+2,rh);
    path body    ; body    := unitsquare xyscaled ( wd, ht+dp) shifted (0,-dp) && ((0,0) -- (wd,0)) ;
    path degree  ; degree  := unitsquare xyscaled ( dw, dh+dd) shifted (0,-dd) && ((0,0) -- (dw,0)) ;

    degree := degree shifted (-3rw-2,rh/3+dd+2) ;

    draw rleft   withcolor "darkblue"  ;
    draw rmiddle withcolor "darkgray"  ;
    draw rright  withcolor "darkblue"  ;
    draw body    withcolor "darkred"  ;
    draw degree  withcolor "darkgreen"  ;

    path line ; line := (baseline currentpicture) leftenlarged 1 rightenlarged 1 ;

    draw line withstacking -1 withcolor "middlegray" ;

    currentpicture := currentpicture scaled 5 ;
\stopMPcode
\stoplinecorrection

Here are some aspects to take care of when rendering a radical like this:

\startitemize[packed]
\startitem The radical symbol goes below the baseline of what it contains. \stopitem
\startitem There is some distance between the left symbol and the body. \stopitem
\startitem There is some distance between the top symbol and the body. \stopitem
\startitem There is some distance between the right symbol and the body. \stopitem
\startitem The degree has to be anchored properly and possibly can stick out left. \stopitem
\startitem The (upto) three elements have to overlap a little to avoid artifacts. \stopitem
\startitem Multiple radicals might have to be made consistent with respect to heights and depths. \stopitem
\stopitemize

Involved commands:

\starttyping
\Uradical
\Uroot
\Urooted
\stoptyping

Relevant parameters:

\starttyping
\Umathradicaldegreeafter
\Umathradicaldegreebefore
\Umathradicaldegreeraise
\Umathradicalextensibleafter
\Umathradicalextensiblebefore
\Umathradicalkern
\Umathradicalrule
\Umathradicalvariant
\Umathradicalvgap
\stoptyping

\stopsection

\startsection[title=Math accents]

todo

\stopsection

\startsection[title=Math fences]

todo

\stopsection

\stopdocument

% \tohexa8 \mathnormalstylepreset            \par
% \tohexa8 \mathcrampedstylepreset           \par
% \tohexa8 \mathsubscriptstylepreset         \par
% \tohexa8 \mathsuperscriptstylepreset       \par
% \tohexa8 \mathsmallstylepreset             \par
% \tohexa8 \mathsmallerstylepreset           \par
% \tohexa8 \mathnumeratorstylepreset         \par
% \tohexa8 \mathdenominatorstylepreset       \par
% \tohexa8 \mathdoublesuperscriptstylepreset \par

% \tohexa8 \Umathoverlayaccentvariant \displaystyle\par
% \tohexa8 \Umathradicalvariant       \displaystyle\par
% \tohexa8 \Umathradicalvariant       \textstyle   \par
% \tohexa8 \Umathunderlinevariant     \displaystyle\par

% $2x^{2^2}$
% $\sqrt{x-\frac{1}{\sqrt{2}}}$
% $\Umathradicalvariant \allmathstyles 0 \sqrt{x-\frac{1}{\sqrt{2}}}$
% $2\frac{1}{2}$

% An alternative description:
%
% \def\blob#1{\mathord{\ruledhbox{\strut\tttf #1}}}
%
% The most common cases of scripts attached to a nucleus are the following where
% vertical spacing depends on the dimensions and combinations:
%
% \startlinecorrection
% \im { \blob{nuc}
%     \superscript   {\blob{sup 1}}
% }
% \quad
% \im { \blob{nuc}
%     \subscript     {\blob{sub 1}}
% }
% \quad
% \im { \blob{nuc}
%     \superscript   {\blob{sup 1}}
%     \subscript     {\blob{sub 1}}
% }
% \stoplinecorrection
%
% We also have primes. These have always been somewhat messy in \TEX\ (macro
% packages) because they are on the one hand already raised (in text they also
% serve as minutes) but because they are supposed to be anchored as super script
% the non raised variant us used. Another complication is that, especially in
% \UNICODE\ the multiple primes are single characters. In \LUAMETATEX\ primes are
% an additional script which makes for better positioning, definitely when we also
% have a superscript. Multiple primes are collected and can be collapsed into a
% single character. \footnote {In \CONTEXT\ \MKIV\ primes are handles at the \LUA\
% end and become scripts. In \MKIV\ as well as in \MKXL\ collapsing is still a
% \LUA\ task, but it is much simpler in \MKXL.}
%
% \startlinecorrection
% \im { \blob{nuc}
%     \primescript   {\blob{pri}}
% }
% \quad
% \im { \blob{nuc}
%     \primescript   {\blob{pri}}
%     \subscript     {\blob{sub 1}}
% }
% \quad
% \im { \blob{nuc}
%     \primescript   {\blob{pri}}
%     \superscript   {\blob{sup 1}}
% }
% \quad
% \im { \blob{nuc}
%     \primescript   {\blob{pri}}
%     \superscript   {\blob{sup 1}}
%     \subscript     {\blob{sub 1}}
% }
% \stoplinecorrection
%
% We also have prescripts and these have to align properly with postscripts
% when present. Combinations of super- and subscripts make for different spacing
% compared to single scripts and you can imagine that looking at four possible
% scripts in various combinations have to obey to the same rules.
%
% \startlinecorrection
% \im { \blob{nuc}
%     \superscript   {\blob{sup 1}}
%     \superprescript{\blob{sup 1}}
% }
% \quad
% \im { \blob{nuc}
%     \subscript   {\blob{sup 1}}
%     \subprescript{\blob{sup 1}}
% }
% \quad
% \im { \blob{nuc}
%     \superscript {\blob{sup 1}}
%     \subprescript{\blob{sup 1}}
% }
% \stoplinecorrection
%
% Finally there can be multiscripts and often we then talk in terms of index. One
% can explicitly mark an index. A single \type {^} means super (post) script (or
% more verbose \typ {\superscript}), while \type {^^} means indexed super (post)
% script (\typ {\indexedsuperscript}). The threesome \type {^^^} makes a super pre
% script and likewise the quadruple \type {^^^^} is an indexed super pre script.
%
% \startlinecorrection
% \im { \blob{nuc}
%     \primescript   {\blob{pri}}
%     \superscript   {\blob{sup 1}}
%     \subscript     {\blob{sub 1}}
%     \superprescript{\blob{presup 1}}
%     \subprescript  {\blob{presub 1}}
%     \superscript   {\blob{sup 2}}
%     \subscript     {\blob{sub 2}}
%     \superprescript{\blob{presup 2}}
%     \subprescript  {\blob{presub 2}}
% }
% \stoplinecorrection
%
% We can move to the next multiscript assembly with \typ {\noscript} so we have some
% control over positioning.
%
% \startlinecorrection
% \im { \blob{nuc}
%     \superscript   {\blob{sup 1}}
%     \noscript
%     \subscript     {\blob{sub 1}}
%     \noscript
%     \superscript   {\blob{sup 2}}
% }
% \stoplinecorrection
