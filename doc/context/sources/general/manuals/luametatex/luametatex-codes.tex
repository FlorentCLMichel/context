\environment luametatex-style

\startcomponent luametatex-codes

\startchapter[title=Primitive codes]

here follows a list with all primitives and their category is shown. When the
engine starts up in ini mode all primitives get defined along with some
properties that makes it possible to do a reverse lookup of a combination of
command code and char code. But, a primitive, being also a regular command can be
redefined later on. The table below shows the original pairs but in \CONTEXT\
some of these primitives are redefined. However, any macro that fits a command
and char pair is (reported as) a primitive in logs and error messages. In the end
all tokens are such a combination, The first 16 command codes are reserved for
characters (the whole \UNICODE\ range can be used as char code) with specific
catcodes and not mentioned in the list.

\ctxlua{document.allprimitives()}

\stopchapter

\stopcomponent
