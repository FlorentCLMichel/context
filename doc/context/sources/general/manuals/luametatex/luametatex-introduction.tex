% language=uk

\environment luametatex-style

\startcomponent luametatex-introduction

\startchapter[title=Introduction]

Around 2005 we started the \LUATEX\ projects and it took about a decade to reach
a state where we could consider the experiments to have reached a stable state.
Already for a while one could use \LUATEX\ in production but some of the
interfaces evolved. In 2018 the functionality was more or less frozen. Of course
we might add some features in due time but nothing fundamental will change as we
consider version 1.10 to be reasonable feature complete. Among the reasons is
that this engine is now used outside \CONTEXT\ too which means that we cannot
simply change much without affecting other macro packages.

However, in reaching that state some decisions were delayed because they didn't
go well with a current stable version. This is why at the 2018 \CONTEXT\ meeting
those present agreed that we could move on with a follow up tagged \METATEX, a
name we already had in mind for a while, but as \LUA\ is an important component,
it got expanded to \LUAMETATEX. This follow up is a lightweight companion to
\LUATEX\ that will be maintained alongside. More about the reasons for this
follow up as well as the philosophy behind it can be found on the document(s)
describing the development. During \LUATEX\ development I kept track of what
happened in a series of documents, parts of which were published as articles in
user group journals, but all are in the \CONTEXT\ distribution. I did the same
with the development of \LUAMETATEX.

The \LUAMETATEX\ engine is, as said, a lightweight version of \LUATEX, that for
now targets \CONTEXT. We will use it for possibly drastic experiments but without
affecting \LUATEX. As we can easily adapt \CONTEXT\ to support both, no other
macro package will be harmed when (for instance) interfaces change as part of an
experiment. Of course, when we consider something to be useful, it can be back
ported to \LUATEX, but only when there are good reasons for doing so. When
considering this follow up one consideration was that a lean and mean version
with an extension mechanism is a bit closer to original \TEX. Of course, because
we have new primitives, this is not entirely true.

This manual currently has quite a bit of overlap with the \LUATEX\ manual but
some chapters are removed, others added and the rest has been adapted. We also
discusses the (main) differences. Some of the new primitives or functions that
show up in \LUAMETATEX\ might show up in \LUATEX\ at some point, others might
not. For now it is an experimental engine in which we can change things at will
but with \CONTEXT\ in tandem so it will keep working.

For \CONTEXT\ users the \LUAMETATEX\ engine will become the default. Because we
can keep both \LUAMETATEX\ and \CONTEXT\ in sync. The \CONTEXT\ variant is tagged
\LMTX. The pair can be used in production, just as with \LUATEX\ and \MKIV. In
fact, most users will probably not really notice the difference.

As this follow up is closely related to \CONTEXT\ development, and because we
expect stock \LUATEX\ to be used outside the \CONTEXT\ proper, there will be no
special mailing list nor coverage (or polution) on the \LUATEX\ related mailing
lists. We have the \CONTEXT\ mailing lists for that. In due time the source code
will be part of the regular \CONTEXT\ distribution.

% \testpage[8]

This manual refers to \LUATEX, when we talk of features common to both engine, as
well as \LUAMETATEX, when it is more specific to the follow up.

\blank[big]

Hans Hagen

% \blank[2*big]
\page

\starttabulate[|||]
\NC Version       \EQ \currentdate \NC \NR
\NC \LUAMETATEX   \EQ \cldcontext{LUATEXENGINE} %
                      \cldcontext{LUATEXVERSION} / %
                      \cldcontext{LUATEXFUNCTIONALITY}
                      \NC \NR
\NC \CONTEXT      \EQ MkIV \contextversion \NC \NR
\NC \LUATEX\ Team \EQ Hans Hagen, Hartmut Henkel, Taco Hoekwater, Luigi Scarso \NC \NR
\stoptabulate

\stopchapter

\stopcomponent
