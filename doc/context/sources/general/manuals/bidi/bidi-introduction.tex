% language=us runpath=texruns:manuals/bidi

\startcomponent bidi-introduction

\environment bidi-style

\startchapter[title=Introduction]

With \CONTEXT\ you can typeset in two directions: from left to right and from
right to left. In fact you can also combine these two directions, like this:

\startbuffer
There are many {\righttoleft \maincolor \it scripts in use} and some run into the
other direction. However, there is {\righttoleft \maincolor \it no fixed relation
{\lefttoright \black \it between the} direction of the script} and cars being
driven left or right of the road.
\stopbuffer

\typebuffer

\getbuffer

Even someone not familiar with right to left typesetting can see what happens
here, or not? In fact Luigi Scarso pointed out that the \type {fixed} reversed
into {\righttoleft \type {fixed}} but not in the example where {\bf fixed}
becomes {\righttoleft \bf fixed}. This signals an important property of the way
the text gets processed: you input something, at some points font features get
applied (like ligatures) and in the end the resulting glyph stream is reversed.
By that time the combination of {\bf f}+{\bf i} has become {\bf fi}! So, be
prepared for surprises.

This manual is written by a left to right user so don't expect a manual on
semitic typesetting. Also don't expect a (yet) complete manual. I'll add whatever
comes to mind. This is not a manual about Hebrew or Arabic, if only because I
can't read any of those scripts (languages). I leave that to others to cover.

This is work in progress and might always be! So expect errors and typos. As with
anything related to typesetting the truth about how it should be done and what
looks best is not absolute. So, the most we can offer is flexibility and the way
\CONTEXT\ is setup permits that.

Of course this is not possible without input. When we moved to \CONTEXT\ \LMTX,
the bidi thread was picked up by Mohammad Hossein Bateni, Idris Samawi Hamid,
Wolfgang Schuster and myself. So, expect more!

\startlines
Hans Hagen
Hasselt, NL
\stoplines

\stopchapter

\stopcomponent
