% language=us runpath=texruns:manuals/math
\environment math-layout

\startcomponent math-subtoptimal

\startchapter[title=Suboptimal]

\startsection[title=Extensibles]

Extensibles are implemented as follows: we start with the default shape, and when
that doesn't cover the body of text, a next size is chosen. When we run out of
sizes, a glyph is made from snippets (often a start glyph, overlapping middle
pieces and an end piece. Of course a font needs to provide these variants and
snippets.

However, the quality of the coverage can differ per font. Here we show how Latin
Modern, Pagella, Cambria, Lucida and Dejavu look like:

\starttexdefinition ShowSample #1#2
    \start
        \showglyphs
        \switchtobodyfont [#1]
        \dontleavehmode#2: \dorecurse{50}{$\vec{\blackrule[width=##1pt]}$\space}\unskip
        \par
    \stop
\stoptexdefinition

\ShowSample{modern}  {Latin Modern} \blank
\ShowSample{pagella} {Pagella}      \blank
\ShowSample{cambria} {Cambria}      \blank
\ShowSample{lucidaot}{Lucida}       \blank
\ShowSample{dejavu}  {Dejavu}

Of course fonts can be improved (or patched) and these samples might come out
better compared to previous renderings.

\stopsection

\stopchapter

\stopcomponent
