% language=us

\environment lowlevel-style

\startdocument
  [title=paragraphs,
   color=middlecyan]

\startsection[title=Introduction]

This manual is mostly discussing a few low level wrappers around low level \TEX\
features. Its writing is triggered by an update to the \METAFUN\ and \LUAMETAFUN\
manuals where we mess a bit with shapes. It gave a good reason to also cover some
more paragraph related topics but it might take a while to complete. Remind me if
you feel that takes too much time.

Because paragraphs and their construction are rather central to \TEX, you can
imagine that the engine exposes dealing with them. This happens via commands
(primitives) but only when it's robust. Then there are callbacks, and some
provide detailed information about what we're dealing with. However, intercepting
node lists can already be hairy and we do that a lot in \CONTEXT. Intercepting
and tweaking paragraph properties is even more tricky, which is why we try to
avoid that in the core. But \unknown\ in the following sections you will see that
there are actually a couple of mechanism that do so. Often new features like this
are built in stepwise and enabled locally for a while and when they seem okay
they get enabled by default. \footnote {For this we have \type
{\enableexperiments} which one can use in \type {cont-loc.mkxl} or \type
{cont-exp.mkxl}, files that are loaded runtime when on the system. When you use
them, make sure they don't interfere; they are not part of the updates, contrary
to \type {cont-new.mkxl}.}

\stopsection

\startsection[title=Paragraphs]

Before we demonstrate some trickery, let's see what a paragraph is. Normally a
document source is formatted like this:

\starttyping[option=TEX]
some text (line 1)
some text (line 2)

some more test (line 1)
some more test (line 2)
\stoptyping

There are two blocks of text here separated by an empty line and they become two
paragraphs. Unless configured otherwise an empty line is an indication that we
end a paragraph. You can also explicitly do that:

\starttyping[option=TEX]
some text (line 1)
some text (line 2)
\par
some more test (line 1)
some more test (line 2)
\stoptyping

When \TEX\ starts a paragraph, it actually also does something think of:

\starttyping[option=TEX]
[\the\everypar]some text      (line 1) some text      (line 2) \par
[\the\everypar]some more test (line 1) some more test (line 2) \par
\stoptyping

or more accurate:

\starttyping[option=TEX]
[\the\everypar]some text      some text      \par
[\the\everypar]some more test some more test \par
\stoptyping

because the end|-|of|-|line character has become a space. As mentioned,
an empty line is actually the end of a paragraph. But in \LUAMETATEX\
we can cheat a bit. If we have this:

\startbuffer
line 1

line 2
\stopbuffer

\typebuffer[option=TEX]

We can do this (watch how we need to permit overloading a primitive when we have
enabled \type {\overloadmode}):

\startbuffer
\pushoverloadmode
\def\linepar{\removeunwantedspaces !\ignorespaces}
\popoverloadmode
line 1

line 2
\stopbuffer

\typebuffer[option=TEX]

This comes out as:

\start \getbuffer \stop

I admit that since it got added (as part of some cleanup halfway the overhaul of
the engine) I never saw a reason to use it, but it is a cheap feature. The \type
{\linepar} primitive is undefined (\type {\undefined}) by default so no user sees
it anyway. Just don't use it unless maybe for some pseudo database trickery (I
considered using it for the database module but it is not needed). In a similar
fashion, just don't redefine \type {\par}: it's asking for troubles and \quote
{not done} in \CONTEXT\ anyway.

Back to reality. In \LUATEX\ we get a node list that starts with a so called
\type {localpar} node and ends with a \type {\parfillskip}. The first node is
prepended automatically. That list travels through the system: hyphenation,
applying font properties, break the effectively one line into lines, wrap them
and add them to a vertical list, etc. Each stage can be intercepted via
callbacks.

When the paragraph is broken into lines hanging indentation or a so called par
shape can be applied, and we will see more of that later, here we talk \type
{\par} and show another \LUAMETATEX\ trick:

\startbuffer
\def\foo{{\bf test:} \ignorepars}

\foo

line
\stopbuffer

\typebuffer[option=TEX]

The macro typesets some text and then skips to the next paragraph:

\start \getbuffer \stop

Think of this primitive as being a more powerful variant of \type
{\ignorespaces}. This leaves one aspect: how do we start a paragraph. Technically
we need to force \TEX\ into so called horizontal mode. When you look at plain
\TEX\ documents you will notice commands like \type {\noindent} and \type
{\indent}. In \CONTEXT\ we have more high level variants, for instance we have
\type {\noindentation}.

A robust way to make sure that you get in horizontal mode is using \type
{\dontleavehmode} which is a wink to \type {\leavevmode}, a command that you
should never use in \CONTEXT, so when you come from plain or \LATEX, it's one of
the commands you should wipe from your memory.

When \TEX\ starts with a paragraph the \type {\everypar} token list is expanded
and again this is a primitive you should not mess with yourself unless in very
controlled situations. If you change its content, you're on your own with respect
to interferences and side effects.

One of the things that \TEX\ does in injecting the indentation. Even when there
is none, it gets added, not as skip but as an empty horizontal box of a certain
width. This is easier on the engine when it constructs the paragraph from the one
liner: starting with a skip demands a bit more testing in the process (a nice
trick so to say). However, in \CONTEXT\ we enable the \LUAMETATEX\ feature that
does use a skip instead of a box. It's part of the normalization that is
discussed later. Instead of checking for a box with property indent, we check for
a skip with such property. This is often easier and cleaner.

A bit off topic is the fact that in traditional \TEX\ empty lines or \type {\par}
primitives can trigger an error. This has to do with the fact that the program
evolved in a time where paper terminals were used and runtime could be excessive.
So, in order to catch a possible missing brace, a concept of \type {\long}
macros, permitting \type {\par} or equivalents in arguments, was introduced as
well as not permitting them in for instance display math. In \CONTEXT\ \MKII\
most macros that could be sensitive for this were defined as \type {\long} so
that users never had to bother about it and probably were not even aware of it.
Right from the start in \LUATEX\ these error|-|triggers could be disabled which
of course we enable in \CONTEXT\ and in \LUAMETATEX\ these features have been
removed altogether. I don't think users will complain about this.

If you want to enforce a newline but not a new paragraph you can use the \type
{\crlf} command. When used on its own it will produce an empty line. Don't use
this to create whitespace between lines.

If you want to do something after so called par tokens are seen you can do this:

\startbuffer
\def\foo{{\bf >>>> }}
\expandafterpars\foo

this is a new paragraph ...

\expandafterpars\foo
\par\par\par\par
this is a new paragraph ...
\stopbuffer

\typebuffer[option=TEX]

This not to be confused with \type {\everypar} which is a token list that \TEX\
itself injects before each paragraph (also nested ones).

\getbuffer

This is typically a primitive that will only be used in macros. You can actually
program it using macros: pickup a token, check and push it back when it's not a
par equivalent token. The primitive is is just nicer (and easier on the log when
tracing is enabled).

\stopsection

\startsection[title=Properties]

A paragraph is just a collection of lines that result from one input line that
got broken. This process of breaking into lines is influenced by quite some
parameters. In traditional \TEX\ and also in \LUAMETATEX\ by default the values
that are in effect when the end of the paragraph is met are used. So, when you
change them in a group and then ends the paragraph after the group, the values
you've set in the group are not used.

However, in \LUAMETATEX\ we can optionally store them with the paragraph. When
that happens the values current at the start are frozen. You can still overload
them but that has to be done explicitly then. The advantage is that grouping no
longer interferes with the line break algorithm. The magic primitive is \type
{\snapshotpar} which takes a number made from categories mentioned below:

\starttabulate[|l|l|r|]
\BC variable                       \BC category        \BC code                                     \NC \NR
\NC \type {\hsize}                 \NC hsize           \NC 0x\uchexnumbers\frozenhsizecode          \NC \NR
\NC \type {\leftskip}              \NC skip            \NC 0x\uchexnumbers\frozenskipcode           \NC \NR
\NC \type {\rightskip}             \NC skip            \NC 0x\uchexnumbers\frozenskipcode           \NC \NR
\NC \type {\hangindent}            \NC hang            \NC 0x\uchexnumbers\frozenhangcode           \NC \NR
\NC \type {\hangafter}             \NC hang            \NC 0x\uchexnumbers\frozenhangcode           \NC \NR
\NC \type {\parindent}             \NC indent          \NC 0x\uchexnumbers\frozenindentcode         \NC \NR
\NC \type {\parfillleftskip}       \NC par fill        \NC 0x\uchexnumbers\frozenparfillcode        \NC \NR
\NC \type {\parfillrightskip}      \NC par fill        \NC 0x\uchexnumbers\frozenparfillcode        \NC \NR
\NC \type {\adjustspacing}         \NC adjust          \NC 0x\uchexnumbers\frozenadjustcode         \NC \NR
\NC \type {\adjustspacingstep}     \NC adjust          \NC 0x\uchexnumbers\frozenadjustcode         \NC \NR
\NC \type {\adjustspacingshrink}   \NC adjust          \NC 0x\uchexnumbers\frozenadjustcode         \NC \NR
\NC \type {\adjustspacingstretch}  \NC adjust          \NC 0x\uchexnumbers\frozenadjustcode         \NC \NR
\NC \type {\protrudechars}         \NC protrude        \NC 0x\uchexnumbers\frozenprotrudecode       \NC \NR
\NC \type {\pretolerance}          \NC tolerance       \NC 0x\uchexnumbers\frozentolerancecode      \NC \NR
\NC \type {\tolerance}             \NC tolerance       \NC 0x\uchexnumbers\frozentolerancecode      \NC \NR
\NC \type {\emergencystretch}      \NC stretch         \NC 0x\uchexnumbers\frozenstretchcode        \NC \NR
\NC \type {\looseness}             \NC looseness       \NC 0x\uchexnumbers\frozenloosenesscode      \NC \NR
\NC \type {\lastlinefit}           \NC last line       \NC 0x\uchexnumbers\frozenlastlinecode       \NC \NR
\NC \type {\linepenalty}           \NC line penalty    \NC 0x\uchexnumbers\frozenlinepenaltycode    \NC \NR
\NC \type {\interlinepenalty}      \NC line penalty    \NC 0x\uchexnumbers\frozenlinepenaltycode    \NC \NR
\NC \type {\interlinepenalties}    \NC line penalty    \NC 0x\uchexnumbers\frozenlinepenaltycode    \NC \NR
\NC \type {\clubpenalty}           \NC club penalty    \NC 0x\uchexnumbers\frozenclubpenaltycode    \NC \NR
\NC \type {\clubpenalties}         \NC club penalty    \NC 0x\uchexnumbers\frozenclubpenaltycode    \NC \NR
\NC \type {\widowpenalty}          \NC widow penalty   \NC 0x\uchexnumbers\frozenwidowpenaltycode   \NC \NR
\NC \type {\widowpenalties}        \NC widow penalty   \NC 0x\uchexnumbers\frozenwidowpenaltycode   \NC \NR
\NC \type {\displaywidowpenalty}   \NC display penalty \NC 0x\uchexnumbers\frozendisplaypenaltycode \NC \NR
\NC \type {\displaywidowpenalties} \NC display penalty \NC 0x\uchexnumbers\frozendisplaypenaltycode \NC \NR
\NC \type {\brokenpenalty}         \NC broken penalty  \NC 0x\uchexnumbers\frozenbrokenpenaltycode  \NC \NR
\NC \type {\adjdemerits}           \NC demerits        \NC 0x\uchexnumbers\frozendemeritscode       \NC \NR
\NC \type {\doublehyphendemerits}  \NC demerits        \NC 0x\uchexnumbers\frozendemeritscode       \NC \NR
\NC \type {\finalhyphendemerits}   \NC demerits        \NC 0x\uchexnumbers\frozendemeritscode       \NC \NR
\NC \type {\parshape}              \NC shape           \NC 0x\uchexnumbers\frozenshapecode          \NC \NR
\NC \type {\baselineskip}          \NC line            \NC 0x\uchexnumbers\frozenlinecode           \NC \NR
\NC \type {\lineskip}              \NC line            \NC 0x\uchexnumbers\frozenlinecode           \NC \NR
\NC \type {\lineskiplimit}         \NC line            \NC 0x\uchexnumbers\frozenlinecode           \NC \NR
\NC \type {\hyphenationmode}       \NC hyphenation     \NC 0x\uchexnumbers\frozenhyphenationcode    \NC \NR
\stoptabulate


As you can see here, there are more paragraph related parameters than in for
instance \PDFTEX\ and \LUATEX\ and these are (to be) explained in the
\LUAMETATEX\ manual. You can imagine that keeping this around with the paragraph
adds some extra overhead to the machinery but most users won't notice that
because is is compensated by gains elsewhere.

This is pretty low level and there are a bunch of helpers that support this but
these are not really user level macros. As with everything \TEX\ you can mess
around as much as you like, and the code gives plenty of examples but when you do
this, you're on your own because it can interfere with \CONTEXT\ core
functionality.

In \LMTX\ taking these snapshots is turned on by default and because it thereby
fundamentally influences the par builder, users can run into compatibility issues
but in practice there has been no complaints (and this feature has been in use
quite a while before this document was written). One reason for users not
noticing is that one of the big benefits is probably handled by tricks mentioned
on the mailing list. Imagine that you have this:

\starttyping[option=TEX]
{\bf watch out:} here is some text
\stoptyping

In this small example the result will be as expected. But what if something magic
with the start of a paragraph is done? Like this:

\starttyping[option=TEX]
\placefigure[left]{A cow!}{\externalfigure[cow.pdf]}

{\bf watch out:} here is some text ... of course much more is needed to
    get a flow around the figure!
\stoptyping

The figure will hang at the left side of the paragraph but it is put there when
the text starts and that happens inside the bold group. It means that the
properties we set in order to get the shape around the figure are lost as soon as
we're at \quote{\type {here is some text}} and definitely is wrong when the
paragraph ends and the par builder has to use them to get the shape right. We get
text overlapping the figure. A trick to overcome this is:

\starttyping[option=TEX]
\dontleavehmode {\bf watch out:} here is some text ... of course much
    more is needed to get a flow around the figure!
\stoptyping

where the first macro makes sure we already start a paragraph before the group is
entered (using a \type {\strut} also works). It's not nice and I bet users have
been bitten by this and by now know the tricks. But, with snapshots such fuzzy
hacks are not needed any more! The same is true with this:

\starttyping[option=TEX]
{\leftskip 1em some text \par}
\stoptyping

where we had to explicitly end the paragraph inside the group in order to retain
the skip. I suppose that users normally use the high level environments so they
never had to worry about this. It's also why users probably won't notice that
this new mechanism has been active for a while. Actually, when you now change a
parameter inside the paragraph its new value will not be applied (unless you
prefix it with \type {\frozen} or snapshot it) but no one did that anyway.

\stopsection

\startsection[title=Wrapping up]

In \CONTEXT\ \LMTX\ we have a mechanism to exercise macros (or content) before a
paragraph ends. This is implemented using the \type {\wrapuppar} primitive. The
to be wrapped up material is bound to the current paragraph which in order to
get this done has to be started when this primitive is used.

Although the high level interface has been around for a while it still needs a
bit more testing (read: use cases are needed). In the few cases where we already
use it application can be different because again it relates to snapshots. This
because in the past we had to use tricks that also influenced the user interface
of some macros (which made them less natural as one would expect). So the
question is: where do we apply it in old mechanisms and where not.

{\em todo: accumulation, interference, where applied, limitations}

% \vbox   {vbox    : \wrapuppar{1}test\par x\wrapuppar{2}test}\blank
% \vtop   {vtop    : \wrapuppar{1}test\par x\wrapuppar{2}test}\blank
% \vcenter{vcenter : \wrapuppar{1}test\par x\wrapuppar{2}test}\blank
% $$x = \vcenter{vcenter : \wrapuppar{1}test\par x\wrapuppar{2}test}$$\blank
% x\vadjust{vadjust : \wrapuppar{1}test\par x\wrapuppar{2}test}x\blank

\stopsection

\startsection[title=Hanging]

There are two mechanisms for getting a specific paragraph shape: rectangular
hanging and arbitrary shapes. Both mechanisms work top|-|down. The first
mechanism uses a combination of \type {\hangafter} and \type {\hangindent}, and
the second one depends on \type {\parshape}. In this section we discuss the
rectangular one.

\startbuffer[demo-5]
\hangafter  4 \hangindent  4cm \samplefile{tufte} \page
\hangafter -4 \hangindent  4cm \samplefile{tufte} \page
\hangafter  4 \hangindent -4cm \samplefile{tufte} \page
\hangafter -4 \hangindent -4cm \samplefile{tufte} \page
\stopbuffer

\typebuffer[demo-5][option=TEX]

As you can see in \in {figure} [fig:hang], the four cases are driven by the sign
of the values. If you want to hang into the margin you need to use different
tricks, like messing with the \type {\leftskip}, \type {\rightskip} or \type
{\parindent} parameters (which then of course can interfere with other mechanisms
uses at the same time).

\startplacefigure[title=Hanging indentation,reference=fig:hang]
\startcombination[nx=2,ny=2]
    {\typesetbuffer[demo-5][page=1,width=.4\textwidth,frame=on]} {\type{\hangafter +4 \hangindent +4cm}}
    {\typesetbuffer[demo-5][page=2,width=.4\textwidth,frame=on]} {\type{\hangafter -4 \hangindent +4cm}}
    {\typesetbuffer[demo-5][page=3,width=.4\textwidth,frame=on]} {\type{\hangafter +4 \hangindent -4cm}}
    {\typesetbuffer[demo-5][page=4,width=.4\textwidth,frame=on]} {\type{\hangafter -4 \hangindent -4cm}}
\stopcombination
\stopplacefigure

\stopsection

\startsection[title=Shapes]

In \CONTEXT\ we don't use \type {\parshape} a lot. It is used in for instance
side floats but even there not in all cases. It's more meant for special
applications. This means that in \MKII\ and \MKIV\ we don't have some high level
interface. However, when \METAFUN\ got upgraded to \LUAMETAFUN, and the manual
also needed an update, one of the examples in that manual that used shapes also
got done differently (read: nicer). And that triggered the arrival of a new low
level shape mechanism.

One important property of the \type {\parshape} mechanism is that it works per
paragraph. You define a shape in terms of a left margin and width of a line. The
shape has a fixed number of such pairs and when there is more content, the last
one is used for the rest of the lines. When the paragraph is finished, the shape
is forgotten. \footnote {Not discussed here is a variant that might end up in
\LUAMETATEX\ that works with the progression, i.e.\ takes the height of the
content so far into account. This is somewhat tricky because for that to work
vertical skips need to be frozen, which is no real big deal but has to be done
careful in the code.}

The high level interface is a follow up on the example in the \METAFUN\ manual and
uses shapes that carry over to the next paragraph. In addition we can cycle over
a shape. In this interface shapes are defined using keyword. Here are some
examples:

\starttyping[option=TEX]
\startparagraphshape[test]
    left 1mm right 1mm
    left 5mm right 5mm
\stopparagraphshape
\stoptyping

This shape has only two entries so the first line will have a 1mm margin while
later lines will get 5mm margins. This translates into a \type {\parshape} like:

\starttyping[option=TEX]
\parshape 2
    1mm \dimexpr\hsize-1mm\relax
    5mm \dimexpr\hsize-5mm\relax
\stoptyping

Watch the number \type {2}: it tells how many specification lines follow. As you
see, we need to calculate the width.

\starttyping[option=TEX]
\startparagraphshape[test]
    left 1mm right 1mm
    left 5mm right 5mm
    repeat
\stopparagraphshape
\stoptyping

This variant will alternate between 1mm and 5mm margins. The repeating feature is
translated as follows. Maybe at some point I will introduce a few more options.

\starttyping[option=TEX]
\parshape 2 options 1
    1mm \dimexpr\hsize-1mm\relax
    5mm \dimexpr\hsize-5mm\relax
\stoptyping

A shape can have some repetition, and we can save keystrokes by copying the last
entry. The resulting \type {\parshape} becomes rather long.

\starttyping[option=TEX]
\startparagraphshape[test]
    left 1mm right 1mm
    left 2mm right 2mm
    left 3mm right 3mm
    copy 8
    left 4mm right 4mm
    left 5mm right 5mm
    left 5mm hsize 10cm
\stopparagraphshape
\stoptyping

Also watch the \type {hsize} keyword: we don't calculate the hsize from the \type
{left} and \type {right} values but explicitly set it.

\starttyping[option=TEX]
\startparagraphshape[test]
    left 1mm right 1mm
    right 3mm
    left 5mm right 5mm
    repeat
\stopparagraphshape
\stoptyping

When a \type {right} keywords comes first the \type {left} is assumed to be zero.
In the examples that follow we will use a couple of definitions:

\startbuffer[setup-0]
\startparagraphshape[test]
    both 1mm both 2mm both 3mm both 4mm both 5mm both 6mm
    both 7mm both 6mm both 5mm both 4mm both 3mm both 2mm
\stopparagraphshape
\stopbuffer

\startbuffer[setup-0-repeat]
\startparagraphshape[test-repeat]
    both 1mm both 2mm both 3mm both 4mm both 5mm both 6mm
    both 7mm both 6mm both 5mm both 4mm both 3mm both 2mm
    repeat
\stopparagraphshape
\stopbuffer

\typebuffer[setup-0,setup-0-repeat][option=TEX]

The last one could also be defines as:

\starttyping[option=TEX]
\startparagraphshape[test-repeat]
    \rawparagraphshape{test} repeat
\stopparagraphshape
\stoptyping

In the previous code we already introduced the \type {repeat} option. This will
make the shape repeat at the engine level when the shape runs out of specified
lines. In the application of a shape definition we can specify a \type {method}
to be used and that determine if the next paragraph will start where we left off
and discard afterwards (\type {shift}) or that we move the discarded lines up
front so that we never run out of lines (\type {cycle}). It sounds complicated
but just keep in mind that \type {repeat} is part of the \type {\parshape} and
act within a paragraph while \type {shift} and \type {cycle} are applied when a
new paragraph is started.

\startbuffer[demo-1]
\startshapedparagraph[list=test]
    \dorecurse{8}{\showparagraphshape\samplefile{tufte} \par}
\stopshapedparagraph
\stopbuffer

\startbuffer[demo-1-repeat]
\startshapedparagraph[list=test-repeat]
    \dorecurse{8}{\showparagraphshape\samplefile{tufte} \par}
\stopshapedparagraph
\stopbuffer

In \in {figure} [fig:shape:discard] you see the following applied:

\typebuffer[demo-1,demo-1-repeat][option=TEX]

\startplacefigure[title=Discarded shaping,reference=fig:shape:discard]
\startcombination[nx=2,ny=2]
    {\typesetbuffer[setup-0,demo-1]       [page=1,width=.4\textwidth,frame=on]} {discard, finite shape,    page 1}
    {\typesetbuffer[setup-0,demo-1]       [page=2,width=.4\textwidth,frame=on]} {discard, finite shape,    page 2}
    {\typesetbuffer[setup-0,demo-1-repeat][page=1,width=.4\textwidth,frame=on]} {discard, repeat in shape, page 1}
    {\typesetbuffer[setup-0,demo-1-repeat][page=2,width=.4\textwidth,frame=on]} {discard, repeat in shape, page 2}
\stopcombination
\stopplacefigure

In \in {figure} [fig:shape:shift] we use this instead:

\startbuffer[demo-2]
\startshapedparagraph[list=test,method=shift]
    \dorecurse{8}{\showparagraphshape\samplefile{tufte} \par}
\stopshapedparagraph
\stopbuffer

\startbuffer[demo-2-shift]
\startshapedparagraph[list=test-repeat,method=shift]
    \dorecurse{8}{\showparagraphshape\samplefile{tufte} \par}
\stopshapedparagraph
\stopbuffer

\typebuffer[demo-2,demo-2-repeat][option=TEX]

\startplacefigure[title=Shifted shaping,reference=fig:shape:shift]
\startcombination[nx=2,ny=2]
    {\typesetbuffer[setup-0,demo-2][page=1,width=.4\textwidth,frame=on]}              {shift, finite shape,    page 1}
    {\typesetbuffer[setup-0,demo-2][page=2,width=.4\textwidth,frame=on]}              {shift, finite shape,    page 2}
    {\typesetbuffer[setup-0-repeat,demo-2-shift][page=1,width=.4\textwidth,frame=on]} {shift, repeat in shape, page 1}
    {\typesetbuffer[setup-0-repeat,demo-2-shift][page=2,width=.4\textwidth,frame=on]} {shift, repeat in shape, page 2}
\stopcombination
\stopplacefigure

Finally, in \in {figure} [fig:shape:cycle] we use:

\startbuffer[demo-3]
\startshapedparagraph[list=test,method=cycle]
    \dorecurse{8}{\showparagraphshape\samplefile{tufte} \par}
\stopshapedparagraph
\stopbuffer

\startbuffer[demo-3-cycle]
\startshapedparagraph[list=test-repeat,method=cycle]
    \dorecurse{8}{\showparagraphshape\samplefile{tufte} \par}
\stopshapedparagraph
\stopbuffer

\typebuffer[demo-3,demo-3-repeat][option=TEX]

\startplacefigure[title=Cycled shaping,reference=fig:shape:cycle]
\startcombination[nx=2,ny=2]
    {\typesetbuffer[setup-0,demo-3][page=1,width=.4\textwidth,frame=on]}              {cycle, finite shape,    page 1}
    {\typesetbuffer[setup-0,demo-3][page=2,width=.4\textwidth,frame=on]}              {cycle, finite shape,    page 2}
    {\typesetbuffer[setup-0-repeat,demo-3-cycle][page=1,width=.4\textwidth,frame=on]} {cycle, repeat in shape, page 1}
    {\typesetbuffer[setup-0-repeat,demo-3-cycle][page=2,width=.4\textwidth,frame=on]} {cycle, repeat in shape, page 2}
\stopcombination
\stopplacefigure

These examples are probably too small to see the details but you can run them
yourself or zoom in on the details. In the margin we show the values used. Here
is a simple example of (non) poetry. There are other environments that can be
used instead but this makes a good example anyway.

\startbuffer
\startparagraphshape[test]
    left 0em right 0em
    left 1em right 0em
    repeat
\stopparagraphshape

\startshapedparagraph[list=test,method=cycle]
    verse line 1.1\crlf verse line 2.1\crlf
    verse line 3.1\crlf verse line 4.1\par
    verse line 1.2\crlf verse line 2.2\crlf
    verse line 3.2\crlf verse line 4.2\crlf
    verse line 5.2\crlf verse line 6.2\par
\stopshapedparagraph
\stopbuffer

\typebuffer[option=TEX]

\start \getbuffer \stop

Because the idea for this feature originates in \METAFUN, we will now kick in
some \METAPOST. The following code creates a shape for a circle. We use a
2mm offset here:

\startbuffer
\startuseMPgraphic{circle}
    path p ; p := fullcircle scaled TextWidth ;
    build_parshape(p,
        2mm, 0, 0,
        LineHeight, StrutHeight, StrutDepth, StrutHeight
    ) ;
\stopuseMPgraphic
\stopbuffer

\typebuffer[option=TEX]

\start \getbuffer \stop

We plug this into the already described macros:

\startbuffer
\startshapedparagraph[mp=circle]%
    \setupalign[verytolerant,stretch,last]%
    \samplefile{tufte}
    \samplefile{tufte}
\stopshapedparagraph
\stopbuffer

\typebuffer[option=TEX]

And get ourself a circular shape. Watch out, at this moment the shape environment
does not add grouping so when as in this case you change the alignment it can
influence the document.

\start \getbuffer \stop

\startbuffer[framed]
\framed[align=normal,width=\textwidth,offset=2mm,strut=no]\bgroup
    \getbuffer
\egroup
\stopbuffer

Assuming that the shape definition above is in a buffer we can do this:

\typebuffer[option=TEX]

The result is shown in \in {figure} [fig:shape:circle]. Because all action
happens in the framed environment, we can also use this definition:

\starttyping[option=TEX]
\startuseMPgraphic{circle}
    path p ; p := fullcircle scaled \the\dimexpr\framedwidth+\framedoffset*2\relax ;
    build_parshape(p,
        \framedoffset, 0, 0,
        LineHeight, StrutHeight, StrutDepth, StrutHeight
    ) ;
    draw p ;
\stopuseMPgraphic
\stoptyping

\startplacefigure[title=A framed circular shape,reference=fig:shape:circle]
    \getbuffer[framed]
\stopplacefigure

A mechanism like this is often never completely automatic in the sense that you
need to keep an eye on the results. Depending on user demands more features can
be added. With weird shapes you might want to set up the alignment to be \type
{tolerant} and have some \type {stretch}.

The interface described in the \METAFUN\ manual is pretty old, the time stamp of
the original code is mid 2000, but the principles didn't change. The examples in
\type {meta-imp-txt.mkxl} can now be written as:

\startuseMPgraphic{test 1}
  begingroup ;
    save p ; path p ; p := fullcircle scaled 6cm ;
    lmt_parshape [
        path        = p,
        offset      = BodyFontSize/2,
        dx          = 0,           % default
        dy          = 0,           % default
        lineheight  = LineHeight,  % default
        strutheight = StrutHeight, % default
        strutdepth  = StrutDepth,  % default
        topskip     = StrutHeight, % default
    ] ;
    draw p withpen pencircle scaled 1pt ;
  endgroup ;
\stopuseMPgraphic

\startuseMPgraphic{test 2}
  begingroup ;
    save p ; path p ; p := fullsquare rotated 45 scaled 5cm ;
    lmt_parshape [
        path   = p,
        offset = BodyFontSize/2,
        trace  = true,
    ] ;
    draw p withpen pencircle scaled 1pt ;
  endgroup ;
\stopuseMPgraphic

\startuseMPgraphic{test 3}
  begingroup ;
    save w, h, p ; path p ; w := h := 6cm ;
    p := (.5w,h) -- (   0,  h) -- (0,0) -- (w,0) &
         (  w,0) .. (.75w,.5h) .. (w,h) &  (w,h) -- cycle ;
    lmt_parshape [
        path   = p,
        offset = BodyFontSize/2,
    ] ;
    draw p withpen pencircle scaled 1pt ;
  endgroup ;
\stopuseMPgraphic

\startuseMPgraphic{test 4}
  begingroup ;
    save d, p, q ; path p, q ; d := BodyFontSize/2;
    vardef shape(expr w, h, o) =
        (o,o) -- (w-o,o) & (w-o,o) .. (.75w-o,.5h) ..
        (w-2o,h-o) & (w-2o,h-o) -- (o,h-o) -- cycle
    enddef ;
    p := shape(6cm, 6cm, d) ; q := shape(6cm, 6cm, 0) ;
    lmt_parshape [
        path       = p,
        offsetpath = q,
        dx         = d,
        dy         = d,
        trace      = true,
    ] ;
    draw q withpen pencircle scaled 1pt ;
  endgroup ;
\stopuseMPgraphic

\defineoverlay[test 1][\useMPgraphic{test 1}]
\defineoverlay[test 2][\useMPgraphic{test 2}]
\defineoverlay[test 3][\useMPgraphic{test 3}]
\defineoverlay[test 4][\useMPgraphic{test 4}]

\startbuffer
  \startshapetext[test 1,test 2,test 3,test 4]
    \setupalign[verytolerant,stretch,normal]%
    \samplefile{douglas} % Douglas R. Hofstadter
  \stopshapetext
  \startcombination[2*2]
    {\framed[offset=overlay,frame=off,background=test 1]{\getshapetext}}
        {test 1}
    {\framed[offset=overlay,frame=off,background=test 2]{\getshapetext}}
        {test 2}
    {\framed[offset=overlay,frame=off,background=test 3]{\getshapetext}}
        {test 3}
    {\framed[offset=overlay,frame=off,background=test 4]{\getshapetext}}
        {test 4}
  \stopcombination
\stopbuffer

\typebuffer[option=TEX]

In \in {figure} [fig:shapes:chain] we see the result. Watch how for two shapes
we have enabled tracing. Of course you need to tweak till all fits well but we're
talking of special situations anyway.

\startplacefigure[Title=Multiple shapes,reference=fig:shapes:chain]
    \getbuffer
\stopplacefigure

Here is a bit more extreme example. Again we use a circle:

\startbuffer
\startuseMPgraphic{circle}
    lmt_parshape [
        path       = fullcircle scaled 136mm,
        offset     = 2mm,
        bottomskip = - 1.5LineHeight,
    ] ;
\stopuseMPgraphic
\stopbuffer

\typebuffer[option=TEX]

But we output a longer text:

\startbuffer
\startshapedparagraph[mp=circle,repeat=yes,method=cycle]%
    \setupalign[verytolerant,stretch,last]\dontcomplain
    {\darkred     \samplefile{tufte}}\par
    {\darkgreen   \samplefile{tufte}}\par
    {\darkblue    \samplefile{tufte}}\par
    {\darkcyan    \samplefile{tufte}}\par
    {\darkmagenta \samplefile{tufte}}\par
\stopshapedparagraph
\stopbuffer

\typebuffer[option=TEX]

We get a multi|-|page shape:

\start \getbuffer \stop

Compare this with:

\startbuffer
\startshapedparagraph[mp=circle,repeat=yes,method=cycle]%
    \setupalign[verytolerant,stretch,last]\dontcomplain
    {\darkred     \samplefile{tufte}}
    {\darkgreen   \samplefile{tufte}}
    {\darkblue    \samplefile{tufte}}
    {\darkcyan    \samplefile{tufte}}
    {\darkmagenta \samplefile{tufte}}
\stopshapedparagraph
\stopbuffer

\typebuffer[option=TEX]

Which gives:

\start \getbuffer \stop

Here the \type {bottomskip} takes care of subtle rounding issues as well as
discarding the last line in the shape so that we get nicer continuation. There is
no full automated solution for all you can come up with.

Mixing a \METAPOST\ specification into a regular one is also possible. The next
example demonstrates this as well as the option to remove some lines from a
specification:

\starttyping[option=TEX]
\startparagraphshape[test]
    left 0em right 0em
    left 1em right 0em
    metapost {circle}
    delete 3
    metapost {circle,circle,circle}
    delete 7
    metapost {circle}
    repeat
\stopparagraphshape
\stoptyping

You can combine a shape with narrowing a paragraph. Watch the \type {absolute}
keyword in the next code. The result is shown in \in {figure} [fig:shape:skips].

\startbuffer[demo-4]
\startuseMPgraphic{circle}
    lmt_parshape [
        path       = fullcircle scaled TextWidth,
        bottomskip = - 1.5LineHeight,
    ] ;
\stopuseMPgraphic

\startparagraphshape[test-1]
    metapost {circle} repeat
\stopparagraphshape

\startparagraphshape[test-2]
    absolute left metapost {circle} repeat
\stopparagraphshape

\startparagraphshape[test-3]
    absolute right metapost {circle} repeat
\stopparagraphshape

\startparagraphshape[test-4]
    absolute both metapost {circle} repeat
\stopparagraphshape

\showframe

\startnarrower[4*left,2*right]
    \startshapedparagraph[list=test-1,repeat=yes,method=repeat]%
        \setupalign[verytolerant,stretch,last]\dontcomplain
        \dorecurse{3}{\samplefile{thuan}}
    \stopshapedparagraph
    \page
    \startshapedparagraph[list=test-2,repeat=yes,method=repeat]%
        \setupalign[verytolerant,stretch,last]\dontcomplain
        \dorecurse{3}{\samplefile{thuan}}
    \stopshapedparagraph
    \page
    \startshapedparagraph[list=test-3,repeat=yes,method=repeat]%
        \setupalign[verytolerant,stretch,last]\dontcomplain
        \dorecurse{3}{\samplefile{thuan}}
    \stopshapedparagraph
    \page
    \startshapedparagraph[list=test-4,repeat=yes,method=repeat]%
        \setupalign[verytolerant,stretch,last]\dontcomplain
        \dorecurse{3}{\samplefile{thuan}}
    \stopshapedparagraph
\stopnarrower
\stopbuffer

\typebuffer[demo-4][option=TEX]

\startplacefigure[title=Skip compensation,reference=fig:shape:skips]
\startcombination[nx=2,ny=2]
    {\typesetbuffer[demo-4][page=1,width=.4\textwidth,frame=on]} {test 1}
    {\typesetbuffer[demo-4][page=2,width=.4\textwidth,frame=on]} {test 2, left}
    {\typesetbuffer[demo-4][page=3,width=.4\textwidth,frame=on]} {test 3, right}
    {\typesetbuffer[demo-4][page=4,width=.4\textwidth,frame=on]} {test 4, both}
\stopcombination
\stopplacefigure

The shape mechanism has a few more tricks but these are really meant for usage
in specific situations, where one knows what one deals with. The following
examples are visualized in \in {figure} [fig:flow].

\startbuffer[jano]
\useMPlibrary[dum]
\usemodule[article-basics]

\startbuffer
    \externalfigure[dummy][width=6cm]
\stopbuffer

\startshapedparagraph[text=\getbuffer]
    \dorecurse{3}{\samplefile{ward}\par}
\stopshapedparagraph

\page

\startshapedparagraph[text=\getbuffer,distance=1em]
    \dorecurse{3}{\samplefile{ward}\par}
\stopshapedparagraph

\page

\startshapedparagraph[text=\getbuffer,distance=1em,
        hoffset=-2em]
    \dorecurse{3}{\samplefile{ward}\par}
\stopshapedparagraph

\page

\startshapedparagraph[text=\getbuffer,distance=1em,
        voffset=-2ex,hoffset=-2em]
    \dorecurse{3}{\samplefile{ward}\par}
\stopshapedparagraph

\page

\startshapedparagraph[text=\getbuffer,distance=1em,
        voffset=-2ex,hoffset=-2em,lines=1]
    \dorecurse{3}{\samplefile{ward}\par}
\stopshapedparagraph

\page

\startshapedparagraph[width=4cm,lines=4]
    \dorecurse{3}{\samplefile{ward}\par}
\stopshapedparagraph
\stopbuffer

\typebuffer[jano]

\startplacefigure[title={Flow around something},reference=fig:flow]
    \startcombination[nx=3,ny=2]
        {\typesetbuffer[jano][page=1,frame=on,width=\measure{combination}]}{}
        {\typesetbuffer[jano][page=2,frame=on,width=\measure{combination}]}{}
        {\typesetbuffer[jano][page=3,frame=on,width=\measure{combination}]}{}
        {\typesetbuffer[jano][page=4,frame=on,width=\measure{combination}]}{}
        {\typesetbuffer[jano][page=5,frame=on,width=\measure{combination}]}{}
        {\typesetbuffer[jano][page=6,frame=on,width=\measure{combination}]}{}
    \stopcombination
\stopplacefigure

\stopsection

% \startsection[title=Linebreaks]
\startsection[title=Modes]

% \ruledvbox{1\ifhmode\writestatus{!}{HMODE 1}\fi}                               % hsize
% \ruledvbox{\hbox{\strut 2}\ifhmode\writestatus{!}{HMODE 2}\fi}                 % fit
% \ruledvbox{\hbox{\strut 3}\hbox{\strut 3}\ifhmode\writestatus{!}{HMODE 3}\fi}  % fit
% \ruledvbox{\hbox{\strut 4}4\ifhmode\writestatus{!}{HMODE 4}\fi}                % hsize
% \ruledvbox{\hbox{\strut 5}5\hbox{\strut 5}\ifhmode\writestatus{!}{HMODE 5}\fi} % hsize
% \ruledvbox{6\hbox{\strut 6}\ifhmode\writestatus{!}{HMODE 6}\fi}                % hsize

{\em todo: some of the side effects of so called modes}

\stopsection

\startsection[title=Normalization]

{\em todo: users don't need to bother about this but it might be interesting anyway}

\stopsection

\stopdocument

\everyhbox \everyvbox : useless unless one resets
\parattr
\snapshotpar
\wrapuppar

% \starttext
%     \tracingoutput1 \tracingonline1
%     \pretolerance9000 test \pretolerance8000 test \par
%     \pretolerance9000 test \pretolerance7000 \updateparagraphproperties test \par
%     \pretolerance9000 test \pretolerance6000 \snapshotpar\frozentolerancecode test \par
%     \pretolerance9000 test {\pretolerance5000 \snapshotpar\frozentolerancecode}test \par
% \stoptext

% \par[newgraf][16=1,17=1], .... pretolerance 9000, ....
% \par[newgraf][16=1,17=1], .... pretolerance 7000, ....
% \par[newgraf][16=1,17=1], .... pretolerance 6000, ....

% \parfillleftskip

% I rewarded myself after writing a section by watching the video "Final Thing On
% My Mind", The Pineapple This, Live, 2020, the usual perfect GH performance,
% wondering if live would turn to normal so that we could go to such concerts once
% again given successive covids. Writing manuals can do with a distraction.
%
% Gavin Harrison: Soundcheck, Drummerworld Jan 27, 2021 ... I wish I could make
% something called a check into pefect solo. Okay, another section and I'll check
% out the latest Simon Phillips and other favourite dummer uploads.


