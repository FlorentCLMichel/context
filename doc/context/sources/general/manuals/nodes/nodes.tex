% interface=english modes=screen

% author    : Alan Braslau
% copyright : ConTeXt Development Team
% license   : Creative Commons Attribution ShareAlike 4.0 International
% reference : pragma-ade.nl | contextgarden.net | texlive (related) distributions
% origin    : the ConTeXt distribution
%
% comment   : Because this manual is distributed with TeX distributions it comes with a rather
%             liberal license. We try to adapt these documents to upgrades in the (sub)systems
%             that they describe. Using parts of the content otherwise can therefore conflict
%             with existing functionality and we cannot be held responsible for that. Many of
%             the manuals contain characteristic graphics and personal notes or examples that
%             make no sense when used out-of-context.
%
% comment   : A prototype of the nodes module was presented at the ConTeXt 2016 user meeting
%             and the first release was presented at the 2017 meeting. The module is part of
%             the MetaFun code modules.
%
% comment   : This manual orginates in an article by Alan so anything wrong in here is Hans
%             fault as he converted it.
%
% comment   : The conver images are form the NASA website.

\definemeasure [layout:margin] [\paperheight/20]

\setuplayout
  [topspace=\measure{layout:margin},
   bottomspace=\measure{layout:margin},
   backspace=\measure{layout:margin},
   header=0pt,
   footer=\measure{layout:margin},
   width=middle,
   height=middle]

\setupbodyfont
  [dejavu,11pt]

\setupwhitespace
  [big]

\setuphead
  [chapter]
  [style=\bfc,
   interaction=all]

\setuppagenumbering
  [alternative=doublesided,
   location=]

\setupfootertexts
  [\documentvariable{title}][\pagenumber]
  [\pagenumber][\documentvariable{title}]

\setuphead
  [section]
  [style=\bfb]

\setuphead
  [subsection]
  [style=\bfa]

\setuphead
  [subsubsection]
  [style=\bf,
   after=]

\setuplist
  [interaction=all]

\setupalign
  [verytolerant,stretch]

\setupnote
  [footnote]
  [next={ },
   split=verystrict,
   scope=page]

\setupinteraction
  [state=start,
   option=bookmark,
   color=darkmagenta,
   contrastcolor=darkmagenta]

\setupinteractionscreen
  [option=bookmark]

\placebookmarks
  [title,subject]
  [title,subject]

\enabledirectives
  [references.bookmarks.preroll]

\kindofpagetextareas\plusone % partial page. HH: low level, no high level switch (yet)

\definetextbackground
  [fiction]
  [location=paragraph,
   frame=off,
   leftoffset=1ex,
   rightoffset=1ex,
   topoffset=1ex,
   bottomoffset=1ex,
   background=color,
   backgroundcolor=lightgray]

\defineparagraphs
  [two]
  [n=2,
   offset=1ex,
   background=color,
   backgroundcolor=gray]

\setupframed
  [node]
  [offset=1pt,
   foregroundstyle=\tfa]

\defineframed
  [nodeGreen]
  [node]
  [foregroundcolor=darkgreen,
   foregroundstyle=italic]

\defineframed
  [nodeSmall]
  [node]
  [foregroundstyle=\tfx]

\startbuffer [bib]
    @ARTICLE{Krebs1946,
        author  = {Krebs, H. A.},
        title   = {Cyclic processes in living matter},
        journal = {Enzymologia},
        year    = {1946},
        volume  = {12},
        pages   = {88--100}
    }

    @ARTICLE{Bethe1939a,
        author    = {Bethe, H. A.},
        title     = {Energy Production in Stars},
        journal   = {Phys. Rev.},
        year      = {1939},
        volume    = {55},
        pages     = {103–103},
        month     = {Jan},
        doi       = {10.1103/PhysRev.55.103},
        issue     = {1},
        publisher = {American Physical Society},
        XXurl     = {http://link.aps.org/doi/10.1103/PhysRev.55.103}
    }

    @ARTICLE{Bethe1939b,
        author    = {Bethe, H. A.},
        title     = {Energy Production in Stars},
        journal   = {Phys. Rev.},
        year      = {1939},
        volume    = {55},
        pages     = {434–456},
        month     = {Mar},
        doi       = {10.1103/PhysRev.55.434},
        issue     = {5},
        publisher = {American Physical Society},
        XXurl     = {http://link.aps.org/doi/10.1103/PhysRev.55.434}
    }
\stopbuffer

\usebtxdefinitions [apa]
\setupbtxrendering [apa] [pagestate=start] % index cite pages in bibliography

\usebtxdataset  [bib.buffer]

\defineframed
  [mynode]
  [node]
  [offset=1pt,
   foregroundcolor=white]

\startreusableMPgraphic{nodes::krebs}

   % The Bethe cycle for energy production in stars (1939), following
   % Krebs (1946)

   save p ; path p[] ;
   p1 := (for i=0 step 60 until 300: dir(90-i) .. endfor cycle) scaled 2.75cm ;
   p0 := p1 scaled  .5 ;
   p2 := p1 scaled 1.5 ;

   draw node(p1,0,"\mynode{\chemical{^{12}C}}") ;
   draw node(p1,1,"\mynode{\chemical{^{13}N}}") ;
   draw node(p1,2,"\mynode{\chemical{^{13}C}}") ;
   draw node(p1,3,"\mynode{\chemical{^{14}N}}") ;
   draw node(p1,4,"\mynode{\chemical{^{15}O}}") ;
   draw node(p1,5,"\mynode{\chemical{^{15}N}}") ;

   drawarrow fromtopaths.urt (true,p1,0,p1,1,"\mynode{a}") withcolor white ;
   drawarrow fromtopaths.rt  (true,p1,1,p1,2,"\mynode{b}") withcolor white ;
   drawarrow fromtopaths.lrt (true,p1,2,p1,3,"\mynode{c}") withcolor white ;
   drawarrow fromtopaths.llft(true,p1,3,p1,4,"\mynode{d}") withcolor white ;
   drawarrow fromtopaths.lft (true,p1,4,p1,5,"\mynode{e}") withcolor white ;
   drawarrow fromtopaths.ulft(true,p1,5,p1,6,"\mynode{f}") withcolor white ;

   draw node(p0,0,"\mynode{\chemical{^1H}}") ;
   draw node(p0,2,"\mynode{\chemical{^1H}}") ;
   draw node(p0,3,"\mynode{\chemical{^1H}}") ;
   draw node(p0,5,"\mynode{\chemical{^1H}}") ;

   drawarrow fromtopaths(3/10,p0,0,p1,0.5) withcolor white ;
   drawarrow fromtopaths(3/10,p0,2,p1,2.5) withcolor white ;
   drawarrow fromtopaths(3/10,p0,3,p1,3.5) withcolor white ;
   drawarrow fromtopaths(3/10,p0,5,p1,5.5) withcolor white ;

   draw node     (p2,0,"\mynode{\chemical{^4He}}")                ;
   draw node     (p2,1,"\mynode{$γ$}")                            ;
   draw node.lrt (p2,2,"\mynode{$\mathrm{e}^+ + ν_\mathrm{e}$}")  ;
   draw node     (p2,3,"\mynode{$γ$}")                            ;
   draw node     (p2,4,"\mynode{$γ$}")                            ;
   draw node.ulft(p2,5,"\mynode{$\mathrm{e}^+ + ν_\mathrm{e}$}")  ;

   drawarrow fromtopaths(-1/10,p1,0.5,p2,1) withcolor white ;
   drawarrow fromtopaths(-1/10,p1,1.5,p2,2) withcolor white ;
   drawarrow fromtopaths(-1/10,p1,2.5,p2,3) withcolor white ;
   drawarrow fromtopaths(-1/10,p1,3.5,p2,4) withcolor white ;
   drawarrow fromtopaths(-1/10,p1,4.5,p2,5) withcolor white ;
   drawarrow fromtopaths(-1/10,p1,5.5,p2,0) withcolor white ;

\stopreusableMPgraphic

\startuseMPgraphic{CoverPage}

    StartPage ;

        % Alan wanted a sun in the background combined somehow with the energy
        % harvesting molecule so here we go. The images that are used come from
        % the NASA website and I used them as screen saver for a while. The
        % version that I generate use another variant than the one on the users
        % machine.

        draw textext("\externalfigure[\framedparameter{imagename}]")
            xsized PaperWidth ysized (PaperHeight+4cm)
            shifted center Page shifted (0,2cm);

        for i=1 upto 512 :
            draw (textext("\reuseMPgraphic{nodes::krebs}") scaled (1/5 randomized 1/5))
            shifted (center Page randomized (PaperWidth,PaperHeight)) ;
        endfor ;

        draw (textext.ulft("\word{\documentvariable{title}}") xsized (PaperWidth/2))
            shifted (lrcorner Page)
            shifted (-PaperWidth/10,2PaperWidth/10)
            withcolor white;

        draw (textext.ulft("\word{\documentvariable{author}}") xsized (PaperWidth/2))
            shifted (lrcorner Page)
            shifted (-PaperWidth/10,PaperWidth/10)
            withcolor white;

    StopPage ;

\stopuseMPgraphic

\startsetups document:start

    \doifmodeelse {atpragma} {
        \startMPpage[imagename=nodes-sun-pia-03150]
            \includeMPgraphic{CoverPage}
        \stopMPpage
    } {
        \startMPpage[imagename=nodes-sun-pia-03149]
            \includeMPgraphic{CoverPage}
        \stopMPpage
    }
\stopsetups

\startdocument
  [title=Nodes,
   author=Alan Braslau,
   copyright=\ConTeXt\ development team,
   version=1.0]

\startsubject [title=Introduction]

The graphical representation of textual diagrams is an extremely useful tool in
the communication of idea. These are composed of graphical objects and blocks of
text or the combination of both, i.e. a decorated label or text block, in some
spatial relation to one another. \footnote {The spatial relation may be a
representation of a temporal relationship.} A simple example is shown \in {in
figure} [fig:AB].

\startplacefigure [reference=fig:AB]
    \startMPcode
        ahlength  := 12pt ;
        ahangle   := 30 ;
        ahvariant := 1 ; % dimpled

        u := 1cm ;
        save nodepath ; path nodepath ;
        nodepath := (left -- right) scaled u ;

        draw node(0,"\node{A}") ;
        draw node(1,"\node{B}") ;
        drawarrow fromto(0,0,1) ;
    \stopMPcode
\stopplacefigure

One often speaks of placing text with their accompanying decorations on \emph
{nodes}, points of intersection or branching or else points on a regular lattice.
The nodes of the above diagram are the two endpoints of a straight segment.

\startbuffer
\startMPcode
  save nodepath ; path nodepath ;
  nodepath := (left -- right) scaled .75u ;
  draw node(0,"A") ;
  draw node(1,"B") ;
  drawarrow fromto(0,0,1) ;
\stopMPcode
\stopbuffer

The code producing \inlinebuffer\ \emph {could} be:

\starttwo
    \METAPOST
    \startTEX
        \startMPcode
          draw node(0,"A") ;
          draw node(1,"B") ;
          drawarrow fromto(0,1) ;
        \stopMPcode
    \stopTEX
\two
    \CONTEXT
    \startTEX
       \startnodes
         \node[0]{A}
         \node[1]{B}
         \fromto [0,1] [option=arrow]
       \stopnodes
    \stopTEX
\stoptwo

drawing an arrow from A to B (or from node 0 to node 1).

\startitemize[packed]
    \startitem
        The \METAPOST\ code shown above has been simplified, as will be seen
        later.
    \stopitem
    \startitem
        The \CONTEXT\ module has yet to be coded (and may not be useful).
    \stopitem
\stopitemize

\stopsubject

\startsubject [title=\METAPOST]

\METAPOST\ is an inherently vectorial graphical language using \TEX\ to typeset
text; the \METAPOST\ language is integrated natively as MPlib in \CONTEXT. This
presents advantages over the use of an external graphical objects in providing a
great coherence of style along with great flexibility without bloat. \METAPOST\
has further advantages over many other graphical subsystems, namely high
precision, high quality and the possibility to solve equations. This last point
is little used but should not be overlooked.

It is quite natural in \METAPOST\ to locate these text nodes along a path or on
differing paths. This is a much more powerful concept than locating nodes at some
pair of coordinates, on a square or rectangular lattice, for example, as in a
table. These paths may be in three dimensions (or more); of course the printed
page will only be some projection onto two dimensions. Nor must the nodes be
located on the points defining a path: they may have as index any \emph {time}
along the path \type {p} ranging from the first defining point ($t = 0$) up to
the last point of a path ($t ≤ \mathtt{length(p)}$). \footnote {The time of a
cyclic path is taken modulo the length of the path.}

For a given path \type {p}, nodes are defined (implicitly) as \type {picture}
elements: \typ {picture p.pic[] ;} This is a pseudo|-|array where the square
brackets indicates a set of numerical tokens, as in \type {p.pic[0]} but also
\type {p.pic0}. This number need not be an integer, and \type {p.pic[.5]} or
\type {p.pic.5} (not to be confused with \type {p.pic5}) are also valid. These
picture elements are taken to be located relative to the path \type {p}, with the
index \type {t} corresponding to a time along the path, as in \typ {draw p.pic[t]
shifted point t of p ;} (although one would not necessarily draw them in this
way). This convention allows the \quote {nodes} to be oriented and offset with
respect to the path in an arbitrary fashion.

Note that a path can be defined and then nodes placed relative to this path, or
else the path may be simply declared yet remain undefined, to be determined
later, after the nodes are declared. Yet another possibility allows for the path
to be adjusted as needed, as a function of whatever nodes are to be occupied.
This will be illustrated through examples later.

\stopsubject

\startsubject [title=A few simple examples]

\startplacefigure [location=left,reference=fig:square]
    \startMPcode
        path p ; p := fullsquare scaled 3cm ;
        draw p ;
        for i=0 upto length p:
            draw point i of p
            withcolor red
            withpen pencircle scaled 5pt ;
        endfor ;
        % this looks better in the figure placement:
        setbounds currentpicture to boundingbox currentpicture
            rightenlarged 2mm ;
    \stopMPcode
\stopplacefigure

This might sound a bit arbitrary, so let's begin with the illustration of a
typical natural transformation of mathematics. A simple path and the points
defining this path are drawn in red here. Although it is trivial, this example
helps to introduce the \METAPOST\ syntax. (Of course, one should be able to use
this system without having to learn much \METAPOST.)

\flushsidefloats

\startTEX
\startMPcode
  path p ; p := fullsquare scaled 3cm ; draw p ;
  for i=0 upto length p:
    draw point i of p
    withcolor red
    withpen pencircle scaled 5pt ;
  endfor ;
\stopMPcode
\stopTEX

\startbuffer
\startMPcode
  save nodepath ;
  path nodepath ; nodepath = p ;
  draw node(0,"\node{$G(X)$}") ;
  draw node(1,"\node{$G(Y)$}") ;
  draw node(2,"\node{$F(Y)$}") ;
  draw node(3,"\node{$F(X)$}") ;
  drawarrow fromto.bot(0,0,1, "\nodeSmall{$G(f)$}") ;
  drawarrow fromto.top(0,3,2, "\nodeSmall{$F(f)$}") ;
  drawarrow fromto.rt (0,2,1, "\nodeSmall{$η_Y$}") ;
  drawarrow fromto.lft(0,3,0, "\nodeSmall{$η_X$}") ;
\stopMPcode
\stopbuffer

\startplacefigure [location=left,reference=fig:natural]
    \getbuffer
\stopplacefigure

Given the named path \type {nodepath}, we can define and draw nodes as well as
connections between them.

\flushsidefloats

\typebuffer [option=TEX]

\startfiction
It is an important good practice with \METAPOST\ to reset or clear a variable
using the directive \type {save} as above for the suffix \type {nodepath}. The
macros used here rely on the creation of certain internal variables and may not
function correctly if the variable structure is not cleared. Indeed, any node may
contain a combination of picture elements, added successively, so it is very
important to \type {save} the variable, making its use local, rather than global.
This point is particularly true under \CONTEXT, where a single MPlib instance is
used and maintained over multiple runs.

The \CONTEXT\ directives \type {\startMPcode} \unknown\ \type {\stopMPcode}
include grouping (\METAPOST\ \type {begingroup ;} \unknown\ \type {endgroup ;})
and the use of \type {save} will make the suffix local to this code block. In the
examples above of \in{Figures} [fig:square] and \in [fig:natural], the path \type
{p} is not declared local (through the use of a \type {save}) therefore remains
available for other \METAPOST\ code blocks. We cannot do this with the default
suffix name \type {nodepath} without getting other unwanted consequences.
\stopfiction

Note that one should not confuse the \METAPOST\ function \type {node()} with the
\CONTEXT\ command \type {\node{}}, defined as:

\starttwo
  \startTEX
  \setupframed
    [frame=off,
     offset=1pt]
  \stopTEX
\two
  \startTEX
  \defineframed
    [nodeSmall]
    [node]
    [foregroundstyle=small]
  \stopTEX
\stoptwo

placing the text within a \CONTEXT\ frame (with the frame border turned|-|off).
The \METAPOST\ function \type {node(i,"...")} sets and returns a picture element
associated with a point on path \type {nodepath} indexed by its first argument.
The second argument here is a string that gets typeset by \TEX.

The \METAPOST\ function \type {fromto()} returns a path segment going from two
points of the path \type {nodepath}, by default. The first argument (\type {0} in
the example above) can be used as a displacement to skew the path away from a
straight line. The last argument is a string to be typeset and placed midpoint of
the segment. The \emph {suffix} appended to the function name gives an offset
around this halfway point. This follows standard \METAPOST\ convention.

\startfiction
  The \CONTEXT\ syntax might be:
  \startTEX
    \startuseMPgraphic{node:square}
        save nodepath ; path nodepath ;
        nodepath := fullsquare scaled 6cm ;
    \stopuseMPgraphic

    \setupfromto [option=arrow,position=auto]

    \startnodes [mp=node:square]
      \node [0] {$G(X)$}
      \node [1] {$G(Y)$}
      \node [2] {$F(Y)$}
      \node [3] {$F(X)$}
      \fromto [0,1] [label={\tfxx $G(f)$}]
      \fromto [3,2] [label={\tfxx $F(f)$}]
      \fromto [2,1] [label={\tfxx $η_Y$}}
      \fromto [3,0] [label={\tfxx $η_X$}]
    \stopnodes
  \stopTEX
\stopfiction

Actually, as will be seen later, one can specify the use of any defined path, not
restricted to the built||in name \type {nodepath} that is used by default.
Furthermore, a function \type {fromtopaths()} can be used to draw segments
connecting any two paths which may be distinct. This, too, will be illustrated
later.

\startbuffer
\startMPcode
save nodepath ; path nodepath ;
nodepath := fullsquare scaled 2cm ;
save A ; A = 3 ; draw node(A,"\node{A}") ;
save B ; B = 2 ; draw node(B,"\node{B}") ;
save C ; C = 0 ; draw node(C,"\node{C}") ;
save D ; D = 1 ; draw node(D,"\node{D}") ;

drawarrow fromto(0,B,C) ;
drawarrow fromto(0,A,D) crossingunder fromto(0,B,C) ;
\stopMPcode
\stopbuffer

\startplacefigure [location=left,reference=fig:crossingunder]
  \getbuffer
\stopplacefigure

Consider now the following code illustrating the \METAFUN\ operator \type
{crossingunder} that draws a path with segments cut|-|out surrounding the
intersection(s) with a second path segment. This operator is of such general use
that it has been added to the \METAFUN\ base.

\flushsidefloats

\typebuffer [option=TEX]

\startplacefigure [reference=fig:sincos,location=left,title={\type{crossingunder}}]
    \startMPcode
        save u ; u := 2cm ;
        save p ; path p[] ;
        n := 64 ;
        p2 := for i=0 upto n : if i>0 : .. fi (3u*(i/n), u*sind(720*(i/n))) endfor ;
        p3 := for i=0 upto n : if i>0 : .. fi (3u*(i/n), u*cosd(720*(i/n))) endfor ;
        p4 := point 0 of p2 -- point (length p2) of p2 shifted (left*.01u) ;

        draw p2 withcolor red ;
        crossingscale := 20 ;
        draw (p3 crossingunder p2) crossingunder p4 withcolor blue ;
        crossingscale := 10 ;
        drawarrow (p4 crossingunder p2) ;
    \stopMPcode
\stopplacefigure

Another illustration of the \type {crossingunder} operator in use is shown in \in
{figure} [fig:sincos]. Because the diagrams are all defined and drawn in
\METAPOST, one can easily extend the simple mode drawing with any sort of
graphical decoration using all the power of the \METAPOST\ language.

This brings up a delicate point that has inhibited the development of a \CONTEXT\
module up to now: \METAPOST\ is such a powerful language that it is difficult to
design a set of commands having equivalent flexibility. Perhaps, as is done
presently with the Chemistry package, one need only allow for the injection of
arbitrary \METAPOST\ code when needed for such special cases.

Should it be done like this?

\startfiction
\startTEX
\startnodes [mp=node:square]
  \node [3] {A}
  \node [2] {B}
  \node [0] {C}
  \node [1] {D}
  \fromto [2,0]
  \fromto [3,1] [option={under[2,0]}] % ?
\stopnodes
\stopTEX
\stopfiction

Such a simple example illustrates why we favor a pure \METAPOST\ interface and
why a \CONTEXT\ interface will not be developed, at least for now.

\stopsubject

\startsubject [title=Cyclic diagrams]

In a slightly more complicated example, that of a catalytic process given in the
\cite[authoryears] [Krebs1946] representation, where the input is indicated
coming into the cycle from the center of a circle and the products of the cycle
are spun|-|off from the outside of the circle, we start by defining a circular
path where each point corresponds to a step in the cyclic process. Our example
will use six steps.

We will want to define a second circular path with the same number of points at
the interior of this first circle for the input and a third circular path at the
exterior for the output. Thus,

\startTEX
\startMPcode
  save p ; path p[] ;
  % define a fullcircle path with nodes at 60° (rather than 45°)
  p1 := (for i=0 step 60 until 300: dir(90-i) .. endfor cycle) scaled 2.75cm ;
  p0 := p1 scaled  .5 ;
  p2 := p1 scaled 1.5 ;
\stopMPcode
\stopTEX

\startplacefigure [location=left,title={The paths that we will use for anchoring for nodes.}]
\startMPcode
    save p ; path p[] ;
    p1 := (for i=0 step 60 until 300: dir(90-i) .. endfor cycle) scaled 2.75cm ;
    p0 := p1 scaled  .5 ;
    p2 := p1 scaled 1.5 ;
    for i=0 upto 2:
        draw p[i] ;
        for j=0 upto length p[i]:
            draw point j of p[i] withpen currentpen scaled 5 withcolor red ;
        endfor
        label.bot("p" & decimal i, point 0 of p[i]) ;
    endfor
\stopMPcode
\stopplacefigure

Nodes will then be drawn on each of these three circles and arrows will be used
to connect these various nodes, either on the same path or else between paths.

The \METAPOST\ function \type {fromto()} is used to give a segment pointing
between nodes. It \emph {assumes} the path \type {nodepath}, and in fact calls
the function \type {fromtopaths} that explicitly takes path names as arguments,
i.e. \type {fromto (d, i, j, ...)} is equivalent to \type {fromtopaths (d,
nodepath, i, nodepath, j, ...)}.

As stated above, this segment can be a straight line or else a path that can be
bowed|-|away from this straight line by a transverse displacement given by the
function's first argument (given in units of the straight segment length). When
both nodes are located on a single, defined path, this segment can be made to lie
on or follow this path, such as one of the circular paths defined above. This
behavior is obtained by using any non|-|numeric value (such as \type {true}) in
place of the first argument. Of course, this cannot work if the two nodes are not
located on the same path.

\startplacefigure [reference=fig:Bethe,
  title={The \cite[author] [Bethe1939a] cycle for the energy production in stars
         \cite[alternative=year,left=(,right=)] [Bethe1939a+Bethe1939b] in a
         \cite[authoryears] [Krebs1946] representation of a catalytic process.}]
\startMPcode

  % The Bethe cycle for energy production in stars (1939), following Krebs (1946)

  save p ; path p[] ;
  p1 := (for i=0 step 60 until 300: dir(90-i).. endfor cycle) scaled 2.75cm ;
  p0 := p1 scaled  .5 ;
  p2 := p1 scaled 1.5 ;

  bboxmargin := 0pt ;

  draw node(p1,0,"\node{\chemical{^{12}C}}") ;
  draw node(p1,1,"\node{\chemical{^{13}N}}") ;
  draw node(p1,2,"\node{\chemical{^{13}C}}") ;
  draw node(p1,3,"\node{\chemical{^{14}N}}") ;
  draw node(p1,4,"\node{\chemical{^{15}O}}") ;
  draw node(p1,5,"\node{\chemical{^{15}N}}") ;

  drawarrow fromtopaths.urt (true,p1,0,p1,1,"\nodeGreen{a}") ;
  drawarrow fromtopaths.rt  (true,p1,1,p1,2,"\nodeGreen{b}") ;
  drawarrow fromtopaths.lrt (true,p1,2,p1,3,"\nodeGreen{c}") ;
  drawarrow fromtopaths.llft(true,p1,3,p1,4,"\nodeGreen{d}") ;
  drawarrow fromtopaths.lft (true,p1,4,p1,5,"\nodeGreen{e}") ;
  drawarrow fromtopaths.ulft(true,p1,5,p1,6,"\nodeGreen{f}") ;

  draw node(p0,0,"\node{\chemical{^1H}}") ;
  draw node(p0,2,"\node{\chemical{^1H}}") ;
  draw node(p0,3,"\node{\chemical{^1H}}") ;
  draw node(p0,5,"\node{\chemical{^1H}}") ;

  drawarrow fromtopaths(3/10,p0,0,p1,0.5) withcolor .6white ;
  drawarrow fromtopaths(3/10,p0,2,p1,2.5) withcolor .6white ;
  drawarrow fromtopaths(3/10,p0,3,p1,3.5) withcolor .6white ;
  drawarrow fromtopaths(3/10,p0,5,p1,5.5) withcolor .6white ;

  draw node     (p2,0,"\node{\chemical{^4He}}") ;
  draw node     (p2,1,"\node{$γ$}") ;
  draw node.lrt (p2,2,"\node{$\mathrm{e}^+ + ν_\mathrm{e}$}") ;
  draw node     (p2,3,"\node{$γ$}") ;
  draw node     (p2,4,"\node{$γ$}") ;
  draw node.ulft(p2,5,"\node{$\mathrm{e}^+ + ν_\mathrm{e}$}") ;

  drawarrow fromtopaths(-1/10,p1,0.5,p2,1) withcolor .6white ;
  drawarrow fromtopaths(-1/10,p1,1.5,p2,2) withcolor .6white ;
  drawarrow fromtopaths(-1/10,p1,2.5,p2,3) withcolor .6white ;
  drawarrow fromtopaths(-1/10,p1,3.5,p2,4) withcolor .6white ;
  drawarrow fromtopaths(-1/10,p1,4.5,p2,5) withcolor .6white ;
  drawarrow fromtopaths(-1/10,p1,5.5,p2,0) withcolor .6white ;

\stopMPcode
\stopplacefigure

The circular arc segments labeled \emph {\darkgreen a–f} are drawn on \in
{figure} [fig:Bethe] using

\startTEX
drawarrow fromtopaths.urt (true,p1,0,p1,1,"\nodeGreen{a}") ;
\stopTEX

for example, where \type {\nodeGreen} is a frame that inherits from \type
{\node}, changing style and color:

\startTEX
\defineframed
  [nodeGreen]
  [node]
  [foregroundcolor=darkgreen,
   foregroundstyle=italic]
\stopTEX
\stopfootnote

The bowed|-|arrows feeding into the cyclic process and leading out to the
products, thus between different paths, from the path \type {p0} to the path
\type {p1} and from the path \type {p1} to the path \type {p2}, respectively, are
drawn using the deviations \type {+3/10} and \type {-1/10} (to and from
half|-|integer indices, thus mid|-|step, on path \type {p1}):

\startTEX
drawarrow fromtopaths( 3/10,p0,0,p1,0.5) withcolor .6white ;
drawarrow fromtopaths(-1/10,p1,0.5,p2,1) withcolor .6white ;
\stopTEX

\startsubsubject [title={A lesson in \METAPOST}]

An \quote {array} of paths is declared through \typ {path p[] ;} it is not a
formal array, rather a syntactic definition of a collection of path variables
\type {p0}, \type {p1}, \unknown\ named beginning with the tag \quote {p}
followed by any number, not necessarily an integer (i.e. \type {p3.14} is a valid
path name). The syntax allows enclosing this \quote {index} within square
brackets, as in \type {p[0]} or, more typically, \type {p[i]}, where \type {i}
would be a numeric variable or the index of a loop. Note that the use of brackets
is required when using a negative index, as in \type {p[-1]} (for \type {p-1} is
interpreted as three tokens, representing a subtraction). Furthermore, the
variable \type {p} itself, would here be a numeric (by default), so \type {p[p]}
would be a valid syntactic construction! One could, additionally, declare a set
of variables \typ {path p[][] ;} and so forth, defining also \type {p[0][0]}
(equivalently, \type {p0 0}) for example as a valid path, coexisting with yet
different from the path \type {p0}.

\METAPOST\ also admits variable names reminiscent of a structure: \typ {picture
p.pic[] ;} for example is used internally in the \type {node} macros, but this
becomes \typ {picture p[]pic[] ;} when using a path \quote {array} syntax. These
variable names are associated with the suffix \type {p} and become all undefined
by \typ {save p ;}.

The \type {node} macros use the default name \type {nodepath} when no path is
explicitly specified. It is the user's responsibility to ensure the local or
global validity of this path if used.

\stopsubsubject

\stopsubject

\startsubject [title=Tree diagrams]

The tree diagram shown below is drawn using four paths, each one defining a row
or generation in the branching. The definition of the spacing of nodes was
crafted by hand and is somewhat arbitrary: 3.8, 1.7, and 1 for the first, second
and third generations.

\startplacefigure [location=force,reference=fig:DNA]
\startMPcode
  % third example: A tree diagram

  save u ; u := 2.25cm ;
  save n ; n := 2 ; % n children per generation

  save p ; path p[] ;
  p0 := origin ;
  numeric d[] ; d1 := 3.8 ; d2 := 1.7 ; d3 := 1 ;
  for g=1 upto 3:
    p[g] :=
      for i=0 upto length(p[g-1]):
        for c=0 upto n-1:
          if (i+c)>0: -- fi
          ((point i of p[g-1]) shifted (d[g]*(c/(n-1)-.5)*u,-u))
        endfor
      endfor ;
  endfor

  draw node(p0,0,  "\node{DNA interactions with surfaces}") ;
  draw node(p1,0,  "\node{repulsive:}") ;
  draw node(p1,1,  "\node{attractive: adsorption}") ;
  draw node(p2,0,  "\node{confinement}") ;
  draw node(p2,1,  "\node[align=middle,location=high]{depletion,\\macromolecular\\crowding}") ;
  draw node(p2,2,  "\node{chemisorption}") ;
  draw node(p2,3,  "\node{physisorption}") ;
  draw node(p3,5.5,"\node{immobilized}") ;
  draw node(p3,7,  "\node{mobile}") ;

  drawarrow fromtopaths(0,p0,0,p1,0) ;
  drawarrow fromtopaths(0,p0,0,p1,1) ;

  drawarrow fromtopaths(0,p1,0,p2,0) ;
  drawarrow fromtopaths(0,p1,0,p2,1) ;
  drawarrow fromtopaths(0,p1,1,p2,2) ;
  drawarrow fromtopaths(0,p1,1,p2,3) ;

  drawarrow fromtopaths(0,p2,2,p3,5.5) ;
  drawarrow fromtopaths(0,p2,3,p3,5.5) ;
  drawarrow fromtopaths(0,p2,3,p3,7) ;

\stopMPcode
\stopplacefigure

One can do better by allowing \METAPOST\ to solve equations and to determine this
spacing automatically. This will be illustrated by a very simple example where
nodes are first placed on a declared but undefined path.

\startTEX
save p ; % path p ;
\stopTEX

The \type {save p ;} assures that the path is undefined. This path will later get
defined based on the contents of the nodes and a desired relative placement. In
fact, it is not even necessary to declare that the suffix will be a path, as the
path will be declared and automatically built once the positions of all the nodes
are determined, which is why the \type {path} declaration is commented|-|out
above.

The user is to be warned that solving equations in \METAPOST\ can be
non|-|trivial for those who are less mathematically inclined. One needs to
establish a coupled set of equations that is solvable: that is, fully but not
over|-|determined.

A few helper functions have been defined: \type {makenode()} returns a suffix
(variable name) corresponding to the node's position. The first such node can be
placed at any finite point, for example the drawing's origin. The following nodes
can be placed in relation to this first node:

% \startframed [frame=off,align=text,offset=overlay] % keep this together on one page.
\startTEX [style=small]
save nodepath ;
save first, second, third, fourth ;
pair first, second, third, fourth ;
first.i  = 0 ; first  = makenode(first.i, "\node{first}") ;
second.i = 1 ; second = makenode(second.i,"\node{second}") ;
third.i  = 2 ; third  = makenode(third.i, "\node{third}") ;
fourth.i = 3 ; fourth = makenode(fourth.i,"\node{fourth}") ;

first  = origin ;
second = first
       + betweennodes.urt(nodepath,first.i, nodepath,second.i,whatever) ;
third  = second
       + betweennodes.lft(nodepath,second.i,nodepath,third.i, whatever) ;
fourth = third
       + betweennodes.bot(nodepath,fourth.i,nodepath,first.i,3ahlength) ;
\stopTEX
% \stopframed

The helper function \type {betweennodes()} returns a vector pointing in a certain
direction, here following the standard \METAPOST\ suffixes: \type {urt}, \type
{lft}, and \type {bot}, that takes into account the bounding boxes of the
contents of each node, plus an (optional) additional distance (here given in
units of the arrow||head length, \type {ahlength}). Using the keyword \type
{whatever} tells \METAPOST\ to adjust this distance as necessary. The above set
of equations is incomplete as written, so a fifth and final relation needs to be
added; the fourth node is also to be located directly to the left of the very
first node: \footnote {Equivalently, we could declare that the first node located
to the right of the fourth node: \typ {first = fourth + betweennodes.rt
(nodepath, first.i, nodepath, fourth.i, 3ahlength) ;}}

\startTEX [style=small]
fourth = first
       + betweennodes.lft(nodepath,fourth.i,nodepath,first.i,3ahlength) ;
\stopTEX

Note that the helper function \type {makenode()} can be used as many times as
needed; if given no content, only returning the node's position. Additional nodes
can be added to this diagram along with appropriate relational equations, keeping
in mind that the equations must be solvable, of course. This last point is the
one challenge that most users might face.

The function \type {node()} that was used previously and returning a picture
element to be drawn itself calls the function \type {makenode()}, used here. The
nodes have not been drawn as yet:

\startTEX
for i = first.i, second.i, third.i, fourth.i :
  draw node(i) ;
  drawarrow fromto(0,i,i+1) ;
endfor
\stopTEX

resulting in \in{figure} [fig:relative]. The path is now defined as one running
through the position of all of the defined nodes, and is cyclic.

\startplacefigure [location=here,reference=fig:relative]
\startMPcode
  save nodepath ;
  save first, second, third, fourth ;
  pair first, second, third, fourth ;
  first.i  = 0 ; first  = makenode(first.i, "\node{first}") ;
  second.i = 1 ; second = makenode(second.i,"\node{second}") ;
  third.i  = 2 ; third  = makenode(third.i, "\node{third}") ;
  fourth.i = 3 ; fourth = makenode(fourth.i,"\node{fourth}") ;

  first  = origin ;
  second = first  + betweennodes.urt(nodepath,first.i, nodepath,second.i,whatever) ;
  third  = second + betweennodes.lft(nodepath,second.i,nodepath,third.i, whatever) ;
  fourth = third  + betweennodes.bot(nodepath,fourth.i,nodepath,first.i,3ahlength) ;
  fourth = first  + betweennodes.lft(nodepath,fourth.i,nodepath,first.i,3ahlength) ;

  for i = first.i, second.i, third.i, fourth.i :
    draw node(i) ;
    drawarrow fromto(0,i,i+1) ;
  endfor
\stopMPcode
\stopplacefigure

Using this approach, that of defining but not drawing the nodes until a complete
set of equations defining their relative positions has been constructed, imposes
several limitations: firstly, the nodes are expected to be numbered from $0$ up
to $n$, continuously, without any gaps for each defined path. This is just an
implicit convention of the path construction heuristic. Secondly, when finally
defining all the nodes and their positions, the path needs to be constructed. A
function, \type {makenodepath(p) ;} accomplishing this gets implicitly called
(once) upon the drawing of any \type {node()} or connecting \type {fromto}. Of
course, \type {makenodepath()} can always be explicitly called once the set of
equations determining the node positions is completely defined.

\startparagraph [style=bold]
We again stress that the writing of a solvable yet not over|-|determined set of
equations can be a common source of error for many \METAPOST\ users.
\stopparagraph

Another such example is the construction of a simple tree of descendance or
family tree. There are many ways to draw such a tree; in \in{figure}
[fig:descendance] we will show only three generations.

\startplacefigure [location=here,reference=fig:descendance,
  title={A tree of descendance}]
\startMPcode
  save p ; % path p[], p[][] ; get automagically defined
  save spacing ; spacing = 5pt ;
  save d ; d = 4ahlength ;

  save mother, father ; pair mother, father ;

  mother = makenode(p,0,"\node{mother}") ;
  father = makenode(p,1,"\node{father}") ;

  mother = origin ;
  father = mother + betweennodes.rt(p,0,p,1,spacing) ;

  % first generation
  save child, spouse ; pair child[], spouse[] ;
  child = 0 ; spouse = 1 ;

  child1  = makenode(p0,child, "\node{child1}") ;
  spouse1 = makenode(p0,spouse,"\node{spouse}") ;
  child2  = makenode(p1,child, "\node{child2}") ;
  spouse2 = makenode(p1,spouse,"\node{spouse}") ;

  .5[child1,child2] = mother + d*down ;
  spouse1 = child1  + betweennodes.rt(p0,child, p0,spouse,spacing) ;
  child2  = spouse1 + betweennodes.rt(p0,spouse,p1,child, whatever) ;
  spouse2 = child2  + betweennodes.rt(p1,child, p1,spouse,spacing) ;

  % second generation
  save grandchild, grandspouse ; pair grandchild[], grandspouse[] ;
  grandchild1  = makenode(p0 0,child, "\node{grandchild1}") ;
  grandspouse1 = makenode(p0 0,spouse,"\node{spouse}") ;
  grandchild2  = makenode(p0 1,child, "\node{grandchild2}") ;
  grandspouse2 = makenode(p0 1,spouse,"\node{spouse}") ;
  grandchild3  = makenode(p1 0,child, "\node{grandchild3}") ;
  grandspouse3 = makenode(p1 0,spouse,"\node{spouse}") ;
  grandchild4  = makenode(p1 1,child, "\node{grandchild4}") ;
  grandspouse4 = makenode(p1 1,spouse,"\node{spouse}") ;

  .5[grandchild1,grandchild2]  = child1 + d*down ;
  .5[grandchild3,grandchild4]  = child2 + d*down ;
  grandchild2  = grandchild1 + betweennodes.rt(p0 0,child,p0 1,child,spacing) ;
  grandchild3  = grandchild2 + betweennodes.rt(p0 1,child,p1 0,child,spacing) ;
  grandchild4  = grandchild3 + betweennodes.rt(p1 0,child,p1 1,child,spacing) ;
  grandspouse1 = grandchild1 + nodeboundingpoint.bot(p0 0,child)
                             + nodeboundingpoint.lrt(p0 0,spouse) ;
  grandspouse2 = grandchild2 + nodeboundingpoint.bot(p0 1,child)
                             + nodeboundingpoint.lrt(p0 1,spouse) ;
  grandspouse3 = grandchild3 + nodeboundingpoint.bot(p1 0,child)
                             + nodeboundingpoint.lrt(p1 0,spouse) ;
  grandspouse4 = grandchild4 + nodeboundingpoint.bot(p1 1,child)
                             + nodeboundingpoint.lrt(p1 1,spouse) ;

  draw node(p,0) ;
  draw node(p,1) withcolor blue ;

  for i=0,1 :
    draw node(p[i],child) ;
    drawarrow fromtopaths(0,p,0,p[i],child) ;
    draw node(p[i],spouse) withcolor blue ;
    for j=0,1 :
      draw node(p[i][j],child) ;
      draw node(p[i][j],spouse) withcolor blue ;
    endfor
    drawarrow fromtopaths(0,p[i],child,p[i][0],child) ;
    drawarrow fromtopaths(0,p[i],child,p[i][1],child) ;
  endfor
\stopMPcode
\stopplacefigure

We leave it as an exercise to the reader to come|-|up with the equations used to
determine this tree (one can look at source of this document, if necessary).

The set of equations could be hidden from the user wishing to construct simple,
pre|-|defined types of diagrams. However, such cases would imply a loss of
generality and flexibility. Nevertheless, the \emph {node} module will probably
be extended in the future to provide a few simple models. One might be a
branching tree structure, yet even the above example as drawn does not fit into a
simple, general model.

A user on the mailing list asked if it is possible to make structure trees for
English sentences with categorical grammar, an example of which is shown in \in
{Figure} [fig:grammar].

\startbuffer
\startMPcode
  save p ; path p[] ; % we work first with unit paths.
  save n ; n = 0 ;
  forsuffixes $=People,from,the,country,can,become,quite,lonely :
    p [n] = makenode(p[n],0,"\node{\it" & (str $) & "}") = (n,0) ;
    n := n + 1 ;
  endfor
  save u ; u := TextWidth/n ;

  % build upward tree

  vardef makeparentnode(text t) =
    save i, xsum, xaverage, ymax ;
    i = xsum = 0 ;
    forsuffixes $ = t :
      clearxy ; z = point infinity of $ ;
      xsum := xsum + x ;
      if unknown ymax : ymax = y ; elseif y > ymax : ymax := y ; fi
      i := i + 1 ;
    endfor
    xaverage = xsum / i ;
    ymax := ymax + 1 ;
    forsuffixes $ = t :
      clearxy ;
      z = point infinity of $ ;
      $ := $ & z -- (x,ymax) if i>1 : -- (xaverage,ymax) fi ;
    endfor
  enddef ;

  makeparentnode(p2,p3) ;
  makeparentnode(p4,p5) ;
  makeparentnode(p6,p7) ;
  makeparentnode(p1,p2) ;
  makeparentnode(p0,p1) ;
  makeparentnode(p4,p6) ;
  makeparentnode(p0,p4) ;
  makeparentnode(p0) ;

  % the paths are all defined but need to be scaled.

  for i=0 upto n-1 :
    p[i] := p[i] xyscaled (u,.8u) ;
    draw node(p[i],0) ;
  endfor

  save followpath ; boolean followpath ; followpath = true ;

  draw fromtopaths(followpath,p0,0,p0,1,"\node{H:N}") ;
  draw fromtopaths(followpath,p1,0,p1,1,"\node{Rel:Prep}") ;
  draw fromtopaths(followpath,p2,0,p2,1,"\node{Dr:Dv}") ;
  draw fromtopaths(followpath,p3,0,p3,1,"\node{H:N}") ;
  draw fromtopaths(followpath,p4,0,p4,1,"\node{M:Aux}") ;
  draw fromtopaths(followpath,p5,0,p5,1,"\node{H:Mv}") ;
  draw fromtopaths(followpath,p6,0,p6,1,"\node{M:Adv}") ;
  draw fromtopaths(followpath,p7,0,p7,1,"\node{H:Adj}") ;

  draw fromtopaths(followpath,p1,1,p1,2) ;
  draw fromtopaths(followpath,p2,3,p2,4) ;
  draw fromtopaths(followpath,p1,2,p1,3,"\node{M:PP}") ;
  draw fromtopaths(followpath,p2,1,p2,2) ;
  draw fromtopaths(followpath,p3,1,p3,2) ;
  draw fromtopaths(followpath,p2,2,p2,3,"\node{Ax:NP}") ;
  draw fromtopaths(followpath,p4,1,p4,2) ;
  draw fromtopaths(followpath,p5,1,p5,2) ;
  draw fromtopaths(followpath,p4,2,p4,3,"\node{P:VP}") ;
  draw fromtopaths(followpath,p6,1,p6,2) ;
  draw fromtopaths(followpath,p7,1,p7,2) ;
  draw fromtopaths(followpath,p6,2,p6,3,"\node{PCs:AdjP}") ;
  draw fromtopaths(followpath,p0,1,p0,2) ;
  draw fromtopaths(followpath,p1,3,p1,4) ;
  draw fromtopaths(followpath,p0,2,p0,3,"\node{S:NP}") ;
  draw fromtopaths(followpath,p4,3,p4,4) ;
  draw fromtopaths(followpath,p6,3,p6,4) ;
  draw fromtopaths(followpath,p4,4,p4,5,"\node{Pred:PredP}") ;
  draw node(p0,4.5,"\node{Cl}") ;
  draw fromtopaths(followpath,p0,3,p0,4.5) ;
  draw fromtopaths(followpath,p4,5,p4,6) ;
\stopMPcode
\stopbuffer

\startplacefigure [reference=fig:grammar,
    title=A categorical grammer structure tree]
    \getbuffer
\stopplacefigure

I chose to define a series of parallel paths, one per word, with one path
terminating whenever it joins another path (or paths) at a common parent.
Labeling each branch of the tree structure requires, of course, a knowledge of
the tree structure.

\typebuffer [option=TEX]

\stopsubject

\startsubject [title=Two final examples]

\defineframed
  [smallnode]
  [node]
  [foregroundstyle=\tfxx,
   background=color,
   backgroundcolor=white]

\startbuffer[mp:tikz-cd]
\startMPcode
  save nodepath ; save l ; l = 5ahlength ;
  save X, Y, Z, XxY, T ;
  pair X, Y, Z, XxY, T ;
  XxY.i = 0 ; XxY = makenode(XxY.i,"\node{$X\times_Z Y$}") ;
  X.i   = 1 ; X   = makenode(X.i,  "\node{$X$}") ;
  Z.i   = 2 ; Z   = makenode(Z.i,  "\node{$Z$}") ;
  Y.i   = 3 ; Y   = makenode(Y.i,  "\node{$Y$}") ;
  T.i   = 4 ; T   = makenode(T.i,  "\node{$T$}") ;
  XxY = origin ;
  X   = XxY + betweennodes.rt (nodepath,XxY.i,nodepath,X.i) + (l,0) ;
  Z   = X   + betweennodes.bot(nodepath,X.i,  nodepath,Z.i) + (0,-.8l);
  Y   = XxY + betweennodes.bot(nodepath,XxY.i,nodepath,Y.i) + (0,-.8l) ;
  T   = XxY + nodeboundingpoint.ulft(XxY.i)
            + nodeboundingpoint.lft (T.i) + l*dir(135) ;
  for i = XxY.i, X.i, Z.i, Y.i, T.i:
    draw node(i) ;
  endfor
  drawarrow fromto.top(0, XxY.i,X.i,"\smallnode{$p$}") ;
  drawarrow fromto.rt (0,   X.i,Z.i,"\smallnode{$f$}") ;
  drawarrow fromto.top(0,   Y.i,Z.i,"\smallnode{$g$}") ;
  drawarrow fromto.rt (0, XxY.i,Y.i,"\smallnode{$q$}") ;
  drawarrow fromto.top( .13,T.i,X.i,"\smallnode{$x$}") ;
  drawarrow fromto.urt(-.13,T.i,Y.i,"\smallnode{$y$}") ;
  drawarrow fromto    (0,   T.i,XxY.i,"\smallnode{$(x,y)$}")
      dashed withdots scaled .5
      withpen currentpen scaled 2 ;
\stopMPcode
\stopbuffer

\startplacefigure [location=left,reference=fig:tikz-cd]
    \getbuffer[mp:tikz-cd]
\stopplacefigure

The following example, shown in \in{figure} [fig:tikz-cd] and drawn here using
the present \METAPOST\ \type {node} macros, is inspired from the TikZ CD
(commutative diagrams) package. \footnote {The TikZ-CD package uses a totally
different approach: the diagram is defined and laid|-|out as a table with
decorations (arrows) running between cells.} The nodes are again given relative
positions rather than being placed on a predefined path. The arrow labeled \quote
{$(x,y)$} is drawn \typ {dashed withdots} and illustrates how the line gets
broken, here \type {crossingunder} its centered label.

\startparagraph [style=small]
    \typebuffer[mp:tikz-cd] [option=TEX]
\stopparagraph

% \doifmode {screen} {\page [yes]}

\in {Figure} [fig:tikz-cd2] is another \quotation{real|-|life} example, also
inspired by TikZ-CD.

\startbuffer[mp:tikz-cd2]
\startMPcode
  save nodepath ; save l ; l = 5ahlength ;
  save A, B, C, D, E ;
  pair A, B, C, D, E ;
  A.i = 0 ; A = makenode(A.i,"\node{$\pi_1(U_1\cap U_2)$}") ;
  B.i = 1 ; B = makenode(B.i,"\node{$\pi_1(U_1)\ast_{\pi_1(U_1\cap U_2)}\pi_1(U_2)$}") ;
  C.i = 2 ; C = makenode(C.i,"\node{$\pi_1(X)$}") ;
  D.i = 3 ; D = makenode(D.i,"\node{$\pi_1(U_2)$}") ;
  E.i = 4 ; E = makenode(E.i,"\node{$\pi_1(U_1)$}") ;
  A = origin ;
  B = A + betweennodes.rt(nodepath,A.i,nodepath,B.i) + (  l,0) ;
  C = B + betweennodes.rt(nodepath,B.i,nodepath,C.i) + (.7l,0) ;
  D = .5[A,B] + (0,-.9l) ;
  E = .5[A,B] + (0, .9l) ;

  for i = A.i, B.i, C.i, D.i, E.i :
      draw node(i) ;
  endfor
  drawarrow fromto.llft( 0,A.i,D.i,"\smallnode{$i_2$}") ;
  drawarrow fromto.ulft( 0,A.i,E.i,"\smallnode{$i_1$}") ;
  drawarrow fromto     ( 0,D.i,B.i) ;
  drawarrow fromto     ( 0,E.i,B.i) ;
  drawarrow fromto.urt( .1,E.i,C.i,"\smallnode{$j_1$}") ;
  drawarrow fromto.lrt(-.1,D.i,C.i,"\smallnode{$j_2$}") ;
  drawarrow fromto.top(  0,B.i,C.i) dashed evenly ;
  draw textext.top("{\tfxx\strut$\simeq$}")
      shifted point .4 of fromto(0,B.i,C.i) ;
\stopMPcode
\stopbuffer

\startplacefigure [location=here,reference=fig:tikz-cd2]
    \getbuffer[mp:tikz-cd2]
\stopplacefigure

\startparagraph [style=small]
    \typebuffer[mp:tikz-cd2] [option=TEX]
\stopparagraph

\placefloats

\stopsubject

\startsubject [title=Conclusions]

It was decided at the 2017 \CONTEXT\ Meeting in Maibach, Germany where this
package was presented that there was little use of developing a purely \CONTEXT\
interface. Rather, the \METAPOST\ package should be sufficiently accessible.

\stopsubject

\startsubject [title=References]

    \placelistofpublications

\stopsubject

\stoptitle

\stoptext
