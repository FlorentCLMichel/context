% language=us runpath=texruns:manuals/colors

% author    : Hans Hagen
% copyright : ConTeXt Development Team
% license   : Creative Commons Attribution ShareAlike 4.0 International
% reference : pragma-ade.nl | contextgarden.net | texlive (related) distributions
% origin    : the ConTeXt distribution
%
% comment   : Because this manual is distributed with TeX distributions it comes with a rather
%             liberal license. We try to adapt these documents to upgrades in the (sub)systems
%             that they describe. Using parts of the content otherwise can therefore conflict
%             with existing functionality and we cannot be held responsible for that. Many of
%             the manuals contain characteristic graphics and personal notes or examples that
%             make no sense when used out-of-context.
%
% comment   : Some chapters might have been published in TugBoat, the NTG Maps, the ConTeXt
%             Group journal or otherwise. Thanks to the editors for corrections. Also thanks
%             to users for testing, feedback and corrections.

%             \nopdfcompression

\enablemode[simple] % ,oversized

\startbuffer[abstract]

    This book is about colors and how to use them in \CONTEXT\ \MKIV, including
    \METAFUN. Although the basics are not that complex, a bit of insight in how
    they are implemented and what can be done might help in creating more
    interesting looking documents.

\stopbuffer

\environment colors-environment

\startdocument
  [author=Hans Hagen,
   title=Coloring \ConTeXt,
   subtitle=explaining luatex and mkiv,
   affiliation=PRAGMA ADE,
   comment=work in progress,
   cover:color:1=darkgreen,
   cover:color:2=darkyellow,
   cover:color:3=darkblue,
   cover:color:4=darkmagenta,
   cover:color:5=darkgray]

    \startfrontmatter
        \component manuals-explaining-contents
        \component colors-introduction
    \stopfrontmatter

    \startbodymatter
        \component colors-basics
        \component colors-metafun
        \component colors-graphics
    \stopbodymatter

\stopdocument
