% language=uk

\environment still-environment

\enabledirectives[visualizers.fraction=2.5]

\startcomponent still-profiling

\startchapter[title=Profiling lines]

\startsection[title=Introduction]

Although \TEX\ is pretty good at typesetting simple texts like novels, in
practice it's often used for getting more complex stuff on paper (or screen).
Math is of course the first thing that comes to mind. If for instance you look at
the books typeset by Don Knuth you will see a rendering that is rather consistent
in spacing. This is no surprise as the author pays a lot of attention to detail
and uses inline versus display math properly. No publisher will complain about
the result.

In the documents that I have to write styles for, the content is rather mixed,
and in particular inline math can have display math properties. In a one-column
layout this is not a real problem especially because lots of short sentences and
white space is used: we're talking of secondary|-|school educational math where
arguments for formatting something this or that way is not always rational and
consistent but more based on \quotation {this is what the student expects},
\quotation {the competitor also does it that way} or just \quotation {we like
this more}. For instance in a recent project, the books with answers to questions
had to be typeset in a multicolumn layout and because math was involved, we end
up with lines with more height and depth than normal. That can not only result in
more pages but also can make the result look a bit messy.

\blank \dontleavehmode \start \showboxes This paragraph demonstrates how lines
are handled: when a paragraph is broken into lines each line becomes a horizontal
box with a height and depth determined by the size of the characters that make up
the line. There is a minimal distance between baselines (\type {baselineskip})
and when lines touch there can optionally be a \type {\lineskip}. In the end we
get a vertical list of boxes and glue (either of not flexible) mixed with
penalties that determine optimal paragraph breaks. This paragraph shows that
there is normally enough space available to do the job. \par \stop \blank

We already have some ways to control this. For instance the dimensions of math
can be limited a bit and lines can be made to snap on a grid (which is what
publishers often want anyway). However, another alternative is to look at the
line and decide if successive lines can be moved closer, of course within the
constraints of the height and and depth of the lines. There is no real way to see
if some ugly clash can happen simply because when we run into boxed material
there can be anything inside and the dimensions can be set on purpose. This means
that we have to honour all dimensions and only can mess around with dimensions
when we're reasonably confident. In \CONTEXT\ this messing is called profiling
and that is what we will discuss next.

\stopsection

\startsection[title=Line heights and depths]

\startbuffer[example-1]
Regelmatig kom je procenten tegen. ‘Pro centum’ is Latijn en betekent per
honderd, dus één van elke honderd, dus \math {\frac {1} {100}} deel. Met
procenten rekenen is daarom rekenen met honderdsten: \math {45 \procent = \frac
{45} {100} = 0,45}. Dus \math {45 \procent} van een geheel is het \math {\frac
{45} {100}} deel ervan en dat kun je berekenen door te vermenigvuldigen met \math
{0,45}.
\stopbuffer

\startbuffer[example-2]
Je gaat uit van de bekende eigenschappen van machten. Bijvoorbeeld: \math {g^r *
g^s = g^{(r+s)}}. Neem je hierin \math {r = ^{g}\log(a)} en \math {s = ^{g}\log
b}, dan vind je: \math {g^{^{g}\log(a) + ^{g}\log(b)} = g^{^{g}\log a} \times
g^{^{g}\log b} = a \times b}. Hierbij gebruik je de definitieformules.
\stopbuffer

\startbuffer[example-3]
Omdat volgens de eigenschappen van machten en exponenten geldt \math {\frac {1}
{x^4} = x^{-4} } is ook hier sprake van een machtsfunctie, namelijk \math {f(x) =
\frac {6} {x^4} = 6 \times \frac {1} {x^4} = 6x^{-4}}.
\stopbuffer

In this section we will use some (Dutch) examples from documents that we've
processed. We show unprofiled versions, with two different paragraph widths, in
\in {figure} [fig:profiling:unprofiled-examples-widths]. All three examples shown
demonstrate that as soon as we use something more complex than a number or
variable in a subscript we exceed the normal line height, and thus the line
spacing becomes somewhat irregular.

\starttexdefinition ExampleRunUnprofiled #1
    \vbox \bgroup
        \switchtobodyfont[dejavu,10pt]
        \raggedright
        \hsize12cm
        \inleft[scope=local]{\ttx hsize 12cm\\unprofiled}
        \nl \vbox{\getbuffer[example-#1]}
        \blank
        \hsize10cm
        \inleft[scope=local]{\ttx hsize 10cm\\unprofiled}
        \nl \vbox{\getbuffer[example-#1]}
    \egroup
\stoptexdefinition

\startplacefigure[reference=fig:profiling:unprofiled-examples-widths,title={Unprofiled examples.}]
    \enabletrackers[profiling.show]
    \startcombination[nx=1,ny=3]
        \startcontent
            \ExampleRunUnprofiled{1}
        \stopcontent
        \startcaption
            example 1
        \stopcaption
        \startcontent
            \ExampleRunUnprofiled{2}
        \stopcontent
        \startcaption
            example 2
        \stopcaption
        \startcontent
            \ExampleRunUnprofiled{3}
        \stopcontent
        \startcaption
            example 3
        \stopcaption
    \stopcombination
    \disabletrackers[profiling.show]
\stopplacefigure

The profiled rendering of the same examples are shown in \in {figure}
[fig:profiling:profiled-examples-widths]. Here we use the minimal heights and
depths plus a minimum distance of 1pt. This default method is called \type
{strict}.

\starttexdefinition ExampleRunProfiled #1
    \vbox \bgroup
        \switchtobodyfont[dejavu,10pt]
        \raggedright
        \hsize12cm
        \inleft[scope=local]{\ttx hsize 12cm\\profiled}
        \nl \profiledbox{\getbuffer[example-#1]}
        \blank
        \hsize10cm
        \inleft[scope=local]{\ttx hsize 10cm\\profiled}
        \nl \profiledbox{\getbuffer[example-#1]}
    \egroup
\stoptexdefinition

\startplacefigure[reference=fig:profiling:profiled-examples-widths,title={Profiled examples.}]
    \enabletrackers[profiling.show]
    \startcombination[nx=1,ny=3]
        \startcontent
            \ExampleRunProfiled{1}
        \stopcontent
        \startcaption
            example 1
        \stopcaption
        \startcontent
            \ExampleRunProfiled{2}
        \stopcontent
        \startcaption
            example 2
        \stopcaption
        \startcontent
            \ExampleRunProfiled{3}
        \stopcontent
        \startcaption
            example 3
        \stopcaption
    \stopcombination
    \disabletrackers[profiling.show]
\stopplacefigure

In the first and last example there are some lines where the depth of one line
combined with the height of the following exceeds the standard line height. This
forces \TEX\ to insert \type {\lineskip} (mentioned in the demonstration
paragraph above), a dimension that is normally set to a fraction of the line
spacing (for instance 1pt for a 10pt body font and 12pt line spacing). When we
are profiling, \type{\lineskip} is ignored and we use a settable distance
instead. The second example (with superscripts) normally comes out fine as the
math stays within limits and we make sure that smaller fractions and scripts stay
within the natural limits of the line, but nested scripts can be an issue.

% \unexpanded\def\fakeinmargin#1%
%   {\dontleavehmode{\resetvisualizers\smash{\llap{#1}}}}

% \unexpanded\def\fakestrut#1%
%   {\fakeinmargin{\showstruts\strut\kern\dimexpr.5em+#1em/2\relax}}

In \in {figure} [fig:profiling:zapf] we see the profile of a regular text with no
math. The average text stays well within the limits of height and depth. If this
doesn't happen for prose then you need to adapt the height|/|depth ratio to the
ascender|/|descender ratio of the bodyfont. For regular text it makes no sense to
use the profiler, it only slows down typesetting.

\startplacefigure[reference=fig:profiling:zapf,title={Normal lines profiled (quote by Hermann Zapf)}]
    \enabletrackers[profiling.show]
    \switchtobodyfont[dejavu,10pt]
    \profiledbox{Coming back to the use of typefaces in electronic
publishing: many of the new typographers receive their
knowledge and information about the rules of typography
from books, from computer magazines or the instruction
manuals which they get with the purchase of a PC or
software. There is not so much basic instruction, as of
now, as there was in the old days, showing the differences
between good and bad typographic design. Many people are
just fascinated by their PC's tricks, and think that a
widely||praised program, called up on the screen, will make
everything automatic from now on.}
    \disabletrackers[profiling.show]
\stopplacefigure

\stopsection

\startsection[title=When lines exceed boundaries]

Let's now take a more detailed look at what happens when lines get too high or
low. First we'll zoom in on a simple example: in \in {figure}
[fig:profiling:simple-text], we compare a sample text rendered using the variants
of profiling currently implemented. (This is still experimental code so there
might be more in the future). Seeing profiles helps to get a picture of the
complications we have to deal with. In addition to the normal \type {vbox}
variant (used in the previous examples), we show \type {none} which only
analyzes, \type {strict} that uses the natural dimensions of lines and \type
{fixed} that is supposed to cooperate with grid snapping.

% \startbuffer[fake-1]
% \hsize 2cm
% \fakestrut2 first  x \par \fakestrut1 first  y \blank
% \fakestrut2 second x \par \fakestrut1 second y \blank[2*big]
% \fakestrut2 third  x \par \fakestrut1 third  y \par
% \stopbuffer

% \startbuffer[fake-2]
% \hsize 2cm
% \fakestrut2 line 1 \lower2ex\hbox{xxx}\par
% \fakestrut1 line 2 \raise2ex\hbox{xxx}\par
% \fakestrut2 line 3 \par
% \stopbuffer

% \startplacefigure[reference=fig:profiling:simple-text,title={Just a simple two line text.}]
%     \switchtobodyfont[10pt]
%     \enabletrackers[profiling.show]
%     \startcombination[nx=1,ny=3]
%         \startcontent
%             \startcombination[nx=4,ny=1,distance=10mm,style=\tt]
%                 {\kern1.5em\showboxes\vbox               {\getbuffer[fake-1]}} {vbox}
%                 {\kern1.5em\showboxes\profiledbox[none]  {\getbuffer[fake-1]}} {none}
%                 {\kern1.5em\showboxes\profiledbox[strict]{\getbuffer[fake-1]}} {strict}
%                 {\kern1.5em\showboxes\profiledbox[fixed] {\getbuffer[fake-1]}} {fixed}
%             \stopcombination
%         \stopcontent
%         \startcaption
%             no excessive height and depth
%         \stopcaption
%         \startcontent
%             \startcombination[nx=4,ny=1,distance=10mm,style=\tt]
%                 {\kern1.5em\showboxes\vbox                             {\getbuffer[fake-2]}} {vbox}
%                 {\kern1.5em\showboxes\profiledbox[none]  [distance=0pt]{\getbuffer[fake-2]}} {none}
%                 {\kern1.5em\showboxes\profiledbox[strict][distance=0pt]{\getbuffer[fake-2]}} {strict}
%                 {\kern1.5em\showboxes\profiledbox[fixed] [distance=0pt]{\getbuffer[fake-2]}} {fixed}
%             \stopcombination
%         \stopcontent
%         \startcaption
%             some excessive height and depth (distance=0pt)
%         \stopcaption
%         \startcontent
%             \startcombination[nx=4,ny=1,distance=10mm,style=\tt]
%                 {\kern1.5em\showboxes\vbox                             {\getbuffer[fake-2]}} {vbox}
%                 {\kern1.5em\showboxes\profiledbox[none]  [distance=2pt]{\getbuffer[fake-2]}} {none}
%                 {\kern1.5em\showboxes\profiledbox[strict][distance=2pt]{\getbuffer[fake-2]}} {strict}
%                 {\kern1.5em\showboxes\profiledbox[fixed] [distance=2pt]{\getbuffer[fake-2]}} {fixed}
%             \stopcombination
%         \stopcontent
%         \startcaption
%             some excessive height and depth (distance=2pt)
%         \stopcaption
%     \stopcombination
%     \disabletrackers[profiling.show]
% \stopplacefigure

\startbuffer[fake-1]
\hsize 2cm \dontleavehmode
line 1 \par
line 2 \par
line 3 \par
\stopbuffer

\startbuffer[fake-2]
\hsize 2cm \dontleavehmode
line 1 x\lower2ex\hbox{xxx}\par
line 2 x\raise2ex\hbox{xxx}\par
line 3 \par
\stopbuffer

\startbuffer[fake-3]
\hsize 2cm \dontleavehmode
x\lower2ex\hbox{xxx} line 1 \par
line 2 x\raise2ex\hbox{xxx}\par
line 3 \par
\stopbuffer

\startplacefigure[reference=fig:profiling:simple-text,title={Variants of profiling, using a constructed two-line text.}]
    \switchtobodyfont[10pt]
    \enabletrackers[profiling.show]
    \startcombination[nx=1,ny=3]
        \startcontent
            \startcombination[nx=5,ny=1,distance=10mm,style=\tt]
                {\showboxes\profiledbox[none]  [distance=0pt]{\getbuffer[fake-1]}} {none}
                {\showboxes\profiledbox[strict][distance=0pt]{\getbuffer[fake-1]}} {strict/0pt}
                {\showboxes\profiledbox[strict][distance=1pt]{\getbuffer[fake-1]}} {strict/1pt}
                {\showboxes\profiledbox[fixed] [distance=0pt]{\getbuffer[fake-1]}} {fixed/0pt}
                {\showboxes\profiledbox[fixed] [distance=1pt]{\getbuffer[fake-1]}} {fixed/1pt}
            \stopcombination
        \stopcontent
        \startcaption
            no excessive height and depth
        \stopcaption
        \startcontent
            \startcombination[nx=5,ny=1,distance=10mm,style=\tt]
                {\showboxes\profiledbox[none]  [distance=0pt]{\getbuffer[fake-2]}} {none}
                {\showboxes\profiledbox[strict][distance=0pt]{\getbuffer[fake-2]}} {strict/0pt}
                {\showboxes\profiledbox[strict][distance=1pt]{\getbuffer[fake-2]}} {strict/1pt}
                {\showboxes\profiledbox[fixed] [distance=0pt]{\getbuffer[fake-2]}} {fixed/0pt}
                {\showboxes\profiledbox[fixed] [distance=1pt]{\getbuffer[fake-2]}} {fixed/1pt}
            \stopcombination
        \stopcontent
        \startcaption
            some excessive height and depth (overlapping)
        \stopcaption
        \startcontent
            \startcombination[nx=5,ny=1,distance=10mm,style=\tt]
                {\showboxes\profiledbox[none]  [distance=0pt]{\getbuffer[fake-3]}} {none}
                {\showboxes\profiledbox[strict][distance=0pt]{\getbuffer[fake-3]}} {strict/0pt}
                {\showboxes\profiledbox[strict][distance=1pt]{\getbuffer[fake-3]}} {strict/1pt}
                {\showboxes\profiledbox[fixed] [distance=0pt]{\getbuffer[fake-3]}} {fixed/0pt}
                {\showboxes\profiledbox[fixed] [distance=1pt]{\getbuffer[fake-3]}} {fixed/1pt}
            \stopcombination
        \stopcontent
        \startcaption
            some excessive height and depth (out of touch)
        \stopcaption
    \stopcombination
    \disabletrackers[profiling.show]
\stopplacefigure

\in {Figure} [fig:profiling:simple-text] we show what happens when we add some
more excessive height and depth to lines. The samples are:

\starttyping
line 1 x\lower2ex\hbox{xxx}\par
line 2 x\raise2ex\hbox{xxx}\par
line 3 \par
\stoptyping

and:

\starttyping
x\lower2ex\hbox{xxx} line 1 \par
line 2 x\raise2ex\hbox{xxx}\par
line 3 \par
\stoptyping

Here the \type {strict} variant has some effect while \type {fixed} only has some
influence on the height and depth of lines. Later we will see that \type {fixed}
operates in steps and the default step is large so here we never meet the
criteria for closing up. \footnote {In \CONTEXT\ we normally use \type {\high}
and \type {\low} and both ensure that we don't exceed the natural height and
depth.}

A profiled box is typeset with \type {\profiledbox}. There is some control
possible but the options are not yet set in stone so we won't use them all here.
Profiling can be turned on for the whole document with \type {\setprofile} but
I'm sure that will seldom happen, and these examples show why: one cannot
beforehand say if the result looks good. Let's now apply profiling to a real
text. If you play with this yourself you can show profiles in gray with a
tracker:

\starttyping
\enabletrackers[profiling.show]
\stoptyping

\newbox\myprofiledbox

\startbuffer[raw-1]
\profiledbox
  [strict]
  [distance=0pt]
  {\nl\getbuffer[example-1]}
\stopbuffer

\startbuffer[raw-2]
\profiledbox
  [strict]
  [distance=1pt]
  {\nl\getbuffer[example-1]}
\stopbuffer

\startbuffer[raw-3]
\profiledbox
  [strict]
  [height=2\strutht,
   depth=2\strutdp,
   distance=1pt]
  {\nl\getbuffer[example-1]}
\stopbuffer

\startplacefigure[reference=fig:profiling:distances,title={Examples width different dimensions.}]
    \enabletrackers[profiling.show]
    \startcombination[nx=1,ny=3]
        \startcontent
            \setbox\myprofiledbox\rawbuffer{raw-1}
            \xdef\LastProfiledHeight{\the\htdp\myprofiledbox}
            \box\myprofiledbox
        \stopcontent
        \startcaption
            zero distance, resulting height \LastProfiledHeight
        \stopcaption
        \startcontent
            \setbox\myprofiledbox\rawbuffer{raw-2}
            \xdef\LastProfiledHeight{\the\htdp\myprofiledbox}
            \box\myprofiledbox
        \stopcontent
        \startcaption
            distance, resulting height \LastProfiledHeight
        \stopcaption
        \startcontent
            \setbox\myprofiledbox\rawbuffer{raw-3}
            \xdef\LastProfiledHeight{\the\htdp\myprofiledbox}
            \box\myprofiledbox
        \stopcontent
        \startcaption
            distance, double height and depth, resulting height \LastProfiledHeight
        \stopcaption
    \stopcombination
    \disabletrackers[profiling.show]
\stopplacefigure

We show the effects of setting distances in \in {figure}
[fig:profiling:distances]. We start with a zero distance:

\typebuffer[raw-1]

Because we don't want lines to touch we then set the minimum distance to a
reasonable value (1pt).

\typebuffer[raw-2]

Finally we also double the height and depth of lines, something that normally will
not be done. The defaults are the standard height and depth (the ones you get
when you inject a so-called \type{\strut}).

\typebuffer[raw-3]

The problem with this kind of analysis is that deciding when and how to use this
information to improve spacing is non|-|trivial. One of the characteristics of
user demand is that it nearly always concerns rather specific situations and that
suggested solutions could work only in those cases. But as soon as we have one
exceptional situation, intervention is needed which in turn means that a
mechanism has to be under complete user control. That itself assumes that the
user still has control, which is not the case in automated workflows. In fact, as
soon as one is in control over the source and rendering, there are often easier
ways to deal with the few cases that need treatment. Possible interference can
come from, for instance:

\startitemize[packed]
\startitem whitespace between paragraphs \stopitem
\startitem section titles (using different fonts and spacing) \stopitem
\startitem descriptions and other intermezzos \stopitem
\startitem images that interrupt the flow, or end up next to text \stopitem
\startitem ornaments like margin words (we catch some) \stopitem
\startitem text backgrounds making spacing assumptions \stopitem
\stopitemize

After a few decades of using \TEX\ and writing solutions, it has become pretty
clear that fully automated typesetting is a dream, if only because the input can
be pretty weird and inconsistent and demands (from those who are accustomed to
tweaking manually in a desktop publishing application) can be pretty weird and
inconsistent too. So, the only real solution is to use some kind of artificial
intelligence that one can feed with demands and constraints and that hopefully is
clever enough to deal with the inconsistencies. As this kind of computing is
unlikely to happen in my lifetime, poor man explicit solutions have to do the job
for now. One can add all kinds of heuristics to the profiler but this can
backfire when control is needed. Alternatively one can end up with many options
like we have in grid snapping.

\stopsection

\startsection[title=Where to use profiling]

In \CONTEXT\ there are four places (maybe a few more eventually) where this kind
of control over spacing makes sense:

\startitemize[packed]
\startitem the main text flow in single column mode \stopitem
\startitem multi|-|column mode, especially mixed columns \stopitem
\startitem framed texts, used for all kinds of content \stopitem
\startitem explicitly (balanced) split boxes \stopitem
\stopitemize

Because framed texts are used all over, for instance in tables, it means that if
we provide control over spacing using profiles, many \CONTEXT\ mechanisms can use
it. However, enabling this for all packaging has a significant overhead so it has
to be used with care so that there is no performance hit when it is not used.
Here is an easy example using \type {\framed}:

\starttyping
\framed
  [align=normal,
   profile=fixed,
   frame=off]
  {some text ...}
\stoptyping

For the following examples we define this helper:

\startbuffer
\starttexdefinition demo-profile-1 #1
  \framed
    [align=normal,profile=#1]
    {xxx$\frac{1}{\frac{1}{\frac{1}{2}}}$
     \par
     $\frac{\frac{1}{\frac{1}{2}}}{2}$xxx}
\stoptexdefinition
\stopbuffer

\typebuffer \getbuffer

We apply this to predefined profiles. The macro is called like this:

\starttyping
\texdefinition{demo-profile-1}{fixed}
\stoptyping

\starttexdefinition unexpanded ProfileSteps #1
    \startcombination[nx=5,ny=1]
        {#1\hbox to 6em{\hss\texdefinition{demo-profile-1}            {}\hss}} {\small vbox}
        {#1\hbox to 6em{\hss\texdefinition{demo-profile-1}       {fixed}\hss}} {\small\type{fixed}}
        {#1\hbox to 6em{\hss\texdefinition{demo-profile-1}   {halffixed}\hss}} {\small\type{halffixed}}
        {#1\hbox to 6em{\hss\texdefinition{demo-profile-1}{quarterfixed}\hss}} {\small\type{quarterfixed}}
        {#1\hbox to 6em{\hss\texdefinition{demo-profile-1} {eightsfixed}\hss}} {\small\type{eightsfixed}}
    \stopcombination
\stoptexdefinition

The outcome can depend on the font used: in \in {figure}
[fig:profiling:profiles-fonts] we show Latin Modern, \TEX\ Gyre Pagella and
Dejavu. Because in \CONTEXT\ the line height depends on the bodyfont; each case
is different.

\startplacefigure[reference=fig:profiling:profiles-fonts,title=A few fonts compared.]
    \enabletrackers[profiling.show]
    \startcombination[nx=1,ny=3]
        \startcontent
            \ProfileSteps{\switchtobodyfont[modern]}
        \stopcontent
        \startcaption
            Latin Modern
        \stopcaption
        \startcontent
            \ProfileSteps{\switchtobodyfont[pagella]}
        \stopcontent
        \startcaption
            Pagella
        \stopcaption
        \startcontent
            \ProfileSteps{\switchtobodyfont[dejavu]}
        \stopcontent
        \startcaption
            Dejavu
        \stopcaption
    \stopcombination
    \disabletrackers[profiling.show]
\stopplacefigure

\startplacefigure[reference=fig:profiling:profiles-boxedcolumns,title={Boxed columns without profile.}]
    \enabletrackers[profiling.show]
    \startcombination[nx=1,ny=3]
        \startcontent
            \startboxedcolumns[distance=2.2em,grid=yes,profile=none,frame=on]
                \nl\getbuffer[example-1]
            \stopboxedcolumns
        \stopcontent
        \startcaption
            none on grid
        \stopcaption
        \startcontent
            \startboxedcolumns[distance=2.2em,grid=yes,profile=strict,frame=on]
                \nl\getbuffer[example-1]
            \stopboxedcolumns
        \stopcontent
        \startcaption
            strict on grid
        \stopcaption
        \startcontent
            \startboxedcolumns[distance=2.2em,grid=yes,profile=fixed,frame=on]
                \nl\getbuffer[example-1]
            \stopboxedcolumns
        \stopcontent
        \startcaption
            fixed on grid
        \stopcaption
    \stopcombination
    \disabletrackers[profiling.show]
\stopplacefigure

As mentioned, we need this kind of profiling in multi|-|column typesetting, so
let us have a look at that now. Columns are processed in grid mode but this is
taken into account. We can simulate this by using boxed columns; see \in {figure}
[fig:profiling:profiles-boxedcolumns]. One of the biggest problems is what to do
with the bottom and top of a page or column. This will probably take a bit more
to get right, and likely we will end up with different strategies. We can also
think of special handlers but that will come with a high speed penalty. In the
\type {strict} variant we don't mess with the dimension of a line too much, but
the \type {fixed} alternative will get some more control.

Although using this feature looks promising it is also dangerous. For instance a
side effect can be that interline spacing becomes inconsistent and even ugly. It
really depends on the content. Also, as soon as some grid snapping is used, the
gain becomes less, simply because the solution space is smaller. Then of course
there is the matter of overall look and feel: most documents that need this kind
of magic look bad anyway, so why bother. In this respect it is comparable to
applying protrusion and expansion. There are hardly any combinations of design
and content where micro|-|typography makes sense to use: in prose perhaps, but
not in mixed content. On the other hand, profiling makes more sense in mixed
content than in prose.

Not everything that is possible should be used. In \in {figure}
[fig:profiling:fake-examples-1] we show some fake paragraphs with profiles
applied, the first series (random range~2) has a few excessive snippets, the last
one (random range~5) has many. In \in {figure} [fig:profiling:fake-examples-2] we
show them in a different arrangement. Although there are differences it is hard
to say if the results look better. We scaled down the results and used gray fake
blurs instead of real text in order to get a better impression of the so-called
(overall) grayness of a text.

% \starttexdefinition profile-sample-set #1#2#3
%     \startuseMPgraphic{random-thing}
%         if round(uniformdeviate(10)) = 5 :
%             fill unitsquare enlarged ((1mm,#2mm) randomized(1mm,#1mm)) withcolor \MPcolor{#3} ;
%         else :
%             fill unitsquare enlarged ((1mm,1mm) randomized(1mm,1mm)) withcolor \MPcolor{darkgray} ;
%         fi ;
%         draw origin withpen pencircle scaled 1mm ;
%     \stopuseMPgraphic
%     \setbox#1\hbox\bgroup
%         \dorecurse {500} {%
%             \dontleavehmode
%             \bgroup
%             \obeyMPboxdepth
%             \useMPgraphic{random-thing}%
%             \egroup
%             \hskip 3mm plus 3mm minus 1mm\relax
%         }%
%     \egroup
% \stoptexdefinition

% \starttexdefinition profile-sample-get #1#2
%     \scale
%       [width=\dimexpr(\textwidth-2em)/4\relax]
%       {\framed
%         [offset=overlay,align=normal,profile=#2]
%         {\unhcopy#1}}
% \stoptexdefinition

% \texdefinition{profile-sample-set}{0}{2}{darkred}
% \texdefinition{profile-sample-set}{2}{3}{darkgreen}
% \texdefinition{profile-sample-set}{4}{4}{darkblue}
% \texdefinition{profile-sample-set}{6}{5}{darkyellow}

\startMPdefinitions
    color FakeRed    ; FakeRed    := \MPcolor{darkred} ;
    color FakeGreen  ; FakeGreen  := \MPcolor{darkgreen} ;
    color FakeBlue   ; FakeBlue   := \MPcolor{darkblue} ;
    color FakeYellow ; FakeYellow := \MPcolor{darkyellow} ;
    color FakeGray   ; FakeGray   := \MPcolor{darkgray} ;
    def FakeInColor(expr h,c) =
        if round(uniformdeviate(10)) = 5 :
            fill unitsquare enlarged ((1mm,h*mm) randomized(1mm,h*mm)) withcolor c ;
        else :
            fill unitsquare enlarged ((1mm,1 mm) randomized(1mm,1 mm)) withcolor FakeGray ;
        fi ;
        draw origin withpen pencircle scaled 1mm ;
    enddef ;
\stopMPdefinitions

\starttexdefinition profile-sample-set #1#2#3
    \setbox#1\hbox\bgroup
        \dorecurse {500} {%
            \dontleavehmode
            \bgroup
            \obeyMPboxdepth
            \startMPcode FakeInColor(#2,#3)\stopMPcode
            \egroup
            \hskip 3mm plus 3mm minus 1mm\relax
        }%
    \egroup
\stoptexdefinition

\starttexdefinition profile-sample-get #1#2
    \scale
      [width=\dimexpr(\textwidth-2em)/4\relax]
      {\framed
        [offset=overlay,align=normal,profile=#2]
        {\unhcopy#1}}
\stoptexdefinition

\texdefinition{profile-sample-set}{0}{2}{FakeRed}
\texdefinition{profile-sample-set}{2}{3}{FakeGreen}
\texdefinition{profile-sample-set}{4}{4}{FakeBlue}
\texdefinition{profile-sample-set}{6}{5}{FakeYellow}

\startplacefigure
  [reference=fig:profiling:fake-examples-1,
  title={Some examples, each row has progressively more excessive snippets.}]
    \startcombination[location=top,nx=4,ny=4]
        {\texdefinition{profile-sample-get}{0}{none}}      {\tttf none      / 2}
        {\texdefinition{profile-sample-get}{0}{strict}}    {\tttf strict    / 2}
        {\texdefinition{profile-sample-get}{0}{fixed}}     {\tttf fixed     / 2}
        {\texdefinition{profile-sample-get}{0}{halffixed}} {\tttf halffixed / 2}
        {\texdefinition{profile-sample-get}{2}{none}}      {\tttf none      / 3}
        {\texdefinition{profile-sample-get}{2}{strict}}    {\tttf strict    / 3}
        {\texdefinition{profile-sample-get}{2}{fixed}}     {\tttf fixed     / 3}
        {\texdefinition{profile-sample-get}{2}{halffixed}} {\tttf halffixed / 3}
        {\texdefinition{profile-sample-get}{4}{none}}      {\tttf none      / 4}
        {\texdefinition{profile-sample-get}{4}{strict}}    {\tttf strict    / 4}
        {\texdefinition{profile-sample-get}{4}{fixed}}     {\tttf fixed     / 4}
        {\texdefinition{profile-sample-get}{4}{halffixed}} {\tttf halffixed / 4}
        {\texdefinition{profile-sample-get}{6}{none}}      {\tttf none      / 5}
        {\texdefinition{profile-sample-get}{6}{strict}}    {\tttf strict    / 5}
        {\texdefinition{profile-sample-get}{6}{fixed}}     {\tttf fixed     / 5}
        {\texdefinition{profile-sample-get}{6}{halffixed}} {\tttf halffixed / 5}
    \stopcombination
\stopplacefigure

\startplacefigure
  [reference=fig:profiling:fake-examples-2,
   title={The same examples, rearranged such that each row has  a different profiling variant.}]
    \startcombination[location=top,nx=4,ny=4]
        {\texdefinition{profile-sample-get}{0}{none}}      {\tttf none      / 2}
        {\texdefinition{profile-sample-get}{2}{none}}      {\tttf none      / 3}
        {\texdefinition{profile-sample-get}{4}{none}}      {\tttf none      / 4}
        {\texdefinition{profile-sample-get}{6}{none}}      {\tttf none      / 5}
        {\texdefinition{profile-sample-get}{0}{strict}}    {\tttf strict    / 2}
        {\texdefinition{profile-sample-get}{2}{strict}}    {\tttf strict    / 3}
        {\texdefinition{profile-sample-get}{4}{strict}}    {\tttf strict    / 4}
        {\texdefinition{profile-sample-get}{6}{strict}}    {\tttf strict    / 5}
        {\texdefinition{profile-sample-get}{0}{fixed}}     {\tttf fixed     / 2}
        {\texdefinition{profile-sample-get}{2}{fixed}}     {\tttf fixed     / 3}
        {\texdefinition{profile-sample-get}{4}{fixed}}     {\tttf fixed     / 4}
        {\texdefinition{profile-sample-get}{6}{fixed}}     {\tttf fixed     / 5}
        {\texdefinition{profile-sample-get}{0}{halffixed}} {\tttf halffixed / 2}
        {\texdefinition{profile-sample-get}{2}{halffixed}} {\tttf halffixed / 3}
        {\texdefinition{profile-sample-get}{4}{halffixed}} {\tttf halffixed / 4}
        {\texdefinition{profile-sample-get}{6}{halffixed}} {\tttf halffixed / 5}
    \stopcombination
\stopplacefigure

\stopsection

\startsection[title=Conclusion]

Although profiling seems interesting, in practice it does not have much value in
an automated flow. Ultimately, in the project for which I investigated this
trickery, only in the final stage was some last minute optimization of the
rendering done. We did that by injecting directives. Think of page breaks that
make the result look more balanced. Optimizing image placement happens in an
earlier stage because the text can refer to images like \quotation {in the
picture on the left, we see \unknown}. Controlling profiles is much harder. In
fact, the more clever we are, the harder it gets to beat it when we want an
exception. All these mechanisms: spacing, snapping, profiling, breaking pages,
image placement, to mention a few, have to work together. For projects that
depend on such placement, it might be better to write dedicated mechanisms than
to try to fight with clever built|-|in features.

In practice, probably only the \type {fixed} alternative makes sense and as that
one has a boundary condition similar to (or equal, depending on other settings)
snapping on gridsteps, the end result might not be that different from doing
nothing. In \in {figure} [fig:profiling:fake-examples-3] you see that the vbox
variant is not that bad. And extremely difficult content is unlikely to ever look
perfect unless some manual intervention happens. Therefore, from the perspective
of \quotation {fine points of text typesetting} some local (manual) control might
be more interesting and relevant.

\texdefinition{profile-sample-set}{0}{3}{FakeGreen}
\texdefinition{profile-sample-set}{2}{3}{FakeGreen}
\texdefinition{profile-sample-set}{4}{3}{FakeGreen}

In the end, I didn't need this profiling feature at all: because there are
expectations with respect to how many pages a book should have, typesetting in
columns was not needed. It didn't save that many pages, and the result would
never look that much better, simply because of the type of content. Large images
were also spoiling the game. Nevertheless we will keep profiles in the core and
it might even get extended. One question remains: at what point do we stop adding
such features? The answer would be easier if \TEX\ wasn't so flexible.

\startplacefigure[location=top,reference=fig:profiling:fake-examples-3,title=Three similar random cases.]
    \startcombination[location=top,nx=3,ny=3]
        {\texdefinition{profile-sample-get}{0}{}}       {\tttf vbox   1}
        {\texdefinition{profile-sample-get}{0}{strict}} {\tttf strict 1}
        {\texdefinition{profile-sample-get}{0}{fixed}}  {\tttf fixed  1}
        {\texdefinition{profile-sample-get}{2}{}}       {\tttf vbox   2}
        {\texdefinition{profile-sample-get}{2}{strict}} {\tttf strict 2}
        {\texdefinition{profile-sample-get}{2}{fixed}}  {\tttf fixed  2}
        {\texdefinition{profile-sample-get}{4}{}}       {\tttf vbox   3}
        {\texdefinition{profile-sample-get}{4}{strict}} {\tttf strict 3}
        {\texdefinition{profile-sample-get}{4}{fixed}}  {\tttf fixed  3}
    \stopcombination
\stopplacefigure

\stopsection

\stopchapter

\page \enabledirectives[visualizers.fraction=default]

\stopcomponent

