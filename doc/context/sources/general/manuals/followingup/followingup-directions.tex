% language=us runpath=texruns:manuals/followingup

\startcomponent followingup-directions

\environment followingup-style

\startchapter[title={Directions}]

\startsection[title={Introduction}]

In \LUATEX\ the directional model taken from \OMEGA\ has been upgraded a bit. For
instance in addition to the \type {\*dir} commands we have \type {\*direction}
commands that take a number instead of a keyword. This is a bit more efficient and
consistent as using these keywords was kind of un|-|\TEX. Internally direction
related nodes (text directions) are not whatsits but first class nodes. We also
use a subtype that indicates the push or pop state.

The \LUATEX\ directional model provides four directions which is a subset of the
many that \OMEGA\ provided, indicated by three letters, like \type {TRT} and
\type {LTT}. In the beginning we had them all fixed\footnote {This was doen by
Hartmut by rigorously checking all possible combinations} and thereby implemented
but being in doubt about their usefulness we dropped most of them, just four were
kept. However, in practice only right|-|to|-|left makes sense. Going from top to
bottom in Japanese or Mongolian can also involve glyph rotation, which actually
is not implemented in the engine at all. Spacing and inter|-|character breaks
have to be implemented and in the end one has to combine the results into a
page body. So, in practice you end up with juggling node list and macro magic in
the page builder. The \type {LTL} (number~2) and \type {RTT} (number~3)
directions are not used for serious work. Therefore, in \LUAMETATEX\ the model
has been adapted. In the end, it was not entirely clear anyway what the three
letters were indicating in each direction property (page, body, par, text, math)
as most had no real meaning.

As a side note: if you leave the (not really working well) vertical directions
out of the picture, directional typesetting is not that hard to deal with and has
hardly any consequences for the code. This is because horizontal dimensions are
not affected by direction, only the final ship out is: when a run (wrapped in an
hbox) goes the other way, the backend effectively has to skip the width and then
with each component goes back. Not much more is involved. This means that a
bidirectional engine is rather simple. The complications are more in the way a
macro package deals with it, in relation to the input as well as the layout. The
backend has to do the real work. \footnote {Of course when one hooks in \LUA\
code taking care of direction can be needed!}

\stopsection

\startsection[title=Two directions]

We now have only two directions left: the default left|-|to|-|right (l2r) and
right|-|to|-|left (r2l). They work the same as before and in the backend we can
get rid of the fuzzy parallel and rotation (which actually was just stacking
nodes) heuristics.

Reducing the lot to two directions simplifies some code in the engine. This is
because when calculating dimensions a change in horizontal direction doesn't
influence the width, height and depth in an orthogonal way. Because there are no
longer top|-|down items we don't need to swap the height and or depth with the
width. This also means that we don't need to keep much track of direction
changes. Technically an hpack doesn't need to know its own direction and we can
set it to any value afterwards if we want because the calculation are not
influenced by it; so that also simplified matters.

The \type {\bodydir} and \type {\pagedir} already didn't make much sense, and in
\CONTEXT\ we actually intercepted them, so now they are removed. The body
direction is always left|-|to|-|right and the page direction was only consulted
in the backend code which we no longer have. Another side effect of going with
only two directions is that rules no longer need to carry the direction property:
there is no flipping of width with height and depth needed.

\stopsection

\startsection[title=Four orientations]

Instead of the top|-|bottom variants we now have four orientations plus a bunch
of anchoring options. Of course one could use the backend save, restore and
matrix whatsits but a natural feature makes more sense. Let's start with what
happens normally:

\startbuffer[1]
This is a \LUAMETATEX\ goodie.
\stopbuffer

\startbuffer[2]
\hbox orientation 2{This is a \LUAMETATEX\ goodie.}
\stopbuffer

\startbuffer[3]
This is a \hbox orientation 2{\LUAMETATEX} goodie.
\stopbuffer

\startbuffer[4]
\hbox orientation 2{This is a \hbox orientation 002{\LUAMETATEX} goodie.}
\stopbuffer

\blank{\showstruts\strut}\quad\ruledhbox{\inlinebuffer[1]}\quad{\showstruts\strut}\blank

This line has height and depth. We can rotate this sentence by 180 degrees around
the baseline in which case the depth and height are flipped.

\blank{\showstruts\strut}\quad\ruledhbox{\inlinebuffer[2]}\quad{\showstruts\strut}\blank

or we flip part:

\blank{\showstruts\strut}\quad\ruledhbox{\inlinebuffer[3]}\quad{\showstruts\strut}\blank

or flip nested:

\blank{\showstruts\strut}\quad\ruledhbox{\inlinebuffer[4]}\quad{\showstruts\strut}\blank

but we're talking boxes, so the above examples are defined as:

\typebuffer[1,2,3,4]

The \type {orientation} keyword does the magic here. There are four such
orientations with zero being the default. We saw that two rotates over 180
degrees, so one and three are left for up and down.

\startbuffer[5]
\hbox orientation 0 {\TEX} and
\hbox orientation 1 {\TEX} and
\hbox orientation 2 {\TEX} and
\hbox orientation 3 {\TEX}
\stopbuffer

\blank{\showstruts\strut}\quad\ruledhbox{\inlinebuffer[5]}\quad{\showstruts\strut}\blank

This is codes as:

\typebuffer[5]

The landscape and seascape variants both sit on top of the baseline while the
flipped variant has its depth swapped with the height. Although this would be
enough a bit more control is possible. The number is actually a three byte hex
number:

\starttyping
0x<X><Y><O>
\stoptyping

or in \TEX\ syntax

\starttyping
"<X><Y><O>
\stoptyping

We saw that the last byte regulates the orientation. The first and second one
deal with anchoring horizontally and vertically. The vertical options of the
horizontal variants anchor on the baseline, lower corner, upper corner or center.

\startbuffer[6]
\hbox orientation "002 {\TEX} and
\hbox orientation "012 {\TEX} and
\hbox orientation "022 {\TEX} and
\hbox orientation "032 {\TEX}
\stopbuffer

\typebuffer[6]

\blank{\showstruts\strut}\quad\ruledhbox{\inlinebuffer[6]}\quad{\showstruts\strut}\blank

\startbuffer[7]
\hbox orientation "002 {\TEX} and
\hbox orientation "102 {\TEX} and
\hbox orientation "202 {\TEX} and
\hbox orientation "302 {\TEX} and
\hbox orientation "402 {\TEX}
\stopbuffer

The horizontal options of the horizontal variants anchor in the center, left, right,
halfway left and halfway right.

\typebuffer[7]

\blank{\showstruts\strut}\quad\ruledhbox{\inlinebuffer[7]}\quad{\showstruts\strut}\blank

All combinations will be shown on the next pages, so we suffice with telling that
for the vertical variants we can vertically anchor on the baseline, top, bottom
or center, while horizontally we center, hang left or right, halfway left or
right, and in addition align on the (rotated) baseline left or right.

The orientation has consequences for the dimensions so they are dealt with in the
expected way in constructing lines, paragraphs and pages, but the anchoring is
virtual. As a bonus, we have two extra variants for orientation zero: on top of
baseline or below, with dimensions taken into account.

\startbuffer[8]
\hbox orientation "000 {\TEX} and
\hbox orientation "004 {\TEX} and
\hbox orientation "005 {\TEX}
\stopbuffer

\typebuffer[8]

\blank{\showstruts\strut}\quad\ruledhbox{\inlinebuffer[8]}\quad{\showstruts\strut}\blank

\definecolor[brcolorh][r=1,t=.5,a=1]
\definecolor[brcolord][b=1,t=.5,a=1]
\definecolor[brcolorm][g=1,t=.5,a=1]

\starttexdefinition ShowAnchor
    \blackrule[width=2pt,height=1pt,depth=1pt,color=darkgray]
\stoptexdefinition

\starttexdefinition DemoRule#1#2#3
    \ShowAnchor
    \ruledhbox {
        \hbox orientation "#1#2#3 {
            \blackrule[height=6mm,depth=0mm,width=8mm,color=brcolorh]\kern-8mm\relax
            \blackrule[height=0mm,depth=3mm,width=8mm,color=brcolord]\kern-8mm\relax
            \blackrule[height=2mm,depth=-1mm,width=8mm,color=brcolorm]
        }
    }
    \ShowAnchor
\stoptexdefinition

\starttexdefinition DemoText#1#2#3
    \ShowAnchor
    \ruledhbox{\hbox orientation "#1#2#3 {\red\TEX}}
    \ShowAnchor
\stoptexdefinition

\starttexdefinition DemoSet #1#2
    \startcombination[nx=4,ny=7,width=10cm]
        {#200#1}{\ttxx 0x00#1} {#201#1}{\ttxx 0x01#1} {#202#1}{\ttxx 0x02#1} {#203#1}{\ttxx 0x03#1}
        {#210#1}{\ttxx 0x10#1} {#211#1}{\ttxx 0x11#1} {#212#1}{\ttxx 0x12#1} {#213#1}{\ttxx 0x13#1}
        {#220#1}{\ttxx 0x20#1} {#221#1}{\ttxx 0x21#1} {#222#1}{\ttxx 0x22#1} {#223#1}{\ttxx 0x23#1}
        {#230#1}{\ttxx 0x30#1} {#231#1}{\ttxx 0x31#1} {#232#1}{\ttxx 0x32#1} {#233#1}{\ttxx 0x33#1}
        {#240#1}{\ttxx 0x40#1} {#241#1}{\ttxx 0x41#1} {#242#1}{\ttxx 0x42#1} {#243#1}{\ttxx 0x43#1}
        {#250#1}{\ttxx 0x50#1} {#251#1}{\ttxx 0x51#1} {#252#1}{\ttxx 0x52#1} {#253#1}{\ttxx 0x53#1}
        {#260#1}{\ttxx 0x60#1} {#261#1}{\ttxx 0x61#1} {#262#1}{\ttxx 0x62#1} {#263#1}{\ttxx 0x63#1}
    \stopcombination
\stoptexdefinition

\startplacefigure[title={orientation 0}]\scale[width=\textwidth]{\framed[offset=1cm]{\DemoSet0\DemoRule}}\stopplacefigure
\startplacefigure[title={orientation 1}]\scale[width=\textwidth]{\framed[offset=1cm]{\DemoSet1\DemoRule}}\stopplacefigure
\startplacefigure[title={orientation 2}]\scale[width=\textwidth]{\framed[offset=1cm]{\DemoSet2\DemoRule}}\stopplacefigure
\startplacefigure[title={orientation 3}]\scale[width=\textwidth]{\framed[offset=1cm]{\DemoSet3\DemoRule}}\stopplacefigure

\startplacefigure[title={orientation 0}]\scale[width=\textwidth]{\framed[offset=1cm]{\DemoSet0\DemoText}}\stopplacefigure
\startplacefigure[title={orientation 1}]\scale[width=\textwidth]{\framed[offset=1cm]{\DemoSet1\DemoText}}\stopplacefigure
\startplacefigure[title={orientation 2}]\scale[width=\textwidth]{\framed[offset=1cm]{\DemoSet2\DemoText}}\stopplacefigure
\startplacefigure[title={orientation 3}]\scale[width=\textwidth]{\framed[offset=1cm]{\DemoSet3\DemoText}}\stopplacefigure

% \page

The anchoring can look somewhat confusing but you need to keep in mind that it is
normally only used in very controlled circumstances and not in running text.
Wrapped in macros users don't see the details. We're talking boxes here, so or
instance:

\startbuffer
test\quad
\hbox orientation 3 \bgroup
    \strut test\hbox orientation "002 \bgroup\strut test\egroup test%
\egroup \quad
\hbox orientation 3 \bgroup
    \strut test\hbox orientation "002 \bgroup\strut test\egroup test%
\egroup \quad
\hbox orientation 3 \bgroup
    \strut test\hbox orientation "012 \bgroup\strut test\egroup test%
\egroup \quad
\hbox orientation 3 \bgroup
    \strut test\hbox orientation "022 \bgroup\strut test\egroup test%
\egroup \quad
\hbox orientation 3 \bgroup
    \strut test\hbox orientation "032 \bgroup\strut test\egroup test%
\egroup \quad
\hbox orientation 3 \bgroup
    \strut test\hbox orientation "042 \bgroup\strut test\egroup test%
\egroup
\quad test
\stopbuffer

\typebuffer

gives:

\startlinecorrection[blank]
\ruledhbox\bgroup \showcolorstruts \showboxes \inlinebuffer \egroup
\stoplinecorrection

\stopsection

\startsection[title={Right|-|to|-|left typesetting}]

Another aspect to keep in mind when we transform is the already mentioned
right|-|to|-|left direction. We show some examples where we do things like this:

\starttyping
\hbox{\hbox
    orientation #1
    {\strut abcd}}
\hbox{\hbox
    orientation #1
    to 15mm
    {\strut abcd}}
\hbox{\hbox
    orientation #1
    direction 1
    {\righttoleft\strut abcd}}
\hbox{\hbox
    orientation #1
    direction 1
    to 15mm {\righttoleft\strut abcd}}
\stoptyping

\starttexdefinition TestH #1
    \dontcomplain
    \setbox\scratchbox\hbox{abcd}%
    x\ruledhbox{\hbox orientation #1             to \wd\scratchbox  {\strut abcd}}x\quad
    x\ruledhbox{\hbox orientation #1             to 15mm            {\strut abcd}}x\quad
    x\ruledhbox{\hbox orientation #1 direction 1 to \wd\scratchbox  {\righttoleft\strut abcd}}x\quad
    x\ruledhbox{\hbox orientation #1 direction 1 to 15mm            {\righttoleft\strut abcd}}x%
\stoptexdefinition

\starttexdefinition TestV #1
    \dontcomplain
    \setbox\scratchbox\hbox{abcd}%
    x\ruledvbox{\vbox orientation #1             {\hsize \wd\scratchbox\strut abcd}}x\quad
    x\ruledvbox{\vbox orientation #1             {\hsize           15mm\strut abcd}}x\quad
    x\ruledvbox{\vbox orientation #1 {\righttoleft\hsize \wd\scratchbox\strut abcd}}x\quad
    x\ruledvbox{\vbox orientation #1 {\righttoleft\hsize           15mm\strut abcd}}x%
\stoptexdefinition

\startplacefigure[title={Horizontal boxes.}]
    \startcombination[nx=2,ny=2]
        {\TestH 0} {orientation 0}
        {\TestH 2} {orientation 2}
        {\TestH 1} {orientation 1}
        {\TestH 3} {orientation 3}
    \stopcombination
\stopplacefigure

\startplacefigure[title={Vertical boxes.}]
    \startcombination[nx=2,ny=2]
        {\TestV 0} {orientation 0}
        {\TestV 2} {orientation 2}
        {\TestV 1} {orientation 1}
        {\TestV 3} {orientation 3}
    \stopcombination
\stopplacefigure

\stopsection

\startsection[title={Vertical typesetting}]

I'm no expert on vertical typesetting and have no application for it either. But
from what I've seen vertically positioned glyphs are normally used in rather
straightforward situations. Here I will just give some examples of how
transformations can be used to achieve certain effects. It is no big deal to make
macros or use \LUA\ to apply magic to node lists but it is beyond this description
to discuss that.

Before we fine tune this example we have to discuss another feature. When a \typ
{orientation} keyword is given optionally \type {xoffset} and \type {yoffset} can
be specified. These offsets are {\em not} taken into account when calculating
dimensions. This is different from the offsets (at the \LUA\ end) used in glyphs
because there the vertical offset is taken into account. Here are some examples
of offsets in packaged lists:

\startbuffer
\hbox
    {test 1}
\hbox
    orientation 0
    yoffset  15pt
    xoffset 150pt
    {test}
\vbox
    orientation 0
    {\hbox{test}}
\vbox
    orientation 0
    yoffset  -5pt
    xoffset 130pt
    {\hbox{test}}
\vbox
    orientation 0
    yoffset 2pt
    {\hbox{test}}
\stopbuffer

\typebuffer

\startlinecorrection[blank]
    \start \showboxes \bfd \getbuffer \stop
\stoplinecorrection

In order to demonstrate some hacking, we first define a font that supports
chinese glyphs:

\startbuffer
\definefont[NotoCJK][NotoSansCJKtc-Regular*default @ 24pt]
\stopbuffer

\typebuffer \getbuffer

We put some text in a horizontal box; it doesn't show up in verbatim but you
get the idea nevertheless:

\startbuffer
\hbox{\NotoCJK 通用规范汉字表}
\stopbuffer

\typebuffer

\startlinecorrection[blank]
    \start \showboxes \getbuffer \stop
\stoplinecorrection

Let's now rotate this line of text:

\startbuffer[1]
\hbox orientation 1 {\NotoCJK 通用规范汉字表}
\stopbuffer

\typebuffer[1]

The result is shown in a while. Because we also need to rotate the glyphs we
deconstruct the box.

\startbuffer[2]
\hbox orientation 1 \bgroup \NotoCJK %
    \vbox {\hbox {通}}%
    \vbox {\hbox {用}}%
    \vbox {\hbox {规}}%
    \vbox {\hbox {test}}%
    \vbox {\hbox {范}}%
    \vbox {\hbox {汉}}%
    \vbox {\hbox {字}}%
    \vbox {\hbox {表}}%
\egroup
\stopbuffer

\typebuffer[2]

Next we rotate the glyphs.

\startbuffer[3]
\hbox orientation 1 \bgroup \NotoCJK %
    \vbox orientation 3 {\hbox {通}}%
    \vbox orientation 3 {\hbox {用}}%
    \vbox orientation 3 {\hbox {规}}%
    \vbox orientation 0 {\hbox {test}}%
    \vbox orientation 3 {\hbox {范}}%
    \vbox orientation 3 {\hbox {汉}}%
    \vbox orientation 3 {\hbox {字}}%
    \vbox orientation 3 {\hbox {表}}%
\egroup
\stopbuffer

\typebuffer[3]

This still looks bad so we kick in some offsets and glue:

\startbuffer[4]
\dontleavehmode\hbox orientation 1 \bgroup \NotoCJK
    \vbox
        orientation 0 yoffset -.1ex
        {\hbox orientation 3 {通}}\hskip.2ex
    \vbox
        orientation 0 yoffset -.1ex
        {\hbox orientation 3 {用}}\hskip.2ex
    \vbox
        orientation 0 yoffset -.1ex
        {\hbox orientation 3 {规}}\hskip.6ex
    \vbox
        {\hbox             {test}}\hskip.2ex
    \vbox
        orientation 0 yoffset -.1ex
        {\hbox orientation 3 {范}}\hskip.2ex
    \vbox
        orientation 0 yoffset -.1ex
        {\hbox orientation 3 {汉}}\hskip.2ex
    \vbox
        orientation 0 yoffset -.1ex
        {\hbox orientation 3 {字}}\hskip.2ex
    \vbox
        orientation 0 yoffset -.1ex
        {\hbox orientation 3 {表}}\hskip.2ex
\egroup
\stopbuffer

\typebuffer[4]

Now we're ready to compare the results

\startlinecorrection[blank]
    \startcombination[9*1]
        {\showboxes \getbuffer[1]} {1}
        {\showboxes \getbuffer[2]} {2}
        {\showboxes \getbuffer[3]} {3}
        {\showboxes \getbuffer[4]} {4}
        {\quad}{}
        {\getbuffer[1]} {1}
        {\getbuffer[2]} {2}
        {\getbuffer[3]} {3}
        {\getbuffer[4]} {4}
    \stopcombination
\stoplinecorrection

This could of course also be done with traditional kerns, raising and|/|or
lowering and messing around with dimensions. It's just that when manipulating
such rather complex constructs a little help (and efficiency) makes a difference,
also at the \LUA\ end. Of course one can argue the result but all is
programmable in the end.

\stopsection

\startsection[title={Considerations}]

Just in case you wonder if using these offsets is better than using normal kerning
and shifting, in practice offsets are not more efficient. Let's compare the
alternatives. We go from most to least efficient.

\starttyping
\setbox\scratchbox\hpack{}
\boxxoffset\scratchbox\scratchdimen
\boxyoffset\scratchbox\scratchdimen
\stoptyping

This sets the offsets and by setting them we also trigger the transform. Scanning
is fast and so is setting them. One million times expanding this takes (as
reference) 0.73 seconds on my current machine.

\starttyping
\setbox\scratchbox\hpack
  orientation \zerocount
  xoffset     \scratchdimen
  yoffset     \scratchdimen
  {}
\stoptyping

This takes a bit more time, 1.11 seconds, because the keywords have to be scanned
which happens on a token by token base.

\starttyping
\setbox\scratchbox\hpack{}
\scratchheight\ht\scratchbox
\scratchdepth\dp\scratchbox
\setbox\scratchbox\hpack
  {\kern\scratchdimen
   \raise\scratchdimen\box\scratchbox
   \kern\scratchdimen}
\ht\scratchbox\scratchheight
\dp\scratchbox\scratchdepth
\stoptyping

Now we're up to 1.69 seconds for the million expansions. Not only do we have some
parsing going on, but we also have assignments and extra packing, which means
calculations taking place.

\starttyping
\setbox\scratchbox\hpack{}
\scratchwidth\wd\scratchbox
\scratchheight\ht\scratchbox
\scratchdepth\dp\scratchbox
\setbox\scratchbox\hpack
  {\kern\scratchdimen
   \raise\scratchdimen\box\scratchbox}
\wd\scratchbox\scratchwidth
\ht\scratchbox\scratchheight
\dp\scratchbox\scratchdepth
\stoptyping

This variant is about as fast, as I measured 1.72 seconds. So, compared to the
0.73 seconds for the first variant, is this better? Does it help when we look at
our existing macros and adapt them?

Normally we don't have an empty box and normally we use \type {\hbox} because we
want the content to be processed. And a million times building a list and
processing content (which means runs over the list) will make the differences
in timing become noise. Add to that garbage collection (in \LUA) and memory
management (in \TEX) and it even becomes unpredictable. Seeing differences of
a factor two in such timings is no exception.

Another aspect is the parsing. When these commands are wrapped in macros we're
talking expanding tokens which is pretty fast. When it comes from the input file
a conversion to tokens has to happen too. And we will never see millions of such
sequences in a source file.

The backend also plays a role. Handling a kern or shift is more efficient than
analyzing transforms (and offsets) especially in a \LUA\ variant. But on the
other hand, we don't have an extra wrapping in a box so that actually saves work.

So, before a \CONTEXT\ user thinks \quotation {Let's update macros and change
policy.}, just consider staying with proven good old \TEX\ approaches. These
features are mostly meant for efficient low level manipulations as discussed in
relation to for instance handling scripts. In the rather large \CONTEXT\ code
base there are really only a few places where it will make code look nicer, but
there I don't expect an impact on performance.

\stopsection

\startsection[title={Integration}]

How these mechanisms are used depends on ones needs and the macro package used.
It makes no sense to cook up generic solutions because integration in a macro
package is too different. But anyhow we'll give an example of some (definitely
non optimized) \LUA\ magic.

\startbuffer
\startluacode
local glyph_id = node.id("glyph")
local fontdata = fonts.hashes.identifiers -- assumes generic font loader

local function is_vertical(c)
  -- more ranges matter but this will do here
  return c >= 0x04E00 and c <= 0x09FFF
end

function document.go_vertical(boxnumber)
  local box = tex.getbox(boxnumber)
  local n   = box.list
  while n do
    if n.id == glyph_id and is_vertical(n.char) then
      local o = .2 * fontdata[n.font].parameters.xheight
      local prev, next = n.prev, n.next
      n.next, n.prev = nil, nil
      local l = nodes.new("hlist")
      l.list = n
      local w, h, d = n.width, n.height, n.depth
      if prev then
          prev.next, l.prev = l, prev
      else
          box.list = l
      end
      if next then
          l.next, next.prev = next, l
      end
      l.width, l.height, l.depth  = h + d + o, w, 0
      l.orientation = 0x003
      l.xoffset, l.yoffset = o/2, -o/2
      l.hoffset, l.doffset = h, d - o
      n = next
    else
      n = n.next
    end
  end
end
\stopluacode
\stopbuffer

\typebuffer \getbuffer

We will use some other magic that we won't discuss here which relates to handling
scripts. For Hangul one needs to inject breakpoints and if needed also glue
between characters. The script environment does this. We also need to bump the
interline spacing. First we define a regular text helper and an auxiliary box.

\startbuffer[1]
\unexpanded\def\stripe#1%
  {\hbox orientation 0 yoffset .2\exheight{\strut #1}}

\newbox\MyVerticalBox
\stopbuffer

\typebuffer[1]

Next we fill that box with some mix of text (I have no clue what, as I just
copied it from some web page).

\startbuffer[2a]
\setbox\MyVerticalBox\hbox \bgroup
    \NotoCJK
    \startscript[hangul]%
    \dorecurse{20}{通用规范汉字表 \stripe{test #1} }%
    \unskip % remove last space
    \stopscript
\egroup
\stopbuffer

\typebuffer[2a]

We then apply the \LUA\ magic to the result:

\startbuffer[3a]
\ctxlua{document.go_vertical(\number\MyVerticalBox)}
\stopbuffer

\typebuffer[3a]

and finally assemble the result:

\startbuffer[4a]
\ruledvbox orientation 1 to \textwidth \bgroup
    \setupinterlinespace[40pt]
    \hsize .95\textheight
    \unhbox\MyVerticalBox
    \vfill
\egroup
\stopbuffer

\typebuffer[4a]

The result is shown in \in {figure} [fig:verticalmagic-1]. Of course this
approach is not that user friendly but it just serves as example. In \CONTEXT\ we
can follow a different route. First we define a new font feature. It is probably
clear that we need some code elsewhere that does something useful with this
information, but I will nos show this as it is rather \CONTEXT\ dependent.

\startbuffer[2b]
\definefontfeature
  [vertical]
  [vertical={%
    orientation=3,%
    down=.1,%
    right=.1,%
    ranges={%
      cjkcompatibility,%
      cjkcompatibilityforms,%
      cjkcompatibilityideographs,%
      cjkcompatibilityideographssupplement,%
      cjkradicalssupplement,%
    % cjkstrokes,%
      cjksymbolsandpunctuation,%
      cjkunifiedideographs,%
      cjkunifiedideographsextensiona,%
      cjkunifiedideographsextensionb,%
      cjkunifiedideographsextensionc,%
      cjkunifiedideographsextensiond,%
      cjkunifiedideographsextensione,%
      cjkunifiedideographsextensionf,%
    }%
  }]
\stopbuffer

\typebuffer[2b]

We apply this feature to a font:

\startbuffer[3b]
\definefont
  [NotoCJKvertical]
  [NotoSansCJKtc-Regular*default,vertical @ 24pt]
\stopbuffer

\typebuffer[3b]

\startbuffer[4b]
\setbox\MyVerticalBox\hbox\bgroup
    \NotoCJKvertical
    \startscript[hangul]%
    \dorecurse{20}{通用规范汉字表 \stripe{test #1} }%
    \unskip
    \stopscript
\egroup
\stopbuffer

\typebuffer[4b]

\startbuffer[5b]
\ruledvbox orientation 1 to \textwidth \bgroup
    \setupinterlinespace[40pt]
    \hsize .95\textheight
    \unhbox\MyVerticalBox
    \vfill
\egroup
\stopbuffer

\typebuffer[5b]

The result is shown in \in {figure} [fig:verticalmagic-2]. Again this approach is
not that user friendly but it already is a bit easier.

\startplacefigure[reference=fig:verticalmagic-1,title={Some vertical magic using manipulations.}]
    \getbuffer[1,2a,3a,4a]
\stopplacefigure

\startplacefigure[reference=fig:verticalmagic-2,title={Some vertical magic using fonts.}]
    \getbuffer[1,2b,3b,4b,5b]
\stopplacefigure

\stopsection

\stopchapter

\stopcomponent
