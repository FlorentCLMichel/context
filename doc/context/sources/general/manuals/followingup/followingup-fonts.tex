% language=us

\startcomponent followingup-fonts

\environment followingup-style

\startchapter[title={Scaled fonts}]

\startsection[title={History}]

The infrastructure for fonts makes for a large part of the code of any \TEX\
macro package. We have to go back in time to understand why. When \TEX\ showed
up, fonts were collections of bitmaps and measures. There were at most 256 glyphs
in a font and in order to do its job, \TEX\ needed to know (and still needs to
know) the width, height and depth of glyphs. If you want ligatures it also needs
to know how to construct them from the input and when you want kerning there has
to be additional information about what neighboring glyphs need a kern in
between. Math is yet another subtask that demands extra information, like chains
of glyphs that grow in size and if needed even recipes of how to construct large
shapes from smaller ones.

Fonts come in sizes, and for instance Latin Modern has quite a few variants where
the shapes are adapted to the size. This means that when you need a 9pt regular
shape alongside a 12pt one, two fonts have to be loaded. This is quite visible in
math where we have three related sizes: text, script and scriptscript, grouped in
so called families.

Plenty has been written (in various documents that come with \CONTEXT) about how
this all works together and how it impacts the design of the system, so here I
just give a short summary of what a font system has to deal with.

\startitemize
\startitem
    In a bodyfont setup different sizes (9pt, 10pt, 12pt) can have their own
    specific set of fonts. This can result in quite some definitions that relate
    to the style, like regular, bold, italic, bolditalic, slanted and
    boldslanted, etc. When possible loading the fonts is delayed.
\stopitem
\startitem
    Some font designs have different shapes per bodyfont size. A minor
    complication is that when one is missing some heuristic best match choice
    might be needed.
\stopitem
\startitem
    Within a bodyfont size we distinguish size variants. We can go smaller (x and
    xx), for instance when we use sub- and superscripts in text, or we can go
    larger, for instance in titling (a, b, c, d, \unknown). Fortunately most of
    the loading of these can be delayed too.
\stopitem
\startitem
    When instances are not available, scaling can be used, as happens for
    instance with 11pt in Computer Modern. Actually, this is why in \CONTEXT\ we
    default to 12pt, because the scaled versions didn't look as nice as the
    others.
\stopitem
\startitem
    Special features, like smallcaps or oldstyle numerals, can demand their own
    definitions. More loading and automatic definitions can be triggered by sizes
    needed in for instance scripts and titling.
\stopitem
\startitem
    A document can have a mixed setup, that is: use different font designs in one
    document, so some kind of namespace subsystem is needed.
\stopitem
\startitem
    In traditional eight bit engine the dependency of hyphenation on the encoding
    of a font can result in the necessity to load a font multiple times in
    different encodings, something that depends on the language mix used.
\stopitem
\startitem
    In the more modern \OPENTYPE\ fonts combinations of features can demand
    additional instances (one can thing of language|/|script combinations,
    substitutions in base mode, special effects like boldening, color fonts,
    etc.).
\stopitem
\startitem
    Math is complicated by the fact that alphabets come from different fonts
    (sometimes a font has several alphabets which means that some mapping can be
    needed). Operating on the size, shape, encoding and style axes puts some
    demands on the font system. Add to this (often) partial (due to lack of
    fonts) bold support and it gets even more complicated.
\stopitem
\startitem
    There is additional math auto|-|definition and loading code for the sizes
    used in text scripts and titling.
\stopitem
\stopitemize

All this has resulted in a pretty complex subsystem. Although going \OPENTYPE\
(and emulated \OPENTYPE\ with \TYPEONE\ fonts as we do in \MKIV) removes some
complications, like encodings, it does also add complexity because of the many
possible font features, either or not dependent on script and language.

Also, in order to let a font subsystem not impact performance to much, let alone
extensive memory usage, the \CONTEXT\ font subsystem is rather optimized. The
biggest burden comes from fonts that have a dynamic (adaptive) definition because
then we need to do quite a bit of testing per font switch, but even that has
always been rather fast.

\stopsection

\startsection[title={Reality}]

In \MKIV\ and therefore also in \LMTX\ more font magic happens. The initial node
lists that makes up a box or paragraph can get manipulated in several ways and
often fonts are involved. The mentioned font features can be defines static (as
part of the definition) or dynamic (resolved on the spot at the cost of some
overhead). Characters can be remapped, fonts can be replaced. The math subsystem
in \MKIV\ was different right from the start: we use a limited number of families
(regular, bold, l2r and r2l), and stay abstract till the moment we need to deal
with the specific alphabets. But still, in \MKIV, we have the families with three
fonts.

In the \LUAMETATEX\ manual we show some math magic and we do so for different
fonts. As a side effect, we set up half a dozen bodyfont collections. Even with
delayed and shared font loading, we end up with 158 instances but quite a bit of
them are math fonts, at least six per bodyfont size: regular and bold (boldened)
text, script and scriptscript. Of course most are just copies with different
scaling that reuse already loaded resources.

If we look at the math fonts that we use today, there is however quite some
overlap. It starts with a text font. From that script and scriptscript variants
are derived, but often these variants use many text size related shapes too. Some
shapes get alternatives (from the \type {ssty} feature), and the whole clone get
scaled. But, much of the logic of for instance extensibles is the same.

A similar situation happens with for instance large \CJK\ fonts: there are hardly
any advanced features involved there, so any size is basically a copy with scaled
dimensions, and these fonts can be real huge!

Actually, when we talk about features, in many cases in \CONTEXT\ you don't
define them as part of the font. For instance small caps can best be triggered by
using a dynamic feature: applied to a specific stretch of text. When the font
handler does its work there are actually four cases: no features get applied,
something that for instance happens with most monospaced fonts, base mode is
used, which means that the \TEX\ machinery takes care of constructing ligatures
and injecting kerns, and node mode, where \LUA\ handles the features. The fourth
case is a special case of node mode where a different features set is applied.
\footnote {We also have so called plug mode where an external rendered can do the
work but that one is only around for some experiments during Idris' font
development.} At the cost of some extra overhead (for each node mode run) dynamic
features are quite powerful and save quite some memory and definitions. The
overhead comes from much more testing regarding the font we deal with because
suddenly the same font can demand different treatments, depending on what dynamic
features is active.

Although the font handling is responsible for much of the time spent in \LUA, it is
still reasonable given what has to be done. Actually, because we have an extensible
system, it's often the extensions that takes additional runtime. Flexibility comes
at a price.

\stopsection

\startsection[title={Progress}]

At some point I started playing with realtime glyph scaling. Here realtime means that
it doesn't depend on the font definition. To get an idea, here is an example:

\startbuffer
test {\glyphxscale 2500 test} test
\stopbuffer

\typebuffer

\getbuffer

The glyphs in the current font get scaled horizontally without the need for an extra
font instance. Now, this kind of trickery puts some constraints on the font handling,
as is demonstrated in the next example. We use Latin Modern because that font has
all these ligatures:

\startbuffer
\definedfont[lmroman-regular*default]%
e{\glyphxscale 2500 ff}icient
ef{\glyphxscale 2500 f}icient
ef{\glyphxscale 2500 fi}cient
e{\glyphxscale 2500 ffi}cient
\stopbuffer

\typebuffer

{\getbuffer}

So, in order to deal with this kind of scaling, we now operate not only on the
font (id) and dynamic feature axes, but also on the scales of which we have three
variants: glyph scale, glyph xscale and glyph y scale). There is actually also a
state dimension but we leave that for now. Think of flagging glyphs as initial or
final. This brings the number of axis to six. It is important to stress that in
these examples the same font instance is used!

Just for the record: there are several approaches to switching fonts possible but
for now we stick to a simple font id switch plus glyph scale settings at the
\TEX\ end. A variant would be to introduce a new mechanism where id's and scales
go together but for now I see no real gain in that.

\stopsection

\startsection[title={Math}]

Given what is written in the previous sections, a logical question is if we can
apply scaling to math and the answer is \quotation {yes, we can}. But we actually
go a bit further and that is partly due to some other properties of the engine.

From \PDFTEX\ the \LUATEX\ engines inherited character protrusion and glyph
expansions, aka hz. However, where in \PDFTEX\ copies of the font are made that
carry the expanded dimensions, in \LUATEX\ at some point this was replaced by an
expansion field in the glyph and kern nodes. So, instead of changing the font id
of expanded glyphs, the same id is used but with the applied expansion factor set
in the glyph. A side effect was that in places where dimensions are needed, calls
to functions that calculate the expanded widths on request (as these can change
during linebreak calculations) in combination with accessing font dimensions
directly, we now always use accessor functions. This level of abstraction is even
more present in \LUAMETATEX. This means that we have an uniform interface to
fonts and as a side effect scaling is to be dealt with in only a few places in
the code.

Now, in math we have a few more complications. First of all, we have three sizes
to consider and we also have lots of parameters that depend on the size. But, as
I wanted to be able to apply scaling to math, the whole machinery was also
abstracted in a way that, at the cost of some extra overhead, made it easier to
work with scaled glyph properties. This means that we can stick to loading only
one bodyfont size of math (note that each math family has three sizes, where the
script and script sizes can have different, fine tuned, shapes) and just scale
that on demand.

Once all that was in place it was a logical next step to see if we could actually
stick to one instance. Because in \LUAMETATEX\ we try to load fonts efficiently
we store only the minimally needed information at the \TEX\ end. A font with no
math therefore has less data per glyph. Again, this brings some abstraction that
actually helped to implement the one instance mechanism. A math glyph has
optional lists of increasing sizes and vertical or horizontal extensibles. So
what got added was an optional chain of smaller sizes. If a character has 3
different glyphs for the three sizes, the text glyph has a pointer to the script
glyph which in turn has a pointer to the scriptscript glyph. This means that when
the math engine needs a specific character at a given size (text, script,
scriptscript) we just follow that chain.

In an \OPENTYPE\ math font the script and scriptscript sizes are specified as
percentages of the text size. So, when the dimensions of a glyph are needed, we
just scale on the fly. Again this adds some overhead but I'm pretty sure that no
user will notice this.

So, to summarize: if we need a character at scriptscript size, we access the text
size glyph, check for a pointer to a script size, go there, and again check for a
smaller size. We just use what fits the bill. And, when we need dimensions we
just scale. In order to scale we need the relative size, so we need to set that
up when we load the font. Because in \CONTEXT\ we also can assemble a virtual
\OPENTYPE\ font from \TYPEONE\ fonts, it was actually that (old) compatibility
feature that took most time to adapt, not so much because it is complicates but
because in \LMTX\ we have to bypass some advanced loading mechanisms.

The end result is that for math we now only need to define two fonts per bodyfont
setup: regular and bold at the natural scale (normally 10pt) and we share these
for all sizes. It is because of this and what we describe in the next section
that the 158 instances for the \LUAMETATEX\ manual can now be brought down to 30.

\stopsection

\startsection[title={Text}]

Sharing instances in text mode is relatively simple, although we do have to keep
in mind that these are extra axis when dealing with font features: two
neighboring glyphs with the same font id and dynamics but with different scales
are effectively from different fonts.

Another complication is that when we use font fallbacks (read: take missing
glyphs from another font) we no longer have a dedicated instance but use a shared
one. This in itself if not a problem but we do need to handle specified relative
scales. This was not that hard to patch in \CONTEXT\ \LMTX.

We can enforce aggressive font sharing with:

\starttyping
\enableexperiments[fonts.compact]
\stoptyping

After that we often use less instances. Just to give an idea, on the \LUAMETATEX\
manual we get these stats:

\starttyping
290 pages, 10.8 sec, 292M lua, 99M tex, 158 instances
290 pages,  9.5 sec, 149M lua, 35M tex,  30 instances
\stoptyping

So, we win on all fronts when we use this glyph scale mechanism. The magic
primitive that deals with this is \type {\glyphscale} which accepts a number,
where \type {1200} and \type {1.2} both mean scaling to 20\percent\ more than
normal. But one can best not use this primitive directly.

A specific font can be defined using the \type {\definefont} command. In \LMTX\ a
regular scaler can be followed by two scale factors. The nest example
demonstrates this:

\startbuffer
\definefont[FooA][Serif*default @ 12pt 1800 500]
\definefont[FooB][Serif*default @ 12pt 0.85 0.4]
\definefont[FooC][Serif*default @ 12pt]

\definetweakedfont[runwider] [xscale=1.5]
\definetweakedfont[runtaller][yscale=2.5,xscale=.8,yoffset=-.2ex]

{\FooA test test \runwider test test \runtaller test test}\par
{\FooB test test \runwider test test \runtaller test test}\par
{\FooC test test \runwider test test \runtaller test test}\par
\stopbuffer

\typebuffer

We also use the new \type {\definetweakedfont} command here. This example not
only shows the two scales but also introduces the offset.

\getbuffer

Here is another one:

\startbuffer
\definetweakedfont[squeezed] [xscale=0.9]
\definetweakedfont[squeezed] [xscale=0.9]

\startlines
$a = b^2 + \sqrt{c}$
{\squeezed $a = b^2 + \sqrt{c}$}
\stoplines
\stopbuffer

\typebuffer

\getbuffer

Watch this:

\startbuffer
\startcombination[4*1]
    {\bTABLE
        \bTR \bTD foo \eTD \bTD[style=\squeezed] $x = 1$ \eTD \eTR
        \bTR \bTD oof \eTD \bTD[style=\squeezed] $x = 2$ \eTD \eTR
     \eTABLE}
    {local}
    {\bTABLE[style=squeezed]
        \bTR \bTD $x = 1$ \eTD \bTD  $x = 3$ \eTD \eTR
        \bTR \bTD $x = 2$ \eTD \bTD  $x = 4$ \eTD \eTR
     \eTABLE}
    {global}
    {\bTABLE[style=\squeezed\squeezed]
        \bTR \bTD $x = 1$ \eTD \bTD  $x = 3$ \eTD \eTR
        \bTR \bTD $x = 2$ \eTD \bTD  $x = 4$ \eTD \eTR
     \eTABLE}
    {multiple}
\stopcombination
\stopbuffer

\typebuffer

\startlinecorrection
\getbuffer
\stoplinecorrection

An additional style parameter is also honored:

\startbuffer
\definetweakedfont[MyLargerFontA][scale=2000,style=bold]
test {\MyLargerFontA test} test
\stopbuffer

\typebuffer

This gives:

\getbuffer

Just for the record: the Latin Modern fonts, when set up to use design sizes, will
still use the specific size related files.

\stopsection

\startsection[title={Hackery}]

You can actually use negative scale values, as is demonstrated in the following
code:

\startbuffer
\bTABLE[align=middle]
  \bTR
    \bTD a{\glyphxscale  1000 \glyphyscale  1000 bc}d \eTD
    \bTD a{\glyphxscale  1000 \glyphyscale -1000 bc}d \eTD
    \bTD a{\glyphxscale -1000 \glyphyscale -1000 bc}d \eTD
    \bTD a{\glyphxscale -1000 \glyphyscale  1000 bc}d \eTD
  \eTR
  \bTR
    \bTD \tttf +1000 +1000 \eTD
    \bTD \tttf +1000 -1000 \eTD
    \bTD \tttf -1000 -1000 \eTD
    \bTD \tttf -1000 +1000 \eTD
  \eTR
\eTABLE
\stopbuffer

\typebuffer

gives:

\startlinecorrection
\getbuffer
\stoplinecorrection

Glyphs can have offsets and these are used for implementing \OPENTYPE\ features.
However, they are also available at the \TEX\ end. Take this

\startbuffer
\ruledhbox{
    \ruledhbox{\glyph yoffset 1ex options 0 123}
    \ruledhbox{\glyph xoffset .5em yoffset 1ex options "C0 123}
    \ruledhbox{oeps{\glyphyoffset 1ex \glyphxscale 800 \glyphyscale\glyphxscale oeps}oeps}
}
\stopbuffer

This example demonstrates that the \type {\glyph} primitive takes keywords: \type
{xoffset}, \type {yoffset}, \type {xscale}, \type {yscale}, \type {options},
\type {font} and \type {id} where the last two take a font identifier or font id
(an positive number).

\startbuffer
\samplefile{jojomayer}
{\glyphyoffset .8ex
 \glyphxscale 700 \glyphyscale\glyphxscale
 \samplefile{jojomayer}}
{\glyphyscale\numexpr3*\glyphxscale/2\relax
 \samplefile{jojomayer}}
{\glyphyoffset -.2ex
 \glyphxscale 500 \glyphyscale\glyphxscale
 \samplefile{jojomayer}}
\samplefile{jojomayer}
\stopbuffer

\typebuffer

To quote Jojo Mayer:

\startnarrower
\darkred \getbuffer
\stopnarrower

Keep in mind that this can interfere badly with font feature processing which also
used offsets. It might work out okay, or not.

The scales, as mentioned, works with pseudo scales but that is sometimes a bit
cumbersome. This is why a special \type {\numericscale} primitive has been
introduced.

\startbuffer
1200 : \the\numericscale1200
1.20 : \the\numericscale1.200
\stopbuffer

\typebuffer

Both these lines produce the same integer:

\startlines\tttf
\getbuffer
\stoplines

You can do strange things with these primitives but keep in mind that you can
also waste the defaults.

\startbuffer[definition]
\def\UnKernedTeX
  {T%
   {\glyph xoffset -.2ex yoffset -.4ex `E}%
   {\glyph xoffset -.4ex options "60 `X}}
\stopbuffer

\startbuffer[example]
We use \UnKernedTeX\ and {\bf \UnKernedTeX} and {\bs \UnKernedTeX}:
the slanted version could use some more left shifting of the E.
\stopbuffer

\typebuffer[definition,example]

This gives is the \TEX\ logos (but of course we normally use the real ones
instead).

\startnarrower
    \darkred \getbuffer[definition,example]
\stopnarrower

Because offsets are (also) used for handling font features like mark and cursive
placement as well as special inter|-|character positioning, the above is
suboptimal. Here is a better alternative:

\startbuffer[definition]
\def\UnKernedTeX
  {T\glyph left .2ex raise -.4ex `E\glyph left .4ex `X\relax}
\stopbuffer

\typebuffer[definition]

The result is the same:

\startnarrower
    \darkgreen \getbuffer[definition,example]
\stopnarrower

But anyway: don't over|-|do it. We could deal with such cases for decades without
these fancy new features. The next example show margins in action:

\startlinecorrection
\bTABLE[align=middle,width=.33\textwidth]
    \bTR
        \bTD \showglyphs \multiply\glyphscale by 2 <{\darkgray \glyph                      `M}>\eTD
        \bTD \showglyphs \multiply\glyphscale by 2 <{\darkgray \glyph raise 3pt            `M}>\eTD
        \bTD \showglyphs \multiply\glyphscale by 2 <{\darkgray \glyph raise -3pt           `M}>\eTD
    \eTR
    \bTR[frame=off]
        \bTD \tttf            \eTD
        \bTD \tttf raise  3pt \eTD
        \bTD \tttf raise -3pt \eTD
    \eTR
    \bTR
        \bTD \showglyphs \multiply\glyphscale by 2 <{\darkgray \glyph left  3pt            `M}>\eTD
        \bTD \showglyphs \multiply\glyphscale by 2 <{\darkgray \glyph           right  2pt `M}>\eTD
        \bTD \showglyphs \multiply\glyphscale by 2 <{\darkgray \glyph left  3pt right  2pt `M}>\eTD
    \eTR
    \bTR[frame=off]
        \bTD \tttf left  3pt           \eTD
        \bTD \tttf           right  2pt\eTD
        \bTD \tttf left  3pt right  2pt\eTD
    \eTR
    \bTR
        \bTD \showglyphs \multiply\glyphscale by 2 <{\darkgray \glyph left -3pt            `M}>\eTD
        \bTD \showglyphs \multiply\glyphscale by 2 <{\darkgray \glyph           right -2pt `M}>\eTD
        \bTD \showglyphs \multiply\glyphscale by 2 <{\darkgray \glyph left -3pt right -2pt `M}>\eTD
    \eTR
    \bTR[frame=off]
        \bTD \tttf left -3pt            \eTD
        \bTD \tttf           right -2pt \eTD
        \bTD \tttf left -3pt right -2pt \eTD
    \eTR
\eTABLE
\stoplinecorrection

Here is another way of looking at it:

\startbuffer
\glyphscale 4000
\vl\glyph                        `M\vl\quad
\vl\glyph raise  .2em            `M\vl\quad
\vl\glyph left   .3em            `M\vl\quad
\vl\glyph             right  .2em`M\vl\quad
\vl\glyph left  -.2em right -.2em`M\vl\quad
\vl\glyph raise -.2em right  .4em`M\vl
\stopbuffer

\typebuffer

The raise as well as left and right margins are taken into account when calculating the
dimensions of a glyph.

{\getbuffer}


\stopsection

\startsection[title={Implementation}]

Discussing the implementation in the engine makes no sense here, also because
details might change. However, it is good to know that quite some properties
travel with the glyph nodes, for instance the scales and offsets. The dimensions
(width, height and depth) are not stored in the glyph node but calculated from
the font, scales and optionally the offsets and expansion factor. One problem is
that the more clever (and nice) solutions we cook up, the more it might impact
performance. So, I will delay some experiments till I have a more powerful
machine.

One reason for {\em not} storing the dimensions in a glyph node is that we often
copy those nodes or change character fields in the font handler and we definitely
don't want the wrong dimensions there. At that moment, offsets and margin fields
don't reflect features yet, so copying them is no big deal. However, dimensions
are rather character bound so every time a character is set, we also would have
to set the dimensions. Even worse, when we can set them, the question arises if
they actually were already set explicitly. So, this is a can of worms we're not
going to open: the basic width, height and depth of the glyph as specified in the
font is used and combines with actual dimensions (likely already scaled according
the glyph scales) in offset and margin fields.

Now, I have to admit that especially playing with using margins to glyphs instead
of font kerns is more of an experiment to see what the consequences are than a
necessity, but what would be the joy of \TEX\ without such experiments. And as
usual, in \CONTEXT\ these will become options in the font handler that one can
either or not enable.

\stopsection

\stopchapter

\stopcomponent

% sample file

\setuplayout[topspace=1cm,header=1cm] \setuplayout[middle]

\setupbodyfont[pagella,10pt]

\setupalign[verytolerant,stretch]

\setupwhitespace[big]

\starttext

\startbuffer
\definetweakedfont[bfe][xscale=2000,yscale=6000,style=bf]

\setuphead[chapter][style=\bfe]

\dostepwiserecurse {10} {2020} {10} {
    \title{Here we go #1!}
    \start
    \glyphxscale#1\relax
    \glyphyscale#1\relax
    \setupinterlinespace
    \samplefile{ward}%
    \start
    \bf
    \samplefile{ward}%
    \glyphxscale\numexpr(#1*2)/3\relax
    \glyphyscale#1\relax
    \samplefile{ward}%
    \par
    \stop
    \def\TEST{y = \sqrt{x^2 + 1}^3}%
    \dontleavehmode
    \ruledhbox{\glyphxscale #1                 \glyphyscale #1                 $\left{\TEST\right}$}
    \ruledhbox{\glyphxscale #1                 \glyphyscale \numexpr#1*3\relax $\left{\TEST\right}$}
    \ruledhbox{\glyphxscale \numexpr#1/2\relax \glyphyscale \numexpr#1*2\relax $\left{\TEST\right}$}
    \par
    \stop
    \page
}
\stopbuffer

\getbuffer

% As often there is a musical timestamp to this text: discovering Julia Hofer
% (bass, came there via checking out (of course) Leland Sklar, and now i have to
% watch a bunch of Victor Wooten videos too) and Neon Vine's (play and combine
% anything) on YT. Both examples of where (imo extensive) practicing real
% instruments can bring you with todays diverse equipment. These are interesting
% times (it's now last days of 2020).
%
% The initial \definescaledfont was replaced by \definetweakedfont in the first
% week of Januari 2021, when I watched the joyful covers of Peter Gabriels
% "Steam" by Yoyoka which shows that the future belongs to the real young kids
% (so much explosive creativity!) and not to the old folks.
