% language=uk

\environment luatex-style

\startcomponent luatex-nodes

\startchapter[reference=nodes,title={Nodes}]

\section{\LUA\ node representation}

\topicindex {nodes}

\TEX's nodes are represented in \LUA\ as userdata object with a variable set of
fields. In the following syntax tables, such the type of such a userdata object
is represented as \syntax {<node>}.

The current return value of \type {node.types()} is:
\startluacode
    for id, name in table.sortedhash(node.types()) do
        context.type(name)
        context(" (%s), ",id)
    end
    context.removeunwantedspaces()
    context.removepunctuation()
\stopluacode
. % period

The \prm {lastnodetype} primitive is \ETEX\ compliant. The valid range is still
$[-1,15]$ and glyph nodes (formerly known as char nodes) have number~0 while
ligature nodes are mapped to~7. That way macro packages can use the same symbolic
names as in traditional \ETEX. Keep in mind that these \ETEX\ node numbers are
different from the real internal ones and that there are more \ETEX\ node types
than~15.

You can ask for a list of fields with the \type {node.fields} (which takes an id)
and for valid subtypes with \type {node.subtypes} (which takes a string because
eventually we might support more used enumerations).

The \type {node.values} function reports some used values. Valid arguments are
\nod {dir}, \type {direction}, \nod {glue}, \whs {pdf_literal}, \whs
{pdf_action}, \whs {pdf_window} and \whs {color_stack}. Keep in mind that the
setters normally expect a number, but this helper gives you a list of what
numbers matter. For practical reason the \type {pagestate} values are also
reported with this helper.

\subsection{Attributes}

\topicindex {attributes}

The newly introduced attribute registers are non|-|trivial, because the value
that is attached to a node is essentially a sparse array of key|-|value pairs. It
is generally easiest to deal with attribute lists and attributes by using the
dedicated functions in the \type {node} library, but for completeness, here is
the low|-|level interface.

\subsubsection{attribute_list nodes}

\topicindex {nodes+attributes}

An \nod {attribute_list} item is used as a head pointer for a list of attribute
items. It has only one user-visible field:

\starttabulate[|l|l|p|]
\DB field       \BC type \BC explanation \NC \NR
\TB
\NC \type{next} \NC node \NC pointer to the first attribute \NC \NR
\LL
\stoptabulate

\subsubsection{\nod {attr} nodes}

A normal node's attribute field will point to an item of type \nod
{attribute_list}, and the \type {next} field in that item will point to the first
defined \quote {attribute} item, whose \type {next} will point to the second
\quote {attribute} item, etc.

\starttabulate[|l|l|p|]
\DB field         \BC type   \BC explanation \NC \NR
\TB
\NC \type{next}   \NC node   \NC pointer to the next attribute \NC \NR
\NC \type{number} \NC number \NC the attribute type id \NC \NR
\NC \type{value}  \NC number \NC the attribute value \NC \NR
\LL
\stoptabulate

As mentioned it's better to use the official helpers rather than edit these
fields directly. For instance the \type {prev} field is used for other purposes
and there is no double linked list.

\subsection{Main text nodes}

\topicindex {nodes+text}

These are the nodes that comprise actual typesetting commands. A few fields are
present in all nodes regardless of their type, these are:

\starttabulate[|l|l|p|]
\DB field          \BC type   \BC explanation \NC \NR
\TB
\NC \type{next}    \NC node   \NC the next node in a list, or nil \NC \NR
\NC \type{id}      \NC number \NC the node's type (\type {id}) number \NC \NR
\NC \type{subtype} \NC number \NC the node \type {subtype} identifier \NC \NR
\LL
\stoptabulate

The \type {subtype} is sometimes just a stub entry. Not all nodes actually use
the \type {subtype}, but this way you can be sure that all nodes accept it as a
valid field name, and that is often handy in node list traversal. In the
following tables \type {next} and \type {id} are not explicitly mentioned.

Besides these three fields, almost all nodes also have an \type {attr} field, and
there is a also a field called \type {prev}. That last field is always present,
but only initialized on explicit request: when the function \type {node.slide()}
is called, it will set up the \type {prev} fields to be a backwards pointer in
the argument node list. By now most of \TEX's node processing makes sure that the
\type {prev} nodes are valid but there can be exceptions, especially when the
internal magic uses a leading \nod {temp} nodes to temporarily store a state.

\subsubsection{\nod {hlist} nodes}

\starttabulate[|l|l|p|]
\DB field             \BC type   \BC explanation \NC \NR
\TB
\NC \type{subtype}    \NC number \NC \showsubtypes{list} \NC \NR
\NC \type{attr}       \NC node   \NC list of attributes \NC \NR
\NC \type{width}      \NC number \NC the width of the box \NC \NR
\NC \type{height}     \NC number \NC the height of the box \NC \NR
\NC \type{depth}      \NC number \NC the depth of the box \NC \NR
\NC \type{shift}      \NC number \NC a displacement perpendicular to the character
                                     progression direction \NC \NR
\NC \type{glue_order} \NC number \NC a number in the range $[0,4]$, indicating the
                                     glue order \NC \NR
\NC \type{glue_set}   \NC number \NC the calculated glue ratio \NC \NR
\NC \type{glue_sign}  \NC number \NC 0 = \type {normal}, 1 = \type {stretching}, 2 =
                                     \type {shrinking} \NC \NR
\NC \type{head/list}  \NC node   \NC the first node of the body of this list \NC \NR
\NC \type{dir}        \NC string \NC the direction of this box, see~\in [dirnodes] \NC \NR
\LL
\stoptabulate

\topicindex {nodes+lists}
\topicindex {lists}

A warning: never assign a node list to the \type {head} field unless you are sure
its internal link structure is correct, otherwise an error may result.

Note: the field name \type {head} and \type {list} are both valid. Sometimes it
makes more sense to refer to a list by \type {head}, sometimes \type {list} makes
more sense.

\subsubsection{\nod {vlist} nodes}

\topicindex {nodes+lists}
\topicindex {lists}

This node is similar to \nod {hlist}, except that \quote {shift} is a displacement
perpendicular to the line progression direction, and \quote {subtype} only has
the values 0, 4, and~5.

\subsubsection{\nod {rule} nodes}

\topicindex {nodes+rules}
\topicindex {rules}

Contrary to traditional \TEX, \LUATEX\ has more \prm {rule} subtypes because we
also use rules to store reuseable objects and images. User nodes are invisible
and can be intercepted by a callback.

\starttabulate[|l|l|p|]
\DB field          \BC type   \BC explanation \NC \NR
\TB
\NC \type{subtype} \NC number \NC \showsubtypes {rule} \NC \NR
\NC \type{attr}    \NC node   \NC list of attributes \NC \NR
\NC \type{width}   \NC number \NC the width of the rule where the special value
                                  $-1073741824$ is used for \quote {running} glue dimensions \NC \NR
\NC \type{height}  \NC number \NC the height of the rule (can be negative) \NC \NR
\NC \type{depth}   \NC number \NC the depth of the rule (can be negative) \NC \NR
\NC \type{left}    \NC number \NC shift at the left end (also subtracted from width) \NC \NR
\NC \type{right}   \NC number \NC (subtracted from width) \NC \NR
\NC \type{dir}     \NC string \NC the direction of this rule, see~\in[dirnodes] \NC \NR
\NC \type{index}   \NC number \NC an optional index that can be referred to \NC \NR
\LL
\stoptabulate

The \type {left} and type {right} keys are somewhat special (and experimental).
When rules are auto adapting to the surrounding box width you can enforce a shift
to the right by setting \type {left}. The value is also subtracted from the width
which can be a value set by the engine itself and is not entirely under user
control. The \type {right} is also subtracted from the width. It all happens in
the backend so these are not affecting the calculations in the frontend (actually
the auto settings also happen in the backend). For a vertical rule \type {left}
affects the height and \type {right} affects the depth. There is no matching
interface at the \TEX\ end (although we can have more keywords for rules it would
complicate matters and introduce a speed penalty.) However, you can just
construct a rule node with \LUA\ and write it to the \TEX\ input.

\subsubsection{\nod {ins} nodes}

\topicindex {nodes+insertions}
\topicindex {insertions}

This node relates to the \prm {insert} primitive.

\starttabulate[|l|l|p|]
\DB field            \BC type   \BC explanation \NC \NR
\TB
\NC \type{subtype}   \NC number \NC the insertion class \NC \NR
\NC \type{attr}      \NC node   \NC list of attributes \NC \NR
\NC \type{cost}      \NC number \NC the penalty associated with this insert \NC \NR
\NC \type{height}    \NC number \NC height of the insert \NC \NR
\NC \type{depth}     \NC number \NC depth of the insert \NC \NR
\NC \type{head/list} \NC node   \NC the first node of the body of this insert \NC \NR
\LL
\stoptabulate

There is a set of extra fields that concern the associated glue: \type {width},
\type {stretch}, \type {stretch_order}, \type {shrink} and \type {shrink_order}.
These are all numbers.

A warning: never assign a node list to the \type {head} field unless you are sure
its internal link structure is correct, otherwise an error may be result. You can use
\type {list} instead (often in functions you want to use local variable swith similar
names and both names are equally sensible).

\subsubsection{\nod {mark} nodes}

\topicindex {nodes+marks}
\topicindex {marks}

This one relates to the \prm {mark} primitive.

\starttabulate[|l|l|p|]
\DB field          \BC type   \BC explanation \NC \NR
\TB
\NC \type{subtype} \NC number \NC unused \NC \NR
\NC \type{attr}    \NC node   \NC list of attributes \NC \NR
\NC \type{class}   \NC number \NC the mark class \NC \NR
\NC \type{mark}    \NC table  \NC a table representing a token list \NC \NR
\LL
\stoptabulate

\subsubsection{\nod {adjust} nodes}

\topicindex {nodes+adjust}
\topicindex {adjust}

This node comes from \prm {vadjust} primitive.

\starttabulate[|l|l|p|]
\DB field            \BC type   \BC explanation \NC \NR
\TB
\NC \type{subtype}   \NC number \NC \showsubtypes{adjust} \NC \NR
\NC \type{attr}      \NC node   \NC list of attributes \NC \NR
\NC \type{head/list} \NC node   \NC adjusted material \NC \NR
\LL
\stoptabulate

A warning: never assign a node list to the \type {head} field unless you are sure
its internal link structure is correct, otherwise an error may be result.

\subsubsection{\nod {disc} nodes}

\topicindex {nodes+discretionaries}
\topicindex {discretionaries}

The \prm {discretionary} and \prm {-}, the \type {-} character but also the
hyphenation mechanism produces these nodes.

\starttabulate[|l|l|p|]
\DB field          \BC type   \BC explanation \NC \NR
\TB
\NC \type{subtype} \NC number \NC \showsubtypes{disc} \NC \NR
\NC \type{attr}    \NC node   \NC list of attributes \NC \NR
\NC \type{pre}     \NC node   \NC pointer to the pre|-|break text \NC \NR
\NC \type{post}    \NC node   \NC pointer to the post|-|break text \NC \NR
\NC \type{replace} \NC node   \NC pointer to the no|-|break text \NC \NR
\NC \type{penalty} \NC number \NC the penalty associated with the break, normally
                                  \prm {hyphenpenalty} or \prm {exhyphenpenalty} \NC \NR
\LL
\stoptabulate

The subtype numbers~4 and~5 belong to the \quote {of-f-ice} explanation given
elsewhere. These disc nodes are kind of special as at some point they also keep
information about breakpoints and nested ligatures.

The \type {pre}, \type {post} and \type {replace} fields at the \LUA\ end are in
fact indirectly accessed and have a \type {prev} pointer that is not \type {nil}.
This means that when you mess around with the head of these (three) lists, you
also need to reassign them because that will restore the proper \type {prev}
pointer, so:

\starttyping
pre = d.pre
-- change the list starting with pre
d.pre = pre
\stoptyping

Otherwise you can end up with an invalid internal perception of reality and
\LUATEX\ might even decide to crash on you. It also means that running forward
over for instance \type {pre} is ok but backward you need to stop at \type {pre}.
And you definitely must not mess with the node that \type {prev} points to, if
only because it is not really an node but part of the disc data structure (so
freeing it again might crash \LUATEX).

\subsubsection{\nod {math} nodes}

\topicindex {nodes+math}
\topicindex {math+nodes}

Math nodes represent the boundaries of a math formula, normally wrapped into
\type {$} signs.

\starttabulate[|l|l|p|]
\DB field           \BC type   \BC explanation \NC \NR
\TB
\NC \type{subtype}  \NC number \NC \showsubtypes{math} \NC \NR
\NC \type{attr}     \NC node   \NC list of attributes \NC \NR
\NC \type{surround} \NC number \NC width of the \prm {mathsurround} kern \NC \NR
\LL
\stoptabulate

There is a set of extra fields that concern the associated glue: \type {width},
\type {stretch}, \type {stretch_order}, \type {shrink} and \type {shrink_order}.
These are all numbers.

\subsubsection{\nod {glue} nodes}

\topicindex {nodes+glue}
\topicindex {glue}

Skips are about the only type of data objects in traditional \TEX\ that are not a
simple value. They are inserted when \TEX\ sees a space in the text flow but also
by \prm {hskip} and \prm {vskip}. The structure that represents the glue
components of a skip is called a \nod {glue_spec}, and it has the following
accessible fields:

\starttabulate[|l|l|p|]
\DB field                \BC type   \BC explanation \NC \NR
\TB
\NC \type{width}         \NC number \NC the horizontal or vertical displacement \NC \NR
\NC \type{stretch}       \NC number \NC extra (positive) displacement or stretch amount \NC \NR
\NC \type{stretch_order} \NC number \NC factor applied to stretch amount \NC \NR
\NC \type{shrink}        \NC number \NC extra (negative) displacement or shrink amount\NC \NR
\NC \type{shrink_order}  \NC number \NC factor applied to shrink amount \NC \NR
\LL
\stoptabulate

The effective width of some glue subtypes depends on the stretch or shrink needed
to make the encapsulating box fit its dimensions. For instance, in a paragraph
lines normally have glue representing spaces and these stretch of shrink to make
the content fit in the available space. The \type {effective_glue} function that
takes a glue node and a parent (hlist or vlist) returns the effective width of
that glue item.

A \nod {glue_spec} node is a special kind of node that is used for storing a set
of glue values in registers. Originally they were also used to store properties
of glue nodes (using a system of reference counts) but we now keep these
properties in the glue nodes themselves, which gives a cleaner interface to \LUA.

The indirect spec approach was in fact an optimization in the original \TEX\
code. First of all it can save quite some memory because all these spaces that
become glue now share the same specification (only the reference count is
incremented), and zero testing is also a bit faster because only the pointer has
to be checked (this is no longer true for engines that implement for instance
protrusion where we really need to ensure that zero is zero when we test for
bounds). Another side effect is that glue specifications are read|-|only, so in
the end copies need to be made when they are used from \LUA\ (each assignment to
a field can result in a new copy). So in the end the advantages of sharing are
not that high (and nowadays memory is less an issue, also given that a glue node
is only a few memory words larger than a spec).

\starttabulate[|l|l|p|]
\DB field          \BC type   \BC explanation \NC \NR
\TB
\NC \type{subtype} \NC number \NC \showsubtypes{glue} \NC \NR
\NC \type{attr}    \NC node   \NC list of attributes \NC \NR
\NC \type{leader}  \NC node   \NC pointer to a box or rule for leaders \NC \NR
\LL
\stoptabulate

In addition there are the \type {width}, \type {stretch} \type {stretch_order},
\type {shrink}, and \type {shrink_order} fields. Note that we use the key \type
{width} in both horizontal and vertical glue. This suits the \TEX\ internals well
so we decided to stick to that naming.

A regular word space also results in a \type {spaceskip} subtype (this used to be
a \type {userskip} with subtype zero).

\subsubsection{\nod {kern} nodes}

\topicindex {nodes+kerns}
\topicindex {kerns}

The \prm {kern} command creates such nodes but for instance the font and math
machinery can also add them.

\starttabulate[|l|l|p|]
\DB field          \BC type   \BC explanation \NC \NR
\TB
\NC \type{subtype} \NC number \NC \showsubtypes{kern} \NC \NR
\NC \type{attr}    \NC node   \NC list of attributes \NC \NR
\NC \type{kern}    \NC number \NC fixed horizontal or vertical advance \NC \NR
\LL
\stoptabulate

\subsubsection{\nod {penalty} nodes}

\topicindex {nodes+penalty}
\topicindex {penalty}

The \prm {penalty} command is one that generates these nodes.

\starttabulate[|l|l|p|]
\DB field          \BC type   \BC explanation \NC \NR
\TB
\NC \type{subtype} \NC number \NC \showsubtypes{penalty} \NC \NR
\NC \type{attr}    \NC node   \NC list of attributes \NC \NR
\NC \type{penalty} \NC number \NC the penalty value \NC \NR
\LL
\stoptabulate

The subtypes are just informative and \TEX\ itself doesn't use them. When you
run into an \type {linebreakpenalty} you need to keep in mind that it's a
accumulation of \type {club}, \type{widow} and other relevant penalties.

\subsubsection[glyphnodes]{\nod {glyph} nodes}

\topicindex {nodes+glyph}
\topicindex {glyph}

These are probably the mostly used nodes and although you can push on tin the
list with for instance \prm {char} \TEX\ will normally do it for you when it
considers some input to be text.

\starttabulate[|l|l|p|]
\DB field                   \BC type    \BC explanation \NC \NR
\TB
\NC \type{subtype}          \NC number  \NC bitfield \NC \NR
\NC \type{attr}             \NC node    \NC list of attributes \NC \NR
\NC \type{char}             \NC number  \NC the chatacter index in the font \NC \NR
\NC \type{font}             \NC number  \NC the font identifier \NC \NR
\NC \type{lang}             \NC number  \NC the language identifier \NC \NR
\NC \type{left}             \NC number  \NC the frozen \type {\lefthyphenmnin} value \NC \NR
\NC \type{right}            \NC number  \NC the frozen \type {\righthyphenmnin} value \NC \NR
\NC \type{uchyph}           \NC boolean \NC the frozen \prm {uchyph} value \NC \NR
\NC \type{components}       \NC node    \NC pointer to ligature components \NC \NR
\NC \type{xoffset}          \NC number  \NC a virtual displacement in horizontal direction \NC \NR
\NC \type{yoffset}          \NC number  \NC a virtual displacement in vertical direction \NC \NR
%NC \type{xadvance}         \NC number  \NC an additional advance after the glyph (experimental) \NC \NR
\NC \type{width}            \NC number  \NC the (original) width of the character \NC \NR
\NC \type{height}           \NC number  \NC the (original) height of the character\NC \NR
\NC \type{depth}            \NC number  \NC the (original) depth of the character\NC \NR
\NC \type{expansion_factor} \NC number  \NC the to be applied expansion_factor \NC \NR
\LL
\stoptabulate

The \type {width}, \type {height} and \type {depth} values are read|-|only. The
\type {expansion_factor} is assigned in the par builder and used in the backend.

A warning: never assign a node list to the components field unless you are sure
its internal link structure is correct, otherwise an error may be result. Valid
bits for the \type {subtype} field are:

\starttabulate[|c|l|]
\DB bit \BC meaning   \NC \NR
\TB
\NC 0   \NC character \NC \NR
\NC 1   \NC ligature  \NC \NR
\NC 2   \NC ghost     \NC \NR
\NC 3   \NC left      \NC \NR
\NC 4   \NC right     \NC \NR
\LL
\stoptabulate

See \in {section} [charsandglyphs] for a detailed description of the \type
{subtype} field.

The \type {expansion_factor} has been introduced as part of the separation
between font- and backend. It is the result of extensive experiments with a more
efficient implementation of expansion. Early versions of \LUATEX\ already
replaced multiple instances of fonts in the backend by scaling but contrary to
\PDFTEX\ in \LUATEX\ we now also got rid of font copies in the frontend and
replaced them by expansion factors that travel with glyph nodes. Apart from a
cleaner approach this is also a step towards a better separation between front-
and backend.

The \type {is_char} function checks if a node is a glyph node with a subtype still
less than 256. This function can be used to determine if applying font logic to a
glyph node makes sense. The value \type {nil} gets returned when the node is not
a glyph, a character number is returned if the node is still tagged as character
and \type {false} gets returned otherwise. When nil is returned, the id is also
returned. The \type {is_glyph} variant doesn't check for a subtype being less
than 256, so it returns either the character value or nil plus the id. These
helpers are not always faster than separate calls but they sometimes permit
making more readable tests. The \type {uses_font} helpers takes a node
and font id and returns true when a glyph or disc node references that font.

\subsubsection{\nod {boundary} nodes}

\topicindex {nodes+boundary}
\topicindex {boundary}

This node relates to the \prm {noboundary}, \prm {boundary}, \prm
{protrusionboundary} and \prm {wordboundary} primitives.

\starttabulate[|l|l|p|]
\DB field          \BC type   \BC explanation \NC \NR
\TB
\NC \type{subtype} \NC number \NC \showsubtypes{boundary} \NC \NR
\NC \type{attr}    \NC node   \NC list of attributes \NC \NR
\NC \type{value}   \NC number \NC values 0--255 are reserved \NC \NR
\LL
\stoptabulate

\subsubsection{\nod {local_par} nodes}

\topicindex {nodes+paragraphs}
\topicindex {paragraphs}

This node in inserted a the start of a paragraph. You should not mess
too much with this one.

\starttabulate[|l|l|p|]
\DB field                  \BC type   \BC explanation \NC \NR
\TB
\NC \type{attr}            \NC node   \NC list of attributes \NC \NR
\NC \type{pen_inter}       \NC number \NC local interline penalty (from \lpr {localinterlinepenalty}) \NC \NR
\NC \type{pen_broken}      \NC number \NC local broken penalty (from \lpr {localbrokenpenalty}) \NC \NR
\NC \type{dir}             \NC string \NC the direction of this par. see~\in [dirnodes] \NC \NR
\NC \type{box_left}        \NC node   \NC the \lpr {localleftbox} \NC \NR
\NC \type{box_left_width}  \NC number \NC width of the \lpr {localleftbox} \NC \NR
\NC \type{box_right}       \NC node   \NC the \lpr {localrightbox} \NC \NR
\NC \type{box_right_width} \NC number \NC width of the \lpr {localrightbox} \NC \NR
\LL
\stoptabulate

A warning: never assign a node list to the \type {box_left} or \type {box_right}
field unless you are sure its internal link structure is correct, otherwise an
error may be result.

\subsubsection[dirnodes]{\nod {dir} nodes}

\topicindex {nodes+direction}
\topicindex {directions}

Direction nodes mark parts of the running text that need a change of direction and \
the \prm {textdir} command generates them.

\starttabulate[|l|l|p|]
\DB field        \BC type   \BC explanation \NC \NR
\TB
\NC \type{attr}  \NC node   \NC list of attributes \NC \NR
\NC \type{dir}   \NC string \NC the direction (but see below) \NC \NR
\NC \type{level} \NC number \NC nesting level of this direction whatsit \NC \NR
\LL
\stoptabulate

Direction specifiers are three|-|letter combinations of \type {T}, \type {B},
\type {R}, and \type {L}. These are built up out of three separate items:

\startitemize[packed]
\startitem
    the first  is the direction of the \quote{top}   of paragraphs.
\stopitem
\startitem
    the second is the direction of the \quote{start} of lines.
\stopitem
\startitem
    the third  is the direction of the \quote{top}   of glyphs.
\stopitem
\stopitemize

However, only four combinations are accepted: \type {TLT}, \type {TRT}, \type
{RTT}, and \type {LTL}. Inside actual \nod {dir} nodes, the representation of
\nod {dir} is not a three|-|letter but a combination of numbers. When printed the
direction is indicated by a \type {+} or \type {-}, indicating whether the value
is pushed or popped from the direction stack.

\subsubsection{\nod {margin_kern} nodes}

\topicindex {nodes+paragraphs}
\topicindex {paragraphs}
\topicindex {protrusion}

Margin kerns result from protrusion.

\starttabulate[|l|l|p|]
\DB field          \BC type   \BC explanation \NC \NR
\TB
\NC \type{subtype} \NC number \NC \showsubtypes{margin_kern} \NC \NR
\NC \type{attr}    \NC node   \NC list of attributes \NC \NR
\NC \type{width}   \NC number \NC the advance of the kern \NC \NR
\NC \type{glyph}   \NC node   \NC the glyph to be used \NC \NR
\LL
\stoptabulate

\subsection{Math noads}

\topicindex {nodes+math}
\topicindex {math+nodes}

These are the so||called \quote {noad}s and the nodes that are specifically
associated with math processing. Most of these nodes contain subnodes so that the
list of possible fields is actually quite small. First, the subnodes:

\subsubsection{Math kernel subnodes}

Many object fields in math mode are either simple characters in a specific family
or math lists or node lists. There are four associated subnodes that represent
these cases (in the following node descriptions these are indicated by the word
\type {<kernel>}).

The \type {next} and \type {prev} fields for these subnodes are unused.

\subsubsubsection{\nod {math_char} and \nod {math_text_char} subnodes}

\starttabulate[|l|l|p|]
\DB field       \BC type   \BC explanation \NC \NR
\TB
\NC \type{attr} \NC node   \NC list of attributes \NC \NR
\NC \type{char} \NC number \NC the character index \NC \NR
\NC \type{fam}  \NC number \NC the family number \NC \NR
\LL
\stoptabulate

The \nod {math_char} is the simplest subnode field, it contains the character
and family for a single glyph object. The \nod {math_text_char} is a special
case that you will not normally encounter, it arises temporarily during math list
conversion (its sole function is to suppress a following italic correction).

\subsubsubsection{\nod {sub_box} and \nod {sub_mlist} subnodes}

\starttabulate[|l|l|p|]
\DB field            \BC type \BC explanation \NC \NR
\TB
\NC \type{attr}      \NC node \NC list of attributes \NC \NR
\NC \type{head/list} \NC node \NC list of nodes \NC \NR
\LL
\stoptabulate

These two subnode types are used for subsidiary list items. For \nod {sub_box},
the \type {head} points to a \quote {normal} vbox or hbox. For \nod {sub_mlist},
the \type {head} points to a math list that is yet to be converted.

A warning: never assign a node list to the \type {head} field unless you are sure
its internal link structure is correct, otherwise an error may be result.

\subsubsubsection{\nod {delim} subnodes}

There is a fifth subnode type that is used exclusively for delimiter fields. As
before, the \type {next} and \type {prev} fields are unused.

\starttabulate[|l|l|p|]
\DB field             \BC type   \BC explanation \NC \NR
\TB
\NC \type{attr}       \NC node   \NC list of attributes \NC \NR
\NC \type{small_char} \NC number \NC character index of base character \NC \NR
\NC \type{small_fam}  \NC number \NC family number of base character \NC \NR
\NC \type{large_char} \NC number \NC character index of next larger character \NC \NR
\NC \type{large_fam}  \NC number \NC family number of next larger character \NC \NR
\LL
\stoptabulate

The fields \type {large_char} and \type {large_fam} can be zero, in that case the
font that is sed for the \type {small_fam} is expected to provide the large
version as an extension to the \type {small_char}.

\subsubsection{Math core nodes}

First, there are the objects (the \TEX book calls then \quote {atoms}) that are
associated with the simple math objects: ord, op, bin, rel, open, close, punct,
inner, over, under, vcent. These all have the same fields, and they are combined
into a single node type with separate subtypes for differentiation.

Some noads have an option field. The values in this bitset are common:

\starttabulate[|l|r|]
\DB meaning         \BC bits                      \NC \NR
\TB
\NC set             \NC               \type{0x08} \NC \NR
\NC internal        \NC \type{0x00} + \type{0x08} \NC \NR
\NC internal        \NC \type{0x01} + \type{0x08} \NC \NR
\NC axis            \NC \type{0x02} + \type{0x08} \NC \NR
\NC no axis         \NC \type{0x04} + \type{0x08} \NC \NR
\NC exact           \NC \type{0x10} + \type{0x08} \NC \NR
\NC left            \NC \type{0x11} + \type{0x08} \NC \NR
\NC middle          \NC \type{0x12} + \type{0x08} \NC \NR
\NC right           \NC \type{0x14} + \type{0x08} \NC \NR
\NC no sub script   \NC \type{0x21} + \type{0x08} \NC \NR
\NC no super script \NC \type{0x22} + \type{0x08} \NC \NR
\NC no script       \NC \type{0x23} + \type{0x08} \NC \NR
\LL
\stoptabulate

\subsubsubsection{simple \nod {noad} nodes}

\starttabulate[|l|l|p|]
\DB field          \BC type        \BC explanation \NC \NR
\TB
\NC \type{subtype} \NC number      \NC \showsubtypes{noad} \NC \NR
\NC \type{attr}    \NC node        \NC list of attributes \NC \NR
\NC \type{nucleus} \NC kernel node \NC base \NC \NR
\NC \type{sub}     \NC kernel node \NC subscript \NC \NR
\NC \type{sup}     \NC kernel node \NC superscript \NC \NR
\NC \type{options} \NC number      \NC bitset of rendering options \NC \NR
\LL
\stoptabulate

\subsubsubsection{\nod {accent} nodes}

\starttabulate[|l|l|p|]
\DB field             \BC type        \BC explanation \NC \NR
\TB
\NC \type{subtype}    \NC number      \NC \showsubtypes{accent} \NC \NR
\NC \type{nucleus}    \NC kernel node \NC base \NC \NR
\NC \type{sub}        \NC kernel node \NC subscript \NC \NR
\NC \type{sup}        \NC kernel node \NC superscript \NC \NR
\NC \type{accent}     \NC kernel node \NC top accent \NC \NR
\NC \type{bot_accent} \NC kernel node \NC bottom accent \NC \NR
\NC \type{fraction}   \NC number      \NC larger step criterium (divided by 1000) \NC \NR
\LL
\stoptabulate

\subsubsubsection{\nod {style} nodes}

\starttabulate[|l|l|p|]
\DB field        \BC type   \BC explanation    \NC \NR
\TB
\NC \type{style} \NC string \NC contains the style \NC \NR
\LL
\stoptabulate

There are eight possibilities for the string value: one of \type {display},
\type {text}, \type {script}, or \type {scriptscript}. Each of these can have
be prefixed by \type {cramped}.

\subsubsubsection{\nod {choice} nodes}

\starttabulate[|l|l|p|]
\DB field               \BC type \BC explanation \NC \NR
\TB
\NC \type{attr}         \NC node \NC list of attributes \NC \NR
\NC \type{display}      \NC node \NC list of display size alternatives \NC \NR
\NC \type{text}         \NC node \NC list of text size alternatives \NC \NR
\NC \type{script}       \NC node \NC list of scriptsize alternatives \NC \NR
\NC \type{scriptscript} \NC node \NC list of scriptscriptsize alternatives \NC \NR
\LL
\stoptabulate

Warning: never assign a node list to the \type {display}, \type {text}, \type
{script}, or \type {scriptscript} field unless you are sure its internal link
structure is correct, otherwise an error may be result.

\subsubsubsection{\nod {radical} nodes}

\starttabulate[|l|l|p|]
\DB field          \BC type           \BC explanation \NC \NR
\TB
\NC \type{subtype} \NC number         \NC \showsubtypes{radical} \NC \NR
\NC \type{attr}    \NC node           \NC list of attributes \NC \NR
\NC \type{nucleus} \NC kernel node    \NC base \NC \NR
\NC \type{sub}     \NC kernel node    \NC subscript \NC \NR
\NC \type{sup}     \NC kernel node    \NC superscript \NC \NR
\NC \type{left}    \NC delimiter node \NC \NC \NR
\NC \type{degree}  \NC kernel node    \NC only set by \lpr {Uroot} \NC \NR
\NC \type{width}   \NC number         \NC required width \NC \NR
\NC \type{options} \NC number         \NC bitset of rendering options \NC \NR
\LL
\stoptabulate

Warning: never assign a node list to the \type {nucleus}, \type {sub}, \type
{sup}, \type {left}, or \type {degree} field unless you are sure its internal
link structure is correct, otherwise an error may be result.

\subsubsubsection{\nod {fraction} nodes}

\starttabulate[|l|l|p|]
\DB field          \BC type           \BC explanation \NC \NR
\TB
\NC \type{attr}    \NC node           \NC list of attributes \NC \NR
\NC \type{width}   \NC number         \NC (optional) width of the fraction \NC \NR
\NC \type{num}     \NC kernel node    \NC numerator \NC \NR
\NC \type{denom}   \NC kernel node    \NC denominator \NC \NR
\NC \type{left}    \NC delimiter node \NC left side symbol \NC \NR
\NC \type{right}   \NC delimiter node \NC right side symbol \NC \NR
\NC \type{middle}  \NC delimiter node \NC middle symbol \NC \NR
\NC \type{options} \NC number         \NC bitset of rendering options \NC \NR
\LL
\stoptabulate

Warning: never assign a node list to the \type {num}, or \type {denom} field
unless you are sure its internal link structure is correct, otherwise an error
may be result.

\subsubsubsection{\nod {fence} nodes}

\starttabulate[|l|l|p|]
\DB field          \BC type           \BC explanation \NC \NR
\TB
\NC \type{subtype} \NC number         \NC \showsubtypes{fence} \NC \NR
\NC \type{attr}    \NC node           \NC list of attributes \NC \NR
\NC \type{delim}   \NC delimiter node \NC delimiter specification \NC \NR
\NC \type{italic}  \NC number         \NC italic correction \NC \NR
\NC \type{height}  \NC number         \NC required height \NC \NR
\NC \type{depth}   \NC number         \NC required depth \NC \NR
\NC \type{options} \NC number         \NC bitset of rendering options \NC \NR
\NC \type{class}   \NC number         \NC spacing related class \NC \NR
\LL
\stoptabulate

Warning: some of these fields are used by the renderer and might get adapted in
the process.

\subsection{whatsit nodes}

Whatsit nodes come in many subtypes that you can ask for by running
\type {node.whatsits()}:
\startluacode
    for id, name in table.sortedpairs(node.whatsits()) do
        context.type(name)
        context(" (%s), ",id)
    end
    context.removeunwantedspaces()
    context.removepunctuation()
\stopluacode
. % period

\subsubsection{front|-|end whatsits}

\subsubsubsection{\whs {open}}

\starttabulate[|l|l|p|]
\DB field         \BC type   \BC explanation \NC \NR
\TB
\NC \type{attr}   \NC node   \NC list of attributes \NC \NR
\NC \type{stream} \NC number \NC \TEX's stream id number \NC \NR
\NC \type{name}   \NC string \NC file name \NC \NR
\NC \type{ext}    \NC string \NC file extension \NC \NR
\NC \type{area}   \NC string \NC file area (this may become obsolete) \NC \NR
\LL
\stoptabulate

\subsubsubsection{\whs {write}}

\starttabulate[|l|l|p|]
\DB field         \BC type   \BC explanation \NC \NR
\TB
\NC \type{attr}   \NC node   \NC list of attributes \NC \NR
\NC \type{stream} \NC number \NC \TEX's stream id number \NC \NR
\NC \type{data}   \NC table  \NC a table representing the token list to be written \NC \NR
\LL
\stoptabulate

\subsubsubsection{\whs {close}}

\starttabulate[|l|l|p|]
\DB field         \BC type   \BC explanation \NC \NR
\TB
\NC \type{attr}   \NC node   \NC list of attributes \NC \NR
\NC \type{stream} \NC number \NC \TEX's stream id number \NC \NR
\LL
\stoptabulate

\subsubsubsection{\whs {user_defined}}

User|-|defined whatsit nodes can only be created and handled from \LUA\ code. In
effect, they are an extension to the extension mechanism. The \LUATEX\ engine
will simply step over such whatsits without ever looking at the contents.

\starttabulate[|l|l|p|]
\DB field          \BC type   \BC explanation \NC \NR
\TB
\NC \type{attr}    \NC node   \NC list of attributes \NC \NR
\NC \type{user_id} \NC number \NC id number \NC \NR
\NC \type{type}    \NC number \NC type of the value \NC \NR
\NC \type{value}   \NC number \NC a \LUA\ number \NC \NR
\NC                \NC node   \NC a node list \NC \NR
\NC                \NC string \NC a \LUA\ string \NC \NR
\NC                \NC table  \NC a \LUA\ table \NC \NR
\LL
\stoptabulate

The \type {type} can have one of six distinct values. The number is the \ASCII\
value if the first character if the type name (so you can use string.byte("l")
instead of \type {108}).

\starttabulate[|r|c|p|]
\DB value \BC meaning \BC explanation \NC \NR
\TB
\NC   97  \NC a       \NC list of attributes (a node list) \NC \NR
\NC  100  \NC d       \NC a \LUA\ number \NC \NR
\NC  108  \NC l       \NC a \LUA\ value (table, number, boolean, etc) \NC \NR
\NC  110  \NC n       \NC a node list \NC \NR
\NC  115  \NC s       \NC a \LUA\ string \NC \NR
\NC  116  \NC t       \NC a \LUA\ token list in \LUA\ table form (a list of triplets) \NC \NR
\LL
\stoptabulate

\subsubsubsection{\whs {save_pos}}

\starttabulate[|l|l|p|]
\DB field       \BC type \BC explanation \NC \NR
\TB
\NC \type{attr} \NC node \NC list of attributes \NC \NR
\LL
\stoptabulate

\subsubsubsection{\whs {late_lua}}

\starttabulate[|l|l|p|]
\DB field         \BC type     \BC explanation \NC \NR
\TB
\NC \type{attr}   \NC node     \NC list of attributes \NC \NR
\NC \type{data}   \NC string   \NC data to execute \NC \NR
\NC \type{string} \NC string   \NC data to execute \NC \NR
\NC \type{name}   \NC string   \NC the name to use for \LUA\ error reporting \NC \NR
\LL
\stoptabulate

The difference between \type {data} and \type {string} is that on assignment, the
\type {data} field is converted to a token list, cf.\ use as \lpr {latelua}. The
\type {string} version is treated as a literal string.

\subsubsection{\DVI\ backend whatsits}

\subsubsection{\whs {special}}

\starttabulate[|l|l|p|]
\DB field       \BC type   \BC explanation \NC \NR
\TB
\NC \type{attr} \NC node   \NC list of attributes \NC \NR
\NC \type{data} \NC string \NC the \prm {special} information \NC \NR
\LL
\stoptabulate

\subsubsection{\PDF\ backend whatsits}

\subsubsubsection{\whs {pdf_literal}}

\starttabulate[|l|l|p|]
\DB field       \BC type   \BC explanation \NC \NR
\TB
\NC \type{attr} \NC node   \NC list of attributes \NC \NR
\NC \type{mode} \NC number \NC the \quote {mode} setting of this literal \NC \NR
\NC \type{data} \NC string \NC the \orm {pdfliteral} information \NC \NR
\LL
\stoptabulate

Possible mode values are:

\starttabulate[|c|p|]
\DB value \BC keyword        \NC \NR
\TB
\NC 0     \NC \type{origin}  \NC \NR
\NC 1     \NC \type{page}    \NC \NR
\NC 2     \NC \type{direct}  \NC \NR
\NC 3     \NC \type{raw}     \NC \NR
\NC 4     \NC \type{text}    \NC \NR
\LL
\stoptabulate

The higher the number, the less checking and the more you can run into troubles.
Especially the \type {raw} variant can produce bad \PDF\ so you can best check
what you generate.

\subsubsubsection{\whs {pdf_refobj}}

\starttabulate[|l|l|p|]
\DB field         \BC type   \BC explanation \NC \NR
\TB
\NC \type{attr}   \NC node   \NC list of attributes \NC \NR
\NC \type{objnum} \NC number \NC the referenced \PDF\ object number \NC \NR
\LL
\stoptabulate

\subsubsubsection{\whs {pdf_annot}}

\starttabulate[|l|l|p|]
\DB field         \BC type   \BC explanation \NC \NR
\TB
\NC \type{attr}   \NC node   \NC list of attributes \NC \NR
\NC \type{width}  \NC number \NC the width (not used in calculations) \NC \NR
\NC \type{height} \NC number \NC the height (not used in calculations) \NC \NR
\NC \type{depth}  \NC number \NC the depth (not used in calculations) \NC \NR
\NC \type{objnum} \NC number \NC the referenced \PDF\ object number \NC \NR
\NC \type{data}   \NC string \NC the annotation data \NC \NR
\LL
\stoptabulate

\subsubsubsection{\whs {pdf_start_link}}

\starttabulate[|l|l|p|]
\DB field            \BC type   \BC explanation \NC \NR
\TB
\NC \type{attr}      \NC node   \NC list of attributes \NC \NR
\NC \type{width}     \NC number \NC the width (not used in calculations) \NC \NR
\NC \type{height}    \NC number \NC the height (not used in calculations) \NC \NR
\NC \type{depth}     \NC number \NC the depth (not used in calculations) \NC \NR
\NC \type{objnum}    \NC number \NC the referenced \PDF\ object number \NC \NR
\NC \type{link_attr} \NC table  \NC the link attribute token list \NC \NR
\NC \type{action}    \NC node   \NC the action to perform \NC \NR
\LL
\stoptabulate

\subsubsubsection{\whs {pdf_end_link}}

\starttabulate[|l|l|p|]
\DB field       \BC type \BC explanation \NC \NR
\TB
\NC \type{attr} \NC node \NC \NC \NR
\LL
\stoptabulate

\subsubsubsection{\whs {pdf_dest}}

\starttabulate[|l|l|p|]
\DB field              \BC type     \BC explanation \NC \NR
\TB
\NC \type{attr}        \NC node     \NC list of attributes \NC \NR
\NC \type{width}       \NC number   \NC the width (not used in calculations) \NC \NR
\NC \type{height}      \NC number   \NC the height (not used in calculations) \NC \NR
\NC \type{depth}       \NC number   \NC the depth (not used in calculations) \NC \NR
\NC \type{named_id}    \NC number   \NC is the \type {dest_id} a string value? \NC \NR
\NC \type{dest_id}     \NC number   \NC the destination id \NC \NR
\NC                    \NC string   \NC the destination name \NC \NR
\NC \type{dest_type}   \NC number   \NC type of destination \NC \NR
\NC \type{xyz_zoom}    \NC number   \NC the zoom factor (times 1000) \NC \NR
\NC \type{objnum}      \NC number   \NC the \PDF\ object number \NC \NR
\LL
\stoptabulate

\subsubsubsection{\whs {pdf_action}}

These are a special kind of item that only appears inside \PDF\ start link
objects.

\starttabulate[|l|l|p|]
\DB field              \BC type             \BC explanation \NC \NR
\TB
\NC \type{action_type} \NC number           \NC the kind of action involved \NC \NR
\NC \type{action_id}   \NC number or string \NC token list reference or string \NC \NR
\NC \type{named_id}    \NC number           \NC the index of the destination \NC \NR
\NC \type{file}        \NC string           \NC the target filename \NC \NR
\NC \type{new_window}  \NC number           \NC the window state of the target \NC \NR
\NC \type{data}        \NC string           \NC the name of the destination \NC \NR
\LL
\stoptabulate

Valid action types are:

\starttabulate[|l|l|]
\DB value \BC meaning       \NC \NR
\TB
\NC 0     \NC \type{page}   \NC \NR
\NC 1     \NC \type{goto}   \NC \NR
\NC 2     \NC \type{thread} \NC \NR
\NC 3     \NC \type{user}   \NC \NR
\LL
\stoptabulate

Valid window types are:

\starttabulate[|l|l|]
\DB value \BC meaning       \NC \NR
\TB
\NC 0     \NC \type{notset} \NC \NR
\NC 1     \NC \type{new}    \NC \NR
\NC 2     \NC \type{nonew}  \NC \NR
\LL
\stoptabulate

\subsubsubsection{\whs {pdf_thread}}

\starttabulate[|l|l|p|]
\DB field              \BC type   \BC explanation \NC \NR
\TB
\NC \type{attr}        \NC node   \NC list of attributes \NC \NR
\NC \type{width}       \NC number \NC the width (not used in calculations) \NC \NR
\NC \type{height}      \NC number \NC the height (not used in calculations) \NC \NR
\NC \type{depth}       \NC number \NC the depth (not used in calculations) \NC \NR
\NC \type{named_id}    \NC number \NC is \type {tread_id} a string value? \NC \NR
\NC \type{tread_id}    \NC number \NC the thread id \NC \NR
\NC                    \NC string \NC the thread name \NC \NR
\NC \type{thread_attr} \NC number \NC extra thread information \NC \NR
\LL
\stoptabulate

\subsubsubsection{\whs {pdf_start_thread}}

\starttabulate[|l|l|p|]
\DB field              \BC type   \BC explanation \NC \NR
\TB
\NC \type{attr}        \NC node   \NC list of attributes \NC \NR
\NC \type{width}       \NC number \NC the width (not used in calculations) \NC \NR
\NC \type{height}      \NC number \NC the height (not used in calculations) \NC \NR
\NC \type{depth}       \NC number \NC the depth (not used in calculations) \NC \NR
\NC \type{named_id}    \NC number \NC is \type {tread_id} a string value? \NC \NR
\NC \type{tread_id}    \NC number \NC the thread id \NC \NR
\NC                    \NC string \NC the thread name \NC \NR
\NC \type{thread_attr} \NC number \NC extra thread information \NC \NR
\LL
\stoptabulate

\subsubsubsection{\whs {pdf_end_thread}}

\starttabulate[|l|l|p|]
\DB field       \BC type \BC explanation \NC \NR
\TB
\NC \type{attr} \NC node \NC \NC \NR
\LL
\stoptabulate

\subsubsubsection{\whs {pdf_colorstack}}

\starttabulate[|l|l|p|]
\DB field          \BC type   \BC explanation \NC \NR
\TB
\NC \type{attr}    \NC node   \NC list of attributes \NC \NR
\NC \type{stack}   \NC number \NC colorstack id number \NC \NR
\NC \type{command} \NC number \NC command to execute \NC \NR
\NC \type{data}    \NC string \NC data \NC \NR
\LL
\stoptabulate

\subsubsubsection{\whs {pdf_setmatrix}}

\starttabulate[|l|l|p|]
\DB field       \BC type   \BC explanation \NC \NR
\TB
\NC \type{attr} \NC node   \NC list of attributes \NC \NR
\NC \type{data} \NC string \NC data \NC \NR
\LL
\stoptabulate

\subsubsubsection{\whs {pdf_save}}

\starttabulate[|l|l|p|]
\DB field       \BC type \BC explanation \NC \NR
\TB
\NC \type{attr} \NC node \NC list of attributes \NC \NR
\LL
\stoptabulate

\subsubsubsection{\whs {pdf_restore}}

\starttabulate[|l|l|p|]
\DB field       \BC type \BC explanation \NC \NR
\TB
\NC \type{attr} \NC node \NC list of attributes \NC \NR
\LL
\stoptabulate

\section{The \type {node} library}

The \type {node} library contains functions that facilitate dealing with (lists
of) nodes and their values. They allow you to create, alter, copy, delete, and
insert \LUATEX\ node objects, the core objects within the typesetter.

\LUATEX\ nodes are represented in \LUA\ as userdata with the metadata type
\type {luatex.node}. The various parts within a node can be accessed using
named fields.

Each node has at least the three fields \type {next}, \type {id}, and \type {subtype}:

\startitemize[intro]

\startitem
    The \type {next} field returns the userdata object for the next node in a
    linked list of nodes, or \type {nil}, if there is no next node.
\stopitem

\startitem
    The \type {id} indicates \TEX's \quote{node type}. The field \type {id} has a
    numeric value for efficiency reasons, but some of the library functions also
    accept a string value instead of \type {id}.
\stopitem

\startitem
    The \type {subtype} is another number. It often gives further information
    about a node of a particular \type {id}, but it is most important when
    dealing with \quote {whatsits}, because they are differentiated solely based
    on their \type {subtype}.
\stopitem

\stopitemize

The other available fields depend on the \type {id} (and for \quote {whatsits},
the \type {subtype}) of the node. Further details on the various fields and their
meanings are given in~\in{chapter}[nodes].

Support for \nod {unset} (alignment) nodes is partial: they can be queried and
modified from \LUA\ code, but not created.

Nodes can be compared to each other, but: you are actually comparing indices into
the node memory. This means that equality tests can only be trusted under very
limited conditions. It will not work correctly in any situation where one of the
two nodes has been freed and|/|or reallocated: in that case, there will be false
positives.

At the moment, memory management of nodes should still be done explicitly by the
user. Nodes are not \quote {seen} by the \LUA\ garbage collector, so you have to
call the node freeing functions yourself when you are no longer in need of a node
(list). Nodes form linked lists without reference counting, so you have to be
careful that when control returns back to \LUATEX\ itself, you have not deleted
nodes that are still referenced from a \type {next} pointer elsewhere, and that
you did not create nodes that are referenced more than once.

There are statistics available with regards to the allocated node memory, which
can be handy for tracing.

\subsection{Node handling functions}

\subsubsection{\type {node.is_node}}

\topicindex {nodes+functions}

\startfunctioncall
<boolean> t =
    node.is_node(<any> item)
\stopfunctioncall

This function returns true if the argument is a userdata object of
type \type {<node>}.

\subsubsection{\type {node.types}}

\startfunctioncall
<table> t =
    node.types()
\stopfunctioncall

This function returns an array that maps node id numbers to node type strings,
providing an overview of the possible top|-|level \type {id} types.

\subsubsection{\type {node.whatsits}}

\startfunctioncall
<table> t =
    node.whatsits()
\stopfunctioncall

\TEX's \quote {whatsits} all have the same \type {id}. The various subtypes are
defined by their \type {subtype} fields. The function is much like \type
{node.types}, except that it provides an array of \type {subtype} mappings.

\subsubsection{\type {node.id}}

\startfunctioncall
<number> id =
    node.id(<string> type)
\stopfunctioncall

This converts a single type name to its internal numeric representation.

\subsubsection{\type {node.subtype}}

\startfunctioncall
<number> subtype =
    node.subtype(<string> type)
\stopfunctioncall

This converts a single whatsit name to its internal numeric representation (\type
{subtype}).

\subsubsection{\type {node.type}}

\startfunctioncall
<string> type =
    node.type(<any> n)
\stopfunctioncall

In the argument is a number, then this function converts an internal numeric
representation to an external string representation. Otherwise, it will return
the string \type {node} if the object represents a node, and \type {nil}
otherwise.

\subsubsection{\type {node.fields}}

\startfunctioncall
<table> t =
    node.fields(<number> id)
<table> t =
    node.fields(<number> id, <number> subtype)
\stopfunctioncall

This function returns an array of valid field names for a particular type of
node. If you want to get the valid fields for a \quote {whatsit}, you have to
supply the second argument also. In other cases, any given second argument will
be silently ignored.

This function accepts string \type {id} and \type {subtype} values as well.

\subsubsection{\type {node.has_field}}

\startfunctioncall
<boolean> t =
    node.has_field(<node> n, <string> field)
\stopfunctioncall

This function returns a boolean that is only true if \type {n} is
actually a node, and it has the field.

\subsubsection{\type {node.new}}

\startfunctioncall
<node> n =
    node.new(<number> id)
<node> n =
    node.new(<number> id, <number> subtype)
\stopfunctioncall

The \type {new} function creates a new node. All its fields are initialized to
either zero or \type {nil} except for \type {id} and \type {subtype}. Instead of
numbers you can also use strings (names). If you create a new \nod {whatsit} node
the second argument is required. As with all node functions, this function
creates a node at the \TEX\ level.

\subsubsection{\type {node.free} and \type {node.flush_node}}

\startfunctioncall
<node> next =
    node.free(<node> n)
flush_node(<node> n)
\stopfunctioncall

Removes the node \type {n} from \TEX's memory. Be careful: no checks are done on
whether this node is still pointed to from a register or some \type {next} field:
it is up to you to make sure that the internal data structures remain correct.

The \type {free} function returns the next field of the freed node, while the
\type {flush_node} alternative returns nothing.

\subsubsection{\type {node.flush_list}}

\startfunctioncall
node.flush_list(<node> n)
\stopfunctioncall

Removes the node list \type {n} and the complete node list following \type {n}
from \TEX's memory. Be careful: no checks are done on whether any of these nodes
is still pointed to from a register or some \type {next} field: it is up to you
to make sure that the internal data structures remain correct.

\subsubsection{\type {node.copy}}

\startfunctioncall
<node> m =
    node.copy(<node> n)
\stopfunctioncall

Creates a deep copy of node \type {n}, including all nested lists as in the case
of a hlist or vlist node. Only the \type {next} field is not copied.

\subsubsection{\type {node.copy_list}}

\startfunctioncall
<node> m =
    node.copy_list(<node> n)
<node> m =
    node.copy_list(<node> n, <node> m)
\stopfunctioncall

Creates a deep copy of the node list that starts at \type {n}. If \type {m} is
also given, the copy stops just before node \type {m}.

Note that you cannot copy attribute lists this way, specialized functions for
dealing with attribute lists will be provided later but are not there yet.
However, there is normally no need to copy attribute lists as when you do
assignments to the \type {attr} field or make changes to specific attributes, the
needed copying and freeing takes place automatically.

\subsubsection{\type {node.next}}

\startfunctioncall
<node> m =
    node.next(<node> n)
\stopfunctioncall

Returns the node following this node, or \type {nil} if there is no such node.

\subsubsection{\type {node.prev}}

\startfunctioncall
<node> m =
    node.prev(<node> n)
\stopfunctioncall

Returns the node preceding this node, or \type {nil} if there is no such node.

\subsubsection{\type {node.current_attr}}

\startfunctioncall
<node> m =
    node.current_attr()
\stopfunctioncall

Returns the currently active list of attributes, if there is one.

The intended usage of \type {current_attr} is as follows:

\starttyping
local x1 = node.new("glyph")
x1.attr = node.current_attr()
local x2 = node.new("glyph")
x2.attr = node.current_attr()
\stoptyping

or:

\starttyping
local x1 = node.new("glyph")
local x2 = node.new("glyph")
local ca = node.current_attr()
x1.attr = ca
x2.attr = ca
\stoptyping

The attribute lists are ref counted and the assignment takes care of incrementing
the refcount. You cannot expect the value \type {ca} to be valid any more when
you assign attributes (using \type {tex.setattribute}) or when control has been
passed back to \TEX.

Note: this function is somewhat experimental, and it returns the {\it actual}
attribute list, not a copy thereof. Therefore, changing any of the attributes in
the list will change these values for all nodes that have the current attribute
list assigned to them.

\subsubsection{\type {node.hpack}}

\startfunctioncall
<node> h, <number> b =
    node.hpack(<node> n)
<node> h, <number> b =
    node.hpack(<node> n, <number> w, <string> info)
<node> h, <number> b =
    node.hpack(<node> n, <number> w, <string> info, <string> dir)
\stopfunctioncall

This function creates a new hlist by packaging the list that begins at node \type
{n} into a horizontal box. With only a single argument, this box is created using
the natural width of its components. In the three argument form, \type {info}
must be either \type {additional} or \type {exactly}, and \type {w} is the
additional (\type {\hbox spread}) or exact (\type {\hbox to}) width to be used.
The second return value is the badness of the generated box.

Caveat: at this moment, there can be unexpected side|-|effects to this function,
like updating some of the \prm {marks} and \type {\inserts}. Also note that the
content of \type {h} is the original node list \type {n}: if you call \type
{node.free(h)} you will also free the node list itself, unless you explicitly set
the \type {list} field to \type {nil} beforehand. And in a similar way, calling
\type {node.free(n)} will invalidate \type {h} as well!

\subsubsection{\type {node.vpack}}

\startfunctioncall
<node> h, <number> b =
    node.vpack(<node> n)
<node> h, <number> b =
    node.vpack(<node> n, <number> w, <string> info)
<node> h, <number> b =
    node.vpack(<node> n, <number> w, <string> info, <string> dir)
\stopfunctioncall

This function creates a new vlist by packaging the list that begins at node \type
{n} into a vertical box. With only a single argument, this box is created using
the natural height of its components. In the three argument form, \type {info}
must be either \type {additional} or \type {exactly}, and \type {w} is the
additional (\type {\vbox spread}) or exact (\type {\vbox to}) height to be used.

The second return value is the badness of the generated box.

See the description of \type {node.hpack()} for a few memory allocation caveats.

\subsubsection{\type {node.dimensions}, \type {node.rangedimensions}}

\startfunctioncall
<number> w, <number> h, <number> d  =
    node.dimensions(<node> n)
<number> w, <number> h, <number> d  =
    node.dimensions(<node> n, <string> dir)
<number> w, <number> h, <number> d  =
    node.dimensions(<node> n, <node> t)
<number> w, <number> h, <number> d  =
    node.dimensions(<node> n, <node> t, <string> dir)
\stopfunctioncall

This function calculates the natural in|-|line dimensions of the node list starting
at node \type {n} and terminating just before node \type {t} (or the end of the
list, if there is no second argument). The return values are scaled points. An
alternative format that starts with glue parameters as the first three arguments
is also possible:

\startfunctioncall
<number> w, <number> h, <number> d  =
    node.dimensions(<number> glue_set, <number> glue_sign, <number> glue_order,
        <node> n)
<number> w, <number> h, <number> d  =
    node.dimensions(<number> glue_set, <number> glue_sign, <number> glue_order,
        <node> n, <string> dir)
<number> w, <number> h, <number> d  =
    node.dimensions(<number> glue_set, <number> glue_sign, <number> glue_order,
        <node> n, <node> t)
<number> w, <number> h, <number> d  =
    node.dimensions(<number> glue_set, <number> glue_sign, <number> glue_order,
        <node> n, <node> t, <string> dir)
\stopfunctioncall

This calling method takes glue settings into account and is especially useful for
finding the actual width of a sublist of nodes that are already boxed, for
example in code like this, which prints the width of the space in between the
\type {a} and \type {b} as it would be if \type {\box0} was used as-is:

\starttyping
\setbox0 = \hbox to 20pt {a b}

\directlua{print (node.dimensions(
    tex.box[0].glue_set,
    tex.box[0].glue_sign,
    tex.box[0].glue_order,
    tex.box[0].head.next,
    node.tail(tex.box[0].head)
)) }
\stoptyping

You need to keep in mind that this is one of the few places in \TEX\ where floats
are used, which means that you can get small differences in rounding when you
compare the width reported by \type {hpack} with \type {dimensions}.

The second alternative saves a few lookups and can be more convenient in some
cases:

\startfunctioncall
<number> w, <number> h, <number> d  =
    node.rangedimensions(<node> parent, <node> first)
<number> w, <number> h, <number> d  =
    node.rangedimensions(<node> parent, <node> first, <node> last)
\stopfunctioncall

\subsubsection{\type {node.mlist_to_hlist}}

\startfunctioncall
<node> h =
    node.mlist_to_hlist(<node> n, <string> display_type, <boolean> penalties)
\stopfunctioncall

This runs the internal mlist to hlist conversion, converting the math list in
\type {n} into the horizontal list \type {h}. The interface is exactly the same
as for the callback \cbk {mlist_to_hlist}.

\subsubsection{\type {node.slide}}

\startfunctioncall
<node> m =
    node.slide(<node> n)
\stopfunctioncall

Returns the last node of the node list that starts at \type {n}. As a
side|-|effect, it also creates a reverse chain of \type {prev} pointers between
nodes.

\subsubsection{\type {node.tail}}

\startfunctioncall
<node> m =
    node.tail(<node> n)
\stopfunctioncall

Returns the last node of the node list that starts at \type {n}.

\subsubsection{\type {node.length}}

\startfunctioncall
<number> i =
    node.length(<node> n)
<number> i =
    node.length(<node> n, <node> m)
\stopfunctioncall

Returns the number of nodes contained in the node list that starts at \type {n}.
If \type {m} is also supplied it stops at \type {m} instead of at the end of the
list. The node \type {m} is not counted.

\subsubsection{\type {node.count}}

\startfunctioncall
<number> i =
    node.count(<number> id, <node> n)
<number> i =
    node.count(<number> id, <node> n, <node> m)
\stopfunctioncall

Returns the number of nodes contained in the node list that starts at \type {n}
that have a matching \type {id} field. If \type {m} is also supplied, counting
stops at \type {m} instead of at the end of the list. The node \type {m} is not
counted.

This function also accept string \type {id}'s.

\subsubsection{\type {node.traverse}}

\startfunctioncall
<node> t, id, subtype =
    node.traverse(<node> n)
\stopfunctioncall

This is a \LUA\ iterator that loops over the node list that starts at \type {n}.
Typically code looks like this:

\starttyping
for n in node.traverse(head) do
   ...
end
\stoptyping

is functionally equivalent to:

\starttyping
do
  local n
  local function f (head,var)
    local t
    if var == nil then
       t = head
    else
       t = var.next
    end
    return t
  end
  while true do
    n = f (head, n)
    if n == nil then break end
    ...
  end
end
\stoptyping

It should be clear from the definition of the function \type {f} that even though
it is possible to add or remove nodes from the node list while traversing, you
have to take great care to make sure all the \type {next} (and \type {prev})
pointers remain valid.

If the above is unclear to you, see the section \quote {For Statement} in the
\LUA\ Reference Manual.

\subsubsection{\type {node.traverse_id}}

\startfunctioncall
<node> t, subtype =
    node.traverse_id(<number> id, <node> n)
\stopfunctioncall

This is an iterator that loops over all the nodes in the list that starts at
\type {n} that have a matching \type {id} field.

See the previous section for details. The change is in the local function \type
{f}, which now does an extra while loop checking against the upvalue \type {id}:

\starttyping
 local function f(head,var)
   local t
   if var == nil then
      t = head
   else
      t = var.next
   end
   while not t.id == id do
      t = t.next
   end
   return t
 end
\stoptyping

\subsubsection{\type {node.traverse_char}}

This iterator loops over the \node {glyph} nodes in a list. Only nodes with a
subtype less than 256 are seen.

\startfunctioncall
<node> n, font, char =
    node.traverse_char(<node> n)
\stopfunctioncall

\subsubsection{\type {node.traverse_glyph}}

This iterator loops over a list and returns the list and filters all glyphs:

\startfunctioncall
<node> n, font, char =
    node.traverse_glyph(<node> n)
\stopfunctioncall

\subsubsection{\type {node.traverse_list}}

This iterator loops over the \nod {hlist} and \nod {vlist} nodes in a list.

\startfunctioncall
<node> n, id, subtype, list =
    node.traverse_list(<node> n)
\stopfunctioncall

The four return values can save some time compared to fetching these fields but
in practice you seldom need them all. So consider it a (side effect of
experimental) convenience.

\subsubsection{\type {node.has_glyph}}

This function returns the first glyph or disc node in the given list:

\startfunctioncall
<node> n =
    node.has_glyph(<node> n)
\stopfunctioncall

\subsubsection{\type {node.end_of_math}}

\startfunctioncall
<node> t =
    node.end_of_math(<node> start)
\stopfunctioncall

Looks for and returns the next \type {math_node} following the \type {start}. If
the given node is a math endnode this helper return that node, else it follows
the list and return the next math endnote. If no such node is found nil is
returned.

\subsubsection{\type {node.remove}}

\startfunctioncall
<node> head, current =
    node.remove(<node> head, <node> current)
\stopfunctioncall

This function removes the node \type {current} from the list following \type
{head}. It is your responsibility to make sure it is really part of that list.
The return values are the new \type {head} and \type {current} nodes. The
returned \type {current} is the node following the \type {current} in the calling
argument, and is only passed back as a convenience (or \type {nil}, if there is
no such node). The returned \type {head} is more important, because if the
function is called with \type {current} equal to \type {head}, it will be
changed.

\subsubsection{\type {node.insert_before}}

\startfunctioncall
<node> head, new =
    node.insert_before(<node> head, <node> current, <node> new)
\stopfunctioncall

This function inserts the node \type {new} before \type {current} into the list
following \type {head}. It is your responsibility to make sure that \type
{current} is really part of that list. The return values are the (potentially
mutated) \type {head} and the node \type {new}, set up to be part of the list
(with correct \type {next} field). If \type {head} is initially \type {nil}, it
will become \type {new}.

\subsubsection{\type {node.insert_after}}

\startfunctioncall
<node> head, new =
    node.insert_after(<node> head, <node> current, <node> new)
\stopfunctioncall

This function inserts the node \type {new} after \type {current} into the list
following \type {head}. It is your responsibility to make sure that \type
{current} is really part of that list. The return values are the \type {head} and
the node \type {new}, set up to be part of the list (with correct \type {next}
field). If \type {head} is initially \type {nil}, it will become \type {new}.

\subsubsection{\type {node.first_glyph}}

\startfunctioncall
<node> n =
    node.first_glyph(<node> n)
<node> n =
    node.first_glyph(<node> n, <node> m)
\stopfunctioncall

Returns the first node in the list starting at \type {n} that is a glyph node
with a subtype indicating it is a glyph, or \type {nil}. If \type {m} is given,
processing stops at (but including) that node, otherwise processing stops at the
end of the list.

\subsubsection{\type {node.ligaturing}}

\startfunctioncall
<node> h, <node> t, <boolean> success =
    node.ligaturing(<node> n)
<node> h, <node> t, <boolean> success =
    node.ligaturing(<node> n, <node> m)
\stopfunctioncall

Apply \TEX-style ligaturing to the specified nodelist. The tail node \type {m} is
optional. The two returned nodes \type {h} and \type {t} are the new head and
tail (both \type {n} and \type {m} can change into a new ligature).

\subsubsection{\type {node.kerning}}

\startfunctioncall
<node> h, <node> t, <boolean> success =
    node.kerning(<node> n)
<node> h, <node> t, <boolean> success =
    node.kerning(<node> n, <node> m)
\stopfunctioncall

Apply \TEX|-|style kerning to the specified node list. The tail node \type {m} is
optional. The two returned nodes \type {h} and \type {t} are the head and tail
(either one of these can be an inserted kern node, because special kernings with
word boundaries are possible).

\subsubsection{\type {node.unprotect_glyphs} and \type {node.unprotect_glyph}}

\startfunctioncall
node.unprotect_glyph(<node> n)
node.unprotect_glyphs(<node> n,[<node> n])
\stopfunctioncall

Subtracts 256 from all glyph node subtypes. This and the next function are
helpers to convert from \type {characters} to \type {glyphs} during node
processing. The second argument is option and indicates the end of a range.

\subsubsection{\type {node.protect_glyphs} and \type {node.protect_glyph}}

\startfunctioncall
node.protect_glyph(<node> n)
node.protect_glyphs(<node> n,[<node> n])
\stopfunctioncall

Adds 256 to all glyph node subtypes in the node list starting at \type {n},
except that if the value is 1, it adds only 255. The special handling of 1 means
that \type {characters} will become \type {glyphs} after subtraction of 256. A
single character can be marked by the singular call. The second argument is
option and indicates the end of a range.

\subsubsection{\type {node.last_node}}

\startfunctioncall
<node> n =
    node.last_node()
\stopfunctioncall

This function pops the last node from \TEX's \quote{current list}. It returns
that node, or \type {nil} if the current list is empty.

\subsubsection{\type {node.write}}

\startfunctioncall
node.write(<node> n)
\stopfunctioncall

This is an experimental function that will append a node list to \TEX's \quote
{current list} The node list is not deep|-|copied! There is no error checking
either!

\subsubsection{\type {node.protrusion_skippable}}

\startfunctioncall
<boolean> skippable =
    node.protrusion_skippable(<node> n)
\stopfunctioncall

Returns \type {true} if, for the purpose of line boundary discovery when
character protrusion is active, this node can be skipped.

\subsection{Glue handling}

\subsubsection{\type {node.setglue}}

You can set the properties of a glue in one go. If you pass no values, the glue
will become a zero glue.

\startfunctioncall
node.setglue(<node> n)
node.setglue(<node> n,width,stretch,shrink,stretch_order,shrink_order)
\stopfunctioncall

When you pass values, only arguments that are numbers are assigned so

\starttyping
node.setglue(n,655360,false,65536)
\stoptyping

will only adapt the width and shrink.

When a list node is passed, you set the glue, order and sign instead.

\subsubsection{\type {node.getglue}}

The next call will return 5 values (or northing when no glue is passed).

\startfunctioncall
<integer> width, <integer> stretch, <integer> shrink, <integer> stretch_order,
    <integer> shrink_order = node.getglue(<node> n)
\stopfunctioncall

When the second argument is false, only the width is returned (this is consistent
with \type {tex.get}).

When a list node is passed, you get back the glue that is set, the order of that
glue and the sign.

\subsubsection{\type {node.is_zero_glue}}

This function returns \type {true} when the width, stretch and shrink properties
are zero.

\startfunctioncall
<boolean> isglue =
    node.is_zero_glue(<node> n)
\stopfunctioncall

\subsection{Attribute handling}

Attributes appear as linked list of userdata objects in the \type {attr} field of
individual nodes. They can be handled individually, but it is much safer and more
efficient to use the dedicated functions associated with them.

\subsubsection{\type {node.has_attribute}}

\startfunctioncall
<number> v =
    node.has_attribute(<node> n, <number> id)
<number> v =
    node.has_attribute(<node> n, <number> id, <number> val)
\stopfunctioncall

Tests if a node has the attribute with number \type {id} set. If \type {val} is
also supplied, also tests if the value matches \type {val}. It returns the value,
or, if no match is found, \type {nil}.

\subsubsection{\type {node.get_attribute}}

\startfunctioncall
<number> v =
    node.get_attribute(<node> n, <number> id)
\stopfunctioncall

Tests if a node has an attribute with number \type {id} set. It returns the
value, or, if no match is found, \type {nil}.

\subsubsection{\type {node.find_attribute}}

\startfunctioncall
<number> v, <node> n =
    node.find_attribute(<node> n, <number> id)
\stopfunctioncall

Finds the first node that has attribute with number \type {id} set. It returns
the value and the node if there is a match and otherwise nothing.

\subsubsection{\type {node.set_attribute}}

\startfunctioncall
node.set_attribute(<node> n, <number> id, <number> val)
\stopfunctioncall

Sets the attribute with number \type {id} to the value \type {val}. Duplicate
assignments are ignored.

\subsubsection{\type {node.unset_attribute}}

\startfunctioncall
<number> v =
    node.unset_attribute(<node> n, <number> id)
<number> v =
    node.unset_attribute(<node> n, <number> id, <number> val)
\stopfunctioncall

Unsets the attribute with number \type {id}. If \type {val} is also supplied, it
will only perform this operation if the value matches \type {val}. Missing
attributes or attribute|-|value pairs are ignored.

If the attribute was actually deleted, returns its old value. Otherwise, returns
\type {nil}.

\subsubsection{\type {node.slide}}

This helper makes sure that the node lists is double linked and returns the found
tail node.

\startfunctioncall
<node> tail =
    node.slide(<node> n)
\stopfunctioncall

After some callbacks automatic sliding takes place. This feature can be turned
off with \type {node.fix_node_lists(false)} but you better make sure then that
you don't mess up lists. In most cases \TEX\ itself only uses \type {next}
pointers but your other callbacks might expect proper \type {prev} pointers too.
Future versions of \LUATEX\ can add more checking but this will not influence
usage.

\subsubsection{\type {node.check_discretionary} and \type {node.check_discretionaries}}

When you fool around with disc nodes you need to be aware of the fact that they
have a special internal data structure. As long as you reassign the fields when
you have extended the lists it's ok because then the tail pointers get updated,
but when you add to list without reassigning you might end up in troubles when
the linebreak routien kicks in. You can call this function to check the list for
issues with disc nodes.

\startfunctioncall
node.check_discretionary(<node> n)
node.check_discretionaries(<node> head)
\stopfunctioncall

The plural variant runs over all disc nodes in a list, the singular variant
checks one node only (it also checks if the node is a disc node).

\subsubsection{\type {node.flatten_discretionaries}}

This function will remove the discretionaries in the list and inject the replace
field when set.

\startfunctioncall
<node> head, count = node.flatten_discretionaries(<node> n)
\stopfunctioncall

\subsubsection{\type {node.family_font}}

When you pass it a proper family identifier the next helper will return the font
currently associated with it. You can normally also access the font with the normal
font field or getter because it will resolve the family automatically for noads.

\startfunctioncall
<integer> id =
    node.family_font(<integer> fam)
\stopfunctioncall

\subsubsection{\type {node.set_synctex_fields} and \type {node.get_synctex_fields}}

You can set and query the synctex fields, a file number aka tag and a line
number, for a glue, kern, hlist, vlist, rule and math nodes as well as glyph
nodes (although this last one are not used in native synctex).

\startfunctioncall
node.set_synctex_fields(<integer> f, <integer> l)
<integer> f, <integer> l =
    node.get_synctex_fields(<node> n)
\stopfunctioncall

Of course you need to know what you're doing as no checking on sane values takes
place. Also, the synctex interpreter used in editors is rather peculiar and has
some assumptions (heuristics).

\section{Two access models}

\topicindex{nodes+direct}
\topicindex{direct nodes}

Deep down in \TEX\ a node has a number which is an numeric entry in a memory
table. In fact, this model, where \TEX\ manages memory is real fast and one of
the reasons why plugging in callbacks that operate on nodes is quite fast too.
Each node gets a number that is in fact an index in the memory table and that
number often gets reported when you print node related information.

There are two access models, a robust one using a so called user data object that
provides a virtual interface to the internal nodes, and a more direct access which
uses the node numbers directly. The first model provide key based access while
the second always accesses fields via functions:

\starttyping
nodeobject.char
getfield(nodenumber,"char")
\stoptyping

If you use the direct model, even if you know that you deal with numbers, you
should not depend on that property but treat it an abstraction just like
traditional nodes. In fact, the fact that we use a simple basic datatype has the
penalty that less checking can be done, but less checking is also the reason why
it's somewhat faster. An important aspect is that one cannot mix both methods,
but you can cast both models. So, multiplying a node number makes no sense.

So our advice is: use the indexed (table) approach when possible and investigate
the direct one when speed might be an real issue. For that reason we also provide
the \type {get*} and \type {set*} functions in the top level node namespace.
There is a limited set of getters. When implementing this direct approach the
regular index by key variant was also optimized, so direct access only makes
sense when we're accessing nodes millions of times (which happens in some font
processing for instance).

We're talking mostly of getters because setters are less important. Documents
have not that many content related nodes and setting many thousands of properties
is hardly a burden contrary to millions of consultations.

Normally you will access nodes like this:

\starttyping
local next = current.next
if next then
    -- do something
end
\stoptyping

Here \type {next} is not a real field, but a virtual one. Accessing it results in
a metatable method being called. In practice it boils down to looking up the node
type and based on the node type checking for the field name. In a worst case you
have a node type that sits at the end of the lookup list and a field that is last
in the lookup chain. However, in successive versions of \LUATEX\ these lookups
have been optimized and the most frequently accessed nodes and fields have a
higher priority.

Because in practice the \type {next} accessor results in a function call, there
is some overhead involved. The next code does the same and performs a tiny bit
faster (but not that much because it is still a function call but one that knows
what to look up).

\starttyping
local next = node.next(current)
if next then
    -- do something
end
\stoptyping

Some accessors are used frequently and for these we provide more efficient helpers:

\starttabulate[|l|p|]
\DB function          \BC explanation \NC \NR
\TB
\NC \type{getnext}    \NC parsing nodelist always involves this one \NC \NR
\NC \type{getprev}    \NC used less but is logical companion to \type {getnext} \NC \NR
\NC \type{getboth}    \NC returns the next and prev pointer of a node \NC \NR
\NC \type{getid}      \NC consulted a lot \NC \NR
\NC \type{getsubtype} \NC consulted less but also a topper \NC \NR
\NC \type{getfont}    \NC used a lot in \OPENTYPE\ handling (glyph nodes are consulted a lot) \NC \NR
\NC \type{getchar}    \NC idem and also in other places \NC \NR
\NC \type{getwhd}     \NC returns the \type {width}, \type {height} and \type {depth} of a list, rule or
                          (unexpanded) glyph as well as glue (its spec is looked at) and unset nodes\NC \NR
\NC \type{getdisc}    \NC returns the \type {pre}, \type {post} and \type {replace} fields and
                          optionally when true is passed also the tail fields. \NC \NR
\NC \type{getlist}    \NC we often parse nested lists so this is a convenient one too \NC \NR
\NC \type{getleader}  \NC comparable to list, seldom used in \TEX\ (but needs frequent consulting
                          like lists; leaders could have been made a dedicated node type) \NC \NR
\NC \type{getfield}   \NC generic getter, sufficient for the rest (other field names are
                          often shared so a specific getter makes no sense then) \NC \NR
\NC \type{getbox}     \NC gets the given box (a list node) \NC \NR
\LL
\stoptabulate

In the direct namespace there are more such helpers and most of them are
accompanied by setters. The getters and setters are clever enough to see what
node is meant. We don't deal with whatsit nodes: their fields are always accessed
by name. It doesn't make sense to add getters for all fields, we just identifier
the most likely candidates. In complex documents, many node and fields types
never get seen, or seen only a few times, but for instance glyphs are candidates
for such optimization. The \type {node.direct} interface has some more helpers.
\footnote {We can define the helpers in the node namespace with \type {getfield}
which is about as efficient, so at some point we might provide that as module.}

The \type {setdisc} helper takes three (optional) arguments plus an optional
fourth indicating the subtype. Its \type {getdisc} takes an optional boolean;
when its value is \type {true} the tail nodes will also be returned. The \type
{setfont} helper takes an optional second argument, it being the character. The
directmode setter \type {setlink} takes a list of nodes and will link them,
thereby ignoring \type {nil} entries. The first valid node is returned (beware:
for good reason it assumes single nodes). For rarely used fields no helpers are
provided and there are a few that probably are used seldom too but were added for
consistency. You can of course always define additional accessor using \type
{getfield} and \type {setfield} with little overhead.

\def\yes{$+$} \def\nop{$-$}

\starttabulate[|l|c|c|]
\DB function                     \BC node \BC direct \NC \NR
\TB
%NC \type {do_ligature_n}        \NC \yes \NC \yes  \NC \NR % was never documented and experimental
\NC \type {check_discretionaries}\NC \yes \NC \yes  \NC \NR
\NC \type {copy_list}            \NC \yes \NC \yes  \NC \NR
\NC \type {copy}                 \NC \yes \NC \yes  \NC \NR
\NC \type {count}                \NC \yes \NC \yes  \NC \NR
\NC \type {current_attr}         \NC \yes \NC \yes  \NC \NR
\NC \type {dimensions}           \NC \yes \NC \yes  \NC \NR
\NC \type {effective_glue}       \NC \yes \NC \yes  \NC \NR
\NC \type {end_of_math}          \NC \yes \NC \yes  \NC \NR
\NC \type {family_font}          \NC \yes \NC \nop  \NC \NR
\NC \type {fields}               \NC \yes \NC \nop  \NC \NR
\NC \type {find_attribute}       \NC \yes \NC \yes  \NC \NR
\NC \type {first_glyph}          \NC \yes \NC \yes  \NC \NR
\NC \type {flush_list}           \NC \yes \NC \yes  \NC \NR
\NC \type {flush_node}           \NC \yes \NC \yes  \NC \NR
\NC \type {free}                 \NC \yes \NC \yes  \NC \NR
\NC \type {get_attribute}        \NC \yes \NC \yes  \NC \NR
\NC \type {getattributelist}     \NC \nop \NC \yes  \NC \NR
\NC \type {getboth}              \NC \yes \NC \yes  \NC \NR
\NC \type {getbox}               \NC \nop \NC \yes  \NC \NR
\NC \type {getchar}              \NC \yes \NC \yes  \NC \NR
\NC \type {getcomponents}        \NC \nop \NC \yes  \NC \NR
\NC \type {getdepth}             \NC \nop \NC \yes  \NC \NR
\NC \type {getdir}               \NC \nop \NC \yes  \NC \NR
\NC \type {getdisc}              \NC \yes \NC \yes  \NC \NR
\NC \type {getfam}               \NC \nop \NC \yes  \NC \NR
\NC \type {getfield}             \NC \yes \NC \yes  \NC \NR
\NC \type {getfont}              \NC \yes \NC \yes  \NC \NR
\NC \type {getglue}              \NC \yes \NC \yes  \NC \NR
\NC \type {getheight}            \NC \nop \NC \yes  \NC \NR
\NC \type {getid}                \NC \yes \NC \yes  \NC \NR
\NC \type {getkern}              \NC \nop \NC \yes  \NC \NR
\NC \type {getlang}              \NC \nop \NC \yes  \NC \NR
\NC \type {getleader}            \NC \yes \NC \yes  \NC \NR
\NC \type {getlist}              \NC \yes \NC \yes  \NC \NR
\NC \type {getnext}              \NC \yes \NC \yes  \NC \NR
\NC \type {getnucleus}           \NC \nop \NC \yes  \NC \NR
\NC \type {getoffsets}           \NC \nop \NC \yes  \NC \NR
\NC \type {getpenalty}           \NC \nop \NC \yes  \NC \NR
\NC \type {getprev}              \NC \yes \NC \yes  \NC \NR
\NC \type {getproperty}          \NC \yes \NC \yes  \NC \NR
\NC \type {getshift}             \NC \nop \NC \yes  \NC \NR
\NC \type {getwidth}             \NC \nop \NC \yes  \NC \NR
\NC \type {getwhd}               \NC \nop \NC \yes  \NC \NR
\NC \type {getsub}               \NC \nop \NC \yes  \NC \NR
\NC \type {getsubtype}           \NC \yes \NC \yes  \NC \NR
\NC \type {getsup}               \NC \nop \NC \yes  \NC \NR
\NC \type {has_attribute}        \NC \yes \NC \yes  \NC \NR
\NC \type {has_field}            \NC \yes \NC \yes  \NC \NR
\NC \type {has_glyph}            \NC \yes \NC \yes  \NC \NR
\NC \type {hpack}                \NC \yes \NC \yes  \NC \NR
\NC \type {id}                   \NC \yes \NC \nop  \NC \NR
\NC \type {insert_after}         \NC \yes \NC \yes  \NC \NR
\NC \type {insert_before}        \NC \yes \NC \yes  \NC \NR
\NC \type {is_char}              \NC \yes \NC \yes  \NC \NR
\NC \type {is_direct}            \NC \nop \NC \yes  \NC \NR
\NC \type {is_glue_zero}         \NC \yes \NC \yes  \NC \NR
\NC \type {is_glyph}             \NC \yes \NC \yes  \NC \NR
\NC \type {is_node}              \NC \yes \NC \yes  \NC \NR
\NC \type {kerning}              \NC \yes \NC \yes  \NC \NR
\NC \type {last_node}            \NC \yes \NC \yes  \NC \NR
\NC \type {length}               \NC \yes \NC \yes  \NC \NR
\NC \type {ligaturing}           \NC \yes \NC \yes  \NC \NR
\NC \type {mlist_to_hlist}       \NC \yes \NC \nop  \NC \NR
\NC \type {new}                  \NC \yes \NC \yes  \NC \NR
\NC \type {next}                 \NC \yes \NC \nop  \NC \NR
\NC \type {prev}                 \NC \yes \NC \nop  \NC \NR
\NC \type {protect_glyphs}       \NC \yes \NC \yes  \NC \NR
\NC \type {protect_glyph}        \NC \yes \NC \yes  \NC \NR
\NC \type {protrusion_skippable} \NC \yes \NC \yes  \NC \NR
\NC \type {rangedimensions}      \NC \yes \NC \yes  \NC \NR
\NC \type {remove}               \NC \yes \NC \yes  \NC \NR
\NC \type {set_attribute}        \NC \nop \NC \yes  \NC \NR
\NC \type {setattributelist}     \NC \nop \NC \yes  \NC \NR
\NC \type {setboth}              \NC \nop \NC \yes  \NC \NR
\NC \type {setbox}               \NC \nop \NC \yes  \NC \NR
\NC \type {setchar}              \NC \nop \NC \yes  \NC \NR
\NC \type {setcomponents}        \NC \nop \NC \yes  \NC \NR
\NC \type {setdepth}             \NC \nop \NC \yes  \NC \NR
\NC \type {setdir}               \NC \nop \NC \yes  \NC \NR
\NC \type {setdisc}              \NC \nop \NC \yes  \NC \NR
\NC \type {setfield}             \NC \yes \NC \yes  \NC \NR
\NC \type {setfont}              \NC \nop \NC \yes  \NC \NR
\NC \type {setglue}              \NC \yes \NC \yes  \NC \NR
\NC \type {setheight}            \NC \nop \NC \yes  \NC \NR
\NC \type {setid}                \NC \nop \NC \yes  \NC \NR
\NC \type {setkern}              \NC \nop \NC \yes  \NC \NR
\NC \type {setlang}              \NC \nop \NC \yes  \NC \NR
\NC \type {setleader}            \NC \nop \NC \yes  \NC \NR
\NC \type {setlist}              \NC \nop \NC \yes  \NC \NR
\NC \type {setnext}              \NC \nop \NC \yes  \NC \NR
\NC \type {setnucleus}           \NC \nop \NC \yes  \NC \NR
\NC \type {setoffsets}           \NC \nop \NC \yes  \NC \NR
\NC \type {setpenalty}           \NC \nop \NC \yes  \NC \NR
\NC \type {setprev}              \NC \nop \NC \yes  \NC \NR
\NC \type {setproperty}          \NC \nop \NC \yes  \NC \NR
\NC \type {setshift}             \NC \nop \NC \yes  \NC \NR
\NC \type {setwidth}             \NC \nop \NC \yes  \NC \NR
\NC \type {setwhd}               \NC \nop \NC \yes  \NC \NR
\NC \type {setsub}               \NC \nop \NC \yes  \NC \NR
\NC \type {setsubtype}           \NC \nop \NC \yes  \NC \NR
\NC \type {setsup}               \NC \nop \NC \yes  \NC \NR
\NC \type {slide}                \NC \yes \NC \yes  \NC \NR
\NC \type {subtypes}             \NC \yes \NC \nop  \NC \NR
\NC \type {subtype}              \NC \yes \NC \nop  \NC \NR
\NC \type {tail}                 \NC \yes \NC \yes  \NC \NR
\NC \type {todirect}             \NC \yes \NC \yes  \NC \NR
\NC \type {tonode}               \NC \yes \NC \yes  \NC \NR
\NC \type {tostring}             \NC \yes \NC \yes  \NC \NR
\NC \type {traverse_char}        \NC \yes \NC \yes  \NC \NR
\NC \type {traverse_id}          \NC \yes \NC \yes  \NC \NR
\NC \type {traverse}             \NC \yes \NC \yes  \NC \NR
\NC \type {types}                \NC \yes \NC \nop  \NC \NR
\NC \type {type}                 \NC \yes \NC \nop  \NC \NR
\NC \type {unprotect_glyphs}     \NC \yes \NC \yes  \NC \NR
\NC \type {unset_attribute}      \NC \yes \NC \yes  \NC \NR
\NC \type {usedlist}             \NC \yes \NC \yes  \NC \NR
\NC \type {uses_font}            \NC \yes \NC \yes  \NC \NR
\NC \type {vpack}                \NC \yes \NC \yes  \NC \NR
\NC \type {whatsitsubtypes}      \NC \yes \NC \nop  \NC \NR
\NC \type {whatsits}             \NC \yes \NC \nop  \NC \NR
\NC \type {write}                \NC \yes \NC \yes  \NC \NR
\NC \type {set_synctex_fields}   \NC \yes \NC \yes  \NC \NR
\NC \type {get_synctex_fields}   \NC \yes \NC \yes  \NC \NR
\LL
\stoptabulate

The \type {node.next} and \type {node.prev} functions will stay but for
consistency there are variants called \type {getnext} and \type {getprev}. We had
to use \type {get} because \type {node.id} and \type {node.subtype} are already
taken for providing meta information about nodes. Note: The getters do only basic
checking for valid keys. You should just stick to the keys mentioned in the
sections that describe node properties.

Some nodes have indirect references. For instance a math character refers to a
family instead of a font. In that case we provide a virtual font field as
accessor. So, \type {getfont} and \type {.font} can be used on them. The same is
true for the \type {width}, \type {height} and \type {depth} of glue nodes. These
actually access the spec node properties, and here we can set as well as get the
values.

\stopchapter

\stopcomponent
