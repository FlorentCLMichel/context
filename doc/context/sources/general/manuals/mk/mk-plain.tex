% language=uk

\environment mk-environment

\startcomponent mk-plain

\chapter {Just plain}

\subject{running}

For testing basic \LUATEX\ functionality it makes sense to have a
minimal system, and traditionally plain \TEX\ has been the most
natural candidate. It is for this reason that it had been on the
agenda for a while to provide basic \OPENTYPE\ font support for
plain \TEX\ as well. Although the \MKIV\ node mode subsystem is
not yet perfect, the time was right to start experimenting with a
subset of the \MKIV\ code.

Using plain roughly comes down to the following. First you need
to generate a format:

\starttyping
luatex --ini --fmt=luatex.fmt luatex-plain.tex
\stoptyping

This format has to be moved to a place where it can be found by
the \KPSE\ library. Since this can differ per distribution there
is no clear recipe for it, but for \TEXLIVE\ some path ending in
\type {web2c/luatex} is probably the right spot. After that you
can run

\starttyping
luatex luatex-test.tex
\stoptyping

This file lives under \type {generic/context}. When it
is run it is quite likely that you will get an error message
because the font name database cannot be found. You
can generate one with the following command (which assumes that
you have \CONTEXT\ installed):

\starttyping
mtxrun --usekpse --script fonts --names
\stoptyping

The resulting file \type {luatex-fonts-names.lua} has to be placed
somewhere in your \TEX\ tree so that it can be found anytime.
Beware: the \type {--usekpse} flag is only used outside \CONTEXT\
and provides very limited functionality, just enough for this
task. Again this is a distribution specific issue so we will not
dwell upon it here.

The way fonts are defined is modelled after \XETEX, as it makes no
sense to support the somewhat more fancy \CONTEXT\ way of doing
things. Keep in mind that although \CONTEXT\ \MKIV\ does support
the \XETEX\ syntax too, the preferred way there is to use a more
symbolic feature definition approach.

As this is an experimental setup, it might not always work out as
expected. Around \LUATEX\ version 0.50 we expect the code to be
more or less okay.

\subject{implementation}

The \type {luatex-fonts.lua} file is the first in a series of
basic functionality enhancements for \LUATEX\ derived from the
\CONTEXT\ \MKIV\ code base. Please don't pollute the \type
{luatex-*} namespace with code not coming from the \CONTEXT\
development team as we may add more files.

This file implements a basic font system for a bare \LUATEX\
system. By default \LUATEX\ only knows about the classic \TFM\
fonts but it can read other font formats and pass them to \LUA.
With some glue code one can then construct a suitable \TFM\
representation that \LUATEX\ can work with. For more advanced font
support a bit more code is needed that needs to be hooked
into the callback mechanism.

This file is currently rather simple: it just loads the \LUA\ file
with the same name. An example of a \type {luatex.tex} file that is
just the plain \TEX\ format:

\starttyping
\catcode`\{=1 % left brace is begin-group character
\catcode`\}=2 % right brace is end-group character

\input plain

\everyjob\expandafter{\the\everyjob\input luatex-fonts\relax}

\dump
\stoptyping

We could load the \LUA\ file in \type {\everyjob} but maybe some
day we will need more here.

When defining a font, in addition to the \XETEX\ way, you can use
two prefixes. A \type {file:}\ prefix forces a file search, while
a \type {name:}\ prefix will result in consulting the names
database. The font definitions shown in figure~1 are all valid.

\starttyping
\font\testa=file:lmroman10-regular at 12pt
\font\testb=file:lmroman12-regular:+liga; at 24pt
\font\testc=file:lmroman12-regular:mode=node;+liga; at 24pt
\font\testd=name:lmroman10bold at 12pt
\font\testh=cmr10
\font\testi=ptmr8t
\font\teste=[lmroman12-regular]:+liga at 30pt
\font\testf=[lmroman12-regular] at 40pt
\font\testj=adobesongstd-light % cid font
\font\testk=cambria(math) {\mathtest 123}
\font\testl=file:IranNastaliq.ttf:mode=node;script=arab;\
    language=dflt;+calt;+ccmp;+init;+isol;+medi;+fina;+liga;\
    +rlig;+kern;+mark;+mkmk at 14pt
\stoptyping

You can load maths fonts but as Plain \TEX\ is set up for Computer
Modern (and as we don't adapt Plain \TEX) loading Cambria does
not give you support for its math features automatically.

If you want access by name you need to generate a font
database, using:

\starttyping
mtxrun --script font --names
\stoptyping

and put the resulting file in a spot where \LUATEX\ can find it.

\subject {remarks}

The code loaded in \type {luatex-fonts.lua} does not come out of
thin air, but is mostly shared with \CONTEXT; however, in that
macro package we go beyond what is provided in the plain variant.
When using this code you need to keep a few things in mind:

\startitemize

\item This subsystem will be extended, improved etc.\ at about the
same pace as \CONTEXT\ \MKIV. However, because \CONTEXT\ provides a
rather high level of integration not all features will be supported
in the same quality. Use \CONTEXT\ if you want more goodies.

\item There is no official \API\ yet, which means that using
functions implemented here is at your own risk, in the sense that
names and namespaces might change. There will be a minimal \API\
defined once \LUATEX\ version 1.0 is out. Instead of patching the
files it's better to overload functions if needed.

\item The modules are not stripped too much, which makes it
possible to benefit from improvements in the code that take place
in the perspective of \CONTEXT\ development. They might be split a
bit more in due time so the baseline might become smaller.

\item The code is maintained and tested by the \CONTEXT\
development team. As such it might be better suited for this macro
package and integration in other systems might demand some
additional wrapping. The plain version discussed here is the
benchmark and should be treated as a kind of black box.

\item Problems can be reported to the team but as we use \CONTEXT\
\MKIV\ as our baseline, you'd better check if the problem is a general
\CONTEXT\ problem too.

\item The more high level support for features that is provided in
\CONTEXT\ is not part of the code loaded here as it makes no sense
elsewhere. Some experimental features are not part of this code
either but some might show up later.

\item Math font support will be added but only in its basic form
once the Latin Modern and \TEX\ Gyre math fonts are
available. Currently traditional and \OPENTYPE\ math fonts can
be loaded.

\item At this moment the more nifty speedups are not enabled
because they work in tandem with the alternative file handling
that \CONTEXT\ uses. Maybe around \LUATEX\ 1.0 we will bring some
speedup into this code too (if it pays off at all).

\item The code defines a few global tables. If this code is used
in a larger perspective then you can best make sure that no
conflicts occur. The \CONTEXT\ package expects users to work in
their own namespace (\type {userdata}, \type {thirddata}, \type
{moduledata} or \type {document}). We give ourselves the freedom to
use any table at the global level but will not use tables that are
named after macro packages. Later, \CONTEXT\ might operate in
a more controlled namespace but it has a low priority.

\item There is some tracing code present but this is not enabled
and not supported as it integrates quite tightly into \CONTEXT. In
case of problems you can use \CONTEXT\ for tracking down problems.

\item Patching the original code in distributions is dangerous as
it might fix your problem but introduce new ones for \CONTEXT. So,
best keep the original code as it is and overload functions and
callbacks when needed. This is trivial in \LUA.

\item Attributes are (automatically) taken from the range 127||255 so
you'd best not use these yourself. Don't count on an attribute number
staying the same and don't mess with these attributes.

\stopitemize

If this all sounds a bit strict, keep in mind that it makes no sense
for us to maintain multiple code bases and we happen to use \CONTEXT.

\subject{advanced features}

The latest versions now also support font extending, slanting,
protrusion and expansion. Here are a few examples:

\starttyping
\pdfprotrudechars2 \pdfadjustspacing2

\font\testa=file:lmroman12-regular:+liga;extend=1.5         at 12pt
\font\testb=file:lmroman12-regular:+liga;slant=0.8          at 12pt
\font\testc=file:lmroman12-regular:+liga;protrusion=default at 12pt
\font\testd=file:lmroman12-regular:+liga;expansion=default  at 12pt
\stoptyping

The extend and slant options are similar to those used in
map files. The extend is limited to 10 and the slant to 1.

In the protrusion and expansion specification the keyword \type
{default} is an entry in a definition table. You can find an
example at the end of \type {font-dum.lua}.

A setup for expansion looks as follows:

\starttyping
fonts.expansions.setups['default'] = {
    stretch = 2, shrink = 2, step = .5, factor = 1,

    [byte('A')] = 0.5, [byte('B')] = 0.7,
    ...........
    [byte('8')] = 0.7, [byte('9')] = 0.7,
}
\stoptyping

The stretch, shrink and steps become font properties
and characters gets a value assigned. In pseudo code
it looks like:

\starttyping
chr(A).expansion_factor = 0.5 * factor
\stoptyping

The protrusion table has left and right protrusion
factors for each relevant character.

\starttyping
fonts.protrusions.setups['default'] = {
    factor = 1, left = 1, right = 1,

    [0x002C] = { 0, 1 }, -- comma
    [0x002E] = { 0, 1 }, -- period
    [0x003A] = { 0, 1 }, -- colon
    ........
    [0x061B] = { 0, 1 }, -- arabic semicolon
    [0x06D4] = { 0, 1 }, -- arabic full stop
}
\stoptyping

So, the comma will stick out in the right margin:

\starttyping
chr(comma).right_protruding = right * 1 * factor
\stoptyping

As we prefer measures relative to the width (precentages) we
actualy use:

\starttyping
chr(comma).right_protruding = right * 1 * factor * (width/quad)
\stoptyping

You can add additional tables and access them by keyword in the
font specification.

The model used in the plain variant is a simplification of the
\CONTEXT\ model so \CONTEXT\ users should not take this as
starting point.

\stopcomponent
