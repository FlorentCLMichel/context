% language=uk

\startcomponent mk-tokenspeak

\environment mk-environment

\chapter {Token speak}

\subject{tokenization}

Most \TEX\ users only deal with (keyed in) characters and (produced) output. Some
will play with boxes, skips and kerns or maybe even leaders (repeated sequences
of the former). Others will be grateful that macro package writers take care of
such things.

Macro writers on the other hand deal properties of characters, like catcodes and
a truckload of other codes, with lists made out of boxes, skips, kerns and
penalties but even they cannot look much deeper into \TEX's internals. Their
deeper understanding comes from reading the \TEX book or even looking at the
source code.

When someone enters the magic world of \TEX\ and starts asking around on a bit,
he or she will at some point get confronted with the concept of \quote {tokens}.
A token is what ends up in \TEX\ after characters have entered its machinery.
Sometimes it even seems that one is only considered a qualified macro writer if
one can talk the right token||speak. So what are those magic tokens and how can
\LUATEX\ shed light on this.

In a moment we will show examples of how \LUATEX\ turns characters into tokens,
but when looking at those sequences, you need to keep a few things in mind:

\startitemize[packed]
\startitem
    A sequence of characters that starts with an escape symbol (normally this is
    the backslash) is looked up in the hash table (which relates those names to
    meanings) and replaced by its reference. Such a reference is much faster than
    looking up the sequence each time.
\stopitem
\startitem
    Characters can have special meanings, for instance a dollar is often used to
    enter and exit math mode, and a percent symbol starts a comment and hides
    everything following it on the same line. These meanings are determined by
    the character's catcode.
\stopitem
\startitem
    All the characters that will end up actually typeset have catcode \quote
    {letter} or \quote {other} assigned. A sequence of items with catcode
    \quote{letter} is considered a word and can potentially become hyphenated.
\stopitem
\stopitemize

\subject{examples}

We will now provide a few examples of how \TEX\ sees your input.

\starttyping
Hi there!
\stoptyping

\starttokens[demo]Hi there!\stoptokens \setups{ShowCollect}

Here we see three kind ot tokens. At this stage a space is still recognizable as
such but later this will become a skip. In our current setup, the exclamation
mark is not a letter.

\starttyping
Hans \& Taco use Lua\TeX \char 33\relax
\stoptyping

\starttokens[demo]Hans \& Taco use Lua\TeX \char 33\relax\stoptokens \setups{ShowCollect}

Here we see a few new tokens, a \quote {char\_given} and a \quote {call}. The
first represents a \type {\chardef} i.e.\ a reference to a character slot in a
font, and the second one a macro that will expand to the \TEX\ logo. Watch how
the space after a control sequence is eaten up. The exclamation mark is a direct
reference to character slot~33.

\starttyping
\noindent {\bf Hans} \par \hbox{Taco} \endgraf
\stoptyping

\starttokens[demo]\noindent {\bf Hans} \par \hbox{Taco} \endgraf\stoptokens \setups{ShowCollect}

As you can see, some primitives and macro's that are bound to them (like \type
{\endgraf}) have an internal representation on top of their name.

\starttyping
before \dimen2=10pt after \the\dimen2
\stoptyping

\starttokens[demo]before \dimen2=10pt after \the\dimen2\stoptokens \setups{ShowCollect}

As you can see, registers are not explicitly named, one needs the associated
register code to determine it's character (a dimension in our case).

\starttyping
before \inframed[width=3cm]{whatever} after
\stoptyping

\starttokens[demo]before \inframed[width=3cm]{whatever} after\stoptokens \setups{ShowCollect}

As you can see, even when control sequences are collapsed into a reference, we
still end up with many tokens, and because each token has three properties (cmd,
chr and id) in practice we end up with more memory used after tokenization.

\starttyping
compound|-|word
\stoptyping

\starttokens[demo]compound|-|word\stoptokens \setups{ShowCollect}

This example uses an active character to handle compound words (a \CONTEXT\
feature).

\starttyping
hm, \directlua 0 { tex.sprint("Hello World") }
\stoptyping

\starttokens[demo]hm, \directlua 0 { tex.sprint("Hello World!") }\stoptokens \setups{ShowCollect}

The previous example shows what happens when we include a bit of \LUA\ code
\unknown\ it is just seen as regular input, but when the string is passed to
\LUA, only the chr property is passed, so we no longer can distinguish between
letters and other characters.

A macro definition converts to tokens as follows.

\starttokens[demo]\def\Test#1#2{[#2][#1]} \Test{A}{B}\stoptokens \setups{ShowCollect}

As we already mentioned, a token has three properties. More details can be found
in the reference manual so we will not go into much detail here.

{\bf The original interceptor for tokens but that one has been replaced by a more
powerful scanning mechanism. The following text is no longer applicable but kept
as historic reference. The new token scanner is discussed in later articles.}

% keep text formatted as it is now:

\starttyping[color=]

A most simple callback is:

\starttyping
callback.register('token_filter', token.get_next)
\stoptyping

In principle you can call \type {token.get_next} anytime you want
to intercept a token. In that case you can feed back tokens into
\TEX\ by using a trick like:

\starttyping
function tex.printlist(data)
   callback.register('token_filter', function ()
       callback.register('token_filter', nil)
       return data
    end)
end
\stoptyping

Another example of usage is:

\starttyping
callback.register('token_filter', function ()
    local t = token.get_next
    local cmd, chr, id = t[1], t[2], t[3]
    -- do something with cmd, chr, id
    return { cmd, chr, id }
end)
\stoptyping

There is a whole repertoire of related functions, one is \type
{token.create}, which can be used as:

\starttyping
tex.printlist{
    token.create("hbox"),
    token.create(utf.byte("{"),  1),
    token.create(utf.byte("?"), 12),
    token.create(utf.byte("}"),  2),
}
\stoptyping

This results in: \ctxlua {
    tex.printlist{
        token.create("hbox"),
        token.create(utf.byte("{"),  1),
        token.create(utf.byte("?"), 12),
        token.create(utf.byte("}"),  2),
    }
}

While playing with this we made a few auxiliary functions that
permit things like:

\starttyping
tex.printlist ( table.unnest ( {
    tokens.hbox,
    tokens.bgroup,
    tokens.letters("12345"),
    tokens.egroup,
} ) )
\stoptyping

Unnesting is needed because the result of the \type {letters} call
is a table, and the \type {printlist} function wants a flattened
table.

The result looks like: \ctxlua {
    local t = table.unnest {
        tokens.hbox,
        tokens.bgroup,
        tokens.letters("12345"),
        tokens.egroup,
    }
    tex.printlist (t)
    tokens.collectors.show(t)
}

In practice, manipulating tokens or constructing lists of tokens
this way is rather cumbersome, but at least we now have some
kind of access, if only for illustrative purposes.

\starttyping
\hbox{12345\hbox{54321}}
\stoptyping

can also be done by saying:

\starttyping
tex.sprint("\\hbox{12345\\hbox{54321}}")
\stoptyping

or under \CONTEXT's basic catcode regime:

\starttyping
tex.sprint(tex.ctxcatcodes, "\\hbox{12345\\hbox{54321}}")
\stoptyping

If you like it the hard way:

\starttyping
tex.printlist ( table.unnest ( {
    tokens.hbox,
        tokens.bgroup,
            tokens.letters("12345"),
            tokens.hbox,
                tokens.bgroup,
                    tokens.letters(string.reverse("12345")),
                tokens.egroup,
        tokens.egroup
} ) )
\stoptyping

This method may attract those who dislike the traditional \TEX\
syntax for doing the same thing. Okay, a careful reader will
notice that reversing the string in \TEX\ takes a bit more
trickery, so \unknown

\stoptyping

% end of verbose text

{\bf The \type {tokens} etc.\ examples shows here make no sense anyway as we have
a more extensive interface to the macro language: \type {context}.}

\stopcomponent
