\startcomponent ma-cb-en-descriptions

\enablemode[**en-us]

\project ma-cb

\startchapter[title=Definitions]

\index{definition}

\Command{\tex{definedescription}}
\Command{\tex{setupdescriptions}}

If you want to display notions, concepts and ideas in a consistent manner you can
use:

\shortsetup{definedescription}

For example:

\startbuffer
\definedescription
  [concept]
  [alternative=serried,headstyle=bold,width=broad]

\concept{Hasselter juffer} A sort of biscuit made of puff pastry and
covered with sugar. It tastes very sweet. \par
\stopbuffer

\typebuffer

It would look like this:

\getbuffer

But you can also choose other layouts:

\startbuffer
\definedescription
  [concept]
  [alternative=top,
   inbetween={\blank[none]},
   headstyle=bold,
   width=broad,
   style=slanted]

\concept{Hasselter bitter} A very strong alcoholic drink (up to 40\%)
mixed with herbs to give it a special taste. It is sold in a stone
flask and it should be served {\em ijskoud} (as cold as ice). \par

\definedescription
  [concept]
  [alternative=inmargin,headstyle=bold,width=broad]

\concept{Euifeest} A harvest home to celebrate the end of a period of
hard work. The festivities take place in the last week of August. \par
\stopbuffer

\start
\getbuffer
\stop

If you want to avoid the \type{\par} or when you have more than one paragraph in
the definition you can use the \type{\start...\stop} construct.

\startbuffer
\definedescription
  [concept]
  [alternative=right,
   headstyle=bold,
   width=broad]

\startconcept{Euifeest} A harvest home to celebrate the end of a
period of hard work.
This event takes place at the end of August and lasts one week. The
city is completely illuminated and the streets are decorated. This
feast week ends with a {\em Braderie}.
\stopconcept
\stopbuffer

\typebuffer

This would become:

\getbuffer

Layout is set up within the second bracket pair of
\type{\definedescription[][]}. But you can also use:

\shortsetup{setupdescriptions}

\stopchapter

\stopcomponent
