% language=us

\usemodule[present-boring,abbreviations-logos,system-tokens]

\startdocument
  [title={TOKENS},
   banner={tokens as I see them},
   location={context\enspace {\bf 2020}\enspace meeting}]

\starttitle[title=About tokens]

\startitemize

\startitem Like nodes, it's a common term used in programming. \stopitem
\startitem In \TEX\ The Program tokens and nodes are therefore omni|-|present. \stopitem
\startitem For most users they are irrelevant concepts. \stopitem
\startitem But we will explain them anyway. \stopitem
\startitem Let's try to avoid the snobbish token|-|speak sometimes heard in the community. \stopitem
\startitem So \unknown\ I won't correct you as long as you don't correct me. \stopitem
\startitem Let's now enter the world of tokens in the na\"ive way. \stopitem

\stopitemize

\stoptitle

\starttitle[title=What are tokens]

\startitemize

\startitem It is an internal data structure, effectively a (32 bit) integer. \stopitem
\startitem This integer encodes a command (opcode) and an char code (operand). \stopitem
\startitem But often it's not a character but more a sub command. \stopitem
\startitem Input is converted into tokens. \stopitem
\startitem Tokens are either expanded (interpreted) or stored. \stopitem
\startitem When they are stored they are part of a larger data structure, a memory word. \stopitem
\startitem Token memory is an array of such memory words. \stopitem
\startitem The token memory \quote {word} has two integers: a token value and an index into token memory. \stopitem
\startitem That way \TEX\ can have forward linked lists of tokens. \stopitem
\startitem A hash table maps control sequences onto indices into token memory. \stopitem

\stopitemize

\starttitle[title=Some implementation details]

\startitemize

\startitem Sometimes there is special head token at the start. \stopitem
\startitem A head token makes for easier appending of extra tokens. \stopitem
\startitem Shared lists use the head node for a reference count. \stopitem
\startitem Original \TEX\ uses global temporary lists. \stopitem
\startitem This is needed when we expand (nested) and need to report issues. \stopitem
\startitem This is not needed when we just serialize (which we do a lot in \LUATEX). \stopitem
\startitem So, this is all optimized for performance and memory consumption. \stopitem
\startitem Freed tokens are collected in a cache so tokens can get scattered. \stopitem
\startitem In \LUAMETATEX\ we stay as close to original \TEX\ as possible. \stopitem
\startitem But the \LUA\ interfaces force us to occasionally divert. \stopitem

\stopitemize

\stoptitle

\starttitle[title=A schematic view of tokens]

A token value:

\startlinecorrection[blank]
    \setupTABLE[each][align=middle]
    \setupTABLE[c][1][width=22mm]
    \setupTABLE[c][2][width=42mm]
    \bTABLE
        \bTR \bTD cmd \eTD \bTD chr \eTD \eTR
    \eTABLE
\stoplinecorrection

Token memory:

\startlinecorrection[blank]
    \setupTABLE[each][align=middle]
    \setupTABLE[c][1][width=8mm]
    \setupTABLE[c][2][width=64mm]
    \setupTABLE[c][3][width=64mm]
    \bTABLE
        \bTR \bTD 1 \eTD \bTD info \eTD \bTD link \eTD \eTR
        \bTR \bTD 2 \eTD \bTD info \eTD \bTD link \eTD \eTR
        \bTR \bTD 3 \eTD \bTD info \eTD \bTD link \eTD \eTR
        \bTR \bTD n \eTD \bTD info \eTD \bTD link \eTD \eTR
    \eTABLE
\stoplinecorrection

\stoptitle

\starttitle[title=Looking up control sequences]

\startitemize

\startitem A very visible to-be-token is a \type {\controlsequence}. \stopitem
\startitem When read, the name will be looked up in the hash table. \stopitem
\startitem When found its value will point to the table of equivalents. \stopitem
\startitem That table keeps track of:
    \startitemize
        \startitem the type (cmd) \stopitem
        \startitem the current level (grouping) \stopitem
        \startitem the current meaning (token list) \stopitem
    \stopitemize
\stopitem
\stopitemize

\stoptitle

\starttitle[title=The (big) table of equivalents (simplified)]

\startlinecorrection[blank]
    \bTABLE
        \bTR \bTD[ny=4] main hash \eTD \bTD null control sequence              \eTD \eTR
        \bTR                           \bTD 128K hash entries                  \eTD \eTR
        \bTR                           \bTD frozen control sequences           \eTD \eTR
        \bTR                           \bTD special sequences (undefined)      \eTD \eTR
        \bTR \bTD[ny=7] registers \eTD \bTD  17 internal & 64K user glues      \eTD \eTR
        \bTR                           \bTD   4 internal & 64K user mu glues   \eTD \eTR
        \bTR                           \bTD  12 internal & 64K user tokens     \eTD \eTR
        \bTR                           \bTD   2 internal & 64K user boxes      \eTD \eTR
        \bTR                           \bTD 116 internal & 64K user integers   \eTD \eTR
        \bTR                           \bTD   0 internal & 64K user attribute  \eTD \eTR
        \bTR                           \bTD  22 internal & 64K user dimensions \eTD \eTR
        \bTR \bTD specifications  \eTD \bTD   5 internal &   0 user            \eTD \eTR
        \bTR \bTD extra hash      \eTD \bTD additional entries (grows dynamic) \eTD \eTR
    \eTABLE
\stoplinecorrection

\stoptable

\starttitle[title=The hash table (simplified)]

The hash table runs parallel to the main hash. On the todo list is is to move the
registers to its own tables and make them dynamic.

\startlinecorrection[blank]
    \setupTABLE[each][align=middle]
    \setupTABLE[c][1][width=16mm]
    \setupTABLE[c][2][width=64mm]
    \setupTABLE[c][3][width=64mm]
    \bTABLE
        \bTR \bTD 1     \eTD \bTD string index \eTD \bTD equivalents or (next > n) index \eTD \eTR
        \bTR \bTD 2     \eTD \bTD string index \eTD \bTD equivalents or (next > n) index \eTD \eTR
        \bTR \bTD n     \eTD \bTD string index \eTD \bTD equivalents or (next > n) index \eTD \eTR
        \bTR \bTD n + 1 \eTD \bTD string index \eTD \bTD equivalents or (next > n) index \eTD \eTR
        \bTR \bTD n + 2 \eTD \bTD string index \eTD \bTD equivalents or (next > n) index \eTD \eTR
        \bTR \bTD n + m \eTD \bTD string index \eTD \bTD equivalents or (next > n) index \eTD \eTR
    \eTABLE
\stoplinecorrection

Equivalents (registers direct, macros indirect i.e.\ token lists):

\startlinecorrection[blank]
    \setupTABLE[each][align=middle]
    \setupTABLE[c][1][width=8mm]
    \setupTABLE[c][2][width=32mm]
    \setupTABLE[c][3][width=32mm]
    \setupTABLE[c][4][width=64mm]
    \bTABLE
        \bTR \bTD 1 \eTD \bTD level \eTD \bTD type \eTD \bTD value \eTD \eTR
        \bTR \bTD 2 \eTD \bTD level \eTD \bTD type \eTD \bTD value \eTD \eTR
        \bTR \bTD 3 \eTD \bTD level \eTD \bTD type \eTD \bTD value \eTD \eTR
        \bTR \bTD n \eTD \bTD level \eTD \bTD type \eTD \bTD value \eTD \eTR
    \eTABLE
\stoplinecorrection

\stoptitle

\starttitle[title=Other data management]

\startitemize
\startitem Grouping is handles by a nesting stack. \stopitem
\startitem Nested conditionals (\type {\if...}) have their own stack. \stopitem
\startitem The values before assignments are saved ion the save stack. \stopitem
\startitem Also other local changes (housekeeping) ends up in the save stack. \stopitem
\startitem Token lists and macro aliases have references pointers (reuse). \stopitem
\startitem Attributes, being linked node lists, have their own management. \stopitem
\stopitemize

\stoptitle

\starttitle[title=Example 1: in the input]

\startbuffer
\luatokentable{1 \bf{2} 3\what {!}}
\stopbuffer

\typebuffer \blank[line] {\switchtobodyfont[8pt] \getbuffer}

\stoptitle

\starttitle[title=Example 1: in the input]

\startbuffer
\luatokentable{a \the\scratchcounter b \the\parindent \hbox to 10pt{x}}
\stopbuffer

\typebuffer \blank[line] {\switchtobodyfont[8pt] \getbuffer}

\stoptitle

\starttitle[title=Example 2: user registers]

\startbuffer
\scratchtoks{foo \framed{\red 123}456}

\luatokentable\scratchtoks
\stopbuffer

\typebuffer \blank[line] {\switchtobodyfont[8pt] \getbuffer}

\stoptitle

\starttitle[title=Example 3: internal variables]

\startbuffer
\luatokentable\everypar
\stopbuffer

\typebuffer \blank[line] {\switchtobodyfont[8pt] \getbuffer}

\stoptitle

\starttitle[title=Example 4: macro definitions]

\startbuffer
\protected\def\whatever#1[#2](#3)\relax{oeps #1 and #2 & #3 done ## error}

\luatokentable\whatever
\stopbuffer

\typebuffer \blank[line] {\switchtobodyfont[8pt] \startcolumns \getbuffer \stopcolumns}

\stoptitle

\starttitle[title=Example 5: commands]

\startbuffer
\luatokentable\startitemize
\stopbuffer

\typebuffer \blank[line] {\switchtobodyfont[8pt] \getbuffer}

\stoptitle

\starttitle[title=Example 6: commands]

\startbuffer
\luatokentable\doifelse
\stopbuffer

\typebuffer \blank[line] {\switchtobodyfont[8pt] \getbuffer }

\stoptitle

\starttitle[title=Example 7: nothing]

\startbuffer
\luatokentable\relax
\stopbuffer

\typebuffer \blank[line] {\switchtobodyfont[8pt] \getbuffer }

\stoptitle

\starttitle[title=Example 8: Hashes]

\startbuffer
\edef\foo#1#2{(#1)(\letterhash)(#2)}  \luatokentable\foo
\stopbuffer

\typebuffer \blank[line] {\switchtobodyfont[8pt] \getbuffer }

\stoptitle

\starttitle[title=Example 9: Nesting]

\startbuffer
\def\foo#1{\def\foo##1{(#1)(##1)}}  \luatokentable\foo
\stopbuffer

\typebuffer \blank[line] {\switchtobodyfont[8pt] \getbuffer }

\stoptitle

\stopdocument
