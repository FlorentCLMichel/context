\usemodule[present-stepwise,present-bars,abr-01]

\startdocument
  [title=After the cleanup,
   color=darkred]

\StartSteps

\startsubject[title=The update]

    \startitemize[packed]

        \startitem The move to \MKIV\ is more than supporting an engine. \stopitem \FlushStep
        \startitem It is a complete rewrite (pruning, extending, cleaning). \stopitem \FlushStep
        \startitem Although somewhat crippled by the fact that we want to remain compatible. \stopitem \FlushStep
        \startitem But sometimes we sacrifice compatibility by getting rid of old stuff. \stopitem \FlushStep

    \stopitemize

\stopsubject

\startsubject[title=The current state]

    \startitemize[packed]

        \startitem A lot of work, more than I had thought, so it takes longer. \stopitem \FlushStep
        \startitem Most \TEX\ code is done (some structure and column code left). \stopitem \FlushStep
        \startitem New namespaces and helpers mostly done, but will be checked for constency. \stopitem \FlushStep

    \stopitemize

\stopsubject

\StopSteps

\page

\StartSteps

\startsubject[title=What is there todo]

    \startitemize[packed]

        \startitem Some code might become generalized (also depends on others). \stopitem \FlushStep
        \startitem Layer and positioning code might get a more extensive \LUA\ and \XML\ interface. \stopitem \FlushStep
        \startitem Structure related code will support setups (some already in place). \stopitem \FlushStep
        \startitem New page builder variants will be explored (anyway more column support and floats). \stopitem \FlushStep
        \startitem Math domains cq.\ dictionaries (basics already in place, just needs time). \stopitem \FlushStep
        \startitem Math list optimization (pet project). \stopitem \FlushStep
        \startitem Generate dependecy trees (easier now) and more consistent code loading order. \stopitem \FlushStep
        \startitem All error messages needs checking (some gone, some not yet translated). \stopitem \FlushStep
        \startitem Update all xml definitions (work in progress, also relates to wiki). \stopitem \FlushStep
        \startitem Optimize positioning system (a bit more powerful now, but also more resources). \stopitem \FlushStep
        \startitem More support for css like styling (makes it easier to share code). \stopitem \FlushStep
        \startitem Modules (especially those for tracing) need to be normalized. \stopitem \FlushStep
        \startitem Some styles (mostly private presentation styles) needs to be fixed. \stopitem \FlushStep
        \startitem Pick up the \quote {lean and mean} \CONTEXT\ variant project. \stopitem \FlushStep
        \startitem Now that we have more code isolated, we can define an api. \stopitem \FlushStep
        \startitem Some manuals need to be updated (most still applies). \stopitem \FlushStep

    \stopitemize

\stopsubject

\StopSteps

\page

\StartSteps

\startsubject[title=What I have to keep in mind]

    \startitemize[packed]

        \startitem What is handy for me is not always handy for all users. \stopitem \FlushStep

    \stopitemize

\stopsubject

\startsubject[title=But nevertheless there will be new things]

    \startitemize[packed]

        \startitem Elements of our processing framework will show up in the distribution. \stopitem \FlushStep
        \startitem It's just more convenient to have one installation for related things. \stopitem \FlushStep
        \startitem This is also why support for databases has been added recently. \stopitem \FlushStep
        \startitem Running (blocking) \TEX\ jobs needs special treatment (ticket management). \stopitem \FlushStep
        \startitem It makes sense to use the well developed \TDS\ infrastructure. \stopitem \FlushStep

    \stopitemize

\stopsubject

\StopSteps

\page

\StartSteps

\startsubject[title=Keep an eye on updates]

    \startitemize[packed]

        \startitem Rewriting the code base leads to bugs but these are often resolved quickly
                   (indeed by Wolfgang). \stopitem \FlushStep
        \startitem Following the mailing list helps and nowadays the wiki is adapted close to
                   realtime (coordinated by Sietse). \stopitem \FlushStep
        \startitem Changes in standards and related tools are supported and followed by those who
                   depend on them (ask Peter). \stopitem \FlushStep
        \startitem Sometimes users have demands and these end up as extensions to existing
                   mechanisms (Aditya's elastic modules). \stopitem \FlushStep
        \startitem Issues with platforms are often quickly dealt with (if Luigi doesn't know it
                   \unknown). \stopitem \FlushStep
        \startitem And of course I add new things driven by projects, challenges (and an occasional
                   stack of new \CD's). \stopitem \FlushStep
        \startitem New releases (and betas) are checked against a growing set of test files (Lukas
                   mails a report after each update). \stopitem \FlushStep
    \stopitemize

\stopsubject

\StopSteps

\page

\StartSteps

\startsubject[title=Just ask]

    \startitemize[packed]

        \startitem We started long ago with what ended up as \MKII\ and \MPII. \stopitem \FlushStep
        \startitem We currently have \MKIV\ and \MPIV. \stopitem \FlushStep
        \startitem It has some features that we tag as \MKVI. \stopitem \FlushStep
        \startitem Recently \MKIX\ and \MKXI\ were introduced. \stopitem \FlushStep
        \blank
        \startitem Examples: \MKIV, \MKVI, \MKIX, \MKXI \stopitem \FlushStep
        \blank
        \startitem So, what should \MKIC\ provide \stopitem \FlushStep

    \stopitemize

\stopsubject

\startsubject[title=What I'm working on]

    \startitemize[packed]

        \startitem Proper dependency chain so that we can make small dedicated formats. \stopitem \FlushStep
        \startitem Support for typesetting from databases (text, graphics). \stopitem \FlushStep
        \startitem Next iteration if (job) tickets processing system cq. framework. \stopitem \FlushStep

    \stopitemize

\stopsubject

\StopSteps

% show latest manuals

\stopdocument
