\usemodule[present-dark]

\usemodule[abr-01]

\startdocument

\StartIdea
    [ title={Hans Hagen},
     remark={PRAGMA ADE, Hasselt NL},
        url={www.pragma-ade.com}]

{\bfd \setstrut \strut TUG 2001}

{\bfa \setstrut \strut A \TEX\ Odyssey}

\blank[2*big]

\startitemize [packed]
    \startitem what way are we heading \stopitem
    \startitem will there be documents \stopitem
    \startitem is typography still needed \stopitem
    \startitem are we still talking \TEX \stopitem
\stopitemize

\StopIdea

\StartIdea

Until now, the main source of information is books. In the next couple of slides,
I will present some quotes from books I read the last couple of years, written
by: Arthur \remark {Clarke} {physics}, Greg \remark {Bear} {psychology}, Graham
\remark {Hancock} {journalism}, Peter \remark {Wilbur} {ergonomics} and Michael
Burke, Jared \remark {Diamond} {history}, Edward \remark {Tufte} {design}, Peter
Ward and Donald \remark {Brownlee} {biology}, Steve \remark{Reich} {music} and
Beryl Korot, Richard \remark {Kadrey} {fantasy}, Brian \remark {Butterworth}
{math} and of course Donald \remark {Knuth} {informatics}.

\StopIdea

\StartIdea
    [ title={ed. David G. Stork},
     remark={Hal's Legacy, 1997}]

[Arthur Clarke:] Although I've never considered 2001 as a strict
predict\-ion|<|but as more of a vision, a way things could work|>|I have long
kept track, informally, of how our vision compares with computer science reality.
Some things we got right---even righter than we ever had reason to suspect.
Others, well, who could have \remark {known} {so, to what extent can we predict
the future of documents}.

[Summary:] much of the science predicted in 1968 is okay, but with regards to
\remark {computers} {in this respect, \TeX\ is surprisingly up|-|to|-|date} a
couple of points are missed: they have become smaller, \remark {AI} {and
automated text processing is still difficult} is far from operational, natural
speech, reasoning and lipreading are not really available, fault tolerance is
there, we have lcd's, graphical user interfaces and windows, don't communicate in
terminal messages, have mice and other means of input.

\StopIdea

\StartIdea
    [ title={Arthur Clarke},
     remark={2001, A space Odyssey, p. 66, 1968}]

After a short walk through a tunnel packed with pipes and \remark {cables} {we
are already going wireless}, and echoing hollowly with rhythmic thumbing and
throbbings, they arrived in executive territory, and Floyd found himself back in
the familiar environment of \remark {typewriters} {the good old times of \quote
{think before you key}}, \remark {office} {indeed, most of today's users run
\quote {office}} computers, \remark {girl} {everyone is now a typist} assistants,
\remark {wall} {when will we go virtual} charts and ringing telephones.

\StopIdea

\StartIdea
    [ title={Arthur Clarke},
     remark={2001, A space Odyssey, p. 67, 1968}]

There was plenty to occupy his time, even if he did nothing but sit and read.
When he tired of official reports and memoranda and minutes he would \remark
{plug} {documents will be in the air} his foolscap|-|sized newspad into the
ship's information circuit and scan the latest reports from Earth. One by one he
would conjure up the world's major electronic papers; he knew the \remark {codes}
{who is using codes today} of the more important ones by heart, and had no need
to consult the list on the back of his pad. Switching to the display's unit's
short|-|term memory, he would hold the front page while he quickly \remark
{searched} {we are very good in quick browsing} the headlines and noted the items
that interested him. Each had its own \remark {two|-|digit} {aren't we running
out of 256.256.256.256 already} reference; when he punched that, the
postage|-|stamp|-|sized rectangle would expand until it neatly filled the screen,
and he could read it with comfort. When had finished he would flash back to the
complete remark {page} {will we keep on using composed mixed content pages} and
select a new subject for detailed examination.

\StopIdea

\StartIdea
    [ title={Arthur Clarke},
     remark={2001, A space Odyssey, p. 109, 1968}]

Bowman had been a student for more than a half his life; he would continue to be
one until he retired. Thanks to the Twentieth Century \remark {revolution} {that
has been a pretty quiet revolution then} in training and information|-|handling
techniques, he already possessed the \remark {equivalent} {we will stop talking
in those qualifications} of two or three college educations|=|and, what was more,
he could \remark {remember} {with or without implant} \remark {ninety} {is this
still needed with information everywhere} per cent of what he had learned.

\StopIdea

\StartIdea
    [ title={Arthur Clarke},
     remark={2001, A space Odyssey, p. 132, 1968}]

The information flashed on the display screen; simultaneously, a sheet of paper
slit out of the slot immediately \remark {beneath} {but aren't screens becoming
like paper} it. Despite all the electronic read|-|outs, there were times when
good, old|-|fashioned printed material was the most \remark {convenient} {good,
because paper is a great invention} form of record.

\StopIdea

\StartIdea
    [ title={Greg Bear},
     remark={Eon, p. 30, 1985}]

The office was neatly organized but still looked cluttered. A small desk
manufactured from OTV tank baffles was flanked by chromium bins filled with
\remark {rolls} {not much paper will be used in space, I guess} of paper. A
narrow shelf of \remark {real} {that sounds pretty sad for around 2000} books
hung next to \remark {racks} {will there be such a physical need} of memory
blocks sealed behind tough, alarm-equipped plastic panels. \remark {Maps} {we
will probably always need an overview} and \remark {diagrams} {and for that we
need large projections} were taped to the wall.

\StopIdea

\StartIdea
    [ title={Greg Bear},
     remark={Eon, p. 132, 1985}]

Still, she agreed with a nod and settled into the seat, manipulating the controls
with one hand. A simple \remark {circular} {will we move away from rectangular
presentations} graphic display \remark {hovered} {that's indeed what we want}
before her, as crisp and \remark {clear} {good} as something \remark {solid}
{even better}. Takahashi had misinformed her on one point, and her fumbling
triggered a tutorial. It corrected her errors and informed her|<|in only slightly
\remark {accented} {-)} American \remark {English} {what a pitty for dislectic
people}|>|how to operate the equipment properly. Then it provided her with call
\remark {numbers} {we really love numbers, don't we} and codes for other types of
information.

\StopIdea

\StartIdea
    [ title={Greg Bear},
     remark={Eon, pp. 132/135, 1985}]

The \remark {illusion} {physical presence will become less important} was
perfect|<|even providing her with a memory of what her apartment looked like. She
could turn her head and look completely behind her if she wished|>|indeed, she
could walk around, even through she knew she was sitting down. \unknown\ The
information had come in \remark {printed} {don't throw away eons of experience}
displays, selected \remark {visuals} {will we keep on changing interfaces} and
even more selected \remark {sounds} {we should have started recording already}.
Where documentation of the multimedia sort was lacking, print took over, but with
subtle and clear vocal accompaniment. Compared to this, simple reading was
\remark {torture} {hm, \unknown} and current video methods as \remark {archaic}
{eh, \unknown} as cave \remark {paintings} {let's be humble then}.

\StopIdea

\StartIdea
    [ title={Greg Bear},
     remark={Eon, p. 258, 1985}]

\quotation {The P.M.\ has no suspicion of this when you alone were sent?} Toller
\remark {picted} {finally ideographic scripts will win the game}. The symbols
that flashed between the two men came from pictor torques around their necks,
\remark {devices} {we really need an physical update} that had developed over the
centuries in the Thistledown and in the Axis City.

\StopIdea

\StartIdea
    [ title={Graham Hancock},
     remark={Fingerprints of the gods, Graham Hancock, p. 120, 1995}]

More systematically, all over Central America, vast repositories of knowledge
\remark {accumulated} {as today in libraries, on servers and in our houses} since
ancient times were painstakingly gathered, heaped up and burned by zealous
friars. In July 1562, for example, in the main square of Mani (just south of
modern Merida in Yucatan Province) Fr Diego de Landa \remark {burned} {and all
can get lost forever} thousands of Maya codices, story paintings and hieroglyphs
inscribed on rolled|-|up deer \remark {skins} {how about bits curled up on
CDROM's}.

\StopIdea

\StartIdea
    [ title={Graham Hancock},
     remark={Fingerprints of the gods, Graham Hancock, p. 520/526, 1995}]

We know that out late twentieth|-|century, post|-|industrial civilization is
about to be destroyed by an \remark {inescapable} {not that imaginary, it has
happened before} cosmic or geological cataclysm.

We know|<|because our science is pretty good|>|that the destruction is going to
be \remark {{\em near|-|total}} {one 10-30 km meteor or even one lunatic
president will do}.

\blank \unknown \blank

I'm sure that we'd want to say more than just \quote {Kilroy was here}.

\blank \unknown \blank

And, yes, \remark {they} {pyramid builders 12,000 years ago} found an ingenious
way to tell \remark {us} {who more and more think short|-|term} that they were
\remark {here} {what will we leave behind}.

\StopIdea

\StartIdea
    [ title={Jared Diamond},
     remark={Guns, Germs and Steel, A Short History of Everybody for the Last
             13,000 years, p. 260, 1997}]

Human technology developed from the first stone tools, in use by two and a half
million years ago, to the 1996 laser \remark {printer} {the ones that produced
sticky fading print|-|outs} that replaced my already outdated 1992 \remark
{laser} {and now we want color on the desktop} printer and that was used to print
this book's manuscript. The rate of development was undetectably slow at the
beginning, when hundreds of thousands of years passed with no discernible change
in out stone tools and with no surviving evidence for artifacts and of other
materials. Today, technology advances so \remark {rapidly} {so let's be careful
in claiming advance} that it is reported in the daily \remark {newspaper} {less
and less people read them}.

\StopIdea

\StartIdea
    [ title={Jared Diamond},
     remark={Guns, Germs and Steel, A Short History of Everybody for the Last
             13,000 years, p. 418, 1997}]

The decision could have gone to another keyboard at any of numerous stages
between the 1860s and the 1880's; nothing about the American environment favored
the \hbox {QWERTY} keyboard over its rivals. \unknown\ For example, if the \hbox
{QWERTY} keyboard of the United States had not been adopted elsewhere in the
world as well|<|say, if Japan or Europe had adopted the more \remark {efficient}
{so why don't we take that one} Dvorak keyboard|>|that trivial decision in the
19\high{th} century might have had big consequences for the competative position
of the 20\high{th}|-|century \remark {American} {isn't \TeX\ also best tuned for
english} technology.

\StopIdea

\StartIdea
    [ title={Peter Wilbur \& Michael Burke},
     remark={Information Graphics, Innovative Solutions in Contemporary Design,
             p. 87, 1998}]

It was generally \remark {agreed} {so let's judge with care} at that time that
products which tried to fulfil two or more \remark {functions} {how many
functions are there in a book} were compromises and therefore inferior to a
single|-|function product.

\StopIdea

\StartIdea
    [ title={Peter Wilbur \& Michael Burke},
     remark={Information Graphics, Innovative Solutions in Contemporary Design,
             p. 17, 1998}]

All of this implies that design students of the future will need to have a much
wider range of skills than most graphic and multimedia students possess today.
The coming \remark {together} {which is better: overloaded CNN news screens or
the more traditional ones} of typography, graphics, the moving image, sound and
music requires training in both \remark {aesthetic} {let's hope for the best}
judgment and technical skills, as well as the ability to implement and commission
\remark {multimedia} {the current hype will become a decent craft} productions.
Such a program hardly exists today, and it may be that \remark {designers} {or
will machines do the work} of the future will find themselves on courses equal in
duration and related in structure to those followed by architects.

\StopIdea

\StartIdea
    [ title={Greg Bear},
     remark={Darwin's radio, p. 271, 1999}]

\quotation {As far as it goes}, Kaye said. \quotation {I believe our genome is
much more \remark {clever} {let's hope that we can cope with the future} than we
are. It's taken us tens of thousands of years to get to to the point where we
have a hope of understanding how life works. \unknown\ The Earth species have
learned how to anticipate climate change and respond to it in advance, get a head
start, and I believe, in our case, our genome is now responding to social \remark
{change} {like writing, reading, processing, collecting information} and the
\remark {stress} {ability to keep track of things} it causes.}

\StopIdea

\StartIdea
    [ title={Greg Bear},
     remark={Darwin's radio, p. 404, 1999}]

She looked at the cover and laughed out loud. It was a copy of WIRED, and on the
brilliant orange cover was printed the black silhouette of a curled fetus with a
green question mark across the middle. The log line read \quotation {\em Human
3.0: Not a Virus, but an \remark {Upgrade} {or: complex talking & communicating
in color, smell and taste}?}

\StopIdea

\StartIdea
    [title={Edward R. Tufte}]

We thrive in information|-|thick worlds because of our marvelous and everyday
\remark {capacity} {that is us, now, or maybe until recently} to select, edit,
single out, structure, highlight, group, pair, merge, harmonize, synthesize,
focus, organize, condense, reduce, boil down, choose, categorize, catalog,
classify, list, abstract, scan, look into, idealize, isolate, discriminate,
distinguish, screen, pigeonhole, pick over, sort, integrate, blend, inspect,
filter, lump, skip, smooth, chunk, average, approximate, cluster, aggregate,
outline, summarize, itemize, review, dip into, flip through, browse, glance into,
leaf through, skim, refine, enumerate, glean, synopsize, \remark {winnow} {do we
really} the wheat from the chaff and separate the sheep from the goats.

\StopIdea

\StartIdea
    [ title={Donald E. Knuth},
     remark={Selected Papers in Computer Science, p. 95, 1996}]

I believe that the real reason underlying the fact that Computer Science has
become a thriving discipline at essential all of the world's universities,
although it was totally \remark {unknown} {much more is yet unknown, but we don't
know what} twenty years ago, is {\em not} that computers exist in quantity; the
real reason is that the algorithmic thinkers among scientists of the world never
before had a home. We are brought \remark {together} {there will be more new
disciplines} in Computer Science departments {\em because we find people who
think like we do}. At least, that seems a viable hypothesis, which hasn't been
contradicted by my observations during the last half dozen or so years since the
possibility occurred to me.

\StopIdea

\StartIdea
    [ title={Brian Butterworth},
     remark={The Mathematical Brain, p. 162, 1999}]

Nevertheless, it is now abundantly clear that infants are born with a \remark
{capacity} {what more is lurking there} to recognize distinct numerosities up to
about~4, and to respond to changes in numerosity. They also possess arithmical
expectations: ....

\StopIdea

\StartIdea
    [ title={Brian Butterworth},
     remark={The Mathematical Brain, p. 275, 1999}]

Imagine, if you can, asking Archimed, the greatest mathematician of antiquity, to
solve the equation:

\startformula
2a^2 + 3ab - 4b^2 = 0
\stopformula

\remark {He} {would your parents recognize \type {<tags>} as such} would have
less chance than an average educated fourteen|-|year|-|old, simply because he
would not know what the strange \remark {symbols} {or recognize hyperlinks} $0$,
$2$, $3$, and $4$ mean because thet weren't invented till seven centuries after
his murder; nor $+$ and $-$, German inventions of the fifteenth century; not to
mention \remark {$=$} {or be able to interpret a regular expression}, which was
invented by the Englishman Robert Recorde in the sixteenth century. He would also
have had a problem with the \remark {idea} {or be able to picture the internet}
that equations can have negative roots.

\StopIdea

\StartIdea
    [ title={Richard Kadrey},
     remark={From Myst to Riven, the Creations and Inspirations, p. 16, 1997}]

Some of basics of the D'ni bookmaking are known, but the most important \remark
{details} {can we still make Gutenberg bibles} have been \remark {lost} {how do
we preserve what we have} over time. \unknown\ From the few existing \remark
{records} {how much is really new} lost it appears that the D'ni have been using
their Linking books for millenia, and that they \remark {linked} {then they
manage their links better than we do} to the earth around 10,000 terrestial years
ago.

\StopIdea

\StartIdea
    [ title={Richard Kadrey},
     remark={From Myst to Riven, the Creations and Inspirations, p. 81,1997}]

Glancing at the surface of thing, {\em Myst} and {\em Riven} might seem more of a
technical achievement in computer \remark {artistry} {for this a real new way of
thinking is needed} and the fine points of modeling frames for objects and
designing surface textures and shader programs to reflect hyper|-|reality. It is
very easy to focus exclusively on the cool factor of what you see and to overlook
what is the underlying key to the success of these games: they are \remark
{story} {authorship will change} driven. What really sucks the player in is that
there is a deeply felt {\em purpose} to playing the \remark {game} {and the less
we need to work, the more we will game}.

\StopIdea

\StartIdea
    [ title={Steve Reich \& Beryl Korot},
     remark={The Cave, 1995}]

The true underpinnings were our interest in making a \remark {new} {the time is
ready for revolutionary new ways of presenting information} kind of musical
theater based on videotaped documentary sources. The idea was that you would be
able to see and hear people as they spoke on the videotape and simultaneously you
would see and hear on|-|stage musicians \remark {doubling} {also accompanied by
char|-|by|-|char typesetting} them|=|actually playing their speech melodies as
they spoke.

\StopIdea

\StartIdea
    [ title={Peter D. Ward \& Donald Brownlee},
     remark={Rare Earth, Why Complex Life us Uncommon in the Universe, p. xxiv,
             2000}]

If it is found to be correct, however, the Rare Earth Hypothesis will reverse
that decentering trend. What if the Earth, with its cargo of advanced animals, is
virtually unique in this quadrant of the galaxy|=|the most diverse planet, say,
in the nearest 10,000 light|-|years? What if it is utterly unique: the only
planet with animals in this galaxy or even in the visible Universe, a bastion of
animals amid a sea of microbe|-|infested worlds? If that is the case, how much
greater the loss the Universe sustains for each species of animals or planet
driven to extinction trough the \remark {careless} {like more and more paper}
stewardship of Homo Sapiens? \crlf Welcome \remark {aboard} {but let's move on
with care}.

\StopIdea

\stopdocument
