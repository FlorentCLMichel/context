
\usemodule[int-load]
\def\loadsetups{}
\setupinteraction[state=start]
\setupcolors[state=start]

\usemodule[bib,set-11,mod-01]

\setuppublications[alternative=num]

\startXMLmapping[zero]
\processXMLfilegrouped{t-bib.xml}
\stopXMLmapping

\setupitemize[each][packed]

\setuphead[section][page=]

\setupoutput[pdftex]

\def\BIBTEX{Bib\TeX}
\def\MAPS{Maps}


\startbuffer[bibexample]
\startpublication[k=me,
                  t=manual,
                  a=Hoekwater,
                  y=2006,
                  s=TH2006,
                  n=1,
                  u=http://contextgarden.net/Bibliography]
\author{Taco}[T.]{}{Hoekwater}
\title{\CONTEXT\ Publication Module, The user documententation}
\pubyear{2006}
\note{In case you didn't know: it's the document you are reading now}
\pages{14}
\stoppublication
\stopbuffer

\getbuffer[bibexample]

\startmodule[type=tex]

\startdocumentation

\module
  [       file=bibmod-doc,
       version=2006.03.13,
         title=Module Documentation,
      subtitle=Bibliographies,
        author={Taco Hoekwater},
          date=\currentdate,
     copyright=Taco Hoekwater]

\completecontent

\section{Introduction}

The bibliographic module (\type{t-bib.tex}) takes care of references
to publications and the typesetting of publication lists, as well as
providing an interface between \BIBTEX and \CONTEXT. This manual
documents version 2006.08.08.

The bibliographic subsystem consists of the main module
\type{t-bib.tex}; four \BIBTEX\ styles (\type{cont-xx.bst}); and a set
of example configuration files (\type{bibl-xxx.tex}) that set up
specific formatting styles for both the citations and the list of
references.


\subsection{General overview}

A typical input file obeys following structure:
\startitemize[n]
\item A call to \type{\usemodule[bib]}.
\item Optionally, a few setup commands for the bibliographic module.
\item A number of definitions of publications to be referenced in the
main text of the article. The source of these definitions can be
a combination of:
    \startitemize
    \item The \type{\jobname.bbl} file (automatically read at \type{\starttext})
    \item extra bbl files
    \item a file or inline macros before \type{\starttext}
    \stopitemize
    These possibilities will be explained below. For now, it is
    only important to realize that of all these definitions have to be known
    {\it before} the first citation in the text.
\item \type{\starttext}
\item The body text, with a number of \type{\cite} and \type{\nocite} commands.
\item The list of publications, called using the command
    \type{\placepublications} or the command\break \type{\completepublications}.
\item \type{\stoptext}
\stopitemize

\section{Setup commands}

Bibliographic references use a specific `style', a collection of rules
for the use of \type{\cite} as well as for the formatting that is
applied to the publication list. The \CONTEXT\ bibliographic module
expects you to define all of these style options in one single
file of which the names starts with the prefix \type{bibl-}. 

Unlike the normal situation in \LATEX, this style {\it also\/}
includes the formatting of the items themselves. Because of this, the
\type{.bbl} file is set up as a database of entries with fields.

\subsection{Global settings: \type{\setuppublications}}

The most important user-level command is
\type{\setuppublications}. Most of the options to this command
shoudl be set by the bibliography style file, but a few of them are of
immediate interest to the casual user as well. 

Like all setup commands, thus command should be given before
\type{\starttext}, as it sets up global information about the
bibliographic references used in the document. \CONTEXT\ needs this
information in order to function correctly.

\setup{setuppublications}

\starttabulate[|l|p|]
\NC alternative\NC This gives the name of a bibliography style. \crlf
    The chosen style defines the other default options, the options
    given in this documentation are the defaults as they are set up 
    by the `apa' style. When this argument is given, the newly set
style is read in first, before the other options are processed. Thus
allowing you to override specific settings from the chosen style.\NC\NR
\NC refcommand    \NC the default option for \type{\cite}\NC \NR 
\NC sorttype\NC How the publications in the final publication
     list should be sorted. `cite' means: by the order in which
     they were first cited in your text. `bbl' tells the
     module to keep the relative ordering in which the publication 
     definitions were found\crlf
     The current default for apa is `cite'\NC\NR 
\NC criterium\NC Whether to list only the referenced
     publications or all of them.\crlf
     If this value is `all', then if `sorttype' equals `cite', this 
     means that all referred-to publications are listed 
     before all others, otherwise (if `sorttype' equals `bbl') you will
     just get a typeset version of the used database(s).\crlf
     The default for apa is `used'.\NC\NR
\NC numbering\NC Whether or not the  publication list
     should be labelled and if so, how. \type{yes} uses the item number in
     the publication list as label. \type{short} uses the short
     label. \type{bib}
     uses the original number in the \BIBTEX\ database as a label. 
     Anything else turns labelling off.\crlf
     The default for apa is `no'\NC\NR 
\NC autohang\NC Whether or not the
     hanging indent should be re-calculated based on the real size of the
     label. This option only applies if numbering is turned on.\crlf
     The default is `no'.\NC\NR  
\NC monthconversion\NC The presentation form of any month field, if it
     is entered in the database as a numeric value. The default is to
     typeset the number without any conversion\NC\NR
\stoptabulate

\subsection{How the entries are formatted: \type{\setuppublicationlist}}

\setup{setuppublicationlist}

The list of publications at the end of the article is comparable with
a sequence of normal \CONTEXT\ `list items' that behaves much like the
list that defines the table of contents. {\it In previous versions, it was 
in fact implemented as a `normal' \CONTEXT\ list, but this is no
longer the true.\/}

The module defines a set of extra options. These option names are static, they
do {\it not} change to follow the selected \CONTEXT\ interface language.

The first two options provide default widths for `autohang':
\starttabulate[|l|p|]
\NC totalnumber\NC The total number of items in the following list (used for autohang).\NC\NR
\NC samplesize\NC The longest short label in the list (used for autohang)\NC\NR
\stoptabulate

A third option can be used to overrule the use of \type{\title} as
heading for \type{\completepublications}

\starttabulate[|l|p|]
\NC title\NC The sectioning command.\NC\NR
\stoptabulate

The other extra options are needed to control micro||typesetting
of things that are buried deep within macros. There is a separate
command to handle the larger layout options
(\type{\setuppublicationlayout}, explained below), but the options
here are the only way to make changes in the formatting used for
editors', authors', and article authors' names.
\starttabulate[|l|p|]
\NC author    \NC command to typeset one author  in the publication list.\NC \NR 
\NC artauthor \NC command to typeset one article author  in the publication list.\NC \NR 
\NC editor    \NC command to typeset one editor  in the publication list.\NC \NR 
\NC namesep   \NC the separation between consecutive names (either
                  editors, authors or artauthors).\NC \NR 
\NC lastnamesep   \NC the separation before the last name in a list of names.\NC \NR 
\NC firstnamesep \NC the separation following the fistname or inits 
                      within a name in the publication list.\NC \NR 
\NC juniorsep     \NC likewise for `junior'.\NC \NR 
\NC vonsep        \NC likewise for `von'.\NC \NR 
\NC surnamesep     \NC likewise for surname.\NC \NR 
\NC authoretallimit \NC Number of authors needed to trigger `et al.' handling.\NC \NR 
\NC authoretaltext \NC Text to show at the end of an abbreviated list.\NC \NR 
\NC authoretaldisplay \NC Number of authors to actually display in an abbreviated list.\NC \NR 
\NC artauthoretallimit \NC Number of authors needed to trigger `et al.' handling.\NC \NR 
\NC artauthoretaltext \NC Text to show at the end of an abbreviated list.\NC \NR 
\NC artauthoretaldisplay \NC Number of authors to actually display in an abbreviated list.\NC \NR 
\NC editoretallimit \NC Number of editors needed to trigger `et al.' handling.\NC \NR 
\NC editoretaltext    \NC Text to show at the end of an abbreviated list.\NC \NR 
\NC editoretaldisplay \NC Number of editors to actually display in an abbreviated list.\NC \NR 
\NC authorcommand \NC A three-argument macro to typeset the list of authors.\NC\NR
\NC artauthorcommand \NC A three-argument macro to typeset the list of authors.\NC\NR
\NC editorcommand \NC A three-argument macro to typeset the list of authors.\NC \NR
\stoptabulate

The commands after `author' e.g. are predefined
macros that control how a single name is typeset. The four supplied
macros provide formatting that looks like this:

{\setvalue{@@currentalternative}{data}
\starttabulate[|l|p|]
\NC\tex{invertedauthor}\NC         \invertedauthor{Taco}{von}{Hoekwater}{T}{jr}\NC\NR
\NC\tex{invertedshortauthor}\NC    \invertedshortauthor{Taco}{von}{Hoekwater}{T}{jr}\NC\NR
\NC\tex{normalauthor}\NC           \normalauthor{Taco}{von}{Hoekwater}{T}{jr}\NC\NR
\NC\tex{normalshortauthor}\NC      \normalshortauthor{Taco}{von}{Hoekwater}{T}{jr}\NC\NR
\stoptabulate
}
As you can see in the examples, there is a connection between certain
styles of displaying a name and the punctuation used. Punctuation in
this document has been set up by the `apa' style, and that style makes
sure that \type{\invertedshortauthor} looks good, since that is the default
command for `apa' style. (Keep in mind that the comma at the end of the
author will be inserted by either `namesep' or `lastnamesep'.)

In case you are not happy with the predefined macros; it is quite simple to
define one of these macros yourself, it is a simple macro with 5
arguments: firstnames, von-part, surname, inits, junior.

For example, here is the definition of \type{\normalauthor},
\starttyping
\def\normalauthor#1#2#3#4#5%
  {\bibdoif{#1}{#1\bibalternative{firstnamesep}}%
   \bibdoif{#2}{#2\bibalternative{vonsep}}%
   #3%
   \bibdoif{#5}{\bibalternative{surnamesep}#5\unskip}}
\stoptyping
but commands can be a lot simpler, like this:
\starttyping
\def\surnameonly#1#2#3#4#5{#3}
\setuppublicationlist[editor=\surnameonly]
\stoptyping

The three-argument macro after `authorcommand' etc. can be used to
overrule the typesetting of the list of authors (normally done by the
internal macro \type{\dospecialbibinsert}). This is mostly a hook for
duplicated author lists in the publication list, that can be handled
like so:

\starttyping
\def\oldlist{}
\def\AbbreviateAuthors#1#2#3%
   {\xdef\newlist{#3}%
    \ifx\oldlist\newlist   \hbox to 2em{\hss---\hss}%
    \else                  \dospecialbibinsert{#1}{#2}{#3}\fi
    \global\let\oldlist\newlist }

\setuppublicationlist
	[artauthorcommand=\AbbreviateAuthors]
\stoptyping
The first argument is a list type `author', `artauthor', or `editor',
the second argument is the number of items that should be typeset,
and the third argument is a macro containing the commalist of persons,
in a form suitable for \tex{invertedauthor} and friends.


The following options are initialized depending on the
global settings for `numbering' and `autohang':
\starttabulate[|l|p|]
\NC width\NC Set to the calculated width of the largest label, but only if autohang is `yes'\NC\NR
\NC distance\NC Set to 0pt, but only if autohang is `yes'\NC\NR
\NC numbercommand\NC A command given in `setuppublications' if numbering is turned on, otherwise empty.\NC\NR
\NC textcommand\NC Set to a macro that outdents the body text if numbering is turned off, otherwise empty\NC\NR
\stoptabulate


\subsection{Setting citation options: \type{\setupcite}}

The \type{\cite} command has a lot of alternatives, as could be seen
above in the setting of `refcommand'. And these alternatives have
their own options:

\setup{setupcite}

\starttabulate[|l|p|]
\NC andtext  \NC separation between two authors (for \type{\cite[author]} styles)\NC \NR 
\NC otherstext  \NC text used for `et.al.' (for \type{\cite[author]} styles)\NC \NR 
\NC pubsep       \NC separator between publication references in a 
                  \type{\cite} command.\NC \NR 
\NC lastpubsep   \NC same, but for the 
                    last publication in the list.\NC \NR 
\NC left  \NC left side of a \type{\cite} (like \type{[})\NC \NR 
\NC inbetween \NC the separator between parts of a single citation.\NC\NR
\NC right     \NC right side of a \type{\cite} (like \type{]})\NC \NR 
\NC compress \NC Whether \type{\cite} should try to
compress it's argument list. \NC\NR
\stoptabulate
Not all options apply to all types of \type{\cite} commands.
E.g. `compress'  does not apply to the citation
list for all options of \type{\cite}, since sometimes compression does
not make sense or is not possible. The `num' version compresses
into a condensed sorted list, and the various `author' styles try
to compress all publications by one author, but e.g. years are
never compressed.

Likewise, `inbetween' only applies to three types: `authoryear' (a
space), `authoryears' (a comma followed by a space), and `num' (where
it is `--' (an endash), the character used to separate number ranges).

\subsection{Setting up \BIBTEX: \type{\setupbibtex}}

\BIBTEX\ bibliographic databases are converted into \type{.bbl} files,
and the generated file is just a more \TEX-minded representation of
the full database(s).

The four \type{.bst} files do not do any actual formatting on the
entries, and they do not subset the database either. Instead, the
{\it entire} database is converted into \TEX-parseable records. About the
only thing the \type{.bst} files do is sorting the entries (and
\BIBTEX\ itself resolves any `STRING' specifications, of course).

The module will read the created \type{\jobname.bbl} file
and select the parts that are needed for the current article.

\setup{setupbibtex}

\starttabulate[|l|p|]
\NC database\NC List of bibtex database file names to be
     used. The module will write a very short \type{.aux} file instructing
     \BIBTEX\ to create a (possibly very large) \type{\jobname.bbl} file,
     that will be \type{\input} by the module (at \type{\starttext}).\NC\NR
\NC sort\NC How the publications in the
     \BIBTEX\ database file should be sorted.\crlf
     The default here is `no' (\type{cont-no.bst}), meaning no sorting at all. 
     `author' (\type{cont-au.bst}) sorts alphabetically on author and within that on year,
     `title' (\type{cont-ti.bst}) sorts alphabetically on title and then on author and
     year, and `short' (\type{cont-ab.bst}) sorts on the short key that is generated
     by \BIBTEX.\NC\NR 
\stoptabulate

Starting with version 2006.08.08, the module registers \BIBTEX\ as a
program to be run by texexec, so you no longer need to run \BIBTEX\ by
hand.

Still, you may want to create the \type{\jobname.bbl} yourself. The
\type{.bbl} syntax is explained below. There is no default
database of course, and you do not {\it have} to use one: it is
perfectly OK to just \type{\input} a file with the bibliographic
records, as long as it has the right input syntax. Or even to include
the definitions themselves in the preamble of your document.

\subsection{Borrowing publications: \type{\usepublications}}

It is also possible to instruct the module to use the bibliographic 
references belonging to another document. This is done by using the command
\type{\usepublications[files]}, where \type{files} is a list of other
\CONTEXT\ documents (without extension). 

\setup{usepublications}

To be precise, this command will use the \type{.bbl} and \type{.tuo}
files from the other document(s), and will therefore not work if these
files cannot be found (the \type{.tuo} file is needed to get correct
page references for \type{\cite[page]}).


\subsection{Legacy database support}

Old \BIBTEX\ databases tend to contain \LaTeX-specific commands and,
especially, command||definitions. To make it easier to handle these
databases, a support module that defines a simplified version of
\LaTeX's \type{\newcommand} is shipped alongside the bib module. 
You can load this support code by adding
\starttyping
\usemodule[bibltx]
\stoptyping
to your document preamble.

\section{Citations}

Citations are normally handled through the \type{\cite} command.

\type{\cite} has two basic appearances:

\subsection{Default and explicit citations}

\setup{cite}

The single-argument form executes the style-defined default citation
command. This is the preferred way of usage, since some styles might
use numeric citations while others might use a variation of the
(author,year) style. 

The two-argument form allows you to manually select the style you want.

\subsubsection{Citation types}

Following is the full list of recognized keywords for \type{\cite},
with a short explanation where the data comes from. Most of the
information that is usable within \type{\cite} comes from the argument
to \type{\startpublication}. This command is covered in detail below. 


All of these options are {\it valid} in all publication styles, since
\CONTEXT\ always has the needed information available. But not all of
these are {\it sensible} in a particular style: using numbered references if
the list of publications itself is not numbered is not a good idea, for
instance. Also, some of the keys are somewhat strange and only
provided for future extensions.

First, here are the simple ones:
\starttabulate[|l|l|p|]
\NC author\NC      \cite[author][me] \NC(from `a')\hfil\NC\NR
\NC doi\NC         \cite[doi][me]\NC (from `d')\hfil\NC\NR
\NC key\NC         \cite[key][me]\NC (from `k')\hfil\NC\NR
\NC serial\NC      \cite[serial][me]\NC (from `n')\hfil\NC\NR
\NC short\NC       \cite[short][me]\NC (from `s')\hfil\NC\NR
\NC type\NC        \cite[type][me]\NC (from `t')\hfil\NC\NR
\NC year\NC        \cite[year][me]\NC (from `y')\hfil\NC\NR
\NC url\NC         \cite[url][me]\NC (from `u')\hfil\NC\NR
\stoptabulate
Keep in mind that `n' is a database sequence number, and not
necesarily the same number that is used in the list of
publications. For instance, if `sorttype' is cite, the list will be
re-ordered, but the `n' value will remain the same. To get to the
number that is finally used, use
\starttabulate[|l|l|p|]
\NC num\NC         \cite[num][me]\NC (this is a reference to
                                        the sequence number used in the publication list)\hfil\NC\NR
\stoptabulate
If the list of publications is not numbered visually, there will still
be a number available.

Three of the options are combinations:
\starttabulate[|l|l|p|]
\NC authoryear\NC  \cite[authoryear][me]\NC(from `a' and `y')\hfil\NC\NR
\NC authoryears\NC \cite[authoryears][me]\NC(from `a' and `y')\hfil\NC\NR
\NC authornum\NC  \cite[authornum][me]\NC(from `a' and `num')\hfil\NC\NR
\NC data\NC        \vtop{\hsize .45\hsize \cite[data][me]}\NC The data content of the entry\hfil\NC\NR
\stoptabulate

And the last one is a page reference to the page where the 
the entry is typeset within the publication list.

\starttabulate[|l|l|p|]
\NC page\NC        \cite[page][me]\NC (a page reference)\hfil\NC\NR
\stoptabulate

\subsection{Citations with local setups}

\setup{citealt}

The arguments in this form are inherited from \type{\setupcite},
except for \type{extras}. The argument of `\type{extras}' is typeset 
at the end of the reference, but before a potential `\type{right}', so
it can be used for e.g. page or chapter specifiers.

\subsection{Invisible citations}

\setup{nocite}

This command registers the references in the argument list, but does
not generate typeset material. It can be used to force certain entries
from the database to appear in the typeset list of publications.

\section{Placing the publication list}

To typset the list of publications, use \type{\completepublications}
or \type{\placepublications} at the location in your text where you
want the publication list to appear. As is normal in \CONTEXT,  
\type{\placepublications} gives you a raw list, and
\type{\completepublications} a list with a title. 


The default for the publication list is to contain only the `locally'
referenced items, so if you want to use your own heading instead of
the default one, you most likely want to call
\type{\placepublications} with an explicit criterium, like so:
\starttyping
\placepublications[criterium=all]
\stoptyping

The module uses the following defaults for the generated head:
\starttyping
\setupheadtext[en][pubs=References]
\setupheadtext[nl][pubs=Literatuur]
\setupheadtext[de][pubs=Literatur]
\setupheadtext[it][pubs=Bibliografia]
\setupheadtext[sl][pubs=Literatura]
\setupheadtext[fr][pubs=Bibliographie]
\stoptyping
These (or new ones) can be redefined as needed.

\section{The bbl file}

A typical bbl file consists of one initial command 
(\type{\setuppublicationlist}) that sets some information
about the number of entries in the bbl file and the widths
of the labels for the list, and that command is followed by a number of
appearances of \type{\startpublication ... \stoppublication}

The full appearance version of \type{\cite} 
accepts a number of option keywords, and we saw earlier that
the argument of the \type{\startpublication} command
defines most of the things we can reference to. This section explains
the precise syntax for \type{\startpublication}.

Each single block defines one bibliographic entry. I apologise
for the use of single||letter keys, but these have the advantage of
being a)\quad short and b)\quad safe w.r.t. the multi-lingual interface.

\setup{startpublication}

Here is the full example that has been used throughout this document:

\typebuffer[bibexample]

\subsection{Defining a publication}

The list of commands that is allowed to appear between \type{\startpublication}
and \type{\stoppublication} is given below. 

Order within an entry is irrelevant, except for the relative ordering
within each of the three commands that might appear more than once:
\type{\artauthor}, \type{\author} and \type{\editor}.

Most of these are `normal' \BIBTEX\ field names (in lowercase), but
some are extra special, either because they come from non-standard
databases that I know of, or because the bst file has pre-processed
the contents of the field. 

\subsubsection{Complex fields}

The three fields that contain names are extra special, because they
have more than one argument. These are: \type{\artauthor},
\type{\author} and \type{\editor}.  These commands require three
arguments, and there can be two extra optional arguments as well.


\starttabulate[|l|l|p|]
\NC\tex{artauthor[]\{\}[]\{\}\{\}}\NC\tfx AUTHOR\NC For an author of any publication
    that appears within a larger publication, like an article that appears
    within a journal or as part of a proceedings. \NC\NR
\NC\tex{author[]\{\}[]\{\}\{\}}\NC\tfx AUTHOR\NC The author of a standalone
    publication, like a monograph.\NC\NR
\NC\tex{editor[]\{\}[]\{\}\{\}}\NC\tfx EDITOR\NC The editor of e.g.
    an edited volume.\NC\NR
\stoptabulate

The argument lists have this form:

\starttyping
\author[junior]{firstnames}[inits]{von}{surname}
\stoptyping

and the meanings are as follows:
\starttabulate[|l|p|]
\NC \type{junior} \NC a designation of lineage, only used if confusion is possible (due to family members having the same name).\NC\NR
\NC \type{firstnames}  \NC individual (given) name(s)\NC\NR
\NC \type{inits}  \NC abbreviated form(s) of \type{firstnames}.\NC\NR
\NC \type{von}    \NC any bits of the family name that are normally disregarded in sorting\NC\NR
\NC \type{surname}   \NC reaminder of the family (last) name\NC\NR
\stoptabulate


\subsubsection{Simple fields}

Rather a large list, this is caused by the desire to support as many
existing \BIBTEX\ databases as possible.  Please note that a few of
the fields have names that are not the same as in \BIBTEX, because a
1~on~1 mapping causes conflicts with predefined macro names in 
\CONTEXT.

\starttabulate[|l|p(2.5cm)|p|]
\NC\type{\abstract}\NC\tfx ABSTRACT\NC just text.\NC\NR
\NC\type{\annotate}\NC\tfx ANNOTATE \NC just text.\NC\NR
\NC\type{\arttitle}\NC\tfx TITLE\NC The title of a partial publication (one that has \type{\artauthor}s).\NC\NR
\NC\type{\assignee}\NC\tfx ASSIGNEE\NC Assigned person for a patent\NC\NR
\NC\type{\bibnumber}\NC\tfx NUMBER \NC \NC\NR
\NC\type{\bibtype}\NC\tfx TYPE \NC See the \BIBTEX\ 
    documentation for it's use. This is {\it not} related
    to the type of entry that is used for deciding on the
    layout.\NC\NR
\NC\type{\biburl}\NC\tfx URL \NC Location on the internet. \NC\NR
\NC\type{\chapter}\NC\tfx CHAPTER \NC the chapter number, if this entry is
referring to a smaller section of a publication. It might actually
be a part number or a (sub)section number. The field \type{\bibtype} (above)
differentiates between these.\NC\NR
\NC\type{\city}\NC\tfx CITY\NC city of publication.\NC\NR
\NC\type{\comment}\NC\tfx COMMENT\NC just text.\NC\NR
\NC\type{\country}\NC\tfx COUNTRY\NC country of publication.\NC\NR
\NC\type{\crossref}\NC\tfx CROSSREF\NC A cross-reference to another 
    bibliographic entry. It will insert a citation
    to that entry, forcing it to be typeset as well.\NC\NR
\NC\type{\day}\NC\tfx DAY \NC Date of publication (for a patent)\NC\NR
\NC\type{\dayfiled}\NC\tfx DAYFILED\NC Filing date for a patent\NC\NR
\NC\type{\doi}\NC\tfx DOI \NC Document Object Identifier\NC\NR
\NC\type{\edition}\NC\tfx EDITION\NC The edition.\NC\NR
\NC\type{\eprint}\NC\tfx EPRINT\NC E-print information\NC\NR
\NC\type{\howpublished}\NC\tfx HOWPUBLISHED\NC \NC\NR
\NC\type{\isbn}\NC\tfx ISNB\NC isbn number (for books)\NC\NR
\NC\type{\issn}\NC\tfx ISSN\NC issn number (for journals)\NC\NR
\NC\type{\issue}\NC\tfx ISSUE\NC issue number (for journals)\NC\NR
\NC\type{\journal}\NC\tfx JOURNAL \NC The journal's name.\NC\NR
\NC\type{\keyword}\NC\tfx KEYWORD \NC just text (for use in indices).\NC\NR
\NC\type{\keywords}\NC\tfx KEYWORDS \NC just text (for use in indices).\NC\NR
\NC\type{\lang}\NC\tfx LANGUAGE \NC The language of the
    current bibliographic record\NC\NR 
\NC\type{\month}\NC\tfx MONTH\NC Month of publication\NC\NR
\NC\type{\monthfiled}\NC\tfx MONTHFILED\NC Filing month for a patent\NC\NR
\NC\type{\names}\NC\tfx NAMES\NC just text (for use in indices).\NC\NR
\NC\type{\nationality}\NC\tfx NATIONALITY\NC Nationality information for a patent\NC\NR
\NC\type{\note}\NC\tfx NOTE \NC just text (this is the `standard' \BIBTEX\ commentary field).\NC\NR
\NC\type{\notes}\NC\tfx NOTES \NC just text.\NC\NR
\NC\type{\organization}\NC\tfx ORGANIZATION\NC Like institute, but for e.g. companies.\NC\NR
\NC\type{\pages}\NC\tfx PAGES\NC Either the number of pages, or the page range
         for a partial publication. The `t' key to startpublication
         will decide automatically what is meant.\NC\NR
\NC\type{\pubname}\NC\tfx INSTITUTION,\crlf PUBLISHER,\crlf SCHOOL\NC Publisher or institution name.\NC\NR
\NC\type{\pubyear}\NC\tfx YEAR \NC Year of publication. Within this command,
        the \BIBTEX\ bst files will sometimes insert the command
        \type{\maybeyear}, which is needed to make sure that 
        the bbl file stay flexible enough to allow all styles of
        formatting.\NC\NR
\NC\type{\revision}\NC\tfx REVISION \NC Release version\NC\NR
\NC\type{\series}\NC\tfx SERIES \NC Possible book series information.\NC\NR
\NC\type{\size}\NC\tfx SIZE \NC Size in KB of a PDF file (this came from
        the NTG \MAPS\ database)\NC\NR
\NC\type{\thekey}\NC\tfx KEY \NC See the \BIBTEX\ 
    documentation for it's use. This is {\it not} related to
    the key used for citing this entry.\NC\NR
\NC\type{\title}\NC\tfx TITLE,\crlf BOOKTITLE \NC The title of a book.\NC\NR
\NC\type{\volume}\NC\tfx VOLUME \NC Volume number for multi-part books or 
    journals.\NC\NR
\NC\type{\yearfiled}\NC\tfx YEARFILED\NC Filing year for a patent\NC\NR
\stoptabulate

When the \type{\lang} field's content is a full word instead of a
two||letter code, correct processing depends on an auxiliary command
\type{\setbiblanguage}, to be used like this:
\starttyping
\setbiblanguage{English}{en}
\stoptyping
The first argument is a literal \type{\lang} argument, the second
argument has to be a two||letter language abbreviation understood by
\CONTEXT. 

Adding in one of your own fields is reasonably simple:

\starttyping
\newbibfield[mycommand]
\stoptyping
This will define \type{\mycommand} for use within
a publication (plus \type{\bib@mycommand}, it's internal form) as
well as the command \type{\insertmycommand} that can be used
within \type{\setuppublicationlayout} to fetch the supplied
value (see below). 


\section{Defining a publication type layout}

Publication style files of course take care of setting defaults for the
commands as explained earlier, but the largest part of a such a
publication style is concerned with specifying layouts for various
types of publications.

The command that does the work is \type{\setuppublicationlayout}:

\setup{setuppublicationlayout}

The first argument that is a publication (\BIBTEX\ entry) type, and
all publications that have this type given as argument to the `t' key
of \type{\startpublication} will be typeset by executing the commands
that appear in the group following the command.

For example, here is a possible way to typeset an article: from \type{bibl-apa}:
\starttyping
\setuppublicationlayout[article]{%
   \insertartauthors{}{ }{\insertthekey{}{ }{}}%
   \insertpubyear{(}{). }{\unskip.}%
   \insertarttitle{\bgroup }{\egroup. }{}%
   \insertjournal{\bgroup \it}{\egroup}
    {\insertcrossref{In }{}{}}%
   \insertvolume
    {, }
    {\insertissue{(}{)}{}\insertpages{:}{.}{.}}
    {\insertpages{, pages }{.}{.}}%
   \insertnote{ }{.}{}%
   \insertcomment{}{.}{}%
}
\stoptyping
For every command in the long list given in the previous paragraph, there is 
a corresponding \type{\insertxxx} command. (As usual, \type{\author}
etc. are special: they have a macro called \type{\insertxxxs}
instead). All of these \type{\insertxxx} macros use the same logic:

\starttyping
\insertartauthors{<before>}{<after>}{<not found>}
\stoptyping

Sounds easy? It is! But it is also often tedious: database entries can
be tricky things: some without issue numbers, others without page
numbers, some even without authors. So, you often need to nest rather
a lot of commands in the \type{<not found>} section of the `upper'
command, and \type{\unskip} and \type{\ignorespaces} are good friends
as well.

Incidentally, the distributed \type{bibl-xxx} files define layouts for
the `standard' publication types that are defined in the example
bibliography that comes with \BIBTEX. The list of possbile types is in
no way limited to that list, but it provides a reasonable starting
point.

\section{References}

\placepublications[criterium=all]

\stopdocumentation

\stopmodule

\stoptext