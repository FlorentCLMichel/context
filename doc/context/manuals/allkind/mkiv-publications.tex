% language=en % uk - Is behaviour really better than behavior, defence than defense?

% NOTE: I have purposely not reformated the source text linebreaks in order to facilitate
% comparaison of my editing using diff & co.

% \usebtxdataset[alan-1.bib]
% \setupbtxrendering[standard][repeat=yes,continue=yes,method=global]
%
% \starttext
%
% \startbodymatter
%     \startchapter[title=First chapter]
%         cite: \cite[Hagen]
%         \blank
%         \placelistofpublications[standard][criterium=chapter]
%     \stopchapter
%     \startchapter[title=Second chapter]
%         cite: \cite[Scarso]
%         \blank
%         cite: \cite[Hagen]
%         \blank
%         \placelistofpublications[standard][criterium=chapter]
%     \stopchapter
% \stopbodymatter
%
% \startbackmatter
%     \startchapter[title=Bibliography,number=no]
%         \placelistofpublications[standard][criterium=all]
%     \stopchapter
% \stopbackmatter
%
% \stoptext

% \usebtxdataset
%  [example]
%  [t:/manuals/publications-mkiv/mkiv-publications.bib]
%
% \definebtxrendering
%   [example]
%   [dataset=example]
%
% \starttext
%     \showbtxdatasetfields[example]
%     some text
%    %\placebtxrendering[example][method=dataset]
%     \placebtxrendering[example][method=dataset,sorttype={{detail:author,detail:editor},{entry:year,9998},entry:journal,entry:title,entry:pages}]
%    %\placebtxrendering[example][method=dataset,sorttype={entry:title,{entry:year,3000},{detail:author,detail:editor}}]
% \stoptext

% do etallimit etaldisplay

\enablemode[export]

% criterium: all + sorttype=cite    => citex before rest
% criterium: all + sorttype=dataset => dataset order
% criterium: used
% numbering: label, short, indexinlist, indexused

% \enabletrackers[publications*]

% \enabletrackers[publications.cite.match]

\dontcomplain

\setupbtxlistvariant [interaction=start]
\setupbtxcitevariant [interaction=start]

\usemodule[abr-02]
\usemodule[set-11]

\startmode[export]

    \setupbackend
      [export=yes,
       xhtml=yes,
       css=export-example.css]

    \setupexport
      [hyphen=yes,
       width=60em]

\stopmode

\startdocument
    [title={\BIBTEX},
     subtitle={The \CONTEXT\ way},
     author={Hans Hagen and Alan Braslau}]

% We need some (\expandafter,\unexpanded,...?) \TEX\ magic here (or above?):
\setupinteraction
  [title={\getvariable{document}{title}},
   subtitle={\getvariable{document}{subtitle}},
   author={\getvariable{document}{author}}]

\loadsetups[publications-en.xml] \enablemode[interface:setup:defaults]

% \input publ-tmp.mkiv

\setupbodyfont
  [dejavu,10pt]

\setuphead
  [chapter]
  [header=high,
   style=\bfc,
   color=darkmagenta]

\setuplayout
  [topspace=2cm,
   bottomspace=1cm,
   header=0cm,
   width=middle,
   height=middle]

\setupwhitespace
  [big]

\setuptyping
  [color=darkmagenta]

\setuptyping
  [keeptogether=yes]

\setuptype
  [color=darkmagenta]%darkcyan]
% How to get: option=TEX ?

\setupfootertexts
  [pagenumber]

\setupMPgraphics
  [mpy=\jobname.mpy]

\setupinteraction
  [state=start,
   color=darkcyan,
   contrastcolor=darkyellow]

\setupalign
  [verytolerant]

% The details of the following are a question of style
% to be used for notes or asides (but more important than footnotes)
\setupframedtexts
  [width=\textwidth,
   background=color,backgroundcolor=lightgray,
   % (may lead to some confusion with \showsetups?)
   offset=0.5cm,
   %before={\blank[small]},
   %after={\blank[small]},
   frame=off,
  ]
\automigrateinserts % needed to handle footnotes within a box...
% BUT, footnotes within framedtext still disappear! ?

\setupnote [footnote] [next={ }] % Why should this be necessary?

% Since this is a manual about bibliographies, let us use citations...
\startbuffer [bibliography]
@Book{Ierusalimschy2006,
    author   ={Ierusalimschy, R.},
    title    ={Programming in Lua},
    year     ={2006},
    publisher={Lua.org},
    isbn     ={8590379817},
    url      ={http://www.lua.org/pil/contents.html},
}
@Book{APA2010,
    title    ={Publication Manual of the American Psychological Association},
    year     ={2010},
    edition  ={Sixth},
    address  ={Washington, DC},
    publisher={American Psychological Association},
    note     ={291 pages},
    url      ={http://www.apa.org/books/},
}
\stopbuffer
\usebtxdataset [standard] [bibliography.buffer]

\startMPpage

    StartPage ;

        % input "mkiv-publications.mpy" ;

        picture pic ; pic := image (
            path pth ; pth := ((0,0) for i=1 step 2 until 20 : -- (i,1) -- (i+1,0) endfor) ;
            for i=0 upto 9 : draw pth shifted (0,2*i) ; endfor ;
        ) ;

        picture btx ; btx := textext("\ssbf\WORDS{\getvariable{document}{title}}") ;
        picture ctx ; ctx := textext("\ssbf\WORDS{\getvariable{document}{subtitle}}") ;
        picture dtx ; dtx := textext("\ssbf       \getvariable{document}{author}") ;
        % why do you use outline fonts here? (and why does it not work for me?)
        %picture btx ; btx := image(graphictext("\ssbf BIBTEX") withfillcolor white) ;
        %picture ctx ; ctx := image(graphictext("\ssbf THE CONTEXT WAY") withfillcolor white) ;

        pic := pic shifted - llcorner pic ;
        btx := btx shifted - llcorner btx ;
        ctx := ctx shifted - llcorner ctx ;
        dtx := dtx shifted - llcorner dtx ;

        pic := pic xysized (PaperWidth,PaperHeight) ;
        btx := btx xsized (2PaperWidth/3) shifted (.25PaperWidth,.225PaperHeight) ;
        ctx := ctx xsized (2PaperWidth/3) shifted (.25PaperWidth,.150PaperHeight) ;
        dtx := dtx xsized (2PaperWidth/3) shifted (.25PaperWidth,.075PaperHeight) ;

        fill Page withcolor \MPcolor{darkcyan} ;

        draw pic withcolor \MPcolor{darkmagenta} ;
        draw btx withcolor \MPcolor{lightgray} ;
        draw ctx withcolor \MPcolor{lightgray} ;
        draw dtx withcolor \MPcolor{lightgray} ;

        % draw boundingbox btx ;
        % draw boundingbox ctx ;
        % draw boundingbox dtx ;

    StopPage ;

\stopMPpage

\startfrontmatter

\starttitle[title=Contents]
    \setuplist[chapter][before=,after=]
    \placelist[chapter][color=black]
\stoptitle

\startchapter[title=Introduction]

This manual describes how \MKIV\ handles bibliographies. Support in \CONTEXT\
started in \MKII\ for \BIBTEX, using a module written by Taco Hoekwater. Later his
code was adapted to \MKIV, but because users demanded more, I decided that
reimplementing made more sense than patching. In particular, through the use of
\LUA, the \BIBTEX\ data files can be easily directly parsed, thus liberating
\CONTEXT\ from the dependency on an external \BIBTEX\ executable. The CritEd
project (by Thomas Schmitz, Alan Braslau, Luigi Scarso and myself) was a good
reason to undertake this rewrite. As part that project users were invited to come
up with ideas about extensions. Not all of them are (yet) honored, but the
rewrite makes more functionality possible.

This manual is dedicated to Taco Hoekwater who in a previous century implemented
the first \BIBTEX\ module and saw it morph into a \TEX||\LUA\ hybrid in this
century. The fact that there was support for bibliographies made it possible for
users to use \CONTEXT\ in an academic environment, dominated by bibliographic
databases encoded in the \BIBTEX\ format.

The subsystem described here is one of the most complex and messy of all \CONTEXT\
subsystems. This has to do with the fact that it combines (multiple) lists and
(multiple) forward and backward references, all kind of rendering of the citation
as well as the entry in the list, rather complex interactivity, multiple
databases, datasets and renderings and of course combinations of this. The
implementation uses a mix of \TEX\ and \LUA\ code with so called setups as
rendering specifications. At the cost of complexity (and some runtime penalty) this
provides a lot of freedom and flexibility.

\startlines
Hans Hagen
PRAGMA ADE
Hasselt NL
\stoplines

\stopchapter

\stopfrontmatter

\startbodymatter

\startchapter[title=The database]

The bibliography subsystem uses a database or a set of databases to construct
a list of citations to be used in a scholarly work. However, it will be seen
later that the database system can be used (and abused) to many ends having
little or nothing to do with citations and bibliographies. Nevertheless, we
will remain focused on the use of bibliography databases.

\startsection[title=\BIBTEX]

The \BIBTEX\ format is rather popular in the \TEX\ community and even with its
shortcomings it will stay around for a while. Many publication websites can
export and many tools are available to work with this database format. It is
rather simple and looks a bit like \LUA\ tables.%
\startfootnote
Indeed, it is said that the \BIBTEX\ format was one of the inspirations for the
constructor syntax in \LUA. \cite[extras={ Chapter 12.}][standard::Ierusalimschy2006]
\stopfootnote

Unfortunately the content can be
polluted with non|-|standardized \TEX\ commands which complicates pre- or
postprocessing outside \TEX. In that sense a \BIBTEX\ database is often not coded
neutrally. Some limitations, like the use of commands to encode accented
characters root in the \ASCII\ world and can be bypassed by using \UTF\ instead
(as handled somewhat in \LATEX\ through extensions such as \type {bibtex8}).

The normal way to deal with a bibliography is to refer to entries using a unique
tag or key. When a list of entries is typeset, this reference can be used for
linking purposes. The typeset list can be processed and sorted using the \type
{bibtex} program that converts the database into something more \TEX\ friendly (a
\type {.bbl} file). I never used the program myself (nor bibliographies) so I
will not go into too much detail here, if only because all I say can be wrong.

In \CONTEXT\ we no longer use the (external) \type {bibtex} program at all: we
simply parse the
database files and deal with the necessary manipulations directly in \CONTEXT.
One or more such databases can be used and combined with additional entries
defined within the document. We can have several such datasets active at the same
time.

A \BIBTEX\ file entry looks like this:

\starttyping
@Article{sometag,
    author  = "An Author and Another One",
    title   = "A hopefully meaningful title",
    journal = maps,
    volume  = "25",
    number  = "2",
    pages   = "5--9",
    month   = mar,
    year    = "2013",
    ISSN    = "1234-5678",
}
\stoptyping

Entries are of the form: \type{@category{…}}%
\startfootnote
Anything outside of a valid \type{@category{…}} construction is ignored and is
taken to be a comment. Within an entry, there are to be no comments but one can
prefix field names, for example, to have them ignored.

\quotation{There is a special entry type named \type {@comment{…}}. The main use of
such an entry type is to comment a large part of the bibliography easily,
since anything outside an entry is already a comment, and commenting out
one entry may be achieved by just removing its initial~\type {@}.} — This is perhaps
of some use, although not very elegant! As one can input multiple bibliography data
files, as will be seen below, it is much better practice to split datafiles for
optional loading.

Furthermore, many \BIBTEX\ data management tools such as \type {jabref} (see below) will
ignore and then throw-away all such handily-crafted comments and data entries turned
into comments. So one must beware!
\stopfootnote
The field names are all cast to lowercase so capitalization is irrelevant;
%\startfootnote
%But, whereas \BIBTEX\ is indifferent to the capitalization of the category names,
%we are a bit more picky here in order to gain performance in the loading of huge data
%files. This may be relaxed in the future.
%\stopfootnote
Spacing is not important and should be used advantageously for readability. The
leading tag (\type {sometag}) cannot contain spaces and must be followed by a comma.%
\startfootnote
The tag is, in fact, optional, as will be seen later.
\stopfootnote

Normally a value is given between quotes (or curly brackets) but single words are
also OK (as there is no real benefit in not using quotes or curly brackets, we advise to always use
them). The order of the fields in an entry is inconsequential and there can be many more fields than those shown above. Instead of string values one can also use
predefined shortcuts. The title for example might quite often contain \TEX\ macros.
Some fields, like \type {pages} have funny characters such as the endash
(typically entered as \type {--}) so we have a mixture of data and typesetting
directives. If you are covering non||english references, you often need
characters that are not in the \ASCII\ subset (but \CONTEXT\ is quite happy with
\UTF). If your database file uses old|-|fashioned \TEX\ accent combinations then these
will be internally converted automatically to \UTF. Commands (macros) found in a database file are
converted to an indirect call, which is quite robust.

\startframedtext
{\sl A note on strings:} the \type {author} (and \type {editor}) fields are parsed
separating multiple authors identified by the conjunction \quote{and}. Each name
is assumed to be in the form: Firstname(s) Lastname (where Lastname may include an
optional \quote{von}: lower-case word(s) such as \quotation{von}, \quotation{de},
\quotation{de la}, etc.) {\em unless} specifically in the form: Lastname(s),
Firstname(s) or Lastnames(s), Suffix(es), Firstname(s) (separated explicitly using
comma(s)).

Other string values such as \type {title} are kept literally (except for an
internal automatic conversion to \UTF\ of certain \TEX\ strings such as accent
combinations, endash, quotations, etc.). Note that the bibliography rendering
style (see below) might specify a capitalization (using the \CONTEXT\ command
\type {\Words}, for example). Capitalized Names and acronyms are respected,
removing a need for the \BIBTEX\ practice of \quote{protecting} such words or
letters with surrounding curly brackets (which are simply stripped off).%
\startfootnote
Since \CONTEXT\ uses \UTF, it suffers neither from all of the complicated
sorting issues that plague \BIBTEX/\LATEX.
\stopfootnote
{\red (WHERE DID THE FOOTNOTE GO?)}
As some styles might not specify the capitalization of words in the title whereas
other styles might, it is recommended that strings be written in lower case
except where upper case is explicitly required so as to be compatible with all
such capitalization styles.%
\startfootnote
Some bibliographic database sources can be quite sloppy and return strings
(titles and even authors) in all capitals, for example. We have made the
design choice {\em not} to follow the \BIBTEX\ practice/feature of managing
strings as we did not want to require the protection through enclosing curly
brackets that would have been a necessary consequence. Thus, some cleaning
of these database files might be needed.
\stopfootnote
{\red (WHERE DID THE FOOTNOTE GO?)}

String values can be enclosed indifferently between matching curly brackets:
\type {{}} or pairs of quotation marks: \type{""}. Multiple string values can
be concatenated using the operator \type {#}.%
\startfootnote
The \BIBTEX\ string concatenation operator \type {\#} is not to be confused
with the \LUA\ operator \type {..}, nor with the \METAPOST\ operator \type {\&}!
\stopfootnote
{\red (WHERE DID THIS FOOTNOTE GO?)}
\stopframedtext

Everything outside of a valid entry is ignored and treated as a comment.

The \BIBTEX\ files are loaded in memory as \LUA\ table but can be converted to
\XML\ so that we can access them in a more flexible way, but that is a subject
for specialists. % to be discussed later?

\stopsection

\startsection[title=\MKII\ definitions]

In the old \MKII\ setup we have two kinds of entries: the ones that come from the
\BIBTEX\ run and additional user supplied ones. We no longer rely on \BIBTEX\ output but we
do still support the user supplied definitions. These were in fact prepared in a
way that suits the processing of \BIBTEX\ generated entries. The next variant
reflects the \CONTEXT\ recoding of the old \BIBTEX\ output.

\starttyping
\startpublication[k=Hagen:Second,t=article,a={Hans Hagen},y=2013,s=HH01]
    \artauthor[]{Hans}[H.]{}{Hagen}
    \arttitle{Who knows more?}
    \journal{MyJournal}
    \pubyear{2013}
    \month{8}
    \volume{1}
    \issue{3}
    \issn{1234-5678}
    \pages{123--126}
\stoppublication
\stoptyping

The split \type {\artauthor} fields are collapsed into a single \type {author}
field as we handle the splitting later when it gets parsed in \LUA. The \type
{\artauthor} syntax is only kept around for backward compatibility with the
previous use of \BIBTEX.

In the new setup we support these variants:

\starttyping
\startpublication[k=Hagen:Third,t=article]
    \author{Hans Hagen}
    \title{Who knows who?}
    ...
\stoppublication
\stoptyping

as well as

\starttyping
\startpublication[tag=Hagen:Third,category=article]
    \author{Hans Hagen}
    \title{Who knows who?}
    ...
\stoppublication
\stoptyping

and

\starttyping
\startpublication
    \tag{Hagen:Third}
    \category{article}
    \author{Hans Hagen}
    \title{Who knows who?}
    ...
\stoppublication
\stoptyping

\stopsection

\startsection[title=\LUA\ tables]

Because internally the entries are \LUA\ tables, we also support the loading of \LUA\
based definitions:

\starttyping
return {
    ["Hagen:First"] = {
        author   = "Hans Hagen",
        category = "article",
        issn     = "1234-5678",
        issue    = "3",
        journal  = "MyJournal",
        month    = "8",
        pages    = "123--126",
        tag      = "Hagen:First",
        title    = "Who knows nothing?",
        volume   = "1",
        year     = "2013",
    },
}
\stoptyping

\stopsection

\startsection[title=\XML]

The following \XML\ input is rather
close in structure, and is also accepted as input.

\starttyping
<?xml version="2.0" standalone="yes" ?>
<bibtex>
    <entry tag="Hagen:First" category="article">
        <field name="author">Hans Hagen</field>
        <field name="category">article</field>
        <field name="issn">1234-5678</field>
        <field name="issue">3</field>
        <field name="journal">MyJournal</field>
        <field name="month">8</field>
        <field name="pages">123--126</field>
        <field name="tag">Hagen:First</field>
        <field name="title">Who knows nothing?</field>
        <field name="volume">1</field>
        <field name="year">2013</field>
    </entry>
</bibtex>
\stoptyping

\stopsection

\startsection[title=Other formats]

Many various bibliographic data file formats are available, such as:
\starttabulate [|lT|p|]
\NC savedrecs.txt     \NC Institute of Scientific Information (ISI) tagged format
                          (e.g. Thomson Reuters™ Web of Science™), \NC \NR
\NC filename.enw      \NC Thomson Reuters™ Endnote™ export format
                          (there is also an Endnote \type {.xml} export), \NC \NR
\NC filename.ris      \NC Research Information Systems, Incorporated, now
                          Thomson Reuters™ Reference Manager™, and \NC \NR
\NC pubmed_result.txt \NC The National Library of Medicine® (NLM®)
                          MEDLINE®/PubMed® data format \NC \NR
\stoptabulate
just to name a few. Filters can be easily written in \LUA\ to also read these
and other bibliography data formats. However, the user has a choice of
bibliography database management programs that can easily convert these
to the \BIBTEX\ format. Notable, open source examples are
\goto {jabref} [url(http://jabref.sourceforge.net)] and
\goto {zotero} [url(http://www.zotero.org)].
Indeed, it is not the vocation of the present \CONTEXT\ bibliography subsystem
to fully manage the bibliography data sources, only to be able to use such data
in the production of documents.

\stopsubsection

\stopsection

\startsection[title=Dataset coverage]

You can load much more data than you actually need. Only entries that are referred to
explicitly (through the \type {\cite} and \type {\nocite} commands) will be shown
in lists. We will cover these details later.

\stopsection

\stopchapter

\startchapter[title=Commands in entries]

One unfortunate aspect commonly found in \BIBTEX\ files is that they may
contain \TEX\ commands. Even worse is that there is no standard on what these
commands can be and what they mean, at least not formally, as \BIBTEX\ is a
program intended to be used with many variants of \TEX\ style: plain, \LATEX, and
others. This means that we need to define our use of these typesetting commands.%
\startfootnote
In particular, one might need to redefine those that are too \LATEX-centric.
\stopfootnote
However, in most cases, they are just abbreviations or font switches and these
are often well known. Therefore, \CONTEXT\ will try to resolve them before reporting
an issue. The log file will announce the commands that have been seen in the
loaded databases. For instance, loading \type {tugboat.bib}%
\startfootnote
\type {tugboat.bib} is distributed with \TeX{}Live.
\stopfootnote
gives a long list of
commands of which we show a small set of the five most frequently encountered ones here:
\startbuffer[tugboat]
%\definebtxdataset[tugboat]
\usebtxdataset[tugboat][tugboat.bib]
\stopbuffer
\getbuffer[tugboat]

\starttyping
publications > start used btx commands

publications    > tugboat              tt                     134 known
publications    > tugboat              Dash                   136 unknown
publications    > tugboat              acro                   137 known
publications    > tugboat              LaTeX                  209 known
publications    > tugboat              TeX                    856 known

publications > stop used btx commands
\stoptyping
%    publications > stop used btx commands
%-> "publications > stop used dataset commands"?

Some are flagged as known and others as unknown.
You can define unknown commands, or overload existing definitions in the
standard way ({\it e.g.} \type {\def\Dash{—}}) or, alternatively, in the
following way:

\startTEX
\definebtxcommand\TUB {TUGboat}
\definebtxcommand\MP  {METAPOST}
\definebtxcommand\sltt{\tt}
\definebtxcommand\<#1>{\type{#1}}
\stopTEX
\definebtxcommand\MP  {METAPOST} % to be used silently below

Custom commands created using \type {\definebtxcommand} have the advantage of
using a separate name space thus allowing isolation from other \CONTEXT\ commands.
Unknown commands do not stall processing, but their names are then typeset in a
mono|-|spaced font so they probably stand out for proofreading. You can
access the commands with \type {\btxcommand {...}}, as in:

\startbuffer
commands like \btxcommand{MySpecialCommand} are handled in an indirect way
\stopbuffer

\typebuffer[option=TEX]

As this is an undefined command we get: \quotation {\inlinebuffer}.

\blank

Finally, the \BIBTEX\ entry \type{@String{}} is preprocessed as expected.%
\startfootnote
Notice that \type {tugboat.bib} also contains:

\type {@Preamble{"\input tugboat.def"}}

\type {@Preamble{"\input path.sty"}}

These are silently ignored as many such commands are most likely not to be
compatible with \CONTEXT, indeed, these ones are not!
\stopfootnote

\starttyping
@String{j-TUGboat               = "TUGboat"}
\stoptyping

\stopchapter

\startchapter[title=Datasets]

Normally in a document you will use only one bibliographic database, whether or
not distributed over multiple files. Nevertheless we support multiple databases as well
which is why we talk of datasets instead. The use of multiple datasets allows the
isolation of different bibliographies (a single bibliography can nevertheless be
isolated by structure element: section, chapter, part, etc. as we shall see later).

A dataset is initiated with the \type
{\definebtxdataset} command.

\startTEX
\definebtxdataset[standard]
\stopTEX

Although currently a new dataset will be implicitly defined%
\startfootnote
\type {standard} is the predefined default database name.
\stopfootnote
it is best do this explicitly because in the future we may provide more options.
And like other commands in \CONTEXT, the dataset options can be setup using the
command \type{\setupbtxdataset}.

A dataset is loaded with the \type {\usebtxdataset} command.
Here are some examples:

\startTEX
\usebtxdataset[standard][tugboat.bib]
\usebtxdataset[standard][mtx-bibtex-output.xml]
\usebtxdataset[standard][test-001-btx-standard.lua]
\usebtxdataset[standard][named.buffer]
\stopTEX

These three suffixes are understood by the loader. Here the dataset has the name
\type {standard} and the three database files are merged, where later entries having the
same tag overload previous ones.%
\startfootnote
\red
Question: should entries having the same tag overload previous entries, or should
they rather be treated as if they had no tag? Are warnings produced in the log file?

Second question: are the suffixes important (or is the data format itself recognized)?
\stopfootnote
\startfootnote
The fourth example shows that a \type {named} buffer can also be employed to add dataset
entries. This may be useful for small additions or examples, but it is generally a good
idea (for convenience of management of data) to place them in files separate from the
document source code.
\stopfootnote
Definitions in the document source (coded in \TEX\
speak) are also added, and they are saved for successive runs. This means that if
you load and define entries, they will be known at a next run beforehand, so that
references to them are independent of when loading and definitions take place.
This is convenient to eventually break-up the dataset loading calls to relevant
sections of the document structure.

\showsetup{setupbtxdataset}

\showsetup{definebtxdataset}

\showsetup{usebtxdataset}

\startplacetable[reference=tab:test.bib,
                 title=The example database file {\type {test.bib}},
                 location=right]
    \small\small
    \startframed[align=text]%height=.4\textheight]
        %\typefile[file][range={5,33},before=,after=]{test.bib} % range??
        \executesystemcommand{head -n 33 test.bib | tail -n 29 > testhead.bib} % ugh!
        \typefile[file][before=,after=]{testhead.bib}
    \stopframed
\stopplacetable

In this document we use some example databases, so let's load one of them now:

\startbuffer
\definebtxdataset[example]

\usebtxdataset[example][test.bib]
\stopbuffer

\typebuffer[option=TEX] \getbuffer

The beginning of the file \type {test.bib} is shown in \in {table} [tab:test.bib].

(This bibliography database test file contains one entry of each type or category,
with the key set to the entry type name.)

You can set the current active dataset with

\startbuffer
\setbtxdataset[example]
\stopbuffer

\typebuffer[option=TEX] \getbuffer

but most publication|-|related commands accept optional arguments that denote the
dataset and references to entries can be prefixed with a dataset identifier.. More
about that later.

You can ask for an overview of entries present in a dataset with:

\startbuffer
\showbtxdatasetfields[example]
\stopbuffer

\typebuffer[option=TEX]

which gives:

\getbuffer

\page [yes]

The \quote{required} field names are colored in green and when one such field
is missing from the dataset entry this is indicated in red. Required means that
if this field is missing, one is probably not using the correct entry type or category.
Some fields might be mutually exclusive for some categories. For example, a
\type {@book} would not normally have both an \type {author} and an \type {editor}
or both a \type {volume} and a \type {number}. {\red Such cases are indicated in orange.}
Fields that normally would be ignored are colored in {\red xxxx}. The notions of
required, optional and ignored fields depends in fact on the bibliography rendering
style. Some, for example, might not use the \type {title} field whereas others
certainly will. More on this later.

Sometimes you might want to check a database, listing all of its entries. One way of doing that is the following:

\startbuffer
\showbtxdatasetcompleteness[example]
\stopbuffer

\typebuffer[option=TEX]

The completeness check colors the field names as indicated above. In addition,
in blue we show what gets inherited. The listing can be rather long.

\getbuffer

\startsection[title=Dataset fields]

Filling a dataset defines a certain number of entries, each having a type or category
and containing a number of fields. These fields (and entry types) can be anything going
far beyond the standard (predefined) ones used in \BIBTEX\ bibliography databases.
How these fields (and categories) will be used depends on the rendering style, discussed
later.

You can see all (currently known) categories and fields with:%
\startfootnote
\red
The required fields should get red {\ast}...
\stopfootnote

\starttyping
\showbtxfields[rotation=...]
\stoptyping

The result is shown \in {table} [tab:fields].

Just as a database can be much larger than needed for a document, the same is true
for the fields that make up an entry; not all entry fields will be necessarily used.
This idea will be developed in the next section describing the rendering of bibliography
lists.

\startplacetable[title={\type{\showbtxfields[rotation=90]}},reference=tab:fields]
    \showbtxfields[rotation=90]
\stopplacetable

\stopsection

\stopchapter

\startchapter[title=Renderings]

A list of publications can be rendered at any place in the document, and multiple
renderings can appear under certain limitations (according to scope).

If you want to see what publications are in the database, the easiest way is to
ask for a complete list:%
\startfootnote
\red
\type {sorttype=dataset} gives a \LUA\ error.
\stopfootnote

\startbuffer
\definebtxrendering
  [example]
  [dataset=example,
   %sorttype=dataset,
   method=dataset]

\placelistofpublications % \placebtxrendering
  [example]
  [criterium=all]
\stopbuffer

\typebuffer[option=TEX]

This gives:%
\startfootnote
\red
There are several issues here with respect to numbering:
As the default \type{alternative} is \type {apa}, to be coherent by default
\type {numbering} should be \type {off} and the citation style should be
\type {authoryears}. Furthermore, numbering should be local to a dataset
unless one explicitly asks for global numbering. This is not the case as one
can see here!

There is also a problem with the year suffixes. The present rendering
(numbered) should not get suffixes which only makes sense when using
\type {authoryear} variants. 
\stopfootnote
\startfootnote
\red
Why \type {\blank} before and after when, by default, the list is not packed?
(Defaults could be: \type {before=\blank,after=\blank,inbetween=\blank})
We should make this coherent with enumerations (and what about handling options
such as \type {joinedup}, \type {packed}?)
\stopfootnote

\blank \getbuffer \blank

Let's try another example (taken from the \goto {TUG bibliography archive}
[url(http://ftp.math.utah.edu/pub/tex/bib/index.html)]):

\startbuffer
\definebtxdataset[template]
\usebtxdataset[template][template.bib]

\definebtxrendering
  [template]
  [dataset=template,
   %?repeat=no,
   %?continue=no,
   method=dataset]

\placelistofpublications
  [template]
  [criterium=all]
\stopbuffer

\typebuffer[option=TEX]
\blank % Grr!
\getbuffer

Notice that the example above contains a reference to an unknown entry
(\type {Chen:SPE-19-9-897}) as well as an unknown command (\type {\path}).

\blank

The default rendering style is that described in \cite[title][standard::APA2010].
\cite[standard::APA2010] Often there is a prescribed style to render bibliographic
descriptions that can be programmed as a standard alternative, for example
(the default) \type {alternative=apa}. Other styles commonly used in the physical
sciences and in the biological sciences might be \type{aps} and \type{vancouver}.
More style schemes will be added in the future.
The rendering style usually also implies a particular bibliography list sorting
scheme as well as the use of a citation style. Indeed, the rendering of bibliography
lists and references to it are intimately coupled. This question will be explained
a bit later.

The rendering itself is somewhat complex to set up because we have not only many
different standards but also many fields that can be set up. This means that
there are several commands involved. A rendering is setup and defined with:

\showsetup[setupbtxrendering]

\showsetup[definebtxrendering]

And a list of such descriptions is generated with:%
\startfootnote
\red
This \type {\showsetup} is missing options?
\stopfootnote

\showsetup[placebtxrendering]

For example, the APA style specifies that the bibliography list be sorted by
author and then by year, then by title. The list is not numbered (and references
are made by author and year). This can be achieved as follows:%
\startfootnote
\red
The year suffixes are incorrect; I cannot figure out in what order they
are generated here!
\stopfootnote

\startbuffer
\definebtxrendering
  [altexample]
  [dataset=example,
   sorttype=authoryear,
   numbering=no,
   method=dataset]

\placebtxrendering
  [altexample]
  [criterium=all]
\stopbuffer

\typebuffer[option=TEX]

\blank \getbuffer \blank

So a dataset can by used with multiple renderings. The most obvious use of
this feature is the placement of bibliography lists localized by structure
elements: parts, chapters in a book, sections, etc. through the option
\type {criterium=chapter}, for example.

Only the default rendering definition (\type {alternative=apa}) is loaded
automatically. Other schemes are made available either by defining them
(as described in the following section) or by loading a predefined set of
definitions:

\startbuffer
\loadbtxdefinitionfile[aps]

\definebtxrendering
  [anotherexample]
  [dataset=example,
   method=dataset,
   alternative=aps]

% aps is presently broken :(
%\placebtxrendering
%  [anotherexample]
%  [criterium=all]
\stopbuffer

\typebuffer[option=TEX]

\blank \getbuffer \blank

The standard rendering alternatives have a small number of global options
(such as \type {numbering=yes}) that can be used to modify the output.
New and custom renderings can be written, to be described in detail in a
later chapter.

The dataset contains entries and each entry is assigned to a category.
It must be stressed that these \quote{categories} can be of any sort,
the meaning of which resides in the rendering style that is chosen. The
entries contain fields, and these too can be of any sort; their use also
depends on the rendering style and the category in which they belong.
\BIBTEX\ has conventionally defined a number of standard categories, each
making use of a number of fields considered either required, optional or
ignored. However, different traditional \BIBTEX\ rendering styles can make
inconsistant use of the standard categories and fields. To make matters
worse, different \type {.bib} database handling programs might use (and
impose) differing \quote{standards} as well.%
\startfootnote
For example, \type {jabref}, in addition to discarding all comments contained
in the database file, will convert all unrecognized, preciously named categories
to \type {@Other}!
\stopfootnote
This situation arises from the complexity of handling bibliographic data of
all sorts.

\startsection[title=Language]

Bibliography lists (and citations in the text, see below) are rendered
in the language of the document (\type {\mainlanguage}). However, a
bibliography entry can contain a \type {language=} field and this will
be used (if present) in the rendering and hyphenation of the \type {title},
for example.

\type{\setupbtxdataset[language=]} ...

\stopsection

\stopchapter

\startchapter[title=Citations]

The rendering of bibliographic lists was described in the previous chapter.
The APA Style Guide as well as good practice demand that {\em all}
bibliographical references be cited at least once in the text%
\startfootnote
Other publishing styles, textbooks in particular, might also include lists
of general references for \quote{further reading}.
\stopfootnote
(and, of course, all citations must have a corresponding bibliographical reference).

The examples of bibliography listing of the previous chapter were all fairly easy
since the entire bibliographical dataset was rendered. In practice, the same dataset
could be used over many documents and it might contain many more references than
are used in any one document. That is, the data source might be more complete than
the final rendered bibliography list or lists. The mechanism of citation allows you
to select references from the dataset(s) to build the list rendering as well as to
place a rendering of the reference in the text of the document to the corresponding
list rendering. These renderings can be of many forms.

A citation is normally pretty short as its main purpose is to refer uniquely to a more
detailed description. But, there are several ways to refer, which is why the citation
subsystem is configurable and extensible. Just look at the following commands:

\startbuffer
\cite[num][example::article]
\cite[year][example::article]
\cite[title][example::article]
\cite[author][example::article]
\cite[authoryear][example::article]
\stopbuffer

\typebuffer[option=TEX]
\startlines \getbuffer \stoplines

\startbuffer
\cite[num][example::article,book]
\cite[year][example::article,book]
\cite[title][example::article,book]
\cite[author][example::article,book]
\cite[authoryear][example::article,book]
\stopbuffer

\typebuffer[option=TEX]
\startlines \getbuffer \stoplines

\startbuffer
\cite[num][example::book,article]
\cite[year][example::book,article]
\cite[title][example::book,article]
\cite[author][example::book,article]
\cite[authoryear][example::book,article]
\stopbuffer

\typebuffer[option=TEX]
\startlines \getbuffer \stoplines

The citation rendering and the bibliographic list rendering are thus intimately
coupled reciprocally and cannot be dissociated. The coupling could be through the
reference number for example, but unnumbered reference lists also contain interacting
hyperlinks. A failure to take into account this interdependence can lead to fundamental
misunderstandings in use.%
\startfootnote
Both the citation and the list must be rendered. For example, omitting or
commenting-out the list rendering during the writing stage of a document will cause
the citations to fail to render properly.
\stopfootnote

\showsetup[cite]

The first argument is optional and if omitted, the default citation rendering
(\type {alternative=num}) will be selected.%
\startfootnote
The cite variant \type {num} refers to a counter that is usually sequential
in the order in which citations are called in the running text, with the rendered
bibliography list sorted in the same order. It is not to be confused with
\type {number} which corresponds to the bibliographic data field that is also
available (see below).
\stopfootnote
For example, to follow the APA specifications, one would change this default:

\startbuffer
\setupbtxcitevariant [alternative=authoryear]
\stopbuffer

\typebuffer[option=TEX]

Another common citation setup will enable interaction (disabled by default):

\startbuffer
\setupbtxcitevariant [interaction=start]
\stopbuffer

\typebuffer[option=TEX]

\showsetup[setupbtxcitevariant]

\startplacetable [reference=tab:citevariants,
    title=Predefined \type {\cite} variants.]
\startluacode
local variants = publications.tracers.citevariants

context.starttabulate { "|l|p|" }
    context.NC() context.bold("variant")
    context.NC() context.bold("rendering")
    context.NC() context.NR() context.FL()
    for i=1,#variants do
        local variant = variants[i]
        context.NC() context.type(variant)
        context.NC() context.citation( { variant }, { "example::article" })
        context.NC() context.NR()
    end
context.stoptabulate()
\stopluacode

{\red Why are the variants listed here hard-coded in lua when the list is
in fact dynamic? New variants can be defined, either based on known variants
or else flushing a field with the same name name (if found).}

Answer:

\startluacode
context.starttabulate { "|l|p|" }
     context.NC() context.bold("key")
     context.NC() context.bold("rendering")
     context.NC() context.NR() context.FL()
     for variant in table.sortedhash(publications.tracers.citevariants) do
         context.NC() context.type(variant)
         context.NC() context.citation( { variant }, { "example::demo-005" })
         context.NC() context.NR()
     end
context.stoptabulate()
\stopluacode

But, specific variants can have them overloaded:

\startTEX
% \showinstancevalues[setupbtxcitevariant][author]
% \showinstancevalues[setupbtxcitevariant][authornum]
\stopTEX

\startluacode
     for variant in table.sortedhash(publications.tracers.citevariants) do
         context.showinstancevalues( { "btxcitevariant" }, { variant })
     end
\stopluacode

\stopplacetable

A certain number of citation variants (or alternatives) have been predefined
(see \in{table} [tab:citevariants]).
The most commonly used are \type {num} and \type {authoryear} introduced above.
The first is typically used in conjunction with a numbered bibliography list,
usually sorted in the citation order in the text; The second is typically used
in conjunction with a bibliography list sorted by author and year of publication.
Other variants may be quite useful, even when used in conjunction with the above
standard schemes. One such example would be \type {\cite[title][key]} that can be
used to include the title of the work in the running text, another one is
\type {\cite[year][key]} that can be used to include the publication date or
\type {\cite[author][key]} that can be used to extract the authors' names. The
rendering of the variants can be somewhat adjusted through
\type {\setupbtxcitevariant}, for example modifying (or suppressing) parenthesis
or activating or deactivating interaction hyperlinks. 
Here we sort by author and color the citation:%
\startfootnote
Note that the hyperlinks activated through \type {interaction=start} are colored
according to their own scheme.
\stopfootnote
\startfootnote
\red
The sorting does not appear to work here!
\stopfootnote

\startbuffer
\setupbtxcitevariant[num]        [sorttype=author,color=darkyellow]
\setupbtxcitevariant[authoryear] [sorttype=author,color=darkyellow]

\citation[num]        [example::article,book]
\citation[authoryear] [example::article,book]
\stopbuffer

\typebuffer[option=TEX]

\startlines \getbuffer \stoplines

\setupbtxcitevariant[num]       [color=]
\setupbtxcitevariant[authoryear][color=]

New cite variants can also be defined, either based on an existing variant or
referring to a bibliography field of the same name (if found). For example, 
\startbuffer
\definebtxcitevariant
    [refnum]
    [num]
    [left={Ref.\nbsp},
    right=]
\cite[refnum][example::book]
\stopbuffer

\typebuffer [option=TEX]

will yield%
\startfootnote
\red
Why does this not work?? There is something that I do not understand!
\stopfootnote
\inlinebuffer.

\showsetup[definebtxcitevariant]

The details behind these cite variants will be described later.

The \type {\citation} command is synonymous to \type {\cite}.%
\startfootnote
Note that the \MKII\ module based on \type {bibtex} allowed the use of curly
brackets enclosing the key (for reasons of backward compatibility with traditional
\LATEX\ practice). As a consequence, this made it a bit picky about spaces between
its arguments, the first of which is optional. We have chosen to remove this
restriction through the use of standard \CONTEXT\ syntax using square brackets,
reserving curly brackets to be used to enclose text that is to be typeset.
\stopfootnote

Unless you use \type {criterium=all} {\red (with \type {method=dataset}?)} only publications that are explicitly cited will end up
in the lists. You can force a citation into a list using \type {\usecitation}, for
example:%
\startfootnote
\red
How does this differ from \type {\cite[none][example::patent]}?
\stopfootnote

\startTEX
\usecitation[example::patent]
\stopTEX

This command has two synonyms: \type {\nocite} and \type {\nocitation} so you can
choose whatever fits you best.

\showsetup[usecitation]

\showsetup[nocite]

\showsetup[nocitation]

\startsection[title=Additional text]

Sometimes one would like to include additional text a bibliography list entry,
for example a specific commentary or page number reference.

left, right, extra, before, after 

\startbuffer
\citation[reference=example::article,lefttext={This is an article:}][foo=bar]

\citation[reference=example::book,righttext={(p. xx)}][foo=bar]
\stopbuffer

\typebuffer[option=TEX]

\getbuffer

\startframed [align=text,framecolor=red,foreground=color,foregroundcolor=red]
This does not work correctly here as \type {example} is typeset as \type {method=dataset}.
There is also a question of syntax, since it will not work without including here
\type {[foo=bar]}. Also, which rendering does this affect (example and/or altexample)?

\blank

Note that the following does work:

\startbuffer
\citation[reference=tugboat::Hagen:TB32-1-9,
          lefttext={ConTeXt's evolution and milestones in its history: }]
         [foo=bar]

\definebtxrendering[tugboat][dataset=tugboat]
\placebtxrendering [tugboat][criterium=section]
\stopbuffer

\typebuffer[option=TEX]

\getbuffer

\stopframed

Note that the injection of left and/or right text only makes sense
when referring to a single list entry.

\stopsection

\startsection[title=Combining citations]

\startbuffer
\cite[example::article,book,booklet]
\stopbuffer

A single citation might refer to several sources, obtained through the use of a
comma separated list of keys, for example: \getbuffer

\typebuffer[option=TEX]

Notice that the comma separated list of three (or more) consecutive numbers gets
collapsed or compressed into a range of numbers.

Some bibliography styles admit the combination of several bibliographical sources
into a single list item having a unique reference number.%
\startfootnote
The combination of multiple bibliographic entries into as single bibliography list
item is more compact and this practice is often encountered in short \quote{letter}
type journal articles.
\stopfootnote
To achieve this, one combines keys using the addition operator symbol (+),
best illustrated though an example:

\startbuffer
\METAFUN\ began as an expression of love for \METAPOST. \citation
[tugboat::Berdnikov:TB21-2-129+Hobby:TB21-2-131]

\definebtxrendering[tugboat][dataset=tugboat]
\placebtxrendering [tugboat][criterium=section]
\stopbuffer

\typebuffer[option=TEX]

\getbuffer

\startframed [align=text,framecolor=red,foreground=color,foregroundcolor=red]
There are several issues here:
\startitemize[joinedup,packed]
\startitem What is with the color darkyellow? \stopitem
\startitem The separator (;) should suppress the period (.). \stopitem
\startitem Numbering? \stopitem
\startitem Automatic handling of tradition \TEX\ quotations (``''). \stopitem
\startitem The second entry should not appear (see below). \stopitem
\stopitemize
\stopframed

Combined entries are joined using a separator that can be specified, as in:

\startbuffer
\setupbtxrendering[tugboat][separator={ {\emdash} }]
\stopbuffer

\typebuffer[option=TEX]

or suppressed (using \type {separator=,}); By default \type {separator={; }}.

Beware that dataset entries that are combined cannot also appear apart
(nor does this really make any logical sense). The following:
\startbuffer
\cite[tugboat::Berdnikov:TB21-2-129]
\cite[tugboat::Berdnikov:TB21-2-129,Hobby:TB21-2-131]
\cite[tugboat::Hobby:TB21-2-131]
\stopbuffer
\typebuffer[option=TEX]
will yield.%
\startfootnote
\red
I would expect the system to handle this better. All instances of any key that
appears in a combined reference should refer to this combined reference. Here,
all three \type{\cite} should give a single (same) number.
\stopfootnote
\getbuffer
This is another instance showing that citations and the rendered bibliography
list interact and cannot be separated; There are indeed many others.


% test:

%\savebtxdataset [tugboat] [tugboat-export.bib] [criterium=section]

\stopsection

\startsection[title=Searching]

Finding the right key in a database can be a pain. On the other hand, asking for
a wildcard also makes no sense. Nevertheless we provide a mechanism for matching
a query. For this we load a small bibliographic database:

\startbuffer
\usebtxdataset[graph][mkiv-publications-graph.bib]
\stopbuffer

\typebuffer \getbuffer

We could switch to this base using:

\starttyping
\setbtxdataset[graph]
\stoptyping

but instead we will use a prefix. For instance, if we have this in our source:

\startbuffer
searching gives a few hits, so we get: \cite [ graph :: match (
author:cleveland and year:1993 ) ].
\stopbuffer

\typebuffer

We will get: \quotation {\inlinebuffer}. Of course this assumes that we also
typeset a list of referred to references, so let's do that:

\startbuffer
\definebtxrendering[graph][dataset=graph]
\placebtxrendering[graph][criterium=chapter]
\stopbuffer

\typebuffer

We get:

\blank \getbuffer \blank


Let's look in more detail at the \type {\cite} command. In order to distinguish
efficiently between a normal reference and a more clever one, we use the \type
{match} keyword:

\startbuffer
dataset::match(query)
dataset :: match ( query )
\stopbuffer

The handler is rather tolerant for spaces:

\startbuffer
dataset :: match ( query )
\stopbuffer

Which is handy if you have long queries that wrap around in the source code.. Of
course the \type {dataset::} prefix is optional in which case the current dataset
is taken.

A query eventually becomes a \LUA\ expression so you can use helpers to achieve
your goal. As a convenience there are some shortcuts to access fields. The
following examples demonstrate this:

\starttyping
match(author:hagen)
match(author:hagen and author:hoekwater and year:1990-2010)
match(author:"Bogusław Jackowski")
match(author:"Bogusław Jackowski" and (tonumber(field:year) or 0) > 2000)
\stoptyping

You can use quotes when spaces are involved. Of course you can use other
characters that the basic alphabet. Ranges (of numbers) are recognized. String
lookups are partial and case insensitive. \footnote {At the time of this
writing, may 2014, this mechanism is still somewhat experimental.}

\startbuffer
Wildcards: \cite [graph::match(author:cleve)].
\stopbuffer

\typebuffer

We get three entries: \quotation {\inlinebuffer}.

% To be checked: are indeed three entries found?

% Match : \cite [match(author:cleveland and year:1993)]      \par

\stopsection

\startsection[title=Placing a single citation]

Sometimes, one would like to place a single citation somewhere in the text
without necessarily adding it to a bibliography list. Take, for example,%
\startfootnote
The default rendering style is given by \placecitation[APA2010]

{\red How is this to be controlled?}
\stopfootnote
\startbuffer
\placecitation[tugboat::Mahajan:TB31-1-88]
\stopbuffer
\getbuffer
obtained by using
\typebuffer[option=TEX]

\stopsection

\stopchapter

\startchapter[title=Custom renderings]

The rendering of citations and bibliography lists is highly configurable and
custom rendering schemes can be added. The details can get quite complicated
so we will begin with a description of how citation variations can be used in
the running text, followed by a description on how to control the rending of
the associated bibliography list.

\startsection[title=Custom citation renderings]

\startframed [align=text,framecolor=red,foreground=color,foregroundcolor=red]
This section needs to be developed…
\stopframed

Because we are dealing with database input and because we generally need to
manipulate entries, much of the work is delegated to \LUA. This makes it easier
to maintain and extend the code. Of course \TEX\ still does the rendering. The
typographic details are controlled by parameters but not all are used in all
variants. As with most \CONTEXT\ commands, it starts out with a general setup
command \type {\setupbtxcitevariant}.
On top of that we can define instances that inherit either from a given parent or
from the topmost setup using \type {\definebtxcitevariant}.

But, specific variants can have them overloaded:

% \showinstancevalues[setupbtxcitevariant][author]
% \showinstancevalues[setupbtxcitevariant][authornum]

\startcolumns[n=2]
\startluacode
    local variants = publications.tracers.citevariants

    for i=1,#variants do
        context.showinstancevalues( { "btxcitevariant" }, { variants[i] })
    end
\stopluacode
\stopcolumns

A citation variant is defined in several steps and if you really want to know
the dirty details, you should look into the source code. Here
we stick to the concept.

\starttyping
\startsetups btx:cite:author
    \btxcitevariant{author}
\stopsetups
\stoptyping

You can overload such setups if needed, but that only makes sense when you cannot
configure the rendering with parameters. The \type {\btxcitevariant} command is
one of the build in accessors and it calls out to \LUA\ where more complex
manipulation takes place if needed. If no manipulation is known, the field with
the same name (if found) will be flushed. A command like \type {\btxcitevariant}
assumes that a dataset and specific tag has been set. This is normally done in
the wrapper macros, like \type {\cite}. For special purposes you can use these
commands

\starttyping
\setbtxdataset[example]
\setbtxentry[hh2013]
\stoptyping

But don't expect too much support for such low level rendering control.

\stopsection

\startsection[title=Custom list renderings]

\startframed [align=text,framecolor=red,foreground=color,foregroundcolor=red]
This entire section is complicated and a bit confusing between renderings
and lists; It needs to be rewritten.
\stopframed

A dataset can have all kind or categories of entries:
\startalignment[flushleft,verytolerant,nothyphenated]
\startluacode
    local categories = publications.tracers.categories

    for i=1,#categories do
        if i > 1 then
            context(", ")
        end
        context.type(categories[i]) -- needs to be cast to a string...
    end
\stopluacode
\stopalignment
each having its own rendering variant. To keep things simple we have their settings
separated. However, these settings are shared for all rendering alternatives.%
\startfootnote
In practice this is seldom a problem in a publication as only one rendering
alternative will be active. If this be not sufficient, you can always group local
settings in a setup and hook that into the specific rendering.
\stopfootnote

\showsetup[setupbtxlistvariant]
\showsetup[definebtxlistvariant]

Examples of list variants are:%
\startfootnote
\red
I do not understand the use of \type {\setupbtxlistvariant}!
\stopfootnote

\startluacode
    local variants = publications.tracers.listvariants

    for i=1,#variants do
        context.showinstancevalues( { "btxlistvariant" }, { variants[i] })
    end
\stopluacode

\startluacode
     for variant in table.sortedhash(publications.tracers.listvariants) do
         context.showinstancevalues( { "btxlistvariant" }, { variant })
     end
\stopluacode

The exact rendering of list entries is determined by the \type {alternative} key
and defaults to \type {apa} which uses definitions from \type
{publ-imp-apa.mkiv}. If you look at that file you will see that each category has
its own setup. You may also notice that additional tests are needed to make sure
that empty fields don't trigger separators and such.

% \showsetup{setuplists}

There are a couple of accessors and helpers to get the job done. When you want to
fetch a field from the current entry you use \type {\btxfield}. In most cases
you want to make sure this field has a value, for instance because you don't want
fences or punctuation that belongs to a field.

\starttyping
\btxdoif {title} {
    \bold{\btxfield{title}},
}
\stoptyping

There are three test macros:

\starttyping
\btxdoifelse{fieldname}{action when found}{action when not found}
\btxdoif    {fieldname}{action when found}
\btxdoifnot {fieldname}                   {action when not found}
\stoptyping

An extra conditional is available for testing interactivity:

\starttyping
\btxdoifelseinteraction{action when true}{action when false}
\stoptyping

In addition there is also a conditional \type {\btxinteractive} which is
more efficient, although in practice efficiency is not so important here.

There are three commands to flush data:

\starttabulate[|l|l|]
\NC \type {\btxfield}  \NC fetch a explicit field (e.g. \type {year}) \NC \NR
\NC \type {\btxdetail} \NC fetch a derived field (e.g. \type {short}) \NC \NR
\NC \type {\btxflush}  \NC fetch a derived or explicit field \NC \NR
\stoptabulate

Normally you can use \type {\btxfield} or \type {\btxflush} as derived fields
just like analyzed author fields are flushed in a special way. There is
experimental support for so called manipulators. You can for instance say this:

\starttyping
\btxflush{lowercase->title}
\stoptyping

A sequence of manipulators is applied to fetched field, where a sequence is one
or more manipulators:

\starttyping
\btxflush{stripperiod->uppercase->title}
\stoptyping

Some actions are recognized (built-in) but you can also use actions from other
namespaces, like in:

\starttyping
\btxflush{converters.Word -> title}
\stoptyping

Watch how we can use spaces around the \type {->} which is nicer for wrapped
around usage. Eventually, we might put some more function in the default
namespace.

You can improve readability by using setups, for instance:

\starttyping
\btxdoifelse {author} {
    \btxsetup{btx:apa:author:yes}
} {
    \btxsetup{btx:apa:author:nop}
}
\stoptyping

Keep in mind that normally you don't need to mess with definitions like this
because standard rendering styles are provided. These styles use a few helpers
that inject symbols but also take care of leading and trailing spaces:

\starttabulate[|||]
\NC \type {\btxspace   } \NC before \btxspace    after \NC \NR
\NC \type {\btxperiod  } \NC before \btxperiod   after \NC \NR
\NC \type {\btxcomma   } \NC before \btxcomma    after \NC \NR
\NC \type {\btxlparent } \NC before \btxlparent  after \NC \NR
\NC \type {\btxrparent } \NC before \btxrparent  after \NC \NR
\NC \type {\btxleftparenthesis}
\NC before \btxleftparenthesis  after \NC \NR
\NC \type {\btxrightparenthesis}
\NC before \btxrightparenthesis after \NC \NR
\NC \type {\btxrightparenthesiscomma}
\NC before \btxrightparenthesiscomma after \NC \NR
\NC \type {\btxrightparenthesisperiod}
\NC before \btxrightparenthesisperiod after \NC \NR
\NC \type {\btxrightparenthesiscomma}
\NC before \btxrightparenthesiscomma after \NC \NR
\NC \type {\btxrightparenthesisperiod}
\NC before \btxrightparenthesisperiod after \NC \NR
\NC \type {\btxleftbracket}
\NC before \btxleftbracket  after \NC \NR
\NC \type {\btxrightbracket}
\NC before \btxrightbracket after \NC \NR
\NC \type {\btxrightbracketcomma}
\NC before \btxrightbracketcomma after \NC \NR
\NC \type {\btxrightbracketperiod}
\NC before \btxrightbracketperiod after \NC \NR
\stoptabulate

So, the previous example setup can be rewritten as:

\starttyping
\btxdoif {title} {
    \bold{\btxfield{title}}
    \btxcomma
}
\stoptyping

There is a special command for rendering a (combination) of authors:

\starttyping
\btxflushauthor{author}
\btxflushauthor{editor}
\btxflushauthor[inverted]{editor}
\stoptyping

Instead of the last one you can also use:

\starttyping
\btxflushauthorinverted{editor}
\stoptyping

You can use a (configurable) default or pass directives: Valid directives are

\starttabulate
\NC \bf conversion       \NC \bf rendering        \NC \NR
\HL
\NC \type{inverted}      \NC the Frog jr, Kermit  \NC \NR
\NC \type{invertedshort} \NC the Frog jr, K       \NC \NR
\NC \type{normal}        \NC Kermit, the Frog, jr \NC \NR
\NC \type{normalshort}   \NC K, the Frog, jr      \NC \NR
\stoptabulate

The list itself is not a list in the sense of a regular \CONTEXT\ structure related
list. We do use the list mechanism to keep track of used entries but that is mostly
because we can then reuse filtering mechanisms. The actual rendering of a reference
and entry runs on top of so called constructions (other examples of constructions are
descriptions, enumerations and notes).

\showsetup[setupbtxlist]

You need to be aware what command is used to achieve the desired result. For instance,
in order to put parentheses around a number reference you say:

\starttyping
\setupbtxlistvariant
  [num]
  [left=(,
   right=)]
\stoptyping

If you want automated width calculations, the following does the trick:

\starttyping
\setupbtxrendering
  [standard]
  [width=auto]
\stoptyping

but if you want to control it yourself you say something:

\starttyping
\setupbtxrendering
  [width=none]

\setupbtxlist
  [standard]
  [width=3cm,
   distance=\emwidth,
   color=red,
   headcolor=blue,
   headalign=flushright]
\stoptyping

In most cases the defaults will work out fine.

Normally the references are numbered using one counter for the whole
document. If you want each list to have its own number, then you can
set the \type {continue} parameter:

\starttyping
\setupbtxrendering[continue=no]
\stoptyping

In a similar fashion you can influence if references are included only once
of in each list:

\starttyping
\setupbtxrendering[repeat=yes]
\stoptyping

\stopsection

\stopchapter

\startchapter[title=Exporting datasets]

\stopchapter

\startchapter[title=The \LUA\ view]

Because we manage data at the \LUA\ end it is tempting to access it there for
other purposes. This is fine as long as you keep in mind that aspects of the
implementation may change over time, although this is unlikely once the modules
become stable.

The entries are collected in datasets and each set has a unique name. In this
document we have the set named \type {example}. A dataset table has several
fields, and probably the one of most interest is the \type {luadata} field. Each
entry in this table describes a publication:

\startluacode
    context.tocontext(publications.datasets.example.luadata["demo-001"])
\stopluacode

This is \type {publications.datasets.example.luadata["demo-001"]}. There can be
a companion entry in the parallel \type {details} table.

\startluacode
    context.tocontext(publications.datasets.example.details["demo-001"])
\stopluacode

These details are accessed as \type
{publications.datasets.example.details["demo-001"]} and by using a separate table
we can overload fields in the original entry without losing the original.

You can loop over the entries using regular \LUA\ code combined with \MKIV\
helpers:

\startbuffer
local dataset = publications.datasets.example

context.starttabulate { "|l|l|l|" }
for tag, entry in table.sortedhash(dataset.luadata) do
    local detail = dataset.details[tag] or { }
    context.NC() context.type(tag)
    context.NC() context(detail.short)
    context.NC() context(entry.title)
    context.NC() context.NR()
end
context.stoptabulate()
\stopbuffer

\typebuffer

This results in:

\ctxluabuffer

You can manipulate a dataset after loading. Of course this assumes that you know
what kind of content you have and what you need for rendering. As example we
load a small dataset.

\startbuffer
\definebtxdataset[drumming]
\usebtxdataset[drumming][mkiv-publications.lua]
\stopbuffer

\typebuffer \getbuffer

Because we're going to do some \LUA, we could also have loaded the dataset
with:

\starttyping
publications.load("drumming","mkiv-publications.lua","lua")
\stoptyping

The dataset has three entries:

\typefile{mkiv-publications.lua}

As you can see, we can have a subtitle. We will combine the title and subtitle
into one:

\startbuffer
\startluacode
local luadata = publications.datasets.drumming.luadata

for tag, entry in next, luadata do
    if entry.subtitle then
        if entry.title then
            entry.title = entry.title .. ", " .. entry.subtitle
        else
            entry.title = entry.subtitle
        end
        entry.subtitle = nil
        logs.report("btx",
            "combining title and subtitle of entry tagged %a into %a",
            tag,entry.title)
    end
end
\stopluacode
\stopbuffer

\typebuffer \getbuffer

As a hash comes in a different order each run (something that demands a lot
of care in multipsass workflows that save data in between), we can use this
instead:

\starttyping
\startluacode
local ordered = publications.datasets.drumming.ordered

for i=1,#ordered do
    local entry = ordered[i]
    if entry.subtitle then
        if entry.title then
            entry.title = entry.title .. ", " .. entry.subtitle
        else
            entry.title = entry.subtitle
        end
        entry.subtitle = nil
        logs.report("btx",
            "combining title and subtitle of entry tagged %a into %a",
            entry.tag,entry.title)
    end
end
\stopluacode
\stoptyping

This loops processes in the order of definition. You can also sort by tag:

\starttyping
\startluacode
local luadata = publications.datasets.drumming.luadata

for tag, entry in table.sortedhash(luadata) do
    if entry.subtitle then
        if entry.title then
            entry.title = entry.title .. ", " .. entry.subtitle
        else
            entry.title = entry.subtitle
        end
        entry.subtitle = nil
        logs.report("btx",
            "combining title and subtitle of entry tagged %a into %a",
            entry.tag,entry.title)
    end
end
\stopluacode
\stoptyping

We can now typeset the entries with:

\startbuffer
\definebtxrendering[drumming][dataset=drumming,method=dataset]
\placebtxrendering[drumming]
\stopbuffer

\typebuffer

Because we just want to show the entries, and have no citations that force them
to be shown, we have to the \type {method} to \type {dataset}. \footnote {Gavin
Harrison is in my opinion one of the most creative, diverse and interesting
drummers of our time. It's also fascinating to watch him play and a welcome
distraction from writing code and manuals.}

\blank \getbuffer \blank

\stopchapter

\startchapter[title=The \XML\ view]

The \type {luadata} table can be converted into an \XML\ representation. This is
a follow up on earlier experiments with an \XML|-|only approach. I decided in the end
to stick to a \LUA\ approach and provide some simple \XML\ support in addition.

Once a dataset is accessible as \XML\ tree, you can use the regular \type {\xml...}
commands. We start with loading a dataset, in this case from just one file.

%\startbuffer
%\usebtxdataset[tugboat][tugboat.bib]
%\stopbuffer
%
%\typebuffer %\getbuffer

The dataset has to be converted to \XML:

\startbuffer
\convertbtxdatasettoxml[tugboat]
\stopbuffer

broken...%\typebuffer \getbuffer

The tree is now accessible by its root reference \type {btx:tugboat}. If we want simple
field access we can use a few setups:

\startbuffer
\startxmlsetups btx:initialize
    \xmlsetsetup{#1}{bibtex|entry|field}{btx:*}
    \xmlmain{#1}
\stopxmlsetups

\startxmlsetups btx:field
    \xmlflushcontext{#1}
\stopxmlsetups

\xmlsetup{btx:tugboat}{btx:initialize}
\stopbuffer

\typebuffer \getbuffer

The two setups are predefined in the core already, but you might want to change them. They are
applied in for instance:

\startbuffer
\starttabulate[|||]
    \NC \type {tag}   \NC \xmlfirst {btx:tugboat}
        {/bibtex/entry[string.find(@tag,'Hagen')]/attribute('tag')}
    \NC \NR
    \NC \type {title} \NC \xmlfirst {btx:tugboat}
        {/bibtex/entry[string.find(@tag,'Hagen')]/field[@name='title']}
    \NC \NR
\stoptabulate
\stopbuffer

\typebuffer \getbuffer

\startbuffer
\startxmlsetups btx:demo
    \xmlcommand
        {#1}
        {/bibtex/entry[string.find(@tag,'Hagen')][1]}{btx:table}
\stopxmlsetups

\startxmlsetups btx:table
\starttabulate[|||]
    \NC \type {tag}   \NC \xmlatt{#1}{tag} \NC \NR
    \NC \type {title} \NC \xmlfirst{#1}{/field[@name='title']} \NC \NR
\stoptabulate
\stopxmlsetups

\xmlsetup{btx:tugboat}{btx:demo}
\stopbuffer

\typebuffer \getbuffer

Here is another example:

\startbuffer
\startxmlsetups btx:row
    \NC \xmlatt{#1}{tag}
    \NC \xmlfirst{#1}{/field[@name='title']}
    \NC \NR
\stopxmlsetups

\startxmlsetups btx:demo
    \xmlfilter {#1} {
        /bibtex
        /entry[@category='article']
        /field[@name='author'
            and (find(text(),'Knuth') or find(text(),'DEK'))]
        /../command(btx:row)
    }
\stopxmlsetups

\starttabulate[|||]
    \xmlsetup{btx:tugboat}{btx:demo}
\stoptabulate
\stopbuffer

\typebuffer \getbuffer

A more extensive example is the following. Of course this assumes that you
know what \XML\ support mechanisms and macros are available.

\startbuffer
\startxmlsetups btx:getkeys
    \xmladdsortentry{btx}{#1}{\xmlfilter{#1}{/field[@name='author']/text()}}
    \xmladdsortentry{btx}{#1}{\xmlfilter{#1}{/field[@name='year'  ]/text()}}
    \xmladdsortentry{btx}{#1}{\xmlatt{#1}{tag}}
\stopxmlsetups

\startxmlsetups btx:sorter
    \xmlresetsorter{btx}
  % \xmlfilter{#1}{entry/command(btx:getkeys)}
    \xmlfilter{#1}{
        /bibtex
        /entry[@category='article']
        /field[@name='author' and find(text(),'Knuth')]
        /../command(btx:getkeys)}
    \xmlsortentries{btx}
    \starttabulate[||||]
        \xmlflushsorter{btx}{btx:entry:flush}
    \stoptabulate
\stopxmlsetups

\startxmlsetups btx:entry:flush
    \NC \xmlfilter{#1}{/field[@name='year'  ]/context()}
    \NC \xmlatt{#1}{tag}
    \NC \xmlfilter{#1}{/field[@name='author']/context()}
    \NC \NR
\stopxmlsetups

\xmlsetup{btx:tugboat}{btx:sorter}
\stopbuffer

\typebuffer \getbuffer

The original data is stored in a \LUA\ table, hashed by tag. Starting with \LUA\ 5.2
each run of \LUA\ gets a different ordering of such a hash. In older versions, when you
looped over a hash, the order was undefined, but the same as long as you used the same
binary. This had the advantage that successive runs, something we often have in document
processing gave consistent results. In today's \LUA\ we need to do much more sorting of
hashes before we loop, especially when we save multi||pass data. It is for this reason
that the \XML\ tree is sorted by hash key by default. That way lookups (especially
the first of a set) give consistent outcomes.

\stopchapter

\startchapter[title=Standards]

The rendering of bibliographic entries is often standardized and prescribed by
the publisher. If you submit an article to a journal, normally it will be
reformatted (or even re|-|keyed) and the rendering will happen at the publishers
end. In that case it may not matter how entries were rendered when writing the
publication, because the publisher will do it his or her way.
This means that most users probably will stick to the standard \APA\ rules and for
them we provide some configuration. Because we use setups it is easy to overload
specifics. If you really want to tweak, best look in the files that deal with it.

Many standards exist and support for other renderings may be added to the core.
Interested users are invited to develop and to test alternate standard renderings
according to their needs.

Todo: maybe a list of categories and fields.

\stopchapter

\startchapter[title=Cleaning up]

Although the \BIBTEX\ format is reasonably well defined, in practice there are
many ways to organize the data. For instance, one can use predefined string
constants that get used (either or not combined with other strings) later on. A string
can be enclosed in curly braces or double quotes. The strings can contain \TEX\ commands
but these are not standardized. The databases often have somewhat complex
ways to deal with special characters and the use of braces in their definition is also
not normalized.

The most complex to deal with are the fields that contain names of people. At some point it
might be needed to split a combination of names into individual ones that then get split into
title, first name, optional inbetweens, surname(s) and additional: \type {Prof. Dr. Alfred
B. C. von Kwik Kwak Jr. II and P. Q. Olet} is just one example of this. The convention seems
to be not to use commas but \type {and} to separate names (often each name will be specified
as lastname, firstname).

We don't see it as challenge nor as a duty to support all kinds of messy definitions. Of
course we try to be somewhat tolerant, but you will be sure to get better results if you
use nicely setup, consistent databases.

Todo: maybe some examples of bad.

\stopchapter

\startchapter[title=Transition]

In the original bibliography support module usage was as follows (example taken
from the contextgarden wiki):

\starttyping
% engine=pdftex

\usemodule[bib]
\usemodule[bibltx]

\setupbibtex
  [database=xampl]

\setuppublications
  [numbering=yes]

\starttext
    As \cite [article-full] already indicated, bibtex is a \LATEX||centric
    program.

    \completepublications
\stoptext
\stoptyping

For \MKIV\ the modules were partly rewritten and ended up in the core so the two
{\usemodule} commands were no longer needed. The overhead associated with the
automatic loading of the bibliography macros can be neglected these days, so
standardized modules such as \type {bib} are all being moved to the core and do
not need to be explicitly loaded.

The first \type {\setupbibtex} command in this example is needed to bootstrap
the process: it tells what database has to be processed by \BIBTEX\ between
runs. The second \type {\setuppublications} command is optional. Each citation
(tagged with \type {\cite}) ends up in the list of publications.

In the new approach we no longer use \BIBTEX so we don't need to setup \BIBTEX.
Instead we define dataset(s). We also no longer set up publications with one
command, but have split that up in rendering-, list-, and cite|-|variants. The
basic \type {\cite} command remains. The above example becomes:

\starttyping
\definebtxdataset
  [document]

\usebtxdataset
  [document]
  [mybibfile.bib]

\definebtxrendering
  [document]

\setupbtxrendering
  [document]
  [numbering=yes]

\starttext
    As \cite [article-full] already indicated, bibtex is a \LATEX||centric
    program.

    \completebtxrendering[document]
\stoptext
\stoptyping

So, we have a few more commands to set up things. If you intend to use just a
single dataset and rendering, the above preamble can be simplified to:

\starttyping
\usebtxdataset
  [mybibfile.bib]

\setupbtxrendering
  [numbering=yes]
\stoptyping

But keep in mind that compared to the old \MKII\ derived method we have moved
some of the options to the rendering, list and cite setup variants.

Another difference is now the use of lists. When you define a rendering, you
also define a list. However, all entries are collected in a common list tagged
\type {btx}. Although you will normally configure a rendering you can still set
some properties of lists, but in that case you need to prefix the list
identifier. In the case of the above example this is \type {btx:document}.

\stopchapter

\startchapter[title=\MLBIBTEX]

Todo: how to plug in \MLBIBTEX\ for sorting and other advanced operations.

\stopchapter

\startchapter[title=Extensions]

As \TEX\ and \LUA\ are both open and accessible in \CONTEXT\ it is possible to
extend the functionality of the bibliography related code. For instance, you can add
extra loaders.

\starttyping
function publications.loaders.myformat(dataset,filename)
    local t = { }
    -- Load data from 'filename' and convert it to a Lua table 't' with
    -- the key as hash entry and fields conforming the luadata table
    -- format.
    loaders.lua(dataset,t)
end
\stoptyping

This then permits loading a database (into a dataset) with the command:

\starttyping
\usebtxdataset[standard][myfile.myformat]
\stoptyping

The \type {myformat} suffix is recognized automatically. If you want to use another
suffix, you can do this:

\starttyping
\usebtxdataset[standard][myformat::myfile.txt]
\stoptyping

If you want to add information to an entry at runtime you can pass so called user
variables with the \type {\cite} command. The following example demonstrates
this. First we define a dataset:

\startbuffer
\startbuffer [knuth]
@Book{knuth-texbook,
    title     = {The TeXbook},
    author    = {Knuth, Donald Ervin},
    isbn      = {0-201-13447-0},
    series    = {Computers {\&} Typesetting},
    volume    = {A},
    year      = {1986},
    publisher = {Addison Wesley},
    address   = {Reading, MA},
}
\stopbuffer

\definebtxdataset[knuth]
\usebtxdataset [knuth] [knuth.buffer]
\definebtxrendering[knuth][dataset=knuth]
\stopbuffer

\typebuffer \getbuffer

\startbuffer[setup]
\startsetups btx:apa:lefttext
    \currentbtxlefttext
    \btxspace
    \btxdoifelseuservariable {notabene} {
        {\bs \currentbtxuservariable{notabene}}
    } {
        % nothing
    }
    \btxspace
\stopsetups
\stopbuffer

\getbuffer[setup]

\startbuffer
We all know the \TeX book by Don Knuth \citation [reference=knuth::knuth-texbook,
lefttext={\bf >}] [notabene=Well known to \TEX\ users:].
\stopbuffer

We use this example where we use \type {\citation} instead of \type {\cite} because
it is more tolerant with spaces. Because we pass user variables as second argument
the first argument also has to be a key|/|value set.

\typebuffer

\blank \getbuffer \blank

The list is typeset using:

\startbuffer
\placelistofpublications [knuth] [criterium=all]
\stopbuffer

\typebuffer

and looks like this:

\blank \getbuffer

The injection of the user variables is up to you. Here we hooked it into an
existing setup that we overload:

\typebuffer[setup]

The \type {lefttext} and \type {righttext} variables are also kept with the
entry but these are checked for automatically. In this case it means that
when no \type {lefttext} is specified, the \type {notabene} doesn't show up.

\stopchapter

\startchapter[title=Authors]

The most complicated part of the rendering is authors. The way names are made up
is quite different and depends on culture, history, country and personal taste.
For instance, in the Netherlands you seldom see junior or senior being used, but
in the Unites States this is quite common. Then there is the matter of several
authors cooperating.

\starttexdefinition TestAuthor#1
    \starttabulate[|lT|p|]
        \HL
        \NC \ttx \rlap {\string\citation[alternative=author,authorconversion=...][#1]} \NC \NC \NR
        \HL
        \NC name          \NC \citation[alternative=author,authorconversion=name]         [#1] \NC \NR
        \NC normal        \NC \citation[alternative=author,authorconversion=normal]       [#1] \NC \NR
        \NC normalshort   \NC \citation[alternative=author,authorconversion=normalshort]  [#1] \NC \NR
        \NC inverted      \NC \citation[alternative=author,authorconversion=inverted]     [#1] \NC \NR
        \NC invertedshort \NC \citation[alternative=author,authorconversion=invertedshort][#1] \NC \NR
        \HL
    \stoptabulate
\stoptexdefinition

Herw we give some examples of rendering. The authornames are taken from the
database, analyzed, split and depending on the demand, reconstructed.

\TestAuthor{example::demo-001}
\TestAuthor{example::demo-003}
\TestAuthor{example::demo-003,demo-004}
\TestAuthor{example::demo-006}

As with all other elements of a bibliographic entry you can also finetune the
author name. It's one of the reasons why this subsystem is so complex deep down.
It makes no sense to have a parameter for each aspect, so again we use setups.
You can tweak individual components. Here we show the user friendly variant, in
\type {publ-imp-author} you can find an optimized version.

\startsetups btx:cite:author:normal
    \fastsetup{btx:cite:author:concat}
    \doifsomething {\btxauthorfield{firstnames}} {
        \btxauthorfield{firstnames}
        \btxcitevariantparameter{firstnamesep}
    }
    \doifsomething {\btxauthorfield{vons}} {
        \btxauthorfield{vons}
        \doifsomething {\btxauthorfield{surnames}} {
            \btxcitevariantparameter{vonsep}
        }
    }
    \doifsomething {\btxauthorfield{surnames}} {
        \btxauthorfield{surnames}
        \doifsomething {\btxauthorfield{juniors}} {
            \btxcitevariantparameter{juniorsep}
            \btxauthorfield{juniors}
        }
    }
    \fastsetup{btx:cite:author:etaltext}
\stopsetups

The two concat setups are not shown here. They can be configured using
parameters: \type {namesep}, \type {lastnamesep}, \type {finalnamesep} and \type
{etaltext}, so there is seldom a need to adapt them directly.

Instead of the generic author field accessors you can use macro names which is more
efficient.

\starttabulate[|l|l|]
\NC \type{\currentbtxinitials}   \NC \type{\btxauthorfield{initials}}   \NC \NR
\NC \type{\currentbtxfirstnames} \NC \type{\btxauthorfield{firstnames}} \NC \NR
\NC \type{\currentbtxvons}       \NC \type{\btxauthorfield{vons}}       \NC \NR
\NC \type{\currentbtxsurnames}   \NC \type{\btxauthorfield{surnames}}   \NC \NR
\NC \type{\currentbtxjuniors}    \NC \type{\btxauthorfield{juniors}}    \NC \NR
\stoptabulate

The advantage of the more verbose ones is that you can use manipulators to
process them. This might come in handy when a database is inconsistent.

The two parameters \type {etallimit} and \type {etaldisplay} control the
maximum number of authors displayed ({\em these names can change}).

\stopchapter

\startchapter[title=Journals]

An experimental feature is the ability to load a list of mapping from complete
journal names to abbreviated forms.

\startbuffer
\btxloadjournallist[journals.txt] % the jabref list

\btxexpandedjournal   {Z. Ökol. Nat.schutz} or
\btxabbreviatedjournal{Z. Ökol. Nat.schutz} or
\btxabbreviatedjournal{Z. Ökol. Nat. schutz}
\stopbuffer

\typebuffer \getbuffer

In this case the text file looks like:

\starttyping
Zeitschrift für Ökologie und Naturschutz = Z. Ökol. Nat.schutz
....
\stoptyping

Instead you can have a \LUA\ file that looks like:

\starttyping
return {
    ["Zeitschrift für Ökologie und Naturschutz"] = "Z. Ökol. Nat.schutz",
    ...
}
\stoptyping

or

\starttyping
return {
    { "Zeitschrift für Ökologie und Naturschutz", "Z. Ökol. Nat.schutz" },
    ...
}
\stoptyping

A file can be saved with:

\starttyping
\btxsavejournallist[journals.lua]
\stoptyping

and then loaded again in a second run. For small lists it makes not much sense
to cache the lists but if you have tens thousands of journals it can be
considered. Normally loading is can be neglected compared to the run. Anyhow,
such a list looks like this:

\starttyping
return {
    ["abbreviations"]={
        ["zeitschriftfürökologieundnaturschutz"] = "Z. Ökol. Nat.schutz",
    },
    ["expansions"]={
        ["zökolnatschutz"] = "Zeitschrift für Ökologie und Naturschutz",
    },
}
\stoptyping

In the future \type {mtx-bibtex} might be able to generate such lists (once we know
what users come up with).

You can add additional entries with:

\starttyping
\btxaddjournal
  [Zeitschrift für Ökologie und Naturschutz]
  [Z. Ökol. Nat.schutz]
\stoptyping

As usual with such mechanisms, internally spaces, punctuation and case are
ignored with a lookup.

There are also two manipulators for journals: \type {expandedjournal} and
\type {abbreviatedjournal}.

\stopchapter

\startchapter[title=Other use]

Because a bibliography is just a kind of database, you can use the publications
mechanism for other purposes as well. During the re|-|implementation Mojca came
up with the following definitions:

\startbuffer
\startbuffer[duane]
@IMAGE {tug2013,
    title       = "TUG 2013",
    url         = "http://tug.org/tug2013/",
    url_image   = "http://tug.org/tug2013/tug2013-color-300.jpg",
    url_thumb   = "http://tug.org/tug2013/t2013-thumb.jpg",
    description = "Official drawing of the TUG 2013 conference",
    author      = "Duane Bibby",
    year        = 2013,
    copyright   = "TUG",
}

@IMAGE {tug2014,
    title       = "TUG 2014",
    url         = "http://tug.org/tug2014/",
    url_image   = "http://tug.org/art/tug2014-color.jpg",
    url_thumb   = "http://tug.org/tug2014/t2014-thumb.jpg",
    description = "Official drawing of the TUG 2014 conference",
    author      = "Duane Bibby",
    year        = 2014,
    copyright   = "TUG",
}
\stopbuffer
\stopbuffer

\typebuffer \getbuffer

For documentation purposes we can have a definition in a buffer so that we can
show it verbatim but also load it. The following code defines a dataset, loads
the buffer and sets up a rendering.

\startbuffer
\definebtxdataset
  [duane]

\usebtxdataset
  [duane]
  [duane.buffer]

\definebtxrendering
  [duane]
  [dataset=duane,
   method=dataset,
   alternative=duane,
   criterium=all]

\setupbtxlist
  [duane]
  [number=no]
\stopbuffer

\typebuffer \getbuffer

Instead of for instance \type {apa} we use \type {duane} as alternative. Because
categrories are rendered with a setup we can do the following:

\startbuffer
\startsetups btx:duane:image
  \tbox \bgroup
    \bTABLE[offset=1ex]
      \bTR
        \bTD[ny=4]
          \dontleavehmode
          \externalfigure[\btxfield{url_thumb}][width=3cm]
        \eTD
        \bTD \btxfield{title} \eTD
      \eTR
      \bTR
        \bTD \btxfield{author} \eTD
      \eTR
      \bTR
        \bTD \btxfield{description} \eTD
      \eTR
      \bTR
        \bTD
          \goto{high res variant}[url(\btxfield{url_image})]
        \eTD
      \eTR
    \eTABLE
  \egroup
\stopsetups

\placebtxrendering[duane][criterium=all]
\stopbuffer

\typebuffer \getbuffer

An alternative rendering is:

\startbuffer
\startsetups btx:duane:image
  \bgroup
    \bTABLE[offset=1ex]
      \bTR
        \bTD[ny=4]
          \dontleavehmode
          \externalfigure[\btxfield{url_thumb}][width=3cm]
        \eTD
        \bTD
          \bold{\btxfield{title}}
          \blank
          \btxfield{description}
          \blank
          \goto{high res variant}[url(\btxfield{url_image})]
        \eTD
      \eTR
    \eTABLE
  \egroup
\stopsetups

\placebtxrendering[duane]
\stopbuffer

\typebuffer \getbuffer

We only get the second rendering because we specified \type {criterium} as
\type {all}. Future version of \CONTEXT\ will probably provide sorting
options and ways to plug in additional functionality.

\stopchapter

\startchapter[title=Tracing]

There are several tracing options. If you want to see where a citations refers to and
where a list entry point back to, you can say:

\starttyping
\enabletrackers[publications.crosslinks]
\stoptyping

This injects markers in both places. One list entry can point to multiple citations. The
other tracers a more for debugging and can generate lots of messages.

\starttyping
publications
publications.cite
publications.cite.missing
publications.cite.references
\stoptyping

\stopchapter

\startchapter[title=Summary]

% beware: we use a new dataset for this as we want a full list

\start
\definebtxdataset    [summary]
\usebtxdataset       [summary] [graph.bib]
\setbtxdataset       [summary]
\definebtxrendering  [summary] [dataset=summary]

There are a lot of combinations possible and not all of them make sense.
Nevertheless we show most of them here. (There will be more.)

\startbuffer[samples]
Cleveland : \cite [Cleveland1993,Cleveland1985,Cleveland1993a] \par
Tufte     : \cite [Tufte1983]                                  \par
Bentley   : \cite [Bentley1990]                                \par
All       : \cite [Tufte1983,Cleveland1993,Bentley1990,Cleveland1985,%
                     Cleveland1993a] \par
\stopbuffer

\starttexdefinition BibSampleSet #1#2
    \subsubsubject{alternative=#1 / compress=#2}
    \startpacked
        \setupalign[flushleft]
        \setupbtxcitevariant[#1][compress=#2]
        \setupbtxcitevariant[alternative=#1]
        \getbuffer[samples]
    \stoppacked
\stoptexdefinition

\BibSampleSet{author}     {no}
\BibSampleSet{authoryear} {no}
\BibSampleSet{authoryear} {yes}
\BibSampleSet{authoryears}{no}
\BibSampleSet{authoryears}{yes}
\BibSampleSet{authornum}  {no}
\BibSampleSet{authornum}  {yes}
\BibSampleSet{year}       {no}
\BibSampleSet{year}       {yes}
\BibSampleSet{short}      {no}
\BibSampleSet{serial}     {no}
\BibSampleSet{serial}     {yes}
\BibSampleSet{tag}        {no}
\BibSampleSet{key}        {no}
\BibSampleSet{doi}        {no}
\BibSampleSet{url}        {no}
\BibSampleSet{type}       {no}
\BibSampleSet{category}   {no}
\BibSampleSet{page}       {no}
\BibSampleSet{num}        {no}
\BibSampleSet{num}        {yes}

\startbuffer
\placebtxrendering[summary][criterium=chapter]
\stopbuffer

We produce a local list with:

\typebuffer

and get a list with (new) entries:

\blank \getbuffer \blank

\stop

\stopchapter

\startchapter[title=Notes]

The move from external \BIBTEX\ processing to internal processing has the
advantage that we stay within the same run. In the traditional approach we had
roughly the following steps:

\startitemize[packed]
\startitem the first run information is collected and written to file \stopitem
\startitem after that run the \BIBTEX\ program converts that file to another one \stopitem
\startitem successive runs use that data for references and producing lists \stopitem
\stopitemize

In the \MKIV\ approach the bibliographic database is loaded in memory each run
and processing also happens each run. On paper this looks less efficient but as
\LUA\ is quite fast, in practice performance is much better.

Probably most demanding is the treatment of authors as we have to analyze names,
split multiple authors and reassemble firstnames, vons, surnames and juniors.
When we sort by author sorting vectors have to be made which also has a penalty.
However, in practice the user will not notice a performance degradation. We did
some tests with a list of 500.000 authors, sorted them and typeset them as list
(producing some 5400 dense pages in a small font and with small margins). This is
typical one of these cases where using \LUAJITTEX\ saves quite time. On my
machine it took just over 100 seconds to get this done. Unfortunately not all
operating systems performed equally well: 32 bit versions worked fine, but 64 bit
\LINUX\ either crashed (stalled) the machine or ran out of memory rather fast,
while \MACOSX\ and \WINDOWS\ performed fine. In practice you will never run into
this, unless you produce massive amounts of bibliographic entries. \LUAJIT\ has
some benefits but also some drawbacks.

\stopchapter

\startchapter[title=APA files]

Here are the possible fields per category for APA: \footnote{Currently we show
\type {publ-imp-test.bib} as we need to check things first.}

\definebtxdataset[apadef]
% \usebtxdataset[apadef][publ-imp-apa.bib]
\usebtxdataset[apadef][\cldcontext{resolvers.findfile("publ-imp-test.bib")}]
\showbtxdatasetcompleteness[apadef]

\stopchapter

\stopbodymatter

\startbackmatter

\startchapter[title=Bibliography]

\placelistofpublications [standard] %[criterium=all]

\stopchapter

\stopbackmatter

\writestatus{!!!!!!}{CHECK HYPHENS}

\stopdocument

Todo:

\setuplabeltext[en][reprint=reprint]
\setuplabeltext[de][reprint=Nachdruck]

note = {\labeltext{reprint} 2004}

