With most science fiction films, the more science you 
understand, the {\em less} you admire the film or respect 
its makers. An evil interstellar spaceship careens across 
the screen. The hero's ship fires off a laser blast, 
demolishing the enemy ship---the audience cheers at the 
explosion. But why is the laser beam visible? There is 
nothing in space to scatter the light back to the viewer. 
And what slowed the beam a billionfold to render its advance
toward the enemy ship perceptible? Why, after the moment of 
the explosion, does the debris remain centered in the screen
instead of continuing forward as dictated by the laws of 
inertia? What could possibly drag and slow down the 
expanding debris (and cause the smoke to billow) in the 
vacuum of outer space? Note too the graceful, falling curve 
of the debris. Have the cinematographers forgotten that 
there is no gravity---no \quote {downward}--- in outer 
space? Of course the scene is accompanied by the obligatory 
deafening boom. But isn't outer space eternally silent? And 
even if there were some magical way to hear the explosion, 
doesn't light travel faster than sound? Shouldn't we {\em 
see} the explosion long before we {\em hear} it, just as we 
do with lightning and thunder? Finally, isn't all this moot?
Shouldn't the enemy ship be invisible anyway, as there are 
no nearby stars to provide illumination?  
