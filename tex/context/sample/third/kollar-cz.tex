
Aj zde leží zem ta před okem mým slzy ronícím,
někdy kolébka, nyní národu mého rakev.
Stoj noho! posvátná místa jsou, kamkoli kráčíš,
k obloze, Tatry synu, vznes se vyvýše pohled,
neb raději k velikému prichyl tomu tam se dubisku,
jenž vzdoruje zhoubným až dosaváde časům.
Však času ten horší je člověk, jenž berlu železnou
v těchto krajích na tvou, Slavie, šíji chopil.
Horší, nežli divé války, hromu, ohně divější,
zaslepenec na své když zlobu plémě kydá.
Ó věkové dávní, jako noc vůkol mne ležící,
ó krajino, všeliké slávy i hanby obraz!
Od Labe zrádného k rovinám až Visly nevěrné,
od Dunaje k hltným Baltu celého pěnám:
Krásnohlasy zmužilých Slavjanů kde se někdy ozýval,
aj oněmělť už, byv k ourazu zášti, jazyk.
A kdo se loupeže té, volající vzhůru, dopustil?
Kdo zhanobil v jednom národu lidstvo celé?
Zardi se závistná Teutonie, sousedo Slávy,
tvé vin těchto počet spáchaly někdy ruky!
Neb krve nikde tolik nevylil černidla že žádný
nepřítel, co vylil k záhubě Slávy Němec.
Sám svobody kdo hoden, svobodu zná vážiti každou,
ten kdo do pout jímá otroky, sám je otrok.
Nechť ruky, nechť by jazyk v okovy své vázal otrocké,
jedno to, neb nezná šetřiti práva jiných.
Ten, kdo trůny bořil, lidskou krev darmo vyléval,
po světě nešťastnou války pochodni nosil:
ten porobu slušnou, buď Goth, buď Skytha zasloužil,
ne kdo divé chválil příkladem ordě pokoj.
