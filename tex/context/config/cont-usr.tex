%D \module
%D   [       file=cont-usr,
%D        version=1997.10.05,
%D          title=\CONTEXT\ User Format Specifications,
%D       subtitle=System Specific Setups,
%D         author=Hans Hagen,
%D           date=\currentdate,
%D      copyright={PRAGMA / Hans Hagen \& Ton Otten}]
%C
%C This module is part of the \CONTEXT\ macro||package and is
%C therefore copyrighted by \PRAGMA. See mreadme.pdf for
%C details.

%D In this file users can specify what hyphenation patterns
%D they want to load into the format file. Normally, when
%D using the \type {cont-..} files, this file can best be
%D left unchanged. The default language and font settings
%D done in the \type {cont-.} files take precedence! This file
%D is subject to changes.

\unprotect

%D Hyphenation patterns are normally sought in filed named
%D \type {lang-xx.pat}. When present on the system, those
%D patterns take precedence. (The next list is inspired on
%D Thomas Esser's \TETEX\ distribution.) This list will be
%D adapted to the actual situation.

\definefilesynonym [lang-da.pat]  [dkhyph.tex]
\definefilesynonym [lang-de.pat]  [dehyphn.tex]
\definefilesynonym [lang-es.pat]  [eshyph.tex]
\definefilesynonym [lang-fi.pat]  [fihyph.tex]
\definefilesynonym [lang-fr.pat]  [frhyph.tex]
\definefilesynonym [lang-hr.pat]  [hrhyph.tex]
\definefilesynonym [lang-hu.pat]  [huhyph.tex]
\definefilesynonym [lang-it.pat]  [ithyph.tex]
\definefilesynonym [lang-la.pat]  [lahyph7.tex]
\definefilesynonym [lang-no.pat]  [nohyph.tex]
\definefilesynonym [lang-pl.pat]  [plhyph.tex]
\definefilesynonym [lang-pt.pat]  [pthyph.tex]
\definefilesynonym [lang-ro.pat]  [rohyph.tex]
\definefilesynonym [lang-ru.pat]  [ruenhyph.tex] % sic: ruen
\definefilesynonym [lang-sl.pat]  [sihyph.tex]   % sic: sl/si
\definefilesynonym [lang-sv.pat]  [svhyph.tex]   % was [sehyph.tex]
\definefilesynonym [lang-tr.pat]  [tkhyph.tex]   % was [trhyph.tex]
\definefilesynonym [lang-ua.pat]  [ukrenhyp.tex] % sic ukren
\definefilesynonym [lang-uk.pat]  [ukhyph.tex]

\definefilesynonym [lang-nl.pat]  [nlhyphen.tex] % symbolic name, see below
\definefilesynonym [lang-af.pat]  [nlhyphen.tex] % symbolic name, see below

\definefilesynonym [lang-en.pat]  [ushyphen.tex] % symbolic name, see below
\definefilesynonym [lang-us.pat]  [ushyphen.tex] % symbolic name, see below

\definefilesynonym [lang-cz.pat]  [czhyphen.tex] % in a different part of the tree, sigh
\definefilesynonym [lang-sk.pat]  [skhyphen.tex] % in a different part of the tree, sigh

%definefilesynonym [lang-cz.hyp]  [czhyphen.ex]  % in a different part of the tree, sigh
%definefilesynonym [lang-sk.hyp]  [skhyphen.ex]  % in a different part of the tree, sigh

\definefilesynonym [lang-deo.pat] [dehypht.tex]  % old german patterns

%D When the dutch spelling changed, new patterns were
%D constructed. For long these were named \type {dutch96.pat}.
%D From 2000 however, the old \type {nehyph} files were
%D replaced by \type {nehyph96.tex}. Typical something that
%D you have to find out by accident. The names of hyphenation
%D files as well as their coding is one of the dark areas of
%D \TEX\ distributions.

 \doiffileelse{nehyph96.tex} {\definefilesynonym[nlhyphen.tex][nehyph96.tex]}
{\doiffileelse{dutch96.pat}  {\definefilesynonym[nlhyphen.tex][dutch96.pat]}
                             {\definefilesynonym[nlhyphen.tex][nehyph.tex]}}

%D Ah, something changed in 2003 with respect to ushyph.tex, so let's
%D fall back when needed. I first noticed this during a workshop at  the
%D practical tex conference 2004 in sf. Yet another proof of a mess in
%D filenames. So, we now use \type {ushyphen} as name and do some
%D searching.

 \doiffileelse{ushyph.tex}  {\definefilesynonym[ushyphen.tex][ushyph.tex]}
{\doiffileelse{ushyph1.tex} {\definefilesynonym[ushyphen.tex][ushyph1.tex]}
{\doiffileelse{ushyph2.tex} {\definefilesynonym[ushyphen.tex][ushyph2.tex]}
                            {\definefilesynonym[ushyphen.tex][ukhyph.tex]}}}

%D In order to get 8 bit characters hyphenated, we need to load
%D patterns under the right circumstances. In some countries, more
%D than one font encoding is in use. I can add more defaults here
%D if users let me know what encoding they use.

\installlanguage [\s!nl] [\s!mapping={texnansi,ec},\s!encoding={texnansi,ec}]
\installlanguage [\s!fr] [\s!mapping={texnansi,ec},\s!encoding={texnansi,ec}]
\installlanguage [\s!de] [\s!mapping={texnansi,ec},\s!encoding={texnansi,ec}]
\installlanguage [\s!it] [\s!mapping={texnansi,ec},\s!encoding={texnansi,ec}]

\installlanguage [\s!hr] [\s!mapping={il2,ec},\s!encoding={il2,ec}]
\installlanguage [\s!pl] [\s!mapping={pl0,ec},\s!encoding={pl0,ec}]
\installlanguage [\s!cz] [\s!mapping={il2,ec},\s!encoding={il2,ec}]
\installlanguage [\s!sk] [\s!mapping={il2,ec},\s!encoding={il2,ec}]
\installlanguage [\s!sl] [\s!mapping={il2,ec},\s!encoding={il2,ec}]

%D Additional languages can be defined here. Beware of
%D encoding incompatibilities. Please take a look at the
%D \type {cont-en.tex}, \type {cont-nl.tex}, enz.\ files
%D first. Normally you don't have to change a byte. If you
%D want to play safe, use \typ {texexec --make --alone
%D --all}.

% \installlanguage [\s!af] [\c!state=\v!start] % afrikaans
% \installlanguage [\s!cz] [\c!state=\v!start] % czech
% \installlanguage [\s!da] [\c!state=\v!start] % danish
% \installlanguage [\s!de] [\c!state=\v!start] % german
% \installlanguage [\s!en] [\c!state=\v!start] % english us
% \installlanguage [\s!es] [\c!state=\v!start] % spanish
% \installlanguage [\s!fi] [\c!state=\v!start] % finnish
% \installlanguage [\s!fr] [\c!state=\v!start] % french
% \installlanguage [\s!hr] [\c!state=\v!start] % croatian
% \installlanguage [\s!hu] [\c!state=\v!start] % hungarian
% \installlanguage [\s!it] [\c!state=\v!start] % italian
% \installlanguage [\s!la] [\c!state=\v!start] % latin
% \installlanguage [\s!nl] [\c!state=\v!start] % dutch
% \installlanguage [\s!no] [\c!state=\v!start] % norwegian
% \installlanguage [\s!pl] [\c!state=\v!start] % polish
% \installlanguage [\s!pt] [\c!state=\v!start] % portuguese
% \installlanguage [\s!ro] [\c!state=\v!start] % romanian
% \installlanguage [\s!ru] [\c!state=\v!start] % russian
% \installlanguage [\s!sk] [\c!state=\v!start] % slovak
% \installlanguage [\s!sl] [\c!state=\v!start] % slovenian
% \installlanguage [\s!sv] [\c!state=\v!start] % swedish
% \installlanguage [\s!tr] [\c!state=\v!start] % turkish
% \installlanguage [\s!ua] [\c!state=\v!start] % ukrainian
% \installlanguage [\s!uk] [\c!state=\v!start] % english uk
% \installlanguage [\s!vn] [\c!state=\v!start] % vietnamese

% \installlanguage [deo]   [\c!state=\v!start] % old german

%D The next lines can be used for setting the language to be
%D used at startup time.

% \setupcurrentlanguage[\s!af]
% \setupcurrentlanguage[\s!cz]
% \setupcurrentlanguage[\s!da]
% \setupcurrentlanguage[\s!de]
% \setupcurrentlanguage[\s!en]
% \setupcurrentlanguage[\s!es]
% \setupcurrentlanguage[\s!fi]
% \setupcurrentlanguage[\s!fr]
% \setupcurrentlanguage[\s!hr]
% \setupcurrentlanguage[\s!hu]
% \setupcurrentlanguage[\s!it]
% \setupcurrentlanguage[\s!la]
% \setupcurrentlanguage[\s!nl]
% \setupcurrentlanguage[\s!no]
% \setupcurrentlanguage[\s!pl]
% \setupcurrentlanguage[\s!pt]
% \setupcurrentlanguage[\s!ro]
% \setupcurrentlanguage[\s!ru]
% \setupcurrentlanguage[\s!sk]
% \setupcurrentlanguage[\s!sl]
% \setupcurrentlanguage[\s!sv]
% \setupcurrentlanguage[\s!tr]
% \setupcurrentlanguage[\s!ua]

%D Local font settings can go here. Normally suitable
%D defaults are already preloaded, almost certainly the
%D Computer Modern Roman typefaces or some derivate. So, the
%D next line is only meant as sample, take a look at the
%D format related files first.

% \definefilesynonym [font-cmr] [font-csr] % czech & slovak
% \definefilesynonym [font-cmr] [font-plr] % polish

%D In some languages, compound characters, like \type {"e}
%D are used to get accented and non latin characters.

  \useencoding[fde] % german
% \useencoding[ffr] % french
% \useencoding[fro] % romanian
% \useencoding[fpl] % polish

%D Don't remove the next few lines.

\protect \endinput
