%D \module
%D   [       file=cont-usr,
%D        version=1997.10.05,
%D          title=\CONTEXT\ User Format Specifications,
%D       subtitle=System Specific Setups,
%D         author=Hans Hagen,
%D           date=\currentdate,
%D      copyright={PRAGMA / Hans Hagen \& Ton Otten}]
%C
%C This module is part of the \CONTEXT\ macro||package and is
%C therefore copyrighted by \PRAGMA. See mreadme.pdf for 
%C details. 

%D In this file users can specify what hyphenation patterns 
%D they want to load into the format file. Normally, when 
%D using the \type {cont-..} files, this file can best be 
%D left unchanged. The default language and font settings 
%D done in the \type {cont-.} files take precedence! This file 
%D is subject to changes.

\unprotect

%D Hyphenation patterns are normally sought in filed named
%D \type {lang-xx.pat}. When present on the system, those 
%D patterns take precedence. (The next list is inspired on 
%D Thomas Esser's \TETEX\ distribution.) This list will be 
%D adapted to the actual situation.

\definefilesynonym [lang-cz.pat]  [czhyph.tex]
\definefilesynonym [lang-da.pat]  [dkhyph.tex]
\definefilesynonym [lang-de.pat]  [dehyphn.tex]
\definefilesynonym [lang-en.pat]  [ushyph1.tex]
\definefilesynonym [lang-es.pat]  [eshyph.tex]
\definefilesynonym [lang-fi.pat]  [fihyph.tex]
\definefilesynonym [lang-fr.pat]  [frhyph.tex]
\definefilesynonym [lang-hr.pat]  [hrhyph.tex]
\definefilesynonym [lang-hu.pat]  [huhyph.tex]
\definefilesynonym [lang-it.pat]  [ithyph.tex]
\definefilesynonym [lang-la.pat]  [lahyph7.tex]
\definefilesynonym [lang-nl.pat]  [nehyph.tex]
\definefilesynonym [lang-no.pat]  [nohyph.tex]
\definefilesynonym [lang-pl.pat]  [plhyph.tex]
\definefilesynonym [lang-pt.pat]  [pthyph.tex]
\definefilesynonym [lang-ro.pat]  [rohyph.tex]
\definefilesynonym [lang-ru.pat]  [ruenhyph.tex]
\definefilesynonym [lang-sk.pat]  [skhyph.tex]
\definefilesynonym [lang-sv.pat]  [sehyph.tex]
\definefilesynonym [lang-tr.pat]  [trhyph.tex]
\definefilesynonym [lang-ua.pat]  [ukrenhyp.tex]
\definefilesynonym [lang-uk.pat]  [ukhyphen.tex]
\definefilesynonym [lang-us.pat]  [ushyph1.tex]

%D When the dutch spelling changed, new patterns were
%D constructed. For long these were named \type {dutch96.pat}.
%D From 2000 however, the old \type {nehyph} files were
%D replaced by \type {nehyph96.tex}. Typical something that
%D you have to find out by accident. The names of hyphenation
%D files as well as their coding is one of the dark areas of
%D \TEX\ distributions. 

\doiffileelse{nehyph96.tex}
  {\definefilesynonym[lang-nl.pat][nehyph96.tex]}
  {\doiffileelse{dutch96.pat}
     {\definefilesynonym[lang-nl.pat][dutch96.pat]}
     {\definefilesynonym[lang-nl.pat][nehyph.tex]}}

%D Pattern files are (can be) encoded! And, alas, not all
%D pattern files are self contained, which is why (for the 
%D moment) we specify encodings here. 

\installlanguage [\s!cz] [\s!mapping={il2,ec},\s!encoding={il2,ec}] 
\installlanguage [\s!hr] [\s!mapping={il2,ec},\s!encoding={il2,ec}] 
\installlanguage [\s!pl] [\s!mapping=pl0,\s!encoding=pl0] 
\installlanguage [\s!sk] [\s!mapping={il2,ec},\s!encoding={il2,ec}] 

%D Sometimes these are not wanted: 

%\definefilesynonym [lang-deo.pat] [dehypht.tex]  % old patterns
%\definefilesynonym [lang-nlx.pat] [dutch96.pat]  % new patterns 

%D Additional languages can be defined here. Beware of 
%D encoding incompatibilities. Please take a look at the 
%D \type {cont-en.tex}, \type {cont-nl.tex}, enz.\ files 
%D first. Normally you don't have to change a byte. 

% \installlanguage [\s!af] [\c!status=\v!start] % afrikaans 
% \installlanguage [\s!cz] [\c!status=\v!start] % czech
% \installlanguage [\s!da] [\c!status=\v!start] % danish    
% \installlanguage [\s!de] [\c!status=\v!start] % german    
% \installlanguage [\s!en] [\c!status=\v!start] % english us  
% \installlanguage [\s!es] [\c!status=\v!start] % spanish    
% \installlanguage [\s!fi] [\c!status=\v!start] % finnish
% \installlanguage [\s!fr] [\c!status=\v!start] % french     
% \installlanguage [\s!hr] [\c!status=\v!start] % croatian
% \installlanguage [\s!hu] [\c!status=\v!start] % hungarian 
% \installlanguage [\s!it] [\c!status=\v!start] % italian 
% \installlanguage [\s!la] [\c!status=\v!start] % latin 
% \installlanguage [\s!nl] [\c!status=\v!start] % dutch     
% \installlanguage [\s!no] [\c!status=\v!start] % norwegian 
% \installlanguage [\s!pl] [\c!status=\v!start] % polish
% \installlanguage [\s!pt] [\c!status=\v!start] % portuguese 
% \installlanguage [\s!ro] [\c!status=\v!start] % romanian 
% \installlanguage [\s!ru] [\c!status=\v!start] % russian 
% \installlanguage [\s!sk] [\c!status=\v!start] % slovak
% \installlanguage [\s!sv] [\c!status=\v!start] % swedish   
% \installlanguage [\s!tr] [\c!status=\v!start] % turkish    
% \installlanguage [\s!ua] [\c!status=\v!start] % ukrainian
% \installlanguage [\s!uk] [\c!status=\v!start] % english uk   

% \installlanguage [deo]   [\c!status=\v!start] % old german
% \installlanguage [nlx]   [\c!status=\v!start] % dutch 8 bit 

%D The next lines can be used for setting the language to be 
%D used at startup time. 

% \setupcurrentlanguage[\s!af]
% \setupcurrentlanguage[\s!cz]
% \setupcurrentlanguage[\s!da]
% \setupcurrentlanguage[\s!de]
% \setupcurrentlanguage[\s!en]
% \setupcurrentlanguage[\s!es]
% \setupcurrentlanguage[\s!fi]
% \setupcurrentlanguage[\s!fr]
% \setupcurrentlanguage[\s!hr]
% \setupcurrentlanguage[\s!hu]
% \setupcurrentlanguage[\s!it]
% \setupcurrentlanguage[\s!la]
% \setupcurrentlanguage[\s!nl]
% \setupcurrentlanguage[\s!no]
% \setupcurrentlanguage[\s!pl]
% \setupcurrentlanguage[\s!pt]
% \setupcurrentlanguage[\s!ro]
% \setupcurrentlanguage[\s!ru]
% \setupcurrentlanguage[\s!sk]
% \setupcurrentlanguage[\s!sv]
% \setupcurrentlanguage[\s!tr]
% \setupcurrentlanguage[\s!ua]

%D Local font settings can go here. Normally suitable 
%D defaults are already preloaded, almost certainly the 
%D Computer Modern Roman typefaces or some derivate. So, the 
%D next line is only meant as sample, take a look at the 
%D format related files first. 

% \definefilesynonym [font-cmr] [font-csr] % czech & slovak 
% \definefilesynonym [font-cmr] [font-plr] % polish 

%D In some languages, compound characters, like \type {"e} 
%D are used to get accented and non latin characters. 

  \useencoding[fde] % german 
% \useencoding[fro] % romanian
% \useencoding[fpl] % polish 

%D Don't remove the next few lines. 

\protect \endinput 
