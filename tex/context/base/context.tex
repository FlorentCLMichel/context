%D \module
%D   [       file=context,
%D        version=1995.10.10,
%D          title=\CONTEXT,
%D       subtitle=\CONTEXT\ Format Generation,
%D         author=Hans Hagen,
%D           date=\currentdate,
%D      copyright={PRAGMA / Hans Hagen \& Ton Otten}]
%C
%C This module is part of the \CONTEXT\ macro||package and is
%C therefore copyrighted by \PRAGMA. See mreadme.pdf for 
%C details. 

\catcode`\{=1 \catcode`\}=2 

\def\contextversion{2001.7.4}

%D Welcome to the main module. When this module is ran through
%D \type{initex} or \type{tex -i} or \type{whatevertex} using
%D \type{whatever switch}, the \CONTEXT\ format file is
%D generated. During this process the user is asked for an
%D interface language. Supplying \type{dutch} will generate a
%D dutch version of \CONTEXT, supplying \type{english} will of
%D course end op in a english version.
%D
%D First we load the system modules. These implement a lot of
%D manipulation macros. The first one loads \PLAIN\ \TEX, as
%D minimal as possible.

\input syst-tex.tex  
\input syst-etx.tex  
\input syst-gen.tex
\input syst-ext.tex
\input syst-new.tex
\input syst-con.tex

%D To enable selective loading, we say:

\CONTEXTtrue

%D In order to conveniently load files, we need a few 
%D support modules.  

\input supp-ini.tex
\input supp-fil.tex

%D After this we're ready for the multi||lingual interface
%D modules.

\input mult-ini.tex
\input mult-sys.tex
\input mult-con.tex
\input mult-com.tex

%D We also use some third party macros. These are loaded by
%D saying:

\input thrd-ran.tex

%D Now we're ready for some general support modules. These
%D modules implement some basic typesetting functionality.

\input supp-box.tex
\input supp-mrk.tex
\input supp-vis.tex
\input supp-fun.tex
\input supp-eps.tex
\input supp-pdf.tex
\input supp-spe.tex
\input supp-mps.tex
\input supp-mpe.tex
\input supp-tpi.tex
\input supp-mat.tex
\input supp-ran.tex
\input supp-ali.tex

%D The next module deals with language specific typographic 
%D extensions. 

\input typo-ini.tex  

%D Verbatim typesetting is implemented in a separate class of
%D modules. The pretty typesetting modules are loaded at run
%D time.

\input verb-ini.tex  % replaces supp-ver

%D When loading the font, color and special modules, we need a
%D bit more advanced file handling as well as some general
%D variables, so next we load:

\input core-var.tex
\input core-fil.tex

%D We already need some synonyms (patterns). At runtime this 
%D file will be reloaded. 

\input cont-fil.tex 

%D \CONTEXT\ does not implement its own table handling. We
%D just go for the best there is and load \TABLE. Just to be
%D sure we do it here, before we redefine \type{|}.

\input thrd-tab.tex   

%D Here comes the last support modules. They take care of
%D some language specific things.

\input supp-lan.tex
\input supp-num.tex

%D The next few modules do what their names state. They
%D load additional definition modules when needed.

\input enco-ini.tex 
\input hand-ini.tex 
\input regi-ini.tex 
\input lang-ini.tex

\input colo-ini.tex

\input spec-mis.tex
\input spec-ini.tex

%D For the moment we load a lot of languages. In the future
%D we'll have to be more space conservative.

\input lang-lab.tex

\input lang-alt.tex
\input lang-ana.tex
\input lang-art.tex
\input lang-bal.tex
\input lang-cel.tex
\input lang-ger.tex
\input lang-grk.tex
\input lang-ind.tex
\input lang-ita.tex
\input lang-sla.tex
\input lang-ura.tex

\input lang-vn.tex    % vietnamese, will be grouped 

%D All kind of symbols are handles in:

\input symb-ini.tex

%D Next we load some core macro's. These implement the
%D macros' that are seen by the users. The order of loading 
%D is important, due to dependancies. 

\input core-gen.tex
\input core-uti.tex
\input core-mar.tex
\input core-mak.tex
\input core-dat.tex
\input core-grd.tex
\input core-ver.tex
\input core-vis.tex
\input core-con.tex
\input core-rul.tex
\input core-new.tex
\input core-tab.tex
\input core-nav.tex
\input core-ref.tex
\input core-obj.tex
\input core-buf.tex
\input core-lst.tex 
\input core-num.tex 
\input core-itm.tex 
\input core-des.tex 
\input core-mat.tex
\input core-syn.tex 
%input core-mul.tex % moved 
\input core-spa.tex 
\input core-sys.tex 

% experimental otr 

\input page-ini.tex 
\input page-one.tex 
\input page-lay.tex 
\input page-log.tex
\input page-txt.tex
\input page-sid.tex 
\input page-flt.tex 
\input page-mul.tex 
\input page-set.tex 
\input page-lyr.tex
\input page-mak.tex
\input page-num.tex
\input page-lin.tex
\input page-mar.tex
\input page-bck.tex

\input core-job.tex

% so far 

\input core-sec.tex 
\input page-imp.tex
\input core-tbl.tex
\input core-int.tex
\input core-ntb.tex

%D Like languages, fonts, encodings and symbols, \METAPOST\ 
%D support is also organized in its own class of modules. 

\input meta-ini.tex
\input meta-pag.tex
\input meta-fig.tex

%D On which the next one depends:

\input core-pos.tex

%D A few more languages, that have specifics using core 
%D functionality: 

\input lang-chi.tex

%D How about fill||in fields and related stuff?

\input java-ini.tex
\input core-fld.tex
\input core-hlp.tex

%D Registers can depend on fields, so we load that now.

\input core-reg.tex

%D Of course we do need fonts. There are no \TFM\ files
%D loaded yet, so the format file is independant of their
%D content. Here we also redefine \type{\it} as {\it italic}
%D instead of italian.

\input font-ini.tex
\input font-uni.tex

\input type-ini.tex

\input math-ini.tex % after enco-ini.tex 

%D Now we're ready for more core modules.

\input core-fnt.tex
\input core-not.tex

\input core-mis.tex

\input core-fig.tex % after page body 
\input core-par.tex

%D Only the basic XML parser is part of the core. 

\input xtag-ini.tex
\input xtag-ext.tex
%input xtag-map.tex

%D The next two modules implement some additional
%D functionality concenring classes of documents and output.
%D These modules probably will be replaced some day.

\input docs-ini.tex
\input list-ini.tex

%D \TEX\ related logo's are always typeset in a special way.
%D Here they come:

\input cont-log.tex

%D At run time, a few more files are loaded, like:
%D
%D \startopsomming[opelkaar]
%D \som \type{cont-sys}: local (system dependant) defaults
%D \som \type{cont-old}: substitutes for old (obsolete) macros
%D \som \type{cont-new}: new macro implementations (for testing)
%D \som \type{cont-fil}: filename and module synonyms
%D \stopopsomming

%D Just to keep the user busy for a while, we say:  

\startinterface english

\writestring{This package is based on Plain TeX. It uses an adapted version of the}
\writestring{extended mark mechanism of J. Fox (1987) and a few parts of the sidefloat}
\writestring{mechanism of D. Comenetz (1993). Most of D.E. Knuth's Plain TeX}
\writestring{(\fmtversion) is available and can be used without problems. This package}
\writestring{uses TaBlE, a package designed and copyrighted by M.J. Wichura (1988).}
\writestring{Only a few auxiliary files are generated, of which some must be processed}
\writestring{by TeXUtil (\utilityversion). The current blockversion is \blockversion.}

\stopinterface

\startinterface dutch

\writestring{Dit pakket is gebaseerd op Plain TeX. Er wordt gebruik gemaakt van een}
\writestring{aangepaste versie van het mark mechanisme van J. Fox (1987) en onderdelen}
\writestring{van het sidefloat mechanisme van D. Comenetz (1993). De functionaliteit}
\writestring{van D.E. Knuth's Plain TeX (\fmtversion) is grotendeels beschikbaar en}
\writestring{kan zonder problemen worden gebruikt. Dit pakket gebruikt TaBlE, ontworpen door}
\writestring{M.J. Wichura (1988), die ook het auteursrecht bezit. Er worden slechts een}
\writestring{paar hulpfiles gegenereerd, waarvan er enkele moeten worden bewerkt door}
\writestring{TeXUtil (\utilityversion). Het blokmechanisme heeft versienummer \blockversion.}

\stopinterface

\startinterface german

\writestring{Dieses Paket basiert auf Plain-TeX und benutzt eine angepasste Version}
\writestring{des erweiterten mark-Mechanismus von J. Fox (1987) und einige Teile des}
\writestring{sidefloat-Mechanismus von D. Comenetz (1993). Ein Grossteil D.E. Knuths}
\writestring{Plain-TeX (\fmtversion) ist verfuegbar und kann ohne Probleme benutzt werden.}
\writestring{Dieses Paket benutzt TaBlE, ein von M.J. Wichura (1988) erstelltes und}
\writestring{geschuetztes Paket. Nur einige Hilfsdateien werden erstellt; einige davon}
\writestring{muessen von TeXUtil (\utilityversion) bearbeitet werden. Die aktuelle Block-}
\writestring{version ist \blockversion.}

\stopinterface

\startinterface czech

\writestring{Tento balik je zalozen na Plain TeXu. Pouziva prizpusobenou verzi}
\writestring{rozsireneho znackovaciho mechanismu J. Foxe (1987) a nekolik casti}
\writestring{sidefloat mechanismu D. Comenetze (1993). Vetsina prikazu Plain TeXu}
\writestring{D. E. Knutha (\fmtversion) je dostupna a muze byt bez problemu pouzita.}
\writestring{Tento balik pouziva balik TaBlE, ktery vytvoril M. J. Wichura (1988).}
\writestring{Je generovano jen nekolik pomocnych souboru, z nichz nektere musi byt}
\writestring{zpracovany programem TeXUtil (\utilityversion). Aktualni verze}
\writestring{}
\writestring{THE CZECH USER INTERFACE IS STILL UNDER DEVELOPMENT!}

\stopinterface

\startinterface italian

\writestring{Questo pacchetto � basato sul Plain TeX. Usa una versione adattata del}
\writestring{meccanismo di marcatura esteso di J. Fox (1987) ad alcune parti del}
\writestring{meccanismo per gli oggetti mobili laterali di D. Comenetz (1993).}
\writestring{La maggior parte del Plain TeX (\fmtversion) di D.E. Knuth � disponibile}
\writestring{e pu� essere usata senza problemi. Questo pacchetto usa TaBlE,}
\writestring{un pacchetto progettato da e con diritti di copia di M.J. Wichura (1988).}
\writestring{Vengono generati pochi file ausiliari, alcuni dei quali devono essere}
\writestring{elaborati da TeXUtil (\utilityversion). La versione attuale del blocco}
\writestring{� \blockversion.}
\writestring{}
\writestring{L'INTERFACCIA UTENTE ITALIANA E' ANCORA IN VIA DI SVILUPPO!}
\writestring{THE ITALIAN USER INTERFACE IS STILL UNDER DEVELOPMENT!}

\stopinterface

\startinterface romanian

\writestring{Acest pachet este bazat pe Plain TeX. Foloseste o versiune adaptata a}
\writestring{mecanismului extins de marcare a lui J. Fox (1987) si cateva parti a mecanismului }
\writestring{blocurilor marginale a lui D. Comenetz (1993). Cea mai mare parte a Plain Tex}
\writestring{(\fmtversion) a lui D.E. Knuth este disponibila si poate fi folosita fara probleme.}
\writestring{Acest pachet foloseste TaBlE, un pachet proiectat si creat de M.J. Wichura (1988).}
\writestring{Numai un numar de fisiere auxiliare sunt generate, din care unele trebuie procesate}
\writestring{de catre TeXUtil (\utilityversion). Versiunea curenta de blocuri este \blockversion.}

\stopinterface

\edef\copyrightversion%
  {Copyright 1990-\the\normalyear\normalspace / 
   PRAGMA ADE / J. Hagen - A.F. Otten}

\writeline\writestring{\copyrightversion}\writeline

% %D Except from english, no hyphenation patterns are loaded
% %D yet. Users can specify their needs in the next module:
%
% \input cont-usr.tex

%D Let's quit this file when doing a \type {cont-..} generation.

\doifparentfileelse{context}
  {\let\next\relax}
  {\let\next\endinput}

\next 

%D Unless we're generating a \type {cont-..} format, we also 
%D do the following. 

%D Except from english, no hyphenation patterns are loaded
%D yet. Users can specify their needs in the next module:

\loaduserspecifications 

%D Next we default to the same language as the interface. 

\unprotect

\installlanguage [\s!en] [\c!status=\v!start] 

\startinterface english 

  \installlanguage [\s!uk] [\c!status=\v!start]

\stopinterface 

\appendtoks \language     [\s!en] \to \everyjob
\appendtoks \mainlanguage [\s!en] \to \everyjob

\startinterface german

  \installlanguage [\s!de] [\c!status=\v!start]

  \appendtoks \language     [\s!de] \to \everyjob
  \appendtoks \mainlanguage [\s!de] \to \everyjob

\stopinterface 

\startinterface dutch

  \installlanguage [\s!nl] [\c!status=\v!start]

  \appendtoks \language     [\s!nl] \to \everyjob
  \appendtoks \mainlanguage [\s!nl] \to \everyjob

\stopinterface 

\startinterface czech 

  \installlanguage [\s!cz] [\c!status=\v!start]

  \appendtoks \language     [\s!cz] \to \everyjob
  \appendtoks \mainlanguage [\s!cz] \to \everyjob

\stopinterface 

\startinterface italian 

  \installlanguage [\s!it] [\c!status=\v!start]

  \appendtoks \language     [\s!it] \to \everyjob
  \appendtoks \mainlanguage [\s!it] \to \everyjob

\stopinterface 

\startinterface romanian 

  \installlanguage [\s!ro] [\c!status=\v!start]

  \appendtoks \language     [\s!ro] \to \everyjob
  \appendtoks \mainlanguage [\s!ro] \to \everyjob

\stopinterface 

\protect

%D Finally we load some fonts. 

\setupbodyfont [cmr,rm,12pt]

%D Now dumping the format is all that's left to be done.

\errorstopmode \dump

\endinput
