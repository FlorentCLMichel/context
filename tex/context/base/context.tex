%D \module
%D   [       file=context,
%D        version=1995.10.10,
%D          title=\CONTEXT,
%D       subtitle=\CONTEXT\ Format Generation, 
%D         author=Hans Hagen,
%D           date=\currentdate,
%D      copyright={PRAGMA / Hans Hagen \& Ton Otten}]
%C
%C This module is part of the \CONTEXT\ macro||package and is
%C therefore copyrighted by \PRAGMA. Non||commercial use is 
%C granted. 

%D Welcome to the main module. When this module is ran through
%D \type{initex} or \type{tex -i} or \type{whatevertex} using
%D \type{whatever switch}, the \CONTEXT\ format file is
%D generated. During this process the user is asked for an
%D interface language. Supplying \type{dutch} will generate a
%D dutch version of \CONTEXT, supplying \type{english} will of
%D course end op in a english version. 
%D 
%D First we load the system modules. These implement a lot of 
%D manipulation macros. The first one loads \PLAIN\ \TEX, as 
%D minimal as possible. 

\input syst-tex.tex  
\input syst-gen.tex 
\input syst-ext.tex
\input syst-new.tex

%D To enable selective loading, we say:

\CONTEXTtrue

%D After this we're ready for the multi||lingual interface 
%D modules. 

\input mult-ini.tex
\input mult-sys.tex
\input mult-con.tex
\input mult-com.tex

%D Now we're ready for some general support modules. These 
%D modules implement some basic typesetting functionality.  

\input supp-ini.tex
\input supp-fil.tex
\input supp-box.tex
\input supp-mrk.tex
\input supp-vis.tex
\input supp-fun.tex
\input supp-pdf.tex
\input supp-spe.tex
\input supp-mps.tex
\input supp-tpi.tex

%D Verbatim typestting is implemented in a separate class of 
%D modules. The pretty typesetting modules are loaded at run 
%D time. 

\input verb-ini.tex  % replaces supp-ver

%D We also use some third party macros. These are loaded by 
%D saying:

\input thrd-ran

%D When loading the font, color and special modules, we need a
%D bit more advanced file handling, so next we load:

\input core-fil.tex

%D \CONTEXT\ does not implement its own table handling. We 
%D just go for the best there is and load \TABLE. Just to be 
%D sure we do it here, before we redefine \type{|}. 

\doinputonce{table}       

%D Here comes the last support module. 

\input supp-lan.tex

%D The next three modules do what their names state. They 
%D load additional definition modules when needed. 

\input lang-ini.tex
\input colo-ini.tex
\input spec-ini.tex

%D The special modules need some additional macro's: 

\input spec-mis.tex

%D For the moment we load a lot of languages. In the future 
%D we'll have to be more space conservative. 

\input lang-lab.tex

\input lang-alt.tex
\input lang-ana.tex
\input lang-art.tex
\input lang-bal.tex
\input lang-cel.tex
\input lang-ger.tex
\input lang-grk.tex
\input lang-hnl.tex
\input lang-ind.tex
\input lang-ita.tex
\input lang-sla.tex
\input lang-ura.tex

%D Next we load some core macro's. These implement the 
%D macros' that are seen by the users. 

\input core-var.tex
\input core-gen.tex
\input core-mak.tex
\input core-grd.tex
\input core-ver.tex
\input core-vis.tex
\input core-con.tex
\input core-rul.tex
\input core-tab.tex
\input core-fil.tex
\input core-new.tex
\input core-nav.tex 
\input core-ref.tex 
\input core-obj.tex
\input core-01a.tex 
\input core-mul.tex 
\input core-pag.tex 

%D Of course we do need fonts. There are no \TFM\ files 
%D loaded yet, so the format file is independant of their 
%D content. Here we also redefine \type{\it} as {\it italic} 
%D instead of italian. 

\input font-ini.tex

%D Now we're ready for more core modules. 

\input core-fnt.tex
\input core-not.tex
% \input core-jav.tex % still experimental and loaded at runtime

\input core-01b.tex
\input core-01c.tex
\input core-01d.tex
\input core-01e.tex

\input core-02a.tex
\input core-02b.tex
%      core-02c.tex
\input core-02d.tex

%D Except from english, no hyphenation pattersn are loaded 
%D yet. Users can specify their needs in the next module: 

\input cont-usr.tex

%D The next two modules implement some additional 
%D functionality concenring classes of documents and output.  
%D These modules probably will be replaced some day. 

\input docs-ini.tex
\input list-ini.tex

%D \TEX\ related logo's are always typeset in a special way. 
%D Here they come: 

\input cont-log.tex

%D At run time, a few more files are loaded, like:
%D 
%D \startopsomming[opelkaar]
%D \som \type{cont-sys}: local (system dependant) defaults
%D \som \type{cont-old}: substitutes for old (obsolete) macros
%D \som \type{cont-new}: new macro implementations (for testing)
%D \som \type{cont-fil}: filename and module synonyms
%D \stopopsomming
 
%D Dumping the format is all that's left to be done. 

\dump
