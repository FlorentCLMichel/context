%D \module
%D   [       file=supp-box,
%D        version=1995.10.10,
%D          title=\CONTEXT\ Support Macros,
%D       subtitle=Boxes,
%D         author=Hans Hagen,
%D           date=\currentdate,
%D      copyright={PRAGMA / Hans Hagen \& Ton Otten}]
%C
%C This module is part of the \CONTEXT\ macro||package and is
%C therefore copyrighted by \PRAGMA. See mreadme.pdf for 
%C details. 

%D This module implements some box manipulation macros. Some
%D are quite simple, some are more advanced and when understood
%D well, all can be of use.

\writestatus{loading}{Context Support Macros / Boxes}

\unprotect

%D \macros
%D   {nextdepth}
%D
%D Let's start with a rather simple declaration. Sometimes we
%D need to save the \TEX\ \DIMENSION\ \type{\prevdepth} and
%D append it later on. The name \type{\nextdepth} suits
%D this purpose well.

\newdimen\nextdepth

%D \macros
%D   {smashbox}
%D
%D Smashing is introduced in \PLAIN\ \TEX, and stands for
%D reducing the dimensions of a box to zero. The most resolute
%D one is presented first.

\def\smashbox#1%
  {\wd#1=\!!zeropoint
   \ht#1=\!!zeropoint
   \dp#1=\!!zeropoint}

%D \macros
%D   {hsmashbox,vsmashbox}
%D
%D Smashing can be used for overlaying boxes. Depending on
%D the mode, horizontal or vertical, one can use:

\def\hsmashbox#1%
  {\wd#1=\!!zeropoint}

\def\vsmashbox#1%
  {\ht#1=\!!zeropoint
   \dp#1=\!!zeropoint}

%D \macros
%D   {hsmash,vsmash,
%D    hsmashed,vsmashed}
%D
%D While the previous macros expected a \BOX, the next act on a
%D content. They are some subtle differences betreen the smash
%D and smashed alternatives. The later ones reduce all
%D dimensions to zero.

\def\hsmash#1%
  {\bgroup
   \setbox0=\normalhbox{#1}%
   \hsmashbox0%
   \box0
   \egroup}

\def\vsmash#1%
  {\bgroup
   \setbox0=\normalvbox{#1}%
   \vsmashbox0%
   \box0
   \egroup}

\def\hsmashed#1%
  {\bgroup
   \setbox0=\normalhbox{#1}%
   \smashbox0%
   \box0
   \egroup}

\def\vsmashed#1%
  {\bgroup
   \setbox0=\normalvbox{#1}%
   \smashbox0%
   \box0
   \egroup}

%D \macros
%D   {getboxheight}
%D
%D Although often needed, \TEX\ does not support arithmics
%D like:
%D
%D \starttypen
%D \dimen0 = \ht0 + \dp0
%D \stoptypen
%D
%D so we implemented:
%D
%D \starttypen
%D \getboxheight ... \of \box...
%D \stoptypen
%D
%D For instance,
%D
%D \starttypen
%D \getboxheight \dimen0 \of \box0
%D \getboxheight \someheight \of \box \tempbox
%D \stoptypen

\def\getboxheight#1\of#2\box#3%
  {#1=\ht#3%
   \advance#1 by \dp#3\relax}

%D \macros 
%D   {doiftextelse, doiftext}
%D
%D When \type {\doifelse} cum suis hopelessly fail, for 
%D instance because we pass data, we can fall back on the next 
%D macro: 
%D 
%D \starttypen
%D \doiftextelse {data} {then branch} {else branch}
%D \doiftext     {data} {then branch} 
%D \stoptypen

\def\doiftextelse#1#2#3%
  {\bgroup
   \setbox0=\hbox{#1}%
   \ifdim\wd0>\!!zeropoint
     \egroup#2%
   \else
     \egroup#3%
   \fi}

\def\doiftext#1#2%
  {\doiftextelse{#1}{#2}{}}

%D \macros
%D   {dowithnextbox,nextbox}
%D
%D Sometimes we want a macro to grab a box and do something
%D on the content. One could pass an argument to a box, but
%D this can violate the specific \CATCODES\ of its content and
%D leads to unexpected results. The next macro treats the
%D following braced text as the content of a box and
%D manipulates it afterwards in a predefined way.
%D
%D The first argument specifies what to do with the content.
%D This content is available in \type{\nextbox}. The second
%D argument is one of \type{\hbox}, \type{\vbox} or
%D \type{\vtop}. The third argument must be grouped with
%D \type{\bgroup} and \type{\egroup}, \type{{...}} or can be
%D a \type{\box} specification.
%D
%D In \CONTEXT\ this macro is used for picking up a box and
%D treating it according to earlier specifications. We use for
%D instance something like:
%D
%D \starttypen
%D \def\getfloat%
%D   {\def\handlefloat{...\box\nextbox...}
%D    \dowithnextbox\handlefloat\vbox}
%D \stoptypen
%D
%D instead of:
%D
%D \starttypen
%D \def\getfloat#1%
%D   {...#1...}
%D \stoptypen
%D
%D In this implementation the \type{\aftergroup} construction
%D is needed because \type{\afterassignment} is executed inside
%D the box.

\newbox\nextbox

\long\def\dowithnextbox#1%
  {\long\def\dodowithnextbox{#1}%
   \afterassignment\dododowithnextbox
   \setbox\nextbox}

% Better but first to be checked:
%
%\long\def\dowithnextbox#1%
%  {\bgroup\long\def\dodowithnextbox{#1\egroup}%
%   \afterassignment\dododowithnextbox
%   \setbox\nextbox}

\def\dododowithnextbox%
  {\aftergroup\dodowithnextbox}

%D So in fact we get:
%D
%D \starttypen
%D \setbox\nextbox { \aftergroup\dodowithnextbox ... }
%D \stoptypen
%D
%D or
%D
%D \starttypen
%D \setbox\nextbox { ... } \dodowithnextbox
%D \stoptypen
%D
%D A slower but more versatile implementation is:
%D
%D \starttypen
%D \long\def\dowithnextbox#1#2%
%D   {\long\def\dodowithnextbox{#1}%
%D    \ifx#2\hbox
%D      \afterassignment\dododowithnextbox
%D    \else\ifx#2\vbox
%D      \afterassignment\dododowithnextbox
%D    \else\ifx#2\vtop
%D      \afterassignment\dododowithnextbox
%D    \else\ifx#2\vcenter
%D      \afterassignment\dododowithnextbox
%D    \else
%D      \afterassignment\dodowithnextbox
%D    \fi\fi\fi\fi
%D    \setbox\nextbox#2}
%D \stoptypen
%D
%D This alternative also accepts \type{\box0} and alike, but 
%D we don't really need this functionality now.  

%D \macros
%D   {beginofshapebox,
%D    reshapebox, doreshapebox,
%D    flushshapebox,
%D    innerflushshapebox,
%D    shapebox,
%D    ifreshapingbox}
%D
%D The next utility macro originates from some linenumbering
%D mechanism. Due to \TEX's advanced way of typesetting
%D paragraphs, it's not easy to do things on a line||by||line
%D basis. This macro is able to reprocess a given box and can
%D act upon its vertical boxed components, such as lines. The
%D unwinding sequence in this macro is inspired by a \NTG\
%D workshop of David Salomon in June 1992.
%D
%D First we have to grab the piece of text we want to act
%D upon. This is done by means of the duo macros:
%D
%D \starttypen
%D \beginofshapebox
%D a piece of text
%D \endofshapebox
%D \stoptypen
%D
%D When all texts is collected, we can call \type{\reshapebox}
%D and do something with it's vertical components. We can make
%D as much passes as needed. When we're done, the box can be
%D unloaded with \type{\flushshapebox}. The only condition in
%D this scheme is that \type{\reshapebox} must somehow unload
%D the \BOX\ \type{\shapebox}.
%D
%D An important aspect is that the content is unrolled
%D bottom||up. The next example illustrates this maybe
%D unexpected characteristic.
%D
%D \startbuffer
%D \beginofshapebox
%D \em \input tufte
%D \endofshapebox
%D
%D \newcounter\LineNumber
%D
%D \reshapebox
%D   {\doglobal\increment\LineNumber
%D    \hbox{\llap{\LineNumber\hskip2em}\box\shapebox}}
%D
%D \flushshapebox
%D \stopbuffer
%D
%D \typebuffer
%D
%D \haalbuffer
%D
%D As we can see, when some kind of numbering is done, we have
%D to add a second pass.
%D
%D \startbuffer
%D \newcounter\LineNumber
%D \newcounter\NumberOfLines
%D
%D \reshapebox
%D   {\doglobal\increment\NumberOfLines
%D    \box\shapebox}
%D
%D \reshapebox
%D   {\doglobal\increment\LineNumber
%D    \hbox
%D      {\llap{\LineNumber\ (\NumberOfLines)\hskip2em}%
%D       \box\shapebox}%
%D    \doglobal\decrement\NumberOfLines}
%D
%D \flushshapebox
%D \stopbuffer
%D
%D \typebuffer
%D
%D \haalbuffer
%D
%D This example shows that the content of the box is still
%D available after flushing. Another feature is that only the
%D last reshaping counts. Multiple reshaping can be done by:
%D
%D \startbuffer
%D \beginofshapebox
%D \flushshapebox 
%D \endofshapebox
%D
%D \reshapebox
%D   {\doglobal\increment\LineNumber
%D    \hbox{\llap{$\star$\hskip1em}\box\shapebox}%
%D    \doglobal\decrement\NumberOfLines}
%D
%D \flushshapebox
%D \stopbuffer
%D
%D \typebuffer
%D
%D \haalbuffer
%D
%D The macros are surprisingly easy to follow and in fact
%D introduce no new concepts. Nearly all books on \TEX\ show
%D similar solutions for unwinding \BOXES.
%D
%D Some macros, like footnote ones, can be sensitive for
%D reshaping, which can result in an endless loop. We
%D therefore offer:
%D
%D \starttypen
%D \ifreshapingbox
%D \stoptypen
%D
%D Some \CONTEXT\ commands are protected this way. Anyhow,
%D reshaping is aborted after 100 dead cycles.
%D
%D By the way, changing the height and depth of \BOX\
%D \type{\shapebox} results in bad spacing. This means that
%D for instance linenumbers etc. should be given zero height
%D and depth before being lapped into the margin. The
%D previous examples ignore this side effect, but beware!

\newif\ifsomeshapeleft
\newif\ifreshapingbox

\def\shapesignal {.12345678pt}

\newbox   \shapebox
\newcount \shapepenalty
\newdimen \shapekern
\newskip  \shapeskip

\newbox\newshapebox
\newbox\oldshapebox

\newcount\shapecounter

\def\reshapebox#1%
  {\doreshapebox
     {#1}
     {\penalty\shapepenalty}
     {\kern\shapekern}
     {\vskip\shapeskip}}

\def\doreshapebox#1#2#3#4% \shapebox, \shapepenalty, \shapekern, \shapeskip
  {\setbox\newshapebox=\normalvbox
     \bgroup
       \unvcopy\oldshapebox
       \setbox\newshapebox=\box\voidb@x
       \shapecounter=0
       \loop
         \someshapelefttrue
         \ifdim\lastskip=\!!zeropoint\relax
           \ifdim\lastkern=\!!zeropoint\relax
             \ifnum\lastpenalty=0
               \setbox\shapebox=\lastbox
               \ifvoid\shapebox
                 \unskip\unpenalty\unkern
               \else
                 \ifdim\wd\shapebox=\shapesignal\relax
                   \someshapeleftfalse
                 \else
                   \shapecounter=0
                   \setbox\newshapebox=
                     \normalvbox{#1\unvbox\newshapebox}
                 \fi
               \fi
             \else
               \shapepenalty=\lastpenalty
               \setbox\newshapebox=
                 \normalvbox{#2\unvbox\newshapebox}
               \unpenalty
             \fi
           \else
             \shapekern=\lastkern
             \setbox\newshapebox=
               \normalvbox{#3\unvbox\newshapebox}
             \unkern
           \fi
         \else
           \shapeskip=\lastskip
           \setbox\newshapebox=
             \normalvbox{#4\unvbox\newshapebox}
           \unskip
         \fi
       \ifnum\shapecounter>100 % can be less
         \message{<<forced exit from shapebox>>}%
         \someshapeleftfalse
       \else
         \advance\shapecounter by 1
       \fi
       \ifsomeshapeleft \repeat
       \unvbox\newshapebox
     \egroup}

\def\beginofshapebox%
  {\setbox\oldshapebox=\normalvbox
     \bgroup
     \reshapingboxtrue
     \hbox to \shapesignal{\hss}}

\def\endofshapebox%
  {\endgraf
   \egroup}

\let\beginshapebox=\beginofshapebox
\let\endshapebox  =\endofshapebox

\def\flushshapebox%
  {\ifdim\ht\newshapebox=\!!zeropoint\relax
   \else
     % make \prevdepth legal
     % \par before the next \vskip gives far worse results
     \ifdim\parskip>\!!zeropoint\vskip\parskip\else\par\fi 
     % and take a look
     \ifdim\prevdepth=-1000pt 
       \prevdepth=\!!zeropoint
     \fi
     \ifdim\prevdepth<\!!zeropoint\relax
       % something like a line or a signal or ...
       \donetrue
     \else\ifinner
       % not watertight and not ok
       \donefalse
     \else\ifdim\pagegoal=\maxdimen
       \donetrue
     \else
       % give the previous line a normal depth
       \donetrue
       \vbox{\forgetall\strut}\nobreak\kern-\lineheight % geen \vskip
       \vskip-\dp\strutbox 
     \fi\fi\fi
     \unvcopy\newshapebox\relax
     % \prevdepth=0pt and \dp\newshapebox depend on last line
     \kern-\dp\newshapebox\relax
     % now \prevdepth=0pt
     \ifdone
       \kern\dp\strutbox 
       \prevdepth\dp\strutbox 
     \fi
   \fi}

%D In real inner situations we can use: 
%D
%D \starttypen
%D \flushinnershapebox
%D \stoptypen
%D
%D This one is used in \type{\framed}. 

\def\innerflushshapebox%
  {\ifdim\ht\newshapebox=\!!zeropoint\relax
   \else
     \unvcopy\newshapebox\relax
     \kern-\dp\newshapebox\relax
   \fi}

%D For absolute control, one can use \type{\doreshapebox} 
%D directly. This macro takes four arguments, that take care 
%D of:
%D 
%D \startopsomming[n,opelkaar]
%D \som \type{\shapebox}
%D \som \type{\shapepenalty}
%D \som \type{\shapekern}
%D \som \type{\shapeskip}
%D \stopopsomming 

%D \macros
%D   {hyphenatedword,
%D    dohyphenateword}
%D
%D The next one is a tricky one. \PLAIN\ \TEX\ provides
%D \type{\showhyphens} for showing macros on the terminal. When
%D preparing a long list of words we decided to show the
%D hyphens, but had to find out that the \PLAIN\ alternative
%D can hardly be used and|/|or adapted to typesetting. The next
%D two macros do the job and a little more.
%D
%D The simple command \type{\hyphenatedword} accepts one
%D argument and gives the hyphenated word. This macro calls for
%D
%D \starttypen
%D \dohyphenateword {n} {pre} {word}
%D \stoptypen
%D
%D The next examples tell more than lots of words:
%D
%D \startbuffer
%D \dohyphenateword{0} {}    {dohyphenatedword}
%D \dohyphenateword{1} {...} {dohyphenatedword}
%D \dohyphenateword{2} {...} {dohyphenatedword}
%D \stopbuffer
%D
%D \typebuffer
%D
%D Here, \type{\hyphenatedword{dohyphenatedword}} is the
%D shorter alternative for the first line.
%D
%D \startvoorbeeld
%D \haalbuffer
%D \stopvoorbeeld
%D
%D These macros are slow but effective and not that hard to
%D program at all.

\def\dohyphenateword#1#2#3%
  {\bgroup
   \setbox0=\hbox
     {\mindermeldingen
      \widowpenalty=0
      \clubpenalty=0
      \setbox0=\vbox
        {\hsize\!!zeropoint \ #3}%
      \ifnum#1>0
        \dorecurse{#1}
          {\setbox2=\hbox
             {\splittopskip=\openstrutheight
              \vsplit0 to \baselineskip}}%
        #2%
      \fi
      \loop
        \setbox2=\hbox
          {\splittopskip=\openstrutheight
           \vsplit0 to \baselineskip}%
        \hbox
          {\unhbox2
           \setbox2=\lastbox
           \vbox
             {\unvbox2
              \setbox2=\lastbox
              \hbox{\unhbox2}}}%
        \ifdim\ht0>\!!zeropoint
      \repeat}%
    \ht0=\ht\strutbox
    \dp0=\dp\strutbox
    \box0
    \egroup}

\def\hyphenatedword%
  {\dohyphenateword{0}{}}

%D \macros
%D   {doboundtext}
%D
%D Sometimes there is not enough room to show the complete
%D (line of) text. In such a situation we can strip of some
%D characters by using \type{\doboundtext}. When the text is
%D wider than the given width, it's split and the third
%D argument is appended. When the text to be checked is packed
%D in a command, we'll have to use \type{\expandafter}.
%D
%D \starttypen
%D \doboundtext{a very, probably to long, text}{3cm}{...}
%D \stoptypen
%D
%D When calculating the room needed, we take the width of the
%D third argument into account, which leads to a bit more
%D complex macro than needed at first sight.

\def\dodoboundtext#1%
  {\setbox0=\hbox{\unhcopy0 #1}%
   \ifdim\wd0>\dimen0
     \let\dodoboundtext=\gobbleoneargument
   \else
     #1\relax
   \fi}

\def\doboundtext#1#2#3%
  {\hbox
     {\setbox0=\hbox{#1}%
      \dimen0=#2\relax
      \ifdim\wd0>\dimen0
        \setbox2=\hbox{#3}%
        \advance\dimen0 by -\wd2
        \setbox0=\hbox{}%
        \processtokens
          {\dodoboundtext}
          {\dodoboundtext}
          {}
          {\space}
          {#1}%
        \box2
      \else
        \box0
      \fi}}

%D \macros
%D   {limitatetext}
%D
%D A bit more beautiful alternative for the previous command is
%D the next one. This command is more robust because we let
%D \TEX\ do most of the job. The previous command works better
%D on text that cannot be hyphenated.
%D
%D \starttypen
%D \limitatetext {text} {width} {sentinel}
%D \stoptypen
%D
%D When no width is given, the whole text comes available. The
%D sentinel is optional. This is about the third version. 

\ifx\fakecompoundhyphen\undefined \let\fakecompoundhyphen\relax \fi

\unexpanded\def\limitatetext%
  {\bgroup
   \dowithnextbox\dolimitatetext\hbox}

\def\dolimitatetext#1#2%
  {\doifelsenothing{#1}
     {\unhbox\nextbox}
     {\fakecompoundhyphen
      \scratchdimen=#1\relax
      \ifdim\wd\nextbox>\scratchdimen
        \setbox\scratchbox=\hbox{ #2}%
        \advance\scratchdimen by -\wd\scratchbox
        \setbox\nextbox=\vbox
          {\hsize=\scratchdimen
           \hfuzz\maxdimen
           \veryraggedright
           \strut\unhbox\nextbox}%
        \setbox\nextbox=\vbox % if omitted: missing brace reported
          {\splittopskip=\openstrutheight
           \setbox\nextbox=\vsplit\nextbox to \ht\strutbox
           \unvbox\nextbox
           \setbox\nextbox=\lastbox
           \global\setbox1=\hbox
             {\unhbox\nextbox\unskip\kern\!!zeropoint\box\scratchbox\unskip}}%
        \unhbox1
      \else
        \unhbox\nextbox
      \fi}%
   \egroup}

%D \macros
%D   {processisolatedwords,
%D    betweenisolatedwords,nothingbetweenisolatedwords}
%D
%D References are often made up of one word or a combination
%D of tightly connected words. The typeset text {\bf
%D chapter~5} is for instance the results of the character
%D sequence:
%D
%D \starttypen
%D The typeset text \in{chapter}[texniques] is for instance
%D \stoptypen
%D
%D When such words are made active in interactive texts, the
%D combination cannot longer be hyphenated. Normally this is no
%D problem, because \TEX\ tries to prevent hyphenation as best
%D as can.
%D
%D Sometimes however we need a few more words to make things
%D clear, like when we want to refer to {\bf \TEX\ by Topic}.
%D The macros that are responsible for typesetting hyperlinks,
%D take care of such sub||sentences by breaking them up in
%D words. Long ago we processed words using the space as a
%D separator, but the more advanced our interactive text became,
%D the more we needed a robust solution. Well, here it is and
%D it called as:
%D
%D \starttypen
%D \processisolatedwords{some words}\someaction
%D \stoptypen
%D
%D The second argument \type{someactions} handles the
%D individual words, like in:
%D
%D \startbuffer
%D \processisolatedwords{some more words}           \ruledhbox \par
%D \processisolatedwords{and some $x + y = z$ math} \ruledhbox \par
%D \processisolatedwords{and a \hbox{$x + y = z$}}  \ruledhbox \par
%D \stopbuffer
%D
%D \typebuffer
%D
%D which let the words turn up as:
%D
%D \startvoorbeeld
%D \haalbuffer
%D \stopvoorbeeld
%D
%D The macro has been made a bit more clever than needed at
%D first sight. This is due to the fact that we don't want to
%D generate more overhead in terms of interactive commands than
%D needed.
%D
%D \startbuffer
%D \processisolatedwords{see this \ruledhskip1em}     \ruledhbox
%D \processisolatedwords{and \ruledhskip1em this one} \ruledhbox
%D \stopbuffer
%D
%D \typebuffer
%D
%D becomes:
%D
%D \startvoorbeeld
%D \startregels
%D \haalbuffer
%D \stopregels
%D \stopvoorbeeld
%D
%D Single word arguments are treated without further
%D processing. This was needed because this command is used in
%D the \type{\goto} command, to which we sometimes pass very
%D strange and|/|or complicated arguments or simply boxes
%D whose dimensions are to be left intact.
%D
%D First we build a \type{\hbox}. This enables us to save the
%D last skip. Next we fill a \type{\vbox} without hyphenating
%D words. After we've tested if there is more than one word, we
%D start processing the individual lines (words). We need some
%D splitting, packing and unpacking to get the spacing and
%D dimensions right.
%D
%D Normally the isolated words are separated by space, but 
%D one can overrule this separator by changing the next macros.
%D
%D When needed, spacing can be suppressed by \type 
%D {\nothingbetweenisolatedwords}.

\newif\ifisolatedwords

\def\betweenisolatedwords%
  {\hskip\currentspaceskip} 

\def\setbetweenisolatedwords#1%
  {\gdef\localbetweenisolatedwords{#1}}

\def\processisolatedwords#1#2%
  {\bgroup
   \fakecompoundhyphen
   \mindermeldingen
   \forgetall
   \global\let\localbetweenisolatedwords\betweenisolatedwords
   \setbox0=\hbox
     {\ignorespaces#1%
      \xdef\isolatedlastskip{\the\lastskip}}%
   \setbox2=\vbox
     {%\hyphenpenalty10000       % this one fails in \url breaking, 
      \lefthyphenmin=\!!maxcard  % but this trick works ok, due to them
      \righthyphenmin=\!!maxcard % total>63, when no hyphenation is done
      \hsize\!!zeropoint
      \unhcopy0}% == #1
   \ifdim\ht0=\ht2
      \isolatedwordsfalse
      #2{\unhcopy0}% == #2{#1}
   \else
     \isolatedwordstrue
     \setbox0=\hbox
       {\ignorespaces
        \loop
          \setbox4=\hbox
            {\splittopskip=\openstrutheight
             \vsplit2 to \baselineskip}%
          \hbox
            {\unhbox4\unskip % recently added 
             \setbox4=\lastbox
             \vbox                  % outer \hbox needed 
               {\unvbox4            % for nested use   
                \setbox4=\lastbox
                \hbox{#2{\hbox{\unhbox4}}}}}% 
        \ifdim\ht2>\!!zeropoint \localbetweenisolatedwords \repeat
        \unskip}%
      \unhbox0\unskip\hskip\isolatedlastskip
    \fi
    \egroup}

%D \macros
%D   {sbox}
%D
%D This is a rather strange command. It grabs some box content
%D and and limits the size to the height and depth of a
%D \type{\strut}. The resulting bottom||alligned box can be used
%D aside other ones, without disturbing the normal baseline
%D distance.
%D
%D \startbuffer
%D \ruledhbox to .5\hsize{\sbox{eerste\par tweede \par derde}}
%D \stopbuffer
%D
%D \typebuffer
%D
%D Shows up as:
%D
%D \startvoorbeeld
%D \vskip3\baselineskip
%D \haalbuffer
%D \stopvoorbeeld
%D
%D Before displaying the result we added some skip, otherwise
%D the first two lines would have ended up in the text. This
%D macro can be useful when building complicated menus, headers
%D and footers and|/|or margin material.

\def\sbox%  in handleiding, voorbeeld \inlinker{xx} \extern..
  {\dowithnextbox
     {\setbox0=\hbox
        {\strut
         \dp\nextbox=0pt
         \lower\strutdepth\box\nextbox}%
      \dp0=\strutdepth
      \ht0=\strutheight
      \box0}%
     \vbox}

%D \macros
%D   {centeredbox, centerednextbox}
%D
%D Here is another strange one. This one offers a sort of overlay
%D with positive or negative offsets. This command can be used
%D in well defined areas where no offset options are available.
%D We first used it when building a button inside the margin
%D footer, where the button should have a horizontal offset and
%D should be centered with respect to the surrounding box. The
%D last of the three examples we show below says:
%D
%D \starttypen
%D \vsize=3cm
%D \hsize=3cm
%D \ruledvbox to \vsize
%D   {\centeredbox height .5cm width -1cm
%D      {\vrule width \hsize height \vsize}}}
%D \stoptypen
%D
%D Here the \type{\ruledvbox} just shows the surrounding box
%D and \type{\vrule} is used to show the centered box.
%D
%D \def\AnExample#1#2%
%D   {\vsize=3cm
%D    \hsize=3cm
%D    \ruledvbox to \vsize
%D      {\centeredbox height #1 width #2
%D         {\color[groen]{\vrule width \hsize height \vsize}}}}
%D
%D \startregelcorrectie
%D \startcombinatie[3*1]
%D   {\AnExample {-1cm}  {.5cm}} {}
%D   {\AnExample {.5cm}  {-1cm}} {}
%D   {\AnExample {-1cm} {-.5cm}} {}
%D \stopcombinatie
%D \stopregelcorrectie
%D
%D This command takes two optional arguments: \type{width} and
%D \type{height}. Observing readers can see that we use \TEX's
%D own scanner for grabbing these arguments: \type{#1#} reads
%D everyting till the next brace and passes it to both rules.
%D The setting of the box dimensions at the end is needed for
%D special cases. The dimensions of the surrounding box are kept
%D intact. This commands handles positive and negative
%D dimensions (which is why we need two boxes with rules).

\def\centeredbox#1#%   height +/-dimen width +/-dimen
  {\bgroup
   \setbox0=\vbox to \vsize
     \bgroup
       \mindermeldingen
       \forgetall
       \setbox0=\hbox{\vrule\!!width \!!zeropoint#1}%
       \setbox2=\vbox{\hrule\!!height\!!zeropoint#1}%
       \advance\vsize by \ht2
       \advance\hsize by \wd0
       \vbox to \vsize
         \bgroup
           \vskip-\ht2
           \vss
           \hbox to \hsize
             \bgroup
               \dowithnextbox
                 {\hskip-\wd0
                  \hss
                  \box\nextbox
                  \hss
             \egroup
           \vss
         \egroup
     \egroup
     \wd0=\hsize
     \ht0=\vsize
     \box0
     \egroup}
   \hbox}

%D For those who don't want to deal with \type {\hsize}
%D and \type {\vsize}, we have: 
%D 
%D \starttypen
%D \centerednextbox width 2bp height 2bp 
%D   {\framed[width=100bp,height=100bp]{}}
%D \stoptypen
%D 
%D Do you see what we call this one \type {next}?

\def\centerednextbox#1#%
  {\bgroup
   \dowithnextbox
     {\hsize\wd\nextbox
      \vsize\ht\nextbox
      \centeredbox#1{\box\nextbox}%
      \egroup}
   \hbox}

%D \macros 
%D   {centerbox}
%D
%D Centering on the available space is done by: 
%D
%D \starttypen
%D \centeredbox <optional specs> {content}
%D \stoptypen
%D
%D When omitted, the current \type {\hsize} and \type 
%D {\vsize} are used. Local dimensions are supported. 

\def\centerbox#1#%   optional height +/-dimen width +/-dimen
  {\bgroup
   \dowithnextbox
     {\setlocalhsize
      \setbox0=\hbox{\vrule\!!width\!!zeropoint#1}%
      \setbox2=\vbox{\hrule\!!height\!!zeropoint#1}%
      \hsize\ifdim\wd0=\!!zeropoint\hsize\else\wd0\fi
      \vsize\ifdim\ht2=\!!zeropoint\vsize\else\ht2\fi
      \vbox to \vsize{\vss\hbox to \hsize{\hss\box\nextbox\hss}\vss}%
      \egroup}%
     \hbox}

%D \macros
%D   {setrigidcolumnhsize,rigidcolumnbalance}
%D
%D These macros are copied from the \TEX book, page~397, and
%D extended by a macro that sets the \type{\hsize}.
%D
%D \starttypen
%D \setrigidcolumnhsize {total width} {distance} {n}
%D \rigidcolumnbalance  {box}
%D \stoptypen
%D
%D Both these macros are for instance used in typesetting
%D footnotes.
%D
%D Men kan het proces van breken enigzins beinvloeden met de 
%D volgende twee switches:

\newif\ifalignrigidcolumns      
\newif\ifstretchrigidcolumns    

%D De eerste switch bepaald het uitlijnen, de tweede rekt de 
%D individuele kolommen op naar \type{\vsize}.

\def\setrigidcolumnhsize#1#2#3%
  {\xdef\savedrigidhsize{\the\hsize}%
   \hsize=#1\relax
   \global\chardef\rigidcolumns=#3\relax
   \scratchdimen=-#2\relax
   \multiply\scratchdimen by #3\relax
   \advance\scratchdimen by #2\relax
   \advance\hsize by \scratchdimen
   \divide\hsize by #3\relax}

\def\rigidcolumnbalance#1%
  {\global\chardef\rigidcolumnbox=#1\relax
   \ifnum\rigidcolumns=1
     \ifinner\ifhmode\box\else\unvbox\fi\else\unvbox\fi\rigidcolumnbox
   \else
     \hbox to \savedrigidhsize % was \hsize
       {\vbadness=10000
        \tabskip\!!zeropoint
        \setbox\rigidcolumnbox=\vbox
          {\unvbox\rigidcolumnbox
           \unpenalty\removelastskip}% get rid of \blank's
        \splittopskip=\openstrutheight
       %\scratchdimen=\ht\rigidcolumnbox      % sensitive for overflow
       %\divide\scratchdimen by \rigidcolumns % therefore we need the hack:
        \scratchdimen=1pt
        \divide\scratchdimen by \rigidcolumns
        \expanded{\scratchdimen=\withoutpt{\the\scratchdimen}\ht\rigidcolumnbox}%
        \advance\scratchdimen by \ht\strutbox
        \valign{##\vfill\cr\dorigidcolumnsplits}}%
   \fi}

\def\dorigidcolumnsplits%
  {\ifnum\rigidcolumns>0
     \setbox\scratchbox=\vsplit\rigidcolumnbox to \scratchdimen
     \hbox to \hsize
       \bgroup
         \ifalignrigidcolumns
           \vbox to \ifstretchrigidcolumns\vsize\else\scratchdimen\fi
             {\unvbox\scratchbox}%
         \else
           \vbox{\unvbox\scratchbox}%
         \fi
         \hss
       \egroup
     \doglobal\decrement\rigidcolumns
     \cr
     \ifnum\rigidcolumns>0\noalign{\hfil}\fi
     \expandafter\dorigidcolumnsplits
   \fi}

%D \macros
%D   {startvboxtohbox,stopvboxtohbox,convertvboxtohbox}
%D
%D Here is another of Knuth's dirty tricks, as presented on
%D pages 398 and 399 of the \TEX book. These macros can be used
%D like:
%D
%D \starttypen
%D \vbox
%D   \bgroup
%D     \startvboxtohbox ... \stopvboxtohbox
%D     \startvboxtohbox ... \stopvboxtohbox
%D     \startvboxtohbox ... \stopvboxtohbox
%D   \egroup
%D
%D \vbox
%D   \bgroup
%D     \converthboxtovbox
%D   \egroup
%D \stoptypen
%D
%D These macros are used in reformatting footnotes, so they do
%D what they're meant for.

\def\setvboxtohbox%
  {\bgroup
   \ifdim\baselineskip<16pt \relax
     \dimen0=\baselineskip
     \multiply\dimen0 by 1024
   \else
     \message{cropping \baselineskip to 16pt}%
     \dimen0=\maxdimen
   \fi
   \divide\dimen0 by \hsize
   \multiply\dimen0 by 64
   \xdef\vboxtohboxfactor{\expandafter\withoutpt\the\dimen0}%
   \egroup}

\def\startvboxtohbox%
 {\bgroup
  \setvboxtohbox
  \setbox0=\hbox\bgroup}

\def\stopvboxtohbox%
  {\egroup
   \dp0=\!!zeropoint
   \ht0=\vboxtohboxfactor\wd0
   \box0
   \egroup}

% % to be done: start halfway a line combined with one line 
% % extra to start with (skip) and one line less than counted.
% 
% \def\stopvboxtohbox%
%   {\egroup
%    \setbox2=\vbox
%      {\forgetall\unhcopy0\par\xdef\globalvhlines{\the\prevgraf}}%
%    \setbox2=\vbox
%      {\unvbox2
%       \setbox2=\lastbox
%       \setbox2=\hbox{\unhbox2}%
%       \xdef\globalvhwidth{\the\wd2}}%
%    \decrement\globalvhlines
%    \dimen0=\globalvhwidth
%    \dimen0=\vboxtohboxfactor\dimen0
%    \advance\dimen0 by \globalvhlines\lineheight
%    \dp0=\!!zeropoint
%    \ht0=\dimen0
%   %\writestatus{guessed size}
%   %  {w:\the\wd0\space\space
%   %   b:\the\baselineskip\space
%   %   l:\globalvhlines\space
%   %   e:\globalvhwidth\space
%   %   h:\the\dimen0}%
%    \box0
%    \egroup}

\def\convertvboxtohbox%
  {\setvboxtohbox
   \makehboxofhboxes
   \setbox0=\hbox{\unhbox0 \removehboxes}%
   \noindent\unhbox0\par}

\def\makehboxofhboxes%
  {\setbox0=\hbox{}%
   \loop                  % \doloop { .. \exitloop .. } 
     \setbox2=\lastbox
     \ifhbox2
       \setbox0=\hbox{\box2\unhbox0}%
   \repeat}

\def\removehboxes%
  {\setbox0=\lastbox
   \ifhbox0
     {\removehboxes}%
     \unhbox0
   \fi}

%D \macros 
%D   {unhhbox}
%D
%D The next macro is used in typesetting inline headings. 
%D Let's first look at the macro and then show an example. 

\newbox   \unhhedbox
\newbox   \hhbox
\newdimen \lasthhboxwidth
\newskip  \hhboxindent

\def\unhhbox#1\with#2%
  {\bgroup
   \mindermeldingen
   \forgetall
   \setbox\unhhedbox=\vbox{\hskip\hhboxindent\strut\unhbox#1}% => \hsize 
   \doloop
     {\setbox\hhbox=\vsplit\unhhedbox to \lineheight 
      \ifvoid\unhhedbox
        \setbox\hhbox=\hbox{\strut\hboxofvbox\hhbox}%
      \fi
      \ht\hhbox=\ht\strutbox
      \dp\hhbox=\dp\strutbox
      \ifdim\hhboxindent=\!!zeropoint\else
        \setbox\hhbox=\hbox{\hskip-\hhboxindent\box\hhbox}%
        \hhboxindent=\!!zeropoint
      \fi
      \global\lasthhboxwidth=\wd\hhbox
      #2\relax
      \ifvoid\unhhedbox
        \exitloop
      \else
        \hskip\!!zeropoint \!!plus \!!zeropoint 
      \fi}%
   \egroup}

\def\dohboxofvbox%
  {\setbox0=\vbox{\unvbox\scratchcounter\global\setbox1=\lastbox}%
   \unhbox1
   \egroup}

\def\hboxofvbox%
  {\bgroup
   \afterassignment\dohboxofvbox
   \scratchcounter=}

%D This macro can be used to break a paragraph apart and treat
%D each line seperately, for instance, making it clickable. The
%D main complication is that we want to be able to continue the
%D paragraph, something that's needed in the in line section
%D headers. 
%D 
%D \startbuffer
%D \setbox0=\hbox{\input tufte \relax}
%D \setbox2=\hbox{\input knuth \relax}
%D \unhhbox0\with{\ruledhbox{\box\hhbox}}
%D \hskip1em plus 1em minus 1em 
%D \hhboxindent=\lasthhboxwidth 
%D \advance\hhboxindent by \lastskip
%D \unhhbox2\with{\ruledhbox{\box\hhbox}}
%D \stopbuffer
%D 
%D \haalbuffer
%D 
%D This piece of text was typeset by saying:
%D 
%D \typebuffer
%D 
%D Not that nice a definition, but effective. Note the stretch 
%D we've build in the line that connects the two paragraphs. 

%D \macros
%D   {doifcontent}
%D
%D When processing depends on the availability of content, one
%D can give the next macro a try.
%D
%D \starttypen
%D \doifcontent{pre content}{post content}{no content}\somebox
%D \stoptypen
%D
%D Where \type{\somebox} is either a \type{\hbox} or
%D \type{\vbox}. If the dimension of this box suggest some
%D content, the resulting box is unboxed and surrounded by the
%D first two arguments, else the third arguments is executed.

\def\doifcontent#1#2#3%
  {\dowithnextbox
     {\ifhbox\nextbox
        \ifdim\wd\nextbox>\!!zeropoint
          #1\unhbox\nextbox#2\relax
        \else
          #3\relax
        \fi
      \else
        \ifdim\ht\nextbox>\!!zeropoint
          #1\unvbox\nextbox#2\relax
        \else
          #3\relax
        \fi
      \fi}}

%D So when we say:
%D
%D \startbuffer
%D \doifcontent{[}{]}{}\hbox{content sensitive typesetting}
%D
%D \doifcontent{}{\pagina}{}\vbox{content sensitive typesetting}
%D
%D \doifcontent{}{}{\message{Didn't you forget something?}}\hbox{}
%D \stopbuffer
%D
%D \typebuffer
%D
%D We get:
%D
%D \haalbuffer
%D
%D Where the last call of course does not show up in this
%D document, but definitely generates a confusing message.

%D \macros 
%D   {processboxes}
%D
%D The next macro gobble boxes and is for instance used for 
%D overlays. First we show the general handler.

\newbox\processbox

\def\processboxes#1% 
  {\bgroup
   \def\doprocessbox{#1}% #1 can be redefined halfway
   \setbox\processbox=\box\voidb@x
   \afterassignment\dogetprocessbox\let\next=}

\def\endprocessboxes%
  {\ifhmode\unskip\fi
   \box\processbox
   \next
   \egroup}

\def\dogetprocessbox%
  {\ifx\next\bgroup
     \expandafter\dodogetprocessbox
   \else
     \expandafter\endprocessboxes
   \fi}

\def\dodogetprocessbox%
  {\dowithnextbox
     {\ifhmode\unskip\fi\doprocessbox % takes \nextbox makes \processbox 
      \afterassignment\dogetprocessbox\let\next=}
   \hbox\bgroup}

%D \macros 
%D   {startoverlay}
%D 
%D We can overlay boxes by saying:
%D 
%D \startbuffer
%D \startoverlay
%D   {\omlijnd{hans}}
%D   {\omlijnd[breedte=3cm]{ton}}
%D   {\omlijnd[hoogte=2cm]{oeps}}
%D \stopoverlay
%D \stopbuffer
%D 
%D \typebuffer
%D
%D shows up as: 
%D
%D \leavevmode\haalbuffer

% \def\dooverlaybox%
%   {\ifhmode\unskip\fi 
%    \ifdim\ht\nextbox>\ht\processbox
%      \setbox\processbox=\vbox to \ht\nextbox
%        {\vss\box\processbox\vss}%
%    \else
%      \setbox\nextbox=\vbox to \ht\processbox
%        {\vss\box\nextbox\vss}%
%    \fi
%    \scratchdimen=\wd
%      \ifdim\wd\nextbox>\wd\processbox 
%        \nextbox
%      \else
%        \processbox
%      \fi        
%    \setbox\processbox=\hbox to \scratchdimen
%      {\hbox to \scratchdimen{\hss\box\processbox\hss}%
%       \hskip-\scratchdimen
%       \hbox to \scratchdimen{\hss\box\nextbox\hss}}}
%  
% \def\startoverlay%
%   {\bgroup
%    \let\stopoverlay\egroup
%    \processboxes\dooverlaybox}

\def\dooverlaybox%
  {\ifhmode\unskip\fi
   \scratchdimen=\dp
     \ifdim\dp\nextbox>\dp\processbox
       \nextbox
     \else
       \processbox
     \fi
   \ifdim\ht\nextbox>\ht\processbox
     \setbox\processbox=\vbox to \ht\nextbox
       {\dp\processbox=\!!zeropoint\vss\box\processbox\vss}%
   \else
     \setbox\nextbox=\vbox to \ht\processbox
       {\dp\nextbox=\!!zeropoint\vss\box\nextbox\vss}%
   \fi
   \dp\nextbox=\scratchdimen
   \dp\processbox=\scratchdimen
   \scratchdimen=\wd
     \ifdim\wd\nextbox>\wd\processbox
       \nextbox
     \else
       \processbox
     \fi
   \setbox\processbox=\hbox to \scratchdimen
     {\hbox to \scratchdimen{\hss\box\processbox\hss}%
      \hskip-\scratchdimen
      \hbox to \scratchdimen{\hss\box\nextbox\hss}}}

\def\startoverlay%
  {\bgroup
   \let\stopoverlay\egroup
   \processboxes\dooverlaybox}

% %D \macros 
% %D   {starthspread}
% %D 
% %D In a similar way we can build a horizontal box, spread 
% %D over the available width.
% %D 
% %D \startbuffer
% %D \starthspread
% %D   {hans}
% %D   {ton}
% %D   {oeps}
% %D \stophspread
% %D 
% %D \stopbuffer
% %D 
% %D \typebuffer
% %D
% %D shows up as: 
% %D
% %D \leavevmode\haalbuffer
% 
% \def\dohspread%
%   {\box\nextbox
%    \def\dohspread{\hfil\box\nextbox}}
% 
% \def\starthspread%
%   {\hbox to \hsize \bgroup
%    \let\stophspread\egroup
%    \processboxes\dohspread}

%D \macros
%D   {fakebox}
%D 
%D The next macro is a rather silly one, but saves space. 
%D
%D \starttypen
%D \hbox{\fakebox0} 
%D \stoptypen
%D
%D returns an empty box with the dimensions of the box 
%D specified, here being zero. 

\def\dofakebox%
  {\setbox\scratchbox=\null
   \wd\scratchbox=\wd\scratchcounter
   \ht\scratchbox=\ht\scratchcounter
   \dp\scratchbox=\dp\scratchcounter
   \ifhbox\scratchcounter\hbox\else\vbox\fi{\box\scratchbox}%
   \egroup}

\def\fakebox%
  {\bgroup
   \afterassignment\dofakebox\scratchcounter}

%D \macros 
%D   {lbox,rbox,cbox,tbox,bbox}
%D
%D Here are some convenient alternative box types:
%D
%D \starttypen
%D \lbox{text ...}
%D \cbox{text ...}
%D \rbox{text ...}
%D \stoptypen
%D 
%D Are similar to \type {\vbox}, which means that they also 
%D accept something like \type{to 3cm}, but align to the left, 
%D middle and right. These box types can be used to typeset 
%D paragraphs. 

\def\lbox{\lrcbox\raggedleft}
\def\cbox{\lrcbox\raggedcenter}
\def\rbox{\lrcbox\raggedright}

\def\lrcbox#1#2#%
  {\vbox#2\bgroup
   \let\\=\endgraf
   \forgetall#1\let\next=}

%D The alternatives \type {\tbox} and \type {\bbox} can be used 
%D to properly allign boxes, like in:
%D 
%D \startbuffer 
%D \starttabel[|||]
%D \HL
%D \VL \tbox{\externfiguur[koe][hoogte=3cm,kader=aan]} \VL top aligned    \VL\SR
%D \HL
%D \VL \bbox{\externfiguur[koe][hoogte=3cm,kader=aan]} \VL bottom aligned \VL\SR
%D \HL
%D \stoptabel
%D \stopbuffer
%D 
%D \typebuffer
%D 
%D The positioning depends on the strut settings: 
%D 
%D \haalbuffer

\def\tbox{\tbbox\ht\dp}
\def\bbox{\tbbox\dp\ht}

\def\tbbox#1#2%  
  {\hbox\bgroup
   \dowithnextbox 
     {\scratchdimen=\ht\nextbox   
      \advance\scratchdimen\dp\nextbox
      \advance\scratchdimen-#1\strutbox
      #1\nextbox=#1\strutbox
      #2\nextbox=\scratchdimen
      \setbox\nextbox=\hbox
        {\lower\dp\nextbox\box\nextbox}%
      #1\nextbox=#1\strutbox
      #2\nextbox=\scratchdimen
      \box\nextbox
      \egroup}
     \hbox}

%D \macros
%D   {boxofsize}
%D
%D Sometimes we need to construct a box with a height or 
%D width made up of several dimensions. Instead of cumbersome 
%D additions, we can use: 
%D
%D \starttypen
%D \boxofsize \vbox 10cm 3cm -5cm {the text to be typeset}
%D \stoptypen
%D
%D This example demonstrates that one can use positive and 
%D negative values. Dimension registers are also accepted.

\newdimen\sizeofbox

\def\boxofsize#1%
  {\bgroup
   \sizeofbox\!!zeropoint
   \scratchdimen\!!zeropoint
   \def\docommando%
     {\advance\sizeofbox\scratchdimen
      \futurelet\next\dodocommando}%
   \def\dodocommando%
     {\ifx\next\bgroup
        \expanded{\egroup#1 to \the\sizeofbox}%
      \else
        \@EA\afterassignment\@EA\docommando\@EA\scratchdimen
      \fi}%
   \docommando}


%D Some new, still undocumented features:

% limitatetext -> beter {text} als laatste !!
%
% \limitvbox
% \limithbox

\def\limitatelines#1#2% size sentinel
  {\dowithnextbox
     {\dimen0=#1\hsize
      \ifdim\wd\nextbox>\dimen0
        \setbox\nextbox=\hbox
          {\advance\dimen0 by -.1\hsize
           \limitatetext{\unhbox\nextbox}{\dimen0}{\nobreak#2}}%
      \fi
      \unhbox\nextbox}
     \hbox}

\def\fittoptobaselinegrid% weg hier 
  {\dowithnextbox
     {\bgroup
      \par
      \dimen0=\ht\nextbox
      \ht\nextbox=\ht\strutbox
      \dp\nextbox=\dp\strutbox
      \hbox{\box\nextbox}
      \prevdepth\dp\strutbox
      \doloop
        {\advance\dimen0 by -\lineheight
         \ifdim\dimen0<\!!zeropoint
           \exitloop
         \else
           \nobreak
           \hbox{\strut}
         \fi}
      \egroup}
     \vbox}

%D Some more undocumented macros (used in m-chart). 

\newif\iftraceboxplacement % \traceboxplacementtrue

\def\boxcursor%
  {\iftraceboxplacement
     \bgroup
     \setbox0=\hbox
       {\hskip-1pt\vrule\!!width2pt\!!height2pt\!!depth2pt}%
     \wd0=\!!zeropoint\ht0=\!!zeropoint\dp0=\!!zeropoint\box0
     \egroup
   \else
     \hbox
       {\vrule\!!width\!!zeropoint\!!height\!!zeropoint\!!depth\!!zeropoint}%
   \fi}

\def\placedbox%
  {\iftraceboxplacement\ruledhbox\else\hbox\fi}

\newdimen\boxoffset

\def\rightbox#1%
  {\hbox
     {\setbox0=\placedbox{#1}%
      \dimen0=.5\ht0\advance\dimen0 by -.5\dp0
      \boxcursor\hskip\boxoffset\lower\dimen0\box0}}

\def\leftbox#1%
  {\hbox
     {\setbox0=\placedbox{#1}%
      \dimen0=.5\ht0\advance\dimen0 by -.5\dp0
      \boxcursor\hskip-\wd0\hskip-\boxoffset\lower\dimen0\box0}}

\def\topbox#1%
  {\hbox
     {\setbox0=\placedbox{#1}%
      \dimen0=\boxoffset\advance\dimen0 by \dp0
      \boxcursor\hskip-.5\wd0\raise\dimen0\box0}}

\def\bottombox#1%
  {\hbox
     {\setbox0=\placedbox{#1}%
      \dimen0=\boxoffset\advance\dimen0 by \ht0
      \boxcursor\hskip-.5\wd0\lower\dimen0\box0}}

\def\lefttopbox#1%
  {\hbox
     {\setbox0=\placedbox{#1}%
      \dimen0=\boxoffset\advance\dimen0 by \dp0
      \advance\boxoffset\wd0
      \boxcursor\hskip-\boxoffset\raise\dimen0\box0}}

\def\righttopbox#1%
  {\hbox
     {\setbox0=\placedbox{#1}%
      \dimen0=\boxoffset\advance\dimen0 by \dp0
      \boxcursor\hskip\boxoffset\raise\dimen0\box0}}

\def\leftbottombox#1%
  {\hbox
     {\setbox0=\placedbox{#1}%
      \dimen0=\boxoffset\advance\dimen0 by \ht0
      \advance\boxoffset\wd0
      \boxcursor\hskip-\boxoffset\lower\dimen0\box0}}

\def\rightbottombox#1%
  {\hbox
     {\setbox0=\placedbox{#1}%
      \dimen0=\boxoffset\advance\dimen0 by \ht0
      \boxcursor\hskip\boxoffset\lower\dimen0\box0}}

\let\topleftbox    \lefttopbox
\let\toprightbox   \righttopbox
\let\bottomleftbox \leftbottombox
\let\bottomrightbox\rightbottombox

\def\middlebox#1%
  {\hbox{\setbox0=\placedbox{#1}\boxoffset=-.5\wd0\rightbox{\box0}}}

%D \macros 
%D   {removedepth, obeydepth}
%D
%D While \type {\removedepth} removes the preceding depth, 
%D \type {\obeydepth} makes sure we have depth. Both macros 
%D leave the \type {\prevdepth} untouched. 

\def\removedepth%
  {\ifvmode \ifdim\prevdepth>\!!zeropoint \kern-\prevdepth \fi \fi}

\def\obeydepth%
  {\par \removedepth \ifvmode \kern\dp\strutbox \fi}

% maybe some day we need this
%
% \def\appendvbox#1%  % uses \box8
%   {\bgroup
%    \ifdim\prevdepth<\!!zeropoint
%      \ifdim\pagetotal=\!!zeropoint
%        \setbox8=\vtop{\unvcopy#1}%
%        \hrule\c!!height\!!zeropoint
%        \kern-\ht8
%        \box#1\relax
%      \else
%        \box#1\relax
%      \fi
%    \else
%      \dimen0=\prevdepth
%      \hrule\c!!height\!!zeropoint
%      \setbox8=\vtop{\unvcopy#1}%
%      \dimen2=\baselineskip
%      \advance\dimen2 by -\dimen0
%      \advance\dimen2 by -\ht8
%      \kern\dimen2
%      \box#1\relax
%    \fi
%    \egroup}

%D Also new:
%D
%D \startbuffer
%D \normbox[1cm][bba]{m}  % b(efore) a(fter) v(box) s(trut) f(rame)
%D \normbox[1cm][bba]{m}
%D \normbox[1cm][bba]{m}
%D \stopbuffer
%D
%D \typebuffer
%D \haalbuffer
% 
% \def\dodonormbox#1#2#3#4#5#6#7%
%   {\doifnumberelse{#1}
%      {\dimen0=#1}{\setbox0=#3{#1}\dimen0=#50}% 
%    \doifinstringelse{f}{#2}
%      {\let\next#4}{\let\next#3}%
%    \next to \dimen0
%      {\counttoken b\in#2\to\!!counta\dorecurse{\!!counta}{#6}#6%
%       #7\nextbox                   
%       \counttoken a\in#2\to\!!counta\dorecurse{\!!counta}{#6}#6}}
% 
% \def\donormbox[#1][#2]%
%   {\bgroup
%    \doifinstringelse{v}{#2}
%      {\let\next\vbox}
%      {\let\next\hbox}%
%    \dowithnextbox
%      {\ifvbox\nextbox
%         \let\\=\par
%         \dodonormbox{#1}{#2}\vbox\ruledvbox\ht\vfil\unvbox
%       \else
%         \let\\=\space
%         \dodonormbox{#1}{#2}\hbox\ruledhbox\wd\hfil\unhbox
%       \fi
%       \egroup}%
%    \next}
% 
% \def\normbox%
%   {\dodoubleempty\donormbox}

\protect

\endinput
