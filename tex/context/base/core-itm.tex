%D \module
%D   [       file=core-itm, % updated 
%D        version=1997.03.31,
%D          title=\CONTEXT\ Core Macros,
%D       subtitle=itemgroups,
%D         author=Hans Hagen,
%D           date=\currentdate,
%D      copyright={PRAGMA / Hans Hagen \& Ton Otten}]
%C
%C This module is part of the \CONTEXT\ macro||package and is
%C therefore copyrighted by \PRAGMA. See mreadme.pdf for
%C details.

% new: text + lefttext=(,righttext=)
%      start=     

\writestatus{loading}{Context Core Macros / itemgroup}

\startmessages  dutch  library: layouts
      9: momenteel maximaal -- niveaus in opsommingen
\stopmessages

\startmessages  english  library: layouts
      9: currently no more than -- levels in itemizations
\stopmessages

\startmessages  german  library: layouts
      9: z.Z. nicht mehr als -- Niveaus in Posten
\stopmessages

\startmessages  czech  library: layouts
      9: aktualne ne vice nez -- urovne/urovni vyctu
\stopmessages

\startmessages  italian  library: layouts
      9: attualmente non pi� di -- livelli di elencazione
\stopmessages

\startmessages  norwegian  library: layouts
      9: for �yeblikket maksimalt -- niv�er i opplisting
\stopmessages

\startmessages  romanian  library: layouts
      9: acum nu se supota mai mult de -- nivele de adancime la iteratii
\stopmessages

\unprotect

% - instellingen in macro
% - [0] voor start op 0
% - start=2

\newif\ifsubitem        \subitemfalse
\newif\ifsymbolitem     \symbolitemfalse
\newif\ifheaditem       \headitemfalse
\newif\ifitemintro      \itemintrofalse
\newif\ifautoitemintro  \autoitemintrofalse
\newif\ifoptimizeitems  \optimizeitemstrue
\newif\ifpackeditems    \packeditemsfalse
\newif\iffirstlist      \firstlistfalse
\newif\ifparagraphitems \paragraphitemsfalse
\newif\iftextitems      \textitemsfalse

\newcounter\noflists
\newcounter\itemlevel
\newcounter\itemcolumndepth
\newcounter\maxitemlevel

\definetwopasslist\s!list

\let\currentitemgroup\empty 

\def\unknownitemreference{0} \let\itemreferences\unknownitemreference

% #1=level #2=parameter 

\def\getitemparameter #1#2{\csname\??op\currentitemgroup#1#2\endcsname}
\def\setitemparameter #1#2{\@EA\def\csname\??op\currentitemgroup#1#2\endcsname}
\def\letitemparameter #1#2{\@EA\let\csname\??op\currentitemgroup#1#2\endcsname}

\def\doitemattributes   #1{\doattributes{\??op\currentitemgroup#1}}

\def\@@globalitemsymbol #1{\??op\currentitemgroup\c!symbool\s!global#1}
\def\@@localitemsymbol  #1{\??op\currentitemgroup\c!symbool\s!local #1}
\def\@@currentitemsymbol#1{\??op\currentitemgroup\c!symbool         #1}

\def\@@itemcounter{\s!itemcount\currentitemgroup}

\def\doitembreak#1{\iftextitems\else\dosomebreak#1\fi}

\def\dolistreference
  {\immediatewriteutilitycommand
     {\twopassentry%
        {\s!list}%
        {\currentlist}%
        {\currentlist:\noflistelements}}}

\def\initializeitemgroupslevel#1%
  {\ifundefined{\@@globalitemsymbol{#1}}%
     \edef\itemreferences{\itemreferences,#1}%
     \makecounter{\@@itemcounter#1}%
     \setevalue{\@@globalitemsymbol{#1}}{#1}%
   \fi}

\def\initializeitemgrouplevel#1% safeguard 
  {\ifundefined{\??op\currentitemgroup#1\c!breedte}%
     \doinitializeitemgrouplevel{#1}%
   \fi}

\def\doinitializeitemgrouplevel#1%
  {\copyparameters
     [\??op\currentitemgroup#1][\??oo]
     [\c!breedte,\c!factor,\c!afstand,\c!uitlijnen,
      \c!letter,\c!marletter,\c!symletter,\c!kopletter,
      \c!kleur,\c!markleur,\c!symkleur,\c!kopkleur,
      \c!kopvoor,\c!kopna,\c!voor,\c!tussen,\c!na,
      \c!afsluiter,\c!plaatsafsluiter,\c!inspringen,
      \c!n,\c!binnen,\c!symbool,\c!marge,\c!items,
      \c!start,\c!linkertekst,\c!rechtertekst]}

\def\setupitemgroups
  {\dosingleargument\dosetupitemgroups}

\def\dosetupitemgroups[#1]% still undocumented
  {\getparameters[\??oo][\c!niveaus=4,#1]%
   % will change (remove) 
   \ifnum\@@ooniveaus>\maxitemlevel
     \edef\maxitemlevel{\@@ooniveaus}%
     \dorecurse\maxitemlevel{\initializeitemgroupslevel\recurselevel}%
   \fi}

\def\doitemreference#1,#2,#3\\%
  {\ifnum\itemlevel>#1
     \ifnum#1>0 \tempsymbol \fi
     \getvalue{\@@currentitemsymbol{#2}}%
     \doitemreference#2,#3\\%
   \fi}

\def\itemreference
  {\expandafter\doitemreference\itemreferences,,\\}

\def\packitems
  {\ifnum\itemlevel=0 \else \packeditemstrue \fi}

\def\dosetupitemgroupvariable[#1]% [#2]%  niveau instellingen
  {\doifelsenothing{#1}
     {\getparameters[\??op\currentitemgroup\itemlevel]}% [#2]}%
     {\getparameters[\??op\currentitemgroup#1]}}% [#2]}}

\newconditional\inlinelistitem \setfalse\inlinelistitem

\def\dododosetupitemgroupconstant[#1][#2#3#4]% * permits [2]
  {\processaction
     [#2#3#4]
     [   \v!opelkaar*=>\packitems,
            \v!intro*=>\itemintrotrue,
        \v!autointro*=>\autoitemintrotrue,
             \v!ruim*=>\setitemparameter{#1}\c!factor{1},
        #2#3*\v!ruim*=>\setitemparameter{#1}\c!factor{#2#3},
          #2*\v!ruim*=>\setitemparameter{#1}\c!factor{#2},
            \v!tekst*=>\textitemstrue
                       \settrue\inlinelistitem 
                       \dosetuppackeditemgroup{#1}%
                       \packitems,
         \v!kolommen*=>\packitems,
            \v!marge*=>\setitemparameter{#1}\c!breedte{-2em}, % signal
          \v!inmarge*=>\setitemparameter{#1}\c!breedte{-2em}, % signal
          \v!opmarge*=>\doifnot{#1}{1}{\setitemparameter{#1}\c!breedte{0em}}, % signal
          \v!intekst*=>\settrue\inlinelistitem, % new
              \v!los*=>\optimizeitemsfalse,
           \v!alinea*=>\paragraphitemstrue
                       \packitems,
      \v!aansluitend*=>\dosetuppackeditemgroup{#1}%
                       \packitems,
        \v!aanelkaar*=>\setitemparameter{#1}\c!factor{-1},
   #2#3*\v!aanelkaar*=>\setitemparameter{#1}\c!factor{-#2#3},
     #2*\v!aanelkaar*=>\setitemparameter{#1}\c!factor{-#2},
        \v!afsluiter*=>\setitemparameter{#1}\c!plaatsafsluiter\v!ja,
        \v!vanelkaar*=>\packeditemsfalse,
        \v!standaard*=>\dosetupstandarditemgroup{#1}]}

\def\dosetupstandarditemgroup#1%
  {\getparameters
     [\??op\currentitemgroup#1]
     [\c!breedte=1.5em,
      \c!factor=0,
      \c!afstand=.5em,
      \c!kopvoor=,
      \c!kopna=\blanko,
      \c!voor=\blanko,
      \c!tussen=\blanko,
      \c!na=\blanko,
      \c!binnen=]}

\def\dosetuppackeditemgroup#1%
  {\letitemparameter{#1}\c!kopvoor\empty
   \letitemparameter{#1}\c!kopna  \empty
   \letitemparameter{#1}\c!voor   \empty
   \letitemparameter{#1}\c!na     \empty
   \letitemparameter{#1}\c!tussen \empty}

\def\dosetupitemgroupconstant[#1][#2]%
  {\def\dodosetupitemgroupconstant##1%
     {\dododosetupitemgroupconstant[#1][##1*]}%
   \processcommacommand[#2]\dodosetupitemgroupconstant} % expansion of #2 is handy for xml

\def\dododododosetupitemgroup[#1][#2]%
  {\doifassignmentelse{#2}
     {\dosetupitemgroupvariable[#1][#2]}
     {\setitemparameter{#1}\empty{\dosetupitemgroupconstant[#1][#2]}}}%

\def\dodododosetupitemgroup[#1][#2]%
  {\ConvertToConstant\doifnot{#2}{}
     {\doifelse{#1}\v!elk
        {\dorecurse\maxitemlevel{\ExpandFirstAfter\dododododosetupitemgroup[\recurselevel][#2]}}
        {\ExpandFirstAfter\dododododosetupitemgroup[#1][#2]}}}

\def\dododosetupitemgroup[#1][#2]%
  {\ConvertToConstant\doifelse{#2}{}
     {\ifcase\itemlevel\relax
        \dodododosetupitemgroup[\v!elk][#1]%
      \else
        \dodododosetupitemgroup[\itemlevel][#1]%
      \fi}
     {\doifelsenothing{#1}
        {\dodododosetupitemgroup[\itemlevel][#2]}
        {\dodododosetupitemgroup[#1][#2]}}}

\def\dodosetupitemgroup[#1][#2][#3][#4]%
  {\pushmacro\currentitemgroup
   \def\currentitemgroup{#1}%
   \dododosetupitemgroup[#2][#3]%
   \ConvertToConstant\doifnot{#4}{}  % anders wordt #2 overruled
     {\dododosetupitemgroup[#2][#4]}%
   \popmacro\currentitemgroup}

\def\dosetupitemgroup[#1][#2][#3][#4]%
  {\def\docommando##1{\dodosetupitemgroup[##1][#2][#3][#4]}%
   \processcommalist[#1]\docommando}

\def\setupitemgroup
  {\doquadrupleempty\dosetupitemgroup}

\def\doadvanceitem
  {\ifsubitem\else\ifsymbolitem\else
     \pluscounter{\@@itemcounter\itemlevel}%
   \fi\fi}

% \def\setitemlevel#1%
%   {\ifnum\itemlevel>0\relax
%      \firstlisttrue
%      \doifnotinset\v!verder{#1}
%        {\resetcounter{\@@itemcounter\itemlevel}}%
%      \def\tempnumber%
%        {\countervalue{\@@itemcounter\itemlevel}}%
%      \doifelse{\getitemparameter\itemlevel\c!plaatsafsluiter}\v!ja
%        {\def\tempsymbol{\getitemparameter\itemlevel\c!afsluiter}}
%        {\let\tempsymbol\empty}%
%    \fi} 

\def\setitemlevel#1%
  {\ifnum\itemlevel>0\relax
     \firstlisttrue
     \doifnotinset\v!verder{#1}
       {\doifinset{0}{#1}{\setitemparameter\itemlevel\c!start{0}}%
        \doifsomething{\getitemparameter\itemlevel\c!start}
          {\setcounter{\@@itemcounter\itemlevel}{\getitemparameter\itemlevel\c!start}%
           \minuscounter{\@@itemcounter\itemlevel}%
           \letitemparameter\itemlevel\c!start\empty}}%
     \def\tempnumber%
       {\countervalue{\@@itemcounter\itemlevel}}%
     \doifelse{\getitemparameter\itemlevel\c!plaatsafsluiter}\v!ja
       {\def\tempsymbol{\getitemparameter\itemlevel\c!afsluiter}}
       {\let\tempsymbol\empty}%
   \fi} 

% PAS OP: ook 'opelkaar' en zo worden getest, nog eens afvangen!

\def\unknownitemsymbol{?}

\def\setitemmark#1% % en pas op: resets \docommando
  {\doifelsenothing{#1}
     {\edef\currentitemsymbol{\itemlevel}}
     {\edef\currentitemsymbol{#1}}%
   \doifsymboldefinedelse\currentitemsymbol
     {\setxvalue{\@@globalitemsymbol\itemlevel}{\currentitemsymbol}%
      \setgvalue{\@@localitemsymbol \itemlevel}{\unknownitemsymbol}%
      \def\listitem{\symbol[\currentitemsymbol]}%
      \let\docommando\gobbleoneargument}
     {\doifconversiondefinedelse\currentitemsymbol
        {\setxvalue{\@@globalitemsymbol\itemlevel}{\currentitemsymbol}%
         \setgvalue{\@@localitemsymbol \itemlevel}%
           {\convertnumber{\currentitemsymbol}{\countervalue{\@@itemcounter\itemlevel}}}%
         \iftextitems
           \doifsomething{\getitemparameter\itemlevel\c!linkertekst}
             {\let\tempsymbol\empty}%
         \fi
         \def\listitem%
           {\getitemparameter\itemlevel
              {\iftextitems\c!linkertekst\else\c!links\fi}%
            \getvalue{\@@localitemsymbol\itemlevel}\tempsymbol
            \getitemparameter\itemlevel
              {\iftextitems\c!rechtertekst\else\c!rechts\fi}}%
         \let\docommando\gobbleoneargument}
       {\let\listitem\empty}}}

\def\calculatelistwidth#1#2%
  {#2=\getitemparameter{#1}\c!afstand\relax
   \ifnum\getitemparameter{#1}\c!factor>0
     \ifdim#2=\zeropoint #2=.5em\fi
   \fi
   \multiply#2 \getitemparameter{#1}\c!factor
   \advance #2 \getitemparameter{#1}\c!breedte\relax}

% The next conditionals deal with \item \startitemgroup. It
% looks like a hack to skip back, but that way we preserve
% the indentation and bullet placement. It's a rather
% untested feature.

\newconditional\concatnextitem     \setfalse\concatnextitem
\newconditional\autoconcatnextitem \settrue \autoconcatnextitem
\newsignal     \itemsignal

\def\startitemgroup
  {\dotripleempty\dostartitemgroup}

\def\dostartitemgroup[#1][#2][#3]%
  {\bgroup
   \def\currentitemgroup{#1}%
   \ifthirdargument
     \dodostartitemgroup[#2][#3]%
   \else
     \doifassignmentelse{#2}
       {\dodostartitemgroup[][#2]}
       {\dodostartitemgroup[#2][]}%
   \fi}

\def\dodostartitemgroup[#1][#2]%
  {\setfalse\inlinelistitem % new, no indent (leftskip)
   \setfalse\concatnextitem % new, concat
   \ifhmode
     \ifconditional\autoconcatnextitem % new, concat
       \ifdim\lastskip=\itemsignal     % new, concat
         \settrue\concatnextitem       % new, concat
       \fi                             % new, concat
     \fi                               % new, concat
     \iftextitems\else\doifnotinset\v!tekst{#1}\par\fi % suboptimal 
   \fi
   \ifnum\itemlevel=\maxitemlevel\relax
     \showmessage\m!layouts9\maxitemlevel
     \def\itemincrement{0}%
   \else
     \def\itemincrement{1}%
   \fi
   \doglobal\increment(\itemlevel,\itemincrement)%
   \initializeitemgrouplevel\itemlevel % safeguard 
   \begingroup
   \ifnum\itemlevel=1 % NIEUW
     \doadaptleftskip{\getitemparameter1\c!marge}%
   \fi
   \dosetraggedcommand{\getitemparameter\itemlevel\c!uitlijnen}\raggedcommand
   \doifsomething{\getitemparameter\itemlevel\c!inspringen}
     {\expanded{\stelinspringenin[\getitemparameter\itemlevel\c!inspringen]}}%
   \doifinset\v!kolommen{#1}%
     {\ifbinnenkolommen\else\ifnum\itemcolumndepth=0
        \global\let\itemcolumndepth\itemlevel
        \getitemparameter\itemlevel\c!voor
        \processfirstactioninset
          [#1]
          [  \v!een=>\!!counta=1\relax,
            \v!twee=>\!!counta=2\relax,
            \v!drie=>\!!counta=3\relax,
            \v!vier=>\!!counta=4\relax,
            \v!vijf=>\!!counta=5\relax,
         \s!unknown=>\@EA\!!counta\getitemparameter\itemlevel\c!n]%
        % new
        \edef\columneditemleftskip{\the\leftskip}%
        \def\postprocesscolumnbox##1%
          {\scratchdimen=\columneditemleftskip
           \divide\scratchdimen \nofcolumns
           \hbox{\hskip\columneditemleftskip\hbox{\box##1}}}%
        \scratchdimen=-\columneditemleftskip
        \multiply\scratchdimen \nofcolumns
        \advance\scratchdimen \columneditemleftskip
        \advance\scratchdimen \hsize
        \edef\columntextwidth{\the\scratchdimen}%
        \leftskip\zeropoint
        % so far
        \startkolommen
          [\c!n=\!!counta, % netter \??op\itemlevel\c!n
           \c!hoogte=,
           \c!lijn=\v!uit,
           \c!balanceren=\v!ja,
           \c!uitlijnen=\v!nee]%
      \fi\fi}%
   \doifinsetelse\v!intro{#1}\itemintrotrue\itemintrofalse
   \doglobal\increment\noflists
   \let\currentlist\noflists
   \newcounter\noflistelements
   \headitemfalse
   \subitemfalse
   \symbolitemfalse
   \let\marsymbol\relax
   \global\let\somdestination\empty
   \let\symsymbol\empty
   \the\itemgroupcommands
   \setitemlevel{#1}%
   \getitemparameter\itemlevel\empty
   \doifelsenothing{#1} % iffirstargument
     {\edef\@@opsymbool{\getitemparameter\itemlevel\c!symbool}%
      \global\letvalue{\@@globalitemsymbol\itemlevel}\empty
      \global\letitemparameter\itemlevel\v!verder\empty
      \setitemmark\@@opsymbool
      \dosetupitemgroupvariable[\itemlevel][#2]}
     {\dosetupitemgroupconstant[\itemlevel][#1]%
      \dosetupitemgroupvariable[\itemlevel][#2]%
      \doifinsetelse\v!verder{#1}% \noexpand, else problems in non-etex with chinese
        {\edef\@@opsymbool{\noexpand\getvalue{\@@globalitemsymbol\itemlevel}}%
         \getitemparameter\itemlevel\v!verder}
        {\edef\@@opsymbool{\noexpand\getitemparameter{\itemlevel}{\c!symbool}}%
         \global\setitemparameter\itemlevel\v!verder
           {\dosetupitemgroupconstant[\itemlevel][#1]%
            \dosetupitemgroupvariable[\itemlevel][#2]}}%
      \def\docommando##1% \setitemmark resets \docommando
        {\doifnot{##1}{0}{\setitemmark{##1}}}% 
      \processcommalist[#1,\@@opsymbool]\docommando}%
   \ifautoitemintro\ifnum\prevgraf<3
     \itemintrotrue
   \fi\fi
   \ifparagraphitems
     \ifnum\itemlevel>1 
       \letitemparameter\itemlevel\c!tussen\empty
     \fi
   \else\ifpackeditems
     \letitemparameter\itemlevel\c!tussen\empty
   \fi\fi
   \calculatelistwidth\itemlevel{\dimen0}%
   \ifdim\dimen0>\zeropoint\relax
     \ifconditional\inlinelistitem\else
       \advance\leftskip \dimen0\relax
     \fi
   \fi}

\def\stopitemgroup
  {\iftextitems
     \removeunwantedspaces\space\ignorespaces
   \else
     \par
   \fi
   \ifnum\itemcolumndepth=0 \dolistreference \fi % beware !
   \iffirstlist \else \endgroup \fi % toegevoegd, eerste \som opent groep
   \ifnum\itemcolumndepth=\itemlevel\relax
     \stopkolommen
     \doglobal\newcounter\itemcolumndepth
     \getitemparameter\itemlevel\c!na
   \else
     \ifnum\itemlevel=1
       \doitembreak\allowbreak           % toegevoegd
       \getitemparameter1\c!na
       \doif\@@oospringvolgendein\v!nee\noindentation
     \fi
   \fi
   \endgroup
   \doglobal\decrement(\itemlevel,\itemincrement)%
   \egroup}

\newtoks\itemgroupcommands

\def\itemgroupitem
  {\doitemgroupitem}

\def\itemgroupbutton[#1]%
  {\gdef\somdestination{#1}%
   \itemgroupitem}

\def\itemgroupdummy
  {\itemgroupsymbol{\strut}\strut}

\def\itemgroupsubitem 
  {\subitemtrue\itemgroupitem}

\def\itemgroupsymbol#1%
  {\def\symsymbol{\doitemattributes\itemlevel\c!symletter\c!symkleur{#1}}%
   \symbolitemtrue
   \itemgroupitem}

\def\itemgroupedge#1%
  {\itemgroupsymbol
     {\calculatelistwidth\itemlevel{\dimen0}%
      \hbox to \dimen0
        {#1\hskip\getitemparameter\itemlevel\c!afstand}}}

\def\itemgrouphead    
  {\headitemtrue\doitemgrouphead}

\def\itemgroupitems
  {\dosingleempty\doitemgroupitems}

\def\doitemgroupitems[#1]%   
  {\itemgroupedge
     {\dorecurse{0\getitemparameter\itemlevel\c!items}{\listitem\hss}%
      \unskip}}

\def\itemgroupmargin#1%  
  {\def\marsymbol%
     {\llap
        {\doitemattributes\itemlevel\c!marletter\c!markleur{#1}%
         \hskip\leftskip\hskip\linkermargeafstand}}%
   \itemgroupitem}

\appendtoks \let\item       \itemgroupitem    \to \itemgroupcommands
\appendtoks \let\itm        \itemgroupitem    \to \itemgroupcommands
\appendtoks \let\but        \itemgroupbutton  \to \itemgroupcommands
\appendtoks \let\nop        \itemgroupdummy   \to \itemgroupcommands
\appendtoks \letvalue\v!sub \itemgroupsubitem \to \itemgroupcommands
\appendtoks \letvalue\v!sym \itemgroupsymbol  \to \itemgroupcommands
\appendtoks \letvalue\v!ran \itemgroupedge    \to \itemgroupcommands
\appendtoks \letvalue\v!kop \itemgrouphead    \to \itemgroupcommands
\appendtoks \letvalue\v!its \itemgroupitems   \to \itemgroupcommands
\appendtoks \letvalue\v!mar \itemgroupmargin  \to \itemgroupcommands

\def\itembreak   % -10
  {\flushnotes\penalty-5\relax}   

\def\itemnobreak %  +5
  {\flushnotes\penalty+5\ifbinnenkolommen\else00\fi\relax} 

\def\dolistitem % evt aantal items opslaan per niveau, scheelt zoeken
  {\iftextitems
    % begin of item 
   \else
     \par
   \fi
   \ignorespaces
   \increment\noflistelements
   \ifnum\itemcolumndepth=0 \ifoptimizeitems
     \ifnum\noflistelements=1               % tgv bv kolommen/nesting
       \findtwopassdata\s!list{\noflists:}% % wordt soms de volgorde
     \fi                                    % verstoord, vandaar \find
\iftwopassdatafound
  \ifcase0\twopassdata\relax \twopassdatafoundfalse \fi
\fi  
     \iftwopassdatafound
       \ifnum\twopassdata=3
         \ifnum\noflistelements>1
           \doitembreak\itemnobreak
         \fi
       \else\ifnum\twopassdata>3
         \ifnum\noflistelements=2
           \ifitemintro
             \doitembreak\nobreak
           \else
             \doitembreak\itemnobreak
           \fi
         \else\ifnum\twopassdata=\noflistelements\relax
           \doitembreak\itemnobreak
         \else\ifnum\noflistelements>2
           \doitembreak\itembreak
         \else
           \ifitemintro\else\doitembreak\itembreak\fi
         \fi\fi\fi
       \fi\fi
     \fi
   \fi\fi
   \noindent
   \setbox8=\hbox
     {\ifheaditem
        \doitemattributes\itemlevel\c!kopletter\c!kopkleur{\listitem}%
      \else\ifsymbolitem
        \symsymbol 
      \else
        \doitemattributes\itemlevel\c!letter\c!kleur{\listitem}%
      \fi\fi}%
   \doifsomething\somdestination
     {\setbox8=\hbox{\naar{\box8}[\somdestination]}}%
   \global\let\somdestination\empty
   \dimen2=\getitemparameter\itemlevel\c!breedte\relax
   % new, prevents loops when symbol is (not yet found) graphic
   \ht8=\strutheight
   \dp8=\strutdepth
   % so that content differs per run (esp mp graphics afterwards)
   \ifdim\dimen2<\zeropoint\relax
     \llap{\ifsubitem\llap{+}\fi\box8\hskip\linkermargeafstand}%
   \else
     \ifdim\dimen2=\zeropoint\relax
       \calculatelistwidth{1}{\dimen0}%
     \else
       \calculatelistwidth{\itemlevel}{\dimen0}%
     \fi
     \iftextitems
       \hbox{\ifsubitem+\fi\box8\hskip\fontdimen2\font}\nobreak
     \else
       \ifconditional\inlinelistitem
         \hbox to \dimen0{\ifsubitem\llap{+}\fi\box8\hfill}%
       \else
         \llap{\hbox to \dimen0{\ifsubitem\llap{+}\fi\box8\hfill}}%
       \fi
     \fi
   \fi
   \setevalue{\@@currentitemsymbol\itemlevel}%
     {\getvalue{\@@localitemsymbol\itemlevel}}% still problems with \uchar ?
    %{\noexpand\getvalue{\@@localitemsymbol\itemlevel}}% no, spoils subrefs
   \headitemfalse
   \subitemfalse
   \symbolitemfalse
   \EveryPar{\ignorespaces}%
   \ignorespaces}

\def\complexdoitemgroupitem[#1]%
  {\iftextitems
     % begin of item
   \else
     \par
   \fi
   \ignorespaces
   \ifconditional\concatnextitem % new, concat
     \doitembreak\nobreak        % new, concat
   \fi                           % new, concat
   \doadvanceitem
   \ifnum\itemcolumndepth=0\relax\ifnum\noflistelements>0\relax
     \doitembreak\nobreak
   \fi\fi
   \iffirstlist
     \firstlistfalse
     \begingroup
     \ifcase\itemlevel
     \or % 1
       \ifnum\itemcolumndepth=0\relax
         \ifitemintro\doitembreak\nobreak\fi
         \getitemparameter1\c!voor
         \ifitemintro\doitembreak\nobreak\fi
       \fi
     \else % 2 en hoger
       \ifparagraphitems \else
         \let\previtemlevel\itemlevel
         \decrement\previtemlevel
         \getitemparameter\previtemlevel\c!tussen % = itemlevel-1
       \fi
     \fi
   \else
     \getitemparameter\itemlevel\c!tussen
   \fi
   \ifconditional\concatnextitem % new, concat
     \vskip-\lastskip            % new, concat
     \vskip-\lineheight          % new, concat
     \nobreak                    % new, concat
   \fi                           % new, concat
   \ignorespaces
   \dolistitem
   \ifpackeditems
     \stelwitruimtein[\v!geen]%
   \fi
   \getitemparameter\itemlevel\c!binnen
   \marsymbol
   \let\marsymbol\relax
   \doifsomething{#1}
     {\doifnot\itemreference\unknownitemreference
        {\bgroup
         \protectconversion
         \rawreference\s!lst{#1}\itemreference
         \egroup}}%
   \strut % added 11-08-99
   \setfalse\concatnextitem % new, concat
   \hskip\itemsignal        % new, concat
   \ignorespaces}

\def\complexitem[#1]#2\par% todo: no two pass data
  {\startitemgroup[#1]
   \complexdoitemgroupitem[]\ignorespaces\begstrut#2\unskip\endstrut\par
   \stopitemgroup}

\definecomplexorsimpleempty\item
\definecomplexorsimpleempty\doitemgroupitem

\def\complexdoitemgrouphead[#1]#2\par% % beter in \complexdosom hangen met een if
  {\iffirstlist\else\doitembreak\allowbreak\fi
   \ifpackeditems\else\getitemparameter\itemlevel\c!kopvoor\fi
   \iffirstlist\ifitemintro\else\ifcase\itemlevel % incr in \complexdosom
      \doitembreak\allowbreak
   \fi\fi\fi
   \complexdoitemgroupitem[#1]{\doitemattributes\itemlevel\c!kopletter\c!kopkleur
     {\ignorespaces#2}}%
   \iftextitems
     \removeunwantedspaces\space\ignorespaces
   \else
     \par
   \fi
   \doitembreak\nobreak
   \ifpackeditems\else\getitemparameter\itemlevel\c!kopna\fi
   \doitembreak\nobreak
   \noindentation}

\def\complexhead[#1]#2\par#3\par
  {\startitemgroup[#1]%
   \complexdoitemgrouphead[]\ignorespaces#2\par#3\par
   \stopitemgroup}

\definecomplexorsimpleempty\head
\definecomplexorsimpleempty\doitemgrouphead

\def\sym#1%
  {\noindent
   \setbox0\hbox{#1}%
   \setbox0\hbox\ifdim\wd0<1em to 1.5\else spread 1\fi em{#1\hfil}%
   \hangindent\wd0\relax
   \box0
   \ignorespaces}

\setupitemgroups % undocumented
  [\c!niveaus=6,
   \c!marge=\!!zeropoint,
   \c!springvolgendein=\v!ja,
   \c!breedte=1.5em,
   \c!factor=0,
   \c!afstand=.5em,
  %\c!uitlijnen=\v!normaal, % definitely not \v!normaal, see mails and
   \c!uitlijnen=, % debug reports of David A & Patrick G on context list 
   \c!kleur=,      
   \c!inspringen=, % untouched if empty
   \c!kleur=,
   \c!letter=, % kan tzt weg
   \c!marletter=\c!type,  % \c! ???
   \c!symletter=,
   \c!kopletter=,
   \c!markleur=,
   \c!symkleur=,
   \c!kopkleur=,
   \c!kopvoor=,
   \c!kopna=\blanko,
   \c!voor=\blanko,
   \c!tussen=\blanko,
   \c!na=\blanko,
   \c!afsluiter=.,
   \c!plaatsafsluiter=\v!ja,
   \c!binnen=,
   \c!n=2,
   \c!items=4,
   \c!linkertekst=(,
   \c!rechtertekst=),
   \c!start=1,
   \c!symbool=\itemlevel] % \v!niveau

\def\defineitemgroup
  {\dodoubleempty\dodefineitemgroup}

\def\dodefineitemgroup[#1][#2]%
  {\doifsomething{#1}
     {\pushmacro\currentitemgroup
      \def\currentitemgroup{#1}%
      \setvalue{\e!start#1}{\startitemgroup[#1]}%
      \setvalue{\e!stop#1}{\stopitemgroup}%
      \setvalue{\e!stel#1\e!in}{\setupitemgroup[#1]}%
\getparameters[\??ig#1][\c!niveaus=3,#2]%
\ifnum\getvalue{\??ig#1\c!niveaus}<\maxitemlevel\relax
  \letvalue{\??ig#1\c!niveaus}\maxitemlevel
\fi 
      \dorecurse{\getvalue{\??ig#1\c!niveaus}}{\initializeitemgrouplevel\recurselevel}%
      \popmacro\currentitemgroup}}

% efficient default itemize as well as upward compatible definition 

% \def\startitemize {\startitemgroup[]} 
% \def\stopitemize  {\stopitemgroup}
% \def\setupitemize {\setupitemgroup[]}

\defineitemgroup [\e!itemize] [\c!niveaus=6]

\protect \endinput
