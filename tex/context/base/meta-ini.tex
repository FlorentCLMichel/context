%D \module
%D   [       file=meta-ini,
%D        version=1999.07.10,
%D          title=\METAPOST\ Graphics,
%D       subtitle=Initialization,
%D         author=Hans Hagen,
%D           date=\currentdate,
%D      copyright={PRAGMA / Hans Hagen \& Ton Otten}]
%C
%C This module is part of the \CONTEXT\ macro||package and is
%C therefore copyrighted by \PRAGMA. See mreadme.pdf for
%C details.

% currently the running color influences the mp graphic in
% pdftex, but this will change [i.e. become optional]; one
% problem is that pdf has no grouping with regards to the
% color

\writestatus{loading}{MetaPost Graphics / Initializations}

\unprotect

\startmessages  dutch  library: metapost
  title: metapost
      1: metapost bibliotheek -- wordt geladen
\stopmessages

\startmessages  english  library: metapost
  title: metapost
      1: loading metapost library --
\stopmessages

\startmessages  german  library: metapost
  title: metapost
      1: loading metapost library --
\stopmessages

\startmessages  czech  library: metapost
  title: metapost
      1: loading metapost library --
\stopmessages

\startmessages  italian  library: metapost
  title: metapost
      1: caricamento della libreria metapost --
\stopmessages

\startmessages  norwegian  library: metapost
  title: metapost
      1: metapost bibliotek -- blir lest inn
\stopmessages

\startmessages  romanian  library: metapost
  title: metapost
      1: se incarca biblioteca metapost --
\stopmessages

\startmessages  french  library: metapost
  title: metapost
      1: chargement de la bibliothèque metapost --
\stopmessages

%D This module extends the functionality of the support module
%D \type {supp-mps}, the module that is responsible for
%D \METAPOST\ inclusion in \CONTEXT. Some basic macros will be
%D extended. Since some support is depends on \METAPOST\
%D macros. so let's first preload a few auxiliary \METAPOST\
%D files.

\maxnofMPgraphics = 4000 % metafun disables the 4K boundary

\startMPextensions
  if unknown context_tool: input mp-tool; fi;
  if unknown context_spec: input mp-spec; fi;
  if unknown context_grph: input mp-grph; fi;
\stopMPextensions

%D Since we want lables to follow the document settings, we
%D also set the font related variables.

\startMPinitializations % scale is not yet ok
  defaultfont:="\truefontname{Regular}";
  defaultscale:=\the\bodyfontsize/10pt;
\stopMPinitializations

%D In order to support fancy text features (like outline
%D fonts), we set:

\startMPextensions
  graphictextformat:="context";
  graphictextdirective "\the\everyMPTEXgraphic";
\stopMPextensions

% \startMPextensions
%   textextdirective "\the\everyMPTEXgraphic";
% \stopMPextensions

%D A signal that we're in combines \CONTEXT||\METAFUN mode:

\startMPextensions
  string contextversion;
  contextversion:="\contextversion";
\stopMPextensions

%D Some safeguards:
%D
%D \starttyping
%D \appendtoks \cleanupfeatures \to \everyMPgraphic
%D \stoptyping
%D
%D No, we don't want that (else we loose UTF etc).

%D Another one:

\prependtoks \MPstaticgraphictrue \to \everyoverlay
\prependtoks \MPstaticgraphictrue \to \everypagebody

%D We save the number of graphics for the sake of \TEXEXEC.

\newcounter\totalnofMPgraphics

\def\thenofMPgraphics{\the\nofMPgraphics} % from supp-mps

\appendtoks
  \savecurrentvalue\totalnofMPgraphics\thenofMPgraphics
\to \everybye

%D \macros
%D   {setupMPvariables}
%D
%D When we build collections of \METAPOST\ graphics, like
%D background and buttons, the need for passing settings
%D arises. By (mis|)|using the local prefix that belongs to
%D \type {\framed}, we get a rather natural interface to
%D backgrounds. To prevent conflicts, we will use the \type
%D {-} in \METAPOST\ specific variables, like:
%D
%D \starttyping
%D \setupMPvariables[meta:button][size=20pt]
%D \stoptyping

\def\@@meta{meta:}

\def\setupMPvariables
  {\dodoubleempty\dosetupMPvariables}

\def\dosetupMPvariables[#1][#2]%
  {\ifsecondargument
     \getrawparameters[#1:][#2]% brr, todo: [\@@meta#1:]
   \else
     \getrawparameters[\@@meta][#1]%
   \fi}

\let\@@framed\s!unknown

% \def\MPvariable#1%
%   {\getvalue{\ifundefined{\@@framed\@@meta#1}\else\@@framed\fi\@@meta#1}}

\beginTEX

\def\MPvariable#1%
  {\csname
     \@EA\ifx\csname\@@framed\@@meta#1\endcsname\relax\else\@@framed\fi\@@meta#1%
   \endcsname}

\endTEX

\beginETEX \ifcsname

\def\MPvariable#1%
  {\csname
     \ifcsname\@@framed\@@meta#1\endcsname\@@framed\fi\@@meta#1%
   \endcsname}

\endETEX

\let\MPvar\MPvariable

\let\setMPvariables\setupMPvariables

\def\MPrawvar#1#2{\csname#1:#2\endcsname}

%D \macros
%D   {startuniqueMPgraphic, uniqueMPgraphic}
%D
%D This macros is probably of most use to myself, since I like
%D to use graphics that adapt themselves. The next \METAPOST\
%D kind of graphic is both unique and reused when possible.
%D
%D \starttyping
%D \defineoverlay[example][\uniqueMPgraphic{test}]
%D
%D \startuniqueMPgraphic {test}
%D   draw unitsquare xscaled \overlaywidth yscaled \overlayheight ;
%D \stopuniqueMPgraphic
%D \stoptyping

%D For educational purposes, we show the original version
%D first. This one used a rather simple method for determining
%D the uniqueness.
%D
%D \starttyping
%D \long\def\startuniqueMPgraphic#1#2\stopuniqueMPgraphic%
%D   {\setvalue{\@@MPG#1}%
%D      {\startreusableMPgraphic{\overlaystamp:#1}#2\stopreusableMPgraphic
%D       \reuseMPgraphic{\overlaystamp:#1}}}
%D
%D \def\uniqueMPgraphic#1%
%D   {\getvalue{\@@MPG#1}}
%D \stoptyping

%\def\overlaystamp % watch the \MPcolor, since colors can be redefined
%  {\overlaywidth:\overlayheight:\overlaydepth
%   :\MPcolor{\overlaycolor}:\MPcolor{\overlaylinecolor}}

\def\overlaystamp % watch the \MPcolor, since colors can be redefined
  {\overlaywidth:\overlayheight:\overlaydepth
   :\MPcolor\overlaycolor:\MPcolor\overlaylinecolor}

%D A better approach is to let additional variables play a role
%D in determining the uniqueness. In the next macro, the
%D second, optional, argument is used to guarantee the
%D uniqueness, as well as prepare variables for passing them to
%D \METAPOST.
%D
%D \starttyping
%D \startuniqueMPgraphic{meta:hash}{gap,angle,...}
%D \stoptyping
%D
%D The calling macro also accepts a second argument. For
%D convenient use in overlay definitions, we use \type {{}}
%D instead of \type {[]}.
%D
%D \starttyping
%D \uniqueMPgraphic{meta:hash}{gap=10pt,angle=30}
%D \stoptyping

\long\def\handleuniqueMPgraphic#1#2#3%
  {\blabelgroup
   \def\@@meta{#1:}%
   \extendMPoverlaystamp{#2}% incl prepare
   \ifundefined{\@@MPG\overlaystamp:#1}%
     \enableincludeMPgraphics
     \startMPgraphic#3\stopMPgraphic
     \doifobjectssupportedelse\donothing\useMPboxfalse
     \ifuseMPbox
       \@EA\douseMPbox
     \else
       \@EA\nouseMPbox
     \fi {\@@MPG\overlaystamp:#1}%
   \fi
   \getvalue{\@@MPG\overlaystamp:#1}%
   \elabelgroup}

\long\def\startuniqueMPgraphic
  {\blabelgroup
   \dodoublegroupempty\dostartuniqueMPgraphic}

\long\def\dostartuniqueMPgraphic
  {\obeyMPlines
   \dodostartuniqueMPgraphic}

\long\def\dodostartuniqueMPgraphic#1#2#3\stopuniqueMPgraphic%
  {\long\setgvalue{\@@MPG#1}{\handleuniqueMPgraphic{#1}{#2}{#3}}%
   \elabelgroup}

\unexpanded\def\uniqueMPgraphic
  {\dodoublegroupempty\douniqueMPgraphic}

\def\douniqueMPgraphic#1#2%
  {\blabelgroup
   \setupMPvariables[#1][#2]%
   \getvalue{\@@MPG#1}{}%
   \elabelgroup}

\let\stopuniqueMPcode \relax % so that we can use it in \expanded

\long\def\handleuseMPgraphic#1#2#3%
  {\blabelgroup
   \def\@@meta{#1:}%
   \prepareMPvariables{#2}%
   \enableincludeMPgraphics
   \startMPgraphic#3\stopMPgraphic
   \ifMPrun \else % see mfun-004 : processing buffer
     \loadMPgraphic{\MPgraphicfile.\the\currentMPgraphic}{}%
     \placeMPgraphic
   \fi
   \deallocateMPslot\currentMPgraphic
   \elabelgroup}

\long\def\startuseMPgraphic
  {\blabelgroup
   \dodoublegroupempty\dostartuseMPgraphic}

\long\def\dostartuseMPgraphic
  {\obeyMPlines
   \dodostartuseMPgraphic}

\long\def\dodostartuseMPgraphic#1#2#3\stopuseMPgraphic
  {\long\setgvalue{\@@MPG#1}{\handleuseMPgraphic{#1}{#2}{#3}}%
   \elabelgroup}

\long\def\startusableMPgraphic % redundant but handy
  {\blabelgroup
   \dodoublegroupempty\dostartusableMPgraphic}

\long\def\dostartusableMPgraphic % redundant but handy
  {\obeyMPlines
   \dodostartusableMPgraphic}

\long\def\dodostartusableMPgraphic#1#2#3\stopusableMPgraphic
  {\long\setgvalue{\@@MPG#1}{\handleuseMPgraphic{#1}{#2}{#3}}%
   \elabelgroup}

\long\def\handlereusableMPgraphic#1#2#3%
  {\blabelgroup
   \def\@@meta{#1:}%
   \prepareMPvariables{#2}%
   \enableincludeMPgraphics
   \startMPgraphic#3\stopMPgraphic
   \doifobjectssupportedelse\donothing\useMPboxfalse
   \ifuseMPbox
     \@EA\douseMPbox
   \else
     \@EA\nouseMPbox
   \fi {\@@MPG#1}%
   \getvalue{\@@MPG#1}%
   \elabelgroup}

\long\def\startreusableMPgraphic
  {\blabelgroup
   \dodoublegroupempty\dostartreusableMPgraphic}

\def\dostartreusableMPgraphic
  {\obeyMPlines
   \dodostartreusableMPgraphic}

\long\def\dodostartreusableMPgraphic#1#2#3\stopreusableMPgraphic
  {\long\setgvalue{\@@MPG#1}{\handlereusableMPgraphic{#1}{#2}{#3}}%
   \elabelgroup}

\unexpanded\def\useMPgraphic
  {\dodoublegroupempty\douseMPgraphic}

\def\douseMPgraphic#1#2%
  {\blabelgroup
   \setupMPvariables[#1][#2]%
   \getvalue{\@@MPG#1}{}%
   \elabelgroup}

\let\reuseMPgraphic\useMPgraphic

\let\stopuseMPcode      \relax % so that we can use it in \expanded
\let\stopusableMPcode   \relax % so that we can use it in \expanded
\let\stopreusableMPcode \relax % so that we can use it in \expanded
\let\stopuniqueMPcode   \relax % so that we can use it in \expanded

\def\enableincludeMPgraphics
  {\let\handleuseMPgraphic     \thirdofthreearguments
   \let\handlereusableMPgraphic\thirdofthreearguments}

% todo: each code/page/buffer a var class

%D \macros
%D   {startuniqueMPpagegraphic,uniqueMPpagegraphic}
%D
%D Experimental.

\def\MPpageprefix{\doifoddpageelse oe:}

\def\overlaypagestamp
  {\MPpageprefix\overlaywidth:\overlayheight:\overlaydepth
   :\MPcolor\overlaycolor:\MPcolor\overlaylinecolor}

\long\def\startuniqueMPpagegraphic
  {\blabelgroup
   \dodoublegroupempty\dostartuniqueMPpagegraphic}

\long\def\dostartuniqueMPpagegraphic
  {\obeyMPlines
   \dodostartuniqueMPpagegraphic}

\long\def\dodostartuniqueMPpagegraphic#1#2#3\stopuniqueMPpagegraphic
  {\long\setgvalue{\@@MPG o:#1}{\handleuniqueMPgraphic{o:#1}{#2}{#3}}%
   \long\setgvalue{\@@MPG e:#1}{\handleuniqueMPgraphic{e:#1}{#2}{#3}}%
   \elabelgroup}

\unexpanded\def\uniqueMPpagegraphic
  {\dodoublegroupempty\douniqueMPpagegraphic}

% \def\douniqueMPpagegraphic#1#2%
%   {\blabelgroup
%    \let\overlaystamp\overlaypagestamp
%    \setupMPvariables[#1][#2]%
%    \getvalue{\@@MPG\MPpageprefix#1}{}%
%    \elabelgroup}

\def\douniqueMPpagegraphic#1#2%
  {\blabelgroup
   \let\overlaystamp\overlaypagestamp
   \setupMPvariables[\MPpageprefix#1][#2]% prefix is new here
   \getvalue{\@@MPG\MPpageprefix#1}{}%
   \elabelgroup}

%D One way of defining a stamp is:
%D
%D \starttyping
%D \def\extendMPoverlaystamp#1%
%D   {\def\docommand##1%
%D      {\edef\overlaystamp{\overlaystamp:\MPvariable{##1}}}%
%D    \processcommalist[#1]\docommand}
%D \stoptyping

%D Since we need to feed \METAPOST\ with expanded dimensions,
%D we introduce a dedicated expansion engine.

\def\prepareMPvariable#1%
  {\ifundefined{\@@framed\@@meta#1}%
     \doprepareMPvariable{\@@meta#1}%
   \else
     \doprepareMPvariable{\@@framed\@@meta#1}%
   \fi}

% \startlines
% \def\xxx{\lineheight}     \doprepareMPvariable{xxx} \xxx
% \def\xxx{2pt}             \doprepareMPvariable{xxx} \xxx
% \def\xxx{2}               \doprepareMPvariable{xxx} \xxx
% \def\xxx{\scratchcounter} \doprepareMPvariable{xxx} \xxx
% \def\xxx{red}             \doprepareMPvariable{xxx} \xxx
% \def\xxx{0.4}             \doprepareMPvariable{xxx} \xxx
% \stoplines

\def\doprepareMPvariable#1%
  {\edef\theMPvariable{\getvalue{#1}}%
   \doifelsenothing\theMPvariable
     {\setevalue{#1}{\MPcolor{black}}}
     {\convertcommand\theMPvariable\to\ascii % otherwise problems
      \doifcolorelse \ascii                  % with 2\bodyfontsize
        {\setevalue{#1}{\MPcolor\theMPvariable}}
        {% can be aux macro
         \setbox\scratchbox\hbox{\scratchdimen\theMPvariable sp}%
         \ifdim\wd\scratchbox=\zeropoint
         % \scratchcounter\theMPvariable
         % \setevalue{#1}{\the\scratchcounter}%
         % also accepts 0.number :
           \setevalue{#1}{\number\theMPvariable}%
         \else
           \scratchdimen\theMPvariable
           \setevalue{#1}{\the\scratchdimen}%
         \fi}}}

%D We redefine \type {\extendMPoverlaystamp} to preprocess
%D variables using \type {\prepareMPvariable}.

\def\doextendMPoverlaystamp#1%
  {\prepareMPvariable{#1}%
   \edef\overlaystamp{\overlaystamp:\MPvariable{#1}}}

\def\extendMPoverlaystamp#1%
  {\processcommalist[#1]\doextendMPoverlaystamp}

\def\prepareMPvariables#1%
  {\processcommalist[#1]\prepareMPvariable}

%D \macros
%D   {MPdatafile}
%D
%D We redefine a macro from \type {supp-mps.tex}:

\def\MPdatafile
  {\bufferprefix mpd-\the\currentMPgraphic.mpd}

%D \macros
%D   {MPrunfile}
%D
%D This one is more abstract and does not assume knowledge
%D of buffer prefixes.

\def\MPrunfile#1%
  {\bufferprefix mprun.#1}

%D We also have to make sure that \METAPOST\ knows this:

\startMPextensions
  if not known _data_prefix_:
    string _data_prefix_,_data_suffix_;
  fi;
  _data_prefix_:="\bufferprefix mpd-";
  _data_suffix_:=".mpd";
\stopMPextensions

%D \macros
%D   {getMPdata}
%D
%D The current data is loaded with:

\def\getMPdata
  {\startreadingfile
   \readlocfile\MPdatafile\donothing\donothing
   \stopreadingfile}

%D When we collect graphics in one file, we run into
%D troubles, since \METAPOST\ has a built in limit (of 4)
%D on the number of files it can handle. It's therefore
%D better to collect all data in one file and filter it.

\def\getMPdata
  {\long\def\MPdata##1##2%
     {\ifnum##1=\currentMPgraphic\relax##2\fi}%
   \startreadingfile
   \readlocfile{\MPgraphicfile.mpd}\donothing\donothing
   \stopreadingfile}

%D We have to enable this mechanism with:

\startMPextensions
  boolean collapse_data; collapse_data:=true;
  _data_suffix_:=".mpd"; % overloads previous one
\stopMPextensions

%D For the moment, the next one is a private macro:

\def\processMPbuffer
  {\dosingleempty\doprocessMPbuffer}

\def\doprocessMPbuffer[#1]%
  {\doifelsenothing{#1}
     {\doprocessMPbuffer[\jobname]}
     {\bgroup
      \setnormalcatcodes
      \obeyMPlines
     %\let\par\empty % oeps, this makes dvi mode graphics hang when not found
      \!!toksa\emptytoks
      \def\copyMPbufferline{\expandafter\appendtoks\fileline\to\!!toksa}%
      \def\dodoprocessMPbuffer##1%
        {\doprocessfile\scratchread{\TEXbufferfile{##1}}\copyMPbufferline}%
      \processcommalist[#1]\dodoprocessMPbuffer
      \@EA\startMPcode\the\!!toksa\stopMPcode % more efficient
      \egroup}}

\def\runMPbuffer
  {\dosingleempty\dorunMPbuffer}

\def\dorunMPbuffer[#1]% processing only
  {{\MPruntrue\doprocessMPbuffer[#1]}}

%D \macros
%D   {startMPenvironment, resetMPenvironment}
%D
%D In order to synchronize the main \TEX\ run and the runs
%D local to \METAPOST, environments can be passed.

\ifx\everyMPTEXgraphic\undefined
  \newtoks\everyMPTEXgraphic
\fi

%D A more general way of passing environments is:

% ok but introduces \relax's
%
% \def\startMPenvironment                 % second arg gobbles spaces, so
%   {\dodoubleempty\dostartMPenvironment} % that reset gives \emptytoks
%
% \long\def\dostartMPenvironment[#1][#2]#3\stopMPenvironment%
%   {\doif{#1}\s!reset\resetMPenvironment % reset mp toks
%    \doif{#1}\v!globaal{#3}%             % use in main doc too
%    \doif{#1}+{#3}%                      % use in main doc too
%    \convertargument#3\to\ascii
%    \expandafter\appendtoks\ascii\to\everyMPTEXgraphic}

\def\startMPenvironment % second arg gobbles spaces, so that reset gives \emptytoks
  {\bgroup
   \catcode`\^^M=\@@space
   \dodoubleempty\dostartMPenvironment}

\long\def\dostartMPenvironment[#1][#2]#3\stopMPenvironment
  {\egroup
   \doif{#1}\s!reset\resetMPenvironment % reset mp toks
   \doif{#1}\v!global{#3}%              % use in main doc too
   \doif{#1}+{#3}%                      % use in main doc too
   \convertargument#3\to\ascii
   \expandafter\appendtoks\ascii\to\everyMPTEXgraphic}

\def\resetMPenvironment
  {\everyMPTEXgraphic\emptytoks % = is really needed !
   \startMPenvironment
     \global\loadfontfileoncetrue
   \stopMPenvironment}

\resetMPenvironment

\def\useMPenvironmentbuffer[#1]%
  {\expanded{\startMPenvironment\noexpand\readfile{\TEXbufferfile{\jobname}}{}{}}\stopMPenvironment}

\useMPenvironmentbuffer[mp]

%D This command takes \type {[reset]} as optional
%D argument.
%D
%D \starttyping
%D \startMPenvironment
%D   \setupbodyfont[pos,14.4pt]
%D \stopMPenvironment
%D
%D \startMPcode
%D   draw btex \sl Hans Hagen etex scaled 5 ;
%D \stopMPcode
%D \stoptyping
%D
%D The \type {\resetMPenvironment} is a quick way to erase
%D the token list.
%D
%D You should be aware of independencies. For instance, if you use a font
%D in a graphic that is not used in the main document, you need to load the
%D typescript at the outer level (either directly or by using the global
%D option).
%D
%D \starttyping
%D \usetypescript[palatino][texnansi]
%D
%D \startMPenvironment
%D     \usetypescript[palatino][texnansi]
%D     \enableregime[utf]
%D     \setupbodyfont[palatino]
%D \stopMPenvironment
%D
%D \startMPpage
%D     draw btex aap‒noot coördinatie – één etex ;
%D \stopMPpage
%D \stoptyping

%D We don't want spurious files, do we?

%\def\initializeMPgraphics
%  {%\ifx\bufferprefix\empty \else
%     \immediate\openout\MPwrite\MPgraphicfile.mp
%     \immediate\write\MPwrite{end.}%
%     \immediate\closeout\MPwrite
%   }%\fi}

% strange :

% \def\initializeMPgraphicfile
%   {\bgroup
%    \doinitializeMPgraphicfile
%    \MPruntrue
%    \doinitializeMPgraphicfile
%    \egroup}

% \def\doinitializeMPgraphicfile
%   {\immediate\openout\scratchwrite\MPgraphicfile.mp
%    \immediate\write\scratchwrite{end.}%
%    \immediate\closeout\scratchwrite}

\def\initializeMPgraphicfile
  {\immediate\openout\scratchwrite\MPgraphicfile.mp
   \immediate\write\scratchwrite{end.}%
   \immediate\closeout\scratchwrite}

\def\initializeMPgraphics
  {\bgroup
   \initializeMPgraphicfile
   \ifx\bufferprefix\empty\else
     \let\bufferprefix\empty
     \initializeMPgraphicfile
   \fi
   \egroup}

%D Loading specific \METAPOST\ related definitions is
%D accomplished by:

\def\douseMPlibrary#1%
  {\ifundefined{\c!file\f!javascriptprefix#1}%
     \letvalueempty{\c!file\f!javascriptprefix#1}%
     \makeshortfilename[\truefilename{\f!metapostprefix#1}]%
     \startreadingfile
       \readsysfile\shortfilename{\showmessage\m!metapost1{#1}}\donothing
     \stopreadingfile
   \fi}

\def\useMPlibrary[#1]%
  {\processcommalist[#1]\douseMPlibrary}

%D \macros
%D   {setMPtext, MPtext, MPstring, MPbetex}
%D
%D To be documented:
%D
%D \starttyping
%D \setMPtext{identifier}{text}
%D
%D \MPtext  {identifier}
%D \MPstring{identifier}
%D \MPbetex {identifier}
%D \stoptyping

\def\@@MPT{@MPT@}

\def\forceMPTEXgraphic
  {\long\def\checkMPTEXgraphic##1{\global\MPTEXgraphictrue}}

\def\setMPtext#1#2% todo : #1 must be made : safe
  {%\forceMPTEXgraphic
   \convertargument#2\to\ascii
   \dodoglobal\letvalue{\@@MPT#1}\ascii}

% \def\MPtext       #1{\getvalue{\@@MPT#1}}
% \def\MPstring    #1{"\getvalue{\@@MPT#1}"}
% \def\MPbetex #1{btex \getvalue{\@@MPT#1} etex}

\def\MPtext       #1{\executeifdefined{\@@MPT#1}\empty}
\def\MPstring    #1{"\executeifdefined{\@@MPT#1}\empty"}
\def\MPbetex #1{btex \executeifdefined{\@@MPT#1}\empty\space etex}

%D Unfortunately \METAPOST\ does not have \CMYK\ support
%D built in, but by means of specials we can supply the
%D information needed to handle them naturaly.

\newif\ifMPcmykcolors \MPcmykcolorstrue
\newif\ifMPspotcolors \MPspotcolorstrue

\startMPinitializations
  cmykcolors:=\ifMPcmykcolors true\else false\fi;
  spotcolors:=\ifMPspotcolors true\else false\fi;
\stopMPinitializations

%D In order to communicate conveniently with the \TEX\
%D engine, we introduce some typesetting variables.

% todo : backgroundoffsets

\startMPextensions
  color OverlayColor,OverlayLineColor;
\stopMPextensions

\startMPinitializations
  OverlayWidth:=\overlaywidth;
  OverlayHeight:=\overlayheight;
  OverlayDepth:=\overlayheight;
  OverlayColor:=\MPcolor{\overlaycolor};
  OverlayLineWidth:=\overlaylinewidth;
  OverlayLineColor:=\MPcolor{\overlaylinecolor};
  %
  BaseLineSkip:=\the\baselineskip;
  LineHeight:=\the\baselineskip;
  BodyFontSize:=\the\bodyfontsize;
  %
  TopSkip:=\the\topskip;
  StrutHeight:=\strutheight;
  StrutDepth:=\strutdepth;
  %
  CurrentWidth:=\the\hsize;
  CurrentHeight:=\the\vsize;
  %
  EmWidth:=\the\emwidth;
  ExHeight:=\the\exheight;
  %
  PageNumber:=\the\pageno;
  RealPageNumber:=\the\realpageno;
  LastPageNumber:= \lastpage;
\stopMPinitializations

\appendtoks
    \disablediscretionaries
    \disablecompoundcharacters
\to\everyMPgraphic

\appendtoks
    \expanded{\definecolor[currentcolor][\currentcolorname]}%
\to \everyMPgraphic

\appendtoks
  \baselineskip1\baselineskip
  \lineheight  1\lineheight
  \topskip     1\topskip
\to \everyMPgraphic

%D Alas, the prologue settings differ per driver.

\ifx\undefined\MPprologues \def\MPprologues{0} \fi

\startMPinitializations
  prologues:=\MPprologues;
  mpprocset:=1;
\stopMPinitializations

\appendtoks
  \def\MPprologues{0}%
  \def\MPOSTdriver{dvips}%
\to \everyresetspecials

%D \macros
%D   {PDFMPformoffset}
%D
%D In \PDF, forms are clipped and therefore we have to take
%D precautions to get this right. Since this is related to
%D objects, we use the same offset as used there.

\def\PDFMPformoffset{\objectoffset}

%D \macros
%D   {insertMPfile}
%D
%D Bypassing the special driver and figure mechanism is not
%D that nice but saves upto 5\% time in embedding \METAPOST\
%D graphics by using the low level \PDF\ converter directly,
%D given of course that we use \PDFTEX. As a result we need to
%D fool around with the object trigger.

\newtoks\everyinsertMPfile

%D First we present the reasonable fast alternative that we
%D happily used for some time.
%D
%D \starttyping
%D \def\insertMPfile#1#2%
%D   {\ifx\undefined\externalfigure
%D      \message{[insert file #1 here]}%
%D    \else
%D      \bgroup
%D      \the\everyinsertMPfile
%D      \externalfigure
%D        [#1]
%D        [\c!type=\c!mps,\c!object=\v!no,%
%D         \c!symbol=\v!yes,\c!reset=\v!yes,%
%D         \c!maxwidth=,\c!maxheight=,%
%D         \c!frame=\v!off,\c!background=,%
%D         #2]%
%D      \egroup
%D    \fi}
%D \stoptyping
%D
%D However, on a 1 Gig Pentium, the next alternative saves
%D us 20 seconds run time for the 300 page \METAFUN\ manual:

\let\insertMPfileARG\insertMPfile

\def\insertMPfile#1#2% in context #2 is empty
  {\doifelsenothing{#2}{\doinsertMPfile{#1}}{\insertMPfileARG{#1}{#2}}}

\def\includeMPasEPS#1% untested !!
  {\bgroup
   \message{[MP as EPS #1]}%
   \the\everyinsertMPfile
   \dogetEPSboundingbox{#1}\!!widtha\!!heighta\!!widthb\!!heightb
   \setbox\scratchbox\vbox to \!!heightb
     {\vfill
      \let \@@DriverImageType  \c!mps
      \def \@@DriverImageFile  {#1}%
      \edef\@@DriverImageWidth {\the\!!widthb }%
      \edef\@@DriverImageHeight{\the\!!heightb}%
      \doinsertfile}%
   \wd\scratchbox\!!widthb
   \dp\scratchbox\zeropoint
   \box\scratchbox
   \egroup}

\def\includeMPasPDF#1%
  {\bgroup
   \the\everyinsertMPfile
   \ifinobject \else \chardef\makeMPintoPDFobject\plustwo \fi % when needed
   \convertMPtoPDF{#1}{1}{1}% no \plusone !
   \egroup}

%D So, using a low level approach (thereby avoiding the slower
%D figure analysis macros) pays off. This kind of
%D optimizations are a bit tricky since we must make sure that
%D special resources end up in the (PDF) files. Because the
%D \METAPOST\ to \PDF\ can handle objects itself, it is not
%D that complicated.

%D We hook a couple of initializations into the graphic
%D macros.

\appendtoks
  \let\figuretypes\c!mps
  \runutilityfilefalse
  \consultutilityfilefalse
\to \everyinsertMPfile

%D One more: (still needed?)

\startMPextensions
  def initialize_form_numbers =
    do_initialize_numbers;
  enddef;
\stopMPextensions

\startMPinitializations
  HSize:=\the\hsize ;
  VSize:=\the\vsize ;
\stopMPinitializations

\startMPextensions
  vardef ForegroundBox =
    unitsquare xysized(HSize,VSize)
  enddef ;
  vardef PageFraction =
    if \lastpage>1: (\realfolio-1)/(\lastpage-1) else: 1 fi
  enddef ;
\stopMPextensions

%D And some more. These are not really needed since we
%D don't use the normal figure inclusion macros any longer.

\appendtoks
  \externalfigurepostprocessors\emptytoks % safeguard
\to \everyinsertMPfile

%D We also take care of disabling fancy figure features, that
%D can terribly interfere when dealing with symbols,
%D background graphics and running (postponed) graphics.
%D You won't believe me if I tell you what funny side effects
%D can occur. One took me over a day to uncover when
%D processing the screen version of the \METAFUN\ manual.

%D For my eyes only:

\def\doifelseMPgraphic#1{\doifdefinedelse{\@@MPG#1}}

%D \macros
%D   {startMPcolor}
%D
%D The following time consuming method uses \METAPOST\ to
%D calculate a color. This enables a match between colors
%D resulting from a complex calculation (e.g. for a title
%D page) and those in the text.

% \startuseMPgraphic{somecolors}
%   color c[] ; c[1] := .7[red,green] ; c[2] := .7[blue,yellow] ;
% \stopuseMPgraphic

% \startMPcolor[shade-1][t=.2,a=1]
%   \includeMPgraphic{somecolors} ; fill fullcircle withcolor c[1] ;
% \stopMPcolor

% \startMPcolor[shade-2][t=.2,a=1]
%   \includeMPgraphic{somecolors} ; fill fullcircle withcolor c[2] ;
% \stopMPcolor

% \blackrule[width=\hsize,height=4cm,color=shade-1]
% \blackrule[width=\hsize,height=4cm,color=shade-2]

\def\startMPcolor
  {\dodoubleempty\dostartMPcolor}

\long\def\dostartMPcolor[#1][#2]#3\stopMPcolor % slow but sometimes handy
  {\startnointerference
     \def\handleMPgraycolor{\expanded{\defineglobalcolor[#1][s=\!MPgMPa1,#2]}}%
     \def\handleMPrgbcolor {\expanded{\defineglobalcolor[#1][r=\!MPgMPa1,g=\!MPgMPa2,b=\!MPgMPa3,#2]}}%
     \def\handleMPcmykcolor{\expanded{\defineglobalcolor[#1][c=\!MPgMPa1,m=\!MPgMPa2,y=\!MPgMPa3,k=\!MPgMPa4,#2]}}%
     \startMPcode#3\stopMPcode
   \stopnointerference}

%D New:

\definelayerpreset % no dx,dy - else nasty non-mp placement
  [mp]
  [\c!y=-\MPury bp,
   \c!x=\MPllx bp,
   \c!method=\v!fit]

\definelayer
  [mp]
  [\c!preset=mp]

%D Usage:
%D
%D \starttyping
%D \defineproperty[one][layer][state=start]
%D \defineproperty[two][layer][state=stop]
%D
%D \startuseMPgraphic{step-1}
%D   fill fullcircle scaled 10cm withcolor red ;
%D \stopuseMPgraphic
%D
%D \startuseMPgraphic{step-2}
%D   fill fullcircle scaled 5cm withcolor green ;
%D \stopuseMPgraphic
%D
%D \setlayer[mp]{\property[one]{\useMPgraphic{step-1}}}
%D \setlayer[mp]{\property[two]{\useMPgraphic{step-2}}}
%D
%D \ruledhbox{\flushlayer[mp]}
%D \stoptyping
%D
%D Reusing graphics is also possible (now):
%D
%D \starttyping
%D \startreusableMPgraphic{axis}
%D   tickstep := 1cm ; ticklength := 2mm ;
%D   drawticks unitsquare xscaled 4cm yscaled 3cm shifted (-1cm,-1cm) ;
%D   tickstep := tickstep/2 ; ticklength := ticklength/2 ;
%D   drawticks unitsquare xscaled 4cm yscaled 3cm shifted (-1cm,-1cm) ;
%D \stopreusableMPgraphic
%D
%D \startuseMPgraphic{demo}
%D   drawpoint "1cm,1.5cm" ;
%D \stopuseMPgraphic
%D
%D \definelayer[mp][preset=mp]
%D \setlayer[mp]{\reuseMPgraphic{axis}}
%D \setlayer[mp]{\useMPgraphic{demo}}
%D \ruledhbox{\flushlayer[mp]}
%D \stoptyping

%D \macros
%D   {startstaticMPfigure,useMPstaticfigure}
%D
%D Static figures are processed only when there has been
%D something changed. Here is Aditya Mahajan's testcase:
%D
%D \startbuffer
%D \startstaticMPfigure{circle}
%D   fill fullcircle scaled 1cm withcolor blue;
%D \stopstaticMPfigure
%D
%D \startstaticMPfigure{axis}
%D   drawarrow (0,0)--(2cm,0) ;
%D   drawarrow (0,0)--(0,2cm) ;
%D   label.llft(textext("(0,0)") ,origin) ;
%D \stopstaticMPfigure
%D \stopbuffer
%D
%D \typebuffer \getbuffer

\def\usestaticMPfigure[#1]%
  {\dodoubleempty\externalfigure[\jobname-#1.pdf]}

\def\startstaticMPfigure
  {\begingroup
   \obeyMPlines
   \dostartstaticMPfigure}

\def\dostartstaticMPfigure#1#2\stopstaticMPfigure
  {\startstaticMPgraphic{\jobname-#1}#2\stopstaticMPgraphic
   \endgroup}

% faster, but more tricky
%
% \def\startstaticMPfigure
%   {\doifmodeelse{*\v!first}
%      {\begingroup
%       \obeyMPlines
%       \dostartstaticMPfigure}
%      {\gobbleuntil\stopstaticMPfigure}}
%
% \def\dostartstaticMPfigure#1#2\stopstaticMPfigure
%   {\startMPstaticgraphic{\jobname-#1}#2\stopMPstaticgraphic
%    % dirty trick, don't register, so no second main run of texexec:
%    \global\advance\nofMPgraphics \minusone
%    \endgroup}}

%D New:

% \appendtoks \closeMPgraphicfiles \to \everystoptext

%D New:

\newconditional\manyMPspecials % when set to true, > 1000 specials can be used

\settrue \manyMPspecials % per 1/4/2006

\prependtoks
  _special_div_ := 1000\ifconditional\manyMPspecials0\fi ;
\to \MPextensions

%D Needed (will become default):

\prependtoks
  \resetlanguagespecifics
\to \everyMPgraphic

% Done.

\protect \endinput

%D Experimental:

\appendtoks
  \ifrunMPgraphics \ifcase\systemcommandmode \or
    \runMPgraphicsfalse
  \fi \fi
\to \everyjob

% also:
%
% linecap  := rounded ;
% linejoin := rounded ;
% drawoptions () ;
