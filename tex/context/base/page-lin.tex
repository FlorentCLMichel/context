%D \module
%D   [       file=page-lin, % copied from main-001
%D        version=1997.03.31, 
%D          title=\CONTEXT\ Core Macros,
%D       subtitle=Line Numbering,
%D         author=Hans Hagen,
%D           date=\currentdate,
%D      copyright={PRAGMA / Hans Hagen \& Ton Otten}]
%C
%C This module is part of the \CONTEXT\ macro||package and is
%C therefore copyrighted by \PRAGMA. See mreadme.pdf for
%C details.

\writestatus{loading}{Context Core Macros / Line Numbering}

\unprotect 

\newif\ifinregels
\newif\ifregelnummersinmarge

\def\stelregelsin%
  {\dodoubleargument\getparameters[\??rg]}

\def\startregels%
  {\@@rgvoor
   \witruimte
  %\pagina[\v!voorkeur]} gaat mis na koppen, nieuw: later \nobreak
   \begingroup
   \def\@@rgstepyes{\parindent\!!zeropoint}%
   \def\@@rgstepno{\parindent\!!zeropoint}%
   \edef\@@rgparindent{\the\parindent}%
   \gdef\@@rglinesteptoggle{1}%
   \processaction
     [\@@rginspringen]
     [    \v!ja=>\def\@@rgstepyes{\parindent\@@rgparindent}%
                 \def\@@rgstepno {\parindent\@@rgparindent},
      \v!oneven=>\def\@@rgstepyes{\parindent\!!zeropoint  }%
                 \def\@@rgstepno {\parindent\@@rgparindent},
        \v!even=>\def\@@rgstepno {\parindent\!!zeropoint  }%
                 \def\@@rgstepyes{\parindent\@@rgparindent}]%
   \inregelstrue
   \stelwitruimtein[\v!geen]%
   \obeylines
   \let\checkindentation=\relax
   \@@rgstepno
   \ignorespaces
   \gdef\afterfirstobeyedline% tzt two pass, net als opsomming
     {\gdef\afterfirstobeyedline%
        {\nobreak
         \global\let\afterfirstobeyedline\relax}}%
   \def\obeyedline%
     {\par
\let\checkindentation=\relax % else problems with odd/even 
      \afterfirstobeyedline
      \ifdim\lastskip>\zeropoint
        \gdef\@@rglinesteptoggle{0}%
      \else
        \doglobal\increment\@@rglinesteptoggle
      \fi
      \ifodd\@@rglinesteptoggle\relax
        \@@rgstepyes
      \else
        \@@rgstepno
      \fi
      \futurelet\next\dobetweenthelines}%
   \GotoPar}

% \def\dobetweenthelines%
%   {\convertcommand \next      \to\!!stringa % very ugly and fuzzy
%    \convertargument\obeyedline\to\!!stringb % but needed anyway
%    \ifx\!!stringa\!!stringb                 % but alas, it fails 
%      \@@rgtussen                            % hopelessly in non 
%    \fi}                                     % etex

\def\dobetweenthelines%
  {\doifmeaningelse{\next}{\obeyedline}{\@@rgtussen}{}}

\def\stopregels%
  {\endgroup
   \@@rgna}

\newcount\linenumber
\newcount\linestepper
\newif\ifinregelnummeren

% het gebruik van \setlocalreference scheelt een hash entry

\def\dodoshowlinenumber% for use elsewhere, to be extended 
  {\doschrijfregelnummer
   \global\advance\linenumber by 1\relax}%

\def\regelweergave%
  {\convertnumber\@@rnconversie\linenumber}%

\def\dostelregelnummerenin[#1]%
  {\getparameters
     [\??rn]
     [\c!start=1,
      \c!stap=1,
      #1]%
   \global\linenumber=1\relax}

\def\stelregelnummerenin%
  {\dosingleargument\dostelregelnummerenin}

\def\dostartnummerenLINE%                % !! \everypar !!
  {\EveryPar{\schrijfregelnummer}}

\def\dostopnummerenLINE%
  {\egroup}

\def\dodoschrijfregelnummer%
  {\setbox0=\hbox{\regelweergave}%
   \vsmashbox0%
   \ifregelnummersinmarge
     \llap{\hbox{\box0\hskip\linkermargeafstand}}%
   \else
     \rlap{\hbox to \@@rnbreedte{\box0\hss}}% was \llap, nog testen !! 
   \fi}

\def\complexstartregelnummeren[#1]%
  {\doifnotinset{\v!verder}{#1}
     {\global\linenumber=1\relax}%
   \doifinsetelse{\@@rnplaats}{\v!inmarge,\v!marge}
     {\regelnummersinmargetrue}
     {\regelnummersinmargefalse}%
   \ifregelnummersinmarge\else
     \advance\leftskip by \@@rnbreedte\relax
   \fi
   \ifinregels
     \let\dostartnummeren=\dostartnummerenLINE
     \let\stopregelnummeren=\dostopnummerenLINE
     \def\schrijfregelnummer%
       {\doschrijfregelnummer
        \global\advance\linenumber by 1\relax}%
   \else
     \let\dostartnummeren=\dostartnummerenPAR
     \let\stopregelnummeren=\dostopnummerenPAR
     \def\schrijfregelnummer%
       {\global\advance\linenumber by -1\relax
        \doschrijfregelnummer}%
   \fi
   \dostartnummeren}

\def\startregelnummeren%
  {\bgroup
   \inregelnummerentrue
   \complexorsimpleempty\startregelnummeren}

\def\doschrijfregelnummer%
  {\ifnum\linenumber<\@@rnstart\relax
   \else
     \!!counta=\linenumber
     \divide\!!counta by \@@rnstap\relax
     \multiply\!!counta by \@@rnstap\relax
     \ifnum\!!counta=\linenumber
       \doattributes\??rn\c!letter\c!kleur{\dodoschrijfregelnummer}%
     \fi
   \fi}

\def\eenregel[#1]%
  {\regelreferentie0[#1]\ignorespaces}

\def\startregel[#1]%
  {\regelreferentie1[#1]\ignorespaces}

\def\stopregel[#1]%
  {\removelastspace\regelreferentie2[#1]}

% \def\inregellabel#1%
%   {\doifinstringelse{--}{#1}
%      {\labeltext{\v!regels}}
%      {\labeltext{\v!regel}}}
% 
% \def\inregel#1[#2]%
%   {\doifelsenothing{#1}
%      {\in{\inregellabel{\currenttextreference}}[\@@rnprefix#2]}
%      {\in{#1}[\@@rnprefix#2]}}
%
% double labels: 

\def\inregel#1[#2]%
  {\doifelsenothing{#1}
     {\doifinstringelse{--}{\currenttextreference}
        {\in{\leftlabeltext\v!regels}{\rightlabeltext\v!regels}[\@@rnprefix#2]}
        {\in{\leftlabeltext\v!regel }{\rightlabeltext\v!regel }[\@@rnprefix#2]}}
     {\in{#1}[\@@rnprefix#2]}}

\def\dostartnummerenPAR%
  {\beginofshapebox
   \doglobal\newcounter\linereference}

% localcrossref heroverwegen

\def\setlinereference#1#2#3#4%
  {\setxvalue{lrf:#1}{\noexpand\dogetlinereference{#2}{#3}{#4}}}

\def\getlinereference#1%
  {\getvalue{lrf:#1}}

\def\dogetlinereference#1#2#3%
  {\edef\linereferencename{#1}%
   \edef\linereferenceline{#2}%
   \edef\linereferenceplus{#3}}

% 1 xxx xxx xxx xxx xxx xxx xxx
% 2 xxx yyy yyy yyy yyy yyy yyy <= start y
% 3 yyy yyy yyy yyy yyy yyy yyy
% 4 yyy yyy yyy yyy yyy xxx xxx <= stop y
% 5 xxx xxx xxx xxx xxx xxx xxx

%\def\regelreferentie#1[#2]%
%  {\bgroup
%   \dimen0=\dp\strutbox
%   \doif{\@@rnrefereren}{\v!aan}
%     {\doglobal\increment\linereference
%      % start 1=>(n=y,l=0,p=1)
%      % stop  2=>(n=y,l=0,p=2)
%      \setxvalue{lrf:n:\linereference}{\@@rnprefix#2}%
%      \setxvalue{lrf:l:\linereference}{0}%
%      \setxvalue{lrf:p:\linereference}{#1}%
%      \advance\dimen0 by \linereference sp}%
%   \prewordbreak
%   \vrule \!!width \!!zeropoint \!!depth \dimen0 \!!height \!!zeropoint
%   \prewordbreak
%   \egroup}

\def\regelreferentie#1[#2]%
  {\bgroup
   \dimen0=\dp\strutbox
   \doif{\@@rnrefereren}{\v!aan}
     {\doglobal\increment\linereference
      % start 1=>(n=y,l=0,p=1)
      % stop  2=>(n=y,l=0,p=2)
      \setlinereference{\linereference}{\@@rnprefix#2}{0}{#1}%
      \advance\dimen0 by \linereference sp}%
   \prewordbreak
   \vrule \!!width \!!zeropoint \!!depth \dimen0 \!!height \!!zeropoint
   \prewordbreak
   \egroup}

\def\dostopnummerenPAR% dp's -> openstrutdepth
  {\endofshapebox
   \checkreferences
   \linestepper=0
   \reshapebox{\global\advance\linestepper by 1\relax}%
   \global\advance\linenumber by \linestepper
   \doifelse{\@@rnrefereren}{\v!aan}
     {\reshapebox % We are going back!
        {\global\advance\linenumber by -1
         \dimen0=\dp\shapebox
         \advance\dimen0 by -\dp\strutbox\relax
         \ifdim\dimen0>\zeropoint
           % 1=>4 | 2=>4 1=>2
           % start 1=>(n=y,l=2,p=1)
           % stop  2=>(n=y,l=4,p=2)
           \dostepwiserecurse{1}{\number\dimen0}{1}
             {\getlinereference\recurselevel
              \setlinereference\recurselevel
                {\linereferencename}{\the\linenumber}{\linereferenceplus}}%
         \fi}%
      \global\advance\linenumber by \linestepper
      \ifnum\linereference>0 % anders vreemde loop in paragraphs+recurse
        \dorecurse{\linereference}
          {\getlinereference\recurselevel
           \ifnum\linereferenceplus=2 % stop
             % ref y: text = 4 / Kan dit buiten referentie mechanisme om?
             \expanded{\setlocalcrossreference
               {\referenceprefix\linereferencename}{}{}{\linereferenceline}}%
           \fi}%
        \dorecurse{\linereference}
          {\getlinereference\recurselevel
           \ifnum\linereferenceplus<2 % start / lone
             \ifnum\linereferenceplus=1 % start
               \getreferenceelements{\linereferencename}% text = 4
               \ifnum\linereferenceline<0\currenttextreference\relax % 0 prevents error
                 \edef\linereferenceline{\linereferenceline--\currenttextreference}%
               \fi
             \fi
             \expanded{\setlocalcrossreference
               {\referenceprefix\linereferencename}{}{}{\linereferenceline}}%
           \fi}%
        \global\let\scratchline=\linenumber  % We are going back!
        \reshapebox
          {\doglobal\decrement\scratchline
           \hbox
             {\dorecurse{\linereference}
                {\getlinereference\recurselevel
                 \getreferenceelements{\linereferencename}%
                 \beforesplitstring\currenttextreference--\at--\to\firstline
                 \ifnum\firstline=\scratchline\relax
                   % beter een rawtextreference
                   \textreference[\linereferencename]{\currenttextreference}%
                   \expanded{\setlocalcrossreference
                     {\referenceprefix\linereferencename}{}{}{0}}% ==done
                 \fi}%
              \dimen0=\dp\shapebox
              \advance\dimen0 by -\dp\strutbox\relax
              \ifdim\dimen0>\zeropoint
                \dp\shapebox=\dp\strutbox
              \fi
              \schrijfregelnummer\box\shapebox}}% no \strut !
      \else
        \reshapebox{\hbox{\schrijfregelnummer\box\shapebox}}% no \strut !
      \fi}
     {\reshapebox{\global\advance\linenumber by -1}%
      \global\advance\linenumber by \linestepper
      \reshapebox{\hbox{\schrijfregelnummer\box\shapebox}}}% no \strut !
   \global\advance\linenumber by \linestepper
   \flushshapebox
   \egroup}

\def\crlf%
  {\ifhmode\unskip\else\strut\fi\ifcase\raggedstatus\hfil\fi\break}

\def\opeenregel%
  {\def\crlf{\removelastspace\space}\let\\\crlf}

\newcount\internalparagraphnumber

\def\stelparagraafnummerenin%
  {\dosingleempty\dostelparagraafnummerenin}

\def\dostelparagraafnummerenin[#1]%
  {\getparameters
     [\??ph][#1]%
   \processaction
     [\@@phstatus]
     [\v!start=>\let\showparagraphnumber\doshowparagraphnumberA,
       \v!stop=>\let\showparagraphnumber\relax,
      \v!regel=>\let\showparagraphnumber\doshowparagraphnumberB,
      \v!reset=>\global\internalparagraphnumber=0  
                \let\showparagraphnumber\doshowparagraphnumberA]}

\def\dodoshowparagraphnumber%
  {\global\advance\internalparagraphnumber 1
   \inleftmargin % \tf normalizes em 
     {\tf{\doattributes\??ph\c!letter\c!kleur{\the\internalparagraphnumber}}%
      \kern\@@phafstand}}

\def\doshowparagraphnumberA%
  {\ifprocessingverbatim
     \iflinepar\dodoshowparagraphnumber\fi
   \else
     \dodoshowparagraphnumber
   \fi}

\def\doshowparagraphnumberB%
  {\ifinregelnummeren
     \doshowparagraphnumberA
   \fi}

\stelregelnummerenin
  [\c!conversie=\v!cijfers,
   \c!start=1,
   \c!stap=1,
   \c!plaats=\v!inmarge,
   \c!letter=,
   \c!kleur=,
   \c!breedte=2em,
   \c!prefix=,
   \c!refereren=\v!aan]

\stelparagraafnummerenin
  [\c!status=\v!stop,
   \c!letter=,
   \c!kleur=,
   \c!afstand=\ifregelnummersinmarge2em\else\!!zeropoint\fi] 

\stelregelsin
  [\c!voor=\blanko,
   \c!na=\blanko,
   \c!tussen=\blanko,
   \c!inspringen=\v!nee]

\protect \endinput 
