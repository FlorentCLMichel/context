%D \module
%D   [       file=page-lin, % copied from main-001
%D        version=1997.03.31,
%D          title=\CONTEXT\ Core Macros,
%D       subtitle=Line Numbering,
%D         author=Hans Hagen,
%D           date=\currentdate,
%D      copyright={PRAGMA / Hans Hagen \& Ton Otten}]
%C
%C This module is part of the \CONTEXT\ macro||package and is
%C therefore copyrighted by \PRAGMA. See mreadme.pdf for
%C details.

\writestatus{loading}{Context Core Macros / Line Numbering}

\unprotect

\newif\ifinregels % \newif\ifregelnummersinmarge

\chardef\linenumberlocation=0

\newtoks\beforeeverylinenumbering
\newtoks\aftereverylinenumbering

\def\stelregelsin
  {\dodoubleargument\getparameters[\??rg]}

\def\startregels
  {\@@rgvoor
   \witruimte
  %\pagina[\v!voorkeur]} gaat mis na koppen, nieuw: later \nobreak
   \begingroup
   \def\@@rgstepyes{\parindent\zeropoint}%
   \def\@@rgstepno {\parindent\zeropoint}%
   \edef\@@rgparindent{\the\parindent}%
   \globallet\@@rglinesteptoggle\!!plusone
   \processaction
     [\@@rginspringen]
     [    \v!ja=>\def\@@rgstepyes{\parindent\@@rgparindent}%
                 \def\@@rgstepno {\parindent\@@rgparindent},
      \v!oneven=>\def\@@rgstepyes{\parindent\zeropoint    }%
                 \def\@@rgstepno {\parindent\@@rgparindent},
        \v!even=>\def\@@rgstepno {\parindent\zeropoint    }%
                 \def\@@rgstepyes{\parindent\@@rgparindent}]%
   \inregelstrue
   \setupwhitespace[\v!geen]%
   \obeylines
   \let\checkindentation\relax
   \@@rgstepno
   \ignorespaces
   \gdef\afterfirstobeyedline % tzt two pass, net als opsomming
     {\gdef\afterfirstobeyedline
        {\nobreak
         \global\let\afterfirstobeyedline\relax}}%
   \def\obeyedline
     {\par
      \let\checkindentation\relax % else problems with odd/even
      \afterfirstobeyedline
      \ifdim\lastskip>\zeropoint
        \globallet\@@rglinesteptoggle\!!zerocount
      \else
        \doglobal\increment\@@rglinesteptoggle
      \fi
      \ifodd\@@rglinesteptoggle\relax
        \@@rgstepyes
      \else
        \@@rgstepno
      \fi
      \futurelet\next\dobetweenthelines}%
   \GotoPar}

% \def\dobetweenthelines%
%   {\convertcommand \next      \to\!!stringa % very ugly and fuzzy
%    \convertargument\obeyedline\to\!!stringb % but needed anyway
%    \ifx\!!stringa\!!stringb                 % but alas, it fails
%      \@@rgtussen                            % hopelessly in non
%    \fi}                                     % etex

\def\dobetweenthelines
  {\doifmeaningelse\next\obeyedline\@@rgtussen\donothing}

\def\stopregels
  {\endgroup
   \@@rgna}

\newcount\linenumber
\newcount\linestepper
\newif\ifinregelnummeren

% het gebruik van \setlocalreference scheelt een hash entry

\def\dodoshowlinenumber % for use elsewhere, to be extended
  {\doschrijfregelnummer
   \global\advance\linenumber \plusone}

\def\regelweergave
  {\@@rnlinks\convertnumber\@@rnconversie\linenumber\@@rnrechts}

\def\dostelregelnummerenin[#1]%
  {\getparameters[\??rn][\c!start=1,\c!stap=1,#1]%
   \global\linenumber\plusone}

\def\stelregelnummerenin
  {\dosingleargument\dostelregelnummerenin}

\def\dostartnummerenLINE
  {\EveryPar{\schrijfregelnummer}}

\def\dostopnummerenLINE
  {\the\aftereverylinenumbering
   \egroup}

\def\dostartnummerenVERB
  {\EveryLine{\schrijfregelnummer}}

\def\dostopnummerenVERB
  {\the\aftereverylinenumbering
   \egroup}

\newevery \everylinenumber \relax

\def\dodoschrijfregelnummer
  {% beware of em's, the font is already switched !
   \setbox\scratchbox\hbox
     {\setbox0\hbox{\@@rncommando{\regelweergave}}\vsmashbox0%
      \ifcase\linenumberlocation
        \rlap{\hbox to \@@rnbreedte{\box0\hss}}% was \llap, nog testen !!
      \or
        \inleftmargin
          {\forgetall
           \doifelse\@@rnbreedte\v!marge
             {\hsize\linkermargebreedte}{\hsize\@@rnbreedte}%
           \alignedline\@@rnuitlijnen\v!rechts{\box0\hskip\@@rnafstand}}%
      \else
        \inrightmargin
          {\forgetall
           \doifelse\@@rnbreedte\v!marge
             {\hsize\rechtermargebreedte}{\hsize\@@rnbreedte}%
           \alignedline\@@rnuitlijnen\v!links{\hskip\@@rnafstand\box0}}%
      \fi}%
   \vsmashbox\scratchbox
   \box\scratchbox
   \the\everylinenumber}

\def\complexstartregelnummeren[#1]%
  {\doifnot{#1}\v!verder
     {\doifnumberelse{#1}
        {\global\linenumber#1\relax}
        {\doifelsenothing\@@rnstart
           {\global\linenumber\plusone}
           {\global\linenumber\@@rnstart}}}%
   \chardef\linenumberlocation\zerocount
   \processaction
     [\@@rnplaats]
     [  \v!inmarge=>\chardef\linenumberlocation1,
       \v!inlinker=>\chardef\linenumberlocation1,
      \v!inrechter=>\chardef\linenumberlocation2,
          \v!marge=>\chardef\linenumberlocation1]%
       %  \v!tekst=>\chardef\linenumberlocation0,
       %\s!unknown=>\chardef\linenumberlocation0,
       %\s!default=>\chardef\linenumberlocation0]%
   \ifcase\linenumberlocation % text
     \advance\leftskip \@@rnbreedte\relax
   \fi
   \chardef\@@rn@@rnmethod
     \ifprocessingverbatim0\else\ifinregels1\else2\fi\fi
   \processaction
     [\@@rnmethode]
     [ \v!type=>\chardef\@@rn@@rnmethod0,
      \v!regel=>\chardef\@@rn@@rnmethod1,
      \v!tekst=>\chardef\@@rn@@rnmethod2,
       \v!file=>\chardef\@@rn@@rnmethod3]%
   \ifcase\@@rn@@rnmethod % verbatim, line by line
     \inregelstrue
     \let\dostartnummeren\dostartnummerenVERB
     \let\stopregelnummeren\dostopnummerenVERB
     \def\schrijfregelnummer
       {\doschrijfregelnummer
        \global\advance\linenumber \plusone}%
   \or % text, line by line
     \let\dostartnummeren\dostartnummerenLINE
     \let\stopregelnummeren\dostopnummerenLINE
     \def\schrijfregelnummer
       {\doschrijfregelnummer
        \global\advance\linenumber \plusone}%
   \or % text, whole lot
     \let\dostartnummeren\dostartnummerenPAR
     \let\stopregelnummeren\dostopnummerenPAR
     \def\schrijfregelnummer
       {\global\advance\linenumber \minusone
        \doschrijfregelnummer}%
   \or % verbatim, selective line by line
     \inregelstrue
     \let\dostartnummeren\dostartnummerenVERB
     \let\stopregelnummeren\dostopnummerenVERB
     \def\schrijfregelnummer
       {\global\linenumber\verbatimlinenumber
        \doschrijfregelnummer}%
   \fi
   \dostartnummeren}

\def\startregelnummeren
  {\bgroup
   \the\beforeeverylinenumbering
   \inregelnummerentrue
   \complexorsimpleempty\startregelnummeren}

\def\doschrijfregelnummer
  {\ifnum\linenumber<\@@rnstart\relax
   \else
     \!!counta\linenumber
     \divide\!!counta \@@rnstap
     \multiply\!!counta \@@rnstap\relax
     \ifnum\!!counta=\linenumber
       \doattributes\??rn\c!letter\c!kleur\dodoschrijfregelnummer
     \fi
   \fi}

\def\eenregel[#1]%
  {\regelreferentie0[#1]\ignorespaces}

\def\startregel[#1]%
  {\regelreferentie1[#1]\ignorespaces}

\def\stopregel[#1]%
  {\removelastspace\regelreferentie2[#1]}

% \def\inregellabel#1%
%   {\doifinstringelse{--}{#1}
%      {\labeltext{\v!regels}}
%      {\labeltext{\v!regel}}}
%
% \def\inregel#1[#2]%
%   {\doifelsenothing{#1}
%      {\in{\inregellabel{\currenttextreference}}[\@@rnprefix#2]}
%      {\in{#1}[\@@rnprefix#2]}}
%
% double labels:

\def\inregel#1[#2]%
  {\doifelsenothing{#1}
     {\doifinstringelse{--}\currenttextreference
        {\in{\leftlabeltext\v!regels}{\rightlabeltext\v!regels}[\@@rnprefix#2]}
        {\in{\leftlabeltext\v!regel }{\rightlabeltext\v!regel }[\@@rnprefix#2]}}
     {\in{#1}[\@@rnprefix#2]}}

\def\dostartnummerenPAR
  {\beginofshapebox
   \doglobal\newcounter\linereference}

% localcrossref heroverwegen

\def\setlinereference#1#2#3#4%
  {\setxvalue{lrf:#1}{\noexpand\dogetlinereference{#2}{#3}{#4}}}

\def\getlinereference#1%
  {\getvalue{lrf:#1}}

\def\dogetlinereference#1#2#3%
  {\edef\linereferencename{#1}%
   \edef\linereferenceline{#2}%
   \edef\linereferenceplus{#3}}

% 1 xxx xxx xxx xxx xxx xxx xxx
% 2 xxx yyy yyy yyy yyy yyy yyy <= start y
% 3 yyy yyy yyy yyy yyy yyy yyy
% 4 yyy yyy yyy yyy yyy xxx xxx <= stop y
% 5 xxx xxx xxx xxx xxx xxx xxx

%\def\regelreferentie#1[#2]%
%  {\bgroup
%   \dimen0\strutdp
%   \doif{\@@rnrefereren}{\v!aan}
%     {\doglobal\increment\linereference
%      % start 1=>(n=y,l=0,p=1)
%      % stop  2=>(n=y,l=0,p=2)
%      \setxvalue{lrf:n:\linereference}{\@@rnprefix#2}%
%      \setxvalue{lrf:l:\linereference}{0}%
%      \setxvalue{lrf:p:\linereference}{#1}%
%      \advance\dimen0 by \linereference sp}%
%   \prewordbreak
%   \vrule \!!width \!!zeropoint \!!depth \dimen0 \!!height \!!zeropoint
%   \prewordbreak
%   \egroup}

\def\regelreferentie#1[#2]%
  {\bgroup
   \dimen0=\strutdp
   \doif\@@rnrefereren\v!aan
     {\doglobal\increment\linereference
      % start 1=>(n=y,l=0,p=1)
      % stop  2=>(n=y,l=0,p=2)
      \setlinereference{\linereference}{\@@rnprefix#2}{0}{#1}%
      \advance\dimen0 \linereference sp}%
   \prewordbreak
   \vrule \!!width \zeropoint \!!depth \dimen0 \!!height \zeropoint
   \prewordbreak
   \egroup}

\def\dostopnummerenPAR % dp's -> openstrutdepth
  {\endofshapebox
   \checkreferences
   \linestepper\zerocount
   \reshapebox{\global\advance\linestepper \plusone}%
   \global\advance\linenumber \linestepper
   \doifelse\@@rnrefereren\v!aan
     {\reshapebox % We are going back!
        {\global\advance\linenumber \minusone
         \dimen0=\dp\shapebox
         \advance\dimen0 -\strutdp\relax
         \ifdim\dimen0>\zeropoint
           % 1=>4 | 2=>4 1=>2
           % start 1=>(n=y,l=2,p=1)
           % stop  2=>(n=y,l=4,p=2)
           \dostepwiserecurse\plusone{\number\dimen0}\plusone
             {\getlinereference\recurselevel
              \setlinereference\recurselevel
                {\linereferencename}{\the\linenumber}{\linereferenceplus}}%
         \fi}%
      \global\advance\linenumber \linestepper
      \ifnum\linereference>\zerocount % anders vreemde loop in paragraphs+recurse
        \dorecurse\linereference
          {\getlinereference\recurselevel
           \ifnum\linereferenceplus=2 % stop
             % ref y: text = 4 / Kan dit buiten referentie mechanisme om?
             \expanded{\setlocalcrossreference
               {\referenceprefix\linereferencename}{}{}{\linereferenceline}}%
           \fi}%
        \dorecurse\linereference
          {\getlinereference\recurselevel
           \ifnum\linereferenceplus<2 % start / lone
             \ifnum\linereferenceplus=1 % start
               \getreferenceelements\linereferencename % text = 4
               \ifnum\linereferenceline<0\currenttextreference\relax % 0 prevents error
                 \edef\linereferenceline{\linereferenceline--\currenttextreference}%
               \fi
             \fi
             \expanded{\setlocalcrossreference
               {\referenceprefix\linereferencename}{}{}{\linereferenceline}}%
           \fi}%
        \global\let\scratchline\linenumber  % We are going back!
        \reshapebox
          {\doglobal\decrement\scratchline
           \hbox
             {\dorecurse\linereference
                {\getlinereference\recurselevel
                 \getreferenceelements\linereferencename
                 \beforesplitstring\currenttextreference--\at--\to\firstline
                 \ifnum\firstline=\scratchline\relax
                   % beter een rawtextreference, i.e. expanded
                   % \textreference[\linereferencename]{\currenttextreference}%
                   \rawtextreference\s!lin\linereferencename\currenttextreference
                   \expanded{\setlocalcrossreference
                     {\referenceprefix\linereferencename}{}{}{0}}% ==done
                 \fi}%
              \dimen0\dp\shapebox
              \advance\dimen0 -\strutdp\relax
              \ifdim\dimen0>\zeropoint
                \dp\shapebox\strutdp
              \fi
              \schrijfregelnummer\box\shapebox}}% no \strut !
      \else
        \reshapebox{\hbox{\schrijfregelnummer\box\shapebox}}% no \strut !
      \fi}
     {\reshapebox{\global\advance\linenumber \minusone}%
      \global\advance\linenumber \linestepper
      \reshapebox{\hbox{\schrijfregelnummer\box\shapebox}}}% no \strut !
   \global\advance\linenumber \linestepper
   \flushshapebox
   \the\aftereverylinenumbering
   \egroup}

\def\emptylines
  {\dosingleempty\doemptylines}

\def\doemptylines[#1]%
  {\endgraf\dorecurse{\iffirstargument#1\else3\fi}\crlf}

\newcount\internalparagraphnumber

\def\stelparagraafnummerenin%
  {\dosingleempty\dostelparagraafnummerenin}

\def\dostelparagraafnummerenin[#1]%
  {\getparameters
     [\??ph][#1]%
   \processaction
     [\@@phstatus]
     [\v!start=>\let\showparagraphnumber\doshowparagraphnumberA,
       \v!stop=>\let\showparagraphnumber\relax,
      \v!regel=>\let\showparagraphnumber\doshowparagraphnumberB,
      \v!reset=>\global\internalparagraphnumber\zerocount
                \let\showparagraphnumber\doshowparagraphnumberA]}

\def\dodoshowparagraphnumber
  {\global\advance\internalparagraphnumber \plusone
   \inleftmargin % \tf normalizes em
     {\tf{\doattributes\??ph\c!letter\c!kleur{\the\internalparagraphnumber}}%
      \kern\@@phafstand}}

\def\doshowparagraphnumberA
  {\ifprocessingverbatim
     \iflinepar\dodoshowparagraphnumber\fi
   \else
     \dodoshowparagraphnumber
   \fi}

\def\doshowparagraphnumberB
  {\ifinregelnummeren
     \doshowparagraphnumberA
   \fi}

\stelregelnummerenin
  [\c!methode=,
   \c!conversie=\v!cijfers,
   \c!start=1,
   \c!stap=1,
   \c!plaats=\v!marge,
   \c!letter=,
   \c!kleur=,
   \c!breedte=2em,
   \c!prefix=,
   \c!refereren=\v!aan]

% new

\stelregelnummerenin
  [\c!breedte=\ifcase\linenumberlocation2em\else\v!marge\fi,
   \c!links=,
   \c!rechts=,
   \c!commando=,
   \c!afstand=\zeropoint,
   \c!uitlijnen=\ifcase\linenumberlocation\v!rechts\or\v!rechts\or\v!links\fi]

\stelparagraafnummerenin
  [\c!status=\v!stop,
   \c!letter=,
   \c!kleur=,
   \c!afstand=\ifcase\linenumberlocation2em\else\!!zeropoint\fi]

\stelregelsin
  [\c!voor=\blanko,
   \c!na=\blanko,
   \c!tussen=\blanko,
   \c!inspringen=\v!nee]

\protect \endinput