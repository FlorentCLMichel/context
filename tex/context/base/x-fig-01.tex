%D \module
%D   [      file=x-fig-01,
%D        version=2001.03.21,
%D          title=\CONTEXT\ Style File,
%D       subtitle=Figure Base Generation,
%D         author=Hans Hagen,
%D           date=\currentdate,
%D      copyright={PRAGMA / Hans Hagen \& Ton Otten}]
%C
%C This module is part of the \CONTEXT\ macro||package and is
%C therefore copyrighted by \PRAGMA. See mreadme.pdf for 
%C details. 

%D See \type {x-fig-00.tex} for more information on how to use 
%D and generate figure databases. This file loads the file 
%D named \type {\inputfilename} (\TEXEXEC\ will set this 
%D variable). You can apply this style to a database by 
%D saying:
%D 
%D \starttypen 
%D texexec --pdf --use=fig-01 yourfile.xml
%D \stoptypen 
%D 
%D The following modes are supported: 
%D 
%D \starttabulatie[|lT|l|]
%D \NC letter  \NC map the preview on letter size  \NC \NR   
%D \NC compact \NC use an alternative presentation \NC \NR   
%D \stoptabulatie
%D
%D The resulting file has the following characteristics: 
%D
%D \startopsomming[opelkaar]
%D \som  the document is split into three sections: first each
%D       figure is shown at its own page, then an overview of 
%D       figures is shown with some data alongside, and 
%D       finally an index and table of contents shows up 
%D \som  there is no title page, which means that one can 
%D       access a figure by page number without offset
%D \som  the document is opened at the first overview page, 
%D       that is, when the viewer supports it 
%D \som  the graphic is shown 3~times: on a page of its own,
%D       scaled to a fixed dimension, and relative to a4 or 
%D       letter paper size 
%D \som  the labels can be accessed in an index and list at 
%D       the end of the document
%D \stopopsomming
%D
%D We use named destinations, which means that one can 
%D access a figure by name from an external application.

\usemodule[fig-00]

\overcomePDFspacefalse

\doifnothing  {\inputfilename}        {\end} 
\doiffileelse {\inputfilename.xml} {} {\end} 

\def\StartDescription%
  {\bgroup
   \defineXMLpush[organization]
   \defineXMLpush[project]
   \defineXMLpush[product]
   \defineXMLpush[comment]}

\def\StopDescription%
  {\subject [begin] {Figure collection}
   \starttabulate[|lBe|p|]
   \doifXMLdataelse{organization}
     {\NC organization \NC \XMLpop{organization} \NC \NR}{}
   \doifXMLdataelse{project}
     {\NC project      \NC \XMLpop{project}      \NC \NR}{}
   \doifXMLdataelse{product}
     {\NC product      \NC \XMLpop{product}      \NC \NR}{}
   \doifXMLdataelse{comment}
     {\NC comment      \NC \XMLpop{comment}      \NC \NR}{}
   \stoptabulate
   \blank[2*big]
   \egroup}

\def\StartFigureA%
  {\bgroup
   \XMLassign{file}{buffer}
   \defineXMLpush[file]
   \defineXMLpush[dummy]
   \defineXMLpush[label]
   \defineXMLpush[copyright]
   \defineXMLpush[comment]
   \defineXMLpush[status]}

\defineoverlay[page][\overlaybutton{Description}]

\startbuffer
  \framed
    [width=\XMLpar{dummy}{width}{12cm},
     height=\XMLpar{dummy}{height}{8cm},
     background=color,
     backgroundcolor=gray,
     foregroundcolor=darkred,
     frame=off]
    {\bf \XMLpop{dummy}}    
\stopbuffer

\def\StopFigureA%
  {\doglobal\increment\CurrentPage
   \setupbackgrounds[page][background=page]
   \doifelsenothing{\XMLpop{label}}
     {\expanded{\definereference[Description][about: \XMLpop{file}]}%
      \pagereference[\XMLpop{file}]}
     {\expanded{\definereference[Description][about: \XMLpop{label}]}%
      \pagereference[\XMLpop{label}]}
   \pagefigure[\XMLpop{file}]
   \setupbackgrounds[page][background=]
   \egroup}

\def\StartFigureB%
  {\StartFigureA}

\defineregister
  [figureindex]
  [figureindices]

\setupregister
  [figureindex]
  [ownnumber=yes,
   criterium=text,
   interaction=text,
   indicator=no]

\definelist
  [figurelist]

\setuplist
  [figurelist]
  [criterium=text,
   pagenumber=no,
   width=2em,
   interaction=all]

\setupcolors
  [state=start]

\setuptolerance
  [verytolerant]

% the next hack will be replaced by a layout definition

\startbuffer[paper]
\startnotmode[letter]
  \framed
    [width=210mm,height=297mm,offset=10pt,frame=off,
     background=color,backgroundcolor=white]
    {\externalfigure[\XMLpop{file}][reset=yes]}
\stopnotmode
\startmode[letter]
  \framed
    [width=8.5in,height=11in,offset=10pt,frame=off,
     background=color,backgroundcolor=white]
    {\externalfigure[\XMLpop{file}][reset=yes]}
\stopmode
\stopbuffer

\setupbuttons
  [offset=10pt,
   width=broad,
   strut=no,
   rulethickness=1pt,
   framecolor=darkred]

\def\StopFigureB%
  {\par
   \doglobal\increment\CurrentPage
   \doifelsenothing{\XMLpop{label}}
     {\expanded{\definereference[Figure][\XMLpop{file}]}}
     {\expanded{\definereference[Figure][\XMLpop{label}]}}%
   \button
     {\hbox to \hsize
        {\forgetall \dontcomplain
         \doifelsenothing{\XMLpop{label}}
           {\pagereference[about: \XMLpop{file}]}
           {\pagereference[about: \XMLpop{label}]}%
         \expanded{\writetolist[figurelist]{\CurrentPage}{\XMLpop{label}}}%
         \expanded{\figureindex{\CurrentPage}{\XMLpop{label}}}%
         \startnotmode[compact]%
           \vbox to 100pt
             {\hsize30pt
              \vskip5pt 
              \hbox to \hsize{\hss\strut\bf\CurrentPage\hss}%
              \vfill}%
           \advance\hsize by -30pt
         \stopnotmode
         \startmode[compact]%
           \advance\hsize by -10pt
           \hskip10pt
         \stopmode
         \framed
           [width=150pt,height=100pt,offset=10pt,frame=off,
            background=color,backgroundcolor=white]
           {\externalfigure
              [\XMLpop{file}]
              [maxheight=80pt,frame=off,maxwidth=130pt,factor=max]}%
         \freezedimenmacro\naturalfigurewidth \let\FigWid\naturalfigurewidth
         \freezedimenmacro\naturalfigureheight\let\FigHei\naturalfigureheight
         \advance\hsize by -150pt
         \hskip10pt
         \advance\hsize by -10pt
         \vbox to 100pt
           {\hsize40pt
            \externalfigure
              [paper]
              [type=buffer,frame=on,
               framecolor=darkred,rulethickness=.5pt,
               width=40pt,object=no]
            \startmode[compact]%
              \vfill
              \hbox to \hsize{\hss\strut\bf\CurrentPage\hss}%
            \stopmode
            \vfill}%
         \advance\hsize by -40pt
         \hskip10pt
         \advance\hsize by -10pt
         \vbox to 100pt
           {\blank[disable]
            \starttabulate[|Bel|p|]
            \NC file \NC \XMLpop{file} \NC \NR
            \doifXMLdataelse{label}
              {\NC label \NC \XMLpop{label} \NC \NR}     
              {}
            \NC w$\times$h \NC \FigWid$\times$\FigHei \NC \NR
            \doifXMLdataelse{copyright}
              {\NC copyright \NC \XMLpop{copyright} \NC \NR}
              {}
            \doifXMLdataelse{status}
              {\doifelse{\XMLpop{status}}{obsolete}
                 {\NC status \NC \bf\darkred\XMLpop{status} \NC \NR}
                 {\NC status \NC \XMLpop{status} \NC \NR}}
              {}
            \doifXMLdataelse{comment}
              {\NC comment \NC \XMLpop{comment} \NC \NR}    
              {}
            \stoptabulate
            \vfill}}}%
      [Figure]
   \vskip10pt
   \egroup}

\def\StartFigureC%
  {\StartFigureA}

\def\StopFigureC%
  {\doglobal\increment\NumberOfFigures
   \egroup}

\setuplayout
  [topspace=15pt,backspace=15pt,
   header=0pt,footer=0pt,bottom=20pt,bottomdistance=10pt,
   width=middle,height=fit]

\setupbackgrounds
  [page]
  [background=,
   backgroundcolor=gray]

\setupinteractionscreen
  [width=max,
   height=max]

\setupcolors     
  [state=start]

\setupinteraction
  [style=,
   color=,
   contrastcolor=,
   state=start]

\setuphead
  [section]
  [style=bfb]

\setupbodyfont   
  [pos]

\setupinteractionmenu
  [bottom]
  [left=\hfill,
   middle=\hskip10pt,
   frame=off,
   style=bold,
   background=color,
   backgroundcolor=darkred,
   foregroundcolor=white]

\startinteractionmenu[bottom]
  \but [begin]         begin   \\
  \but [index]         index   \\
  \but [list]          list    \\
  \but [CloseDocument] close   \\
  \but [PreviousJump]  go back \\
\stopinteractionmenu

\setupinteraction
  [openaction=page(\NumberOfFigures)]

\defineXMLenvironment [figurelibrary] \StartLibrary \StopLibrary

\starttext 

\def\StartLibrary{\mainlanguage[\XMLpar{figurelibrary}{language}{en}]}
\def\StopLibrary {}

\defineXMLignore      
  [description]

\defineXMLenvironment 
  [figure]       
  \StartFigureC      
  \StopFigureC

\doglobal\newcounter\CurrentPage

\processXMLfilegrouped{\inputfilename.xml} 

\increment\NumberOfFigures

\defineXMLignore      
  [description]

\defineXMLenvironment 
  [figure]      
  \StartFigureA 
  \StopFigureA

\doglobal\newcounter\CurrentPage

\processXMLfilegrouped{\inputfilename.xml} 

\setuppapersize
  [S6][S6]

\setupbackgrounds
  [page]
  [background=color]

\setupinteraction
  [menu=on]

\defineXMLenvironment [description] \StartDescription \StopDescription
\defineXMLenvironment [figure]      \StartFigureB     \StopFigureB

\doglobal\newcounter\CurrentPage

\processXMLfilegrouped{\inputfilename.xml} \page

\subject [list] {List of figures}

\placelist[figurelist] \page

\subject [index] {Index of figures}

\startcolumns
\placeregister[figureindex]
\stopcolumns

\stoptext 
