%D \module
%D   [       file=math-tex,
%D        version=2001.04.12,
%D       subtitle=Plain Specials,
%D         author={Hans Hagen, Taco Hoekwater \& Aditya Mahajan},
%D           date=\currentdate,
%D      copyright=\PRAGMA]
%C
%C This module is part of the \CONTEXT\ macro||package and is
%C therefore copyrighted by \PRAGMA. See mreadme.pdf for
%C details.

% beware: in mkiv we will do it differently

\unprotect

\mathcode`\ ="8000 \mathcode`\_="8000 \mathcode`\'="8000

\definefamilysynonym [default] [calligraphic] [sy]
\definefamilysynonym [default] [oldstyle]     [mi]
\definefamilysynonym [default] [blackboard]   [mr]
\definefamilysynonym [default] [gothic]       [mr]
\definefamilysynonym [default] [fraktur]      [mr]

\definefamilysynonym [default] [lcgreek]      [mi]
\definefamilysynonym [default] [ucgreek]      [mr]
\definefamilysynonym [default] [vargreek]     [mi]


%D \macros
%D   {\setupmathematics}
%D
%D Configuration for integrals. (If needed we can speed this up and make it
%D installable; no processaction is needed then).

\chardef\intlimitcode\zerocount  % 0 nolimits 1 displaylimits 2 limits

\def\intlimits
  {\ifcase\intlimitcode \nolimits \or \displaylimits \or \limits \fi}

\def\setupmathematics
  {\dosingleargument\dosetupmathematics}

\def\dosetupmathematics[#1]%
  {\getparameters[\??mo][#1]
   \processaction[\@@mointegral]
      [ nolimits=>\chardef\intlimitcode\zerocount,
   displaylimits=>\chardef\intlimitcode\plusone,
          limits=>\chardef\intlimitcode\plustwo]}

%D \startbuffer
%D   $\int_a^b f(x) dx$ and also
%D   $\iint_a^b f(x,y) dxdy$, $\iiint_a^b f(x,y) dxdy$,
%D   $\iiiint_a^b f(x) dx$
%D   \startformula
%D     \int_a^b f(x) dx \quad
%D     \iint_a^b f(x) dx \quad
%D     \iiint_a^b f(x) dx \quad
%D     \iiiint_a^b f(x) dx \quad
%D   \stopformula
%D \stopbuffer
%D
%D Default: \getbuffer
%D
%D Displaylimits: \setupmathematics[integral=displaylimits] \getbuffer
%D
%D Limits: \setupmathematics[integral=limits] \getbuffer

\startmathcollection [default]

\definemathsymbol [alpha]   [nothing] [lcgreek] ["0B]
\definemathsymbol [beta]    [nothing] [lcgreek] ["0C]
\definemathsymbol [gamma]   [nothing] [lcgreek] ["0D]
\definemathsymbol [delta]   [nothing] [lcgreek] ["0E]
\definemathsymbol [epsilon] [nothing] [lcgreek] ["0F]
\definemathsymbol [zeta]    [nothing] [lcgreek] ["10]
\definemathsymbol [eta]     [nothing] [lcgreek] ["11]
\definemathsymbol [theta]   [nothing] [lcgreek] ["12]
\definemathsymbol [iota]    [nothing] [lcgreek] ["13]
\definemathsymbol [kappa]   [nothing] [lcgreek] ["14]
\definemathsymbol [lambda]  [nothing] [lcgreek] ["15]
\definemathsymbol [mu]      [nothing] [lcgreek] ["16]
\definemathsymbol [nu]      [nothing] [lcgreek] ["17]
\definemathsymbol [xi]      [nothing] [lcgreek] ["18]
\definemathsymbol [omicron] [nothing] [lcgreek] ["6F]
\definemathsymbol [pi]      [nothing] [lcgreek] ["19]
\definemathsymbol [rho]     [nothing] [lcgreek] ["1A]
\definemathsymbol [sigma]   [nothing] [lcgreek] ["1B]
\definemathsymbol [tau]     [nothing] [lcgreek] ["1C]
\definemathsymbol [upsilon] [nothing] [lcgreek] ["1D]
\definemathsymbol [phi]     [nothing] [lcgreek] ["1E]
\definemathsymbol [chi]     [nothing] [lcgreek] ["1F]
\definemathsymbol [psi]     [nothing] [lcgreek] ["20]
\definemathsymbol [omega]   [nothing] [lcgreek] ["21]

\stopmathcollection

\startmathcollection [default]

\definemathsymbol [varepsilon] [nothing] [vargreek] ["22]
\definemathsymbol [vartheta]   [nothing] [vargreek] ["23]
\definemathsymbol [varpi]      [nothing] [vargreek] ["24]
\definemathsymbol [varrho]     [nothing] [vargreek] ["25]
\definemathsymbol [varsigma]   [nothing] [vargreek] ["26]
\definemathsymbol [varphi]     [nothing] [vargreek] ["27]

\stopmathcollection

\startmathcollection [default]

\definemathsymbol [Alpha]   [alpha] [ucgreek] ["41] % A
\definemathsymbol [Beta]    [alpha] [ucgreek] ["42] % B
\definemathsymbol [Gamma]   [alpha] [ucgreek] ["00]
\definemathsymbol [Delta]   [alpha] [ucgreek] ["01]
\definemathsymbol [Epsilon] [alpha] [ucgreek] ["45] % E
\definemathsymbol [Zeta]    [alpha] [ucgreek] ["5A] % Z
\definemathsymbol [Eta]     [alpha] [ucgreek] ["48] % H
\definemathsymbol [Theta]   [alpha] [ucgreek] ["02]
\definemathsymbol [Iota]    [alpha] [ucgreek] ["49] % I
\definemathsymbol [Kappa]   [alpha] [ucgreek] ["4B] % K
\definemathsymbol [Lambda]  [alpha] [ucgreek] ["03]
\definemathsymbol [Mu]      [alpha] [ucgreek] ["4D] % M
\definemathsymbol [Nu]      [alpha] [ucgreek] ["4E] % N
\definemathsymbol [Xi]      [alpha] [ucgreek] ["04]
\definemathsymbol [Omicron] [alpha] [ucgreek] ["4F] % O
\definemathsymbol [Pi]      [alpha] [ucgreek] ["05]
\definemathsymbol [Rho]     [alpha] [ucgreek] ["52] % R
\definemathsymbol [Sigma]   [alpha] [ucgreek] ["06]
\definemathsymbol [Tau]     [alpha] [ucgreek] ["54] % T
\definemathsymbol [Upsilon] [alpha] [ucgreek] ["07]
\definemathsymbol [Phi]     [alpha] [ucgreek] ["08]
\definemathsymbol [Chi]     [alpha] [ucgreek] ["58] % X
\definemathsymbol [Psi]     [alpha] [ucgreek] ["09]
\definemathsymbol [Omega]   [alpha] [ucgreek] ["0A]

\stopmathcollection

% The \mfunction macro is an alternative for \hbox with a
% controlable font switch.

\startmathcollection[default]

\definemathcommand [arccos]  [nolop] {\mfunction{arccos}}
\definemathcommand [arcsin]  [nolop] {\mfunction{arcsin}}
\definemathcommand [arctan]  [nolop] {\mfunction{arctan}}
\definemathcommand [arg]     [nolop] {\mfunction{arg}}
\definemathcommand [cosh]    [nolop] {\mfunction{cosh}}
\definemathcommand [cos]     [nolop] {\mfunction{cos}}
\definemathcommand [coth]    [nolop] {\mfunction{coth}}
\definemathcommand [cot]     [nolop] {\mfunction{cot}}
\definemathcommand [csc]     [nolop] {\mfunction{csc}}
\definemathcommand [deg]     [nolop] {\mfunction{deg}}
\definemathcommand [det]     [limop] {\mfunction{det}}
\definemathcommand [dim]     [nolop] {\mfunction{dim}}
\definemathcommand [exp]     [nolop] {\mfunction{exp}}
\definemathcommand [gcd]     [limop] {\mfunction{gcd}}
\definemathcommand [hom]     [nolop] {\mfunction{hom}}
\definemathcommand [inf]     [limop] {\mfunction{inf}}
\definemathcommand [injlim]  [limop] {\mfunction{inj\,lim}}
\definemathcommand [ker]     [nolop] {\mfunction{ker}}
\definemathcommand [lg]      [nolop] {\mfunction{lg}}
\definemathcommand [liminf]  [limop] {\mfunction{lim\,inf}}
\definemathcommand [limsup]  [limop] {\mfunction{lim\,sup}}
\definemathcommand [lim]     [limop] {\mfunction{lim}}
\definemathcommand [ln]      [nolop] {\mfunction{ln}}
\definemathcommand [log]     [nolop] {\mfunction{log}}
\definemathcommand [median]  [limop] {\mfunction{median}}
\definemathcommand [max]     [limop] {\mfunction{max}}
\definemathcommand [min]     [limop] {\mfunction{min}}
\definemathcommand [mod]     [limop] {\mfunction{mod}}
\definemathcommand [div]     [limop] {\mfunction{div}}
\definemathcommand [projlim] [limop] {\mfunction{proj\,lim}}
\definemathcommand [Pr]      [limop] {\mfunction{Pr}}
\definemathcommand [sec]     [nolop] {\mfunction{sec}}
\definemathcommand [sinh]    [nolop] {\mfunction{sinh}}
\definemathcommand [sin]     [nolop] {\mfunction{sin}}
\definemathcommand [sup]     [limop] {\mfunction{sup}}
\definemathcommand [tanh]    [nolop] {\mfunction{tanh}}
\definemathcommand [tan]     [nolop] {\mfunction{tan}}

\stopmathcollection

\startmathcollection[default]

\definemathcommand [integers]       {\mfunction{Z}}
\definemathcommand [reals]          {\mfunction{R}}
\definemathcommand [rationals]      {\mfunction{Q}}
\definemathcommand [naturalnumbers] {\mfunction{N}}
\definemathcommand [complexes]      {\mfunction{C}}
\definemathcommand [primes]         {\mfunction{P}}

\stopmathcollection

\startmathcollection[default]

\definemathcharacter [!]   [close] [mr] ["21]
\definemathcharacter [(]   [open]  [mr] ["28]
\definemathcharacter [)]   [close] [mr] ["29]
\definemathcharacter [*]   [bin]   [sy] ["03]
\definemathcharacter [+]   [bin]   [mr] ["2B]
\definemathcharacter [,]   [punct] [mi] ["3B]
\definemathcharacter [-]   [bin]   [sy] ["00]
\definemathcharacter [.]   [ord]   [mi] ["3A]
\definemathcharacter [/]   [ord]   [mi] ["3D]
\definemathcharacter [:]   [rel]   [mr] ["3A]
\definemathcharacter [;]   [punct] [mr] ["3B]
\definemathcharacter [<]   [rel]   [mi] ["3C]
\definemathcharacter [=]   [rel]   [mr] ["3D]
\definemathcharacter [>]   [rel]   [mi] ["3E]
\definemathcharacter [?]   [close] [mr] ["3F]
\definemathcharacter [91]  [open]  [mr] ["5B] % [
\definemathcharacter [92]  [ord]   [sy] ["6E] % \
\definemathcharacter [93]  [close] [mr] ["5D] % ]
\definemathcharacter [123] [open]  [sy] ["66] % {
\definemathcharacter [124] [ord]   [sy] ["6A] % |
\definemathcharacter [125] [close] [sy] ["67] % }

\stopmathcollection

\startmathcollection[default]

\definemathcharacter [(]   [nothing] [mr] ["28] [ex] ["00]
\definemathcharacter [)]   [nothing] [mr] ["29] [ex] ["01]
\definemathcharacter [91]  [nothing] [mr] ["5B] [ex] ["02] % [
\definemathcharacter [93]  [nothing] [mr] ["5D] [ex] ["03] % ]
\definemathcharacter [<]   [nothing] [sy] ["68] [ex] ["0A]
\definemathcharacter [>]   [nothing] [sy] ["69] [ex] ["0B]
\definemathcharacter [/]   [nothing] [mr] ["2F] [ex] ["0E]
\definemathcharacter [124] [nothing] [sy] ["6A] [ex] ["0C] % |
\definemathcharacter [92]  [nothing] [sy] ["6E] [ex] ["0F] % \

\stopmathcollection

\def\PLAINangle
  {{\vbox{\ialign{$\m@th\scriptstyle##$\crcr
    \not\mathrel{\mkern14mu}\crcr
    \noalign{\nointerlineskip}
    \mkern2.5mu\leaders\hrule height.34pt\hfill\mkern2.5mu\crcr}}}}

\startmathcollection[default]

\definemathsymbol [aleph]    [nothing] [sy] ["40]
\definemathsymbol [imath]    [nothing] [mi] ["7B]
\definemathsymbol [jmath]    [nothing] [mi] ["7C]
\definemathsymbol [ell]      [nothing] [mi] ["60]
\definemathsymbol [wp]       [nothing] [mi] ["7D]
\definemathsymbol [Re]       [nothing] [sy] ["3C]
\definemathsymbol [Im]       [nothing] [sy] ["3D]
\definemathsymbol [partial]  [nothing] [mi] ["40]
\definemathsymbol [infty]    [nothing] [sy] ["31]
\definemathsymbol [prime]    [nothing] [sy] ["30]
\definemathsymbol [emptyset] [nothing] [sy] ["3B]
\definemathsymbol [nabla]    [nothing] [sy] ["72]
\definemathsymbol [top]      [nothing] [sy] ["3E]
\definemathsymbol [bot]      [nothing] [sy] ["3F]

\definemathcommand [hbar]  {{\mathchar'26\mkern-9muh}}
\definemathcommand [surd]  {{\mathchar"1270}}          % ?
\definemathcommand [angle] {\PLAINangle}

\stopmathcollection

\startmathcollection[default]

\definemathsymbol [triangle]    [ord] [sy] ["34]
\definemathsymbol [forall]      [ord] [sy] ["38]
\definemathsymbol [exists]      [ord] [sy] ["39]
\definemathsymbol [neg]         [ord] [sy] ["3A]
\definemathsymbol [flat]        [ord] [mi] ["5B]
\definemathsymbol [natural]     [ord] [mi] ["5C]
\definemathsymbol [sharp]       [ord] [mi] ["5D]
\definemathsymbol [clubsuit]    [ord] [sy] ["7C]
\definemathsymbol [diamondsuit] [ord] [sy] ["7D]
\definemathsymbol [heartsuit]   [ord] [sy] ["7E]
\definemathsymbol [spadesuit]   [ord] [sy] ["7F]

\stopmathcollection

\startmathcollection[default]

\definemathcommand [square] {\hbox{\hsmash{$\sqcup$}$\sqcap$}}

\stopmathcollection

\startmathcollection[default]

\definemathsymbol [coprod]    [op]  [ex] ["60]
\definemathsymbol [bigvee]    [op]  [ex] ["57]
\definemathsymbol [bigwedge]  [op]  [ex] ["56]
\definemathsymbol [biguplus]  [op]  [ex] ["55]
\definemathsymbol [bigcap]    [op]  [ex] ["54]
\definemathsymbol [bigcup]    [op]  [ex] ["53]
\definemathsymbol [intop]     [op]  [ex] ["52]
\definemathsymbol [prod]      [op]  [ex] ["51]
\definemathsymbol [sum]       [op]  [ex] ["50]
\definemathsymbol [bigotimes] [op]  [ex] ["4E]
\definemathsymbol [bigoplus]  [op]  [ex] ["4C]
\definemathsymbol [bigodot]   [op]  [ex] ["4A]
\definemathsymbol [ointop]    [op]  [ex] ["48]
\definemathsymbol [bigsqcup]  [op]  [ex] ["46]
\definemathsymbol [smallint]  [op]  [sy] ["73]

\stopmathcollection

\startmathcollection[default]

\definemathsymbol [triangleleft]    [bin] [mi] ["2F]
\definemathsymbol [triangleright]   [bin] [mi] ["2E]
\definemathsymbol [bigtriangleup]   [bin] [sy] ["34]
\definemathsymbol [bigtriangledown] [bin] [sy] ["35]
\definemathsymbol [wedge]           [bin] [sy] ["5E]
\definemathsymbol [vee]             [bin] [sy] ["5F]
\definemathsymbol [cap]             [bin] [sy] ["5C]
\definemathsymbol [cup]             [bin] [sy] ["5B]
\definemathsymbol [ddagger]         [bin] [sy] ["7A]
\definemathsymbol [dagger]          [bin] [sy] ["79]
\definemathsymbol [sqcap]           [bin] [sy] ["75]
\definemathsymbol [sqcup]           [bin] [sy] ["74]
\definemathsymbol [uplus]           [bin] [sy] ["5D]
\definemathsymbol [amalg]           [bin] [sy] ["71]
\definemathsymbol [diamond]         [bin] [sy] ["05]
\definemathsymbol [bullet]          [bin] [sy] ["0F]
\definemathsymbol [wr]              [bin] [sy] ["6F]
\definemathsymbol [div]             [bin] [sy] ["04]
\definemathsymbol [odot]            [bin] [sy] ["0C]
\definemathsymbol [oslash]          [bin] [sy] ["0B]
\definemathsymbol [otimes]          [bin] [sy] ["0A]
\definemathsymbol [ominus]          [bin] [sy] ["09]
\definemathsymbol [oplus]           [bin] [sy] ["08]
\definemathsymbol [mp]              [bin] [sy] ["07]
\definemathsymbol [pm]              [bin] [sy] ["06]
\definemathsymbol [circ]            [bin] [sy] ["0E]
\definemathsymbol [bigcirc]         [bin] [sy] ["0D]
\definemathsymbol [setminus]        [bin] [sy] ["6E]
\definemathsymbol [cdot]            [bin] [sy] ["01]
\definemathsymbol [ast]             [bin] [sy] ["03]
\definemathsymbol [times]           [bin] [sy] ["02]
\definemathsymbol [star]            [bin] [mi] ["3F]

\stopmathcollection

\startmathcollection[default]

\definemathsymbol [propto]         [rel] [sy] ["2F]
\definemathsymbol [sqsubseteq]     [rel] [sy] ["76]
\definemathsymbol [sqsupseteq]     [rel] [sy] ["77]
\definemathsymbol [parallel]       [rel] [sy] ["6B]
\definemathsymbol [mid]            [rel] [sy] ["6A]
\definemathsymbol [dashv]          [rel] [sy] ["61]
\definemathsymbol [vdash]          [rel] [sy] ["60]
\definemathsymbol [nearrow]        [rel] [sy] ["25]
\definemathsymbol [searrow]        [rel] [sy] ["26]
\definemathsymbol [nwarrow]        [rel] [sy] ["2D]
\definemathsymbol [swarrow]        [rel] [sy] ["2E]
\definemathsymbol [Leftrightarrow] [rel] [sy] ["2C]
\definemathsymbol [Leftarrow]      [rel] [sy] ["28]
\definemathsymbol [Rightarrow]     [rel] [sy] ["29]

\stopmathcollection

\startmathcollection[default]

\definemathcommand [lnot]   {\neg}
\definemathcommand [int]    {\intop \intlimits}
\definemathcommand [oint]   {\ointop\intlimits}
\definemathcommand [land]   {\wedge}
\definemathcommand [lor]    {\vee}
\definemathcommand [neq]    {\not=}
\definemathcommand [ne]     {\neq}
\definemathcommand [le]     {\leq}
\definemathcommand [ge]     {\geq}
\definemathcommand [eq]     {=}
\definemathcommand [gt]     {>}
\definemathcommand [lt]     {<}
\definemathcommand [gets]   {\leftarrow}
\definemathcommand [owns]   {\ni}
\definemathcommand [to]     {\rightarrow}
\definemathcommand [mapsto] {\mapstochar\rightarrow}

\stopmathcollection

\startmathcollection[default]

\definemathsymbol [leq]              [rel] [sy] ["14]
\definemathsymbol [geq]              [rel] [sy] ["15]
\definemathsymbol [succ]             [rel] [sy] ["1F]
\definemathsymbol [prec]             [rel] [sy] ["1E]
\definemathsymbol [approx]           [rel] [sy] ["19]
\definemathsymbol [succeq]           [rel] [sy] ["17]
\definemathsymbol [preceq]           [rel] [sy] ["16]
\definemathsymbol [supset]           [rel] [sy] ["1B]
\definemathsymbol [subset]           [rel] [sy] ["1A]
\definemathsymbol [supseteq]         [rel] [sy] ["13]
\definemathsymbol [subseteq]         [rel] [sy] ["12]
\definemathsymbol [in]               [rel] [sy] ["32]
\definemathsymbol [ni]               [rel] [sy] ["33]
\definemathsymbol [gg]               [rel] [sy] ["1D]
\definemathsymbol [ll]               [rel] [sy] ["1C]
\definemathsymbol [not]              [rel] [sy] ["36]
\definemathsymbol [leftrightarrow]   [rel] [sy] ["24]
\definemathsymbol [leftarrow]        [rel] [sy] ["20]
\definemathsymbol [rightarrow]       [rel] [sy] ["21]
\definemathsymbol [mapstochar]       [rel] [sy] ["37]
\definemathsymbol [sim]              [rel] [sy] ["18]
\definemathsymbol [simeq]            [rel] [sy] ["27]
\definemathsymbol [perp]             [rel] [sy] ["3F]
\definemathsymbol [equiv]            [rel] [sy] ["11]
\definemathsymbol [asymp]            [rel] [sy] ["10]
\definemathsymbol [smile]            [rel] [mi] ["5E]
\definemathsymbol [frown]            [rel] [mi] ["5F]
\definemathsymbol [leftharpoonup]    [rel] [mi] ["28]
\definemathsymbol [leftharpoondown]  [rel] [mi] ["29]
\definemathsymbol [rightharpoonup]   [rel] [mi] ["2A]
\definemathsymbol [rightharpoondown] [rel] [mi] ["2B]
\definemathsymbol [lhook]            [rel] [mi] ["2C]
\definemathsymbol [rhook]            [rel] [mi] ["2D]

\stopmathcollection

\def\PLAINldots
  {\ldotp\ldotp\ldotp}

\def\PLAINcdots
  {\cdotp\cdotp\cdotp}

\def\PLAINvdots
  {\vbox{\baselineskip4\p@ \lineskiplimit\z@
   \kern6\p@\hbox{.}\hbox{.}\hbox{.}}}

\def\PLAINddots
   {\mkern1mu\raise7\p@\vbox{\kern7\p@\hbox{.}}\mkern2mu
    \raise4\p@\hbox{.}\mkern2mu\raise\p@\hbox{.}\mkern1mu}

\startmathcollection[default]

\definemathcommand [hookrightarrow] {\lhook\joinrel\rightarrow}
\definemathcommand [hookleftarrow]  {\leftarrow\joinrel\rhook}
\definemathcommand [bowtie]         {\mathrel\triangleright\joinrel\mathrel\triangleleft}
\definemathcommand [models]         {\mathrel|\joinrel=}
\definemathcommand [iff]            {\;\Longleftrightarrow\;}

\definemathsymbol [ldotp] [punct] [mi] ["3A]
\definemathsymbol [cdotp] [punct] [sy] ["01]
\definemathsymbol [colon] [punct] [mr] ["3A]

\definemathcommand [ldots] [inner]   {\PLAINldots}
\definemathcommand [cdots] [inner]   {\PLAINcdots}
\definemathcommand [vdots] [nothing] {\PLAINvdots}
\definemathcommand [ddots] [inner]   {\PLAINddots}

\stopmathcollection

\startmathcollection[default]

\definemathsymbol [acute]     [accent] [mr] ["13]
\definemathsymbol [grave]     [accent] [mr] ["12]
\definemathsymbol [ddot]      [accent] [mr] ["7F]
\definemathsymbol [tilde]     [accent] [mr] ["7E]
\definemathsymbol [mathring]  [accent] [mr] ["17]
\definemathsymbol [bar]       [accent] [mr] ["16]
\definemathsymbol [breve]     [accent] [mr] ["15]
\definemathsymbol [check]     [accent] [mr] ["14]
\definemathsymbol [hat]       [accent] [mr] ["5E]
\definemathsymbol [vec]       [accent] [mi] ["7E] % [ord]
\definemathsymbol [dot]       [accent] [mr] ["5F]
\definemathsymbol [widetilde] [accent] [ex] ["65] % [ord]
\definemathsymbol [widehat]   [accent] [ex] ["62] % [ord]

\stopmathcollection

\startmathcollection[default]

\definemathsymbol [lmoustache]  [open]    [ex] ["7A] [ex] ["40]
\definemathsymbol [rmoustache]  [close]   [ex] ["7B] [ex] ["41]
\definemathsymbol [lgroup]      [open]    [mr] ["28] [ex] ["3A] % ?
\definemathsymbol [rgroup]      [close]   [mr] ["29] [ex] ["3B] % ?
\definemathsymbol [arrowvert]   [nothing] [sy] ["6A] [ex] ["3C]
\definemathsymbol [Arrowvert]   [nothing] [sy] ["6B] [ex] ["3D]
\definemathsymbol [bracevert]   [nothing] [ex] ["3E] % ?
\definemathsymbol [Vert]        [nothing] [sy] ["6B] [ex] ["0D]
\definemathsymbol [vert]        [nothing] [sy] ["6A] [ex] ["0C]
\definemathsymbol [uparrow]     [rel]     [sy] ["22] [ex] ["78]
\definemathsymbol [downarrow]   [rel]     [sy] ["23] [ex] ["79]
\definemathsymbol [updownarrow] [rel]     [sy] ["6C] [ex] ["3F]
\definemathsymbol [Uparrow]     [rel]     [sy] ["2A] [ex] ["7E]
\definemathsymbol [Downarrow]   [rel]     [sy] ["2B] [ex] ["7F]
\definemathsymbol [Updownarrow] [rel]     [sy] ["6D] [ex] ["77]
\definemathsymbol [backslash]   [nothing] [sy] ["6E] [ex] ["0F]
\definemathsymbol [rangle]      [close]   [sy] ["69] [ex] ["0B]
\definemathsymbol [langle]      [open]    [sy] ["68] [ex] ["0A]
\definemathsymbol [rbrace]      [close]   [sy] ["67] [ex] ["09]
\definemathsymbol [lbrace]      [open]    [sy] ["66] [ex] ["08]
\definemathsymbol [rceil]       [close]   [sy] ["65] [ex] ["07]
\definemathsymbol [lceil]       [open]    [sy] ["64] [ex] ["06]
\definemathsymbol [rfloor]      [close]   [sy] ["63] [ex] ["05]
\definemathsymbol [lfloor]      [open]    [sy] ["62] [ex] ["04]

\definemathsymbol [sqrt]        [radical] [sy] ["70] [ex] ["70]

\stopmathcollection

%D By request:

\startmathcollection[default]

\definemathsymbol [lvert] [open]  [sy] ["6A] [ex] ["0C]
\definemathsymbol [rvert] [close] [sy] ["6A] [ex] ["0C]

\definemathsymbol [lVert] [open]  [sy] ["6B] [ex] ["0D]
\definemathsymbol [rVert] [close] [sy] ["6B] [ex] ["0D]

\stopmathcollection

%D For brooks:
%D
%D \starttyping
%D $\sqrt[3]{10}$
%D \stoptyping

\ifx\normalsqrt\undefined  \let\normalsqrt\sqrt \fi      % just set to: \dohandlemathtoken {sqrt}

\def\notsosqrt[#1]{\root#1\of}

\unexpanded\def\sqrt{\doifnextcharelse[\notsosqrt\normalsqrt}

\def\PLAINbig {\@@dobig{0.85}}
\def\PLAINBig {\@@dobig{1.15}}
\def\PLAINbigg{\@@dobig{1.45}}
\def\PLAINBigg{\@@dobig{1.75}}

\startrawmathcollection [default]

\definemathcommand [bigl]  [open]  [one] {\big}
\definemathcommand [bigm]  [rel]   [one] {\big}
\definemathcommand [bigr]  [close] [one] {\big}
\definemathcommand [Bigl]  [open]  [one] {\Big}
\definemathcommand [Bigm]  [rel]   [one] {\Big}
\definemathcommand [Bigr]  [close] [one] {\Big}
\definemathcommand [biggl] [open]  [one] {\bigg}
\definemathcommand [biggm] [rel]   [one] {\bigg}
\definemathcommand [biggr] [close] [one] {\bigg}
\definemathcommand [Biggl] [open]  [one] {\Bigg}
\definemathcommand [Biggm] [rel]   [one] {\Bigg}
\definemathcommand [Biggr] [close] [one] {\Bigg}

\definemathcommand [big]  {\PLAINbig}
\definemathcommand [Big]  {\PLAINBig}
\definemathcommand [bigg] {\PLAINbigg}
\definemathcommand [Bigg] {\PLAINBigg}

\stoprawmathcollection

\startmathcollection[default]

\definemathsymbol [dag]  [box] [sy] ["79]
\definemathsymbol [ddag] [box] [sy] ["7A]
\definemathsymbol [S]    [box] [sy] ["78]
\definemathsymbol [P]    [box] [sy] ["7B]
\definemathsymbol [Orb]  [box] [sy] ["0D]

\stopmathcollection

\def\PLAINroot#1#2%
  {\setbox\z@\hbox{$\m@th#1\sqrt{#2}$}\dimen@\ht\z@
   \advance\dimen@-\dp\z@
   \mkern5mu\raise.6\dimen@\copy\rootbox \mkern-10mu\box\z@}

\def\PLAINmatrix#1%
  {\null\,\vcenter{\normalbaselines\m@th
   \ialign{\hfil$##$\hfil&&\quad\hfil$##$\hfil\crcr
   \mathstrut\crcr\noalign{\kern-\baselineskip}
   #1\crcr\mathstrut\crcr\noalign{\kern-\baselineskip}}}\,}

\startrawmathcollection[default]

\definemathcommand [mathstrut] {\vphantom{(}}
\definemathcommand [joinrel]   {\mathrel{\mkern-3mu}}
\definemathcommand [r@@t]      {\PLAINroot}
\definemathcommand [matrix]    {\PLAINmatrix}

\definemathcommand [over]      {\normalover} % hack, to do

\stoprawmathcollection

\def\{{\lbrace}
\def\}{\rbrace}

%def\bbd{\fam\purefamily{blackboard}}
%def\cal{\fam\purefamily{calfamily}}

\def\mit{\fam\purefamily{mitfamily}}

\def\Bbb{\blackboard} % conforming amstex

\startmathcollection[default]

  \definemathsymbol [mathperiod] [ord]   [mi] ["3A]
  \definemathsymbol [textperiod] [punct] [mi] ["3A]

  \definemathsymbol [mathcomma]  [ord]   [mi] ["3B]
  \definemathsymbol [textcomma]  [punct] [mi] ["3B]

\stopmathcollection

\definemathpunctuation . mathperiod textperiod
\definemathpunctuation , mathcomma  textcomma

%D The following colon related definitions are provided by Aditya
%D Mahajan who derived them from \type {mathtools.sty} and \type
%D {colonequals.sty}.

%D \macros
%D   {centercolon, colonminus, minuscolon, colonequals, equalscolon,
%D    colonapprox, approxcolon, colonsim, simcolon, coloncolon,
%D    coloncolonminus, minuscoloncolon, coloncolonequals,
%D    equalscoloncolon, coloncolonapprox, approxcoloncolon,
%D    colonsim, simcoloncolon}
%D
%D In $a := b$ the colon is not vertically centered with the equal
%D to. Also the distance between colon and equal to is a bit large.
%D So, we define a vertically centered colon \tex {centercolon} and
%D a few macros for colon and double colon relation symbols.
%D
%D \startlines
%D \formula {A \centercolon       B}
%D \formula {A \colonminus        B}
%D \formula {A \minuscolon        B}
%D \formula {A \colonequals       B}
%D \formula {A \equalscolon       B}
%D \formula {A \colonapprox       B}
%D \formula {A \approxcolon       B}
%D \formula {A \colonsim          B}
%D \formula {A \simcolon          B}
%D \formula {A \coloncolon        B}
%D \formula {A \coloncolonminus   B}
%D \formula {A \minuscoloncolon   B}
%D \formula {A \coloncolonequals  B}
%D \formula {A \equalscoloncolon  B}
%D \formula {A \coloncolonapprox  B}
%D \formula {A \approxcoloncolon  B}
%D \formula {A \colonsim          B}
%D \formula {A \simcoloncolon     B}
%D \stoplines

%D The next macros take care of the space between the colon and the
%D relation symbol.

\definemathcommand  [colonsep]        {\mkern-1.2mu}
\definemathcommand  [doublecolonsep]  {\mkern-0.9mu}

%D The next macro vertically centeres its contents.

\def\@center@math#1%
  {\vcenter{\hbox{$\m@th#1$}}}

\def\@center@colon
  {\mathpalette\@center@math{\colon}}

%D Now we define all the colon relations.

\definemathcommand [centercolon]      [rel] {\@center@colon}
\definemathcommand [colonminus]       [rel] {\centercolon\colonsep\mathrel{-}}
\definemathcommand [minuscolon]       [rel] {\mathrel{-}\colonsep\centercolon}
\definemathcommand [colonequals]      [rel] {\centercolon\colonsep=}
\definemathcommand [equalscolon]      [rel] {=\centercolon\colonsep}
\definemathcommand [colonapprox]      [rel] {\centercolon\colonsep\approx}
\definemathcommand [approxcolon]      [rel] {\approx\centercolon\colonsep}
\definemathcommand [colonsim]         [rel] {\centercolon\colonsep\sim}
\definemathcommand [simcolon]         [rel] {\sim\centercolon\colonsep}

\definemathcommand [coloncolon]       [rel] {\centercolon\doublecolonsep\centercolon}
\definemathcommand [coloncolonminus]  [rel] {\coloncolon\colonsep\mathrel{-}}
\definemathcommand [minuscoloncolon]  [rel] {\mathrel{-}\colonsep\coloncolon}
\definemathcommand [coloncolonequals] [rel] {\coloncolon\colonsep=}
\definemathcommand [equalscoloncolon] [rel] {=\coloncolon\colonsep}
\definemathcommand [coloncolonapprox] [rel] {\coloncolon\colonsep\approx}
\definemathcommand [approxcoloncolon] [rel] {\approx\coloncolon\colonsep}
\definemathcommand [colonsim]         [rel] {\coloncolon\colonsep\sim}
\definemathcommand [simcoloncolon]    [rel] {\sim\coloncolon\colonsep}

%D More integrals (AM):

\def\dointkern{\mkern-6mu\mathchoice{\mkern-3mu}{}{}{}}

\definemathcommand [iint]   {\repeatintegral\plusone  }
\definemathcommand [iiint]  {\repeatintegral\plustwo  }
\definemathcommand [iiiint] {\repeatintegral\plusthree}

\def\repeatintegral#1%
  {\scratchtoks\emptytoks
   \let\dointlimits\donothing
   \let\dodointlimits\intlimits
   \dorecurse{#1}{\appendtoks \intop \dointkern \to \scratchtoks}
   \appendtoks \intop \dointlimits \dodointlimits \to \scratchtoks
   \edef\dodorepeatintegral{\the\scratchtoks}%
   \futurelet\next\dorepeatintegral}

%D If the \type{\limits} option is used after \type{\iint}, use
%D \type{\mathop} and fudge the left hand space a bit to make the
%D subscript visually centered.

\def\dorepeatintegral
   {\ifx\next\limits          \dointlimitcorrection \else
    \ifx\next\displaylimits   \dointlimitcorrection \else
    \ifx\next\nolimits        \donothing            \else
    \ifcase\intlimitcode\else \dointlimitcorrection \fi\fi\fi\fi
    \dodorepeatintegral}

\def\dointlimitcorrection
  {\mkern-7mu\mathchoice{\mkern-2mu}{}{}{}%
   \mathop\bgroup
     \mkern7mu\mathchoice{\mkern2mu}{}{}{}%
     \let\dointlimits\egroup}


\setupmathematics
  [integral=nolimits]

\protect \endinput
