%D \module
%D   [       file=type-pre,
%D        version=2001.04.12,
%D          title=\CONTEXT\ Typescript Macros,
%D       subtitle=Compatibility scripts,
%D         author=Hans Hagen,
%D           date=\currentdate,
%D      copyright={PRAGMA / Hans Hagen \& Ton Otten}]
%C
%C This module is part of the \CONTEXT\ macro||package and is
%C therefore copyrighted by \PRAGMA. See mreadme.pdf for
%C details.

%D This file defines some typescripts that simulate the pre-typescript way
%D of defining fonts. This file will not be extended.

\starttypescriptcollection[previous]

%D The Computer Modern Roman is derived from the Monotype~8a
%D Times Roman. In this module, that is loaded by default, we
%D define all relevant alternatives.

\starttypescript [cmr]
  \usetypescript [serif,sans,mono,math] [computer-modern,latin-modern] [default,name,size]
  \usemathcollection[default]
\stoptypescript

%D This script remaps the default Computer Modern Font Files
%D onto the EC ones, so that hyphenations work well. (The proper
%D latin modern ec variants have replaed the ae ones.)

\starttypescript [aer]
  \usetypescript [serif,sans,mono,math] [computer-modern,latin-modern] [default,name,size,ec]
  \usemathcollection[default]
\stoptypescript

%D This script defines the Computer Modern Roman with iso
%D latin 2 encoding, as needed for Czech and other languages.

\starttypescript [csr]
  \usetypescript [serif,sans,mono,math] [computer-modern,latin-modern] [default,name,size,il2]
  \usemathcollection[default]
\stoptypescript

%D This script defines the Computer Modern Roman with a
%D polish encoding, as needed for Czech and other languages.

\starttypescript [plr]
  \usetypescript [serif,sans,mono,math] [computer-modern,latin-modern] [default,name,size,pl0]
  \usemathcollection[default]
\stoptypescript

%D Vietnamese.

\starttypescript [vnr]
  \usetypescript [serif,sans,mono,math] [computer-modern,latin-modern] [default,name,size,t5]
  \usemathcollection[default]
\stoptypescript

%D Cyrillic alternatives are available under the symbolic
%D name \type {cyr}.

\starttypescript [cyr]
  \usetypescript [serif,sans,mono,math] [computer-modern,latin-modern] [default,name,size,cyr]
  \usemathcollection[default]
\stoptypescript

\starttypescript [lh-ec]
  \usetypescript [serif,sans,mono,math] [computer-modern,latin-modern] [default,name,size,ec]
  \usemathcollection[default]
\stoptypescript

\starttypescript [lh-t2a]
  \usetypescript [serif,sans,mono,math] [computer-modern,latin-modern] [default,name,size,t2a]
  \usemathcollection[default]
\stoptypescript

\starttypescript [lh-t2b]
  \usetypescript [serif,sans,mono,math] [computer-modern,latin-modern] [default,name,size,t2b]
  \usemathcollection[default]
\stoptypescript

\starttypescript [lh-t2c]
  \usetypescript [serif,sans,mono,math] [computer-modern,latin-modern] [default,name,size,t2c]
  \usemathcollection [default]
\stoptypescript

\starttypescript [lh-x2]
  \usetypescript [serif,sans,mono,math] [computer-modern,latin-modern] [default,name,size,x2]
  \usemathcollection [default]
\stoptypescript

\starttypescript [lh-lcy]
  \usetypescript [serif,sans,mono,math] [computer-modern,latin-modern] [default,name,size,lcy]
  \usemathcollection [default]
\stoptypescript

%D Here we implement the symbol fonts as provided by the
%D American Mathematical Society. The names of these symbols
%D can be found in The Joy of \TeX\ by M.~Spivak.

\starttypescript [ams]
  \usetypescript [math] [ams] [all]
  \usemathcollection[default]
\stoptypescript

%D The Concrete Modern Roman is just an alternative Computer
%D Modern Roman.

\starttypescript [con]
  \usetypescript [serif] [concrete] [all]
  \usemathcollection[default]
\stoptypescript

%D The Euler Fonts are designed by Herman Zapf and can be
%D used with the Concrete Fonts defined elsewhere.

\starttypescript [eul]
  \usetypescript [math] [euler] [all]
  \usemathcollection[eul]
\stoptypescript

%D The Lucida Bright fonts are both good looking and and
%D complete. These fonts have prebuilt accented characters,
%D which means that we use another encoding vector: \YandY\
%D texnansi. These fonts are a good illustration that a 12
%D point bodyfont is indeed never that size. The Lucida Bright
%D fonts come in one design size.

\starttypescript [lbr]
  \usetypescript [serif,sans,mono,math,calligraphy,handwriting] [lucida]  [name,special,\defaultencoding]
  \usetypescript [serif,sans,mono,math,calligraphy,handwriting] [default] [size]
  \usemathcollection[lbr]
  \usetypescript [all] [lucida] [\defaultencoding]
\stoptypescript

% %D With thanks to Berthold Horn from YandY for providing me
% %D evaluation copies of the MathTimePlus fonts.

% todo: \starttypescript [mt,tim]

\starttypescript [tim]
  \usetypescript [math] [times]   [all]
  \usetypescript [math] [default] [size]
  \usemathcollection[tim]
  \usetypescript [all] [times] [\defaultencoding]
\stoptypescript

%D The Antikwa Torunska font family is a rather nice
%D and subtle one. Although primary meant for the polish
%D language, it can be used for other languages as well.

\starttypescript [ant]
  \usetypescript [serif] [antykwa-torunska] [name,\defaultencoding]
  \usetypescript [serif] [default]          [size]
  \usetypescript [all] [antykwa-torunska] [\defaultencoding]
\stoptypescript

%D This script defines the Standard Adobe Courier fonts.

\starttypescript [pcr]
  \usetypescript [mono] [courier] [name,\defaultencoding]
  \usetypescript [mono] [default] [size]
  \usetypescript [all] [courier] [\defaultencoding]
\stoptypescript

%D This script defines the Standard Adobe Helvetica fonts.

\starttypescript [phv]
  \usetypescript [sans] [helvetica] [name,\defaultencoding]
  \usetypescript [sans] [default]   [size]
  \usetypescript [all] [helvetica] [\defaultencoding]
\stoptypescript

%D This script defines the Standard Adobe Times fonts.

\starttypescript [ptm]
  \usetypescript [serif] [times]   [name,\defaultencoding]
  \usetypescript [serif] [default] [size]
  \usetypescript [all] [times] [\defaultencoding]
\stoptypescript

%D This script loads the Adobe Times Roman, Helvetica and
%D Courier.

\starttypescript [pos]
  \usetypescript [serif] [times]     [name,\defaultencoding]
  \usetypescript [sans]  [helvetica] [name,\defaultencoding]
  \usetypescript [mono]  [courier]   [name,\defaultencoding]
  \usetypescript [serif,sans,mono] [default] [size]
% \usetypescript [math] [times]   [all]
% \usetypescript [math] [default] [size]
  \usetypescript [all] [times,helvetica,courier] [\defaultencoding]
\stoptypescript

%D This script defines the Palatino font.

\starttypescript [ppl]
  \usetypescript [serif] [palatino] [name,\defaultencoding]
  \usetypescript [serif] [default]  [size]
  \usetypescript [all]   [palatino] [\defaultencoding]
\stoptypescript

%D The following scripts fake the old \type {font-ber} and
%D alike files.

\starttypescript[fil]
  % fake to prevent loading font-fil.tex and signal 'done'
\stoptypescript

%D But for old time sake we provide:

\starttypescript[ber]
  \usetypescript [berry] [ec,t5] % could be [all]
\stoptypescript

\stoptypescriptcollection

\endinput
