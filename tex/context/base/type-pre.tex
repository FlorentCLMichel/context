%D \module
%D   [       file=type-pre,
%D        version=2001.04.12, 
%D          title=\CONTEXT\ Typescript Macros,
%D       subtitle=Compatibility scripts, 
%D         author=Hans Hagen,
%D           date=\currentdate,
%D      copyright={PRAGMA / Hans Hagen \& Ton Otten}]
%C
%C This module is part of the \CONTEXT\ macro||package and is
%C therefore copyrighted by \PRAGMA. See mreadme.pdf for
%C details.

%D The Computer Modern Roman is derived from the Monotype~8a
%D Times Roman. In this module, that is loaded by default, we
%D define all relevant alternatives.

\starttypescript [cmr] 

\usetypescript [all] [computer-modern] [default,name,size]

\enablemathcollection[default]

\stoptypescript

%D This script remaps the default Computer Modern Font Files
%D onto the virtual EC ones, so that hyphenations work well.

\starttypescript [aer]

\usetypescript [all] [computer-modern] [default,name,size,ec]

\enablemathcollection[default]

\stoptypescript

%D This script defines the Computer Modern Roman with iso
%D latin 2 encoding, as needed for Czech and other languages.

\starttypescript [csr]

\usetypescript [all] [computer-modern] [default,name,size,il2]

\enablemathcollection[default]

\stoptypescript

%D This script defines the Computer Modern Roman with a
%D polish encoding, as needed for Czech and other languages.

\starttypescript [plr]

\usetypescript [all] [computer-modern] [default,name,size,pl0]

\enablemathcollection[default]

\stoptypescript

%D Here we implement the symbol fonts as provided by the
%D American Mathematical Society. The names of these symbols
%D can be found in The Joy of \TeX\ by M.~Spivak.

\starttypescript [ams]

\usetypescript [math] [ams] [all]

\enablemathcollection[ams]

\stoptypescript

%D The Concrete Modern Roman is just an alternative Computer
%D Modern Roman.

\starttypescript [con]

\usetypescript [serif] [concrete] [all]
 
\stoptypescript

%D The Euler Fonts are designed by Herman Zapf and can be
%D used with the Concrete Fonts defined elsewhere.

\starttypescript [eul]

\usetypescript [math] [euler] [all]

\enablemathcollection[eul]

\stoptypescript

%D The Lucida Bright fonts are both good looking and and
%D complete. These fonts have prebuilt accented characters,
%D which means that we use another encoding vector: \YandY\
%D texnansi. These fonts are a good illustration that a 12
%D point bodyfont is indeed never that size. The Lucida Bright
%D fonts come in one design size.

\starttypescript [lbr]

\usetypescript [all] [lucida]  [name,special]
\usetypescript [all] [default] [size]

\enablemathcollection[lbr]

\stoptypescript

%D With thanks to Berthold Horn from YandY for providing me
%D evaluation copies of the MathTimePlus fonts.

\starttypescript [mt,tim]

\usetypescript [math] [times]   [all]
\usetypescript [math] [default] [size]

\enablemathcollection[tim]

\stoptypescript

%D The Antikwa Torunska font family is a rather nice
%D and subtle one. Although primary meant for the polish
%D language, it can be used for other languages as well.

\starttypescript [ant]

\usetypescript [serif] [antykwa-torunska] [name]
\usetypescript [serif] [default]          [size]

\stoptypescript

%D This script defines the Standard Adobe Courier fonts.

\starttypescript [pcr]

\usetypescript [mono] [courier] [name]
\usetypescript [mono] [default] [size] 

\stoptypescript

%D This script defines the Standard Adobe Helvetica fonts.

\starttypescript [phv]

\usetypescript [sans] [helvetica] [name]
\usetypescript [mono] [default]   [size] 

\stoptypescript

%D This script defines the Standard Adobe Times fonts.

\starttypescript [ptm]

\usetypescript [serif] [times]   [name]
\usetypescript [serif] [default] [size] 

\stoptypescript

%D This script loads the Adobe Times Roman, Helvetica and
%D Courier.

\starttypescript [pos]

\usetypescript [serif] [times]     [name] 
\usetypescript [sans]  [helvetica] [name] 
\usetypescript [mono]  [courier]   [name] 

\usetypescript [serif,sans,mono] [default] [size] 

\stoptypescript

%D This script defines the Palatino font.

\starttypescript [ppl]

\usetypescript [serif] [palatino] [name]
\usetypescript [serif] [default]  [size]

\stoptypescript

%D The following scripts fake the old \type {font-ber} and
%D alike files.

\starttypescript[ber]

\usetypescript
  [all]
  [computer-modern,concrete,euler]
  [default]

\usetypescript
  [all]
  [courier,helvetica,times,lucida,antykwa-torunska,baskerville,palatino]
  [ec]

\stoptypescript

\starttypescript[fil]

\usetypescript
  [all]
  [computer-modern,concrete,euler]
  [default]

\usetypescript
  [all]
  [courier,helvetica,times,lucida,antykwa-torunska,baskerville,palatino]
  [texnansi]

\stoptypescript

\endinput
