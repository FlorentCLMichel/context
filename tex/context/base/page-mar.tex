%D \module
%D   [       filefile=page-mar, % moved here from main-001
%D        version=1997.03.31,
%D          title=\CONTEXT\ Core Macros,
%D       subtitle=Marginal Things
%D         author=Hans Hagen,
%D           date=\currentdate,
%D      copyright={PRAGMA / Hans Hagen \& Ton Otten}]
%C
%C This module is part of the \CONTEXT\ macro||package and is
%C therefore copyrighted by \PRAGMA. See mreadme.pdf for
%C details.

%D Support for margin words is one of the reasons for writing
%D \CONTEXT. Over time support for marginal content has been
%D extended en enhanced. Therefore it's always good to watch
%D out for unexpected side effects.

\writestatus{loading}{Context Core Macros / Maginal Things}

\unprotect

%D There are three categories and their historically grown meaning is
%D as follows:
%D
%D marginlines: these are flushed relative to the start of a line and
%D need to be invoked there.
%D
%D marginwords: these can be issued in the text flow and will migrate
%D sidewards; in spite of the name, it can be a paragraph of text as
%D well, but normally it's words.
%D
%D margintexts: these can be set beforehand and are flushed at the
%D next paragraph of text (of header)
%D
%D While these mechanisms were rather separated, they now are slightly
%D more integrated. Instead of low level instances we now have a mechanism
%D for defining additional ones.

%D \macros
%D   {inleftedge,inleftmargin,inrightmargin,inrightedge}
%D
%D The fast and clean way of putting things in the margin is
%D using \type{\rlap} or \type{\llap}. Unfortunately these
%D macro's don't handle indentation, left and right skips. We
%D therefore embed them in some macro's that (force and)
%D remove the indentation and restore it afterwards.

\def\definemarginline
  {\dodoubleargument\dodefinemarginline}

\def\dodefinemarginline[#1][#2]%
  {\getparameters
     [\??im\??im#1]
     [\c!location=\v!left,
      \c!distance=\zeropoint,
      \c!width=\leftmarginwidth,
      \c!offset=\leftmargindistance,
      \c!command=,
      #2]%
   \setvalue{#1}{\domarginline{#1}}}

\def\domarginline#1% #2
  {\getvalue{\s!do\??im\??im\executeifdefined{\??im\??im#1\c!location}\v!left}{#1}} % {#2}

\def\doleftmarginline#1#2%
  {\pushindentation
   \llap
     {\postsignalrightpage
      \hsize\getvalue{\??im\??im#1\c!width}\relax
      \executeifdefined{\??im\??im#1\c!command}\firstofoneargument{#2}\relax
      \hskip\leftskip
      \hskip\getvalue{\??im\??im#1\c!offset }\relax
      \hskip\getvalue{\??im\??im#1\c!distance}}%
   \popindentation
   \ignorespaces}

\def\dorightmarginline#1#2%
  {\pushindentation
   \rlap
     {\postsignalrightpage
      \hskip\hsize
      \hskip-\rightskip
      \hskip\getvalue{\??im\??im#1\c!offset }\relax
      \hskip\getvalue{\??im\??im#1\c!distance}\relax
      \hsize\getvalue{\??im\??im#1\c!width}\relax
      \executeifdefined{\??im\??im#1\c!command}\firstofoneargument{#2}}%
   \popindentation
   \ignorespaces}

\setvalue{\s!do\??im\??im\v!left }{\doleftmarginline}
\setvalue{\s!do\??im\??im\v!right}{\dorightmarginline}
\setvalue{\s!do\??im\??im\v!inner}{\presignalrightpage\doifrightpageelse\doleftmarginline \dorightmarginline}
\setvalue{\s!do\??im\??im\v!outer}{\presignalrightpage\doifrightpageelse\dorightmarginline\doleftmarginline }

\definemarginline[inleftmargin] [\c!location=\v!left, \c!width=\leftmarginwidth, \c!offset=\leftmargindistance]
\definemarginline[inrightmargin][\c!location=\v!right,\c!width=\rightmarginwidth,\c!offset=\rightmargindistance]
\definemarginline[inleftedge]   [\c!location=\v!left, \c!width=\leftedgewidth,  \c!offset=\leftedgedistance, \c!distance=\leftmargintotal]
\definemarginline[inrightedge]  [\c!location=\v!right,\c!width=\rightedgewidth, \c!offset=\rightedgedistance,\c!distance=\rightmargintotal]

\definemarginline[inoutermargin][\c!location=\v!outer,\c!width=\outermarginwidth,\c!offset=\outermargindistance]
\definemarginline[ininnermargin][\c!location=\v!inner,\c!width=\innermarginwidth,\c!offset=\innermargindistance]
\definemarginline[inouteredge]  [\c!location=\v!outer,\c!width=\outeredgewidth,  \c!offset=\outeredgedistance,\c!distance=\outermargintotal]
\definemarginline[ininneredge]  [\c!location=\v!inner,\c!width=\inneredgewidth,  \c!offset=\inneredgedistance,\c!distance=\innermargintotal]

\definemarginline[atleftmargin] [\c!location=\v!left, \c!command=\lrlap,\c!distance=\zeropoint,\c!width=\zeropoint,\c!offset=\zeropoint]
\definemarginline[atrightmargin][\c!location=\v!right,\c!command=\rllap,\c!distance=\zeropoint,\c!width=\zeropoint,\c!offset=\zeropoint]

\ifx\lrlap\undefined \def\lrlap#1{\llap{\rlap{#1}\hskip\leftskip}}               \fi
\ifx\rllap\undefined \def\rllap#1{\rlap{\hskip\hsize\hskip-\rightskip\llap{#1}}} \fi

%D We want to keep things efficient and therefore only handle
%D situations like:
%D
%D \startbuffer
%D                  \inleftedge    {fine} some text \par
%D \strut           \inleftmargin  {fine} some text \par
%D \noindent        \inrightmargin {fine} some text \par
%D \noindent \strut \inrightedge   {fine} some text \par
%D \stopbuffer
%D
%D \typebuffer
%D
%D which looks like:
%D
%D \bgroup
%D \getbuffer
%D \parindent 30pt
%D \getbuffer
%D \egroup
%D
%D A torture test:
%D
%D \starttyping
%D \def\TestLine#1#2{\backgroundline[#1]{\strut\white\tttf#2~\recurselevel}}
%D
%D \startbuffer
%D \inleftmargin {\TestLine{red}   {lm}}  test test test \par
%D \inrightmargin{\TestLine{green} {rm}}  test test test \par
%D \inleftedge   {\TestLine{red}   {le}}  test test test \par
%D \inrightedge  {\TestLine{green} {re}}  test test test \par
%D \inoutermargin{\TestLine{blue}  {om}}  test test test \par
%D \ininnermargin{\TestLine{yellow}{im}}  test test test \par
%D \inouteredge  {\TestLine{blue}  {oe}}  test test test \par
%D \ininneredge  {\TestLine{yellow}{ie}}  test test test \par
%D \atleftmargin {\TestLine{red}   {alm}} \hfill test    \par
%D \atrightmargin{\TestLine{green} {arm}} test \hfill    \par
%D \stopbuffer
%D
%D \dorecurse{40}\getbuffer \page
%D \stoptyping

%D New, yet undocumented:
%D
%D used for pascal:
%D
%D \starttyping
%D \index {test} test \index {west} west \index {rest} rest
%D
%D \startnarrower
%D \placeregister[index][alternative=b,command=\atleftmargin]
%D \stopnarrower
%D \stoptyping

% todo: compensate distance when setuplayout[textwidth=..]
% todo: generalize margin/edge model, now too much duplication

%D The next bunch of macros looks messy which is due to its
%D multi-purpose character.

\chardef\margincontentdisplacement \zerocount
\let    \margincontentdistance     \!!zeropoint
\def    \margincontentlines        {1}
\def    \margincontenttag          {0}
\let    \margincontentseparator    \empty
\def    \margincontentstrutheight  {\strutht}

\newcounter\margincontentlevel
\newdimen  \margincontentheight

\def\setupinmargin
  {\dodoubleempty\dosetupinmargin}

\def\dosetupinmargin[#1][#2]%
  {\ifsecondargument
     \processcommalist[#1]{\dodosetupinmargin[#2]}% becomes [#2]{##1}
   \else
     \getparameters[\??im][#1]%
   \fi}

\def\dodosetupinmargin[#1]#2% [settings]{class}
  {\checkinmargin[#2]%
   \getparameters[\??im#2][#1]}

\def\checkinmargin[#1]%
  {\ifundefined{\??im#1\c!offset}%
     \addtocommalist{#1}\inmargintaglist
     \presetmargintext[#1]%
   \fi}

\def\presetmargintext[#1]%
  {\presetlocalframed
     [\??im#1]%
   \getparameters
     [\??im#1]
     [\c!frame=\v!off,
      \c!offset=\v!overlay,
      \c!line=1,
      \c!separator=,
      \c!width=\v!broad,
      \c!distance=\zeropoint,
      \c!style=\@@imstyle,
      \c!color=\@@imcolor,
      \c!location=\@@imlocation,
      \c!align=\@@imalign,
      \c!before=\@@imbefore,
      \c!after=\@@imafter]}

\newdimen\naturalmargincontentheight

\def\makemargintextblock#1#2#3#4% width l r content
  {\bgroup
   \forgetall % added, else problems with 'center' and nested itemize
   \dontcomplain
   \hsize#1\relax
   \doifnumberelse\margincontenttag
     {\ifcase\margincontenttag\relax
        \def\margincontenttag{#2}% first one is setups id as well
      \fi}
     \donothing
   \doifnumberelse\margincontenttag
     {\ifnum\margincontenttag>25 % to be translated
        \writestatus\m!systems{potential margin stack overflow (\margincontenttag)}%
      \fi}
     \donothing
   % we need to preserve {a,b,c} kind of settings
   \let\margincontentalign#2%
   \processallactionsinset
     [\getvalue{\??im\margincontenttag\c!align}]
     [     \v!yes=>\let\margincontentalign#2,
          \v!no=>\let\margincontentalign\v!normal,
       \v!inner=>\let\margincontentalign#2,
       \v!outer=>\let\margincontentalign#3,
        \v!left=>\let\margincontentalign\v!left,
       \v!middle=>\let\margincontentalign\v!middle,
       \v!right=>\let\margincontentalign\v!right]%
   \doifvaluesomething{\??im\margincontenttag\c!align} % watch {} around set
     {\edef\margincontentalign{{\getvalue{\??im\margincontenttag\c!align},\margincontentalign}}}%
   %
   \expanded{\getparameters
     [\??im\margincontenttag]
     [\c!strut=\v!no,\c!offset=\v!overlay,\c!align=\margincontentalign]}%
   %
   \savestrut %
   \setbox\scratchbox\vbox\localframed
     [\??im\margincontenttag]
     {\decrement\margincontentlines
      \dorecurse\margincontentlines{\savedstrut\endgraf\nointerlineskip}% ! savedstrut
      \@@imbefore
      \dostartattributes{\??im\margincontenttag}\c!style\c!color\empty
        \setstrut % yes or no
        \begstrut#4\endstrut\endgraf
        \xdef\margincontentstrutheight{\the\strutht}% so that it's known outside the framed
      \dostopattributes
      \@@imafter}%
   \global        \naturalmargincontentheight\ht\scratchbox
   \global\advance\naturalmargincontentheight\dp\scratchbox
   \doif\@@imstack\v!yes
     {\def\overlappingmargin{-20\scaledpoint}% test value, maybe .25\strutboxdp, maybe configurable
      \setbox\scratchbox\vbox{\stackeddown\vbox{\box\scratchbox}}}% new
   \ht\scratchbox\strutht
\dp\scratchbox\strutdp % nieuw
   \box\scratchbox
   \egroup}

%D The stacker permits constructs like:
%D
%D \starttyping
%D \setupinmargin[stack=yes]
%D
%D \inleft{test 1}test\break
%D \inleft{test 2}test\break
%D \inleft{test 1}
%D \input tufte
%D \inleft{test 1}
%D \inleft{test 2}
%D \inleft{test 3}
%D \input tufte
%D \inleft{test 1}
%D \inleft{test 2\endgraf test 3}
%D \inleft{test 4}
%D \input tufte
%D \inleft{test 1}
%D \inleft{test 2\endgraf test 3}
%D \inleft{test 4\endgraf test 5\endgraf test 6}
%D \inleft{test 7\endgraf test 8\endgraf test 9}
%D \input tufte
%D \stoptyping

%D This approach permits us to implement a better mechanism
%D later. We need the \type {\graphicvadjust} in order to
%D handle:
%D
%D \starttyping
%D \inleft{test} {\red \dorecurse{40}{test }\par}
%D {\red \inleft{test} \dorecurse{40}{test }\par}
%D \stoptyping
%D
%D The outer margin color is either black or color set as
%D main text color.

\newif\ifrightmargin % documenteren

\ifx\dopositionmarginbox\undefined
  \def\dopositionmarginbox#1{\graphicvadjust{\box#1}}
\fi

% watch out, margin dimensions are swapped locally (\swapmargins)

% with \margincontentmethod one can control pagebreaks
%
% 0 no break
% 1 each entry is one line
% 2 only natural height
% 3 also stack height

\chardef\margincontentmethod\plusthree % beware: 1 = old method
\def\margincontentextralines{1}        % old method, play safe

\def\doplacemargintext#1#2#3#4%
  {\strut
   \doifsomething{#1}
     {\def\margincontenttag{#1}}%
   \doifinsetelse{\margincontenttag}{\v!left,\v!right} % ugly hack
     {\let \margincontentdistance                                                      \zeropoint}
     {\edef\margincontentdistance {\executeifdefined{\??im\margincontenttag\c!distance }\zeropoint}}%
   \edef\margincontentlines    {\executeifdefined{\??im\margincontenttag\c!line   }\plusone  }%
   \edef\margincontentseparator{\executeifdefined{\??im\margincontenttag\c!separator}\donothing}%
   \setbox\scratchbox\hbox{#4}% % todo: make sure that color stack works
\ifcase\margincontentmethod
  \scratchdimen\zeropoint
\or % old method
  \scratchdimen\ht\scratchbox
  \advance\scratchdimen\dp\scratchbox
\or
  \scratchdimen\naturalmargincontentheight
\or
  \scratchdimen\naturalmargincontentheight
  \ifx\laststackvmove\undefined\else\global\advance\scratchdimen\laststackvmove\fi
\fi
   \ifdim\scratchdimen>\margincontentheight
     \global\margincontentheight\scratchdimen
   \fi
   \setbox\scratchbox\hbox
     {#2{\hskip#3\strut
         \ifcase\margincontentdisplacement
           % normal, move strutheight up
           \scratchdimen\strutdp
           \advance\scratchdimen \margincontentstrutheight
           \advance\scratchdimen -\strutht
           \raise\scratchdimen
         \or
           % low, obey vadjust
         \fi
         \box\scratchbox}}%
   \ht\scratchbox\zeropoint
   \dp\scratchbox\zeropoint
   \gdef\margincontentstrutheight{\the\strutht}%
  %\graphicvadjust{\box\scratchbox}}     % fails in high math lines, let it be
  %\hbox{\lower\strutdp\box\scratchbox}} % alas, wrong lapping, therefore useless
   \dopositionmarginbox\scratchbox}

\def\leftmargintextdistance {\getvalue{\??im\v!left \c!distance}}
\def\rightmargintextdistance{\getvalue{\??im\v!right\c!distance}}

\def\leftmargintextwidth    {\getvalue{\??im\v!left \c!width}}
\def\rightmargintextwidth   {\getvalue{\??im\v!right\c!width}}

\def\doleftmarginblock#1#2%
  {\doplacemargintext{#1}\llap\zeropoint
     {\llap{\placemargincontentseparator}%
      \makemargintextblock\leftmargintextwidth\v!left\v!right{#2}%
      \hskip\margincontentdistance
      \hskip\leftmargintextdistance}}

\def\dorightmarginblock#1#2%
  {\doplacemargintext{#1}\rlap\hsize
     {\hskip\rightmargintextdistance
      \hskip\margincontentdistance
      \hskip\textwidth\hskip-\hsize % new: hsize correction
      \makemargintextblock\rightmargintextwidth\v!right\v!left{#2}%
      \rlap{\placemargincontentseparator}}}

\def\placemargincontentseparator
  {\ifnum\margincontentlevel>\zerocount
     \ifx\margincontentseparator\empty\else
       \bgroup
       \scratchdimen\margincontentlines\lineheight
       \advance\scratchdimen -\lineheight
       \lower\scratchdimen\hbox{\margincontentseparator}%
       \egroup
     \fi
   \fi}

%

\newbox\marginconstructbox

% \def\doinmarginswapped#1#2#3#4%
%   {\iffirstsidefloatparagraph\geenwitruimte\fi % zo laat mogelijk
%    \setbox\marginconstructbox\hbox\bgroup % prevents page break in the middle of construction
%      \signalrightpage
%      \doifswappedrightpageelse
%        {\rightmargintrue \freezepagestate#2}
%        {\rightmarginfalse\freezepagestate#1}
%        {#3}% setups
%        {\freezepagestate#4}% content
%    \egroup
%    \unhbox\marginconstructbox}

\def\doinmarginswapped#1#2#3#4%
  {\iffirstsidefloatparagraph\nowhitespace\fi % zo laat mogelijk
   \setbox\marginconstructbox\hbox\bgroup % prevents page break in the middle of construction
     \startsignalrightpage
       \doifswappedrightpageelse
         {\rightmargintrue #2}
         {\rightmarginfalse#1}
         {#3}% setups
         {#4}% content
     \stopsignalrightpage
   \egroup
   \unhbox\marginconstructbox}


% history made this a bit complicated, the +/- was needed before
% we had enough mem/hash to do the page correction

\edef\inmargintaglist{+,-,\v!low,\v!left,\v!right,\v!inner,\v!outer}

% the old one:
%
% \def\doinmargin[#1][#2][#3][#4][#5]% #6 #7
%   {\doifcommonelse{+,-,\v!laag}{#4}
%      {\dodoinmargin[#1][#2][#3][#4][#5]}
%      {\dodoinmargin[#1][#2][#3][][#4]}}
%
% an alternative:
%
% \letvalue{\??im\v!laag\c!offset}\empty
% \letvalue{\??im      +\c!offset}\empty
% \letvalue{\??im      -\c!offset}\empty
%
% \def\doinmargin[#1][#2][#3][#4][#5]% #6 #7
%   {\doifnumberelse{#4}
%      {\dodoinmargin[#1][#2][#3][#4][#5]}
%      {\doifdefinedelse{\??im#4\c!offset}
%         {\dodoinmargin[#1][#2][#3][#4][#5]}
%         {\dodoinmargin[#1][#2][#3][][#4]}}}
%
% the problem is that we ned to keep downward compatibility
% with respect to the first argument thing a reference or a
% directive; the alternative is to force users to pass a
% directive along with a reference; anyhow, as long as one
% does not use references that have the same name as a
% directive we can use the (slow) alternative

\def\doinmargin[#1][#2][#3][#4][#5]% #6 #7
% {\expanded{\doifcommonels{#4}{\inmargintaglist}}
  {\expanded{\doifinsetelse{#4}{\inmargintaglist}}
     {\dodoinmargin[#1][#2][#3][#4][#5]}
     {\dodoinmargin[#1][#2][#3][][#4]}}

\def\defineinmargin
  {\doquadrupleempty\dodefineinmargin}

\def\dodefineinmargin[#1][#2][#3][#4]%
  {\doifassignmentelse{#4}
     {\setupinmargin[#1][#4]%
      \setvalue{#1}{\indentation\doquintupleempty\doinmargin[#2][#3][#1]}}
     {\setvalue{#1}{\indentation\doquintupleempty\doinmargin[#2][#3][#4]}}}

\defineinmargin[inleft]  [\v!left]    [\v!normal]    % takes left settings
\defineinmargin[inright] [\v!right]   [\v!normal]    % takes right settings
\defineinmargin[ininner] [\v!inner]   [\v!normal]    % takes left/right settings
\defineinmargin[inouter] [\v!outer]   [\v!normal]    % takes left/right settings
\defineinmargin[inmargin][\@@imlocation] [\v!normal]    % takes left/right settings
\defineinmargin[inother] [\@@imlocation] [\v!reverse]  % takes left/right settings

\def\inothermargin{\inother}

%D This permits definitions like:
%D
%D \starttyping
%D \defineinmargins[SomePlace] [inner] [normal] [distance=1cm]
%D \defineinmargins[SomePlace] [inner] [normal] [SomePlace]    \setupinmargin[SomePlace][distance=1cm]
%D \defineinmargins[MyPlace]   [inner] [normal] [SomePlace]
%D \defineinmargins[YourPlace] [inner] [normal] [SomePlace]
%D \stoptyping
%D
%D A torture test:
%D
%D \starttyping
%D \startbuffer
%D \inleft       {\TestLine{red}    {l}} test test test \par
%D \inright      {\TestLine{green}  {r}} test test test \par
%D \inmargin     {\TestLine{blue}   {m}} test test test \par
%D \inothermargin{\TestLine{yellow} {x}} test test test \par
%D \ininner      {\TestLine{cyan}   {i}} test test test \par
%D \inouter      {\TestLine{magenta}{o}} test test test \par
%D \stopbuffer
%D
%D \dorecurse{80}\getbuffer \page
%D stoptypen
%D
%D and
%D
%D \starttyping
%D \defineinmargin[InOuterA] [outer] [normal] [distance=0cm]
%D \defineinmargin[InOuterB] [outer] [normal] [distance=1cm]
%D \defineinmargin[InOuterC] [outer] [normal] [distance=2cm,line=2]
%D
%D \startbuffer
%D \InOuterA{\TestLine{red}  {A}} test test test \par
%D \InOuterB{\TestLine{green}{B}} test test test \par
%D \InOuterC{\TestLine{blue} {C}} test test test \par
%D \stopbuffer
%D
%D \dorecurse{80}\getbuffer \page
%D
%D \dorecurse{10}{\inleft {one} test \inleft {two} test } \page
%D
%D \start
%D   \margintext {one} \margintext {two} \input thuan \par
%D   \setupinmargin[1][line=3,distance=1cm]
%D   \margintext [1]{one}
%D   \margintext [2]{two}
%D   \input thuan \page
%D \stop
%D
%D \setupinmargin[3][location=inner,distance=1cm]
%D \setupinmargin[4][location=outer,distance=2cm]
%D
%D % \setupinmargin[left] [line=2]
%D % \setupinmargin[right][line=2]
%D
%D \dorecurse
%D   {10}
%D   {\margintext       {\kern3cm\TestLine{blue}{none}}
%D    \margintext[3]    {\TestLine{darkgray}{3}}
%D    \margintext[4]    {\TestLine{darkgray}{4}}
%D    \margintext[left] {\TestLine{red}     {left}}
%D    \margintext[right]{\TestLine{green}   {right}}
%D    \margintext[inner]{\TestLine{cyan}    {inner}}
%D    \margintext[outer]{\TestLine{magenta} {outer}}
%D    \input thuan \endgraf}
%D
%D \dorecurse{10}{\margintext{test\\test\\test} \input thuan \endgraf}
%D \stoptyping

\def\dodoinmargin[#1][#2][#3][#4][#5]#6%
  {\bgroup
   \postponefootnotes % group is (somehow) needed
   \doifinsetelse\v!low{#4}
     {\chardef\margincontentdisplacement\plusone}
     {\chardef\margincontentdisplacement\zerocount}%
   \doif\v!reverse{#2}
     {\swapmacros\dorightmarginblock\doleftmarginblock}%
   \processaction
     [#1]
     [  \v!left=>\let\next\doleftmarginblock,  % no swapping
       \v!right=>\let\next\dorightmarginblock, % no swapping
       \v!inner=>\def\next{\doinmarginswapped\dorightmarginblock\doleftmarginblock },
       \v!outer=>\def\next{\doinmarginswapped\doleftmarginblock \dorightmarginblock},
      \s!unknown=>\ifdubbelzijdig
                    \doifcommonelse{+,-}{#4}
                      {\def\next{\doinmarginswapped\dorightmarginblock\doleftmarginblock }}
                      {\def\next{\doinmarginswapped\doleftmarginblock \dorightmarginblock}}%
                  \else
                    \let\next\doleftmarginblock
                  \fi]%
   \next{#3}{#6}%
   \rawpagereference\s!mar{#5}% naar binnen ! ! ! !
   \flushnotes
   \egroup % don't forget the group
   \ignorespaces}

% dit zijn voorlopig lokale commando's / vervallen
%
% \def\woordinmarge {\indentation\doquintupleempty\doinmargin[\@@implaats][\inleftmargin][\inrightmarge]}
%
% \def\woordinlinker {\inleftmargin  } % vervallen
% \def\woordinrechter{\inrechtermarge} % vervallen

% Some day: \definemarking[\v!margetitel]

%D Now come the margin text collectors. The collected content is
%D flushed at every paragraph by the following macro. Note for
%D myself: here the location (plaats) is no longer a tag (number).

% gone: \def\doflushmargincontent{\doinmargin[\@@implaats][\v!normaal][]} % + [#1][#2]{#3}}

%D These are now all the same (long ago they had different
%D implementations, somewhere in Sork time if I remember
%D right).

\def\margintext {\dodoubleempty\domargincontent}
\def\marginword {\margintext}
\def\margintitle{\margintext} % txt mark as well

\newtoks\collectedmargintexts
\chardef\margintextcollected  \zerocount

\def\domargincontent[#1][#2]#3% we used to check for #2/#1 being number, no longer now
  {\global\chardef\margintextcollected\plusone
   \edef\margincontenttag{#1}%
   \ifx\margincontenttag\empty
     \doglobal\increment\margincontentlevel
     \let\margincontenttag\margincontentlevel
   \fi
   \checkinmargin[\margincontenttag]%
   \doglobal \appendetoks
     \noexpand \checkinmargin[\margincontenttag]%
     \noexpand \doinmargin[\executeifdefined{\??im\margincontenttag\c!location}\@@imlocation][\v!normal][\margincontenttag][\margincontenttag][#2]%
   \to \collectedmargintexts
   \doglobal \appendtoks
     {#3}%
   \to \collectedmargintexts}

\let\restoreinterlinepenalty\relax

\def\flushmargincontents % plural
  {\restoreinterlinepenalty % here?
   \ifcase\margintextcollected\else     % called quite often, so we
     \expandafter\doflushmargincontents % speed up the \fi scan by
   \fi}                                 % using a \do..

\def\doflushmargincontents % links + rechts
  {\bgroup
   \forgetall
   \global\margincontentheight\zeropoint
   \startsignalrightpage
     \the\collectedmargintexts
     \signalrightpage
   \stopsignalrightpage
   \resetmargincontent
   % dirty tricks
   \ifcase\margincontentmethod
     \donefalse
   \else\ifinsidecolumns % brrrr
     \donetrue  % how fuzzy
   \else\ifdim\margincontentheight>\lineheight\relax
     \donetrue  % how dirty
   \else
     \donefalse % how needed
   \fi\fi\fi
   \ifdone
     \advance\margincontentheight \margincontentextralines\lineheight
     \bgroup % preserve \margincontentheight
     \advance\margincontentheight \pagetotal
     \ifdim\margincontentheight>\pagegoal
       \egroup
       \setmargincontentpenalties
     \else
       \egroup
     \fi
   \else % We need the above because interlinepenalties overrule vadjusted \nobreaks.
     \vadjust{\nobreak}%
   \fi
   \egroup}

\beginETEX

  \def\setmargincontentpenalties
    {\xdef\restoreinterlinepenalty{\global\resetpenalties\interlinepenalties}%
     \getnoflines\margincontentheight
     \global\setpenalties\interlinepenalties\noflines\!!tenthousand}

\endETEX

\beginTEX

  \def\setmargincontentpenalties
    {\xdef\restoreinterlinepenalty
       {\global\let\restoreinterlinepenalty\relax
        \global\interlinepenalty=\the\interlinepenalty}% keep = here
     \global\interlinepenalty\!!tenthousand}

\endTEX

% Yet undocumented, for a manual flush in for instance headers.

\def\resetmargincontent
  {\doglobal\newcounter\margincontentlevel
   \global\chardef\margintextcollected\zerocount
   \global\collectedmargintexts\emptytoks}

\def\placemargincontent
  {\ifcase\margincontentlevel\else
     \bgroup
       \chardef\graphicvadjustmode\zerocount
       \doflushmargincontents
     \egroup
   \fi}

% For old times sake (i use it in project styles) we provide

\def\placemargintexts {\placemargincontent}
\def\resetmargetitels {\resetmargincontent}
\def\margewoordpositie{\margewoord} % obsolete, now no longer range

% but never use them yourself since they may disappear.

\def\oplinker#1%
  {\strut
   \graphicvadjust
     {\dontcomplain
      \setbox\scratchbox\vtop{\forgetall\strut#1}%
      \getboxheight\scratchdimen\of\box\scratchbox
      \vskip-\scratchdimen % waarom stond hier een \ ?
      \box\scratchbox}}

\setupinmargin
  [\c!style=\v!bold,
   \c!color=,
   \c!location=\v!both,
   \c!align=\v!inner,
   \c!stack=\v!no,
   \c!before=,
   \c!after=]

\setupinmargin
  [\v!left]
  [\c!distance=\leftmargindistance,
   \c!width=\leftmarginwidth,
  %\c!align=\v!left, % no
   \c!location=\v!left]

\setupinmargin
  [\v!right]
  [\c!distance=\rightmargindistance,
   \c!width=\rightmarginwidth,
  %\c!align=\v!right, % no
   \c!location=\v!right]

% bonus needed when [inner/outer] is used as tag

\setupinmargin[\v!inner][\c!location=\v!inner,\c!align=\v!inner]
\setupinmargin[\v!outer][\c!location=\v!outer,\c!align=\v!inner]

% more efficient (5K less fotmat file)
%
% \letvalue{\??im\v!binnen\c!plaats}\v!binnen \letvalue{\??im\v!binnen\c!uitlijnen}\v!binnen
% \letvalue{\??im\v!buiten\c!plaats}\v!buiten \letvalue{\??im\v!buiten\c!uitlijnen}\v!binnen

\protect \endinput
