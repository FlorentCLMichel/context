%D \module
%D   [       file=xtag-mmc,
%D        version=2000.12.20,
%D          title=\CONTEXT\ XML Support,
%D       subtitle=Math ML,
%D         author=Hans Hagen,
%D           date=\currentdate,
%D      copyright={PRAGMA / Hans Hagen \& Ton Otten}]
%C
%C This module is part of the \CONTEXT\ macro||package and is
%C therefore copyrighted by \PRAGMA. See mreadme.pdf for
%C details.

\writestatus{loading}{Context XML Macros (content math ml)}

% this is the first, experimental, shabby implementation, as
% always, the third will do -) 

% okay, now this is the second one, with a more kid based
% approach, which we needed for proper nesting, but the 
% code is still too ugly 

\startXMLmapping[mml]

\unprotect

\def\MMLccomma{{,}}

\def\getmmlarguments#1#2#3#4#5#6% {class} {defaults} mmlargs
  {\getXMLarguments{#1}{#2 #5}}%

\def\doifMMLoperatorelse#1#2#3% temp hack
  {\pushmacro\xmlrent
   \let\xmlrent\firstofoneargument
   \edef\mmloperator{#1}%
   \@EA\unspaceargument\mmloperator\to\mmloperator
   \popmacro\xmlrent
   \doifXMLentityelse\mmloperator
     {\doifdefinedelse{doMMLcFN\mmloperator}{#2}{#3}}
     {#3}}

\remapXMLsequence [ci]      [CPA] \MMLcCI
\remapXMLsequence [cn]      [CPA] \MMLcCN
\remapXMLsequence [csymbol] [CPA] \MMLcCSYMBOL

\setupMMLappearance[polar][\c!variant=\v!a] % a|b|c

\def\MMLcpolar#1%
  {\doifdefinedelse{MMLcdopolar\@@MMLpolarvariant}
     {\getvalue{MMLcdopolar\@@MMLpolarvariant}{#1}}
     {#1}}

\def\MMLcdopolara#1%
  {\def\MMLcsep{\MMLccomma}\getXMLentity{polar}\left(#1\right)}

\def\MMLcdopolarb#1%
  {\def\MMLcsep##1\empty%
    {\getXMLentity{exponentiale}^{##1\mskip2mu\getXMLentity{imaginaryi}}}#1}

\def\MMLcdopolarc#1%
  {\def\MMLcsep##1\empty%
     {\getXMLentity{exp}\left(##1\mskip2mu\getXMLentity{imaginaryi}\right)}#1}

\def\MMLccartesian#1%
  {\def\MMLcsep{+}#1\getXMLentity{imaginaryi}}

% float will be obsolete, an is replace by e-notation

\setupMMLappearance[float][\c!symbool=\v!nee] % \v!ja|dot

\def\MMLcfloat#1%
  {\doifelse\@@MMLfloatsymbool\v!nee
     {{\rm#1}} % make sure that e shows up ok 
     {% we should ignore \entities ! 
      \beforesplitstring#1\at e\to\first
      \aftersplitstring #1\at e\to\last
      \ifx\first\empty #1\else
      \ifx\last \empty #1\else
        \first
        \doifelse\@@MMLfloatsymbool{dot}\cdot\times
        10^{\last}%
      \fi\fi}}

% we now have: 

\setupMMLappearance[enotation][\c!symbool=\v!nee] % \v!ja|dot

\def\MMLcenotation#1%
  {\doifelse\@@MMLenotationsymbool\v!nee
     {\def\MMLcsep% 
        {\unskip{\rm e}\ignorespaces}}
     {\def\MMLcsep##1\empty
        {\doifelse\@@MMLenotationsymbool{dot}\cdot\times10^{##1}}}%
   #1\empty}

\def\MMLcCI#1% #2% can be sped up with \doifundefined ... 
  {\getXMLarguments{ci}{type="" #1}%
   \doifXMLparelse{ci}{type}
     {\processaction
        [\XMLpar{ci}{type}{}]
        [     set=>\let\next\MMLcCIset,
           vector=>\let\next\MMLcCIvector,
           matrix=>\let\next\MMLcCImatrix,
         function=>\let\next\MMLcCIfunction,
complex-cartesian=>\let\next\MMLccartesian,
          complex=>\let\next\MMLccartesian,
    complex-polar=>\let\next\MMLcpolar,
            polar=>\let\next\MMLcpolar,
               fn=>\let\next\MMLcCIfunction,
          unknown=>\let\next\firstofoneargument]} % integer constant real float 
     {\let\next\firstofoneargument}%              
   \next}

\def\MMLcCIset#1%
  {{\bbd#1}}

\def\widevec#1%
  {\vbox{\m@th\ialign{##\crcr
     \rightarrowfill\crcr\noalign{\nointerlineskip}%
      $\hfil\displaystyle{#1}\hfil$\crcr}}}

\def\MMLcCIvector#1%
  {\widevec{#1}}

\def\MMLcCImatrix#1%
  {{\bi#1}}

\def\MMLcCIfunction#1%
  {\doifMMLoperatorelse{#1}  
     {\let\next\MMLcCIdefault}
     {\let\next\MMLcFNdefault}%
   \next{}{#1}}

\def\MMLcCIdefault#1#2% #1=dummy 
  {#2%\!% \getXMLentity{NegThinSpace}% looks bad in A(x) 
   \doifnextcharelse\xmlr
     {\encapsulatenextXMLRelements{\left(}{\MMLccomma}{\right)}}
     {}}

\def\MMLcCN#1% #2% can be sped up with \doifundefined ... 
  {\getXMLarguments{cn}{type="" #1}%
   \doifXMLparelse{cn}{type}
     {\processaction
        [\XMLpar{cn}{type}{}]
        [ integer=>\let\next\MMLcCNinteger,
          logical=>\let\next\MMLcCNlogical,
         rational=>\let\next\MMLcCNrational,
complex-cartesian=>\let\next\MMLccartesian,
          complex=>\let\next\MMLccartesian,
    complex-polar=>\let\next\MMLcpolar,
            polar=>\let\next\MMLcpolar,
       e-notation=>\let\next\MMLcenotation, % new 
            float=>\let\next\MMLcfloat,     % obsolete 
             real=>\let\next\MMLcfloat,     % ? 
          unknown=>\let\next\firstofoneargument]}
     {\let\next\firstofoneargument}
  \next}

\setupMMLappearance[base][\c!symbool=\v!cijfers] % cijfers|letters|tekst|nee

\def\MMLcCNinteger#1%
  {\doifXMLparelse{cn}{base}
     {\def\MMLcCNbase{\XMLpar{cn}{base}{}}% 
      \doifelse\@@MMLbasesymbool\v!nee
        {\MMLcCNbasedata{#1}}
        {\MMLcCNbasedata{#1}_{\hbox{$\rm\scriptscriptstyle
         \processaction
           [\@@MMLbasesymbool]
           [\v!letters=>\MMLcCNbasestring BODH,
              \v!tekst=>\MMLcCNbasestring{BIN}{OCT}{DEC}{HEX},
            \s!unknown=>\MMLcCNbase]$}}}}
     {#1}}

\def\MMLcCNbasedata#1%
  {\ifnum\MMLcCNbase>10{\rm#1}\else#1\fi}

\def\MMLcCNbasestring#1#2#3#4%
  {\ifnum\MMLcCNbase= 2 #1\else
   \ifnum\MMLcCNbase= 8 #2\else
   \ifnum\MMLcCNbase=10 #3\else
   \ifnum\MMLcCNbase=16 #4\else
         \MMLcCNbase      \fi\fi\fi\fi}

\def\MMLcCNlogical#1%
  {{\rm#1}}

\def\MMLcCNrational#1%
  {\ConvertConstantAfter\doifinstringelse{sep}{#1}
     {\doMMLcCNrational#1\empty}
     {#1}}

\def\doMMLcCNrational#1\xmlr#2#3#4#5\empty
  {\hbox{$\frac{#1}{#5}$}}

\def\MMLcCSYMBOL#1#2% TO DO
  {\getXMLarguments{csymbol}{encoding="text" #1}%
   \doifelse{\XMLpar{csymbol}{encoding}{}}{text}
     {{\rm#2}}{#2}}

%%% basic content elements

\remapXMLsequence [apply] [CPA] \MMLcAPPLY

\newcount\@MMLlevel \def\MMLcreset{\@MMLlevel=0 }

\let\MMLctempresetlist\empty \def\setMMLcreset{\edef\MMLctempresetlist}

\def\MMLcmainresetlist%
  {times/,divide/,power/,% 
   lt/,gt/,eq/,leq/,geq/,%
   in/,inverse/,%
   fn,%
   floor/,ceiling/,%
   mean/,%
   selector/,%
   abs/,int/,limit/,sum/,product/,%
   outerproduct/,innerproduct/,scalarproduct/}

\def\MMLcfunctionlist
  {sin/,arcsin/,sinh/,arcsinh/,%
   cos/,arccos/,cosh/,arccosh/,%
   tan/,arctan/,tanh/,arctanh/,%
   cot/,arccot/,coth/,arccoth/,%
   csc/,arccsc/,csch/,arccsch/,%
   sec/,arcsec/,sech/,arcsech/,%
   ln/,exp/,log/,%
   abs/,int/,limit/,sum/,product/,%
   fn} % ?  

\def\MMLcconstructlist
  {diff/,partialdiff/,root/} % apply goes wrong on 1/2 * (a_2 + b_3) 

% better: no () here but explicitly when needed, which is in
% less cases

\def\MMLcAPPLY#1#2%
  {\mathinner{\begingroup % new, else 1/2(1+2) problem / inner: ask taco
   \pushmacro\xmlr
   \@EA\@EA\@EA\doifXMLRchild\@EA\@EA\@EA{\@EA\MMLcmainresetlist\@EA,\MMLctempresetlist}{#2}
     {\MMLcreset}%
   \ifcase\@MMLlevel
     \getXMLarguments{apply}{open="" close=""   #1}%
   \else
     \getXMLarguments{apply}{open="(" close=")" #1}%
   \fi
   \doifXMLparelse{apply}{open}{\left\XMLpar{apply}{open}{}}{}%
   \advance\@MMLlevel+1
   \begingroup
     \let\MMLctempresetlist\empty
     \let\xmlr\naturalxmlr % beter een \pushmacro\xmlr
     #2\empty
   \endgroup % ook level push
   \advance\@MMLlevel-1
   \doifXMLparelse{apply}{close}{\right\XMLpar{apply}{close}{}}{}%
   \popmacro\xmlr
   \endgroup}} 

\remapXMLsequence [reln] [CPA] \MMLcRELN

\def\MMLcRELN%
  {\writestatus{XML}{MathML element "reln" is obsolete}\MMLcAPPLY}

\remapXMLsequence [fn] [CPA] \MMLcFN

\def\MMLcFN#1#2% neg space needed because of \left 
  {\doifXMLRchildelse{ci}{#2}
     {\collectXMLRchild{ci}{#2}%
      \@EA\doifMMLoperatorelse\@EA{\the\XMLRtoks}  
        {\let\next\MMLcFNoperator}
        {\let\next\MMLcFNdefault}}
     {\let\next\MMLcFNdefault}%
   \next{#1}{#2}} 

\def\MMLcFNdefault#1#2% neg space needed because of \left 
  {#2\!% \getXMLentity{NegThinSpace}%
   \MMLcreset\encapsulatenextXMLRelements{\left(}{\MMLccomma}{\right)}}

% special function handlers 

\def\MMLcFNoperator
  {\getvalue{doMMLcFN\mmloperator}}

% \def\doMMLcFNplusminus#1#2#3\empty
%   {\countXMLRchildren{#3}% 
%    \ifcase\nofXMLRchildren\or
%      #2#3%
%    \else % suboptimal for instance under root 
%      \encapsulateXMLR{\left(}{#2}{\right)}{\MMLcreset#3}%
%    \fi}

\def\doMMLcFNplusminus#1#2#3\empty
  {\countXMLRchildren{#3}% 
   \ifcase\nofXMLRchildren\or
     \def\next{#2#3}%
   \else 
     \def\next
       {\doifnextcharelse\xmlr
          {\encapsulateXMLR{\left(}{#2}{\right)}{\MMLcreset#3}}%
          {\encapsulateXMLR{}{#2}{}{\MMLcreset#3}}}%
   \fi
   \next}

\let\doMMLcFNminusplus\doMMLcFNplusminus

\def\doMMLcFNcontinued#1#2#3\empty
  {#3\getXMLentity{continued}}

% so far for the special handlers  

\remapXMLsequence [interval] [CPA] \MMLcINTERVAL

\def\MMLcINTERVAL#1#2%
  {\withnexttwoXMLRelements
     {\getXMLarguments{interval}{closure="closed" #1}%
      \processaction
        [\XMLpar{interval}{closure}{closed}]
        [     closed=>{[\firstXMLRelement,\secondXMLRelement]},
         open-closed=>{(\firstXMLRelement,\secondXMLRelement]},
         closed-open=>{[\firstXMLRelement,\secondXMLRelement)},
                open=>{(\firstXMLRelement,\secondXMLRelement)},
          \s!unknown=>{[\firstXMLRelement,\secondXMLRelement]}]}%
   #2\empty}

\remapXMLsingular [inverse] [CPA] \MMLcINVERSE

\def\MMLcINVERSE#1#2%
  {\withnextXMLRelement
     {\nextXMLRelement\empty^{-1}% \empty kills MMLcfunction lookahead
      \withnextXMLRelement
        {\ifx\nextXMLRelement\empty \else
           \left[\nextXMLRelement\right]%
         \fi}}}

\remapXMLsingular [sep] [CPA] \MMLcSEP

\def\MMLcSEP#1#2{\MMLcsep} \def\MMLcsep{\,}

\remapXMLsequence [condition] [CPA] \MMLcCONDITION

\def\MMLcCONDITION#1#2%
  {\pushmacro\MMLcBVAR
   \doifXMLRchild{bvar}{#2}{\processXMLRchild{bvar}{#2}\mid}%
   \let\MMLcBVAR\ignoreXMLRelement
   \processXMLRchildren{#2}%
   \popmacro\MMLcBVAR}

\remapXMLsequence [declare] [CPA] \MMLcDECLARE

\setupMMLappearance[declare][\c!status=\v!start]

\def\MMLcDECLARE#1#2%
  {\doif\@@MMLdeclarestatus\v!start
     {\withnextXMLRelement
        {\getXMLentity{declare}\nextXMLRelement
         \countXMLRchildren{#2}%
         \ifnum\nofXMLRchildren>1 
            \;% \getXMLentity{ThickSpace}%
            \getXMLentity{as}%
            \;% \getXMLentity{ThickSpace}%
         \fi}%
      #2\empty}}

\remapXMLsequence [lambda] [CPA] \MMLcLAMBDA

\setupMMLappearance[lambda][\c!variant=b]

\def\MMLcLAMBDA#1#2%
  {\doifelse\@@MMLlambdavariant\v!a
     {\encapsulateXMLR{\getXMLentity{lambda}(}{\MMLccomma}{)}{#2}}
     {\countXMLRchild{bvar}{#2}%
      \ifnum\nofXMLRchildren>1
        \collectXMLRchild{bvar}{#2}%
        \encapsulateXMLR{\left(}{\MMLccomma}{\right)}{\the\XMLRtoks}%
      \else
        \processXMLRchild{bvar}{#2}%
      \fi
      \getXMLentity{mapsto}%
      \processXMLRchild{apply,reln,ci,cn}{#2}}}

\remapXMLsingular [compose] [CPA] \MMLcCOMPOSE

\def\MMLcCOMPOSE#1#2#3\empty
  {\doifMMLfunctionelse{#3}
     {\encapsulateXMLR{\left(}{\circ}{\right)}{#3}}
     {\encapsulateXMLR{}{\circ}{}{#3}}}

\remapXMLsingular [ident] [CPA] \MMLcIDENT

\def\MMLcIDENT#1#2{\getXMLentity{identity}}

\remapXMLsingular [domain]   [CPA] \MMLcDOMAIN
\remapXMLsingular [codomain] [CPA] \MMLcCODOMAIN

\def\MMLcDOMAIN  #1#2#3\empty
  {\getXMLentity{domain}(\MMLcreset\processXMLRchildren{#3})}

\def\MMLcCODOMAIN#1#2#3\empty
  {\getXMLentity{codomain}(\MMLcreset\processXMLRchildren{#3})}

\remapXMLsingular [image] [CPA] \MMLcIMAGE

\def\MMLcIMAGE#1#2%
  {\withnextXMLRelement{\getXMLentity{image}(\nextXMLRelement)}}

\remapXMLsequence [domainofapplication] [CPA] \MMLcDOMAINOFAPPLICATION

\def\MMLcDOMAINOFAPPLICATION#1#2{#2}

\remapXMLsequence [piecewise] [CPA] \MMLcPIECEWISE
\remapXMLsequence [piece]     [CPA] \MMLcPIECE
\remapXMLsequence [otherwise] [CPA] \MMLcOTHERWISE

\setupMMLappearance[piece][\c!scheider=] 

\def\MMLcPIECEWISE#1#2%
  {\cases{#2}}

%\def\MMLcPIECE#1#2% \@@MMLpiecescheider
%  {\withnexttwoXMLRelements
%     {\@EA\XMLRtoks\@EA{\firstXMLRelement\@col@amp@}%
%      \@EA\appendtoks\@EA\mathematics\@EA{\secondXMLRelement}\to\XMLRtoks
%      \the\XMLRtoks\crcr}%
%   #2}%
%
%\def\MMLcOTHERWISE#1#2%
%  {#2\@col@amp@\mathematics{\getXMLentity{otherwise}}\crcr}

\def\MMLcPIECEseparator{\doif\@@MMLpiecescheider\v!ja,}

\def\MMLcPIECE#1#2% 
  {\withnexttwoXMLRelements
     {\@EA\XMLRtoks\@EA{\firstXMLRelement\MMLcPIECEseparator\@col@amp@}%
      \@EA\appendtoks\@EA\mathematics\@EA{\secondXMLRelement}\to\XMLRtoks
      \the\XMLRtoks\crcr}%
   #2}

\def\MMLcOTHERWISE#1#2%
  {#2\MMLcPIECEseparator\@col@amp@\mathematics{\getXMLentity{otherwise}}\crcr}

%%% arithmic, algebra and logic

\remapXMLsingular [quotient] [CPA] \MMLcQUOTIENT

\def\MMLcQUOTIENT#1#2%
  {\withnexttwoXMLRelements{\lfloor\firstXMLRelement/\secondXMLRelement\rfloor}}

\remapXMLsingular [factorial] [CPA] \MMLcFACTORIAL

\def\MMLcFACTORIAL#1#2%
  {\withnextXMLRelement{\nextXMLRelement !}}

\remapXMLsingular [divide] [CPA] \MMLcDIVIDE

\setupMMLappearance [divide] [\c!niveau=\!!maxcard]

\def\MMLcDIVIDE#1#2%
  {\withnexttwoXMLRelements
     {\increment\MMLcDIVIDElevel
      \ifnum\MMLcDIVIDElevel>\@@MMLdivideniveau\space
        \firstXMLRelement/\secondXMLRelement
      \else
        \frac{\MMLcreset\firstXMLRelement}{\MMLcreset\secondXMLRelement}%
      \fi
      \decrement\MMLcDIVIDElevel}}

\remapXMLsingular [min] [CPA] \MMLcMIN
\remapXMLsingular [max] [CPA] \MMLcMAX

\def\MMLcMIN#1#2#3\empty
  {\encapsulateXMLR
     {\min\doifXMLRchild{bvar}{#3}{_{\processXMLRchild{bvar}{#3}}}\left\{}
     {\MMLccomma}{\right\}}{#3}}

\def\MMLcMAX#1#2#3\empty
  {\encapsulateXMLR
     {\max\doifXMLRchild{bvar}{#3}{_{\processXMLRchild{bvar}{#3}}}\left\{}
     {\MMLccomma}{\right\}}{#3}}

\remapXMLsingular [minus] [CPA] \MMLcMINUS
\remapXMLsingular [plus]  [CPA] \MMLcPLUS

\setupMMLappearance[sign][\c!reductie=\v!ja]

\def\MMLcMINUS % expandafter needed ? 
  {\doifelse\@@MMLsignreductie\v!ja
     {\expandafter\MMLcMINUSyes}{\expandafter\MMLcMINUSno}}

\def\MMLcPLUS % expandafter needed ? 
  {\doifelse\@@MMLsignreductie\v!ja
     {\expandafter\MMLcPLUSyes}{\expandafter\MMLcPLUSno}}

\def\MMLcMINUSno#1#2#3\empty
  {\countXMLRchildren{#3}%
   \encapsulateXMLR{\ifcase\nofXMLRchildren\or-\fi}{-}{}{#3}}

\def\MMLcPLUSno#1#2#3\empty
  {\countXMLRchildren{#3}%
   \encapsulateXMLR{\ifcase\nofXMLRchildren\or+\fi}{+}{}{#3}}

\def\MMLcMINUSyes#1#2#3\empty 
  {\setMMLcreset{\MMLcfunctionlist}%
   \countXMLRchildren{#3}%
   \encapsulateXMLR{\ifcase\nofXMLRchildren\or-\fi}{-}{}{#3}}

\def\MMLcPLUSyes#1#2#3\empty
  {\setMMLcreset{plus/,minus/,\MMLcfunctionlist,\MMLcconstructlist}%
   \encapsulatenextXMLRelements
     {}
     {\doifXMLRchildelse{apply}{\nextXMLRelement}%
        {\collectXMLRchild{apply}{\nextXMLRelement}%
         \doifXMLRchildelse{minus/}{\the\XMLRtoks}
           {\countXMLRchildren{\the\XMLRtoks}
            \ifnum\nofXMLRchildren>2 +\else\fi}{+}}
        {+}}
     {}
     {#3\empty}}

\remapXMLsingular [power] [CPA] \MMLcPOWER

\setupMMLappearance[power][\c!reductie=\v!ja]

\let\MMLpowerelement\empty

\def\MMLcPOWER#1#2%
  {\withnexttwoXMLRelements
     {\doifXMLRchildelse{apply}{\firstXMLRelement}
        {\doifelse\@@MMLpowerreductie\v!ja
           {\collectXMLRchild{apply}{\firstXMLRelement}%
            \@EA\doifXMLRchildelse\@EA{\MMLcfunctionlist}{\the\XMLRtoks}
              {\let\MMLpowerelement\secondXMLRelement
               \MMLcreset\firstXMLRelement\empty}
              {\left(\MMLcreset\firstXMLRelement\empty\right)^{\MMLcreset\secondXMLRelement}}}
           {\left(\MMLcreset\firstXMLRelement\empty\right)^{\MMLcreset\secondXMLRelement}}}
        {\firstXMLRelement^{\MMLcreset\secondXMLRelement}}}}

\remapXMLsingular [rem] [CPA] \MMLcREM

\def\MMLcREM#1#2%
  {\withnexttwoXMLRelements
     {\firstXMLRelement\getXMLentity{mod}\secondXMLRelement}}

\remapXMLsingular [times] [CPA] \MMLcTIMES

\setupMMLappearance[times][\c!symbool=\v!nee]

\def\MMLcTIMES#1#2#3\empty
  {\setMMLcreset{\MMLcfunctionlist,\MMLcconstructlist}%
   \doifelse\@@MMLtimessymbool\v!ja
     {\encapsulateXMLR{}{\times}{}{#3\empty}}
     {\doifelse\@@MMLtimessymbool{dot}
        {\encapsulateXMLR{}{\cdot}{}{#3\empty}}
        {#3\empty}}}

\remapXMLsingular [root] [CPA] \MMLcROOT

% \def\MMLcROOT#1#2#3\empty
%   {\doifXMLRchildelse{degree}{#3}
%      {\root\processXMLRchild{degree}{#3}\of
%         {\let\MMLcDEGREE\ignoreXMLRelement
%          \MMLcreset\processXMLRchildren{#3}}}
%      {\sqrt{\MMLcreset\processXMLRchildren{#3}}}}

\setupMMLappearance[root][\c!symbool=\v!ja]

\def\MMLcROOT#1#2#3\empty
  {\doifXMLRchildelse{degree}{#3}\donetrue\donefalse
   \doif\@@MMLrootsymbool\v!nee\donefalse
   \ifdone  
     \root\processXMLRchild{degree}{#3}\of
       {\let\MMLcDEGREE\ignoreXMLRelement
        \MMLcreset\processXMLRchildren{#3}}%
   \else
     \sqrt
       {\MMLcreset\processXMLRchildren{#3}}%
   \fi}

\remapXMLsingular [gcd] [CPA] \MMLcGCD

\def\MMLcGCD#1#2#3\empty{\encapsulateXMLR{\gcd(}{\MMLccomma}{)}{#3}}

\remapXMLsingular [and]     [CPA] \MMLcAND
\remapXMLsingular [or]      [CPA] \MMLcOR
\remapXMLsingular [xor]     [CPA] \MMLcXOR
\remapXMLsingular [implies] [CPA] \MMLcIMPLIES

\def\MMLcAND    #1#2#3\empty{\encapsulateXMLR{}{\getXMLentity{and}}{}{#3}}
\def\MMLcOR     #1#2#3\empty{\encapsulateXMLR{}{\getXMLentity{or}}{}{#3}}
\def\MMLcXOR    #1#2#3\empty{\encapsulateXMLR{}{\getXMLentity{xor}}{}{#3}}
\def\MMLcIMPLIES#1#2#3\empty{\encapsulateXMLR{}{\getXMLentity{implies}}{}{#3}}

\remapXMLsingular [not]    [CPA] \MMLcNOT

\def\MMLcNOT#1#2#3\empty{\getXMLentity{not}\processXMLRchildren{#3}}

\remapXMLsingular [forall] [CPA] \MMLcFORALL
\remapXMLsingular [exists] [CPA] \MMLcEXISTS

\def\MMLcFORALL%
  {\getXMLentity{forall}%
   \!% \getXMLentity{NegThinSpace}%
   \MMLcFORALLandEXISTS}

\def\MMLcEXISTS%
  {\getXMLentity{exists}%
   \MMLcFORALLandEXISTS}

\def\MMLcFORALLandEXISTS#1#2#3\empty
  {\MMLcreset
   _{\encapsulateXMLRchildren{bvar}{}{\MMLccomma}{}{#3}}%
   \doifXMLRchildelse{condition}{#3}
     {\;% \getXMLentity{ThickSpace}%
      \processXMLRchild{condition}{#3}
      \doifXMLRchildelse{apply,reln,ci,cn}{#3}
        {\countXMLRchild{apply,reln,ci,cn}{#3}%
         \ifcase\nofXMLRchildren\or % == snelle volgende
           \encapsulateXMLRchildren{apply,reln,ci,cn}
             {\left\vert}% 
             {}
             {\right.}
             {#3\empty}%
         \else % special case
           \pushmacro\MMLcCONDITION
           \let\MMLcCONDITION\gobbletwoarguments
           \collectbetweenXMLRchild{apply,reln,ci,cn}{\hfill\crcr}{#3}%
           \left\vert\matrix{\the\XMLRtoks}\right.%
           \popmacro\MMLcCONDITION
         \fi}
        {}}
     {:\processXMLRchild{apply,reln,ci,cn}{#3\empty}}}

\remapXMLsingular [abs] [CPA] \MMLcABS

\def\MMLcABS#1#2#3\empty
  {\left\vert\MMLcreset\processXMLRchildren{#3}\right\vert}

\remapXMLsingular [conjugate] [CPA] \MMLcCONJUGATE

\def\MMLcCONJUGATE#1#2#3\empty%
  {\overline{\MMLcreset\processXMLRchildren{#3}}}

\remapXMLsingular [arg] [CPA] \MMLcARG

\def\MMLcARG#1#2#3\empty
  {\getXMLentity{arg}(\MMLcreset\processXMLRchildren{#3})}

\remapXMLsingular [real]      [CPA] \MMLcREAL
\remapXMLsingular [imaginary] [CPA] \MMLcIMAGINARY

\def\MMLcREAL#1#2#3\empty
  {\getXMLentity{real}(\MMLcreset\processXMLRchildren{#3})}

\def\MMLcIMAGINARY#1#2#3\empty
  {\getXMLentity{imaginary}(\MMLcreset\processXMLRchildren{#3})}

\remapXMLsingular [lcm] [CPA] \MMLcLCM

\def\MMLcLCM#1#2#3\empty
  {\encapsulateXMLR{\getXMLentity{lcm}(}{\MMLccomma}{)}{#3}}

\remapXMLsingular [floor]   [CPA] \MMLcFLOOR
\remapXMLsingular [ceiling] [CPA] \MMLcCEILING

\def\MMLcFLOOR#1#2#3\empty
  {\getXMLentity{lfloor}\processXMLRchildren{#3}\getXMLentity{rfloor}}

\def\MMLcCEILING#1#2#3\empty
  {\getXMLentity{lceiling}\processXMLRchildren{#3}\getXMLentity{rceiling}}

%%% relations

\remapXMLsingular [eq]         [CPA] \MMLcEQ
\remapXMLsingular [neq]        [CPA] \MMLcNEQ
\remapXMLsingular [gt]         [CPA] \MMLcGT
\remapXMLsingular [lt]         [CPA] \MMLcLT
\remapXMLsingular [geq]        [CPA] \MMLcGEQ
\remapXMLsingular [leq]        [CPA] \MMLcLEQ
\remapXMLsingular [equivalent] [CPA] \MMLcEQUIVALENT
\remapXMLsingular [approx]     [CPA] \MMLcAPPROX
\remapXMLsingular [factorof]   [CPA] \MMLcFACTOROF

\setupMMLappearance[relation][\c!uitlijnen=\v!nee]

\def\noMMLcrelation#1#2%
  {\encapsulatenextXMLRelements{}{#1}{}#2\empty}

\def\lastMMLcrelation#1#2%
  {\countXMLRchildren{#2}%
   \rawcollectbetweenXMLR
     {\advance\nofXMLRchildren -1 
      \ifnum\nofXMLRchildren>1 
        \appendtoks\@col@amp@#1\crcr\to\XMLRtoks
      \else
        \appendtoks\@col@amp@#1\to\XMLRtoks
      \fi}
     {#2}%
   \eqalign{\the\XMLRtoks\crcr}}

\def\firstMMLcrelation#1#2%
  {\nofXMLRchildren=0
   \rawcollectbetweenXMLR
     {\advance\nofXMLRchildren 1 
      \ifnum\nofXMLRchildren=1 
        \appendtoks\@col@amp@#1\to\XMLRtoks
      \else
        \appendtoks\crcr\@col@amp@#1\to\XMLRtoks
      \fi}
     {#2}%
   \eqalign{\the\XMLRtoks\crcr}}

\def\leftMMLcrelation#1#2%
  {\collectbetweenXMLR{\@col@amp@#1\crcr}{#2}%
   \eqalign{\the\XMLRtoks}}

\def\rightMMLcrelation#1#2%
  {\collectbetweenXMLR{\crcr#1{}\@col@amp@}{#2}% watch the {}
   \prependtoks\@col@amp@\to\XMLRtoks
   \eqalign{\the\XMLRtoks\crcr}}

\def\MMLcrelation#1#2\empty
  {\MMLcreset
   \processaction
     [\@@MMLrelationuitlijnen]
     [\v!laatste=>\let\next\lastMMLcrelation ,
       \v!eerste=>\let\next\firstMMLcrelation,
           \v!ja=>\let\next\leftMMLcrelation ,
        \v!links=>\let\next\leftMMLcrelation ,
       \v!rechts=>\let\next\rightMMLcrelation,  
      \s!default=>\let\next\noMMLcrelation   ,
      \s!unknown=>\let\next\noMMLcrelation   ]
   \next{#1}{#2}}

\def\MMLcEQ        #1#2{\MMLcrelation=}
\def\MMLcNEQ       #1#2{\MMLcrelation\neq}
\def\MMLcGT        #1#2{\MMLcrelation>}
\def\MMLcLT        #1#2{\MMLcrelation<}
\def\MMLcGEQ       #1#2{\MMLcrelation\geq}
\def\MMLcLEQ       #1#2{\MMLcrelation\leq}
\def\MMLcEQUIVALENT#1#2{\MMLcrelation\equiv}
\def\MMLcAPPROX    #1#2{\MMLcrelation\approx}
\def\MMLcFACTOROF  #1#2{\MMLcrelation\mid}

%%% introduced when making an example for Thorsten Bahne:

\remapXMLsingular [becomes] [CPA] \MMLcBECOMES

\def\MMLcBECOMES#1#2{\MMLcrelation{:=}}

\addtocommalist{becomes/}\MMLcmainresetlist 

%%% calculus and vector calculus

\setupMMLappearance[int][\c!plaats=\v!boven]

\remapXMLsingular [int] [CPA] \MMLcINT

\def\doMMLlimits#1%
  {\doifelsevalue{@@MML#1\c!plaats}\v!boven\limits\nolimits}

\def\MMLcINT#1#2#3\empty
  {\pushmacro\xmlr
   \MMLcreset
   \doifXMLRchildelse{domainofapplication}{#3}
     {\int\doMMLlimits{int}_{\processXMLRchild{domainofapplication}{#3}}}
     {\doifXMLRchildelse{condition}{#3}
        {\int\doMMLlimits{int}_{\processXMLRchild{condition}{#3}}}
        {\doifXMLRchildelse{lowlimit}{#3}
           {\int\doMMLlimits{int}
              _{\processXMLRchild{lowlimit}{#3}}
              ^{\processXMLRchild {uplimit}{#3}}}
           {\int}}}%
  \doifXMLRchildelse{apply}{#3}
    {\doifMMLfunctionelse{#3}%
       {\MMLcreset % TEST 
        \processXMLRchild{apply}{#3}}
       {\left(\MMLcreset % TEST 
              \processXMLRchild{apply}{#3}\right)}}
     {\MMLcreset % TEST 
      \processXMLRchild{ci}{#3}}%
   \doifXMLRchild{bvar}{#3}
     {\,% \getXMLentity{ThinSpace}%
      {\rm\getXMLentity{mathematicald}}\processXMLRchild{bvar}{#3}}%
   \popmacro\xmlr}

\remapXMLsingular [diff] [CPA] \MMLcDIFF

\setupMMLappearance[diff][\c!plaats=\v!boven]

\def\MMLcDIFF#1#2#3\empty
  {\pushmacro\xmlr
   \MMLcreset
   \doifXMLRchildelse{bvar}{#3}
     {\frac
        {{\rm \getXMLentity{mathematicald}}%
         \collectXMLRchild{bvar}{#3}%
         \let\MMLcDEGREE\superMMLcelement
         \expanded{\processXMLRchild{degree}{\the\XMLRtoks}}%
         \doif\@@MMLdiffplaats\v!boven
           {\doifXMLRchildelse{ci}{#3}
              {\processXMLRchild{ci}{#3}}
              {\doifMMLfunctionelse{#3}
                 {\MMLcreset\processXMLRchild{apply}{#3}}
                 {\left(\MMLcreset\processXMLRchild{apply}{#3}\right)}}}}
        {\let\MMLcDEGREE\superMMLcelement
         \begingroup\rm\getXMLentity{mathematicald}\endgroup
         \processXMLRchild{bvar}{#3}}%
      \doifnot\@@MMLdiffplaats\v!boven
        {\left(\MMLcreset\processXMLRchild{apply,ci}{#3}\right)}}
     {\processXMLRchildren{#3}^\prime}%
   \popmacro\xmlr}

\remapXMLsingular [partialdiff] [CPA] \MMLcPARTIALDIFF

\def\MMLcBVARpartialdiff#1#2%
  {\getXMLentity{differentiald}\processXMLRchild{apply,reln,ci,cn}{#2}
   \doifXMLRchild{degree}{#2}{^{\processXMLRchild{degree}{#2}}}}

\def\MMLcPARTIALDIFF#1#2#3\empty
  {\pushmacro\xmlr
   \doifXMLRchildelse{list}{#3}
     {\getXMLentity{capitaldifferentiald}_{\encapsulateXMLRchild{list}{}{\MMLccomma}{}{#3}}%
      \processXMLRchild{apply,reln,ci,cn}{#3}}
     {\countXMLRchild{bvar}{#3}%
      \ifnum\nofXMLRchildren>0
        \pushmacro\MMLcDEGREE
        \pushmacro\MMLcBVAR
        \frac
          {\doifXMLRchildelse{degree}{#3}
             {\collectXMLRchild{degree}{#3}}
             {\collectXMLRchild{bvar}{#3}%
              \expanded{\collectXMLRchild{degree}{\the\XMLRtoks}}}%
           \getXMLentity{differentiald}^{\encapsulateXMLR{}{+}{}{\the\XMLRtoks}}%
           \let\MMLcDEGREE\gobbletwoarguments
           \doifMMLfunctionelse{#3}{\MMLcreset}{}%
           \processXMLRchild{apply,reln,ci,cn}{#3}}
          {\let\MMLcBVAR\MMLcBVARpartialdiff
           \processXMLRchild{bvar}{#3}}%
        \popmacro\MMLcBVAR
        \popmacro\MMLcDEGREE
      \else
        \processXMLRchild{apply,reln,ci,cn}{#3}%
      \fi}%
   \popmacro\xmlr}

\def\doifMMLfunctionelse#1#2#3%
  {\collectXMLRchild{apply}{#1}%
   \doifXMLRchildelse{fn}{\the\XMLRtoks#1} 
     {#2}
     {\doifXMLRchildelse{ci}{\the\XMLRtoks#1} 
        {\pushmacro\xmlr
         \getXMLarguments{dummy}{type=""}%
         \def\xmlr{\getmmlarguments{dummy}{}}%
         \the\XMLRtoks % what if more than one ? 
         \popmacro\xmlr
         \doif{\XMLpar{dummy}{type}{}}{fn}{#2}{#3}}
        {#2}}}

\def\subMMLcelement  #1#2{^{#2}}
\def\superMMLcelement#1#2{^{#2}}

\remapXMLsequence [lowlimit][CPA] \MMLcLOWLIMIT
\remapXMLsequence [uplimit] [CPA] \MMLcUPLIMIT
\remapXMLsequence [bvar]    [CPA] \MMLcBVAR
\remapXMLsequence [degree]  [CPA] \MMLcDEGREE

\def\MMLcLOWLIMIT#1#2{#2}
\def\MMLcUPLIMIT #1#2{#2}
\def\MMLcBVAR    #1#2{#2}
\def\MMLcDEGREE  #1#2{#2}

\remapXMLsingular [divergence] [CPA] \MMLcDIVERGENCE
\remapXMLsingular [grad]       [CPA] \MMLcGRAD
\remapXMLsingular [curl]       [CPA] \MMLcCURL
\remapXMLsingular [laplacian]  [CPA] \MMLcLAPLACIAN

\def\MMLcDIVERGENCE#1#2#3\empty{\getXMLentity{divergence}\processXMLRchildren{#3}}
\def\MMLcGRAD      #1#2#3\empty{\getXMLentity{grad}\processXMLRchildren{#3}}
\def\MMLcCURL      #1#2#3\empty{\getXMLentity{curl}\processXMLRchildren{#3}}
\def\MMLcLAPLACIAN #1#2#3\empty{\getXMLentity{laplacian}\processXMLRchildren{#3}}

%%% theory of sets

\remapXMLsequence [set] [CPA] \MMLcSET

\def\MMLcSET#1#2%
  {\doifXMLRchildelse{condition}{#2}
     {\{\processXMLRchild{bvar}{#2}\,\vert\,\processXMLRchild{condition}{#2}\}}
     {\encapsulateXMLR{\{}{\MMLccomma}{\}}{#2}}}

\remapXMLsequence [list] [CPA] \MMLcLIST

\def\MMLcLIST#1#2{\encapsulateXMLR{[}{\MMLccomma}{]}{#2}}

\remapXMLsingular [union]            [CPA] \MMLcUNION
\remapXMLsingular [intersect]        [CPA] \MMLcINTERSECT
\remapXMLsingular [in]               [CPA] \MMLcIN
\remapXMLsingular [notin]            [CPA] \MMLcNOTIN
\remapXMLsingular [subset]           [CPA] \MMLcSUBSET
\remapXMLsingular [prsubset]         [CPA] \MMLcPRSUBSET
\remapXMLsingular [notsubset]        [CPA] \MMLcNOTSUBSET
\remapXMLsingular [notprsubset]      [CPA] \MMLcNOTPRSUBSET
\remapXMLsingular [setdiff]          [CPA] \MMLcSETDIFF

\def\MMLcset#1{\withnexttwoXMLRelements{\firstXMLRelement#1\secondXMLRelement}}

\def\MMLcUNION           #1#2{\MMLcset\cup}
\def\MMLcINTERSECT       #1#2{\MMLcset\cap}
\def\MMLcIN              #1#2{\MMLcset\in}
\def\MMLcNOTIN           #1#2{\MMLcset{\not\in}}
\def\MMLcSUBSET          #1#2{\MMLcset\subset}
\def\MMLcPRSUBSET        #1#2{\MMLcset\subseteq}
\def\MMLcNOTSUBSET       #1#2{\MMLcset{\not\subset}}
\def\MMLcNOTPRSUBSET     #1#2{\MMLcset{\not\subseteq}}
\def\MMLcSETDIFF         #1#2{\MMLcset\setminus}

\remapXMLsingular [card] [CPA] \MMLcCARD

\def\MMLcCARD#1#2#3\empty{\encapsulateXMLR{\vert}{}{\vert}{#3}}

\remapXMLsingular [cartesianproduct] [CPA] \MMLcCARTESIANPRODUCT

\def\MMLcCARTESIANPRODUCT#1#2#3\empty{\encapsulateXMLR{}{\times}{}{#3}}

%%% sequences and series

\remapXMLsingular [sum]     [CPA] \MMLcSUM
\remapXMLsingular [product] [CPA] \MMLcPRODUCT

\def\MMLcSUM    {\MMLcSUMandPRODUCT{sum}\sum}
\def\MMLcPRODUCT{\MMLcSUMandPRODUCT{product}\prod}

\setupMMLappearance[sum][\c!plaats=\v!boven]
\setupMMLappearance[product][\c!plaats=\v!boven]

\def\stackMMLsubscripts#1%
  {\vbox
     {\baselineskip\!!zeropoint % hack, taco vragen 
      \halign{$\scriptstyle\hss##\hss$\cr#1\crcr}}}

\def\MMLcSUMandPRODUCT#1#2#3#4\empty
  {\pushmacro\xmlr
   \doifXMLRchildelse{condition,bvar,lowlimit}{#4}
     {\def\MMLcSUMlow 
        {_{\doifXMLRchildelse{condition}{#4}
             {\collectXMLRchild{condition}{#4}%
              \expanded{\collectbetweenXMLR{\crcr}{\the\XMLRtoks}}%
              \stackMMLsubscripts{\the\XMLRtoks}}
             {\doifXMLRchild{bvar}{#4}
                {\processXMLRchild{bvar}{#4}%
                 \doifXMLRchild{lowlimit}{#4}{=}}%
              \processXMLRchild{lowlimit}{#4}}}}}
     {\let\MMLcSUMlow\empty}%
   \doifXMLRchildelse{uplimit}{#4}
     {\def\MMLcSUMup{^{\processXMLRchild{uplimit}{#4}}}}
     {\let\MMLcSUMup\empty}%
   \MMLcreset#2\doMMLlimits{#1}\MMLcSUMup\MMLcSUMlow
   \MMLcreset
   \processXMLRchild{apply,ci}{#4}%
   \popmacro\xmlr}

\remapXMLsingular [limit] [CPA] \MMLcLIMIT

\setupMMLappearance[limit][\c!plaats=\v!boven]

\def\MMLcLIMIT#1#2#3\empty
  {\pushmacro\xmlr
   \MMLcreset
   \lim\doMMLlimits{limit}_
     {\MMLcreset
      \doifXMLRchildelse{condition}{#3}
        {\processXMLRchild{condition}{#3}}
        {\doifXMLRchild{bvar}{#3}
           {\processXMLRchild{bvar}{#3}\rightarrow}%
         \processXMLRchild{lowlimit}{#3}}}%
   \processXMLRchild{apply}{#3}%
   \popmacro\xmlr}

\remapXMLsingular [tendsto] [CPA] \MMLcTENDSTO

\def\MMLcTENDSTO#1#2%
  {\getXMLarguments{tendsto}{type="default" #1}%
   \withnexttwoXMLRelements
     {\MMLcreset\firstXMLRelement
      \processaction
        [\XMLpar{tendsto}{type}{default}]
        [  above=>\downarrow,
           below=>\uparrow,
         unknown=>\rightarrow]%
      \MMLcreset\secondXMLRelement}}

%%% elementary classical functions

\remapXMLsingular [exp] [CPA] \MMLcEXP

\def\MMLcEXP#1#2#3\empty
  {\getXMLentity{exponentiale}^{\MMLcreset#3}}

\remapXMLsingular [ln]  [CPA] \MMLcLN

%\def\MMLcLN#1#2#3\empty
%  {\ln\left(\MMLcreset#3\right)}

\def\MMLcLN#1#2%
  {\doMMLcfunction{ln}}

\remapXMLsingular [log] [CPA] \MMLcLOG

\setupMMLappearance[log][\c!plaats=\v!rechts]

\def\MMLcLOG#1#2#3\empty
  {\pushmacro\MMLcLOGBASE
   \let\MMLcLOGBASE\secondoftwoarguments
   \doifXMLRchildelse{logbase}{#3}
     {\doifelse\@@MMLlogplaats\v!links
        {\mathop{{}^{\processXMLRchild{logbase}{#3}}\!\getXMLentity{log}}}
        {\getXMLentity{log}_{\processXMLRchild{logbase}{#3}}}}
     {\getXMLentity{log}}%
   \let\MMLcLOGBASE\ignoreXMLRelement
   \nodoMMLcfunction#3\empty 
   \popmacro\MMLcLOGBASE}

\remapXMLsequence [logbase] [CPA] \MMLcLOGBASE

\def\MMLcLOGBASE#1#2{#2}

\remapXMLsingular [sin]     [CPA] \MMLcSIN
\remapXMLsingular [arcsin]  [CPA] \MMLcARCSIN
\remapXMLsingular [sinh]    [CPA] \MMLcSINH
\remapXMLsingular [arcsinh] [CPA] \MMLcARCSINH
\remapXMLsingular [cos]     [CPA] \MMLcCOS
\remapXMLsingular [arccos]  [CPA] \MMLcARCCOS
\remapXMLsingular [cosh]    [CPA] \MMLcCOSH
\remapXMLsingular [arccosh] [CPA] \MMLcARCCOSH
\remapXMLsingular [tan]     [CPA] \MMLcTAN
\remapXMLsingular [arctan]  [CPA] \MMLcARCTAN
\remapXMLsingular [tanh]    [CPA] \MMLcTANH
\remapXMLsingular [arctanh] [CPA] \MMLcARCTANH
\remapXMLsingular [cot]     [CPA] \MMLcCOT
\remapXMLsingular [arccot]  [CPA] \MMLcARCCOT
\remapXMLsingular [coth]    [CPA] \MMLcCOTH
\remapXMLsingular [arccoth] [CPA] \MMLcARCCOTH
\remapXMLsingular [csc]     [CPA] \MMLcCSC
\remapXMLsingular [arccsc]  [CPA] \MMLcARCCSC
\remapXMLsingular [csch]    [CPA] \MMLcCSCH
\remapXMLsingular [arccsch] [CPA] \MMLcARCCSCH
\remapXMLsingular [sec]     [CPA] \MMLcSEC
\remapXMLsingular [arcsec]  [CPA] \MMLcARCSEC
\remapXMLsingular [sech]    [CPA] \MMLcSECH
\remapXMLsingular [arcsech] [CPA] \MMLcARCSECH

\setupMMLappearance[function][\c!reductie=\v!ja]

\def\doMMLcfunction#1%
  {\doifnextcharelse\xmlr{\dodoMMLcfunction{#1}}{\getXMLentity{#1}}}

\def\dodoMMLcfunction#1%
  {\getXMLentity{#1}%
   \nodoMMLcfunction}

\def\nodoMMLcfunction#1\empty
  {\ifx\MMLpowerelement\empty\else
     ^{\MMLcreset\MMLpowerelement\empty}\!\let\MMLpowerelement\empty
   \fi
   \doifelse\@@MMLfunctionreductie\v!ja
     {\doifXMLRchildelse{apply}{#1}
        {\collectXMLRchild{apply}{#1}%
         \@EA\doifXMLRchildelse\@EA{\MMLcfunctionlist}{\the\XMLRtoks}
           {\MMLcreset#1\empty}
           {\left(\MMLcreset#1\empty\right)}}
        {\MMLcreset#1\empty}}
     {\left(\MMLcreset#1\empty\right)}}

\def\MMLcSIN    #1#2{\doMMLcfunction{sin}}
\def\MMLcARCSIN #1#2{\doMMLcfunction{arcsin}}
\def\MMLcSINH   #1#2{\doMMLcfunction{sinh}}
\def\MMLcARCSINH#1#2{\doMMLcfunction{arcsinh}}
\def\MMLcCOS    #1#2{\doMMLcfunction{cos}}
\def\MMLcARCCOS #1#2{\doMMLcfunction{arccos}}
\def\MMLcCOSH   #1#2{\doMMLcfunction{cosh}}
\def\MMLcARCCOSH#1#2{\doMMLcfunction{arccosh}}
\def\MMLcTAN    #1#2{\doMMLcfunction{tan}}
\def\MMLcARCTAN #1#2{\doMMLcfunction{arctan}}
\def\MMLcTANH   #1#2{\doMMLcfunction{tanh}}
\def\MMLcARCTANH#1#2{\doMMLcfunction{arctanh}}
\def\MMLcCOT    #1#2{\doMMLcfunction{cot}}
\def\MMLcARCCOT #1#2{\doMMLcfunction{arccot}}
\def\MMLcCOTH   #1#2{\doMMLcfunction{coth}}
\def\MMLcARCCOTH#1#2{\doMMLcfunction{arccoth}}
\def\MMLcCSC    #1#2{\doMMLcfunction{csc}}
\def\MMLcARCCSC #1#2{\doMMLcfunction{arccsc}}
\def\MMLcCSCH   #1#2{\doMMLcfunction{csch}}
\def\MMLcARCCSCH#1#2{\doMMLcfunction{arccsch}}
\def\MMLcSEC    #1#2{\doMMLcfunction{sec}}
\def\MMLcARCSEC #1#2{\doMMLcfunction{arcsec}}
\def\MMLcSECH   #1#2{\doMMLcfunction{sech}}
\def\MMLcARCSECH#1#2{\doMMLcfunction{arcsech}}

%%% statistics

\remapXMLsingular [mean]     [CPA] \MMLcMEAN
\remapXMLsingular [sdev]     [CPA] \MMLcSDEV
\remapXMLsingular [variance] [CPA] \MMLcVARIANCE
\remapXMLsingular [median]   [CPA] \MMLcMEDIAN
\remapXMLsingular [mode]     [CPA] \MMLcMODE

\def\MMLcMEAN    #1#2{\withnextXMLRelement{\overline{\nextXMLRelement}}}
\def\MMLcSDEV    #1#2{\withnextXMLRelement{\sigma(\MMLcreset\nextXMLRelement)}}
\def\MMLcVARIANCE#1#2{\withnextXMLRelement{\sigma(\MMLcreset\nextXMLRelement)^2}}
\def\MMLcMEDIAN  #1#2{\withnextXMLRelement{\getXMLentity{median}(\MMLcreset\nextXMLRelement)}}
\def\MMLcMODE    #1#2{\withnextXMLRelement{\getXMLentity{mode}(\MMLcreset\nextXMLRelement)}}

\remapXMLsingular [moment]      [CPA] \MMLcMOMENT
\remapXMLsequence [momentabout] [CPA] \MMLcMOMENTABOUT

\def\MMLcMOMENT#1#2#3\empty
 {\left\langle\processXMLRchild{apply,reln,ci,cn}{#3}%
  ^{\processXMLRchild{degree}{#3}}\right\rangle}

\def\MMLcMOMENTABOUT#1#2{}

%%% linear algebra

\remapXMLsequence [vector] [CPA] \MMLcVECTOR

\setupMMLappearance[vector][\c!richting=\v!horizontaal] % \v!vertikaal

\def\MMLcVECTOR#1#2%
   {\countXMLRchildren{#2}% 
    \ifnum\nofXMLRchildren>1
      \doifelse\@@MMLvectorrichting\v!horizontaal
        {\encapsulateXMLR{\left(}{\MMLccomma}{\right)}{#2}}
        {\collectbetweenXMLR{\crcr}{#2}%
         \MMLcreset\left(\matrix{\the\XMLRtoks}\right)}%
    \else
      \overrightarrow{#2}%
    \fi}

\remapXMLsequence [matrix]    [CPA] \MMLcMATRIX
\remapXMLsequence [matrixrow] [CPA] \MMLcMATRIXROW

\unexpanded\def\@col@amp@{&}

\def\doMMLcMATRIX#1#2#3%
  {\pushmacro\MMLcMATRIXROW
   \let\MMLcMATRIXROW\normalMMLcMATRIXROW
   \MMLcreset
   \ifcase#1\matrix{#3}\else\left(\matrix{#3}\right)\fi
   \popmacro\MMLcMATRIXROW}

\def\MMLcMATRIX#1#2%
  {\doMMLcMATRIX1{#1}{#2}}

\def\MMLcMATRIXROW#1#2%
  {\collectbetweenXMLR{\@col@amp@}{#2}\the\XMLRtoks\crcr}

\let\normalMMLcMATRIXROW\MMLcMATRIXROW

\def\MMLcMATRIXROW#1#2%
  {\collectbetweenXMLR{\@col@amp@}{#2}%
   \left(\matrix{\the\XMLRtoks\crcr}\right)}

\remapXMLsingular [determinant] [CPA] \MMLcDETERMINANT

% \def\MMLcDETERMINANT#1#2{\getXMLentity{determinant}} % optie

\def\MMLcDETERMINANT#1#2\empty
  {\pushmacro\MMLcMATRIX
   \def\MMLcMATRIX##1##2{\doMMLcMATRIX0{##1}{##2}}%
   \left|#2\empty\right|%
   \popmacro\MMLcMATRIX}

\remapXMLsingular [transpose] [CPA] \MMLcTRANSPOSE

\def\MMLcTRANSPOSE#1#2{\withnextXMLRelement{\nextXMLRelement^{\rm T}}}

\remapXMLsingular [selector] [CPA] \MMLcSELECTOR

\def\MMLcSELECTOR#1#2#3\empty
  {\pushmacro\xmlr
   \withnextXMLRelement
     {\mathinner{\nextXMLRelement}% hm, assumes a matrix or so 
      \doMMLcSELECTOR}%
   #3\empty
   \popmacro\xmlr}

\def\doMMLcSELECTOR#1\empty
  {_{\MMLcreset\encapsulateXMLR{}{\MMLccomma}{}{#1\empty}}}

\remapXMLsingular [vectorproduct] [CPA] \MMLcVECTORPRODUCT
\remapXMLsingular [scalarproduct] [CPA] \MMLcSCALARPRODUCT
\remapXMLsingular [outerproduct]  [CPA] \MMLcOUTERPRODUCT

\def\MMLcvector#1%
  {\withnexttwoXMLRelements{\firstXMLRelement#1\secondXMLRelement}}

\def\MMLcVECTORPRODUCT#1#2{\MMLcvector{\getXMLentity{vectorproduct}}#2}% #2?%
\def\MMLcSCALARPRODUCT#1#2{\MMLcvector{\getXMLentity{scalarproduct}}#2}
\def\MMLcOUTERPRODUCT #1#2{\MMLcvector{\getXMLentity{outerproduct}}#2}

%%% semantic mapping elements

\remapXMLsequence [semantics] [CPA] \MMLcSEMANTICS

\def\MMLcSEMANTICS#1#2%
  {\doifXMLRchildelse{annotation}{#2}
     {\processXMLRchild{annotation}{#2}}
     {#2}}

\remapXMLsequence [annotation] [CPA] \MMLcANNOTATION

\def\MMLcANNOTATION#1#2% we need a better unmapper 
  {\getXMLarguments{annotation}{encoding="" #1}%
   \doif{\XMLpar{annotation}{encoding}{}}{TeX}%
     {\begingroup
      \setnormalXMLentities % better: \simplifyXMLentities ; test first  
      \let\xmlrent\expandedXMLentity 
      \edef\mmlascii{#2}%           
      \setnormalcatcodes
      \ifx\mmlascii\empty
        \donefalse
      \else
        \def\do##1##2\end            % hack
          {\edef\!!stringa{\string##1}%
           \edef\!!stringb{\letterdollar}%
           \ifx\!!stringa\!!stringb
             \donetrue
           \else
             \donefalse
           \fi}%
        \expandafter\do\mmlascii\end
      \fi
      \ifdone
        \hbox{\scantokens\@EA{\mmlascii}}%
      \else
        \scantokens\@EA{\mmlascii}%
      \fi
      \endgroup}}

\remapXMLsequence [annotation-xml] [CPA] \MMLcANNOTATIONXML

\def\MMLcANNOTATIONXML#1#2{}

%%% constant and symbol elements

\remapXMLsingular [integers]       [CPA] \MMLcINTEGERS
\remapXMLsingular [reals]          [CPA] \MMLcREALS
\remapXMLsingular [rationals]      [CPA] \MMLcRATIONALS
\remapXMLsingular [naturalnumbers] [CPA] \MMLcNATURALNUMBERS
\remapXMLsingular [complexes]      [CPA] \MMLcCOMPLEXES
\remapXMLsingular [primes]         [CPA] \MMLcPRIMES
\remapXMLsingular [exponentiale]   [CPA] \MMLcEXPONENTIALE
\remapXMLsingular [imaginaryi]     [CPA] \MMLcIMAGINARYI
\remapXMLsingular [notanumber]     [CPA] \MMLcNOTANUMBER
\remapXMLsingular [true]           [CPA] \MMLcTRUE
\remapXMLsingular [false]          [CPA] \MMLcFALSE
\remapXMLsingular [emptyset]       [CPA] \MMLcEMPTYSET
\remapXMLsingular [pi]             [CPA] \MMLcPI
\remapXMLsingular [eulergamma]     [CPA] \MMLcEULERGAMMA
\remapXMLsingular [infinity]       [CPA] \MMLcINFINITY

\def\MMLcINTEGERS      #1#2{\getXMLentity{integers}}
\def\MMLcREALS         #1#2{\getXMLentity{reals}}
\def\MMLcRATIONALS     #1#2{\getXMLentity{rationals}}
\def\MMLcNATURALNUMBERS#1#2{\getXMLentity{naturalnumbers}}
\def\MMLcCOMPLEXES     #1#2{\getXMLentity{complexes}}
\def\MMLcPRIMES        #1#2{\getXMLentity{primes}}
\def\MMLcEXPONENTIALE  #1#2{\getXMLentity{ExponentialE}}
\def\MMLcIMAGINARYI    #1#2{\getXMLentity{ImaginaryI}}
\def\MMLcNOTANUMBER    #1#2{\mathop{\rm NaN}}
\def\MMLcTRUE          #1#2{\mathop{\rm true}}
\def\MMLcFALSE         #1#2{\mathop{\rm false}}
\def\MMLcEMPTYSET      #1#2{\mathop{\hbox{\O}}}
\def\MMLcPI            #1#2{\pi}
\def\MMLcEULERGAMMA    #1#2{\gamma}
\def\MMLcINFINITY      #1#2{\infty}

\stopXMLmapping

\protect \endinput
