%D \module
%D   [       file=xtag-ext,
%D        version=2001.03.21,
%D          title=\CONTEXT\ XML Support,
%D       subtitle=Extra Macros, 
%D         author=Hans Hagen,
%D           date=\currentdate,
%D      copyright={PRAGMA / Hans Hagen \& Ton Otten}]
%C
%C This module is part of the \CONTEXT\ macro||package and is
%C therefore copyrighted by \PRAGMA. See mreadme.pdf for
%C details.

\unprotect 

\beginTEX 
  \writestatus{xml}{sorry, xml is only supported in (pdf)etex}
  \endinput 
\endTEX 

\writestatus{loading}{Context XML Macros (extras)}

\unprotect 

%D \macros 
%D  {startXMLnamespace}
%D 
%D You can define macros within a namespace, so that they 
%D will not conflict (don't confuse this with \XML\ 
%D namespaces.
%D
%D \starttypen
%D \startXMLnamespace [tag] | [-] [tag] | [+] [tag]
%D   definitions 
%D \stopXMLnamespace 
%D \stoptypen 
%D 
%D When a \type {[+]} is specified, the namespaces will 
%D nest.  

\let\normal@@XMLelement\@@XMLelement
\let\XMLnamespace\empty

\def\startXMLnamespace%
  {\dodoubleempty\dostartXMLnamespace}

\def\dostartXMLnamespace[#1][#2]%
  {\pushmacro\@@XMLelement
   \pushmacro\@@XMLnamespace
   \ifsecondargument
     \doifelse{#1}{-}
       {\donostartXMLnamespace{#1}}
       {\doifelse{#1}{+}
          {\dodostartXMLnamespace{#1}{#2}}
          {\donostartXMLnamespace{#1}}}%
   \else
     \donostartXMLnamespace{#1}%
   \fi
   \unprotect}

\def\donostartXMLnamespace#1%
  {\protect
   \edef\XMLnamespace{#1}%
   \let\@@XMLelement\normal@@XMLelement}

\def\dodostartXMLnamespace#1#2%
  {\edef\XMLnamespace{\XMLnamespace:#2}%
   \edef\@@XMLelement{\@@XMLelement:#2}}

\def\stopXMLnamespace%
  {\popmacro\@@XMLnamespace
   \popmacro\@@XMLelement}

\protect \endinput
