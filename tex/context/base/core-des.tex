%D \module
%D   [       file=core-des,
%D        version=1997.03.31,
%D          title=\CONTEXT\ Core Macros,
%D       subtitle=Descriptions,
%D         author=Hans Hagen,
%D           date=\currentdate,
%D      copyright={PRAGMA / Hans Hagen \& Ton Otten}]
%C
%C This module is part of the \CONTEXT\ macro||package and is
%C therefore copyrighted by \PRAGMA. See mreadme.pdf for
%C details.

\writestatus{loading}{Context Core Macros / Descriptions}

\unprotect

% Dit kan en moet dus anders:
%
% \start... :  \vbox\bgroup
% \stop...  :  \egroup
% llap enz.
% geen indent!
%
% enz. enz.
%
% Op die manier is meer mogelijk en worden \par's geskipt.
%
% De macro \??dd#1\s!do\c!commando levert de koppeling tussen
% \doornummeren en \doordefinieren. Deze constructie is nodig
% omdat doornummeren geen argument heeft en omdat subnummers
% niet worden genest binnen het hogere niveau. Het commando
% \??dd#1\s!do\c!status moet in dat geval \v!start zijn.
%
% herimplementeren met \nextbox en \unhbox\unvbox

\newbox\@@definitiebox

\def\@@definitiewoord#1%
  {\getvalue{\??dd#1\s!do\c!commando}{#1}}

% \def\normal@@definitiewoord#1[#2]#3#4%
%   {\doattributes
%      {\??dd#1}\c!kopletter\c!kopkleur
%      {\getvalue{\??dd#1\c!commando}% NAAR BUITENSTE NIVEAU !
%         {\begstrut\getvalue{\??dd#1\c!tekst}#4\endstrut}}%
%    \rawreference\s!def{#2}{#3}}

\def\normal@@definitiewoord#1[#2]#3#4%
  {\doattributes
     {\??dd#1}\c!kopletter\c!kopkleur
     {\getvalue{\??dd#1\c!commando}{#4}}% NAAR BUITENSTE NIVEAU !
   \rawreference\s!def{#2}{#3}} % brrr moet in #4

\setvalue{@@definitie\v!links}#1%
  {\@@definitiehang{#1}\@@definitielinkspure\@@definitielinkshang}

\setvalue{@@definitie\v!rechts}#1%
  {\@@definitiehang{#1}\@@definitierechtspure\@@definitierechtshang}

\def\@@definitiehang#1#2#3%
  {\processaction
     [\getvalue{\??dd#1\c!hang}]
     [   \v!geen=>\let\next#2,%
               0=>\let\next#2,%
      \s!unknown=>\let\next#3,%
      \s!default=>\let\next#2]%
   \next{#1}}

\def\@@definitielinkspure#1[#2]#3%
  {\@@dostartdefinitie{#1}[#2]{#3}%
   \noindent\ignorespaces
   \leftskip\@@leftdefinitieskip
   \rightskip\@@rightdefinitieskip
   \advance\leftskip \!!widtha
   \@@makedefinitiepurebox{#1}\raggedright
   \advance\leftskip \!!widthb
   \llap
     {\hbox to \leftskip
        {\hskip\@@leftdefinitieskip
         \copy\@@definitiebox\hss}}%
   \@@dodefinitie{#1}}

\def\@@definitierechtspure#1[#2]#3%
  {\@@dostartdefinitie{#1}[#2]{#3}%
   \noindent\ignorespaces
   \leftskip\@@leftdefinitieskip
   \rightskip\@@rightdefinitieskip
   \advance\rightskip \!!widtha
   \@@makedefinitiepurebox{#1}\raggedleft
   \rlap
     {\hskip\hsize
      \hskip-\leftskip
      \hskip-\rightskip
      \copy\@@definitiebox
      \hskip\@@rightdefinitieskip}%
   \advance\rightskip \!!widthb
   \@@dodefinitie{#1}}

\def\@@makedefinitiepurebox#1#2%
  {\setbox\@@definitiebox\vtop
     {\mindermeldingen
      \hsize\!!widtha
      \leftskip\zeropoint
      \rightskip\zeropoint
      #2\setupalign[\getvalue{\??dd#1\c!uitlijnen}]%
      \ifhbox\@@definitiebox\unhcopy\else\copy\fi\@@definitiebox}%
   \ht\@@definitiebox\strutht
   \dp\@@definitiebox\strutdp}

\def\@@definitielinkshang#1[#2]#3%
  {\@@dostartdefinitie{#1}[#2]{#3}%
   \mindermeldingen
   \advance\!!widtha \!!widthb
   \hangindent\!!widtha
   \@@makedefinitiehangbox{#1}\raggedright{\advance\rightskip \!!widthb}%
   \noindent\ignorespaces
   \llap
     {\dontshowcomposition
      \vtop to \zeropoint{\box\@@definitiebox}}%
   \@@dodefinitie{#1}}%

\def\@@definitierechtshang#1[#2]#3%
  {\@@dostartdefinitie{#1}[#2]{#3}%
   \mindermeldingen
   \advance\!!widtha \!!widthb
   \hangindent-\!!widtha
   \@@makedefinitiehangbox{#1}\raggedleft{\advance\leftskip \!!widthb}%
   \noindent\ignorespaces
   \rlap
     {\mindermeldingen
      \dontshowcomposition
      \dimen0=\hsize
      \advance\dimen0 -\leftskip
      \advance\dimen0 -\rightskip
      \hbox to \dimen0
        {\hss\vtop to \zeropoint{\box\@@definitiebox}}}%
   \@@dodefinitie{#1}}

\def\@@makedefinitiehangbox#1#2#3%
  {\setbox\@@definitiebox\vtop % \vbox gaat fout in hang
     {\forgetall
      \mindermeldingen
      \hsize\!!widtha
      #2\setupalign[\getvalue{\??dd#1\c!uitlijnen}]#3%
      \ifhbox\@@definitiebox\unhcopy\else\copy\fi\@@definitiebox}%
   \ht\@@definitiebox\strutht
   \dp\@@definitiebox\strutdp
   \doifinsetelse{\getvalue{\??dd#1\c!hang}}{\v!passend,\v!ruim}
     {\dimen0=\ht\@@definitiebox
      \advance\dimen0 \dp\@@definitiebox
      \doifvalue{\??dd#1\c!hang}\v!ruim
        {\advance\dimen0 .5\strutht}%
      \getnoflines{\dimen0}%
      \hangafter-\noflines}
     {\hangafter-\getvalue{\??dd#1\c!hang}}}%

\setvalue{@@definitie\v!boven}#1[#2]#3%
  {%\pagina[\v!voorkeur]%       % Weg ermee!
   %\dosomebreak{\goodbreak}%   % Dit is beter en nodig!
   \dohandelpaginaafX1          % En dit moet het maar worden.
   \@@dostartdefinitie{#1}[#2]{\let\\=\space#3}%
   \noindent\ignorespaces
   \copy\@@definitiebox\par
   \nobreak
   \getvalue{\??dd#1\c!tussen}%
   \nobreak
   \@@dodefinitie{#1}}

% \setvalue{@@definitie\v!inmarge}#1[#2]#3%
%   {\@@dostartdefinitie{#1}[#2]{#3}%
%    \noindent\ignorespaces
%    \inmarge{\ifhbox\@@definitiebox\unhcopy\else\copy\fi\@@definitiebox}%
%    \@@dodefinitie{#1}}
%
% \setvalue{@@definitie\v!inlinker}#1[#2]#3%
%   {\@@dostartdefinitie{#1}[#2]{#3}%
%    \noindent\ignorespaces
%    \inlinker{\ifhbox\@@definitiebox\unhcopy\else\copy\fi\@@definitiebox}%
%    \@@dodefinitie{#1}}
%
% \setvalue{@@definitie\v!inrechter}#1[#2]#3%
%   {\@@dostartdefinitie{#1}[#2]{#3}%
%    \noindent\ignorespaces
%    \inrechter{\ifhbox\@@definitiebox\unhcopy\else\copy\fi\@@definitiebox}%
%    \@@dodefinitie{#1}}

\def\do@@definitie#1#2[#3]#4%
  {\@@dostartdefinitie{#2}[#3]{#4}%
   \noindent\ignorespaces
   #1{\ifhbox\@@definitiebox\unhcopy\else\copy\fi\@@definitiebox}%
   \@@dodefinitie{#2}}

\setvalue{@@definitie\v!inmarge     }{\do@@definitie\inmarge  }
\setvalue{@@definitie\v!inlinker    }{\do@@definitie\inlinker }
\setvalue{@@definitie\v!inrechter   }{\do@@definitie\inrechter}
\setvalue{@@definitie\v!marge       }{\do@@definitie\inmarge  }
\setvalue{@@definitie\v!linkermarge }{\do@@definitie\inlinker }
\setvalue{@@definitie\v!rechtermarge}{\do@@definitie\inrechter}
\setvalue{@@definitie\v!binnenmarge }{\do@@definitie\inbinnen }
\setvalue{@@definitie\v!buitenmarge }{\do@@definitie\inbuiten }

\def\@@definitieaanelkaarpassend#1[#2]#3%
  {\@@dostartdefinitie{#1}[#2]{#3}%
   \noindent\ignorespaces
   \ifhbox\@@definitiebox\unhcopy\else\copy\fi\@@definitiebox
   \hskip\!!widthb % toegevoegd
   \@@dodefinitie{#1}}

\def\@@definitieaanelkaarruim#1[#2]#3%
  {\@@dostartdefinitie{#1}[#2]{#3}%
   \noindent\ignorespaces
   \ifhbox\@@definitiebox\unhcopy\else\copy\fi\@@definitiebox
   \hskip\!!widthb \!!plus .5\!!widthb \!!minus .25\!!widthb
   \@@dodefinitie{#1}}

\def\@@definitieaanelkaarbreed#1[#2]#3%
  {\@@dostartdefinitie{#1}[#2]{#3}%
   \noindent\ignorespaces
   \hbox to \!!widtha
     {\ifhbox\@@definitiebox\unhcopy\else\copy\fi\@@definitiebox\hss}%
   \hskip\!!widthb
   \ignorespaces
   \@@dodefinitie{#1}}

\setvalue{@@definitie\v!aanelkaar}#1[#2]#3%
  {\processaction
      [\getvalue{\??dd#1\c!breedte}]
      [\v!passend=>\let\next\@@definitieaanelkaarpassend,
          \v!ruim=>\let\next\@@definitieaanelkaarruim,
       \s!unknown=>\let\next\@@definitieaanelkaarbreed,
       \s!default=>\let\next\@@definitieaanelkaarruim]%
   \next{#1}[#2]{#3}}

\setvalue{@@definitie\v!hangend}#1[#2]#3%
  {\@@dostartdefinitie{#1}[#2]{#3}% % adds \c!marge to \leftskip
   \noindent\ignorespaces
   \advance\leftskip -\leftskipadaption \relax
   \ifdim\leftskipadaption=\zeropoint
     \leftskipadaption1.5em % just some default
     \ifnum\insidedefinition=\plusone
       \ifdim\leftskip>\zeropoint \relax
         \leftskipadaption\leftskip
       \fi
     \fi
   \fi
   \ifnum\insidedefinition=\plusone
     \advance\leftskip \leftskipadaption
   \fi
   \hskip-\leftskipadaption
   \ifhbox\@@definitiebox\unhcopy\else\copy\fi\@@definitiebox
   \ifdim\!!widthb=\zeropoint
     \kern.75em                     % another default
   \else
     \kern\!!widthb
   \fi
   \ignorespaces
   \@@dodefinitie{#1}}

%D A new key 'titeluitlijnen' in definitions.

\chardef\insidedefinition=0

\let\@@leftdefinitieskip \!!zeropoint
\let\@@rightdefinitieskip\!!zeropoint

\def\@@dostartdefinitie#1[#2]#3%
  {\getvalue{\??dd#1\c!voor}%
   \begingroup
   \doadaptleftskip{\getvalue{\??dd#1\c!marge}}%
   \showcomposition
   \!!widthb\getvalue{\??dd#1\c!afstand}\relax
   \ifdim\!!widthb=\zeropoint\relax
     \doifvalue{\??dd#1\c!breedte}\v!ruim{\!!widthb=1em}%
   \fi
   % temp hack, we need to avoid this kind of preprocessing
   \setbox\@@definitiebox\hbox % preroll
     {\forgetall
      \trialtypesettingtrue
      \mindermeldingen
      \def\\{\crcr}%
      \@@definitiewoord{#1}[#2]{#3}%
        {\begstrut\getvalue{\??dd#1\c!tekst}\ignorespaces#3\endstrut}}%
   % so far
   \assignwidth
     {\!!widtha}%
     {\getvalue{\??dd#1\c!breedte}}%
     {\doifelsevaluenothing{\??dd#1\c!monster}%
        {% preroll can move here (test first)
         \ifhbox\@@definitiebox\unhcopy\else\copy\fi \@@definitiebox}%
        {\@@definitiewoord{#1}[#2]{#3}%
           {\getvalue{\??dd#1\c!tekst}\getvalue{\??dd#1\c!monster}}}}
     {\!!widthb}%
   \setbox\@@definitiebox\hbox
     {\forgetall
      \mindermeldingen
      \let\\\endgraf
      \doifelsevalue{\??dd#1\c!plaats}\v!aanelkaar
        {\@@definitiewoord{#1}[#2]{#3}%
           {\begstrut\getvalue{\??dd#1\c!tekst}#3\endstrut}}
        {\@@definitiewoord{#1}[#2]{#3}%
           {\vtop{\hsize\!!widtha\advance\hsize-\!!widthb
            \begstrut\getvalue{\??dd#1\c!tekst}\ignorespaces#3\endstrut}}}}%
   \doifelsevalue{\??dd#1\c!titeluitlijnen}\v!nee
     {\edef\@@leftdefinitieskip {\the\leftskip }%
      \edef\@@rightdefinitieskip{\the\rightskip}}
     {\ifcase\insidedefinition
        \edef\@@leftdefinitieskip {\the\leftskip }%
        \edef\@@rightdefinitieskip{\the\rightskip}%
      \fi}%
   \expanded{\inspringen[\getvalue{\??dd#1\c!inspringen}]}%
   \ifcase\insidedefinition % better a system mode
     \chardef\insidedefinition\plusone
   \or
     \chardef\insidedefinition\plustwo
   \fi} % now happens elsewhere : \noindent\ignorespaces

\def\@@stopdefinitie#1%
  {\par % maybe better after \dostopattributes
   \dostopattributes
   \endgroup
   \getvalue{\??dd#1\c!na}%
   \egroup % temporary hack
   \dochecknextindentation{\??dd#1}}

\def\@@dodefinitie#1%
  {\dostartattributes{\??dd#1}\c!letter\c!kleur\empty
   \ignorespaces}

% \def\@@somedefinitie#1[#2]#3%
%   {\bgroup % temporary hack
%    \BeforePar{\executedoordefinitie{#1}[#2]{#3}}%
%    \AfterPar{\@@stopdefinitie{#1}}%
%    \GetPar}
%
% nicer and better:

\def\@@somedefinitie#1[#2]#3%
  {\dowithpar
     {\bgroup\executedoordefinitie{#1}[#2]{#3}}%
     {\@@stopdefinitie{#1}}}

\def\@@startsomedefinitie#1[#2]#3%
  {\bgroup % temporary hack
   \BeforePar{\executedoordefinitie{#1}[#2]{#3}}%
   \GotoPar}

\def\dodosteldoordefinierenin[#1][#2]%
  {\getparameters[\??dd#1][#2]}

\def\dosteldoordefinierenin[#1][#2]% % beter: \iffirstargument
  {\ConvertToConstant\doifelse{#2}{}
     {\dodosteldoordefinierenin[][#1]}
     {\dodoubleargumentwithset\dodosteldoordefinierenin[#1][#2]}}

\def\steldoordefinierenin%
  {\dodoubleempty\dosteldoordefinierenin}

\def\executedoordefinitie#1[#2]%
  {\ExpandAfter\doifundefined{@@definitie\getvalue{\??dd#1\c!plaats}}
     {\setvalue{\??dd#1\c!plaats}{\v!links}}%
   \getvalue{@@definitie\getvalue{\??dd#1\c!plaats}}{#1}[#2]}

\def\dodoordefinieren[#1][#2]%
  {\copyparameters[\??dd#1][\??dd]
     [\c!plaats,\c!kopletter,\c!letter,\c!kleur,\c!kopkleur,
      \c!breedte,\c!hang,\c!monster,\c!voor,\c!tussen,\c!na,\c!marge,
      \c!inspringen,\c!springvolgendein,\c!uitlijnen,
      \c!tekst,\c!afstand,\c!commando]%
   \getparameters[\??dd#1]
     [\s!do\c!status=\v!stop,
      \s!do\c!commando=\normal@@definitiewoord,
      #2]%
   \doifvalue{\??dd#1\c!plaats}\v!boven
     {\doassign[\??dd#1][\c!tussen={\blanko}]}%
   \setvalue{#1}%
     {\dodoubleempty\@@definitie[#1]}%
   \setvalue{\e!start#1}%
     {\dodoubleempty\@@startdefinitie[#1]}%
   \setvalue{\e!stop#1}%
     {\@@stopdefinitie{#1}}}%

\def\@@startdefinitie[#1][#2]%
  {\doifelsevalue{\??dd#1\s!do\c!status}\v!start
     {\@@startsomedefinitie{#1}[#2]{}}
     {\dowithwargument{\@@startsomedefinitie{#1}[#2]}}}

\def\@@definitie[#1][#2]%
  {\doifelsevalue{\??dd#1\s!do\c!status}\v!start
     {\@@somedefinitie{#1}[#2]{}}
     {\dowithwargument{\@@somedefinitie{#1}[#2]}}}

\def\doordefinieren%
  {\dodoubleemptywithset\dodoordefinieren}

\def\showdnpuretext#1%
  {\strut\getvalue{\??dd#1\c!tekst}} % geen spatie

% \def\showdntext#1%
%   {\doifelsevaluenothing{\??dd#1\c!tekst}
%      {\ignorespaces}
%      {\strut\getvalue{\??dd#1\c!tekst}\fixedspace}}

\def\showdntext#1%
  {\doifelsevaluenothing{\??dd#1\c!tekst}
     {\ignorespaces}
     {\strut
      \getvalue{\??dd#1\c!tekst}%
      \removeunwantedspaces\fixedspace}}

% \def\showdnnummer#1%
%   {\voorafgaandenummer
%    \nummer[\getvalue{\??dd#1\??dd\c!nummer}]}

\def\showdnnummer#1%
  {%\preparethenumber{\??dd#1}\voorafgaandenummer\preparednumber
   \preparednumber
   \nummer[\getvalue{\??dd#1\??dd\c!nummer}]}

\def\showdnsubnummer#1%
  {\showdnnummer{#1}%
   \getvalue{\??dd#1\c!scheider}%
   \nummer[\v!sub\getvalue{\??dd#1\??dd\c!nummer}]}

\def\showdnsubsubnummer#1%
  {\showdnsubnummer{#1}%
   \getvalue{\??dd#1\c!scheider}%
   \nummer[\v!sub\v!sub\getvalue{\??dd#1\??dd\c!nummer}]}

\def\showdnsubsubsubnummer#1%
  {\showdnsubsubnummer{#1}%
   \getvalue{\??dd#1\c!scheider}%
   \nummer[\v!sub\v!sub\v!sub\getvalue{\??dd#1\??dd\c!nummer}]}

\def\domakednnummer#1#2#3%
  {\getvalue{\??dd#2#1\c!links}%
   \strut#3{#1}%
   \getvalue{\??dd#2#1\c!afsluiter}%
   \getvalue{\??dd#2#1\c!rechts}}

% #1=name  #2=level #3=\show  #4[#5]#6#7=#1[#2]#3#4 van definitie

% \def\special@@definitiewoord#1#2#3#4[#5]#6#7%
%   {\strut
%    \doifelsevalue{\??dd#1\c!nummer}\v!nee
%      \!!doneafalse
%      {\doifelse{#5}{-}
%         \!!doneafalse
%         \!!doneatrue}%
%    \chardef\definitiekoppeling\zerocount
%    \iflocation
%      \doifvaluesomething{\??dd#1\c!koppeling}
%        {\processaction % genereert > of <
%           [\getvalue{\??dd#1\c!koppelwijze}]
%           [ \v!lokaal=>\chardef\definitiekoppeling1, % old: default
%            \v!globaal=>\chardef\definitiekoppeling2]}% new: global crosslinking
%    \fi
%    \stelnummerin % the number is called indirectly
%      [\getvalue{\??dd#1\??dd\c!nummer}]
%      [\c!sectienummer=\getvalue{\??dd#1\c!sectienummer}]%
%    \if!!donea
%      \getvalue{\e!volgende#2#1}%
%      \iflocation
%        \bgroup
%        \letvalue{\??dd#1\c!sectienummer}\v!ja
%        \protectconversion
%        \maakvoorafgaandenummer[\getvalue{\??dd#1\??dd\c!nummer}]%
%        \preparethenumber{\??dd#1}\voorafgaandenummer\preparednumber
%        \ifcase\definitiekoppeling \or
%          \xdef\internaldoornummer{#3{#1}}%
%          \rawreference\s!num{#1:\internaldoornummer}{}%
%        \or
%          \xdef\internaldoornummer{\countervalue{\??dd\c!koppeling#1}}%
%          \rawreference\s!num{#1:\internaldoornummer}{}%
%        \fi
%        \egroup
%      \fi
%      \maakvoorafgaandenummer[\getvalue{\??dd#1\??dd\c!nummer}]%
%      \preparethenumber{\??dd#1}\voorafgaandenummer\preparednumber
%      \hbox
%        {\let\normalkap\relax % sorry, uppercase causes troubles
%         \doattributes        % \nocase primitive needed
%           {\??dd#2#1}\c!kopletter\c!kopkleur
%           {\getvalue{\??dd#1\c!commando}% hook for taco
%              {\showdntext{#2#1}%
%               \domakednnummer{#1}{#2}{#3}}}%
%         \iflocation\ifcase\definitiekoppeling \else
%           \edef\localconnection{\getvalue{\??dd#1\c!koppeling}:\internaldoornummer}%
%           \doifreferencefoundelse\localconnection
% %             {\in[\localconnection]}{}% genereert > of <
%         \fi\fi}%
%      \doifnot{#5}{-}{\rawreference\s!num{#5}{#3{#1}}}%
%    \else % Why was this strange expansion needed?
%      \hbox
%        {\edef\!!stringa{\showdnpuretext{#2#1}}% nog eens testen binnen \expanded
%         \expanded{\doattributes{\??dd#1}\noexpand\c!kopletter\noexpand\c!kopkleur
%           {\noexpand\getvalue{\??dd#1\c!commando}{\!!stringa}}}%
%         \doifnot{#5}{-}{\rawreference\s!num{#5}{}}}%
%    \fi}

\def\special@@definitiewoord#1#2#3#4[#5]#6#7%
  {\strut
   \doifelsevalue{\??dd#1\c!nummer}\v!nee
     \!!doneafalse{\doifelse{#5}{-}\!!doneafalse\!!doneatrue}%
   \chardef\definitiekoppeling\zerocount
   \iflocation
     \doifvaluesomething{\??dd#1\c!koppeling}
       {\processaction % genereert > of <
          [\getvalue{\??dd#1\c!koppelwijze}]
          [ \v!lokaal=>\chardef\definitiekoppeling1, % old: default
           \v!globaal=>\chardef\definitiekoppeling2]}% new: global crosslinking
   \fi
   \stelnummerin % the number is called indirectly
     [\getvalue{\??dd#1\??dd\c!nummer}]
     [\c!sectienummer=\getvalue{\??dd#1\c!sectienummer}]%
   \if!!donea
\iftrialtypesetting\startlocal\fi
     \getvalue{\e!volgende#2#1}% tricky but we need the preroll
\iftrialtypesetting\stoplocal\fi
     % \getvalue{\e!volgende#2#1}%
     \iflocation
       \bgroup
       \letvalue{\??dd#1\c!sectienummer}\v!ja
       \protectconversion
       \maakvoorafgaandenummer[\getvalue{\??dd#1\??dd\c!nummer}]%
       \preparethenumber{\??dd#1}\voorafgaandenummer\preparednumber
       \ifcase\definitiekoppeling \or
         \xdef\internaldoornummer{#3{#1}}%
         \rawreference\s!num{#1:\internaldoornummer}{}%
       \or
         \xdef\internaldoornummer{\countervalue{\??dd\c!koppeling#1}}%
         \rawreference\s!num{#1:\internaldoornummer}{}%
       \fi
       \egroup
     \fi
     \maakvoorafgaandenummer[\getvalue{\??dd#1\??dd\c!nummer}]%
     \preparethenumber{\??dd#1}\voorafgaandenummer\preparednumber
     \let\normalkap\relax % sorry, uppercase causes troubles
     \doattributes        % \nocase primitive needed
       {\??dd#1}\c!kopletter\c!kopkleur
       {\getvalue{\??dd#1\c!commando}% hook for taco
          {\showdntext{#2#1}%
           \domakednnummer{#1}{#2}{#3}}}%
     \iflocation\ifcase\definitiekoppeling \else
       \edef\localconnection{\getvalue{\??dd#1\c!koppeling}:\internaldoornummer}%
       \doifreferencefoundelse\localconnection
         {\in[\localconnection]}\donothing % genereert > of <
     \fi\fi
     \doifnot{#5}{-}{\rawreference\s!num{#5}{#3{#1}}}%
   \else % Why was this strange expansion needed?
     \edef\!!stringa{\showdnpuretext{#2#1}}% nog eens testen binnen \expanded
     \expanded{\doattributes{\??dd#1}\noexpand\c!kopletter\noexpand\c!kopkleur
       {\noexpand\getvalue{\??dd#1\c!commando}{\!!stringa}}}%
     \doifnot{#5}{-}{\rawreference\s!num{#5}{}}%
   \fi}

\def\@@ddsetsubsubsubnummer#1%
  {\edef\doornummer{\getvalue{\??dd#1\??dd\c!nummer}}%
   \setnumber[\v!sub\v!sub\v!sub\doornummer]}

\def\@@ddsetsubsubnummer#1%
  {\@@ddresetsubsubsubnummer{#1}%
   \setnumber[\v!sub\v!sub\doornummer]}

\def\@@ddsetsubnummer#1%
  {\@@ddresetsubsubnummer{#1}%
   \setnumber[\v!sub\doornummer]}

\def\@@ddsetnummer#1%
  {\@@ddresetsubnummer{#1}%
   \setnumber[\doornummer]}

\def\@@ddresetsubsubsubnummer#1%
  {\edef\doornummer{\getvalue{\??dd#1\??dd\c!nummer}}%
   \resetnumber[\v!sub\v!sub\v!sub\doornummer]}

\def\@@ddresetsubsubnummer#1%
  {\@@ddresetsubsubsubnummer{#1}%
   \resetnumber[\v!sub\v!sub\doornummer]}

\def\@@ddresetsubnummer#1%
  {\@@ddresetsubsubnummer{#1}%
   \resetnumber[\v!sub\doornummer]}

\def\@@ddresetnummer#1%
  {\@@ddresetsubnummer{#1}%
   \resetnumber[\doornummer]}

\def\@@ddvolgendesubsubsubnummer#1[#2]%
  {\edef\doornummer{\getvalue{\??dd#1\??dd\c!nummer}}%
   \verhoognummer[\v!sub\v!sub\v!sub\doornummer]%
   \rawreference\s!num{#2}{\showdnsubsubsubnummer{\doornummer}}}%

\def\@@ddvolgendesubsubnummer#1[#2]%
  {\@@ddresetsubsubsubnummer{#1}%
   \verhoognummer[\v!sub\v!sub\doornummer]%
   \rawreference\s!num{#2}{\showdnsubsubnummer{\doornummer}}}

\def\@@ddvolgendesubnummer#1[#2]%
  {\@@ddresetsubsubnummer{#1}%
   \verhoognummer[\v!sub\doornummer]%
   \rawreference\s!num{#2}{\showdnsubnummer{\doornummer}}}

\def\@@ddvolgendenummer#1[#2]%
  {\@@ddresetsubnummer{#1}%
   \verhoognummer[\doornummer]%
   \rawreference\s!num{#2}{\showdnnummer{\doornummer}}}

\def\dodosteldoornummerenin[#1][#2]%
  {\getparameters[\??dd#1][#2]%
   \doifdefined{\??dd#1\c!start}
     {\stelnummerin[#1][\c!start=\getvalue{\??dd#1\c!start}]}%
   \stelnummerin[#1][\c!conversie=\getvalue{\??dd#1\c!conversie}]}

\def\dosteldoornummerenin[#1][#2]%
  {\ConvertToConstant\doifelse{#2}{}
     {\getparameters[\??dn][#1]}
     {\dodoubleargumentwithset\dodosteldoornummerenin[#1][#2]}}

\def\steldoornummerenin%
  {\dodoubleempty\dosteldoornummerenin}

\def\dododoornummeren#1#2#3[#4][#5]#6%
  {\makecounter{\??dd\c!koppeling#1}% new: global cross linking
   \dodoordefinieren[#3#1]%
     [\s!do\c!status=\v!start,
      \s!do\c!commando=\special@@definitiewoord{#1}{#3}{#6}]%
   \copyparameters[\??dd#3#1][\??dn]
     [\c!plaats,\c!kopletter,\c!letter,\c!kleur,\c!kopkleur,
      \c!breedte,\c!nummer,\c!afstand,\c!commando,
      \c!monster,\c!hang,\c!uitlijnen,\c!voor,\c!tussen,\c!na,
      \c!niveaus,\c!wijze,\c!blokwijze,\c!scheider,\c!marge,
      \c!inspringen,\c!springvolgendein,\c!afsluiter,\c!sectienummer,
      \c!nummer]%
   \doifassignmentelse{#4}
     {\getparameters[\??dd#3#1]%
        [\c!tekst=#1,\??dd\c!nummer=#1,\c!conversie=,
         \c!links=,\c!rechts=,\c!koppeling=,\c!koppelwijze=\v!lokaal,#4]}%
     {\doifelsenothing{#4}
        {\getparameters[\??dd#3#1]%
           [\c!tekst=#1,\??dd\c!nummer=#1,\c!conversie=,
            \c!afsluiter=,
            \c!links=,\c!rechts=,\c!koppeling=,\c!koppelwijze=,#4]}%
        {\copyparameters[\??dd#3#1][\??dd#3#4]
           [\c!plaats,\c!kopletter,\c!letter,\c!kleur,\c!kopkleur,
            \c!breedte,\c!nummer,\c!afstand,\c!commando,\c!marge,
            \c!monster,\c!hang,\c!uitlijnen,\c!voor,\c!tussen,\c!na,
            \c!afsluiter,\c!inspringen,\c!springvolgendein,\c!links,\c!rechts,
            \c!koppeling,\c!koppelwijze]%
         \getparameters[\??dd#3#1]
           [\c!tekst=#1,\??dd\c!nummer=#4,\c!conversie=,#5]}}%
   \ExpandBothAfter\doif{\getvalue{\??dd#3#1\??dd\c!nummer}}{#1}
     {\definieernummer
        [#3#1]
        [\c!wijze=\getvalue{\??dd#1\c!wijze},
         \c!blokwijze=\getvalue{\??dd#1\c!blokwijze},
         \c!sectienummer=\getvalue{\??dd#1\c!sectienummer}]%
      \doifvalue{\??dd#1\c!niveaus}{#2}%                           % for
        {\doifsomething{\getvalue{\??dd#1\c!conversie}}%           % old
           {\stelnummerin[#3#1]                                    % times
              [\c!conversie=\getvalue{\??dd#1\c!conversie}]}}}%    % sake
   \setvalue{\s!set#3#1}%
     {\dosetdoornummer[#1][#3]}%
   \setvalue{\s!reset#3#1}%
     {\doresetdoornummer[#1][#3]}%
   \setvalue{\e!volgende#3#1}%
     {\dotripleempty\dovolgendedoornummer[#1][#3]}}

\def\dovolgendedoornummer[#1][#2]%
  {\pluscounter{\??dd\c!koppeling#1}% new: global crosslinking
   \getvalue{\??dd\c!volgende#2\c!nummer}{#1}}%

\def\doresetdoornummer[#1][#2]%
  {\getvalue{\??dd\s!reset#2\c!nummer}{#1}}%

\def\dosetdoornummer[#1][#2]%
  {\getvalue{\??dd\s!set#2\c!nummer}{#1}}%

\def\dodoornummeren[#1][#2][#3]%
  {\dododoornummeren{#1}{1}{}[#2][#3]\showdnnummer
   \dododoornummeren{#1}{2}{\v!sub}[#2][#3]\showdnsubnummer
   \dododoornummeren{#1}{3}{\v!sub\v!sub}[#2][#3]\showdnsubsubnummer
   \dododoornummeren{#1}{4}{\v!sub\v!sub\v!sub}[#2][#3]\showdnsubsubsubnummer}

\def\doornummeren%
  {\dotripleemptywithset\dodoornummeren}

%  Het default-mechanisme kan mooier: leegtest, enz.
%
%  Werkprocedure buiten definitie

\def\dodosteldoorspringenin[#1][#2]%
  {\getparameters[\??ds#1][#2]}

\def\dosteldoorspringenin[#1][#2]%
  {\ConvertToConstant\doifelse{#2}{}
     {\dodosteldoorspringenin[][#1]}
     {\dodoubleargumentwithset\dodosteldoorspringenin[#1][#2]}}

\def\steldoorspringenin
  {\dodoubleempty\dosteldoorspringenin}

\def\startdoorspringen
  {\witruimte
   \@@dsvoor
   \dosomebreak{\goodbreak}% \pagina[\v!voorkeur]
   \begingroup
   \parskip\zeropoint\relax}

\def\stopdoorspringen
  {\endgroup
   \@@dsna}

\def\dododoorspringen#1#2#3%
  {\par
   \getvalue{\??ds#1\c!voor}%
   \begingroup
   \doifvaluenothing{\??ds#1\c!monster}
     {\setvalue{\??ds#1\c!monster}%
        {\getvalue{\??ds#1\c!tekst}}}%
   \assignwidth
     {\!!widtha}
     {\getvalue{\??ds#1\c!breedte}}
     {\doattributes
        {\??ds#1}\c!kopletter\c!kopkleur
        {\getvalue{\??ds#1\c!monster}\getvalue{\??ds#1\c!scheider}}}
     {\getvalue{\??ds#1\c!afstand}}%
   \advance\!!widtha \getvalue{\??ds#1\c!afstand}%
   \setbox2\hbox to \!!widtha
     {\doattributes
        {\??ds#1}\c!kopletter\c!kopkleur
        {\strut
         \getvalue{\??ds#1\c!tekst}%
         \hss
         \getvalue{\??ds#1\c!scheider}%
         \hskip\getvalue{\??ds#1\c!afstand}}}%
   \parindent\zeropoint
   \hskip#2\!!widtha\indent\box2%
   \hangindent#3\!!widtha
   \doattributes{\??ds#1}\c!letter\c!kleur\empty
   \AfterPar% must be redone
     {\endgroup
      \getvalue{\??ds#1\c!na}}%
   \GetPar}

\def\dodoorspringen[#1][#2]%
  {\copyparameters[\??ds#1][\??ds]
      [\c!tekst,\c!scheider,\c!breedte,\c!letter,\c!kleur,
       \c!kopletter,\c!monster,\c!voor,\c!na,\c!afstand]%
   \getparameters[\??ds#1][#2]%
   \setvalue{#1}%
     {\dododoorspringen{#1}{0}{1}}%
   \setvalue{\v!sub#1}%
     {\dododoorspringen{#1}{1}{2}}%
   \setvalue{\v!sub\v!sub#1}%
     {\dododoorspringen{#1}{2}{3}}}

\def\doorspringen%
  {\dodoubleargumentwithset\dodoorspringen}

% \def\dodoorlabel[#1][#2]%
%   {\getvalue{\s!number#1\c!voor}%
%    \bgroup
%    \doifvalue{\s!number#1\c!plaats}{\v!marge}
%      {\setvalue{\s!number#1\c!plaats}{\v!inmarge}}%
%    \doattributes{\s!number#1}\c!kopletter\c!kopkleur
%      {\getvalue{\e!volgende#1}[#2]}%
%    \egroup
%    \getvalue{\s!number#1\c!na}}%
%
% \def\dovolgendedoorlabel[#1][#2]%
%   {\volgendenummer[#1][\s!lab][#2]}
%
% \def\dodoorlabelen[#1][#2]%
%   {\definieernummer
%      [#1][\c!voor=,\c!na=,\c!kopletter=,\c!wijze=\@@nrwijze,#2]%
%    \setvalue           {#1}{\dodoubleempty\dodoorlabel[#1]}%
%    \setvalue{\s!reset   #1}{\resetnumber[#1]}%
%    \setvalue{\e!verhoog #1}{\verhoognummer[#1]}%
%    \setvalue{\e!volgende#1}{\dodoubleempty\dovolgendedoorlabel[#1]}%
%    \setvalue{\c!huidige #1}{\huidigenummer[#1]}}
%
% \def\doorlabelen%
%   {\dodoubleargumentwithset\dodoorlabelen}

\def\dodoorlabel[#1][#2]%
  {\getvalue{\??lb#1\c!voor}%
   \getvalue{\??lb#1\c!commando}%
     {\doattributes{\??lb#1}\c!kopletter\c!kopkleur
        {\dotextprefix{\getvalue{\??lb#1\c!tekst}}%
         \getvalue{\e!volgende#1}[#2]}}%
   \getvalue{\??lb#1\c!na}}%

\def\dovolgendedoorlabel[#1][#2]%
  {\volgendenummer[#1][\s!lab][#2]}

\def\dodoorlabelen[#1][#2]%
  {\getparameters
     [\??lb#1]
     [\c!wijze=\@@nrwijze,\c!commando=,\c!plaats=,#2]%
   % downward compatible
   \processaction
     [\getvalue{\??lb#1\c!plaats}]
     [  \v!inmarge=>\setvalue{\??lb#1\c!commando}{\inmarge  },
       \v!inlinker=>\setvalue{\??lb#1\c!commando}{\inlinker },
      \v!inrechter=>\setvalue{\??lb#1\c!commando}{\inrechter},
          \v!marge=>\setvalue{\??lb#1\c!commando}{\inmarge  }]%
   % inefficient, we need to redesign this command
   \definieernummer
     [#1]
     [\c!wijze=\getvalue{\??lb#1\c!wijze}]%
   % generated commands
   \setvalue           {#1}{\dodoubleempty\dodoorlabel[#1]}%
   \setvalue{\s!reset   #1}{\resetnumber[#1]}%
   \setvalue{\e!verhoog #1}{\verhoognummer[#1]}%
   \setvalue{\e!volgende#1}{\dodoubleempty\dovolgendedoorlabel[#1]}%
   \setvalue{\c!huidige #1}{\huidigenummer[#1]}}

\def\doorlabelen
  {\dodoubleargumentwithset\dodoorlabelen}

\steldoordefinierenin
  [\c!plaats=\v!links,
   \c!kopletter=\v!vet,
   \c!letter=\v!normaal,
   \c!kleur=,
   \c!kopkleur=,
   \c!breedte=8em,
   \c!afstand=0pt,
   \c!hang=,
   \c!monster=,
   \c!uitlijnen=,
   \c!marge=\v!nee,
   \c!voor=\blanko,
   \c!tussen=\blanko,
   \c!na=\blanko,
   \c!springvolgendein=\v!ja,
   \c!inspringen=\v!nooit,
   \c!commando=]

\steldoornummerenin
  [\c!plaats=\v!boven,
   \c!kopletter=\v!vet,
   \c!kopkleur=,
   \c!letter=\v!normaal,
   \c!kleur=,
   \c!breedte=8em,
   \c!afstand=0pt,
   \c!hang=,
   \c!monster=,
   \c!uitlijnen=,
   \c!marge=\v!nee,
   \c!voor=\blanko,
   \c!tussen=\blanko,
   \c!na=\blanko,
   \c!springvolgendein=\v!ja,
   \c!inspringen=\v!nooit,
   \c!tekst=,
   \c!niveaus=3,                % to be upward compatible
   \c!conversie=,               % to be upward compatible
   \c!wijze=\v!per\v!tekst,
   \c!sectienummer=\v!ja,
   \c!scheider=.,
   \c!afsluiter=,
   \c!nummer=,
   \c!commando=]

\steldoorspringenin
  [\c!letter=\v!normaal,
   \c!kopletter=\v!normaal,
   \c!kleur=,
   \c!kopkleur=,
   \c!breedte=\v!passend,
   \c!tekst=\unknown,
   \c!monster=,
   \c!voor=\blanko,
   \c!na=\blanko,
   \c!afstand=1em,
   \c!scheider={ :}]

\protect \endinput
