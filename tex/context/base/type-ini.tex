% wat te doen met casual, evt `cs', danwel een manier om te
% mappen (zie showcase)

%D \module
%D   [       file=type-ini,
%D        version=2001.03.05,
%D          title=\CONTEXT\ Typescript Macros,
%D       subtitle=Initialization,
%D         author=Hans Hagen,
%D           date=\currentdate,
%D      copyright={PRAGMA / Hans Hagen \& Ton Otten}]
%C
%C This module is part of the \CONTEXT\ macro||package and is
%C therefore copyrighted by \PRAGMA. See mreadme.pdf for
%C details.

\writestatus{loading}{Context Typescript Macros (ini)}

\unprotect

\let\typescriptfiles\empty

\unexpanded\def\usetypescriptfile[#1]%
  {\addtocommalist{#1}\typescriptfiles}

\usetypescriptfile[\f!typeprefix syn] % font file synonyms
\usetypescriptfile[\f!typeprefix enc] % files and encodings
\usetypescriptfile[\f!typeprefix siz] % specific font sizes
\usetypescriptfile[\f!typeprefix map] % pdftex mapping
\usetypescriptfile[\f!typeprefix spe] % special macros
\usetypescriptfile[\f!typeprefix exa] % some examples
\usetypescriptfile[\f!typeprefix loc] % local scripts
\usetypescriptfile[\f!typeprefix akb] % adobe karl berry names

%usetypescriptfile[\f!typeprefix pre] % predefined scripts (compatible)
%usetypescriptfile[typeface]          % project scripts

\let\currenttypescripts\empty

\newif\iftypescriptfound

\let\@@typescriptone  \empty \let\typescriptone  \empty
\let\@@typescripttwo  \empty \let\typescripttwo  \empty
\let\@@typescriptthree\empty \let\typescriptthree\empty

\unexpanded\def\usetypescript
  {\dotripleempty\dousetypescript}

\def\dousetypescript[#1][#2][#3]%
  {\expanded{\dodousetypescript[#1][#2][#3]}}

\def\dodousetypescript[#1][#2][#3]% also loads type-loc, a user file
  {\pushmacro\@@typescriptone  \edef\@@typescriptone  {\truetypescript{#1}}%
   \pushmacro\@@typescripttwo  \edef\@@typescripttwo  {\truetypescript{#2}}%
   \pushmacro\@@typescriptthree\edef\@@typescriptthree{\truetypescript{#3}}%
   \pushmacro\typescriptone
   \pushmacro\typescripttwo
   \pushmacro\typescriptthree
   \pushmacro\stoptypescript
   \typescriptfoundfalse
   \writestatus\m!fonts
     {[\@@typescriptone] [\@@typescripttwo] [\@@typescriptthree]}%
   \processcommacommand[\typescriptfiles]\dododousetypescript
   \firsttypescriptpassfalse % testen
   \popmacro\stoptypescript
   \popmacro\typescriptthree
   \popmacro\typescripttwo
   \popmacro\typescriptone
   \popmacro\@@typescriptthree
   \popmacro\@@typescripttwo
   \popmacro\@@typescriptone}

\def\dododousetypescript#1%
  {\startreadingfile
   \pushmacro\currenttypefile
   \def\currenttypefile{#1}%
   \readfile\currenttypefile\donothing\donothing
   \popmacro\currenttypefile
   \stopreadingfile}

\def\usetypescriptonce
  {\dotripleempty\dousetypescriptonce}

\def\dousetypescriptonce[#1][#2][#3]%
  {\doifelseflagged{ts:#1:#2:#3}%
     {\writestatus\m!fonts{once (#1) (#2) (#3)}}
     {\setflag{ts:#1:#2:#3}%
      \expanded{\dodousetypescript[#1][#2][#3]}}}

% \definetypescriptsynonym[lbr][cmr]

\def\definetypescriptsynonym
  {\dodoubleempty\dodefinetypescriptsynonym}

\def\dodefinetypescriptsynonym[#1][#2]%
  {\ifsecondargument\setevalue{\??tm#1}{#2}\fi}

\beginTEX

\def\truetypescript#1%
  {\expandafter\ifx\csname\??tm#1\endcsname\relax
     #1%
   \else
     \@EA\truetypescript\csname\??tm#1\endcsname
   \fi}

\endTEX

\beginETEX \ifcsname

\def\truetypescript#1%
  {\ifcsname\??tm#1\endcsname
     \@EA\truetypescript\csname\??tm#1\endcsname
   \else
     #1%
   \fi}

\endETEX

% script [serif] [default]         [size]
% script [serif] [computer-modern] [size]
% script [serif] [computer-modern] [ec]
% script [serif] [computer-modern] [name]
% script [serif] [computer-modern] [special]

\newif\iffirsttypescriptpass \firsttypescriptpasstrue

\prependtoks\firsttypescriptpasstrue\to\everyjob

% \def\dochecktypescript#1#2#3% script use value
%   {\donefalse
%    \doifelsenothing{#1}\donetrue
%      {\doifelse{#2}{all}\donetrue
%         {\doifelse{#1}{all}\donetrue
%            {\fullexpandtwoargsafter\doifcommonelse{#1}{#2}\donetrue\donefalse
%             \ifdone\let#3\commalistelement\fi}}}}
%
% \def\starttypescript%
%   {\dotripleempty\dostarttypescript}
%
% \long\def\dostarttypescript[#1][#2][#3]#4\stoptypescript
%   {\iffirstargument
%      \let\typescriptone  \@@typescriptone
%      \let\typescripttwo  \@@typescripttwo
%      \let\typescriptthree\@@typescriptthree
%      \dochecktypescript{#1}\@@typescriptone  \typescriptone   \ifdone
%      \dochecktypescript{#2}\@@typescripttwo  \typescripttwo   \ifdone
%      \dochecktypescript{#3}\@@typescriptthree\typescriptthree \ifdone
%        %\writestatus
%        \debuggerinfo
%          {typescript}
%          {\currenttypefile: use=scr (val)
%             [\@@typescriptone  =#1 (\typescriptone)]
%             [\@@typescripttwo  =#2 (\typescripttwo)]
%             [\@@typescriptthree=#3 (\typescriptthree)]}%
%         #4\typescriptfoundtrue
%      \fi\fi\fi
%    \else\iffirsttypescriptpass
%      \pushmacro\fontclass
%      #4%
%      \popmacro\fontclass
%    \else
%      % skip this since it may do unwanted resets, like
%      % setting symbolic font names to unknown, especially
%      % in run time user type scripts
%    \fi\fi}

\def\typescript@@all{all}

\def\dochecktypescript#1#2#3% script use value
  {\donefalse
   \def\@@typescriptcheck{#1}%
   \ifx\@@typescriptcheck\empty
     \donetrue
   \else\ifx#2\typescript@@all
     \donetrue
   \else\ifx\@@typescriptcheck\typescript@@all
     \donetrue
   \else\fullexpandtwoargsafter\doifcommonelse{#1}{#2}\donetrue\donefalse\ifdone
     \let#3\commalistelement
   \fi\fi\fi\fi}

\def\starttypescript
  {\dotripleempty\dostarttypescript}

% \long\def\dostarttypescript[#1][#2][#3]% #4\stoptypescript
%   {\iffirstargument
%      \let\typescriptone  \@@typescriptone
%      \let\typescripttwo  \@@typescripttwo
%      \let\typescriptthree\@@typescriptthree
%      \dochecktypescript{#1}\@@typescriptone  \typescriptone
%      \ifdone
%        \dochecktypescript{#2}\@@typescripttwo  \typescripttwo
%        \ifdone
%          \dochecktypescript{#3}\@@typescriptthree\typescriptthree
%          \ifdone
%            %\debuggerinfo
%            %  {typescript}
%            %  {\currenttypefile: use=scr (val)
%            %     [\@@typescriptone  =#1 (\typescriptone)]
%            %     [\@@typescripttwo  =#2 (\typescripttwo)]
%            %     [\@@typescriptthree=#3 (\typescriptthree)]}%
%            \typescriptfoundtrue
%            \let\next\dostarttypescriptA
%          \else
%            \let\next\dostarttypescriptC
%          \fi
%        \else
%          \let\next\dostarttypescriptC
%        \fi
%      \else
%        \let\next\dostarttypescriptC
%      \fi
%    \else\iffirsttypescriptpass
%      \let\next\dostarttypescriptB
%    \else
%      % skip this since it may do unwanted resets, like
%      % setting symbolic font names to unknown, especially
%      % in run time user type scripts
%      \let\next\dostarttypescriptC
%    \fi\fi
%    \next}

\newif\iftracetypescripts

\long\def\dostarttypescript[#1][#2][#3]% #4\stoptypescript
  {\iftracetypescripts\writestatus\m!fonts{enter [#1] [#2] [#3]}\fi%
   \iffirstargument
     \iftracetypescripts\writestatus\m!fonts{check [\@@typescriptone] [\@@typescripttwo] [\@@typescriptthree]}\fi
     \let\typescriptone  \@@typescriptone
     \let\typescripttwo  \@@typescripttwo
     \let\typescriptthree\@@typescriptthree
     \dochecktypescript{#1}\@@typescriptone\typescriptone
     \ifdone
       \dochecktypescript{#2}\@@typescripttwo\typescripttwo
       \ifdone
         \dochecktypescript{#3}\@@typescriptthree\typescriptthree
         \ifdone
           \typescriptfoundtrue
           \let\next\dostarttypescriptA
         \else
           \let\next\dostarttypescriptC
         \fi
       \else
         \let\next\dostarttypescriptC
       \fi
     \else
       \let\next\dostarttypescriptC
     \fi
     \iftracetypescripts
       \writestatus\m!fonts{\ifdone match\else pass\fi ed}%
     \fi
   \else\iffirsttypescriptpass
     \iftracetypescripts\writestatus\m!fonts{honored}\fi
     \let\next\dostarttypescriptB
   \else
     % skip this since it may do unwanted resets, like
     % setting symbolic font names to unknown, especially
     % in run time user type scripts
     \iftracetypescripts\writestatus\m!fonts{ignored}\fi
     \let\next\dostarttypescriptC
   \fi\fi
   \next}

\def\dostarttypescriptA
  {\pushmacro\fontclass}

\def\dostarttypescriptB
  {\pushmacro\fontclass}

\long\def\dostarttypescriptC#1\stoptypescript
  {}

\def\stoptypescript
  {\popmacro\fontclass}

% status
%
% 1 loaded
% 2 reported
% 3 preloaded

% flags ipv \c!status, more flag values

% \def\preloadmapfile[#1]%
%   {\def\docommando##1%
%      {\doifinstringelse{.}{##1}
%         {\writestatus{pdftex}{compensate map file: ##1}%
%          \setxvalue{##1 \c!status}{3}%
%          \doglobal\removefromcommalist{##1}\allfontmapsfiles}
%         {\expanded{\docommando{##1.\f!fontmapextension}}}}%
%    \expanded{\processcommalist[#1]}\docommando}
%
% \def\loadmapfile[#1]% last add first
%   {\def\docommando##1%
%      {\doifinstringelse{.}{##1}
%         {\doglobal\pretocommalist{##1}\allfontmapsfiles}
%         {\expanded{\docommando{##1.\f!fontmapextension}}}}%
%    \expanded{\processcommalist[#1]}\docommando
%    \ifproductionrun \loadallfontmapfiles \fi}

\def\dopreloadmapfile#1%
  {\doifinstringelse{.}{#1}
     {\writestatus{pdftex}{compensate map file: #1}%
      \setxvalue{#1 \c!status}{3}%
      \doglobal\removefromcommalist{#1}\allfontmapsfiles}
     {\expanded{\dopreloadmapfile{#1.\f!fontmapextension}}}}

\def\preloadmapfile[#1]%
  {\expanded{\processcommalist[#1]}\dopreloadmapfile}

\def\doloadmapfile#1%
  {\doifinstringelse{.}{#1}
     {\doglobal\pretocommalist{#1}\allfontmapsfiles}
     {\expanded{\doloadmapfile{#1.\f!fontmapextension}}}}

\def\loadmapfile[#1]% last added first !
  {\expanded{\processcommalist[#1]}\doloadmapfile
   \ifproductionrun \loadallfontmapfiles \fi}

\def\doloadfontmapfile#1% will be special
  {\doifundefined{#1 \c!status}
     {\ifnum\realpageno>\plusone
        \writestatus{pdftex}{too late for map file: #1}%
      \else
        \writestatus{pdftex}{using map file: #1}%
      \fi
      \doiffileelse{#1}{\pdfmapfile{+#1}}\donothing
      \setxvalue{#1 \c!status}{1}}}

\def\doreportfontmapfile#1%
  {\doifundefined{#1 \c!status}
     {\writestatus{pdftex}{needs map file: #1}%
      \setxvalue{#1 \c!status}{2}}}

\def\loadallfontmapfiles
  {%\message{[\allfontmapsfiles]}%
   \ifx\allfontmapsfiles\empty \else
     \ifautoloadmapfiles
       \ifcase\pdfoutput \else \ifx\pdfmapfile\undefined \else
         \processcommacommand[\allfontmapsfiles]\doloadfontmapfile
         \forgetmapfiles
       \fi \fi
     \else
       \processcommacommand[\allfontmapsfiles]\doreportfontmapfile
       \forgetmapfiles
     \fi
   \fi}

\appendtoks \loadallfontmapfiles \to \everyjob
\appendtoks \loadallfontmapfiles \to \everystarttext
\appendtoks \loadallfontmapfiles \to \everybeforepagebody

\newif\ifautoloadmapfiles

\def\forgetmapfiles
  {\global\let\allfontmapsfiles\empty}

\forgetmapfiles

\def\disablemapfiles
  {\ifcase\pdfoutput \else \ifx\pdfmapfile\undefined \else
     \pdfmapfile{}% this will (kind of reset) the map file loading
   \fi \fi}

% \definetypeface [#1:joke] [#2:rm]
% \definetypeface [#1:joke] [#2:rm] [#3:...]
% \definetypeface [#1:joke] [#2:rm] [#3:serif] [#4:lucida] [#5:size] [#6:...]

\def\definetypeface
  {\dosixtupleargument\dodefinetypeface}

\def\tsvar#1#2%
  {\@EA\ifx\csname\??ts#1\endcsname\empty
     #2%
   \else
     \csname\??ts#1\endcsname
   \fi}

% #1=main #2=rm #3=serif #4=fontname #5=size #6=settings

\def\typefaceencoding{\defaultencoding}

\def\dodefinetypeface[#1][#2][#3][#4][#5][#6]%
  {\dododefinetypeface[#1][#2]
   \iffifthargument % sixth is optional
   % \getparameters[\??ts][rscale=1,\s!encoding=\defaultencoding,#6]
   % we need to expand since in #6 there can be a \typescripttwo
\expanded{\getparameters[\??ts][\s!rscale=1,\s!encoding=\defaultencoding,#6]}%
% toch niet \expanded{\getparameters[\??ts][\s!rscale=1,\s!encoding=\s!default,#6]}%
     \pushmacro\relativefontsize
     \pushmacro\typefaceencoding
     \pushmacro\fontclass
     \let\relativefontsize\@@tsrscale
     \let\typefaceencoding\@@tsencoding
     \setcurrentfontclass{#1}
     \saverelativefontsize{#2}{\relativefontsize}% fall back
    %\writestatus{typeface}{[#1] [#2] [#3] [#4]}
     \writestatus\m!fonts{[#1] [#2] [#3] [#4]}
     \usetypescript[map][\typefaceencoding]
     \usetypescript[#3][#4][name,default,\typefaceencoding,special]
     \usetypescript[#3][#5][size]
     \popmacro\fontclass
     \popmacro\typefaceencoding
     \popmacro\relativefontsize
   \else\iffourthargument
     \definetypeface[#1][#2][#3][#4][\s!default]%
   \else\ifthirdargument
     \getparameters[\??tf#1#2][#3]
   \fi\fi\fi}

\def\dododefinetypeface[#1][#2]% saveguard against redefinition
  {\doifundefined{\??tf#1\s!default}{\setgvalue{\??tf#1\s!default}{#2}}%
   \doifundefined{#1}{\unexpanded\setgvalue{#1}{\switchtotypeface[#1][#2]}}}

\def\setuptypeface% [class] [settings]
  {\doquadrupleempty\doswitchtotypeface[\setupbodyfont][\fontclass]}

\unexpanded\def\switchtotypeface% [class] [settings]
  {\doquadrupleempty\doswitchtotypeface[\switchtobodyfont][\globalfontclass]}

\def\doswitchtotypeface[#1][#2][#3][#4]%
  {%\doifinsetelse{\s!default,\v!reset}{#3}
   %  {\setcurrentfontclass\empty}
   %  {\setcurrentfontclass{#3}}%
   \setcurrentfontclass{#3}%
   \let\globalfontclass#2%
   \iffourthargument
     #1[#4]%
   \else\ifx\fontclass\empty
     #1[\c!rm]%
   \else
     \doifdefinedelse{\??tf\fontclass\s!default}
       {#1[\getvalue{\??tf\fontclass\s!default}]}
       {#1[\c!rm]}%
   \fi \fi
   \ifmmode\mr\else\tf\fi} % needed ?

\def\usetypefile[#1]% recurses on path !
  {\readfile{\f!typeprefix#1}{}{}}% \relax\relax}

%D For backward compatibility we reimplement the font file
%D loading macro.

\ifx\normaldoreadfontdefinitionfile\undefined
  \let\normaldoreadfontdefinitionfile\doreadfontdefinitionfile
\fi

\def\doreadfontdefinitionfile#1#2% #1 = set/switch state
  {\ifundefined{\??tf#2\c!default}%
     \pushmacro\fontclass
     \setcurrentfontclass\empty
     \pushmacro\@@typescriptone \edef\@@typescriptone {\truetypescript{#2}}
     \pushmacro\@@typescripttwo  \let\@@typescripttwo  \empty
     \pushmacro\@@typescriptthree\let\@@typescriptthree\empty
     \typescriptfoundfalse
     \dododousetypescript{\f!typeprefix pre}
     \popmacro\@@typescriptthree
     \popmacro\@@typescripttwo
     \popmacro\@@typescriptone
     \iftypescriptfound \else
       \normaldoreadfontdefinitionfile{#1}{#2}
     \fi
     \setcurrentfontclass\empty
     \popmacro\fontclass
   \else\ifcase#1\relax
     \switchtotypeface[#2]%
   \else
     \setuptypeface[#2]%
   \fi\fi}

\fetchruntimecommand \typetypescript {\f!typeprefix\s!run}

% \usetypescript [berry] [ec]

\protect \endinput