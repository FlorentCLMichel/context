% wat te doen met casual, evt `cs', danwel een manier om te 
% mappen (zie showcase) 

%D \module
%D   [       file=type-ini,
%D        version=2001.03.05,
%D          title=\CONTEXT\ Typescript Macros,
%D       subtitle=Initialization,
%D         author=Hans Hagen,
%D           date=\currentdate,
%D      copyright={PRAGMA / Hans Hagen \& Ton Otten}]
%C
%C This module is part of the \CONTEXT\ macro||package and is
%C therefore copyrighted by \PRAGMA. See mreadme.pdf for
%C details.

\writestatus{loading}{Context Typescript Macros (ini)}

\unprotect

\let\typescriptfiles\empty

\def\usetypescriptfile[#1]%
  {\addtocommalist{#1}\typescriptfiles}

\usetypescriptfile[\f!typeprefix syn] % font file synonyms  
\usetypescriptfile[\f!typeprefix enc] % files and encodings 
\usetypescriptfile[\f!typeprefix siz] % specific font sizes 
\usetypescriptfile[\f!typeprefix map] % pdftex mapping 
\usetypescriptfile[\f!typeprefix spe] % special macros 
\usetypescriptfile[\f!typeprefix exa] % some examples 
\usetypescriptfile[\f!typeprefix loc] % local scripts 
%                 [\f!typeprefix pre] % predefined scripts (compatible)

% \usetypescriptfile[typeface] % project scripts 

\let\currenttypescripts\empty

\newif\iftypescriptfound

\def\usetypescript%
  {\dotripleempty\dousetypescript}

\let\typescriptone  \empty
\let\typescripttwo  \empty
\let\typescriptthree\empty

\def\dousetypescript[#1][#2][#3]% also loads type-loc, a user file
  {\pushmacro\typescriptone  \edef\typescriptone  {\truetypescript{#1}}%
   \pushmacro\typescripttwo  \edef\typescripttwo  {\truetypescript{#2}}%
   \pushmacro\typescriptthree\edef\typescriptthree{\truetypescript{#3}}%
   \typescriptfoundfalse
   \writestatus
     {typescript}
     {[\typescriptone] [\typescripttwo] [\typescriptthree]}%
   \processcommacommand[\typescriptfiles]\dodousetypescript
   \firsttypescriptpassfalse % testen 
   \popmacro\typescriptthree
   \popmacro\typescripttwo
   \popmacro\typescriptone}

\def\dodousetypescript#1%
  {\startreadingfile
   \pushmacro\currenttypefile
   \def\currenttypefile{#1}%
   \readfile{\currenttypefile}{}{}% \relax\relax
   \popmacro\currenttypefile
   \stopreadingfile}

% \definetypescriptsynonym[lbr][cmr]

\def\definetypescriptsynonym%
  {\dodoubleempty\dodefinetypescriptsynonym}

\def\dodefinetypescriptsynonym[#1][#2]%
  {\ifsecondargument\setevalue{\??tm#1}{#2}\fi}

\beginTEX

\def\truetypescript#1%
  {\expandafter\ifx\csname\??tm#1\endcsname\relax
     #1%
   \else
     \@EA\truetypescript\csname\??tm#1\endcsname
   \fi}

\endTEX

\beginETEX \ifcsname

\def\truetypescript#1%
  {\ifcsname\??tm#1\endcsname
     \@EA\truetypescript\csname\??tm#1\endcsname
   \else
     #1%
   \fi}

\endETEX

% script [serif] [default]         [size]            
% script [serif] [computer-modern] [size]            
% script [serif] [computer-modern] [ec]
% script [serif] [computer-modern] [name]
% script [serif] [computer-modern] [special]

\def\dochecktypescript#1#2% script use 
  {\donefalse
   \doifelsenothing{#1}\donetrue
     {\doifelse{#2}{all}\donetrue
        {\doifelse{#1}{all}\donetrue
           {\ExpandBothAfter\doifcommonelse{#1}{#2}\donetrue\donefalse}}}}

\def\starttypescript%
  {\dotripleempty\dostarttypescript}

\newif\iffirsttypescriptpass \firsttypescriptpasstrue 

\prependtoks\firsttypescriptpasstrue\to\everyjob 

\long\def\dostarttypescript[#1][#2][#3]#4\stoptypescript
  {\iffirstargument
     \dochecktypescript{#1}{\typescriptone  }\ifdone
     \dochecktypescript{#2}{\typescripttwo  }\ifdone
     \dochecktypescript{#3}{\typescriptthree}\ifdone
       %\writestatus 
       \debuggerinfo
         {typescript}
         {\currenttypefile: pat=scr 
            [\typescriptone  =#1] 
            [\typescripttwo  =#2] 
            [\typescriptthree=#3]}%
        #4\typescriptfoundtrue
     \fi\fi\fi
   \else\iffirsttypescriptpass
     \pushmacro\fontclass
     #4%
     \popmacro\fontclass
   \else
     % skip this since it may do unwanted resets, like 
     % setting symbolic font names to unknown, especially 
     % in run time user type scripts 
   \fi\fi}

\def\loadmapfile[#1]%
  {\processcommalist[#1]\doloadmapfile}

\def\doloadmapfile#1% will be special 
  {\ifcase\pdfoutput\else\ifx\pdfmapfile\undefined\else
     \doifundefined{map+#1}
       {\global\letvalue{map+#1}\empty\pdfmapfile{+#1}}%
   \fi\fi}

% \definetypeface [#1:joke] [#2:rm] 
% \definetypeface [#1:joke] [#2:rm] [#3:...]
% \definetypeface [#1:joke] [#2:rm] [#3:serif] [#4:lucida] [#5:size] [#6:...]

\def\definetypeface%
  {\dosixtupleargument\dodefinetypeface}

\def\tsvar#1#2%
  {\@EA\ifx\csname\??ts#1\endcsname\empty
     #2%
   \else
     \csname\??ts#1\endcsname
   \fi}

\let\@@tslabel    \empty
\let\@@tsstyle    \empty
\let\@@tsfont     \empty
\let\@@tssize     \empty
\let\@@tsencoding \empty

\def\dodefinetypeface[#1][#2][#3][#4][#5][#6]% 
  {\dododefinetypeface[#1][#2]
   \iffifthargument % sixth is optional 
     \getparameters[\??ts][rscale=1,\s!encoding=\s!default,#6]
     \let\relativefontsize\@@tsrscale
     \let\savedfontclass\fontclass
     \setcurrentfontclass{#1}
     \def\@@tslabel{#1}
     \def\@@tsstyle{#2}
     \def\@@tsfont {#3}
     \def\@@tssize {#4}
     \writestatus
       {typeface}
       {[#1] [#2] [#3] [#4]}%
     \expanded{\usetypescript[#3][#4][name,default,\@@tsencoding,special]} 
     \expanded{\usetypescript[#3][#5][size]}          
     \let\@@tslabel\empty
     \let\@@tsstyle\empty
     \let\@@tsfont \empty
     \let\@@tssize \empty
     \setcurrentfontclass\savedfontclass           
     \def\relativefontsize{1}
   \else\ifthirdargument
     \getparameters[\??tf#1#2][#3]
   \fi\fi}

\def\dododefinetypeface[#1][#2]% saveguard against redefinition
  {\doifundefined{\??tf#1\s!default}{\setgvalue{\??tf#1\s!default}{#2}}%
   \doifundefined{#1}{\setgvalue{#1}{\switchtotypeface[#1][#2]}}}

\def\setuptypeface% [class] [settings]
  {\doquadrupleempty\doswitchtotypeface[\setupbodyfont][\fontclass]}

\def\switchtotypeface% [class] [settings]
  {\doquadrupleempty\doswitchtotypeface[\switchtobodyfont][\globalfontclass]}

\def\doswitchtotypeface[#1][#2][#3][#4]% 
  {%\doifinsetelse{\s!default,\v!reset}{#3}
   %  {\setcurrentfontclass\empty}
   %  {\setcurrentfontclass{#3}}%
   \setcurrentfontclass{#3}%
   \let\globalfontclass#2%
   \iffourthargument
     #1[#4]%
   \else
     \ifx\fontclass\empty
       #1[\c!rm]%
     \else
       \doifdefinedelse{\??tf\fontclass\s!default}
         {#1[\getvalue{\??tf\fontclass\s!default}]}
         {#1[\c!rm]}%
     \fi
   \fi
   \tf}

\def\usetypefile[#1]%
  {\readfile{\f!typeprefix#1}{}{}}% \relax\relax}

%D For backward compatibility we reimplement the font file 
%D loading macro.  

\ifx\normaldoreadfontdefinitionfile\undefined
  \let\normaldoreadfontdefinitionfile\doreadfontdefinitionfile
\fi

\def\doreadfontdefinitionfile#1#2% #1 = set/switch state
  {\ifundefined{\??tf#2\c!default}%
     \setcurrentfontclass\empty
     \pushmacro\typescriptone \edef\typescriptone {\truetypescript{#2}}
     \pushmacro\typescripttwo  \let\typescripttwo  \empty
     \pushmacro\typescriptthree\let\typescriptthree\empty
     \typescriptfoundfalse
     \dodousetypescript{\f!typeprefix pre}
     \popmacro\typescriptthree
     \popmacro\typescripttwo
     \popmacro\typescriptone
     \iftypescriptfound \else
       \normaldoreadfontdefinitionfile{#1}{#2}
     \fi
     \setcurrentfontclass\empty
   \else\ifcase#1\relax
     \switchtotypeface[#2]%
   \else
     \setuptypeface[#2]%
   \fi\fi}

\fetchruntimecommand \typetypescript {\f!typeprefix\s!run}

\setupbodyfont[fil] % default filenames 

\protect \endinput
