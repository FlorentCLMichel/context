%D \module
%D   [       file=type-ini,
%D        version=2001.03.05,
%D          title=\CONTEXT\ Typescript Macros,
%D       subtitle=Initialization,
%D         author=Hans Hagen,
%D           date=\currentdate,
%D      copyright={PRAGMA / Hans Hagen \& Ton Otten}]
%C
%C This module is part of the \CONTEXT\ macro||package and is
%C therefore copyrighted by \PRAGMA. See mreadme.pdf for
%C details.

\writestatus{loading}{Context Typescript Macros (ini)}

%D The default fontclass is empty. We could demand always using fontclasses,
%D and then make the calling macros simplier (always assume fontclass) but
%D apart from downward compatibility issues, it would make global, class
%D spanning definitions a pain. Some day we will introduce a default class.

% torture test : proper typefaces (and namespace)
%
% \starttext
%   \usetypescript[modern][default]  default:  \setupbodyfont[modern] \eacute \eogonek \zcaron \acircumflexgrave
%   \usetypescript[modern][texnansi] texnansi: \setupbodyfont[modern] \eacute \eogonek \zcaron \acircumflexgrave
%   \usetypescript[modern][ec]       ec:       \setupbodyfont[modern] \eacute \eogonek \zcaron \acircumflexgrave
%   \usetypescript[modern][qx]       qx:       \setupbodyfont[modern] \eacute \eogonek \zcaron \acircumflexgrave
%   \usetypescript[modern][t5]       t5:       \setupbodyfont[modern] \eacute \eogonek \zcaron \acircumflexgrave
% \stoptext
%
% torture test : no typeface, just definitions
%
% \starttext
%   {\setupbodyfont[reset] \setupbodyfont[cmr] cmr: \eacute \eogonek \zcaron \acircumflexgrave \endgraf}
%   {\setupbodyfont[reset] \setupbodyfont[plr] plr: \eacute \eogonek \zcaron \acircumflexgrave \endgraf}
%   {\setupbodyfont[reset] \setupbodyfont[csr] csr: \eacute \eogonek \zcaron \acircumflexgrave \endgraf}
%   {\setupbodyfont[reset] \setupbodyfont[vnr] vnr: \eacute \eogonek \zcaron \acircumflexgrave \endgraf}
%   {\setupbodyfont[reset] \setupbodyfont[aer] aer: \eacute \eogonek \zcaron \acircumflexgrave \endgraf}
% \stoptext

% \usetypescript [modern]  [texnansi]
%
% normal   : 1450 ms
% exa quit : 1300 ms (150 ms)
% preload  :  825 ms (635 ms) (40-50%)
%
% \usetypescript [modern]  [texnansi]
% \usetypescript [palatino][texnansi]
% \usetypescript [times]   [texnansi]
%
% normal   : 3200 ms
% exa quit : 2700 ms ( 500 ms)
% preload  : 1300 ms (1900 ms) (60-70%)

% \usetypescript [modern][\defaultencoding]
% \usetypescript [map]   [latin-modern-os] [\defaultencoding]
% \setupbodyfont[modern]
% test 1234 test

\unprotect

\def\starttypescriptcollection
  {\dosingleempty\dostarttypescriptcollection}

\def\dostarttypescriptcollection[#1]%
  {}

\def\stoptypescriptcollection
  {}

\let\typescriptfiles\empty

\unexpanded\def\usetypescriptfile[#1]%
  {\doifelse{#1}\v!reset
     {\let\typescriptfiles\empty}
     {\splitfiletype{#1}%
      \addtocommalist\splitoffname\typescriptfiles}}

% \usetypescriptfile[\f!typeprefix exa] % some examples
% \usetypescriptfile[\f!typeprefix syn] % font file synonyms
% \usetypescriptfile[\f!typeprefix enc] % files and encodings
% %usetypescriptfile[\f!typeprefix dis] % funny name remappings, obsolete
% \usetypescriptfile[\f!typeprefix siz] % specific font sizes
% \usetypescriptfile[\f!typeprefix map] % pdftex mapping
% \usetypescriptfile[\f!typeprefix spe] % special macros
% \usetypescriptfile[\f!typeprefix akb] % adobe karl berry names
%
% \beginXETEX \font
%   \usetypescriptfile[\f!typeprefix xtx] % xetex definitions
% \endXETEX

\usetypescriptfile[\f!typeprefix tmf]
\usetypescriptfile[\f!typeprefix siz]

\beginOLDTEX \font
    \usetypescriptfile[\f!typeprefix one]
    %usetypescriptfile[\f!typeprefix gyr] % to be merged into one
    \usetypescriptfile[\f!typeprefix akb] % will go away when gyre is merged
\endOLDTEX

\beginNEWTEX \font
  \usetypescriptfile[\f!typeprefix otf]
\endNEWTEX

\beginXETEX \font
  \usetypescriptfile[\f!typeprefix xtx]
\endXETEX

\usetypescriptfile[\f!typeprefix loc]

% SO FAR

\let\currenttypescripts\empty

\newif\iftypescriptfound

\let\@@typescriptone  \empty \let\typescriptone  \empty
\let\@@typescripttwo  \empty \let\typescripttwo  \empty
\let\@@typescriptthree\empty \let\typescriptthree\empty

% method 2 is for Hans van der Meer

\newtoks\documenttypescripts

\chardef\typescriptmethod\plusone % 1: empty==all==true  2: empty==false
\chardef\typescriptstate \plustwo % 1: process 2: store

\unexpanded\def\usetypescript     {\chardef\typescriptmethod\plusone\dotripleempty\dousetypescript}
\unexpanded\def\usetypescriptexact{\chardef\typescriptmethod\plustwo\dotripleempty\dousetypescript}

\def\dousetypescript[#1][#2][#3]%
  {\expanded{\dodousetypescript[#1][#2][#3]}}

\def\dodousetypescript[#1][#2][#3]% also loads type-loc, a user file
  {\pushmacro\@@typescriptone  \edef\@@typescriptone  {\truetypescript{#1}}%
   \pushmacro\@@typescripttwo  \edef\@@typescripttwo  {\truetypescript{#2}}%
   \pushmacro\@@typescriptthree\edef\@@typescriptthree{\truetypescript{#3}}%
   \pushmacro\typescriptone
   \pushmacro\typescripttwo
   \pushmacro\typescriptthree
   \pushmacro\typescriptmethod
   \pushmacro\typescriptstate  \chardef\typescriptstate\plusone % process
   \pushmacro\stoptypescript
   \typescriptfoundfalse
   \iftracetypescripts\writestatus\m!fonts{request: [\@@typescriptone] [\@@typescripttwo] [\@@typescriptthree]}\fi
   \processcommacommand[\typescriptfiles]\dododousetypescript
   \the\documenttypescripts
   \firsttypescriptpassfalse % testen
   \popmacro\stoptypescript
   \popmacro\typescriptstate
   \popmacro\typescriptmethod
   \popmacro\typescriptthree
   \popmacro\typescripttwo
   \popmacro\typescriptone
   \popmacro\@@typescriptthree
   \popmacro\@@typescripttwo
   \popmacro\@@typescriptone}

% simple version:
%
% \def\dododousetypescript#1%
%   {\startreadingfile
%    \pushmacro\currenttypefile
%    \def\currenttypefile{#1}%
%    \def\currenttypefile{#1}%
%    \readfile\currenttypefile\donothing\donothing
%    \popmacro\currenttypefile
%    \stopreadingfile}
%
% tricky version:

\newconditional\preloadingtypescripts

\def\preloadtypescripts{\ifproductionrun\settrue\preloadingtypescripts\fi}

\def\dododousetypescript#1%
  {\setfalse\quittingtypescript
   \pushmacro\currenttypefile
   \def\currenttypefile{#1}%
   \ifconditional\preloadingtypescripts
     % load files once, and use saved data
     \def\loadedtypescripts{\csname\??ts:\c!file:#1\endcsname}%
     \@EAEAEA\ifx\loadedtypescripts\relax
       \@EAEAEA\newtoks\loadedtypescripts
       \bgroup
       \long\def\starttypescript##1\stoptypescript
         {\global\loadedtypescripts\@EA{\the\loadedtypescripts
            \starttypescript##1\stoptypescript}}%
       \long\def\starttypescriptcollection##1\stoptypescriptcollection
         {\global\loadedtypescripts\@EA{\the\loadedtypescripts
            \starttypescriptcollection##1\stoptypescriptcollection}}%
       \startreadingfile
       \pushendofline
       \unprotect
       \readfile\currenttypefile\donothing\donothing
       \protect
       \popendofline
       \stopreadingfile
       \egroup
     \fi
     %\message{[\space\currenttypefile}%
     \the\loadedtypescripts
     %\message{\ifconditional\quittingtypescript quit\space\fi]}%
   \else
     % process files each time
     \startreadingfile
     \pushendofline
     \unprotect
     \readfile\currenttypefile\donothing\donothing
     \protect
     \popendofline
     \stopreadingfile
   \fi
   \popmacro\currenttypefile
   \ifconditional\quittingtypescript
     \quitcommalist
     \setfalse\quittingtypescript
   \fi}

\def\usetypescriptonce
  {\dotripleempty\dousetypescriptonce}

\def\dousetypescriptonce[#1][#2][#3]%
  {\doifelseflagged{ts:#1:#2:#3}%
     {\writestatus\m!fonts{once (#1) (#2) (#3)}}
     {\setflag{ts:#1:#2:#3}%
      \expanded{\dodousetypescript[#1][#2][#3]}}}

% \definetypescriptsynonym[lbr][cmr]

\def\definetypescriptsynonym
  {\dodoubleempty\dodefinetypescriptsynonym}

\def\dodefinetypescriptsynonym[#1][#2]%
  {\ifsecondargument\setevalue{\??tm#1}{#2}\fi}

\def\truetypescript#1%
  {\ifcsname\??tm#1\endcsname
     \@EA\truetypescript\csname\??tm#1\endcsname\else#1%
   \fi}

% script [serif] [default]         [size]
% script [serif] [computer-modern] [size]
% script [serif] [computer-modern] [ec]
% script [serif] [computer-modern] [name]
% script [serif] [computer-modern] [special]

% todo, make firsttypescriptpass conditional

\newif\iffirsttypescriptpass \firsttypescriptpasstrue

\prependtoks\firsttypescriptpasstrue\to\everyjob

\def\typescript@@all{all}

\newif\iftracetypescripts

\def\starttypescript
  {\ifcase\typescriptstate
     % 0 = skip
     \@EA\gobbleuntil\@EA\stoptypescript
   \or
     % 1 = process
     \expandafter\dostarttypescript
   \or
     % 2 = store
     \expandafter\nostarttypescript
   \else
     % ? = skip
     \@EA\gobbleuntil\@EA\stoptypescript
   \fi}

\long\def\nostarttypescript#1\stoptypescript
  {\appendtoks\starttypescript#1\stoptypescript\to\documenttypescripts}

\def\dostarttypescript
  {\let\typescriptone  \@@typescriptone
   \let\typescripttwo  \@@typescripttwo
   \let\typescriptthree\@@typescriptthree
   \let\typescriptmatch\empty
   \doifnextcharelse[\dostarttypescriptone\dostarttypescriptall}

\long\def\dostarttypescriptall
  {\iffirsttypescriptpass
     \expandafter\doprocesstypescript
   \else
     % skip this since it may do unwanted resets, like
     % setting symbolic font names to unknown, especially
     % in run time user type scripts
     \expandafter\noprocesstypescript
   \fi}

\long\def\dostarttypescriptyes
  {\ifdone
     \typescriptfoundtrue
     \iftracetypescripts\writestatus\m!fonts{match:\ifx\currenttypefile\relax\space *\fi \typescriptmatch}\fi
     \expandafter\doprocesstypescript
   \else
     \expandafter\noprocesstypescript
   \fi}

\long\def\dostarttypescriptone
  {\dochecktypescript\@@typescriptone\typescriptone\redostarttypescriptone}

\long\def\dostarttypescripttwo
  {\dochecktypescript\@@typescripttwo\typescripttwo\redostarttypescripttwo}

\long\def\dostarttypescriptthree
  {\dochecktypescript\@@typescriptthree\typescriptthree\redostarttypescriptthree}

\long\def\redostarttypescriptone
  {\doifnextcharelse[\dostarttypescripttwo\dostarttypescriptyes}

\long\def\redostarttypescripttwo
  {\doifnextcharelse[\dostarttypescriptthree\dostarttypescriptyes}

\long\def\redostarttypescriptthree
  {\dostarttypescriptyes}

\def\doprocesstypescript
  {\pushmacro\fontclass}

\def\stoptypescript
  {\popmacro\fontclass}

\long\def\noprocesstypescript#1\stoptypescript
  {}

\let\typescriptmatch\empty

\def\dochecktypescript#1#2#3[#4]% script use value next
  {\donefalse
   \def\@@typescriptcheck{#4}%
   \ifx\@@typescriptcheck\empty % no longer needed / met
     \ifcase\typescriptmethod\or\donetrue\else\donefalse\fi
   \else\ifx#1\typescript@@all
     \donetrue
   \else\ifx\@@typescriptcheck\typescript@@all
     \donetrue
   \else\expanded{\doifcommonelse{\@@typescriptcheck}{#1}}\donetrue\donefalse\ifdone
     \let#2\commalistelement
   \fi\fi\fi\fi
   \ifdone
     \edef\typescriptmatch{\typescriptmatch\space[#4]}%
     \expandafter#3%
   \else
     \expandafter\noprocesstypescript
   \fi}

%D Yet another speed up: when issued inside typescript, the call
%D
%D \starttyping
%D \quittypescriptscanning
%D \stoptyping
%D
%D quits further loading. For an example, see type-exa:

\newconditional\quittingtypescript \setfalse\quittingtypescript

\def\quittypescriptscanning{\settrue\quittingtypescript}

% status
%
% 1 loaded
% 2 reported
% 3 preloaded

% flags ipv \c!state, more flag values

\def\dopreloadmapfile#1%
  {\splitfiletype{#1}%
   \writestatus\m!fonts{assuming map file: \splitoffname}%
   \setxvalue{\splitoffname \c!state}{3}%
   \doglobal\removefromcommalist\splitoffname\allfontmapsfiles}

\def\preloadmapfile[#1]{\expanded{\processcommalist[#1]}\dopreloadmapfile}
\def\loadmapfile   [#1]{\expanded{\processcommalist[#1]}\loadthemapfile}
\def\unloadmapfile [#1]{\expanded{\processcommalist[#1]}\unloadthemapfile}

%  too soon, no driver known, \ifproductionrun \loadallfontmapfiles \fi

% temp hack, will become just \addto

\ifx\pdftexversion\undefined

  \def\loadthemapfile#1%
    {\splitfiletype{#1}%
     \doglobal\addtocommalist\splitoffname\allfontmapsfiles}

\else\ifnum\pdftexversion<120 % no overloading

  \def\loadthemapfile#1%
    {\splitfiletype{#1}%
     \doglobal\pretocommalist\splitoffname\allfontmapsfiles}

\else

  \def\loadthemapfile#1%
    {\splitfiletype{#1}%
     \doglobal\addtocommalist\splitoffname\allfontmapsfiles}

\fi \fi

\def\unloadthemapfile#1%
  {\splitfiletype{#1}%
   \doglobal\removefromcommalist\splitoffname\allfontmapsfiles}

\let\usedmapfile\gobbletwoarguments
\let\usedmapline\gobbletwoarguments

\def\doregisterloadedmapfile#1%
  {\immediatewriteutilitycommand{\usedmapfile{=}{#1}}}

\beginLUATEX
    \let\doregisterloadedmapfile\gobbleoneargument
\endLUATEX

\def\doloadfontmapfile#1%
  {\ifundefined{#1\c!state}%
     \writestatus\m!fonts{using map file: #1}%
     \doregisterloadedmapfile{#1}%
     \doloadmapfile{=}{#1.\f!fontmapextension}% +/add =/replace -/remove
     \setxvalue{#1\c!state}{1}%
   \fi}

\def\doreportfontmapfile#1%
  {\ifundefined{#1\c!state}%
     \writestatus\m!fonts{needs map file: #1}%
     \setxvalue{#1\c!state}{2}%
    \fi}

\def\loadallfontmapfiles
  {%\message{[\allfontmapsfiles]}%
   \ifconditional\resettingmapfile
     \writestatus\m!fonts{resetting map file list}%
     \doresetmapfilelist
     \global\setfalse\resettingmapfile
   \fi
   \ifx\allfontmapsfiles\empty \else
     \ifautoloadmapfiles
       \processcommacommand[\allfontmapsfiles]\doloadfontmapfile
     \else
       \processcommacommand[\allfontmapsfiles]\doreportfontmapfile
     \fi
     \forgetmapfiles
   \fi}

\def\loadmapline
  {\dodoubleempty\dodoloadmapline}

\def\dodoloadmapline[#1][#2]%
  {\loadallfontmapfiles % ! ! !
   \ifsecondargument
     \immediatewriteutilitycommand{\usedmapline{#1}{#2}}%
     \doloadmapline{#1}{#2}% special
   \else
     \loadmapline[=][#1]%
     %\immediatewriteutilitycommand{\usedmapline{=}{#2}}%
     %\doloadmapline{=}{#1}% special
   \fi}

% since this is driver dependent, and since we may set map files
% before an output format is defined, we need to postpone it; we
% cannot use starttext as hook because an output switch can be part
% of a style; an alternative is to no longer permit driver switching
% after the first \starttext, but that will break compatibility
% because \startcomponent ... \environment ... is pretty legal.

% the map directives need to end up in the right place in the stream

% hm, the timing of when pdftex needs the map file info keeps changing;
% it's really time to move to map line support

% \appendtoks \loadallfontmapfiles \to \everyPDFximage
% \appendtoks \loadallfontmapfiles \to \everystarttext
% \appendtoks \loadallfontmapfiles \to \everybeforepagebody

\everybeforeshipout \expandafter
  {\expandafter\appendtoks
   \expandafter\loadallfontmapfiles
   \expandafter\to
   \expandafter\pageboundsettings
   \the\everybeforeshipout}

\newif\ifautoloadmapfiles

\let\allfontmapsfiles\empty

\def\forgetmapfiles
  {\globallet\allfontmapsfiles\empty}

\newconditional\resettingmapfile

\def\resetmapfiles
  {\global\settrue\resettingmapfile}

\def\disablemapfiles
  {\resetmapfiles
   \forgetmapfiles}

%D A handy shortcut:

% \definetypescriptprefix[serif][Serif]
% \definetypescriptprefix[sans] [Sans]
% \definetypescriptprefix[mono] [Mono]
%
%\starttypescript [serif,sans,mono] [handling,hanging,hz] [pure,normal,hz,quality]
%  \setupfontsynonym [\typescriptprefix\typescriptone] [handling=\typescriptthree]
% \stoptypescript

\def\definetypescriptprefix
  {\dodoubleargument\dodefinetypescriptprefix}

\def\dodefinetypescriptprefix[#1][#2]%
  {\setvalue{\??ts::#1}{#2}}

% without testing:
%
% \def\typescriptprefix#1%
%   {\csname\??ts::#1\endcsname}
%
% with testing:

\def\typescriptprefix#1%
  {\executeifdefined{\??ts::#1}{#1}}

% defining typefaces:
%
% \definetypeface [#1:joke] [#2:rm]
% \definetypeface [#1:joke] [#2:rm] [#3:...]
% \definetypeface [#1:joke] [#2:rm] [#3:serif] [#4:lucida] [#5:size] [#6:...]

\def\definetypeface
  {\dosixtupleargument\dodefinetypeface}

\def\tsvar#1#2% \executeifdefined{\??ts#1}{#2}
  {\@EA\ifx\csname\??ts#1\endcsname\empty
     #2%
   \else
     \csname\??ts#1\endcsname
   \fi}

% #1=main #2=rm #3=serif #4=fontname #5=size #6=settings

\def\typefaceencoding{\defaultencoding}

% we can use \rawgetparameters or \rawgeteparameters

\ifx\mkdodefinetypeface\undefined
    \let\mkdodefinetypeface\gobblefivearguments
\fi

\def\dodefinetypeface[#1][#2][#3][#4][#5][#6]%
  {\dododefinetypeface[#1][#2]%
   \iffifthargument % sixth is optional
     % we need to expand since in #6 there can be a \typescripttwo
     \expanded{\getparameters[\??ts][\s!rscale=\plusone,\s!encoding=\defaultencoding,\s!features=,\s!fallbacks=,\s!text=,#6]}% \geteparameters
     \pushmacro\relativefontsize
     \pushmacro\typefaceencoding
     \pushmacro\fontclass
     \let\relativefontsize\@@tsrscale
     \let\typefaceencoding\@@tsencoding
     \setcurrentfontclass{#1}%
     \letvalue{\fontclass\s!features }\@@tsfeatures  % new per 16/6/2007
     \letvalue{\fontclass\s!fallbacks}\@@tsfallbacks % new per 12/10/2008
     \saverelativefontsize{#2}\relativefontsize % fall back
     \savemathtextstyle\@@tstext % math text style (new per 28/4/2006)
     \iftracetypescripts\writestatus\m!fonts{define: [#1] [#2] [#3] [#4] / \typefaceencoding}\fi
     \usetypescript[#3,\t!map][#4][\t!name,\t!default,\typefaceencoding,\t!special]% map is needed for backward cmp
     \usetypescript[#3][#5][\t!size]%
     \popmacro\fontclass
     \popmacro\typefaceencoding
     \popmacro\relativefontsize
   \else\iffourthargument
     \definetypeface[#1][#2][#3][#4][\s!default]%
   \else\ifthirdargument
     \getparameters[\??tf#1#2][#3]%
   \fi\fi\fi}

% eventually we will split this module (no time now)

% hm, math needs encoding

% \beginLUATEX
%     \def\dodefinetypeface[#1][#2][#3][#4][#5][#6]%
%       {\dododefinetypeface[#1][#2]%
%        \iffifthargument % sixth is optional
%          % we need to expand since in #6 there can be a \typescripttwo
%          \expanded{\getparameters[\??ts][\s!rscale=\plusone,\s!features=,\s!fallbacks=,\s!text=,#6]}% \geteparameters
%          \pushmacro\relativefontsize
%          \pushmacro\fontclass
%          \let\relativefontsize\@@tsrscale
%          \setcurrentfontclass{#1}%
%          \letvalue{\fontclass\s!features }\@@tsfeatures % new per 16/6/2007
%          \letvalue{\fontclass\s!fallbacks}\@@tsfallbacks % new per 12/10/2008
%          \saverelativefontsize{#2}\relativefontsize % fall back
%          \savemathtextstyle\@@tstext % math text style (new per 28/4/2006)
%          \iftracetypescripts\writestatus\m!fonts{define: [#1] [#2] [#3] [#4]}\fi
%          \usetypescript[#3][#4][\t!name,\t!default,\t!special]%
%          \usetypescript[#3][#5][\t!size]%
%          \popmacro\fontclass
%          \popmacro\relativefontsize
%        \else\iffourthargument
%          \definetypeface[#1][#2][#3][#4][\s!default]%
%        \else\ifthirdargument
%          \getparameters[\??tf#1#2][#3]%
%        \fi\fi\fi}
% \endLUATEX

\def\dododefinetypeface[#1][#2]% saveguard against redefinition
  {\doifsomething{#1}
     {\doifundefined{\??tf#1\s!default}{\setgvalue{\??tf#1\s!default}{#2}}%
      \doifundefined{#1}{\unexpanded\setgvalue{#1}{\switchtotypeface[#1][#2]}}}}

\def\setuptypeface% [class] [settings]
  {\doquadrupleempty\doswitchtotypeface[\setupbodyfont][\fontclass]}

\unexpanded\def\switchtotypeface% [class] [settings]
  {\doquadrupleempty\doswitchtotypeface[\switchtobodyfont][\globalfontclass]}

\def\doswitchtotypeface[#1][#2][#3][#4]%
  {%\doifinsetelse{\s!default,\v!reset}{#3}
   %  {\setcurrentfontclass\empty}
   %  {\setcurrentfontclass{#3}}%
   \setcurrentfontclass{#3}%
   \let\globalfontclass#2%
   \iffourthargument
     #1[#4]%
   \else\ifx\fontclass\empty
     #1[\c!rm]%
   \else
     \doifdefinedelse{\??tf\fontclass\s!default}
       {#1[\getvalue{\??tf\fontclass\s!default}]}
       {#1[\c!rm]}%
   \fi \fi
   \ifmmode\mr\else\tf\fi} % needed ?

\def\usetypefile[#1]% recurses on path !
  {\readfile{\f!typeprefix#1}\donothing\donothing}

%D For Taco:
%D
%D \starttyping
%D \inherittypeface[palatino][rm][postscript]
%D \inherittypeface[palatino][rm][\fontclass]
%D \inherittypeface[palatino][rm] % == \fontclass
%D \inherittypeface[palatino] % == [rm,ss,tt,mm]
%D \stoptyping

\def\inherittypeface
  {\dotripleempty\doinherittypeface}

\def\doinherittypeface[#1][#2][#3]%
  {\doifelsenothing{#2}
     {\doinherittypeface[#1][\c!rm,\c!ss,\c!tt,\c!mm][\fontclass]}
     {\doifnot{#1}{#3}
        {\def\docommand##1{\setevalue{#1-##1}{#3}}%
         \processcommalist[#2]\docommand}}}

%D This hook sinto the font mechanism with:

\def\checkfontclass#1%
  {\edef\fontclass{\executeifdefined{\fontclass-#1}{\fontclass}}}

%D For backward compatibility we reimplement the font file
%D loading macro.

\ifx\normaldoreadfontdefinitionfile\undefined
  \let\normaldoreadfontdefinitionfile\doreadfontdefinitionfile
\fi

% old and obsolete
%
% \def\doreadfontdefinitionfile#1#2% #1 = set/switch state
%   {\ifundefined{\??tf#2\c!default}%
%      \pushmacro\fontclass
%      \setcurrentfontclass\empty
%      \pushmacro\@@typescriptone \edef\@@typescriptone {\truetypescript{#2}}%
%      \pushmacro\@@typescripttwo  \let\@@typescripttwo  \empty
%      \pushmacro\@@typescriptthree\let\@@typescriptthree\empty
%      \typescriptfoundfalse
%      \dododousetypescript{\f!typeprefix pre}%
%      \popmacro\@@typescriptthree
%      \popmacro\@@typescripttwo
%      \popmacro\@@typescriptone
%      \iftypescriptfound \else
%        \normaldoreadfontdefinitionfile{#1}{#2}%
%      \fi
%      \setcurrentfontclass\empty
%      \popmacro\fontclass
%    \else\ifcase#1\relax
%      \switchtotypeface[#2]%
%    \else
%      \setuptypeface[#2]%
%    \fi\fi}
%
% new and obeying fontclasses (but still obsolete)

\def\doreadfontdefinitionfile#1#2% #1 = set/switch state
  {\ifundefined{\??tf#2\c!default}%
     \pushmacro\starttypescript
     \scratchtoks\emptytoks
     % locate downward compatibility definitions, one argument !
     \long\def\starttypescript[##1]##2\stoptypescript
       {\doif{##1}{#2}{\scratchtoks{##2}}}
     \startreadingfile
     \readfile{\f!typeprefix pre}\donothing\donothing
     \stopreadingfile
     \popmacro\starttypescript
     \the\scratchtoks
   \else\ifcase#1\relax
     \switchtotypeface[#2]%
   \else
     \setuptypeface[#2]%
   \fi\fi}

\fetchruntimecommand \typetypescript {\f!typeprefix\s!run}

% \usetypescript [berry] [ec]

\protect \endinput
