%D \module
%D   [       file=core-not,
%D        version=1997.09.15,
%D          title=\CONTEXT\ Core Macros,
%D       subtitle=Footnote Handling,
%D         author=Hans Hagen,
%D           date=\currentdate,
%D      copyright={PRAGMA / Hans Hagen \& Ton Otten}]
%C
%C This module is part of the \CONTEXT\ macro||package and is
%C therefore copyrighted by \PRAGMA. Non||commercial use is
%C granted.

\writestatus{loading}{Context Core Macros / Footnote Handling}

%D There are some (still) dutch core commands used in this
%D file.

\unprotect

%D Footnotes are can be characterized by three components:
%D
%D \startopsomming[opelkaar]
%D \som a small number \voetnoot{a footnote number} or
%D symbol {\stelvoetnotenin[conversie=set 2]\voetnoot{a
%D footnote}}
%D \som and a similar mark at the bottom of the page
%D \som followed by some additional text
%D \stopopsomming
%D
%D Because footnotes are declared at the location of their
%D reference. Footnotes can be seen as a special kind of
%D floating bodies. There placement is postponed but has to be
%D taken into account in the pagebreak calculations. This kind
%D of calculations are forced by using \type{\insert}.

\ifx\footins\undefined \newinsert\footins \fi

%D \macros
%D   {setupfootnotes}
%D
%D We can influence footnote typesetting with the setup
%D command:
%D
%D \showsetup{\y!setupfootnotes}
%D
%D It's sort of a custom to precede footnotes by a horizontal
%D rule and although fancy rules like
%D
%D \starttypen
%D \hbox to 10em{\hskip-3em\dotfill}
%D \stoptypen
%D
%D Are quite ligitimate, we default to a simple one 20\% of the
%D text width.
%D
%D When \type{n} exceeds~1, footnotes are typeset in
%D multi||columns, using the algoritm presented on page~397
%D of \TEX book. Footnotes can be places on a per page basis
%D or whereever suitable. When we set~\type{n} to~0, we get a
%D rearanged paragraph, typeset by the algoritms on pages 398
%D and~389. We definitely did not reinvent that wheel.

\newif\ifendnotes \endnotesfalse

\def\dosetupfootnotes[#1]%
  {\getparameters[\??vn][#1]%
   \processaction
     [\@@vnlijn]
     [    \v!aan=>\def\footnoterule{\hrule \!!width .2\hsize},
          \v!uit=>\let\footnoterule=\relax,
      \s!unknown=>\let\footnoterule=\@@vnlijn]%
   \bgroup
   \setbox0=\vbox
     {\vskip\parskip % still to be tested in multi-column mode
      \@@vnvoor
      \footnoterule
      \@@vnna}%
   \global\skip\footins =\ht0
   \global\count\footins=1000
   \ifnum\@@vnn=0 \chardef\@@vnn=1 \fi % temporary (grouped)
   \global\divide\count\footins by \@@vnn
   \global\dimen\footins=\@@vnhoogte
   \global\multiply\dimen\footins by \@@vnn
   \egroup
   \ifnum\@@vnn=0
     \settextfootnotes
   \else
     \setcolumnfootnotes
   \fi
   \doifelse{\@@vnplaats}{\v!pagina}
     {\global\endnotesfalse}
     {\global\endnotestrue
      \postponefootnotes}}

\def\setupfootnotes%
  {\dosingleargument\dosetupfootnotes}

\def\setcolumnfootnotes%
  {\def\formatfootnumber##1{\indent\llap{##1\hskip\@@vnmargeafstand}}%
   \def\startpushfootnote  {\setrigidcolumnhsize\@@vnbreedte\@@vnkolomafstand\@@vnn}%
   \def\stoppushfootnote   {}%
   \def\startpopfootnotes  {\setbox0=\vbox\bgroup}%
   \def\stoppopfootnotes   {\egroup\rigidcolumnbalance0\@@vnn}}

\def\settextfootnotes%
  {\def\formatfootnumber##1{\noindent##1}%
   \def\startpushfootnote  {\startvboxtohbox}%
   \def\stoppushfootnote   {\hskip\@@vnkolomafstand % plus.5em minus.5em
                            \stopvboxtohbox}%
   \def\startpopfootnotes  {\vbox\bgroup\hsize=\@@vnbreedte}% \setvboxtohbox}%
   \def\stoppopfootnotes   {\convertvboxtohbox\egroup}}

%D The numbers that accompany a footnote are generated using
%D the standard \CONTEXT\ numbering mechanism, and thereby can
%D be assigned on a per whatever sectioning basis.

\definieernummer
  [\v!voetnoot]
  [\c!wijze=\@@vnwijze,
   \c!sectienummer=\@@vnwijze,
   \c!conversie=\@@vnconversie]

%D Both these parameters are coupled to the setup command we
%D will implement in a moment. This means that, given a
%D suitable symbol set, symbols can be used instead of numbers,
%D by saying:
%D
%D \starttypen
%D \setupfootnotes[conversion=set 2]
%D \stoptypen

%D \macros
%D   {footnote}
%D
%D A footnote can have a reference as optional argument and
%D therefore its formal specification looks like:
%D
%D \showsetup{\y!footnote}
%D
%D This command has one optional command: the reference. By
%D saying \type{[-]} the number is omitted. The footnote
%D command is not that sensitive to spacing, so it's quite
%D legal to say:
%D
%D \startbuffer
%D Users of \CONTEXT\ must keep both feet \footnote{Given they
%D have two.} on the ground and not get confused \footnote{Or
%D even crazy.} by all those obscure \footnote{But fortunately
%D readable.} parameters.
%D \stopbuffer
%D
%D \typebuffer
%D
%D When setting the \type{conversion} to \type{set 2} we get
%D something like:
%D
%D \bgroup
%D \startsmaller
%D \stelvoetnotenin[conversie=set 1]
%D \haalbuffer
%D \stopsmaller
%D \egroup
%D
%D Typesetting footnotes is, at least for the moment, disabled
%D when reshaping boxes.

\definecomplexorsimpleempty\footnote

\def\complexfootnote[#1]%
  {\unskip
   \ifvisible
     \ifreshapingbox
       \let\next=\gobbletwoarguments
     \else
       \let\next=\dofootnote
     \fi
   \else
     \let\next=\gobbletwoarguments
   \fi
   \next{#1}}

%D \macros
%D   {footnotesenabled}
%D
%D Before we come to typesetting a footnote, we first check
%D if we have to typeset a number. When a \type{-} is passed
%D instead of a reference, no number is typeset. We can
%D temporary disable footnotes by saying
%D
%D \starttypen
%D \footnotesenabledfalse
%D \stoptypen
%D
%D which can be handy while for instance typesetting tables
%D of contents.

\newif\iffootnotesenabled \footnotesenabledtrue

\def\dofootnote#1%
  {\iffootnotesenabled
     \doifelse{#1}{-}
       {\let\footnotenumber=\empty}
       {\verhoognummer[\v!voetnoot]%
        \maakhetnummer[\v!voetnoot]%
        \rawtextreference{\s!fnt}{#1}{\hetnummer}%
        \def\footnotenumber{\hoog{\hetnummer}}}%
     \let\next=\dostartfootnote
   \else
     \let\next=\gobbleoneargument
   \fi
   \next}

%D The main typesetting routine is more or less the same as the
%D \PLAIN\ \TEX\ one, except that we only handle one type while
%D \PLAIN\ also has something \type{\v...}. In most cases
%D footnotes can be handled by a straight insert, but we do so
%D by using an indirect call to the \type{\insert} primitive.

\let\localfootinsert=\insert

%D Making footnote numbers active is not always that logical,
%D especially when we keep the reference and text at one page.
%D On the other hand we need interactivity when we refer to
%D previous notes or use end notes. Therefore we support
%D interactive footnote numbers in two ways \voetnoot{This
%D feature was implemented years after we were able to do so,
%D mainly because endnotes had to be supported.} that is,
%D automatically (vise versa) and by user supplied reference.

\newcounter\internalfootreference

\let\startpushfootnote = \relax
\let\stoppushfootnote  = \relax

\newsignal\footnotesignal

\def\dostartfootnote% nog gobble als in pagebody
  {\bgroup
   \unskip\unskip
   \ifdim\lastkern=\footnotesignal
     \hoog{\kern\@@vnafstand}% gets the font right
   \fi
   \ignorespaces
   \doglobal\increment\internalfootreference
   \nobreak
   \iflocation
     \naarbox{\footnotenumber}[\s!fnt:t:\internalfootreference]%
     \rawreference{\s!fnt}{\s!fnt:f:\internalfootreference}{}%
   \else
     \footnotenumber
   \fi
   \localfootinsert\footins\bgroup
     \forgetall
     \setfootnotecorps
     \redoconvertfont % to undo \undo calls in in headings etc
     \startpushfootnote
     \dostartattributes\??vn\relax
     \interlinepenalty=\interfootnotelinepenalty
     \splittopskip\ht\strutbox  % not actually needed here
     \splitmaxdepth\dp\strutbox % not actually needed here
     \ifx\footnotenumber\empty \else
       \@@vnnummercommando
         {\iflocation
            \naarbox{\footnotenumber}[\s!fnt:f:\internalfootreference]%
          \else
            \footnotenumber
          \fi}%
     \fi
     \iflocation
       \rawreference{\s!fnt}{\s!fnt:t:\internalfootreference}{}%
     \fi
     \bgroup
     \aftergroup\dostopfootnote
     \begstrut
     \let\next}

\def\dostopfootnote%
  {\endstrut
   \dostopattributes
   \stoppushfootnote
   \egroup
   \egroup
   \kern\footnotesignal\relax} % \relax is needed to honor spaces

%D \macros
%D   {note}
%D
%D Refering to a note is accomplished by the rather short
%D command:
%D
%D \showsetup{\y!note}
%D
%D This command is implemented rather straightforward as:

\def\note[#1]%
  {\iffootnotesenabled
     \bgroup
     \unskip
     \gettextreference{#1}%
     \naarbox{\hoog{\currenttextreference}}[#1]%
     \egroup
   \fi}

%D Normally footnotes are saved as inserts that are called upon
%D as soon as the pagebody is constructed. The footnote
%D insertion routine looks just like the \PLAIN\ \TEX\ one,
%D except that we check for the end note state.

\let\startpopfootnotes = \relax
\let\stoppopfootnotes  = \relax

\def\placefootnoteinserts%
  {\ifvoid\footins \else
     \ifendnotes \else
       \vfill
       \witruimte % kan dit \vskip\parskip zijn? testen
       \@@vnvoor
       \footnoterule
       \bgroup
       \setfootnotecorps
       \startpopfootnotes
       \unvbox\footins
       \stoppopfootnotes
       \egroup
       \@@vnna
     \fi
   \fi}

%D Supporting end notes is surprisingly easy. Even better, we
%D can combine this feature with solving the common \TEX\
%D problem of disappearing inserts when they're called for in
%D deeply nested boxes. The general case looks like:
%D
%D \starttypen
%D \postponefootnotes
%D \.box{whatever we want with footnotes}
%D \flushfootnotes
%D \stoptypen
%D
%D This alternative can be used in headings, captions, tables
%D etc. The latter one sometimes calls for notes local to
%D the table, which can be realized by saying
%D
%D \starttypen
%D \setlocalfootnotes
%D some kind of table with local footnotes
%D \placelocalfootnotes
%D \stoptypen
%D
%D Postponing is accomplished by simply redefining the (local)
%D insert operation. A not too robust method uses the
%D \type{\insert} primitive when possible. This method fails in
%D situations where it's not entirely clear in what mode \TEX\
%D is. Therefore the auto method can is to be overruled when 
%D needed.  

\newbox\postponedfootnotes

\def\autopostponefootnotes%
  {\gdef\localfootinsert%
     {\ifinner
        %\message{[postponed footnote]}%
        \global\setbox\postponedfootnotes=\vbox\bgroup
          \unvbox\postponedfootnotes
          \let\next=\gobbletwoarguments
      \else
        %\message{[inserted footnote]}%
        \let\next=\insert
      \fi
      \next}}

\def\postponefootnotes%
  {\let\autopostponefootnotes=\postponefootnotes
   \gdef\localfootinsert%
     {%\message{[postponed footnote]}%
      \global\setbox\postponedfootnotes=\vbox\bgroup
        \unvbox\postponedfootnotes
        \gobbletwoarguments}}

\def\flushfootnotes%
  {\ifinpagebody \else \ifinner \else
     \ifendnotes \else \ifvoid\postponedfootnotes \else
      %\ifvmode % less interference, but also less secure
         \bgroup
           \scratchdimen=\pagegoal
           \advance\scratchdimen by -\pagetotal
           \ifdim\scratchdimen<\ht\postponedfootnotes
             \message{[moved footnote]}%
           \fi
         \egroup
         \global\let\localfootinsert=\insert
         \insert\footins\bgroup\unvbox\postponedfootnotes\egroup
       %\fi
     \fi\fi
   \fi\fi}

\def\placefootnotesintext#1%
  {\bgroup
   \@@vnvoor
   \setfootnotecorps
   \startpopfootnotes
   \unvbox#1\relax
   \stoppopfootnotes
   \@@vnna
   \egroup}

%D \macros
%D   {startlocalfootnotes,placelocalfootnotes}
%D
%D The next two macros can be used in for instance tables, as
%D we'll demonstrate later on.
%D
%D \showsetup{\y!startlocalfootnotes}
%D \showsetup{\y!placelocalfootnotes}

\newbox\localpostponedfootnotes

\def\collectlocalfootnotes%
  {\gdef\localfootinsert##1%
     {%\message{[local footnote]}%
      \global\setbox\localpostponedfootnotes=\vbox\bgroup
        \unvbox\localpostponedfootnotes
        \let\next}}

\def\dostartlocalfootnotes[#1]%
  {\let\autopostponefootnotes=\postponefootnotes
   \let\postponefootnotes=\collectlocalfootnotes
   \setupfootnotes[#1]%
   \savenumber[\v!voetnoot]%
   \resetnummer[\v!voetnoot]%
   \collectlocalfootnotes}

\def\startlocalfootnotes%
  {\bgroup % here because we support \vbox\startlocalfootnotes
   \dosingleempty\dostartlocalfootnotes}

\def\stoplocalfootnotes%
  {\restorenumber[\v!voetnoot]%
   \egroup}

\def\doplacelocalfootnotes[#1]%
  {\bgroup
   \setupfootnotes[#1]%
   \placefootnotesintext\localpostponedfootnotes
   \egroup}

\def\placelocalfootnotes%
  {\dosingleempty\doplacelocalfootnotes}

%D These commands can be used like:
%D
%D \startbuffer
%D \startlocalfootnotes[breedte=.3\hsize,n=0]
%D   \plaatstabel
%D     {Some Table}
%D     \plaatsonderelkaar
%D       {\starttabel[|l|r|]
%D        \HL
%D        \VL Nota\voetnoot{Bene} \VL Bene\voetnoot{Nota} \VL\SR
%D        \VL Bene\voetnoot{Nota} \VL Nota\voetnoot{Bene} \VL\SR
%D        \HL
%D        \stoptabel}
%D       {\placelocalfootnotes}
%D \stoplocalfootnotes
%D \stopbuffer
%D
%D \typebuffer
%D
%D Because this table placement macro expect box content, and
%D thanks to the grouping of the local footnotes, we don't need
%D additional braces.
%D
%D \haalbuffer

%D \macros
%D   {placefootnotes}
%D
%D We still have no decent command for placing footnotes
%D somewhere else than at the bottom of the page (for which no
%D user action is needed). Footnotes (endnotes) can be
%D placed by using
%D
%D \showsetup{\y!placefootnotes}

\def\doplacefootnotes[#1]%
  {\bgroup
   \setupfootnotes[#1]%
   \ifendnotes
     \ifinpagebody \else
       \placefootnotesintext\postponedfootnotes
     \fi
   \else
     \placefootnoteinserts
   \fi
   \egroup}

\def\placefootnotes%
  {\dosingleempty\doplacefootnotes}

%D Now how can this mechanims be hooked into \CONTEXT\ without
%D explictly postponing footnotes? The solution turned out to
%D be rather simple:
%D
%D \starttypen
%D \everypar  {...\flushfootnotes...}
%D \neverypar {...\postponefootnoes}
%D \stoptypen
%D
%D and
%D
%D \starttypen
%D \def\ejectinsert%
%D   {...
%D    \flushfootnotes
%D    ...}
%D \stoptypen
%D
%D We can use \type{\neverypar} because in most commands
%D sensitive to footnote gobbling we disable \type{\everypar}
%D in favor for \type{\neverypar}. In fact, this footnote
%D implementation is the first to use this scheme.

%D When typesetting footnotes, we have to return to the
%D footnote specific corps size, which is in most cases derived
%D from the global document corps size. In the previous macros
%D we already used a footnote specific font setting macro.

\def\setfootnotecorps%
  {\restoreglobalcorps
   \switchtocorps[\@@vnkorps]}

%D The footnote mechanism defaults to a traditional one
%D column way of showing them. By default we precede them by
%D a small line.

\setupfootnotes
  [\c!plaats=\v!pagina,
   \c!wijze=\v!per\v!deel,
   \c!sectienummer=\v!nee,
   \c!conversie=,
   \c!lijn=\v!aan,
   \c!voor=\blanko,
   \c!korps=\v!klein,
   \c!letter=,
   \c!na=,
   \c!margeafstand=.5em,
   \c!kolomafstand=1em,
   \c!afstand=.125em,
   \c!breedte=\zetbreedte,
   \c!hoogte=\teksthoogte,
   \c!nummercommando=\formatfootnumber,
   \c!n=1]

\protect

\endinput
