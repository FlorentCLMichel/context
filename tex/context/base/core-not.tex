%D \module
%D   [       file=core-not,
%D        version=2002.05.10,    % 1997.09.15
%D          title=\CONTEXT\ Core Macros,
%D       subtitle=Note Handling, % Footnote Handling
%D         author=Hans Hagen,
%D           date=\currentdate,
%D      copyright={PRAGMA / Hans Hagen \& Ton Otten}]
%C
%C This module is part of the \CONTEXT\ macro||package and is
%C therefore copyrighted by \PRAGMA. See mreadme.pdf for
%C details.

\writestatus{loading}{Context Core Macros / Note Handling}

%D Unfortunately we cannot force an even number of lines in
%D a two column footnote placement.

%D There are some (still) dutch core commands used in this
%D file.

\unprotect

% \dochecknote in processnotes

% splitskips setten

%D Footnotes are can be characterized by three components:
%D
%D \startitemize[packed]
%D \item a small number \footnote {a footnote number} or
%D      symbol {\setupfootnotes [conversion=set 2]\footnote
%D      {a footnote}}
%D \item and a similar mark at the bottom of the page
%D \item followed by some additional text
%D \stopitemize
%D
%D Because footnotes are declared at the location of their
%D reference. Footnotes can be seen as a special kind of
%D floating bodies. There placement is postponed but has to be
%D taken into account in the pagebreak calculations. This kind
%D of calculations are forced by using \type{\insert}.

%D \macros
%D   {setupnote,setupnotedefinition}
%D
%D We can influence footnote typesetting with the setup
%D command:
%D
%D \showsetup{setupfootnotes} % ! !
%D
%D It's sort of a custom to precede footnotes by a horizontal
%D rule and although fancy rules like
%D
%D \starttyping
%D \hbox to 10em{\hskip-3em\dotfill}
%D \stoptyping
%D
%D Are quite ligitimate, we default to a simple one 20\% of the
%D text width.
%D
%D When \type{n} exceeds~1, footnotes are typeset in
%D multi||columns, using the algoritm presented on page~397
%D of \TEX book. Footnotes can be places on a per page basis
%D or whereever suitable. When we set~\type{n} to~0, we get a
%D rearanged paragraph, typeset by the algoritms on pages 398
%D and~389. We definitely did not reinvent that wheel.

\newif\ifendnotes    \endnotesfalse
\newif\ifbottomnotes \bottomnotestrue

\chardef\clevernotes=\zerocount % 0=page 1=firstcolumn 2=lastcolumn

%D The next definitions indicate that we can frame the footnote
%D area. The footnotes themselves are treated as definitions.
%D
%D \showsetup{setupfootnotes}

\let\currentnote\v!footnote

\def\noteparameter     #1{\csname\??vn     \currentnote#1\endcsname}
\def\notedefparameter  #1{\csname\??vn\??vn\currentnote#1\endcsname}
\def\footnoteparameter #1{\csname\??vn      \v!footnote#1\endcsname}

\def\startnotedef        {\csname\e!start\??vn\??vn\currentnote\endcsname}
\def\stopnotedef         {\csname\e!stop \??vn\??vn\currentnote\endcsname}

\def\noteinsertion     #1{\csname\??vn:#1\endcsname}
\def\currentnoteins      {\csname\??vn:\currentnote\endcsname}
\def\currentsaveins      {\csname\??vn-\currentnote\endcsname}
\def\localpostponednotes {\csname\??vn+\currentnote\endcsname}

\def\backupnoteins     #1{\@EA\backupinsertion\csname\??vn:#1\endcsname}
\def\currentbackupnoteins{\@EA\backupinsertion\csname\??vn:\currentnote\endcsname}

%D The numbers that accompany a footnote are generated using
%D the standard \CONTEXT\ numbering mechanism, and thereby can
%D be assigned on a per whatever sectioning basis.

\ifx\noteinsertions\undefined \let\noteinsertions\empty \fi % permits reload

\def\doprocessnotes#1#2% #1 may be { ... }
  {\def\currentnote{#2}#1}

\def\doprocessnotescs#1#2% #1 == \cs that takes arg
  {\def\currentnote{#2}\@EA#1\csname\??vn:\currentnote\endcsname}

\def\processnotes  #1{\processcommacommand[\noteinsertions]{\doprocessnotes {#1}}}
\def\processnotescs#1{\processcommacommand[\noteinsertions]{\doprocessnotescs#1}}

\def\savenotecontent   {\processnotescs\saveinsertionbox    }
\def\erasenotebackup   {\processnotescs\eraseinsertionbackup}
\def\savenotedata      {\processnotescs\saveinsertiondata   }
\def\restorenotecontent{\processnotescs\restoreinsertionbox }
\def\restorenotedata   {\processnotescs\restoreinsertiondata}

%D ... due to invisibility of inserts ... maybe save them twice
%D and split new part ... todo ...

\def\doenablenotes % brrr
  {\global\count\currentnoteins\plusthousand
   \global\skip \currentnoteins1\baselineskip\relax}

\def\dodisablenotes
  {\global\count\currentnoteins\zerocount
   \global\skip \currentnoteins\zeropoint}

\def\enablenotes {\processnotes\doenablenotes }
\def\disablenotes{\processnotes\dodisablenotes}

\def\dosavenotes
  {\global\setbox\currentsaveins\vbox
     {\ifvoid\currentsaveins\else\unvbox\currentsaveins\fi
      \box\currentnoteins}}

\def\doflushsavednotes
  {\ifvoid\currentsaveins\else
     \insert\currentnoteins{\unvbox\currentsaveins}%
   \fi}

\def\savenotes      {\processnotes\dosavenotes }
\def\flushsavednotes{\processnotes\doflushsavednotes}

%D Both these parameters are coupled to the setup command we
%D will implement in a moment. This means that, given a
%D suitable symbol set, symbols can be used instead of numbers,
%D by saying:
%D
%D \starttyping
%D \setupfootnotes[conversion=set 2]
%D \stoptyping

% experiment: (compare scope=text and scope=page)
%
% \definenote[mynote][way=bytext,location=text,width=\leftmarginwidth,scope=page,rule=,before=,after=,factor=0]
% \setuptexttexts[margin][\vbox to \textheight{\placenotes[mynote]\vfill}][]

\def\definenote
  {\dodoubleempty\dodefinenote}

% maybe we should inherit (todo)

\def\dodefinenote[#1][#2]%
  {\def\currentnote{#1}%
   \ifundefined{\??vn:\currentnote}%
     \@EA\installinsertion      \csname\??vn:\currentnote\endcsname\relax
     \@EA\installbackupinsertion\csname\??vn:\currentnote\endcsname\relax
%    \@EA\newbox\csname\??vn::\currentnote\endcsname % scratch box % needed ?
     \@EA\newbox\csname\??vn+\currentnote\endcsname % local box
     \@EA\newbox\csname\??vn-\currentnote\endcsname % local box
     \doglobal\addtocommalist{#1}\noteinsertions
   \fi
   \definedescription
     [\??vn\??vn\currentnote]
     [\c!location=\v!inleft,
      \c!width=\v!fit,
      \c!headstyle=\noteparameter\c!style,
      \c!headcolor=\noteparameter\c!color,
      \c!before=,
      \c!after=]%
   \presetlocalframed
     [\??vn\currentnote]%
   \getparameters
     [\??vn\currentnote]
     [\c!location=\v!page,
      \c!way=\v!by\v!part,
      \c!sectionnumber=\v!no,
      \c!conversion=,
      \c!rule=\v!on,
      \c!before=\blank,
      \c!bodyfont=\v!small,
      \c!style=,
      \c!color=,
      \c!after=,
      \c!rulecolor=,
      \c!rulethickness=\linewidth,
      \c!frame=\v!off,
      \c!margindistance=.5em,
      \c!columndistance=1em,
      \c!distance=.125em,
      \c!align=\v!normal,
      \c!tolerance=\v!tolerant,
      \c!split=\v!tolerant,
     %\c!width=\makeupwidth,
     %\c!width=\ifdim\hsize<\makeupwidth\hsize\else\makeupwidth\fi,
      \c!width=\defaultnotewidth,
      \c!height=\textheight,
      \c!numbercommand=\high,
      \c!command=\noteparameter\c!numbercommand, % downward compatible
      \c!separator=\@@koseparator,
      \c!textcommand=\high,
      \c!textstyle=\tx,
      \c!textcolor=,
      \c!interaction=\v!yes,
      \c!factor=,
      \c!scope=, % \v!text \v!page
      \c!n=1]%
  \definenumber
    [\currentnote]
    [\c!way=\noteparameter\c!way,
     \c!sectionnumber=\noteparameter\c!way,
     \c!conversion=\noteparameter\c!conversion]%
  \letvalue{\??vn\c!rule:\currentnote}\normalnoterule
  \unexpanded\setvalue{\currentnote}{\setnote[#1]}%
  \unexpanded\setvalue{\currentnote\v!text}{\setnotetext[#1]}%
  \setupnote[\currentnote][#2]}

\def\setupnotedefinition[#1]%
  {\setupdescriptions[\??vn\??vn#1]}

\def\setupnote
  {\dodoubleempty\dosetupnote}

\def\dosetupnote[#1][#2]%
  {\edef\currentnote{#1}%
   \ifsecondargument
     \getparameters
       [\??vn\currentnote][#2]%
     \processaction
       [\noteparameter\c!rule]
       [     \v!on=>\letvalue{\??vn\c!rule:\currentnote}\normalnoterule,
            \v!off=>\letvalue{\??vn\c!rule:\currentnote}\relax,
        \s!default=>\letvalue{\??vn\c!rule:\currentnote}\relax,
        \s!unknown=>\setvalue{\??vn\c!rule:\currentnote}{\noteparameter\c!rule}]%
     \processaction % todo
       [\noteparameter\c!split]
       [  \v!tolerant=>\notepenalty\zeropoint,
            \v!strict=>\notepenalty9999,
        \v!verystrict=>\notepenalty\maxdimen,
           \s!default=>\notepenalty\zeropoint,
           \s!unknown=>\notepenalty\commalistelement]%
   \fi
   \dochecknote}

\def\dolocalsetupnotes#1#2%
  {\ifsecondargument
     \edef\noteinsertions{#1}%
     \processnotes{\setupnote[\currentnote][#2]}%
   \else\iffirstargument
     \doifassignmentelse{#1}
       {\processnotes{\setupnote[\currentnote][#1]}}
       {\edef\noteinsertions{#1}}%
   \fi\fi}

\def\dochecknote % only to be called locally, some bools will become class-ones
  {% for the moment no mixed text/endnotes modes, so we use
   % \footnoteparameter and not \noteparameter (**)
   \setnotedistance
   \count\currentnoteins\plusthousand
   \expanded{\doifcommonelse{\v!columns,\v!lastcolumn}{\noteparameter\c!location}}% **
       {\chardef\clevernotes\plustwo}
       {\expanded{\doifinsetelse{\v!firstcolumn}{\noteparameter\c!location}}% **
          {\chardef\clevernotes\plusone}%
          {\chardef\clevernotes\zerocount}}%
   \ifcase\clevernotes\relax
     % notes not in column areas
     \ifnum\noteparameter\c!n=\zerocount
       \settextnotes
       \scratchcounter\plusone
     \else
       \setcolumnnotes
       \scratchcounter\noteparameter\c!n\relax
       \divide\count\currentnoteins \scratchcounter
     \fi
     \global\endnotesfalse
     \expanded{\doifinsetelse{\v!page}{\noteparameter\c!location}}% **
       {\expanded{\doifinsetelse{\v!high}{\noteparameter\c!location}}% **
          {\global\bottomnotesfalse}
          {\global\bottomnotestrue}}
       {\global\endnotestrue
        \global\bottomnotestrue}% not: \postponenotes, else global
    \else
      % notes in column areas
      \ifnum\@@kln=\zerocount % brrr dependency on \??kl
        \scratchcounter\plusone
      \else
        \scratchcounter\footnoteparameter\c!n\relax % **
      \fi
      \global\endnotesfalse
      \global\bottomnotestrue
      \setclevernotes
   \fi
   \doifsomething{\noteparameter\c!factor}
     {\ifnum\noteparameter\c!factor<\zerocount\else
        \count\currentnoteins\noteparameter\c!factor
      \fi}%
   \ifnotelimit
     \dimen\currentnoteins\noteparameter\c!height
     \multiply\dimen\currentnoteins \scratchcounter
   \fi
   \ifendnotes
     \dimen\currentnoteins\maxdimen
     \count\currentnoteins\zerocount
     \skip \currentnoteins\zeropoint
   \fi}

\def\checknotes
  {\processnotes\dochecknote}

% Example of using factor:
%
% \definenote[mynote][way=bypage,location=text,width=\marginwidth,rule=,before=,factor=0]
% \setuplayout[backspace=5cm,margin=3cm,margindistance=.5cm,width=middle]
% \setuptexttexts[margin][\vbox to \textheight{\placenotes[mynote]\vfill}][]
% \starttext
% \dorecurse{10}{test \mynote{one one one one one one} \input zapf \mynote{one one one one one one} }
% \stoptext

%D The noterule can be a graphic and therefore calling this
%D setup macro at every skipswitch is tricky (many many MP
%D runs). Let's just reserve a few points, that probably match
%D those of the stretch component.

\def\placenoterule
  {\getvalue{\??vn\c!rule:\currentnote}}

\def\normalnoterule
  {\ifvmode
     \color
       [\noteparameter\c!rulecolor]
       {\hrule
          \!!width .2\hsize
          \!!height\noteparameter\c!rulethickness
          \!!depth \zeropoint}%
     \kern\strutdepth
   \fi}

%D The following switch can be used to disable limiting the
%D height of the footnote area, something that is needed in
%D multi column balancing. Use this switch with care.

\newif\ifnotelimit \notelimittrue

\def\setnotedistance
  {\setbox\scratchbox\vbox
     {\forgetall
      \noteparameter\c!before
      \placenoterule
      \noteparameter\c!after}%
   \global\skip\currentnoteins\ht\scratchbox
   \setbox\scratchbox\box\voidb@x} % scratchbox can be in use

\ifx\setnotehsize\undefined

  \def\setnotehsize{\hsize\noteparameter\c!width} % can be overloaded

\fi

\def\setclevernotes
  {\def\startpushnote  {\bgroup % wellicht ooit kopuitlijnen
                        \setupinmargin[\c!align=\v!left]%
                        \startnotedef}%
   \def\stoppushnote   {\stopnotedef
                        \egroup}%
   \let\startpopnotes   \donothing
   \let\stoppopnotes    \donothing}

\def\setcolumnnotes
  {\def\startpushnote  {\setnotehsize % possibly overloaded
                        \setrigidcolumnhsize\hsize{\noteparameter\c!columndistance}{\noteparameter\c!n}%
                        \bgroup
                        \setupinmargin[\c!align=\v!left]%
                        \startnotedef}%
   \def\stoppushnote   {\stopnotedef
                        \egroup}%
   \def\startpopnotes  {\bgroup
                        \setnotehsize
                        \setrigidcolumnhsize\hsize{\noteparameter\c!columndistance}{\noteparameter\c!n}%
                        \setbox0\vbox\bgroup}%
   \def\stoppopnotes   {\egroup
                        \setbox0\vbox
                          {\unvbox0\setbox0\lastbox
                           \ifvbox0\unvbox\else\box\fi0}%
                        \rigidcolumnbalance0\egroup}}

\def\settextnotes
  {\def\startpushnote  {\startvboxtohbox
                        \dostartattributes{\??vn\currentnote}\c!style\c!color\empty}%
   \def\stoppushnote   {\hskip\noteparameter\c!columndistance % plus.5em minus.5em
                        \dostopattributes
                        \stopvboxtohbox}%
   \def\startpopnotes  {\vbox\bgroup
                        \doifnotinset{\noteparameter\c!width}{\v!fit,\v!broad}\setnotehsize
                        \beginofshapebox}%
   \def\stoppopnotes   {\endofshapebox
                        \reshapebox{\ifhbox\shapebox\unhbox\else\box\fi\shapebox\endgraf}%
                        \flushshapebox
                        \egroup}}

%D The formatting depends on the width of the table, so we
%D have to set \type {n} to zero.
%D
%D \starttyping
%D \startbuffer
%D \bTABLE
%D \bTR \bTD one \footnote{\dorecurse{10}{abcd }} \eTD \bTD two \eTD \eTR
%D \bTR \bTD three fout five six seven eight nine \eTD \bTD ten \eTD \eTR
%D \eTABLE
%D \stopbuffer
%D
%D \startlocalfootnotes[n=0,location={text,none}]
%D \placelegend[n=2]{\getbuffer}{\placelocalfootnotes}
%D \stoplocalfootnotes
%D \stoptyping

%D \macros
%D   {footnote}
%D
%D A footnote can have a reference as optional argument and
%D therefore its formal specification looks like:
%D
%D \showsetup{footnote}
%D
%D This command has one optional command: the reference. By
%D saying \type{[-]} the number is omitted. The footnote
%D command is not that sensitive to spacing, so it's quite
%D legal to say:
%D
%D \startbuffer
%D Users of \CONTEXT\ must keep both feet \footnote{Given they
%D have two.} on the ground and not get confused \footnote{Or
%D even crazy.} by all those obscure \footnote{But fortunately
%D readable.} parameters.
%D \stopbuffer
%D
%D \typebuffer
%D
%D When setting the \type{conversion} to \type{set 2} we get
%D something like:
%D
%D \bgroup
%D \startnarrower
%D \setupfootnotes[conversion=set 1]
%D \getbuffer
%D \stopnarrower
%D \egroup
%D
%D Typesetting footnotes is, at least for the moment, disabled
%D when reshaping boxes.
%D
%D The additional macro \type {\footnotetext} and the
%D associated \type {\note} macro were implemented at
%D request of users on the mailing list and a suggestion by
%D taco to split of the symbol placement. I decided to
%D merge this functionality with the existing \type {\note}
%D functionality.

\newif\ifnotesymbol

\unexpanded\def\setnote    {\dotripleempty\dosetnote[1]}
\unexpanded\def\setnotetext{\dotripleempty\dosetnote[0]}

\def\dosetnote[#1][#2][#3]%
  {\unskip
   \def\currentnote{#2}%
   \ifcase#1\relax
     \global\notesymbolfalse
   \else
     \global\notesymboltrue
   \fi
   \ifvisible % misty feature, make it obsolete
     \ifreshapingbox
       \@EAEAEA\gobbletwoarguments
     \else
       \@EAEAEA\dodonote
     \fi
   \else % todo: \iftrialtypesetting
     \@EA\gobbletwoarguments
   \fi{#3}}

%D \macros
%D   {notesenabled}
%D
%D Before we come to typesetting a footnote, we first check
%D if we have to typeset a number. When a \type{-} is passed
%D instead of a reference, no number is typeset. We can
%D temporary disable footnotes by saying
%D
%D \starttyping
%D \notesenabledfalse
%D \stoptyping
%D
%D which can be handy while for instance typesetting tables
%D of contents. The pagewise footnote numbering is dedicated
%D to Han The Thanh, who needed it first.

\newif\ifnotesenabled \notesenabledtrue

\appendtoks \notesenabledfalse \to \everymarking

\newconditional\pagewisenotes % saves two hash entries

\def\lastnotepage{1}

\def\domovednote#1#2%
  {\ifconditional\pagewisenotes
     \doifreferencefoundelse{\s!fnt:t:\internalfootreference}
       {\let\savedrealreference\currentrealreference
        \doifreferencefoundelse{\s!fnt:f:\internalfootreference}
          {\ifnum\savedrealreference<\currentrealreference\relax\symbol[#1]\else
           \ifnum\savedrealreference>\currentrealreference\relax\symbol[#2]\fi\fi}
          \donothing}
       \donothing
   \fi}

\def\dodonote
  {\ifnotesenabled
     \iftrialtypesetting
       \@EAEAEA\nododonote
     \else
       \@EAEAEA\dododonote
     \fi
   \else
     \@EA\gobbletwoarguments
   \fi}

\def\nododonote#1%
  {\doifnot{#1}{-}{\kern.5em}% quick hack, approximation
   \gobbleoneargument}

\def\dododonote#1%
  {\doglobal\increment\internalfootreference
   \doifelse{\noteparameter\c!way}{\v!by\v!page}
     {\settrue\pagewisenotes}
     {\setfalse\pagewisenotes}%
   \doifelse{#1}{-}
     {\let\lastnotenumber\empty}
     {\ifconditional\pagewisenotes
        \doifreferencefoundelse{\s!fnt:t:\internalfootreference}
          {\ifnum\currentrealreference>\lastnotepage\relax
             \globallet\lastnotepage\currentrealreference
             \resetnumber[\currentnote]%
           \fi}
          {}%
      \fi
      \incrementnumber[\currentnote]%
      \makesectionnumber[\currentnote]%
      \rawreference\s!fnt{#1}\composedsectionnumber
      \let\lastnotenumber\composedsectionnumber}%
   \dostartnote}

%D The main typesetting routine is more or less the same as the
%D \PLAIN\ \TEX\ one, except that we only handle one type while
%D \PLAIN\ also has something \type{\v...}. In most cases
%D footnotes can be handled by a straight insert, but we do so
%D by using an indirect call to the \type{\insert} primitive.

\def\dostartlocalnoteinsert
  {\dochecknote
   \ifendnotes
     \global\setbox\localpostponednotes\vbox\bgroup
     \ifvoid\localpostponednotes\else\unvbox\localpostponednotes\fi
   \else
     \insert\currentnoteins\bgroup
   \fi}

\let\startlocalnoteinsert\dostartlocalnoteinsert
\let\stoplocalnoteinsert \egroup

%D Making footnote numbers active is not always that logical,
%D Making footnote numbers active is not always that logical,
%D especially when we keep the reference and text at one page.
%D On the other hand we need interactivity when we refer to
%D previous notes or use end notes. Therefore we support
%D interactive footnote numbers in two ways \footnote{This
%D feature was implemented years after we were able to do so,
%D mainly because endnotes had to be supported.} that is,
%D automatically (vise versa) and by user supplied reference.

\newcounter\internalfootreference

\let\startpushnote=\relax
\let\stoppushnote =\relax

\newsignal\notesignal
\newcount \notepenalty

\notepenalty=0 % needed in order to split in otrset

\newconditional\processingnote

\def\dostartnote% nog gobble als in pagebody
  {\bgroup
   \settrue\processingnote
  %\restorecatcodes % to be tested first
   \iftypesettinglines % otherwise problems with \type <crlf> {xxx}
     \ignorelines % makes footnotes work in \startlines ... \stoplines
   \fi
   \ifnotesymbol
     \dolastnotesymbol
   \else
     \unskip\unskip
     \globallet\lastnotesymbol\dolastnotesymbol
   \fi
   \startlocalnoteinsert
     \doif{\noteparameter\c!scope}\v!page{\floatingpenalty\maxdimen}% experiment
     \penalty\notepenalty
     \forgetall
     \setnotebodyfont
     \redoconvertfont % to undo \undo calls in in headings etc
     \splittopskip\strutht  % not actually needed here
     \splitmaxdepth\strutdp % not actually needed here
     \leftmargindistance\noteparameter\c!margindistance
     \rightmargindistance\leftmargindistance
     \ifcase\noteparameter\c!n\relax % new 31-07-99 ; always ?
       \doifnotinset{\noteparameter\c!width}{\v!fit,\v!broad}\setnotehsize
     \fi
     \startpushnote
       {\ifx\lastnotenumber\empty \else
          \preparethenumber{\??vn\currentnote}\lastnotenumber\preparednumber
          \doifelse{\noteparameter\c!interaction}\v!no
            {\noteparameter\c!numbercommand
               {\preparednumber\domovednote\v!nextpage\v!previouspage}}%
            {\gotobox{\noteparameter\c!command % was \c!numbercommand, but compatible
               {\preparednumber\domovednote\v!nextpage\v!previouspage}}%
               [\s!fnt:f:\internalfootreference]}%
        \fi
        \doifelse{\noteparameter\c!interaction}\v!no
          {\ifconditional\pagewisenotes
             \rawreference\s!fnt{\s!fnt:t:\internalfootreference}{}%
           \fi}%
          {\rawreference\s!fnt{\s!fnt:t:\internalfootreference}{}}}%
     \bgroup
     \postponenotes
     \aftergroup\dostopnote
     \begstrut
     \let\next}

\def\dostopnote
  {\endstrut
   \stoppushnote
   \egroup
   \stoplocalnoteinsert
   \kern\notesignal\relax} % \relax is needed to honor spaces

\def\dolastnotesymbol
  {\removeunwantedspaces
   \doifitalicelse\/\donothing % Charles IV \footnote{the fourth}
   \ifdim\lastkern=\notesignal
     \dodonotesymbol{\kern\noteparameter\c!distance}% gets the font right, hack !
   \fi
   \nobreak
   \doifelse{\noteparameter\c!interaction}\v!no
     {\dodonotesymbol{\lastnotenumber\domovednote\v!previouspage\v!nextpage}%
      \ifconditional\pagewisenotes
        \rawreference\s!fnt{\s!fnt:f:\internalfootreference}{}%
      \fi}
     {\gotobox
        {\dodonotesymbol{\lastnotenumber\domovednote\v!previouspage\v!nextpage}}%
        [\s!fnt:t:\internalfootreference]%
      \rawreference\s!fnt{\s!fnt:f:\internalfootreference}{}}%
   \globallet\lastnotesymbol\relax}

\let\lastnotesymbol\relax

%D \macros
%D   {note}
%D
%D Refering to a note is accomplished by the rather short
%D command:
%D
%D \showsetup{note}
%D
%D This command is implemented rather straightforward as:

\def\notesymbol
  {\dodoubleempty\donotesymbol}

% \def\donotesymbol[#1][#2]%
%   {\bgroup
%    \ifnotesenabled
%      \def\currentnote{#1}%
%      \ifsecondargument
%        \ifx\lastnotesymbol\relax
%          \unskip
%          \naarbox{\high{\tx\currenttextreference}}[#2]%
%        \else
%          \lastnotesymbol
%        \fi
%      \else
%        \lastnotesymbol
%      \fi
%    \fi
%    \egroup}

\def\dodonotesymbol#1%
  {\noteparameter\c!textcommand{\doattributes{\??vn\currentnote}\c!textstyle\c!textcolor{#1}}}

% \def\donotesymbol[#1][#2]%
%   {\bgroup
%    \ifnotesenabled
%      \def\currentnote{#1}%
%      \ifsecondargument
%        \ifx\lastnotesymbol\relax % bugged
%          \unskip
%          \gotobox{\dodonotesymbol\currenttextreference}[#2]%
%        \else
%          \lastnotesymbol
%        \fi
%      \else
%        \lastnotesymbol
%      \fi
%    \fi
%    \egroup}

\def\donotesymbol[#1][#2]%
  {\bgroup
   \ifnotesenabled
     \def\currentnote{#1}%
     \ifsecondargument
       \unskip
       \gotobox{\dodonotesymbol\currenttextreference}[#2]%
     \else
       \lastnotesymbol
     \fi
   \fi
   \egroup}

%D Normally footnotes are saved as inserts that are called upon
%D as soon as the pagebody is constructed. The footnote
%D insertion routine looks just like the \PLAIN\ \TEX\ one,
%D except that we check for the end note state.

\let\startpopnotes = \relax
\let\stoppopnotes  = \relax

\def\placenoteinserts
  {\processnotes\doplacenoteinserts}

\def\doplacenoteinserts
  {%\ifvoid\currentnoteins \else % unsafe, strange
   \relax\ifdim\ht\currentnoteins>\zeropoint\relax
     \dochecknote
     \ifendnotes \else
       \noteparameter\c!before
       \placenoterule  % alleen in ..mode
       \bgroup
       \setnotebodyfont
       \setbox0\hbox
         {\startpopnotes
          \setnotebodyfont
%          % this should be checked, smells like a mix-up
%          % does not split: \ifcase\noteparameter\c!n\unvbox\else\box\fi\currentnoteins
%          \ifcase\noteparameter\c!n
             \box\currentnoteins
%          \else
%            \unvbox\currentnoteins
%          \fi
          % this is too ugly actually
          \stoppopnotes}%
       \setbox2\hbox
         {\localframed
            [\??vn\currentnote]
            [\c!width=\v!fit,
             \c!height=\v!fit,
             \c!strut=\v!no,
             \c!offset=\v!overlay]
            {\ifdim\dp0=\zeropoint         % this hack is needed because \vadjust
               \hbox{\lower\strutdp\box0}% % in margin number placement
             \else                         % hides the (always) present depth
               \box0
             \fi}}%
       \setbox2\hbox{\lower\strutdepth\box2}%
       \dp2=\strutdepth % so we know that it has the note bodyfont depth
       \box2
       \egroup
       \noteparameter\c!after
     \fi
   \fi}

%D Supporting end notes is surprisingly easy. Even better, we
%D can combine this feature with solving the common \TEX\
%D problem of disappearing inserts when they're called for in
%D deeply nested boxes. The general case looks like:
%D
%D \starttyping
%D \postponenotes
%D \.box{whatever we want with footnotes}
%D \flushnotes
%D \stoptyping
%D
%D This alternative can be used in headings, captions, tables
%D etc. The latter one sometimes calls for notes local to
%D the table, which can be realized by saying
%D
%D \starttyping
%D \setlocalfootnotes
%D some kind of table with local footnotes
%D \placelocalfootnotes
%D \stoptyping
%D
%D Postponing is accomplished by simply redefining the (local)
%D insert operation. A not too robust method uses the
%D \type{\insert} primitive when possible. This method fails in
%D situations where it's not entirely clear in what mode \TEX\
%D is. Therefore the auto method can is to be overruled when
%D needed.

\newconditional\postponednote

\def\autopostponenotes
  {\def\startlocalnoteinsert % not global
     {\ifinner
        %\message{[postponed note]}%
        \global\setbox\localpostponednotes\vbox\bgroup
          \global\settrue\postponednote
          \ifvoid\localpostponednotes\else\unvbox\localpostponednotes\fi
      \else
        %\message{[inserted note]}%
        \expandafter\dostartlocalnoteinsert
      \fi}}

\def\postponenotes
  {\let\autopostponenotes\postponenotes
   \let\postponenotes\relax % prevent loops
   \def\startlocalnoteinsert % not global
     {%\message{[postponed note]}%
      \global\setbox\localpostponednotes\vbox\bgroup
      \global\settrue\postponednote
%         \unvbox\localpostponednotes}}
      \ifvoid\localpostponednotes\else\unvbox\localpostponednotes\fi}}

\def\dodoflushnotes % per class, todo: handle endnotes here
  {\ifdim\ht\localpostponednotes>\zeropoint
     \bgroup
     \dochecknote
     \ifendnotes \else
       % not that accurate when multiple notes
       \scratchdimen\pagegoal
       \advance\scratchdimen -\pagetotal
       \ifdim\scratchdimen<\ht\localpostponednotes
         \message{[moved note \currentnote]}%
       \fi
       \insert\currentnoteins\bgroup\unvbox\localpostponednotes\egroup
     \fi
     \egroup
   \fi}

\def\doflushnotes % also called directly, \ifvoid is needed !
  {\ifconditional\processingnote \else \ifconditional\postponednote
     \let\localnoteinsert\normalnoteinsert % not global
     \processnotes\dodoflushnotes
     \global\setfalse\postponednote
   \fi \fi}

\def\flushnotes
  {\ifconditional\processingnote \else \ifconditional\postponednote
     \ifinner \else \ifinpagebody \else
       %\ifvmode % less interference, but also less secure
       \doflushnotes
       %\fi
     \fi \fi
   \fi \fi}

%D This is a nasty and new secondary footnote flusher. It
%D can be hooked into \type {\everypar} like:
%D
%D \starttyping
%D \appendtoks \synchronizenotes \to \everypar
%D \stoptyping

\def\dosynchronizenotes
  {\ifvoid\currentnoteins\else\insert\currentnoteins{\unvbox\currentnoteins}\fi}

\def\synchronizenotes
  {\processnotes\dosynchronizenotes}

%D There are several placement alternatives.

\def\placenotesintext#1%
  {\ifdim\ht#1>\zeropoint
     \endgraf
     \ifvmode
       \whitespace
       \noteparameter\c!before
     \fi
     \snaptogrid\hbox
       {\setnotebodyfont
        \setbox0\hbox
          {\startpopnotes
           \unvbox#1\endgraf\relax
           \stoppopnotes}%
        \doif{\noteparameter\c!width}\v!fit % new, auto width
          {\setbox0\hbox                          % uggly but ok.
             {\beginofshapebox
              \unhbox0\setbox0=\lastbox\unvbox0
              \endofshapebox
              \reshapebox{\hbox{\unhbox\shapebox}}%
              \vbox{\flushshapebox}}}%
        \localframed
          [\??vn\currentnote]
          [\c!width=\v!fit,
            \c!height=\v!fit,
             \c!strut=\v!no,
            \c!offset=\v!overlay]
          {\ifdim\dp0=\zeropoint   % this hack is needed because \vadjust
             \hbox{\lower\strutdp\box0}% % in margin number placement
           \else                   % hides the (always) present depth
             \box0
           \fi}}%
     \ifvmode
       \noteparameter\c!after
     \fi
   \fi}

%D A stupid alternative is also provided:
%D
%D \starttyping
%D \setupfootnotes[location={text,none}]
%D \stoptyping

\def\placenotesasnone#1% is grouped already
  {\ifdim\ht#1>\zeropoint
     \noteparameter\c!before
     \setnotebodyfont
     \startpopnotes % make sure that fake height is killed
     \unvbox#1\endgraf
     \stoppopnotes
     \setbox0=\lastbox \ifvbox0 \unvbox0\else\box0\fi % enable columns
     \noteparameter\c!after
   \fi}

%D \macros
%D   {startlocalfootnotes,placelocalfootnotes}
%D
%D The next two macros can be used in for instance tables, as
%D we'll demonstrate later on.
%D
%D \showsetup{startlocalfootnotes}
%D \showsetup{placelocalfootnotes}

\def\defaultnotewidth{\makeupwidth}

% \def\collectlocalnotes
%   {\def\localnoteinsert##1% was \gdef, but never reset!
%      {%\message{[local note]}%
%       \global\setbox\localpostponednotes\vbox\bgroup
%         \ifvoid\localpostponednotes \else
%           \unvbox\localpostponednotes
%         \fi
%         \let\next}}

\def\collectlocalnotes
  {\def\startlocalnoteinsert% was \gdef, but never reset!
     {%\message{[local note]}%
      \global\setbox\localpostponednotes\vbox\bgroup
      \ifvoid\localpostponednotes\else\unvbox\localpostponednotes\fi}}

\def\startlocalnotes
  {\bgroup % here because we support \vbox\startlocalnotes
   \dosingleempty\dostartlocalnotes}

\def\dostartlocalnotes[#1]%
  {\let\autopostponenotes\postponenotes
   \let\postponenotes\collectlocalnotes
   \def\defaultnotewidth{\ifdim\hsize<\makeupwidth\hsize\else\makeupwidth\fi}%
   \processnotes
     {\doifsomething{#1}{\setupnote[\currentnote][#1]}%
      \savenumber[\currentnote]%
      \resetnumber[\currentnote]}%
   \collectlocalnotes}

\def\stoplocalnotes
  {\processnotes{\restorenumber[\currentnote]}%
   \egroup
   \checknotes} % really needed, else wrong main settings

\def\placelocalnotes
  {\dodoubleempty\doplacelocalnotes}

\def\doplacelocalnotes[#1][#2]%
  {\bgroup
   \dolocalsetupnotes{#1}{#2}%
   \processnotes
     {\ExpandBothAfter\doifinsetelse\v!none{\noteparameter\c!location}
        \placenotesasnone\placenotesintext\localpostponednotes}%
   \egroup
   \checknotes}

%D These commands can be used like:
%D
%D \startbuffer
%D \startlocalnotes[width=.3\hsize,n=0]
%D   \placetable
%D     {Some Table}
%D     \placeontopofeachother
%D       {\starttable[|l|r|]
%D        \HL
%D        \VL Nota\footnote{Bene} \VL Bene\footnote{Nota} \VL\SR
%D        \VL Bene\footnote{Nota} \VL Nota\footnote{Bene} \VL\SR
%D        \HL
%D        \stoptable}
%D       {\placelocalnotes}
%D \stoplocalnotes
%D \stopbuffer
%D
%D \typebuffer
%D
%D Because this table placement macro expect box content, and
%D thanks to the grouping of the local footnotes, we don't need
%D additional braces.
%D
%D \getbuffer

%D \macros
%D   {placefootnotes}
%D
%D We still have no decent command for placing footnotes
%D somewhere else than at the bottom of the page (for which no
%D user action is needed). Footnotes (endnotes) can be
%D placed by using
%D
%D \showsetup{placefootnotes}

\def\placebottomnotes
  {\processnotes\dodoplacenotes}

% \definecomplexorsimple\placenotes

% \def\simpleplacenotes
%   {\processnotes\dodoplacenotes}

% \def\complexplacenotes[#1]%
%   {\bgroup
%    \edef\noteinsertions{#1}%
%    \simpleplacenotes
%    \egroup}

\def\placenotes
  {\dodoubleempty\doplacenotes}

\def\doplacenotes[#1][#2]%
  {\bgroup
   \dolocalsetupnotes{#1}{#2}%
   \processnotes\dodoplacenotes
   \egroup}

\def\dodoplacenotes
  {\dochecknote
   \ifendnotes
     \ifinpagebody \else \ifdim\ht\localpostponednotes>\zeropoint
       \expanded{\doifinsetelse{\v!none}{\noteparameter\c!location}}
         \placenotesasnone\placenotesintext\localpostponednotes
     \fi \fi
   \else \ifdim\ht\currentnoteins>\zeropoint
     \placenoteinserts
   \fi \fi}

%D \macros
%D   {fakenotes}

\def\fakenotes
  {\ifhmode\endgraf\fi\ifvmode
     \calculatetotalclevernoteheight
     \ifdim\totalnoteheight>\zeropoint \kern\totalnoteheight \fi
   \fi}

\def\fakepagenotes
  {\ifhmode\endgraf\fi\ifvmode
     \calculatetotalpagenoteheight
     \ifdim\totalnoteheight>\zeropoint \kern\totalnoteheight \fi
   \fi}

\newdimen\totalnoteheight


\def\doaddtototalnoteheight#1%
  {\ifdim\ht#1>\zeropoint
     \advance\totalnoteheight\ht  #1%
     \advance\totalnoteheight\skip#1%
   \fi}

\def\docalculatetotalnoteheight
  {\ifcase\clevernotes % tricky here ! ! ! to be sorted out ! ! !
     \doaddtototalnoteheight\currentnoteins
   \else
     \doaddtototalnoteheight\currentbackupnoteins
   \fi}

\def\docalculatetotalclevernoteheight
  {\ifcase\clevernotes \else % tricky here ! ! ! to be sorted out ! ! !
     \doaddtototalnoteheight\currentnoteins
   \fi}

\def\docalculatetotalpagenoteheight
  {\doaddtototalnoteheight\currentnoteins}

\def\calculatetotalnoteheight      {\totalnoteheight\zeropoint\processnotes\docalculatetotalnoteheight}
\def\calculatetotalclevernoteheight{\totalnoteheight\zeropoint\processnotes\docalculatetotalclevernoteheight}
\def\calculatetotalpagenoteheight  {\totalnoteheight\zeropoint\processnotes\docalculatetotalpagenoteheight}

\newif\ifnotespresent

\def\dochecknotepresence
  {\ifdim\ht\currentnoteins>\zeropoint
     \notespresenttrue
   \fi}

\def\checknotepresence
  {\notespresentfalse
   \processnotes\dochecknotepresence}

%D Now how can this mechanism be hooked into \CONTEXT\ without
%D explictly postponing footnotes? The solution turned out to
%D be rather simple:
%D
%D \starttyping
%D \everypar  {...\flushnotes...}
%D \neverypar {...\postponenotes}
%D \stoptyping
%D
%D and
%D
%D \starttyping
%D \def\ejectinsert%
%D   {...
%D    \flushnotes
%D    ...}
%D \stoptyping
%D
%D We can use \type{\neverypar} because in most commands
%D sensitive to footnote gobbling we disable \type{\everypar}
%D in favor for \type{\neverypar}. In fact, this footnote
%D implementation is the first to use this scheme.

%D When typesetting footnotes, we have to return to the
%D footnote specific bodyfont size, which is in most cases derived
%D from the global document bodyfont size. In the previous macros
%D we already used a footnote specific font setting macro.

\def\setnotebodyfont
  {\let\setnotebodyfont\relax
   \restoreglobalbodyfont
   \switchtobodyfont[\noteparameter\c!bodyfont]%
   \setuptolerance[\noteparameter\c!tolerance]%
   \setupalign[\noteparameter\c!align]}

%D The footnote mechanism defaults to a traditional one
%D column way of showing them. By default we precede them by
%D a small line.

\ifx\v!endnote\undefined \def\v!endnote{endnote} \fi

\definenote [\v!footnote  ]
\definenote [\v!endnote   ] [\c!location=\v!none] % else no break

% \definenote
%   [mynote]
%   [way=bypage,
%    location={page,high},
%    factor=0,
%    width=\leftmarginwidth,
%    scope=page,
%    rule=,
%    before=,
%    after=]
%
% \setuptexttexts
%   [margin]
%   [\vbox to \textheight{\placenotes[mynote]\vfill}]
%   []

%D Compatibility macros:

            \def\setupfootnotedefinition{\setupnotedefinition                [\v!footnote]}
            \def\setupfootnotes         {\setupnote                          [\v!footnote]}
\unexpanded \def\footnote               {\setnote                            [\v!footnote]}
\unexpanded \def\footnotetext           {\setnotetext                        [\v!footnote]}
            \def\note                   {\dodoubleempty\notesymbol           [\v!footnote]} %  alleen footnote
            \def\placefootnotes         {\dodoubleempty\doplacefootnotes     [\v!footnote]}
            \def\placelocalfootnotes    {\dodoubleempty\doplacelocalfootnotes[\v!footnote]}
            \def\startlocalfootnotes    {\startlocalnotes}
            \def\stoplocalfootnotes     {\stoplocalnotes }

\def\doplacefootnotes     [#1][#2]%
  {\ifsecondargument\placenotes     [#1][#2,\c!height=\textheight]\else\placenotes     [#1]\fi}

\def\doplacelocalfootnotes[#1][#2]%
  {\ifsecondargument\placelocalnotes[#1][#2,\c!height=\textheight]\else\placelocalnotes[#1]\fi}

%D Backward compatibility command:

\def\footins              {\noteinsertion\currentnote}
\def\postponefootnotes    {\postponenotes}
\def\autopostponefootnotes{\autopostponenotes}

%D New trickery:

\def\ownnotesymbol#1% #1 gets number passed
  {\executeifdefined{\??vn::\currentnote}\empty}

\def\setnotesymbol[#1]#2#3%
  {\prewordbreak % prevent lookback
   \gdef\lastnotenumber{#2}%
   \setgvalue{\??vn::#1}{#3}
   \dolastnotesymbol}

\def\ownnote[#1]#2#3#4%
  {\setnotesymbol[#1]{#2}{#3}%
   \setnotetext  [#1]{#4}}

\defineconversion
  [ownnote]
  [\ownnotesymbol]

%D Usage:

% maybe we should predefine this one

% \definenote
%   [glossary]
%   [way=bypage,
%    location={page,high},
%    factor=0,
%    width=\leftmarginwidth,
%    scope=page,
%    conversion=ownnote,
%    numbercommand=,
%    textcommand=,
%    textstyle=,
%    rule=,
%    before=,
%    after=]
%
% \setupnotedefinition
%   [glossary]
%   [location=left,
%    width=fit,
%    distance=.5em,
%    align={right,tolerant,stretch},
%    headstyle=bold,
%    hang=1]
%
% \setuplayout
%   [width=middle,
%    height=middle,
%    backspace=5cm,
%    margin=4cm,
%    margindistance=.25cm,
%    cutspace=2cm]
%
% \setuptexttexts
%   [margin]
%   [\setups{glossary}]
%   [\setups{glossary}]
%
% \startsetups glossary
%     \vbox to \textheight {\placenotes[glossary]\vfill}
% \stopsetups
%
% \dorecurse{10}
%   {\dorecurse{5}
%      {\ownnote[glossary]{whow}{whatever needs to be glossed:~\recurselevel}%
%       \input tufte \relax}}

\protect \endinput

% \def\myfootnote[#1]#2% let's guess that #2 is without catcode problems
%   {\setgvalue{note:t:#1}%
%      {\setxvalue{note:l:#1}{\getvalue{note:n:#1}}%
%       \footnote[note:a:#1:\getvalue{note:l:#1}]{#2}}%
%    \setgvalue{note:n:#1}%
%      {1}%
%    \getvalue{note:t:#1}}%

% \def\mynote[#1]%
%   {\removeunwantedspaces
%    \scratchcounter\getvalue{note:n:#1}\relax
%    \edef\NoteNumber{\the\scratchcounter}%
%    \doglobal\incrementvalue{note:n:#1}\relax
%    \doifreferencefoundelse{note:a:#1:1}
%      {\edef\NotePageA{\number\currentrealreference}%
%       \doifreferencefoundelse{note:a:#1:\getvalue{note:n:#1}}
%         {\edef\NotePageB{\number\currentrealreference}%
%          \doifreferencefoundelse{note:a:#1:\NoteNumber}
%            {\ifnum\currentrealreference=\NotePageB\relax
%               \pagereference[note:a:#1:\getvalue{note:n:#1}]%
%               \note[note:a:#1:\getvalue{note:l:#1}]%
%             \else\ifnum\NotePageA=\NotePageB\relax
%               \pagereference[note:a:#1:\getvalue{note:n:#1}]%
%               \note[note:a:#1:\getvalue{note:l:#1}]%
%             \else
%               \getvalue{note:t:#1}%
%             \fi\fi}
%            {\ifnum\NotePageA=\NotePageB\relax
%               \pagereference[note:a:#1:\getvalue{note:n:#1}]%
%               \note[note:a:#1:\getvalue{note:l:#1}]%
%             \else
%               \getvalue{note:t:#1}%
%             \fi}}
%         {\pagereference[note:a:#1:\getvalue{note:n:#1}]%
%          \note[note:a:#1:\getvalue{note:l:#1}]}}
%      {\pagereference[note:a:#1:\getvalue{note:n:#1}]%
%       \note[note:a:#1:\getvalue{note:l:#1}]}}

% \starttext

% funny \myfootnote[funny]{funny} funny \mynote[funny] \page

% funny \mynote[funny] funny \mynote[funny] funny \mynote[funny] \page
% funny \mynote[funny] funny \mynote[funny] \page
% funny \mynote[funny] funny \mynote[funny] funny \mynote[funny] \page

% \stoptext
