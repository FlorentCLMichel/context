%D \module
%D   [      file=x-fig-01,
%D        version=2001.03.21,
%D          title=\CONTEXT\ Style File,
%D       subtitle=Figure Base Generation,
%D         author=Hans Hagen,
%D           date=\currentdate,
%D      copyright={PRAGMA / Hans Hagen \& Ton Otten}]
%C
%C This module is part of the \CONTEXT\ macro||package and is
%C therefore copyrighted by \PRAGMA. See mreadme.pdf for
%C details.

%D See \type {x-fig-00.tex} and \type {x-fig-04.tex} for more
%D information on how to use and generate figure databases.
%D This file loads the file named \type {\jobfilename}
%D (\TEXEXEC\ will set this variable). You can apply this
%D style to a database by saying:
%D
%D \starttypen
%D texexec --pdf --use=fig-make yourfile.xml
%D \stoptypen
%D
%D The following modes are supported:
%D
%D \starttabulatie[|lT|l|]
%D \NC letter  \NC map the preview on letter size  \NC \NR
%D \NC compact \NC use an alternative presentation \NC \NR
%D \stoptabulatie
%D
%D The resulting file has the following characteristics:
%D
%D \startopsomming[opelkaar]
%D \som  the document is split into three sections: first each
%D       figure is shown at its own page, then an overview of
%D       figures is shown with some data alongside, and
%D       finally an index and table of contents shows up
%D \som  there is no title page, which means that one can
%D       access a figure by page number without offset
%D \som  the document is opened at the first overview page,
%D       that is, when the viewer supports it
%D \som  the graphic is shown 3~times: on a page of its own,
%D       scaled to a fixed dimension, and relative to a4 or
%D       letter paper size
%D \som  the labels can be accessed in an index and list at
%D       the end of the document
%D \stopopsomming
%D
%D We use named destinations, which means that one can
%D access a figure by name from an external application.

\usemodule[res-00]

\autoXMLnamespace[rl]

\setupoutput[pdftex] \overcomePDFspacefalse

\doifnothing  {\jobfilename}        {\end}
\doiffileelse {\jobfilename.xml} {} {\end}

\def\StartDescription
  {\bgroup}

\def\StopDescription
  {\subject {Figure collection}
   \starttabulate[|lBe|p|]
   \doifXMLdataelse{rl:organization}
     {\NC organization \NC \XMLflush{rl:organization} \NC \NR}{}
   \doifXMLdataelse{rl:project}
     {\NC project      \NC \XMLflush{rl:project}      \NC \NR}{}
   \doifXMLdataelse{rl:product}
     {\NC product      \NC \XMLflush{rl:product}      \NC \NR}{}
   \doifXMLdataelse{rl:comment}
     {\NC comment      \NC \XMLflush{rl:comment}      \NC \NR}{}
   \stoptabulate
   \blank[2*big]
   \egroup}

\def\StartFigureA
  {\bgroup
   \XMLassign{rl:file}{unknown}}

\defineoverlay[page][\overlaybutton{Description}]

\startbuffer[unknown]
  \framed
    [width=\XMLpar{rl:dummy}{width}{12cm},
     height=\XMLpar{rl:dummy}{height}{8cm},
     background=color,
     backgroundcolor=gray,
     foregroundcolor=darkred, 
     align=normal,
     frame=off]
    {\bf \XMLflush{rl:dummy}}
\stopbuffer

\useexternalfigure[unknown][unknown][type=buffer,object=no]

% \def\externalfigurereplacement#1#2#3%
%   {\getbuffer[rl-unknown]}

\def\StopFigureA
  {\doglobal\increment\CurrentPage
   \setupbackgrounds[page][background=page]
   \doifelsenothing{\XMLflush{rl:label}}
     {\expanded{\definereference[Description][about: \XMLflush{rl:file}]}%
      \expanded{\pagereference[\XMLflush{rl:file}]}}
     {\expanded{\definereference[Description][about: \XMLflush{rl:label}]}%
      \expanded{\pagereference[\XMLflush{rl:label}]}}
   \pagefigure[\XMLflush{rl:file}]
   \setupbackgrounds[page][background=]
   \egroup}

\def\StartFigureB
  {\StartFigureA}

\defineregister
  [figureindex]
  [figureindices]

\setupregister
  [figureindex]
  [ownnumber=yes,
   criterium=text,
   interaction=text,
   indicator=no]

\definelist
  [figurelist]

\setuplist
  [figurelist]
  [criterium=text,
   pagenumber=no,
   width=2em,
   interaction=all]

\setupcolors
  [state=start]

\setuptolerance
  [verytolerant]

% Ulgy:

\startnotmode[previewpage-letter,previewpage-S6]
  \enablemode[previewpage-A4]
\stopnotmode

\startmode[letter] % downward compatible
  \enablemode[previewpage-letter]
\stopmode

%startbuffer[paper]
\startsetups[paper]
\startmode[previewpage-A4]
  \framed
    [width=210mm,height=297mm,offset=overlay,frame=off,
     background=color,backgroundcolor=white]
    {\externalfigure[\XMLflush{rl:file}][reset=yes]}
\stopmode
\startmode[previewpage-letter]
  \framed
    [width=8.5in,height=11in,offset=overlay,frame=off,
     background=color,backgroundcolor=white]
    {\externalfigure[\XMLflush{rl:file}][reset=yes]}
\stopmode
\startmode[previewpage-S6]
  \framed
    [width=600pt,height=450pt,offset=overlay,frame=off,
     background=color,backgroundcolor=white]
    {\externalfigure[\XMLflush{rl:file}][reset=yes]}
\stopmode
\stopsetups
%stopbuffer

\setupbuttons
  [offset=10pt,
   width=broad,
   strut=no,
   rulethickness=1pt,
   framecolor=darkred]

\definecolor[XMLRLcolor][white]

\def\StopFigureB
  {\doglobal\increment\CurrentPage
   \doifelsenothing{\XMLflush{rl:label}}
     {\expanded{\definereference[Figure][\XMLflush{rl:file}]}%
      \expanded{\definereference[GridPg][grid:\XMLflush{rl:file}]}}
     {\expanded{\definereference[Figure][\XMLflush{rl:label}]}%
      \expanded{\definereference[GridPg][grid:\XMLflush{rl:label}]}}%
   \button
     {\hbox to \hsize
        {\forgetall \dontcomplain
         \doifelsenothing{\XMLflush{rl:label}}
           {\expanded{\pagereference[about: \XMLflush{rl:file}]}}
           {\expanded{\pagereference[about: \XMLflush{rl:label}]}}%
         % moved here, because descriptions may be absent
         \ifnum\CurrentPage=1 \pagereference[begin]\fi
         %
         \expanded{\writetolist[figurelist]{\CurrentPage}{\XMLflush{rl:label}}}%
         \expanded{\figureindex{\CurrentPage}{\XMLflush{rl:label}}}%
         \startnotmode[compact]%
           \vbox to 100pt
             {\hsize30pt
              \vskip5pt
              \hbox to \hsize{\hss\strut\bf\CurrentPage\hss}%
              \vfill}%
           \advance\hsize by -30pt
         \stopnotmode
         \startmode[compact]%
           \advance\hsize by -10pt
           \hskip10pt
         \stopmode
         \button % \framed
           [width=150pt,height=100pt,offset=10pt,frame=off,
            background=color,backgroundcolor=white,color=]
           {\externalfigure
              [\XMLflush{rl:file}]
              [maxheight=80pt,frame=off,maxwidth=130pt,factor=max]}%
           [GridPg]%
         \freezedimenmacro\naturalfigurewidth \let\FigWid\naturalfigurewidth
         \freezedimenmacro\naturalfigureheight\let\FigHei\naturalfigureheight
         \advance\hsize by -150pt
         \hskip10pt
         \advance\hsize by -10pt
         \vbox to 100pt
           {\hsize40pt
           %\externalfigure
           %  [paper]
           %  [type=buffer,frame=on,
           %   framecolor=darkred,rulethickness=.5pt,
           %   width=40pt,object=no]
            \framed
              [offset=overlay,
               framecolor=darkred,
               rulethickness=.5pt]
              {\scale[width=40pt]{\setups[paper]}}% {\disableXML\getbuffer[paper]}}%
            \startmode[compact]%
              \vfill
              \hbox to \hsize{\hss\strut\bf\CurrentPage\hss}%
            \stopmode
            \vfill}%
         \advance\hsize by -40pt
         \hskip10pt
         \advance\hsize by -10pt
         \vbox to 100pt
           {\blank[disable]
            \starttabulate[|Bel|p|]
            \NC file \NC \XMLflush{rl:file} \NC \NR
            \doifXMLdata{rl:label}
              {\NC label \NC \XMLflush{rl:label} \NC \NR}
            \NC w$\times$h \NC \FigWid$\times$\FigHei \NC \NR
            \doifXMLdata{rl:copyright}
              {\NC copyright \NC \XMLflush{rl:copyright} \NC \NR}
            \doifXMLdata{rl:status}
              {\doif{\XMLflush{rl:status}}{obsolete}
                 {\NC status \NC \bf\darkred\XMLflush{rl:status} \NC \NR}
                 {\NC status \NC \XMLflush{rl:status} \NC \NR}}
            \doifXMLdata{rl:comment}
              {\NC comment \NC \XMLflush{rl:comment} \NC \NR}
            \stoptabulate
            \vfill}}}%
     [Figure]
  \vskip10pt
  \egroup}

\def\StartFigureC
  {\StartFigureA}

\def\StopFigureC
  {\doglobal\increment\NumberOfFigures
   \egroup}

\setuplayout
  [topspace=15pt,backspace=15pt,
   header=0pt,footer=0pt,bottom=20pt,bottomdistance=10pt,
   width=middle,height=fit]

\setupbackgrounds
  [page]
  [background=,
   backgroundcolor=gray]

\setupinteractionscreen
  [width=max,
   height=max]

\setupcolors
  [state=start]

\setupinteraction
  [style=,
   color=,
   contrastcolor=,
   state=start]

\setuphead
  [section]
  [style=bfb]

\setupbodyfont
  [pos]

\setupinteractionmenu
  [bottom]
  [left=\hfill,
   middle=\hskip10pt,
   frame=off,
   style=bold,
   background=color,
   backgroundcolor=darkred,
   foregroundcolor=white]

\startinteractionmenu[bottom]
  \but [begin]         begin   \\
  \but [index]         index   \\
  \but [list]          list    \\
  \but [CloseDocument] close   \\
  \but [PreviousJump]  go back \\
\stopinteractionmenu

\setupinteraction
  [openaction=begin]

\defineXMLenvironment [rl:figurelibrary] \StartLibrary \StopLibrary
\defineXMLenvironment [rl:library]       \StartLibrary \StopLibrary

\starttext

\def\StartLibrary{\mainlanguage[\XMLpar{rl:library}{language}{en}]}
\def\StopLibrary {}

\defineXMLignore      [rl:description]
\defineXMLenvironment [rl:figure]      \StartFigureC \StopFigureC

\doglobal\newcounter\CurrentPage

\processXMLfilegrouped{\jobfilename.xml}

\increment\NumberOfFigures

\defineXMLignore      [rl:description]
\defineXMLenvironment [rl:figure]      \StartFigureA \StopFigureA

\doglobal\newcounter\CurrentPage

\processXMLfilegrouped{\jobfilename.xml}

\setuppapersize
  [S6][S6]

\setupbackgrounds
  [page]
  [background=color]

\setupinteraction
  [menu=on]

\defineXMLenvironment [rl:description] \StartDescription \StopDescription
\defineXMLenvironment [rl:figure]      \StartFigureB     \StopFigureB

\doglobal\newcounter\CurrentPage

\processXMLfilegrouped{\jobfilename.xml} \page

\subject [list] {List of figures}

\placelist[figurelist] \page

\subject [index] {Index of figures}

\startcolumns
\placeregister[figureindex]
\stopcolumns

\doifmodeelse{clipgrid-distance,clipgrid-steps}{\page}{\stoptext}

\startuniqueMPgraphic{clipgrid}{dx,dy,nx,ny,type}
  numeric gdx, gdy, lbx, lby ;
  if \MPvar{type}=1 :
    gdx := \MPvar{dy} ;
    gdy := \MPvar{dx} ;
  else :
    gdx := OverlayWidth /\MPvar{nx} ;
    gdy := OverlayHeight/\MPvar{ny} ;
  fi ;
  lbx := gdx ;
  lby := gdy ;
  defaultfont  := "\truefontname{Mono}" ;
  defaultscale := .5 ;
  numeric pen ; pen := .25pt ;
  def MyGrid text t =
    draw vlingrid (0,OverlayWidth ,gdy,OverlayWidth ,OverlayHeight) t ;
    draw hlingrid (0,OverlayHeight,gdx,OverlayHeight,OverlayWidth ) t ;
  enddef ;
  pickup pencircle scaled pen ;
  MyGrid withcolor white ;
  MyGrid dashed evenly scaled pen ;
  draw OverlayBox withcolor white ;
  draw OverlayBox dashed evenly scaled pen ;
  draw vlinlabel.bot(0,eps+OverlayWidth /lby,2,OverlayWidth ) ;
  draw hlinlabel.lft(0,eps+OverlayHeight/lbx,2,OverlayHeight) ;
  setbounds currentpicture to OverlayBox enlarged (2*EmWidth) ;
\stopuniqueMPgraphic

\presetMPvariable[clipgrid][dx=10pt]
\presetMPvariable[clipgrid][dy=10pt]
\presetMPvariable[clipgrid][nx=10]
\presetMPvariable[clipgrid][ny=10]

\startmode[clipgrid-distance]
  \defineoverlay[grid][\uniqueMPgraphic{clipgrid}{type=1}]
\stopmode

\startmode[clipgrid-steps]
  \defineoverlay[grid][\uniqueMPgraphic{clipgrid}{type=2}]
\stopmode

\setupexternalfigures
  [background={color,foreground,grid},
   backgroundcolor=white]

\def\StartFigureD
  {\StartFigureA}

\def\StopFigureD
  {\doglobal\increment\CurrentPage
   \setupbackgrounds[page][background=page]
   \startpagefigure[\XMLflush{rl:file}][offset=20pt]%
     \doifelsenothing{\XMLflush{rl:label}}
       {\expanded{\definereference[Description][about: \XMLflush{rl:file}]}%
        \expanded{\pagereference[grid:\XMLflush{rl:file}]}}
       {\expanded{\definereference[Description][about: \XMLflush{rl:label}]}%
        \expanded{\pagereference[grid:\XMLflush{rl:label}]}}
   \stoppagefigure
  %\pagefigure[\XMLflush{rl:file}][offset=20pt]
   \setupbackgrounds[page][background=]
   \egroup}

\defineXMLignore      [rl:description]
\defineXMLenvironment [rl:figure]      \StartFigureD \StopFigureD

\doglobal\newcounter\CurrentPage

\processXMLfilegrouped{\jobfilename.xml} \page

\stoptext
