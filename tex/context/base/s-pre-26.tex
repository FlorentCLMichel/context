%D \module
%D   [      file=s-pre-26,
%D        version=2001.02.18,
%D          title=\CONTEXT\ Style File,
%D       subtitle=Presentation Environment 26,
%D         author=Hans Hagen,
%D           date=\currentdate,
%D      copyright={PRAGMA / Hans Hagen \& Ton Otten}]
%C
%C This module is part of the \CONTEXT\ macro||package and is
%C therefore copyrighted by \PRAGMA. See mreadme.pdf for
%C details.

%D modes: reverse

%D This is a nice and simple style, written in februari
%D 2001. It uses a square papersize, derived from \type {S4}.
%D Because this style is meant to be used with Zapf
%D Chancery, I dedicate this style to Volker Schaa, a fan of
%D Zapf.

\setuppapersize
  [S44][S44]

\startmode[asintended]
  \definetypeface[zaphy][cg][calligraphy][chancery]
  \setupbodyfont[zaphy,cg,12pt]
\stopmode

\startnotmode[asintended]
  \setupbodyfont[13pt]
\stopnotmode

%D We use the whole page.

\setuplayout
  [backspace=0pt,
   topspace=0pt,
   header=0pt,
   footer=0pt,
   bottom=0pt,
   width=middle,
   height=middle]

%D We will be very tolerant in alignment.

\setuptolerance
  [verytolerant,stretch]

%D Of course use navigation, but we hide the in this case
%D ugly reverse video hyper spot.

\setupinteraction
  [state=start,
   color=white,
   contrastcolor=white,
   style=\underbar,
   click=no]

%D This style looks best in a dark room, full screen.

\setupinteractionscreen
  [option=max]

%D We use colors and remap a couple of standard colors.

\setupcolors
  [state=start]

\definecolor[white] [s=.8]
\definecolor[red]   [r=.7]
\definecolor[green] [g=.7]
\definecolor[blue]  [b=.7]
\definecolor[yellow][r=.7,g=.7]

\definecolor [PageColor][black]

%D These colors will cyclic be assigned to \type
%D {TextColor}.

\definecolor [TextColor 0][white]
\definecolor [TextColor 1][red]
\definecolor [TextColor 2][green]
\definecolor [TextColor 3][blue]
\definecolor [TextColor 4][yellow]

\definecolor [TextColor] [TextColor 0]

%D We will collect everything in a layer.

\definelayer
  [main]
  [state=repeat]

%D We have quite some overlays.

\defineoverlay [page] [\reuseMPgraphic{page}]
\defineoverlay [text] [\useMPgraphic{text}]
\defineoverlay [next] [\overlaybutton{nextpage}]
\defineoverlay [prev] [\overlaybutton{previouspage}]
\defineoverlay [main] [\composedlayer{main}]

%D These end up as paper, page and text backgrounds. We need
%D to locate the foreground, otherwise hyperlinks will not
%D work.

\setupbackgrounds % otherwise in acrobat 5 rounding error
  [paper]         % and one pixel white line
  [backgroundcolor=Pagecolor,
   background=page]

\setupbackgrounds
  [page]
  [background={page,prev,foreground,main}]

\setupbackgrounds
  [text]
  [background=next,
   backgroundoffset=-10pt]

%D This means that clicking on the center brings you to the
%D next page, while clicking on teh page frame brings you one
%D page back.

%D As usual, the graphics are handled by \METAPOST:

\startuseMPgraphic{text}
  path p ; p := unitsquare xyscaled (OverlayWidth,OverlayHeight) ;
  color c ; c := (.7+uniformdeviate.3)*\MPcolor{TextColor} ;
  p := p enlarged -1.25pt ;
  filldraw p withcolor c ;
  draw p withpen pencircle scaled 2.5pt withcolor .75c ;
\stopuseMPgraphic

\startreusableMPgraphic{page}
  path p ; p := unitsquare xyscaled (OverlayWidth,OverlayHeight) ;
  color c ; c := \MPcolor{PageColor} ;
  filldraw p enlarged 5pt withcolor c ; % bleeding
\stopreusableMPgraphic

%D The text is typeset in a framed text. We cycle trough the
%D colors by means of a counter. This counter also determines
%D the positioning on the main layer. The width is slightly
%D random.

\newcounter\KindOfTopic  % and cycle through corners
\newdimen  \TopicWidth   % with randomized widths

\defineframedtext
  [TopicText]
  [frame=off,
   offset=10pt,
   style=bold,
   width=\TopicWidth,
   background=text,
   before=,
   after=,
   align=normal]

\def\BeforeTopic
  {\ifcase\KindOfTopic\relax
     \TopicWidth=.7\textwidth
     \definecolor[CharColor][black]
   \else
     \getrandomdimen\TopicWidth{.55\textwidth}{.7\textwidth}
     \definecolor[CharColor][white]
   \fi
   \doifmode{reverse}
     {\setupframedtexts[TopicText][foregroundcolor=CharColor]}
   \definecolor[TextColor][TextColor \KindOfTopic]
   \ifcase\KindOfTopic\relax
     \setuplayer[main][x=.5\textwidth,y=.5\textheight,location=c]  \or
     \setuplayer[main][x=0pt,         y=0pt,          location=rb] \or
     \setuplayer[main][x=\textwidth,  y=0pt,          location=lb] \or
     \setuplayer[main][x=\textwidth,  y=\textheight,  location=lt] \or
     \setuplayer[main][x=0pt,         y=\textheight,  location=rt] \fi}

\def\AfterTopic
  {\ifnum\KindOfTopic=4
     \gdef\KindOfTopic{1}
   \else
     \doglobal\increment\KindOfTopic
   \fi}

\def\StartTopic
  {\BeforeTopic
   \startstandardmakeup
   \setlayer[main] \bgroup \startTopicText[none]
     }%\setupwhitespace[big]} % generates an empty line

\def\StopTopic
  {\stopTopicText \egroup
   \stopstandardmakeup
   \AfterTopic}

%D The title and colofon page are centered on the page.

\def\StartNopic
  {\doglobal\newcounter\KindOfTopic % centered at the page
   \StartTopic
     \bfd\setupinterlinespace
     \setupinteraction[color=,contrastcolor=]%
     \def\\{\blank\bfb\setupinterlinespace\def\\{\blank}}%
     \raggedcenter\ignorespaces}

\def\StopNopic
  {\StopTopic}

\let\StartTitlePage\StartNopic \let\StartColofonPage\StartNopic
\let\StopTitlePage \StopNopic  \let\StopColofonPage \StopNopic

\def\TitlePage  #1{\StartTitlePage  #1\StopTitlePage}
\def\ColofonPage#1{\StartColofonPage#1\StopColofonPage}

%D We provide a minimum of title commands.

\definehead
  [Title]
  [title]

\definehead
  [Subject]
  [subject]

\setuphead
  [Title]
  [style=\bfb,
   page=,
   before=,
   after=\blank]

\setuphead
  [Subject]
  [style=\bfa,
   before=\blank,
   after=\blank]

\doifnotmode{demo}{\endinput}

\def\Sample #1 {\input #1 \par \rightaligned{--- #1 ---}}

\starttext

\StartNopic The \ConTeXt\ Test Quotes \\ \currentdate \StopNopic

\StartTopic \Sample tufte   \StopTopic
\StartTopic \Sample knuth   \StopTopic
\StartTopic \Sample zapf    \StopTopic
\StartTopic \Sample douglas \StopTopic
\StartTopic \Sample stork   \StopTopic
\StartTopic \Sample materie \StopTopic

\StartNopic There Will Be Some More \StopNopic

\stoptext
