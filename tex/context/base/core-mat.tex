%D \module
%D   [       file=core-mat,
%D        version=1998.12.07,
%D          title=\CONTEXT\ Core Macros,
%D       subtitle=Math Fundamentals,
%D         author=Hans Hagen,
%D           date=\currentdate,
%D      copyright={PRAGMA / Hans Hagen \& Ton Otten}]
%C
%C This module is part of the \CONTEXT\ macro||package and is
%C therefore copyrighted by \PRAGMA. See mreadme.pdf for 
%C details. 

% engels maken

\writestatus{loading}{Context Core Macros / Math Fundamentals}

\unprotect

\def\mathortext
  {\ifmmode
     \expandafter\firstoftwoarguments
   \else
     \expandafter\secondoftwoarguments
   \fi}

% force text mode, will be overloaded later 

\ifx\text\undefined \let\text\hbox \fi 

% will move to page-ini 

\newevery \everybeginofpar \EveryBeginOfPar
\newevery \everyendofpar   \EveryEndOfPar

\def\bpar{\the\everybeginofpar}
\def\epar{\the\everyendofpar\endgraf}

\newdimen\lastlinewidth 

\def\setlastlinewidth
  {\resetlastlinewidth
   \ifmmode\else\ifhmode\else\ifoptimizedisplayspacing
     \bgroup
     \forgetdisplayskips
     $$\global\lastlinewidth\predisplaysize$$
     \vskip-\baselineskip
     \egroup
   \fi\fi\fi}

\def\resetlastlinewidth
  {\global\lastlinewidth\zeropoint\relax}

\appendtoks \setlastlinewidth \to \everyendofpar

%D moved from main-001

%\def\EveryMathPar{\EveryPar}
%
%\newevery \everymath \EveryMath

\abovedisplayskip      = \!!zeropoint
\abovedisplayshortskip = \!!zeropoint % evt. 0pt minus 3pt
\belowdisplayskip      = \!!zeropoint
\belowdisplayshortskip = \!!zeropoint % evt. 0pt minus 3pt

\predisplaypenalty     = 0
\postdisplaypenalty    = 0  % -5000 gaat mis, zie penalty bij \paragraaf

% we don't use the skip's

\def\displayskipsize#1#2% obsolete
  {\ifdim\tussenwit>\!!zeropoint
     #1\tussenwit\!!plus#2\tussenwit\!!minus#2\tussenwit\relax
   \else
     #1\lineheight\!!plus#2\lineheight\!!minus#2\lineheight\relax
   \fi}

\def\displayskipfactor          {1.0} % obsolete
\def\displayshortskipfactor     {0.8} % obsolete
\def\displayskipgluefactor      {0.3} % obsolete
\def\displayshortskipgluefactor {0.2} % obsolete

\def\abovedisplayskipsize% obsolete
  {\displayskipsize\displayskipfactor\displayskipgluefactor}

\def\belowdisplayskipsize% obsolete
  {\displayskipsize\displayskipfactor\displayskipgluefactor}

\def\abovedisplayshortskipsize% obsolete
  {\displayskipsize\displayshortskipfactor\displayshortskipgluefactor}

\def\belowdisplayshortskipsize% obsolete
  {\displayskipsize\displayshortskipfactor\displayshortskipgluefactor}

\def\setdisplayskip#1#2#3% obsolete
  {#1=#2\relax
   \advance#1 -\parskip
   \advance#1 -#3\relax}

\def\setdisplayskips % obsolete
  {\setdisplayskip\abovedisplayskip     \abovedisplayskipsize     \!!zeropoint
   \setdisplayskip\belowdisplayskip     \belowdisplayskipsize     \!!zeropoint
   \setdisplayskip\abovedisplayshortskip\abovedisplayshortskipsize\baselineskip
   \setdisplayskip\belowdisplayshortskip\belowdisplayshortskipsize\baselineskip}

% so far for unused stuff

\def\forgetdisplayskips % to do 
  {\abovedisplayskip     \zeropoint
   \belowdisplayskip     \zeropoint
   \abovedisplayshortskip\zeropoint
   \belowdisplayshortskip\zeropoint}

\definieernummer % \doorlabelen
  [\v!formule]
  [\c!tekst=\v!formule,
   \c!wijze=\@@fmwijze,
   \c!blokwijze=\@@fmblokwijze,
   \c!plaats=\v!intekst]

\def\setupformulas
  {\dodoubleargument\getparameters[\??fm]}

\newconditional\handleformulanumber
\newconditional\incrementformulanumber

\def\dododoformulenummer#1#2#3#4% (#1,#2)=outer(ref,sub) (#3,#4)=inner(ref,sub)
  {\hbox\bgroup
   \ifconditional\handleformulanumber
     \ifconditional\incrementformulanumber
       \verhoognummer[\v!formule]%
     \fi
     \maakhetnummer[\v!formule]%
     \setbox0\hbox{\ignorespaces#2\unskip}%
     \ifdim\wd0>\zeropoint
       \edef\hetsubnummer{#2}%
     \else
       \let\hetsubnummer\empty
     \fi
     \doifsomething{#1}{\rawreference{\s!for}{#1}{\hetnummer\hetsubnummer}}%
     \setbox0\hbox{\ignorespaces#4\unskip}%
     \ifdim\wd0>\zeropoint
       \edef\hetsubnummer{#4}%
     \fi
     \doifsomething{#3}{\rawreference\s!for{#3}{\hetnummer\hetsubnummer}}%
     \rm % nodig ? 
     \@@fmnummercommando
       {\dostartattributes\??fm\c!nummerletter\c!nummerkleur
        \strut
        \@@fmlinks
        \preparethenumber\??fm\hetnummer\preparednumber
        \labeltexts\v!formule
          {\ignorespaces\preparednumber\ignorespaces\hetsubnummer\unskip}%
        \@@fmrechts
        \dostopattributes}%
   \fi
   \egroup}

\def\dodoformulenummer[#1][#2][#3]%
  {\doquadruplegroupempty\dododoformulenummer{#1}{#2}{#3}}

\def\doformulenummer
  {\dotripleempty\dodoformulenummer}

\setvalue{\e!start\v!formule}{\dostartformula{}}
\setvalue{\e!stop \v!formule}{\dostopformula}

\def\definieerformule%
  {\dodoubleempty\dodefinieerformule}

\def\dodefinieerformule[#1][#2]%
  {\doifsomething{#1}
     {\copyparameters
        [\??fm#1][\??fm]
        [\c!voorwit,\c!nawit,\c!grid,
         \c!linkermarge,\c!rechtermarge,\c!marge,
         \c!springvolgendein,\c!variant,
         \c!strut,\c!uitlijnen,\c!afstand]%
      \setupformulas[#1][#2]%
      \setvalue{\e!start#1\v!formule}{\dostartformula{#1}}%
      \setvalue{\e!stop #1\v!formule}{\dostopformula}}}

\def\setupformulas
  {\dodoubleempty\dosetupformulas}

\def\dosetupformulas[#1][#2]%
  {\ifsecondargument
     \getparameters[\??fm#1][#2]%
   \else
     \getparameters[\??fm][#1]%
   \fi}

\setupformulas
  [\c!wijze=\@@nrwijze,
   \c!blokwijze=,
   \c!sectienummer=\@@nrsectienummer,
   \c!plaats=\v!rechts,
   \c!links=(,
   \c!rechts=),
   \c!voorwit=,
   \c!nawit=\@@fmvoorwit,
   \c!linkermarge=\!!zeropoint,
   \c!rechtermarge=\!!zeropoint,
   \c!marge=,
   \c!springvolgendein=\v!nee,
   \c!variant=\s!default,
   \c!uitlijnen=,
   \c!strut=\v!nee,
   \c!scheider=\@@koscheider,
   \c!afstand=1em]

\def\currentformula          {}
\def\predisplaysizethreshhold{3em}

\def\leftdisplayskip    {\leftskip}
\def\rightdisplayskip   {\rightskip}
\def\leftdisplaymargin  {\getvalue{\??fm\currentformula\c!linkermarge}}
\def\rightdisplaymargin {\getvalue{\??fm\currentformula\c!rechtermarge}}
\def\displaygridsnapping{\getvalue{\??fm\currentformula\c!grid}}

\def\beforedisplayspace
  {\doifnotvalue{\??fm\currentformula\c!voorwit}\v!geen
     {\blanko[\getvalue{\??fm\currentformula\c!voorwit}]}}

\def\afterdisplayspace 
  {\doifnotvalue{\??fm\currentformula\c!nawit}\v!geen
     {\blanko[\getvalue{\??fm\currentformula\c!nawit}]}}

\def\setpredisplaysize#1%
  {\predisplaysize#1\relax
   \ifdim\predisplaysize>\zeropoint
     \advance\predisplaysize \predisplaysizethreshhold
   \fi
   \advance\predisplaysize \displayindent % needed ? 
   \ifdim\predisplaysize>\hsize
     \predisplaysize\hsize
   \fi}

\def\setdisplaydimensions
  {\displayindent\leftdisplayskip 
   \advance\displayindent\leftdisplaymargin 
   \displaywidth\hsize
   \ifdim\hangindent>\zeropoint
     \advance\displayindent\hangindent
   \else
     \advance\displaywidth\hangindent
   \fi
   \advance\displaywidth-\displayindent
   \advance\displaywidth-\rightdisplayskip
   \advance\displaywidth-\rightdisplaymargin}

\newif\ifoptimizedisplayspacing

\def\dostartformula#1%
  {\dodoubleempty\dodostartformula[#1]}

\def\dodostartformula[#1][#2]% setting leftskip adaption is slow ! 
  {\bgroup 
   \switchtoformulabodyfont[#2]%
   \def\currentformula{#1}%
   \doifvaluesomething{\??fm\currentformula\c!marge}% so we test first 
     {\dosetleftskipadaption{\getvalue{\??fm\currentformula\c!marge}}%
      \edef\leftdisplaymargin{\the\leftskipadaption}}% overloaded
   \long\def\dostartformula##1{\bgroup\let\dostopformula\egroup}%
   \freezedimenmacro\leftdisplayskip
   \freezedimenmacro\rightdisplayskip
   \freezedimenmacro\leftdisplaymargin
   \freezedimenmacro\rightdisplaymargin
   \freezedimenmacro\predisplaysizethreshhold
   \forgetdisplayskips
   \let\setdisplayskips\relax
   \ifoptimizedisplayspacing
     \ifdim\lastlinewidth>\zeropoint
       \abovedisplayshortskip-\strutht\relax
     \fi
   \else
     \resetlastlinewidth
   \fi
   \getvalue{\e!start\getvalue{\??fm\currentformula\c!variant}\v!formule}}

\def\switchtoformulabodyfont{\switchtobodyfont}

\setvalue{\v!formule}{\dosingleempty\doformula}

\def\doformula[#1]#2%
  {\begingroup
   \switchtoformulabodyfont[#1]%
   % not : \def\doformula[##1]##2{\mathematics{##2}}%
   \mathematics{#2}%   
   \endgroup}

\let\doplaceformulanumber\empty

\def\dostopformula
  {\doplaceformulanumber
   \getvalue{\e!stop\getvalue{\??fm\currentformula\c!variant}\v!formule}%
    \resetlastlinewidth
    \nonoindentation
    \dochecknextindentation{\??fm\currentformula}%
    \egroup}

\newif\ifinformula

\def\startdisplaymath
  {\ifgridsnapping
     \beforedisplayspace   
     \snapmathtogrid\vbox
     \bgroup 
     \informulatrue   
    %\forgetall % breaks side floats 
   \else
     \bgroup
     \informulatrue   
    %\forgetall % otherwise backgrounds fail 
     \ifdim\lastskip<\!!zeropoint\else
       \par
       \ifvmode \ifdim\parskip>\zeropoint\relax
         \vskip-\parskip
       \fi \fi
     \fi
     \beforedisplayspace   
     \par
     \ifvmode
       \verticalstrut
       \vskip-\struttotal
       \vskip-\baselineskip
     \fi 
   \fi
   $$\setdisplaydimensions
   \setpredisplaysize\lastlinewidth
   \startinnermath}

\def\stopdisplaymath
  {\stopinnermath
   $$%
   \ifgridsnapping 
     \egroup
     \afterdisplayspace
   \else
     \par\ifvmode\ifdim\parskip>\zeropoint\vskip-\parskip\fi\fi
     \afterdisplayspace
     \egroup
   \fi} 

\newif\ifclipdisplaymath \clipdisplaymathtrue 
\def\displaymathclipfactor{1.1}

\def\snapmathtogrid % to do \dp 
  {\dowithnextbox
     {\bgroup
      \donefalse
      \ifclipdisplaymath 
        \ifdim\nextboxht<\displaymathclipfactor\lineheight 
          \donetrue
        \fi 
      \fi
      \ifdone
        \nextboxht\lineheight 
      \else
        \getnoflines\nextboxht
        \setbox\nextbox\vbox to \noflines\lineheight
          {\vfill\flushnextbox\vfill}%
        \setbox\nextbox\hbox{\lower\strutdepth\flushnextbox}%
      \fi
      \snaptogrid[\displaygridcorrection]\hbox{\flushnextbox}%
      \gdef\displaygridcorrection{\displaygridsnapping}%
      \egroup}}

\def\displaygridcorrection{\displaygridsnapping}

\def\moveformula
  {\dosingleempty\domoveformula}

\def\domoveformula[#1]% brr gaat mogelijk fout 
  {\ifgridsnapping
     \iffirstargument
       \xdef\displaygridcorrection{#1}%
     \else
       \gdef\displaygridcorrection{-\v!boven}% handy with short preline
     \fi
   \else
     \gdef\displaygridcorrection{\displaygridsnapping}%
   \fi}

\let\startinnermath\empty
\let\stopinnermath \empty

\def\defineformulaalternative
  {\dotripleargument\dodefineformulaalternative}

\def\dodefineformulaalternative[#1][#2][#3]%  
  {\setvalue{\e!start#1\v!formule}{#2}%
   \setvalue{\e!stop #1\v!formule}{#3}}

\defineformulaalternative[\s!default][\startdisplaymath][\stopdisplaymath]

% sp = single line paragraph  sd = single line display 
% mp = multi  line paragraph  md = multy  line display 

\defineformulaalternative[single][\startdisplaymath][\stopdisplaymath]
\defineformulaalternative[multi] [\startdisplaymath][\stopdisplaymath]

\definieerformule
  [sp]
  [\c!voorwit=\v!geen,\c!nawit=\v!geen,
   \c!springvolgendein=\v!nee,
   \c!variant=single]

\definieerformule
  [sd]
  [\c!voorwit=\v!geen,\c!nawit=\v!geen,
   \c!springvolgendein=\v!ja,
   \c!variant=single]

\definieerformule
  [mp]
  [\c!springvolgendein=\v!nee,
   \c!variant=multi]

\definieerformule
  [md]
  [\c!springvolgendein=\v!ja,
   \c!variant=multi]

% in m-math
%
% \defineformulaalternative[multi][\begindmath][\enddmath]
% 
% \fakewords{20}{40}\epar 
% \plaatsformule {a} $$              \fakespacingformula $$
% \fakewords{20}{40}\epar 
% \plaatsformule {b} \startformule   \fakespacingformula \stopformule
% \plaatsformule {b} \startformule   \fakespacingformula \stopformule
% \fakewords{20}{40}\epar 
% \plaatsformule {c} \startmdformule \fakespacingformula \stopmdformule
% \plaatsformule {c} \startmdformule \fakespacingformula \stopmdformule
% \fakewords{20}{40}\epar 
% \plaatsformule {d} \startmpformule \fakespacingformula \stopmpformule
% \plaatsformule {d} \startmpformule \fakespacingformula \stopmpformule
% \fakewords{20}{40}\epar 
% \plaatsformule {e} \startsdformule \fakespacingformula \stopsdformule
% \plaatsformule {e} \startsdformule \fakespacingformula \stopsdformule
% \fakewords{20}{40}\epar 
% \plaatsformule {f} \startspformule \fakespacingformula \stopspformule
% \plaatsformule {f} \startspformule \fakespacingformula \stopspformule
% \fakewords{20}{40}

% \convertcommand\next\to\ascii \getfirstcharacter\ascii
% \ifx\firstcharacter\letterbackslash % a \cs
 
\def\plaatsformule
  {\settrue\incrementformulanumber
   \dodoubleempty\doplaatsformule}

\def\plaatssubformule
  {\setfalse\incrementformulanumber
   \dodoubleempty\doplaatsformule}

\def\doplaatsformule[#1][#2]% #2 = dummy, gobbles spaces
  {\def\redoplaatsformule
     {\bgroup\def\dostartformula####1{\relax}%
      \ifx\next\bgroup
        \@EA\moreplaatsformule % [ref]{}
      \else
        \expandafter\convertargument\e!start\to\asciiA
        \expandafter\convertargument\next   \to\asciiB
        \ExpandBothAfter\doifincsnameelse\asciiA\asciiB
          {\egroup \@EA\dodoplaatsformule}% [ref]\start
          {\egroup \@EA\dispplaatsformule}% [ref]$$
      \fi[#1]{}}%
   \futurelet\next\redoplaatsformule}

\long\def\moreplaatsformule[#1]#2#3#4% #2 dummy #4 gobbles spaces
  {\def\redoplaatsformule
     {\expandafter\convertargument\e!start\to\asciiA
      \expandafter\convertargument\next   \to\asciiB
      \ExpandBothAfter\doifincsnameelse\asciiA\asciiB
        {\egroup \dodoplaatsformule}% [ref]\start
        {\egroup \dispplaatsformule}% [ref]$$
      [#1]{#3}}%
   \futurelet\next\redoplaatsformule#4}

\def\dispplaatsformule[#1]#2$$#3$$%
  {\dodoplaatsformule[#1]{#2}\dostartformula{}#3\dostopformula}

\let\normalreqno\eqno
\let\normalleqno\leqno

\def\dodoplaatsformule[#1]#2% messy, needs a clean up 
  {\doifelse{#1}{-}
     {\setfalse\handleformulanumber}
     {\doifelse{#2}{-}
        {\setfalse\handleformulanumber}
        {\settrue\handleformulanumber}}%
   \ifconditional\handleformulanumber
     \def\formulenummer
       {%\global\let\subformulenummer\doformulenummer % no, bug 
        \doformulenummer[#1][#2]}%
     \def\subformulenummer
       {\setfalse\incrementformulanumber
        \formulenummer}%
     \gdef\doplaceformulanumber
       {\global\let\doplaceformulanumber\empty
        \doifelse\@@fmplaats\v!links
          {\normalleqno{\doformulenummer[#1][#2][]{}}}
          {\normalreqno{\doformulenummer[#1][#2][]{}}}}%
   \else
     \def\formulenummer{\doformulenummer[#1][#2]}%
     \global\let\subformulenummer\doformulenummer
     \global\let\doplaceformulanumber\empty
   \fi} 

%D We need a hook into the plain math alignment macros
%D
%D \starttypen
%D \displaylines
%D \eqalignno
%D \eqalignno
%D \stoptypen
%D
%D Otherwise we get a missing \type {$$} error reported.

\def\resetdisplaymatheq
  {\let\normalleqno\relax \let\leqno\relax
   \let\normalreqno\relax \let\eqno \relax
   \let\doplaceformulanumber\relax}

\let\normaldispl@y\displ@y

\def\displ@y{\resetdisplaymatheq\normaldispl@y}

%D Here we implement a basic math alignment mechanism. Numbers
%D are also handled. The macros \type {\startinnermath} and
%D \type {\stopinnermath} can be overloaded in specialized
%D modules. 

\def\startinnermath
  {\getvalue{\e!start\??fm\getvalue{\??fm\currentformula\c!uitlijnen}}}

\def\stopinnermath
  {\getvalue{\e!stop \??fm\getvalue{\??fm\currentformula\c!uitlijnen}}}

\def\mathinnerstrut
  {\doifvalue{\??fm\currentformula\c!strut}\v!ja\strut}

\long\def\defineinnermathhandler#1#2#3%
  {\setvalue{\e!start\??fm#1}{#2}%
   \setvalue{\e!stop \??fm#1}{#3}}

\newif\iftracemath 

\def\mathhbox
  {\iftracemath\ruledhbox\else\hbox\fi}  

\def\startmathbox#1#2%
  {\hsize\displaywidth
   \global\let\@eqno \empty \def\eqno {\gdef\@eqno }%
   \global\let\@leqno\empty \def\leqno{\gdef\@leqno}%
   % added 
   \let\normalreqno\eqno
   \let\normalleqno\leqno
   % added 
   \doplaceformulanumber
   \mathhbox to \displaywidth\bgroup
     \mathinnerstrut
     $\displaystyle
     \ifx\@leqno\empty\else \ifcase#2 
       \rlap{\@leqno}%
     \else
       \@leqno\hskip\getvalue{\??fm\currentformula\c!afstand}%
     \fi \fi 
     #1}

\def\stopmathbox#1#2%
  {$#1%
   \ifx\@eqno\empty\else \ifcase#2 
     \llap{\@eqno}%
   \else
     \hskip\getvalue{\??fm\currentformula\c!afstand}\@eqno
   \fi \fi 
   \egroup}

\defineinnermathhandler\v!rechts{\startmathbox\empty1}{\stopmathbox\hfill0}
\defineinnermathhandler\v!links {\startmathbox\hfill0}{\stopmathbox\empty1}
\defineinnermathhandler\v!midden{\startmathbox\hfill0}{\stopmathbox\hfill0}

%D [The examples below are in english and don't process in the 
%D  documentation style, which will be english some day.]
%D 
%D Normally a formula is centered, but in case you want to
%D align it left or right, you can set up formulas to behave
%D that way. Normally a formula will adapt is left indentation
%D to the environment: 
%D 
%D \startbuffer
%D \fakewords{20}{40}\epar 
%D \startitemize
%D \item \fakewords{20}{40}\epar 
%D       \placeformula \startformula \fakeformula \stopformula
%D \item \fakewords{20}{40}\epar 
%D \stopitemize
%D \fakewords{20}{40}\epar 
%D \stopbuffer
%D 
%D % \getbuffer
%D 
%D In the next examples we explicitly align formulas to the
%D left (\type {\raggedleft}), center and right (\type
%D {\raggedright}): 
%D 
%D \startbuffer
%D \setupformulas[align=left]   
%D \startformula\fakeformula\stopformula
%D \setupformulas[align=middle] 
%D \startformula\fakeformula\stopformula
%D \setupformulas[align=right]  
%D \startformula\fakeformula\stopformula
%D \stopbuffer
%D 
%D \typebuffer
%D 
%D Or in print: 
%D 
%D % {\getbuffer}
%D 
%D With formula numbers these formulas look as follows: 
%D 
%D \startbuffer
%D \setupformulas[align=left]   
%D \placeformula \startformula\fakeformula\stopformula
%D \setupformulas[align=middle] 
%D \placeformula \startformula\fakeformula\stopformula
%D \setupformulas[align=right]  
%D \placeformula \startformula\fakeformula\stopformula
%D \stopbuffer
%D 
%D % {\getbuffer}
%D 
%D This was keyed in as: 
%D 
%D \typebuffer
%D 
%D When tracing is turned on (\type {\tracemathtrue}) you can 
%D visualize the bounding box of the formula, 
%D 
%D % {\tracemathtrue\getbuffer}
%D 
%D As you can see, the dimensions are the natural ones, but if 
%D needed you can force a normalized line: 
%D 
%D \startbuffer
%D \setupformulas[strut=yes]
%D \placeformula \startformula \fakeformula \stopformula
%D \stopbuffer
%D 
%D \typebuffer
%D 
%D This time we get a more spacy result. 
%D 
%D % {\tracemathtrue\getbuffer}
%D 
%D We will now show a couple of more settings and combinations
%D of settings. In centered formulas, the number takes no space
%D 
%D \startbuffer
%D \setupformulas[align=middle]
%D \startformula \fakeformula \stopformula
%D \placeformula \startformula \fakeformula \stopformula
%D \stopbuffer
%D 
%D \typebuffer % {\tracemathtrue\getbuffer}
%D 
%D You can influence the placement of the whole box with the
%D parameters \type {leftmargin} and \type {rightmargin}. 
%D 
%D \startbuffer
%D \setupformulas[align=right,leftmargin=3em]
%D \startformula \fakeformula \stopformula
%D \placeformula \startformula \fakeformula \stopformula
%D 
%D \setupformulas[align=left,rightmargin=1em]
%D \startformula \fakeformula \stopformula
%D \placeformula \startformula \fakeformula \stopformula
%D \stopbuffer
%D 
%D \typebuffer % {\tracemathtrue\getbuffer}
%D 
%D You can also inherit the margin from the environment. 
%D 
%D \startbuffer
%D \setupformulas[align=right,margin=standard]
%D \startformula \fakeformula \stopformula
%D \placeformula \startformula \fakeformula \stopformula
%D \stopbuffer
%D 
%D \typebuffer % {\tracemathtrue\getbuffer}
%D 
%D The distance between the formula and the number is only
%D applied when the formula is left or right aligned. 
%D 
%D \startbuffer
%D \setupformulas[align=left,distance=2em]
%D \startformula \fakeformula \stopformula
%D \placeformula \startformula \fakeformula \stopformula
%D \stopbuffer
%D 
%D \typebuffer % {\tracemathtrue\getbuffer}

%D \macros 
%D   {big..}
%D
%D Because they are related to the bodyfontsize, we redefine 
%D some \PLAIN\ macros. 

\def\@@dobig#1#2%
  {{\hbox{$\left#2\vbox\!!to#1\bodyfontsize{}\right.\n@space$}}}

\def\big {\@@dobig{0.85}}
\def\Big {\@@dobig{1.15}}
\def\bigg{\@@dobig{1.45}}
\def\Bigg{\@@dobig{1.75}}

%D \macros 
%D   {bordermatrix}
%D
%D We already redefined \type {\bordermatrix} in \type 
%D {font-ini}.

%D \macros 
%D   {setuptextformulas}
%D
%D This command sets up in||line math. Most features deals 
%D with grid snapping and are experimental. 

\newevery \everysetuptextformulas \relax 

\def\setuptextformulas
  {\dosingleempty\dosetuptextformulas}

\def\dosetuptextformulas[#1]%
  {\getparameters[\??mt][#1]%   
   \the\everysetuptextformulas}

%D \macros
%D   {super, sub}
%D
%D \TEX\ uses \type{^} and \type{_} for entering super- and 
%D subscript mode. We want however a bit more control than 
%D normally provided, and therefore provide \type {\super} 
%D and \type{sub}. 

\global\let\normalsuper=^ 
\global\let\normalsuber=_ 

\newcount\supersubmode

\newevery\everysupersub \EverySuperSub

\appendtoks \advance\supersubmode \plusone \to \everysupersub

% \def\dodosuper#1{\normalsuper{\the\everysupersub#1}}
% \def\dodosuber#1{\normalsuber{\the\everysupersub#1}}
% 
% \def\dosuper{\ifx\next\bgroup\expandafter\dodosuper\else\normalsuper\fi}
% \def\dosuber{\ifx\next\bgroup\expandafter\dodosuber\else\normalsuber\fi}
% 
% \def\super{\futurelet\next\dosuper}
% \def\suber{\futurelet\next\dosuber}
%
% \def\super#1{\normalsuper{\the\everysupersub#1}}
% \def\suber#1{\normalsuber{\the\everysupersub#1}}

\appendtoks 
  \gridsupsubstyle 
\to \everysupersub

\appendtoks 
  \doifelse\@@mtformaat\v!klein
    {\let\gridsupsubstyle    \scriptscriptstyle 
     \let\gridsupsubbodyfont \setsmallbodyfont}%
    {\let\gridsupsubstyle    \scriptstyle
     \let\gridsupsubbodyfont \relax}%
\to \everysetuptextformulas 

\setuptextformulas 
  [\c!formaat=\v!normaal]

\def\dogridsupsub#1#2%
  {\begingroup
   \setbox\nextbox\iftracegridsnapping\ruledhbox\else\hbox\fi
     {\gridsupsubbodyfont
      $\strut^{\the\everysupersub#1}_{\the\everysupersub#2}$}%
   \nextboxht\strutheight
   \nextboxdp\strutdepth
   \flushnextbox
   \endgroup}

\def\gridsupsub
  {\ifconditional\crazymathsnapping  
     \ifgridsnapping
       \@EAEAEA\dogridsupsub
     \else
       \@EAEAEA\normalsupsub
     \fi
   \else
     \@EA\normalsupsub
   \fi}

\def\normalsupsub#1#2%
  {^{\the\everysupersub#1}_{\the\everysupersub#2}}

\appendtoks 
  \let\gridsupsubstyle   \relax
  \let\gridsupsubbodyfont\relax
  \let\gridsupsub        \normalsupsub
\to \everydisplay

\def\super#1{^{\the\everysupersub#1}}
\def\suber#1{_{\the\everysupersub#1}}
\def\supsub#1#2{\super{#1}\suber{#2}}
\def\subsup#1#2{\suber{#1}\super{#2}}

%\def\super#1{\gridsupsub{#1}{}} % 
%\def\suber#1{\gridsupsub{}{#1}} % 
%
%\def\supsub#1#2{\gridsupsub{#1}{#2}}
%\def\subsup#1#2{\gridsupsub{#2}{#1}}

\def\gridsuper#1{\gridsupsub{#1}{}}
\def\gridsuber#1{\gridsupsub{}{#1}}

% \let\sup\super % math char 
% \let\sub\suber

% test set: 
%
% \startbuffer
% \sform{x\frac{1}{2}}
% \sform{x\sup{\frac{1}{2}} + x\sup{2} + 2}
% \sform{x\supsub{\frac{1}{2}}{\frac{1}{2}} + x\sup{2} + 2}
% \stopbuffer
% 
% \typebuffer 
% 
% \startlines
% \getbuffer
% \stoplines
% 
% \startbuffer
% $x\frac{1}{2}$ 
% $x\sup{\frac{1}{2}} + x^2 + 2$
% $x\supsub{\frac{1}{2}}{\frac{1}{2}} + x^2 + 2$
% \stopbuffer
% 
% \typebuffer 
% 
% \start
% \enablesupersub 
% \enableautomath
% \startlines
% \getbuffer
% \stoplines
% \stop

%D \macros
%D   {enablesupersub}
%D 
%D We can let \type {^} and \type {_} act like \type {\super} 
%D and \type {\sub} by saying \type {\enablesupersub}. 

\bgroup
\catcode`\^=\@@active
\catcode`\_=\@@active
\gdef\enablesupersub
  {\catcode`\^=\@@active
   \def^{\ifmmode\expandafter\super\else\expandafter\normalsuper\fi}%
   \catcode`\_=\@@active
   \def_{\ifmmode\expandafter\suber\else\expandafter\normalsuber\fi}}
\egroup

%D \macros
%D   {enableautomath}
%D
%D The next one can be dangerous, but handy in controlled 
%D situations.  

\bgroup \catcode`\$=\active

\gdef\enableautomath
  {\catcode`\$=\active
   \def$##1${\snappedinlineformula{##1}}}

% \gdef\enableautomath
%   {\catcode`\$=\active
%    \def${\doifnextcharelse$\doautodmath\doautoimath}%
%    \def\doautoimath##1${\snappedinlineformula{##1}}%
%    \def\doautodmath$##1$${\startformula##1\stopformula}}

\egroup

%D \macros 
%D   {...} 
%D
%D New and experimental: snapping big inline math! 

\newconditional\halfcrazymathlines % \settrue\halfcrazymathlines 
\newconditional\crazymathsnapping  % \settrue\crazymathsnapping

\appendtoks 
 \doifelse\@@mtgrid\v!ja
   {\settrue \crazymathsnapping }{\setfalse\crazymathsnapping}%
 \doifelse\@@mtstap\v!halveregel
   {\settrue \halfcrazymathlines}{\setfalse\halfcrazymathlines}%
\to \everysetuptextformulas 

\setuptextformulas
  [\c!grid=\v!ja,
   \c!stap=\v!regel]

\newcount\crazymathhack

\let\lastcrazymathline     \!!zeropoint
\let\lastcrazymathpage     \!!zerocount
\let\lastcrazymathprelines \!!zerocount
\let\lastcrazymathpostlines\!!zerocount

\def\crazymathtag{amh:\the\crazymathhack}
\def\crazytexttag{\v!tekst:\lastcrazymathpage}

\def\crazymathindent{\hskip\MPx\crazymathtag\hskip-\MPx\crazytexttag}

\def\flushcrazymathbox
  {\nextboxht\strutheight
   \nextboxdp\strutdepth 
   \hbox{\iftracegridsnapping\ruledhbox\fi{\flushnextbox}}}

% possible pdftex bug: 
%
% \dorecurse{100}{gest \vadjust     {\strut} \par} \page 
% \dorecurse{100}{gest \vadjust pre {\strut} \par} \page 
%
% duplicate depth compensation with pre 

\def\snappedinlineformula
  {\dosingleempty\dosnappedinlineformula}

%D \starttabulatie[|Tl|l|]
%D \NC - \NC half lines        \NC \NR
%D \NC + \NC full lines        \NC \NR
%D \NC = \NC force             \NC \NR
%D \NC < \NC force, minus pre  \NC \NR
%D \NC > \NC force, minus post \NC \NR
%D \stoptabulatie

\newif\if!!donee
\newif\if!!donef

\def\inlinemathmargin{1pt}

\settrue\autocrazymathsnapping 

% FROM NOW ON, CHANGES AS OPTIONS 

% TODO: SKYLINE (PREV LINE POS SCAN) 

\def\dosnappedinlineformula[#1]#2%
  {\ifvmode\dontleavehmode\fi % tricky 
   \strut % prevents funny space at line break  
   \begingroup % interesting: \bgroup can make \vadjust disappear
   \ifconditional\crazymathsnapping  
     \ifgridsnapping
       \ifx\pdftexversion\undefined
         \donefalse
       \else
         \checktextbackgrounds % we need pos tracking, to be made less redundant 
         \donetrue
       \fi
     \else
       \donefalse
     \fi
   \else
     \donefalse
   \fi
   \!!doneafalse % forced or not auto 
   \!!donebfalse % too heigh
   \!!donecfalse % too low 
   \!!donedfalse % less before 
   \!!doneefalse % less after 
   \ifdone
     \setbox\nextbox\hbox{$#2$}% 
     \iftracegridsnapping
       \setbox\nextbox\ruledhbox
         {\incolortrue\localcolortrue
          \backgroundline[gray]{\showstruts\strut\flushnextbox}}%
     \fi
     \def\docommando##1%
       {\doif{##1}-{\settrue \halfcrazymathlines}%
        \doif{##1}+{\setfalse\halfcrazymathlines}%
        \doif{##1}={\!!doneatrue}%
        \doif{##1}<{\!!donedtrue}%
        \doif{##1}>{\!!doneetrue}}%
     \processcommalist[#1]\docommando
\if!!doneb 
  \if!!donec \else
    \setfalse\halfcrazymathlines
  \fi 
\else
  \if!!donec 
    \setfalse\halfcrazymathlines
  \fi
\fi 
     \donefalse
     \if!!donea
       \donetrue
\scratchdimen \nextboxht
\advance\scratchdimen .5\lineheight
\nextboxht\scratchdimen 
\scratchdimen \nextboxdp
\advance\scratchdimen .5\lineheight
\nextboxdp\scratchdimen 
     \else\ifdim\nextboxht>\strutht
       \donetrue 
     \else\ifdim\nextboxdp>\strutdp
       \donetrue 
     \fi\fi\fi
     \ifconditional\autocrazymathsnapping \else \if!!donea \else
       % don't compensate, just snap to strut 
       \donefalse
       % signal for next else, snap line to strut  
       \!!doneatrue 
     \fi \fi
   \fi
   \ifdone
     % analyze height 
     \scratchdimen\inlinemathmargin 
     \advance\scratchdimen \strutht
     \ifdim\nextboxht<\scratchdimen \else \!!donebtrue \fi
     % analyze depth 
     \scratchdimen\inlinemathmargin 
     \advance\scratchdimen \strutdp
     \ifdim\nextboxdp<\scratchdimen \else \!!donectrue \fi 
     % analyzed or forced 
     \ifdone
       \global\advance\crazymathhack\plusone
       \donefalse
       \ifnum\MPp\crazymathtag=\lastcrazymathpage\relax
         \ifdim\MPy\crazymathtag=\lastcrazymathline\relax
           \donetrue
         \fi
       \fi
       \ifnum\MPp\crazymathtag=\zerocount \donefalse \fi
       \ifdim\MPy\crazymathtag=\zeropoint \donefalse \fi
       \ifdone
         % same page and same line 
       \else
         \global\let\lastcrazymathprelines \!!zerocount
         \global\let\lastcrazymathpostlines\!!zerocount
         \xdef\lastcrazymathpage{\MPp\crazymathtag}%
         \xdef\lastcrazymathline{\MPy\crazymathtag}%
       \fi
       \if!!doneb
       % \getrawnoflines\nextboxht 
         \scratchdimen\nextboxht
         \advance\scratchdimen-\strutht
         \getnoflines\scratchdimen
         \if!!doned \advance\noflines\minusone \fi
         \scratchcounter\noflines
         \advance\noflines-\lastcrazymathprelines\relax
         \ifnum\noflines>\zerocount
           \xdef\lastcrazymathprelines{\the\scratchcounter}%
           \scratchdimen\noflines\lineheight 
           \ifconditional\halfcrazymathlines
             \advance\scratchdimen-.5\lineheight
           \fi
           \advance\scratchdimen-\strutdepth
           \setbox\scratchbox\null
           \wd\scratchbox2\bodyfontsize
           \ht\scratchbox\scratchdimen
           \dp\scratchbox\strutdepth 
           %%% top correction code (see below) 
           \normalvadjust pre 
             {%\allowbreak % sometimes breaks spacing 
              \forgetall
              \crazymathindent
              \iftracegridsnapping
                \setbox\scratchbox\hbox
                  {\incolortrue\localcolortrue\green
                   \ruledhbox{\box\scratchbox}}%
              \fi
              \box\scratchbox
              \endgraf
              \nobreak}%
         \else\ifnum\scratchcounter>\zerocount
           \normalvadjust pre 
             {\nobreak}%
         \fi\fi
       \fi
       \if!!donec
       % \getrawnoflines\nextboxdp 
         \scratchdimen\nextboxdp
         \advance\scratchdimen-\strutdp
         \getnoflines\scratchdimen 
         \if!!donee \advance\noflines\minusone \fi
         \scratchcounter\noflines
         \advance\noflines-\lastcrazymathpostlines\relax     
         \ifnum\noflines>\zerocount
           \donetrue
         \else\ifnum\lastcrazymathpostlines=\zerocount
           \donetrue
         \else
           \donefalse
         \fi\fi
       \else
         \donefalse
       \fi
       \ifdone
         \xdef\lastcrazymathpostlines{\the\scratchcounter}%
         \ifnum\lastcrazymathpostlines=\zerocount
           \global\let\lastcrazymathpostlines\!!plusone
         \fi
         \hbox{\setposition\crazymathtag\flushcrazymathbox}%
         \scratchdimen\noflines\lineheight 
         \advance\scratchdimen-\lineheight 
         \advance\scratchdimen+\strutheight
\ifdim\scratchdimen>\zeropoint \else
  \scratchdimen=\strutheight % todo : test for half lines 
\fi
         \ifconditional\halfcrazymathlines
           \advance\scratchdimen-.5\lineheight
         \fi
         \setbox\scratchbox\null
         \wd\scratchbox2\bodyfontsize
         \ht\scratchbox\scratchdimen
         \dp\scratchbox\strutdepth
         \normalvadjust 
           {\forgetall
            \crazymathindent
            \iftracegridsnapping
              \setbox\scratchbox\hbox
                {\incolortrue\localcolortrue\blue
                 \ruledhbox{\box\scratchbox}}%
            \fi
            \box\scratchbox
            \endgraf
            % precaution: else we stick below the text bottom
            \ifconditional\halfcrazymathlines
              \allowbreak
            \else
              \vskip-\lineheight
              \vskip \lineheight  
           \fi}% 
       \else
         \hbox{\setposition\crazymathtag\flushcrazymathbox}%
       \fi
     \else
       \flushcrazymathbox
     \fi
   \else\if!!donea
     \flushcrazymathbox
   \else
     \mathematics{#2}%     
   \fi\fi
   \endgroup}


%%% top correction code 
%%%
%%% correct for fuzzy top of page situations 
%
% \scratchdimen\lastcrazymathprelines\lineheight
% \advance\scratchdimen\MPy\crazymathtag
% \advance\scratchdimen\lineheight
% \advance\scratchdimen\topskip
% \advance\scratchdimen-\strutheight
% \dimen0=\MPy\crazytexttag
% \advance\dimen0 \MPh\crazytexttag
% \advance\scratchdimen-\dimen0\relax
% % do we need correction at all 
% \ifdim\scratchdimen>\strutdepth\relax
%   \donefalse
% \else\ifdim\scratchdimen<\zeropoint
%   \donefalse
% \else
%   \donetrue
% \fi\fi 
% % analysis done 
% \donefalse
% \ifdone
%   \edef\crazymathcorrection{\the\scratchdimen}%
%   \advance\scratchdimen-\dp\scratchbox
%   \dp\scratchbox-\scratchdimen 
% \else
%   \let\crazymathcorrection\zeropoint
% \fi 
%
%%%
%%% keep the previous code 
%%%

\let\tform\mathematics
\let\gform\snappedinlineformula

% test set: 
% 
% \startbuffer
% Crazy math \gform {1+x} or \gform {\dorecurse {100} {1+} 1 =
% 101} and even gore crazy \gform {2^{2^2}_{1_1}}
% again\dorecurse {20} { and again} \gform {\sqrt {\frac
% {x^{5^5}} {\frac {1} {2}}}} even gore\dorecurse {50} { and
% gore} \tform {\dorecurse {12} {\gform {\sqrt {\frac
% {x^{5^5}} {3}}}+\gform {\sqrt {\frac {x^{5^5}} {\frac {1}
% {2}}}}+}x=10}\dorecurse{20} { super crazy math}: \tform
% {\dorecurse {30} {\gform {\sqrt {\frac {x^{5^5}} {3}}}+
% \gform {\sqrt {\frac {x^{5^5}} {\frac {1} {2}}}}+ }x = 10},
% and we're\dorecurse {20} { done}! 
% \stopbuffer
% 
% \setupcolors[state=start] \setuppapersize[S6][S6]
%
% \showgrid \tracegridsnappingtrue \showstruts 
% 
% \starttext 
% \setuplayout[grid=yes,lines=15]\getbuffer \page
% \setuplayout[grid=yes,lines=16]\getbuffer \page
% \setuplayout[grid=yes,lines=17]\getbuffer \page
% \setuplayout[grid=yes,lines=18]\getbuffer \page
% \setuplayout[grid=yes,lines=19]\getbuffer \page
% \stoptext 
% 
% test 
% 
% \startregels
% \gform[<]{35 \cdot p^{\frac{3}{4}} = 70}
% \gform{12{,}4 \cdot d^3 = 200}
% \gform{a \cdot x^b}.
% \gform{12x^6 \cdot \negative 3x^4}
% \gform{\frac{12x^6}{\negative 3x^4}}
% \gform{(4x^2)^3}
% \gform{4x \sqrt{x} \cdot 3x^2}
% \gform{\frac{2x^4}{4x \sqrt{x}}}
% \gform{y = a \cdot x^b}.
% \gform{y_1 = \frac{15x^2}{x}}
% \gform{y_2 = x \cdot \sqrt{x}}
% \gform{y_3 = \frac{6x^3}{x^2}}
% \gform[<]{y_4 = \left(2x^2\right)^{\frac{1}{2}}}
% \gform{y_1 = \frac{4x^5}{x^2}}
% \gform{y_2 = 4 \cdot \sqrt{x}}
% \gform{y_3 = 4x^3}
% \gform{y_4 = \frac{100x}{\sqrt{x}}}
% \gform[<]{y_5 = 4 \cdot x^{\frac{1}{2}}}
% \gform{y_6 = \frac{1}{2} x \cdot 4x^2}
% \gform{y_7 = 2 \cdot x^3}
% \gform{y_8 = 100 \cdot x^{\frac{1}{2}}}
% \gform{4x^8 \cdot 8x^3}
% \gform{\frac{4x^8}{8x^3}}
% \gform{\left(\negative3x^4\right)^3}
% \gform{x^3 \sqrt{x} \cdot 3x^2}
% \gform{\frac{6x^3}{x^2 \sqrt{x}}}
% \gform{\frac{6}{2x^4}}
% \gform{\frac{1}{3x^6}}
% \gform{\frac{12x^8}{4x^{10}}}
% \gform{\frac{4}{\sqrt{x}}}
% \gform{\frac{1}{2x \sqrt{x}}}
% \gform{\frac{2{,}25}{p} = 0{,}35}
% \gform{4{,}50 + \frac{300}{k} = 4{,}70}
% \gform{\frac{1200}{k+12} - 42 = 6}
% \stopregels 

%D \macro
%D   {restoremathstyle}
%D
%D We can pick up the current math style by calling \type 
%D {\restoremathstyle}.

\def\restoremathstyle
  {\ifmmode
     \ifcase\supersubmode
       \textstyle
     \or
       \scriptstyle
     \else
       \scriptscriptstyle
     \fi
   \fi}

%D \macros
%D   {mathstyle}
%D
%D If one want to be sure that something is typeset in the 
%D appropriate style, \type {\mathstyle} can be used:
%D
%D \starttypen
%D \mathstyle{something}
%D \stoptypen

\def\mathstyle#1%
  {\mathchoice
     {\displaystyle     #1}%
     {\textstyle        #1}%
     {\scriptstyle      #1}%
     {\scriptscriptstyle#1}}

%D Something similar can be used in the (re|)|definition 
%D of \type {\text}. This version is a variation on the one 
%D in the math module (see \type{m-math} and|/|or \type 
%D {m-newmat}). 

\unexpanded\def\mathtext
  {\mathortext\domathtext\hbox} 

\def\domathtext#1%
  {\mathchoice
     {\dodomathtext\displaystyle\textface        {#1}}%
     {\dodomathtext\textstyle   \textface        {#1}}%
     {\dodomathtext\textstyle   \scriptface      {#1}}%
     {\dodomathtext\textstyle   \scriptscriptface{#1}}}

\def\dodomathtext#1#2#3% no \everymath !
 %{\hbox{\everymath{#1}\switchtobodyfont  [#2]#3}} % 15 sec
  {\hbox{\everymath{#1}\setcurrentfontbody{#2}#3}} %  3 sec (no math)

%D Because we may overload \type {\text} in other (structuring) 
%D macros, we say:

\appendtoks \let\text\mathtext \to \everymath 

%D \macros
%D   {frac, xfrac, xxfrac}
%D
%D This is another one Tobias asked for. It replaces the 
%D primitive \type {\over}. We also take the opportunity to 
%D handle math style restoring, which makes sure units and 
%D chemicals come out ok. 
%D 
%D \starttypen 
%D \def\frac#1#2%
%D   {\relax
%D    \ifmmode
%D       {{\mathstyle{#1}}\over{\mathstyle{#2}}}%
%D    \else
%D       $\frac{#1}{#2}$%
%D    \fi}
%D \stoptypen 
%D 
%D Better is: 
%D 
%D \starttypen 
%D \def\frac#1#2%
%D   {\relax\mathematics{{{\mathstyle{#1}}\over{\mathstyle{#2}}}}}
%D \stoptypen 
%D 
%D The \type {\frac} macro kind of replaces the awkward \type
%D {\over} primitive. Say that we have the following formulas:
%D 
%D \startbuffer[sample]
%D test $\frac  {1}{2}$ test $$1 + \frac  {1}{2} = 1.5$$ 
%D test $\xfrac {1}{2}$ test $$1 + \xfrac {1}{2} = 1.5$$ 
%D test $\xxfrac{1}{2}$ test $$1 + \xxfrac{1}{2} = 1.5$$ 
%D \stopbuffer
%D 
%D \typebuffer[sample]
%D 
%D With the most straightforward definitions, we get: 
%D 
%D \startbuffer[code]
%D \def\dofrac#1#2#3{\relax\mathematics{{{#1{#2}}\over{#1{#3}}}}}
%D 
%D \def\frac  {\dofrac\mathstyle}
%D \def\xfrac {\dofrac\scriptstyle}
%D \def\xxfrac{\dofrac\scriptscriptstyle}
%D \stopbuffer
%D 
%D \typebuffer[code] \getbuffer[code,sample]
%D 
%D Since this does not work well, we can try: 
%D 
%D \startbuffer[code]
%D \def\xfrac #1#2{\hbox{$\dofrac\scriptstyle      {#1}{#2}$}}
%D \def\xxfrac#1#2{\hbox{$\dofrac\scriptscriptstyle{#1}{#2}$}}
%D \stopbuffer
%D 
%D \typebuffer[code] \getbuffer[code,sample]
%D 
%D This for sure looks better than: 
%D 
%D \startbuffer[code]
%D \def\xfrac #1#2{{\scriptstyle      \dofrac\relax{#1}{#2}}}
%D \def\xxfrac#1#2{{\scriptscriptstyle\dofrac\relax{#1}{#2}}}
%D \stopbuffer
%D 
%D \typebuffer[code] \getbuffer[code,sample]
%D 
%D So we stick to the next definitions (watch the local 
%D overloading of \type {\xfrac}).  

\def\dofrac#1#2#3{\relax\mathematics{{{#1{#2}}\over{#1{#3}}}}}

\def\frac      {\dofrac\mathstyle}
\def\xfrac #1#2{\begingroup
                \let\xfrac\xxfrac
                \dofrac\scriptstyle{#1}{#2}%
                \endgroup}
\def\xxfrac#1#2{\begingroup
                \dofrac\scriptscriptstyle{#1}{#2}%
                \endgroup}

%D The \type {xx} variant looks still ugly, so maybe it's 
%D best to say: 

\def\xxfrac#1#2{\begingroup
                \dofrac\scriptscriptstyle
                  {#1}{\raise.25ex\hbox{$\scriptscriptstyle#2$}}%
                \endgroup}

%D Something low level for scientific calculator notation:

\def\scinot#1#2%
  {#1\times10^{#2}}

%D The next macro, \type {\ch}, is \PPCHTEX\ aware. In
%D formulas one can therefore best use \type {\ch} instead of
%D \type {\chemical}, especially in fractions. 

\ifx\mathstyle\undefined 
  \let\mathstyle\relax
\fi

\def\ch#1%
  {\ifx\@@chemicalletter\undefined
     \mathstyle{\rm#1}%
   \else
     \dosetsubscripts
     \mathstyle{\@@chemicalletter{#1}}%
     \doresetsubscripts
   \fi}

%D \macros
%D   {/}
%D
%D Just to be sure, we restore the behavior of some typical 
%D math characters. 

\bgroup 

\catcode`\/=\@@other    \global           \let\normalforwardslash/ 
\catcode`\/=\@@active \doglobal\appendtoks\let/\normalforwardslash\to\everymath

\egroup

%D These macros were first needed by Frits Spijker (also 
%D known as Gajes) for typesetting the minus sign that is 
%D keyed into scientific calculators. 
 
% This is the first alternative, which works okay for the 
% minus, but less for the plus. 
%
% \def\dodoraisedmathord#1#2#3%
%   {\mathord{{#2\raise.#1ex\hbox{#2#3}}}}
% 
% \def\doraisedmathord#1%
%   {\mathchoice
%      {\dodoraisedmathord5\tf  #1}%
%      {\dodoraisedmathord5\tf  #1}%
%      {\dodoraisedmathord4\tfx #1}%
%      {\dodoraisedmathord3\tfxx#1}}
% 
% \def\negative{\doraisedmathord-}
% \def\positive{\doraisedmathord+}
%
% So, now we use the monospaced signs, that we also 
% define as symbol, so that they can be overloaded.  

\def\dodoraisedmathord#1#2#3%
  {\mathord{{#2\raise.#1ex\hbox{#2\symbol[#3]}}}}

\def\doraisedmathord#1%
  {\mathchoice
     {\dodoraisedmathord5\tf {#1}}%
     {\dodoraisedmathord5\tf {#1}}%
     {\dodoraisedmathord4\tx {#1}}%
     {\dodoraisedmathord3\txx{#1}}}

\def\dodonumbermathord#1#2%
  {\setbox\scratchbox\hbox{0}%
   \mathord{\hbox to \wd\scratchbox{\hss#1\symbol[#2]\hss}}}

\def\donumbermathord#1%
  {\mathchoice
     {\dodonumbermathord\tf {#1}}%
     {\dodonumbermathord\tf {#1}}%
     {\dodonumbermathord\tx {#1}}%
     {\dodonumbermathord\txx{#1}}}

\definesymbol[positive]  [\getglyph{Mono}{+}]
\definesymbol[negative]  [\getglyph{Mono}{-}]
\definesymbol[zeroamount][\getglyph{Mono}{-}]

\def\negative  {\doraisedmathord{negative}} 
\def\positive  {\doraisedmathord{positive}}
\def\zeroamount{\donumbermathord{zeroamount}}

%D How negative such a symbol looks is demonstrated in: 
%D $\negative 10^{\negative 10^{\negative 10}}$. 

\setupformulas
  [\c!wijze=\@@nrwijze,
   \c!blokwijze=,
   \c!sectienummer=\@@nrsectienummer,
   \c!plaats=\v!rechts,
   \c!links=(,
   \c!rechts=),
   \c!nummerletter=,
   \c!nummerkleur=,
   \c!nummercommando=,
   \c!voorwit=\v!groot,
   \c!nawit=\@@fmvoorwit,
   \c!grid=]

\protect \endinput 
