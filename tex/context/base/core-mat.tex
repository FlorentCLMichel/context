%D \module
%D   [       file=core-mat,
%D        version=1998.12.07,
%D          title=\CONTEXT\ Core Macros,
%D       subtitle=Math Fundamentals,
%D         author=Hans Hagen,
%D           date=\currentdate,
%D      copyright={PRAGMA / Hans Hagen \& Ton Otten}]
%C
%C This module is part of the \CONTEXT\ macro||package and is
%C therefore copyrighted by \PRAGMA. See mreadme.pdf for 
%C details. 

% engels maken

\writestatus{loading}{Context Core Macros / Math Fundamentals}

\unprotect

% will move to page-ini 

\newevery \everybeginofpar \EveryBeginOfPar
\newevery \everyendofpar   \EveryEndOfPar

\def\bpar{\the\everybeginofpar}
\def\epar{\the\everyendofpar\endgraf}

\newdimen\lastlinewidth 

\def\setlastlinewidth% 
  {\resetlastlinewidth
   \ifmmode\else\ifhmode\else\ifoptimizedisplayspacing
     \bgroup
     \forgetdisplayskips
     $$\global\lastlinewidth\predisplaysize$$
     \vskip-\baselineskip
     \egroup
   \fi\fi\fi}

\def\resetlastlinewidth%
  {\global\lastlinewidth\!!zeropoint\relax}

\appendtoks \setlastlinewidth \to \everyendofpar

%D moved from main-001

%\def\EveryMathPar{\EveryPar}
%
%\newevery \everymath \EveryMath

\abovedisplayskip      = \!!zeropoint
\abovedisplayshortskip = \!!zeropoint % evt. 0pt minus 3pt
\belowdisplayskip      = \!!zeropoint
\belowdisplayshortskip = \!!zeropoint % evt. 0pt minus 3pt

\predisplaypenalty     = 0
\postdisplaypenalty    = 0  % -5000 gaat mis, zie penalty bij \paragraaf

% we don't use the skip's

\def\displayskipsize#1#2% obsolete
  {\ifdim\tussenwit>\!!zeropoint
     #1\tussenwit\!!plus#2\tussenwit\!!minus#2\tussenwit\relax
   \else
     #1\lineheight\!!plus#2\lineheight\!!minus#2\lineheight\relax
   \fi}

\def\displayskipfactor          {1.0} % obsolete
\def\displayshortskipfactor     {0.8} % obsolete
\def\displayskipgluefactor      {0.3} % obsolete
\def\displayshortskipgluefactor {0.2} % obsolete

\def\abovedisplayskipsize% obsolete
  {\displayskipsize\displayskipfactor\displayskipgluefactor}

\def\belowdisplayskipsize% obsolete
  {\displayskipsize\displayskipfactor\displayskipgluefactor}

\def\abovedisplayshortskipsize% obsolete
  {\displayskipsize\displayshortskipfactor\displayshortskipgluefactor}

\def\belowdisplayshortskipsize% obsolete
  {\displayskipsize\displayshortskipfactor\displayshortskipgluefactor}

\def\setdisplayskip#1#2#3% obsolete
  {#1=#2\relax
   \advance#1 by -\parskip
   \advance#1 by -#3\relax}

\def\setdisplayskips% obsolete
  {\setdisplayskip\abovedisplayskip     \abovedisplayskipsize     \!!zeropoint
   \setdisplayskip\belowdisplayskip     \belowdisplayskipsize     \!!zeropoint
   \setdisplayskip\abovedisplayshortskip\abovedisplayshortskipsize\baselineskip
   \setdisplayskip\belowdisplayshortskip\belowdisplayshortskipsize\baselineskip}

% so far for unused stuff

\def\forgetdisplayskips% to do 
  {\abovedisplayskip     \!!zeropoint
   \belowdisplayskip     \!!zeropoint
   \abovedisplayshortskip\!!zeropoint
   \belowdisplayshortskip\!!zeropoint}

\doorlabelen
  [\v!formule]
  [\c!tekst=\v!formule,
   \c!wijze=\@@fmwijze,
   \c!blokwijze=\@@fmblokwijze,
   \c!plaats=\v!intekst]

\def\setupformulas%
  {\dodoubleargument\getparameters[\??fm]}

\newconditional\handleformulanumber
\newconditional\incrementformulanumber

\def\dododoformulenummer#1#2#3#4% (#1,#2)=outer(ref,sub) (#3,#4)=inner(ref,sub)
  {\hbox\bgroup
   \ifconditional\handleformulanumber
     \ifconditional\incrementformulanumber
       \verhoognummer[\v!formule]%
     \fi
     \maakhetnummer[\v!formule]%
     \setbox0=\hbox{\ignorespaces#2\unskip}%
     \ifdim\wd0>\zeropoint
       \edef\hetsubnummer{#2}%
     \else
       \let\hetsubnummer\empty
     \fi
     \doifsomething{#1}{\rawreference{\s!for}{#1}{\hetnummer\hetsubnummer}}%
     \setbox0=\hbox{\ignorespaces#4\unskip}%
     \ifdim\wd0>\zeropoint
       \edef\hetsubnummer{#4}%
     \fi
     \doifsomething{#3}{\rawreference{\s!for}{#3}{\hetnummer\hetsubnummer}}%
     \rm % nodig ? 
     \@@fmnummercommando
       {\dostartattributes\??fm\c!nummerletter\c!nummerkleur
        \strut
        \@@fmlinks
        \preparethenumber\??fm\hetnummer\preparednumber
        \labeltexts\v!formule
          {\ignorespaces\preparednumber\ignorespaces\hetsubnummer\unskip}%
        \@@fmrechts
        \dostopattributes}%
   \fi
   \egroup}

\def\dodoformulenummer[#1][#2][#3]%
  {\doquadruplegroupempty\dododoformulenummer{#1}{#2}{#3}}

\def\doformulenummer%
  {\dotripleempty\dodoformulenummer}

\setvalue{\e!start\e!formule}{\dostartformula{}}
\setvalue{\e!stop \e!formule}{\dostopformula}

\def\definieerformule%
  {\dodoubleempty\dodefinieerformule}

\def\dodefinieerformule[#1][#2]%
  {\doifsomething{#1}
     {\copyparameters
        [\??fm#1][\??fm]
        [\c!voorwit,\c!nawit,
         \c!linkermarge,\c!rechtermarge,\c!marge,
         \c!springvolgendein,\c!variant,
         \c!strut,\c!uitlijnen,\c!afstand]%
      \setupformulas[#1][#2]%
      \setvalue{\e!start#1\e!formule}%  
        {\dostartformula{#1}}%
      \setvalue{\e!stop #1\e!formule}%
        {\dostopformula}}}

\def\setupformulas%
  {\dodoubleempty\dosetupformulas}

\def\dosetupformulas[#1][#2]%
  {\ifsecondargument
     \getparameters[\??fm#1][#2]%
   \else
     \getparameters[\??fm][#1]%
   \fi}

\setupformulas
  [\c!wijze=\@@nrwijze,
   \c!blokwijze=,
   \c!sectienummer=\@@nrsectienummer,
   \c!plaats=\v!rechts,
   \c!links=(,
   \c!rechts=),
   \c!voorwit=,
   \c!nawit=\@@fmvoorwit,
   \c!linkermarge=\!!zeropoint,
   \c!rechtermarge=\!!zeropoint,
   \c!marge=,
   \c!springvolgendein=\v!nee,
   \c!variant=\s!default,
   \c!uitlijnen=,
   \c!strut=\v!nee,
   \c!scheider=\@@koscheider,
   \c!afstand=1em]

\def\currentformula          {}
\def\predisplaysizethreshhold{3em}

\def\leftdisplayskip   {\leftskip}
\def\rightdisplayskip  {\rightskip}
\def\leftdisplaymargin {\getvalue{\??fm\currentformula\c!linkermarge}}
\def\rightdisplaymargin{\getvalue{\??fm\currentformula\c!rechtermarge}}

\def\beforedisplayspace
  {\doifnotvalue{\??fm\currentformula\c!voorwit}{\v!geen}
     {\blanko[\getvalue{\??fm\currentformula\c!voorwit}]}}

\def\afterdisplayspace 
  {\doifnotvalue{\??fm\currentformula\c!nawit}{\v!geen}
     {\blanko[\getvalue{\??fm\currentformula\c!nawit}]}}

\def\setpredisplaysize#1%
  {\predisplaysize=#1\relax
   \ifdim\predisplaysize>\!!zeropoint
     \advance\predisplaysize \predisplaysizethreshhold
   \fi
   \advance\predisplaysize \displayindent % needed ? 
   \ifdim\predisplaysize>\hsize
     \predisplaysize\hsize
   \fi}

\def\setdisplaydimensions%
  {\displayindent=\leftdisplayskip 
   \advance\displayindent\leftdisplaymargin 
   \displaywidth=\hsize
   \ifdim\hangindent>\!!zeropoint
     \advance\displayindent\hangindent
   \else
     \advance\displaywidth\hangindent
   \fi
   \advance\displaywidth-\displayindent
   \advance\displaywidth-\rightdisplayskip
   \advance\displaywidth-\rightdisplaymargin}

\newif\ifoptimizedisplayspacing

\def\dostartformula#1% setting leftskip adaption is slow ! 
  {\bgroup 
   \def\currentformula{#1}%
   \doifvaluesomething{\??fm\currentformula\c!marge}% so we test first 
     {\dosetleftskipadaption{\getvalue{\??fm\currentformula\c!marge}}%
      \edef\leftdisplaymargin{\the\leftskipadaption}}% overloaded
   \long\def\dostartformula##1{\bgroup\let\dostopformula\egroup}%
   \freezedimenmacro\leftdisplayskip
   \freezedimenmacro\rightdisplayskip
   \freezedimenmacro\leftdisplaymargin
   \freezedimenmacro\rightdisplaymargin
   \freezedimenmacro\predisplaysizethreshhold
   \forgetdisplayskips
   \let\setdisplayskips\relax
   \ifoptimizedisplayspacing
     \ifdim\lastlinewidth>\!!zeropoint\relax
       \abovedisplayshortskip-\ht\strutbox\relax
     \fi
   \else
     \resetlastlinewidth
   \fi
   \getvalue{\e!start\getvalue{\??fm\currentformula\c!variant}\e!formule}}

\let\doplaceformulanumber\empty

\def\dostopformula%
  {\doplaceformulanumber
   \getvalue{\e!stop\getvalue{\??fm\currentformula\c!variant}\e!formule}%
   \egroup
   \resetlastlinewidth
   \nonoindentation
   \doifvalue{\??fm\currentformula\c!springvolgendein}{\v!nee}
     {\noindentation}}

% \def\predisplaycorrection%
%   {\par\ifvmode\vbox{\strut}\vskip-2\baselineskip\fi} 
% 
% \def\startdisplaymath%
%   {\ifgridsnapping
%      \beforedisplayspace   
%      \snaptogrid\vbox\bgroup 
%      % \vskip\parskip 
%      \forgetall
%    \else
%      \bgroup
%      \par\ifvmode\vskip-\parskip\fi
%      \beforedisplayspace   
%      \predisplaycorrection
%    \fi
%    $$\setdisplaydimensions
%    \setpredisplaysize\lastlinewidth}
% 
% \def\stopdisplaymath%
%   {$$\egroup
%    \par\ifvmode\vskip-\parskip\fi
%    \afterdisplayspace}

% \def\startdisplaymath%
%   {\ifgridsnapping
%      \beforedisplayspace   
%      \snaptogrid\vbox\bgroup 
%      % \vskip\parskip 
%      \forgetall
%    \else
%      \bgroup
%      \ifdim\lastskip<\!!zeropoint\else
%        \par\ifvmode\ifdim\parskip>\!!zeropoint\relax\vskip-\parskip\fi\fi
%      \fi
%      \beforedisplayspace   
%      \par\ifvmode\vbox{\strut}\vskip-2\baselineskip\fi 
%    \fi
%    $$\setdisplaydimensions
%    \setpredisplaysize\lastlinewidth
%    \startinnermath}
% 
% \def\stopdisplaymath%
%   {\stopinnermath
%     $$\egroup
%    \par\ifvmode\ifdim\parskip>\!!zeropoint\relax\vskip-\parskip\fi\fi
%    \afterdisplayspace}

\newif\ifinformula

\def\startdisplaymath%
  {\ifgridsnapping
     \beforedisplayspace   
     \snaptogrid\vbox
     \bgroup 
     \informulatrue   
     \forgetall
  \else
     \bgroup
     \informulatrue   
    %\forgetall % otherwise backgrounds fail 
     \ifdim\lastskip<\!!zeropoint\else
       \par\ifvmode\ifdim\parskip>\!!zeropoint\relax\vskip-\parskip\fi\fi
     \fi
     \beforedisplayspace   
     \par\ifvmode\vbox{\strut}\vskip-2\baselineskip\fi 
   \fi
   $$\setdisplaydimensions
   \setpredisplaysize\lastlinewidth
   \startinnermath}

\def\stopdisplaymath%
  {\stopinnermath
   $$%
   \ifgridsnapping 
   \else
     \par\ifvmode\ifdim\parskip>\zeropoint\vskip-\parskip\fi\fi
   \fi 
   \afterdisplayspace
   \egroup}

\let\startinnermath\empty
\let\stopinnermath \empty

\def\defineformulaalternative%
  {\dotripleargument\dodefineformulaalternative}

\def\dodefineformulaalternative[#1][#2][#3]%  
  {\setvalue{\e!start#1\e!formule}{#2}%
   \setvalue{\e!stop #1\e!formule}{#3}}

\defineformulaalternative[\s!default][\startdisplaymath][\stopdisplaymath]

% sp = single line paragraph  sd = single line display 
% mp = multi  line paragraph  md = multy  line display 

\defineformulaalternative[single][\startdisplaymath][\stopdisplaymath]
\defineformulaalternative[multi] [\startdisplaymath][\stopdisplaymath]

\definieerformule
  [sp]
  [\c!voorwit=\v!geen,\c!nawit=\v!geen,
   \c!springvolgendein=\v!nee,
   \c!variant=single]

\definieerformule
  [sd]
  [\c!voorwit=\v!geen,\c!nawit=\v!geen,
   \c!springvolgendein=\v!ja,
   \c!variant=single]

\definieerformule
  [mp]
  [%\c!voorwit=,\c!nawit=,
   \c!springvolgendein=\v!nee,
   \c!variant=multi]

\definieerformule
  [md]
  [%\c!voorwit=,\c!nawit=,
   \c!springvolgendein=\v!ja,
   \c!variant=multi]

% in m-math
%
% \defineformulaalternative[multi][\begindmath][\enddmath]
% 
% \fakewords{20}{40}\epar 
% \plaatsformule {a} $$              \fakespacingformula $$
% \fakewords{20}{40}\epar 
% \plaatsformule {b} \startformule   \fakespacingformula \stopformule
% \plaatsformule {b} \startformule   \fakespacingformula \stopformule
% \fakewords{20}{40}\epar 
% \plaatsformule {c} \startmdformule \fakespacingformula \stopmdformule
% \plaatsformule {c} \startmdformule \fakespacingformula \stopmdformule
% \fakewords{20}{40}\epar 
% \plaatsformule {d} \startmpformule \fakespacingformula \stopmpformule
% \plaatsformule {d} \startmpformule \fakespacingformula \stopmpformule
% \fakewords{20}{40}\epar 
% \plaatsformule {e} \startsdformule \fakespacingformula \stopsdformule
% \plaatsformule {e} \startsdformule \fakespacingformula \stopsdformule
% \fakewords{20}{40}\epar 
% \plaatsformule {f} \startspformule \fakespacingformula \stopspformule
% \plaatsformule {f} \startspformule \fakespacingformula \stopspformule
% \fakewords{20}{40}

\def\plaatsformule%
  {\settrue\incrementformulanumber
   \dodoubleempty\doplaatsformule}

\def\plaatssubformule%
  {\setfalse\incrementformulanumber
   \dodoubleempty\doplaatsformule}

\def\doplaatsformule[#1][#2]% #2 = dummy, gobbles spaces
  {\def\redoplaatsformule%
     {\bgroup\def\dostartformula####1{\relax}%
      \ifx\next\bgroup
        \@EA\moreplaatsformule % [ref]{}
      \else\if\next\relax % a \cs
        \egroup \@EA\@EA\@EA\dodoplaatsformule % [ref]\start 
      \else
        \egroup \@EA\@EA\@EA\dispplaatsformule % [ref]$$
      \fi\fi
      [#1]{}}%
   \futurelet\next\redoplaatsformule}

\long\def\moreplaatsformule[#1]#2#3#4% #2 dummy #4 gobbles spaces
  {\def\redoplaatsformule%
     {\if\next\relax % a \cs
        \egroup \expandafter\dodoplaatsformule % [ref]{}\start
      \else
        \egroup \expandafter\dispplaatsformule % [ref]{}$$
      \fi
      [#1]{#3}}%
   \futurelet\next\redoplaatsformule#4}

\def\dispplaatsformule[#1]#2$$#3$$%
  {\dodoplaatsformule[#1]{#2}\dostartformula{}#3\dostopformula}

\let\normalreqno\eqno
\let\normalleqno\leqno

\def\dodoplaatsformule[#1]#2% messy, needs a clean up 
  {\doifelse{#1}{-}
     {\setfalse\handleformulanumber}
     {\doifelse{#2}{-}
        {\setfalse\handleformulanumber}
        {\settrue\handleformulanumber}}%
   \ifconditional\handleformulanumber
     \def\formulenummer%
       {%\global\let\subformulenummer\doformulenummer % no, bug 
        \doformulenummer[#1][#2]}%
     \def\subformulenummer%
       {\setfalse\incrementformulanumber
        \formulenummer}%
     \gdef\doplaceformulanumber%
       {\global\let\doplaceformulanumber\empty
        \doifelse{\@@fmplaats}{\v!links}
          {\normalleqno{\doformulenummer[#1][#2][]{}}}
          {\normalreqno{\doformulenummer[#1][#2][]{}}}}%
   \else
     \def\formulenummer{\doformulenummer[#1][#2]}%
     \global\let\subformulenummer\doformulenummer
     \global\let\doplaceformulanumber\empty
   \fi} 

%D We need a hook into the plain math alignment macros
%D
%D \starttypen
%D \displaylines
%D \eqalignno
%D \eqalignno
%D \stoptypen
%D
%D Otherwise we get a missing \type {$$} error reported.

\def\resetdisplaymatheq
  {\let\normalleqno\relax \let\leqno\relax
   \let\normalreqno\relax \let\eqno \relax
   \let\doplaceformulanumber\relax}

\let\normaldispl@y\displ@y

\def\displ@y{\resetdisplaymatheq\normaldispl@y}

%D Here we implement a basic math alignment mechanism. Numbers
%D are also handled. The macros \type {\startinnermath} and
%D \type {\stopinnermath} can be overloaded in specialized
%D modules. 

\def\startinnermath%
  {\getvalue{\e!start\??fm\getvalue{\??fm\currentformula\c!uitlijnen}}}

\def\stopinnermath%
  {\getvalue{\e!stop \??fm\getvalue{\??fm\currentformula\c!uitlijnen}}}

\def\mathinnerstrut%
  {\doifvalue{\??fm\currentformula\c!strut}{\v!ja}{\strut}}

\long\def\defineinnermathhandler#1#2#3%
  {\setvalue{\e!start\??fm#1}{#2}%
   \setvalue{\e!stop \??fm#1}{#3}}

\newif\iftracemath 

\def\mathhbox%
  {\iftracemath\ruledhbox\else\hbox\fi}  

\def\startmathbox#1#2%
  {\hsize\displaywidth
   \global\let\@eqno \empty \def\eqno {\gdef\@eqno }%
   \global\let\@leqno\empty \def\leqno{\gdef\@leqno}%
   % added 
   \let\normalreqno\eqno
   \let\normalleqno\leqno
   % added 
   \doplaceformulanumber
   \mathhbox to \displaywidth\bgroup
     \mathinnerstrut
     $\displaystyle
     \ifx\@leqno\empty\else \ifcase#2 
       \rlap{\@leqno}%
     \else
       \@leqno\hskip\getvalue{\??fm\currentformula\c!afstand}%
     \fi \fi 
     #1}

\def\stopmathbox#1#2%
  {$#1%
   \ifx\@eqno\empty\else \ifcase#2 
     \llap{\@eqno}%
   \else
     \hskip\getvalue{\??fm\currentformula\c!afstand}\@eqno
   \fi \fi 
   \egroup}

\defineinnermathhandler\v!rechts{\startmathbox\empty1}{\stopmathbox\hfill0}
\defineinnermathhandler\v!links {\startmathbox\hfill0}{\stopmathbox\empty1}
\defineinnermathhandler\v!midden{\startmathbox\hfill0}{\stopmathbox\hfill0}

%D [The examples below are in english and don't process in the 
%D  documentation style, which will be english some day.]
%D 
%D Normally a formula is centered, but in case you want to
%D align it left or right, you can set up formulas to behave
%D that way. Normally a formula will adapt is left indentation
%D to the environment: 
%D 
%D \startbuffer
%D \fakewords{20}{40}\epar 
%D \startitemize
%D \item \fakewords{20}{40}\epar 
%D       \placeformula \startformula \fakeformula \stopformula
%D \item \fakewords{20}{40}\epar 
%D \stopitemize
%D \fakewords{20}{40}\epar 
%D \stopbuffer
%D 
%D % \getbuffer
%D 
%D In the next examples we explicitly align formulas to the
%D left (\type {\raggedleft}), center and right (\type
%D {\raggedright}): 
%D 
%D \startbuffer
%D \setupformulas[align=left]   
%D \startformula\fakeformula\stopformula
%D \setupformulas[align=middle] 
%D \startformula\fakeformula\stopformula
%D \setupformulas[align=right]  
%D \startformula\fakeformula\stopformula
%D \stopbuffer
%D 
%D \typebuffer
%D 
%D Or in print: 
%D 
%D % {\getbuffer}
%D 
%D With formula numbers these formulas look as follows: 
%D 
%D \startbuffer
%D \setupformulas[align=left]   
%D \placeformula \startformula\fakeformula\stopformula
%D \setupformulas[align=middle] 
%D \placeformula \startformula\fakeformula\stopformula
%D \setupformulas[align=right]  
%D \placeformula \startformula\fakeformula\stopformula
%D \stopbuffer
%D 
%D % {\getbuffer}
%D 
%D This was keyed in as: 
%D 
%D \typebuffer
%D 
%D When tracing is turned on (\type {\tracemathtrue}) you can 
%D visualize the bounding box of the formula, 
%D 
%D % {\tracemathtrue\getbuffer}
%D 
%D As you can see, the dimensions are the natural ones, but if 
%D needed you can force a normalized line: 
%D 
%D \startbuffer
%D \setupformulas[strut=yes]
%D \placeformula \startformula \fakeformula \stopformula
%D \stopbuffer
%D 
%D \typebuffer
%D 
%D This time we get a more spacy result. 
%D 
%D % {\tracemathtrue\getbuffer}
%D 
%D We will now show a couple of more settings and combinations
%D of settings. In centered formulas, the number takes no space
%D 
%D \startbuffer
%D \setupformulas[align=middle]
%D \startformula \fakeformula \stopformula
%D \placeformula \startformula \fakeformula \stopformula
%D \stopbuffer
%D 
%D \typebuffer % {\tracemathtrue\getbuffer}
%D 
%D You can influence the placement of the whole box with the
%D parameters \type {leftmargin} and \type {rightmargin}. 
%D 
%D \startbuffer
%D \setupformulas[align=right,leftmargin=3em]
%D \startformula \fakeformula \stopformula
%D \placeformula \startformula \fakeformula \stopformula
%D 
%D \setupformulas[align=left,rightmargin=1em]
%D \startformula \fakeformula \stopformula
%D \placeformula \startformula \fakeformula \stopformula
%D \stopbuffer
%D 
%D \typebuffer % {\tracemathtrue\getbuffer}
%D 
%D You can also inherit the margin from the environment. 
%D 
%D \startbuffer
%D \setupformulas[align=right,margin=standard]
%D \startformula \fakeformula \stopformula
%D \placeformula \startformula \fakeformula \stopformula
%D \stopbuffer
%D 
%D \typebuffer % {\tracemathtrue\getbuffer}
%D 
%D The distance between the formula and the number is only
%D applied when the formula is left or right aligned. 
%D 
%D \startbuffer
%D \setupformulas[align=left,distance=2em]
%D \startformula \fakeformula \stopformula
%D \placeformula \startformula \fakeformula \stopformula
%D \stopbuffer
%D 
%D \typebuffer % {\tracemathtrue\getbuffer}

%D \macros 
%D   {big..}
%D
%D Because they are related to the bodyfontsize, we redefine 
%D some \PLAIN\ macros. 

\def\@@dobig#1#2%
  {{\hbox{$\left#2\vbox\!!to#1\bodyfontsize{}\right.\n@space$}}}

\def\big {\@@dobig{0.85}}
\def\Big {\@@dobig{1.15}}
\def\bigg{\@@dobig{1.45}}
\def\Bigg{\@@dobig{1.75}}

%D \macros 
%D   {bordermatrix}
%D
%D We already redefined \type {\bordermatrix} in \type 
%D {font-ini}.

%D \macros
%D   {super, sub}
%D
%D \TEX\ uses \type{^} and \type{_} for entering super- and 
%D subscript mode. We want however a bit more control than 
%D normally provided, and therefore provide \type {\super} 
%D and \type{sub}. 

\global\let\normalsuper=^ 
\global\let\normalsuber=_ 

\newcount\supersubmode

\newevery\everysupersub \EverySuperSub

\appendtoks \advance\supersubmode by 1 \to \everysupersub

% \def\dodosuper#1{\normalsuper{\the\everysupersub#1}}
% \def\dodosuber#1{\normalsuber{\the\everysupersub#1}}
% 
% \def\dosuper{\ifx\next\bgroup\expandafter\dodosuper\else\normalsuper\fi}
% \def\dosuber{\ifx\next\bgroup\expandafter\dodosuber\else\normalsuber\fi}
% 
% \def\super{\futurelet\next\dosuper}
% \def\suber{\futurelet\next\dosuber}

\def\super#1{\normalsuper{\the\everysupersub#1}}
\def\suber#1{\normalsuber{\the\everysupersub#1}}

%D \macros
%D   {enablesupersub}
%D 
%D We can let \type {^} and \type {_} act like \type {\super} 
%D and \type {\sub} by saying \type {\enablesupersub}. 

\bgroup
\catcode`\^=\@@active
\catcode`\_=\@@active
\gdef\enablesupersub%
  {\catcode`\^=\@@active
   \def^{\ifmmode\expandafter\super\else\expandafter\normalsuper\fi}%
   \catcode`\_=\@@active
   \def_{\ifmmode\expandafter\suber\else\expandafter\normalsuber\fi}}
\egroup

%D \macro
%D   {restoremathstyle}
%D
%D We can pick up the current math style by calling \type 
%D {\restoremathstyle}.

\def\restoremathstyle%
  {\ifmmode
     \ifcase\supersubmode
       \textstyle
     \or
       \scriptstyle
     \else
       \scriptscriptstyle
     \fi
   \fi}

%D \macros
%D   {mathstyle}
%D
%D If one want to be sure that something is typeset in the 
%D appropriate style, \type {\mathstyle} can be used:
%D
%D \starttypen
%D \mathstyle{something}
%D \stoptypen

\def\mathstyle#1%
  {\mathchoice
     {\displaystyle     #1}%
     {\textstyle        #1}%
     {\scriptstyle      #1}%
     {\scriptscriptstyle#1}}

%D \macros
%D   {frac, xfrac, xxfrac}
%D
%D This is another one Tobias asked for. It replaces the 
%D primitive \type {\over}. We also take the opportunity to 
%D handle math style restoring, which makes sure units and 
%D chemicals come out ok. 
%D 
%D \starttypen 
%D \def\frac#1#2%
%D   {\relax
%D    \ifmmode
%D       {{\mathstyle{#1}}\over{\mathstyle{#2}}}%
%D    \else
%D       $\frac{#1}{#2}$%
%D    \fi}
%D \stoptypen 
%D 
%D Better is: 
%D 
%D \starttypen 
%D \def\frac#1#2%
%D   {\relax\mathematics{{{\mathstyle{#1}}\over{\mathstyle{#2}}}}}
%D \stoptypen 
%D 
%D The \type {\frac} macro kind of replaces the awkward \type
%D {\over} primitive. Say that we habve the following formulas:
%D 
%D \startbuffer[sample]
%D test $\frac  {1}{2}$ test $$1 + \frac  {1}{2} = 1.5$$ 
%D test $\xfrac {1}{2}$ test $$1 + \xfrac {1}{2} = 1.5$$ 
%D test $\xxfrac{1}{2}$ test $$1 + \xxfrac{1}{2} = 1.5$$ 
%D \stopbuffer
%D 
%D \typebuffer[sample]
%D 
%D With the most straightforward definitions, we get: 
%D 
%D \startbuffer[code]
%D \def\dofrac#1#2#3{\relax\mathematics{{{#1{#2}}\over{#1{#3}}}}}
%D 
%D \def\frac  {\dofrac\mathstyle}
%D \def\xfrac {\dofrac\scriptstyle}
%D \def\xxfrac{\dofrac\scriptscriptstyle}
%D \stopbuffer
%D 
%D \typebuffer[code] \getbuffer[code,sample]
%D 
%D Since this does not work well, we can try: 
%D 
%D \startbuffer[code]
%D \def\xfrac #1#2{\hbox{$\dofrac\scriptstyle      {#1}{#2}$}}
%D \def\xxfrac#1#2{\hbox{$\dofrac\scriptscriptstyle{#1}{#2}$}}
%D \stopbuffer
%D 
%D \typebuffer[code] \getbuffer[code,sample]
%D 
%D This for sure looks better than: 
%D 
%D \startbuffer[code]
%D \def\xfrac #1#2{{\scriptstyle      \dofrac\relax{#1}{#2}}}
%D \def\xxfrac#1#2{{\scriptscriptstyle\dofrac\relax{#1}{#2}}}
%D \stopbuffer
%D 
%D \typebuffer[code] \getbuffer[code,sample]
%D 
%D So we stick to the next definitions (watch the local 
%D overloading of \type {\xfrac}).  

\def\dofrac#1#2#3{\relax\mathematics{{{#1{#2}}\over{#1{#3}}}}}

\def\frac{\dofrac\mathstyle}
\def\xfrac#1#2{\hbox{$\let\xfrac\xxfrac\dofrac\scriptstyle{#1}{#2}$}}
\def\xxfrac#1#2{\hbox{$\dofrac\scriptscriptstyle{#1}{#2}$}}

%D The next macro, \type {\ch}, is \PPCHTEX\ aware. In
%D formulas one can therefore best use \type {\ch} instead of
%D \type {\chemical}, especially in fractions. 

\ifx\mathstyle\undefined 
  \let\mathstyle\relax
\fi

\def\ch#1%
  {\ifx\@@chemicalletter\undefined
     \mathstyle{\rm#1}%
   \else
     \dosetsubscripts
     \mathstyle{\@@chemicalletter{#1}}%
     \doresetsubscripts
   \fi}

%D \macros
%D   {/}
%D
%D Just to be sure, we restore the behavior of some typical 
%D math characters. 

\bgroup 

\catcode`\/=\@@other    \global           \let\normalforwardslash/ 
\catcode`\/=\@@active \doglobal\appendtoks\let/\normalforwardslash\to\everymath

\egroup

%D These macros were first needed by Frits Spijker (also 
%D known as Gajes) for typesetting the minus sign that is 
%D keyed into scientific calculators. 
 
% This is the first alternative, which works okay for the 
% minus, but less for the plus. 
%
% \def\dodoraisedmathord#1#2#3%
%   {\mathord{{#2\raise.#1ex\hbox{#2#3}}}}
% 
% \def\doraisedmathord#1%
%   {\mathchoice
%      {\dodoraisedmathord5\tf  #1}%
%      {\dodoraisedmathord5\tf  #1}%
%      {\dodoraisedmathord4\tfx #1}%
%      {\dodoraisedmathord3\tfxx#1}}
% 
% \def\negative{\doraisedmathord-}
% \def\positive{\doraisedmathord+}
%
% So, now we use the monospaced signs, that we also 
% define as symbol, so that they can be overloaded.  

\def\dodoraisedmathord#1#2#3%
  {\mathord{{#2\raise.#1ex\hbox{#2\symbol[#3]}}}}

\def\doraisedmathord#1%
  {\mathchoice
     {\dodoraisedmathord5\tf {#1}}%
     {\dodoraisedmathord5\tf {#1}}%
     {\dodoraisedmathord4\tx {#1}}%
     {\dodoraisedmathord3\txx{#1}}}

\def\dodonumbermathord#1#2%
  {\setbox\scratchbox\hbox{0}%
   \mathord{\hbox to \wd\scratchbox{\hss#1\symbol[#2]\hss}}}

\def\donumbermathord#1%
  {\mathchoice
     {\dodonumbermathord\tf {#1}}%
     {\dodonumbermathord\tf {#1}}%
     {\dodonumbermathord\tx {#1}}%
     {\dodonumbermathord\txx{#1}}}

\definesymbol[positive]  [\getglyph{Mono}{+}]
\definesymbol[negative]  [\getglyph{Mono}{-}]
\definesymbol[zeroamount][\getglyph{Mono}{-}]

\def\negative  {\doraisedmathord{negative}} 
\def\positive  {\doraisedmathord{positive}}
\def\zeroamount{\donumbermathord{zeroamount}}

%D How negative such a symbol looks is demonstrated in: 
%D $\negative 10^{\negative 10^{\negative 10}}$. 

\setupformulas
  [\c!wijze=\@@nrwijze,
   \c!blokwijze=,
   \c!sectienummer=\@@nrsectienummer,
   \c!plaats=\v!rechts,
   \c!links=(,
   \c!rechts=),
   \c!nummerletter=,
   \c!nummerkleur=,
   \c!nummercommando=,
   \c!voorwit=\v!groot,
   \c!nawit=\@@fmvoorwit]

\protect \endinput 
