%D \module
%D   [      file=s-pre-02,
%D        version=1998.04.21,
%D          title=\CONTEXT\ Style File,
%D       subtitle=Presentation Environment 2,
%D         author=Hans Hagen,
%D           date=\currentdate,
%D      copyright={PRAGMA / Hans Hagen \& Ton Otten}]
%C
%C This module is part of the \CONTEXT\ macro||package and is
%C therefore copyrighted by \PRAGMA. See mreadme.pdf for 
%C details. 

%D This environment can be used to typeset interactive
%D presentations. This module was first used at the 1998 
%D publishers conference of the European Portable Document 
%D Association.

\usemodule[pre-general] % mode=step 

%D \macros
%D   {setupbodyfont, switchtobodyfont, setuplayout}
%D
%D At \PRAGMA\ we prefer using the Lucida Bright fonts, but 
%D one can of course load another typeface. 

\setupbodyfont
  [lbr,14.4pt]

\setuplayout
  [style={\switchtobodyfont[12pt]}]

%D \macros
%D   {setuppapersize, setuplayout}
%D
%D The papersize suits the screen dimensions. The layout is 
%D rather simple. We use the whole width of the screen and only
%D have navigational tools at the bottom of the screen.

\setuppapersize
  [S6][S6]

\setuplayout
  [backspace=1cm,
   topspace=1cm,
   margin=0pt,
   header=0pt,
   footer=0pt,
   bottomdistance=.875cm,
   bottom=1cm,
   width=fit,
   height=fit]

%D \macros
%D   {setupwhitespace, setuptyping}
%D
%D We don't have much height, so we use a more cramped 
%D spacing. Verbatim text looks better when indented. 
%D

\setupwhitespace
  [medium]

\setuptyping
  [margin=standard]

%D \macros
%D   {definecolor, setupcolors}
%D
%D Of course we enable color. We define some logical colors,
%D of which most default to the same green shade. 

\definecolor [Backgrounds] [r= 0, g=.7, b=.4]
\definecolor [Actions]     [Backgrounds]
\definecolor [Lines]       [Actions]
\definecolor [Page]        [r=.8, g=.8, b=.8]

%\definecolor [Actions] [r=  1, g=.9, b=.5]
%\definecolor [Actions] [r= .1, g=.5, b=.9]
%\definecolor [Actions] [r=  1, g=.3, b=.1]

\setupcolors
  [state=start]

%D \macros
%D   {setupinteraction, setupinteractionscreen}
%D   
%D We still have to enable interaction mode. We go full 
%D screen!

\setupinteraction
  [state=start,
   color=Actions]

\setupinteractionscreen
  [option=max,
   width=fit,
   height=fit]

%D \macros
%D   {setupitemize}
%D
%D And why not bring some color in itemizations too?

\setupitemize
  [color=Backgrounds] 

%D \macros
%D   {defineoverlay, setupbackgrounds}
%D 
%D The navigational elements and the backgrounds are 
%D provided by \METAPOST. The next few macros are meant for the
%D more experienced \CONTEXT\ user. When, as demonstrated 
%D here, we make the dimensions part of the name, we can be 
%D sure that new ones are generated when we change the page 
%D size in between. 
%D 
%D When \METAPOST\ is used, it makes sense to generate the
%D graphics at runtime. This is supported when one enables
%D system calls in the local \type {texmf.cnf} file and add the
%D switch \type {\runMPgraphicstrue} to the local file \type
%D {cont-sys.tex}. When direct processing is disabled or not 
%D supported, \TEXEXEC\ will tke care of graphic generation. 

\MPinclusions{input mp-tool}

\def\MPpagebackground#1#2#3%
  {\startreusableMPgraphic{pb:#1#2#3}
     delta := #3 ;
     fill unitsquare
       xscaled #1 yscaled #2
       withcolor \MPcolor{Page};
     draw unitsquare
       xscaled (#1-2delta) yscaled (#2-2delta)
       shifted (delta,delta)
       withpen pencircle scaled .5delta
       withcolor \MPcolor{Backgrounds} ; 
   \stopreusableMPgraphic
   \reuseMPgraphic{pb:#1#2#3}}

\defineoverlay
  [PageBackground]
  [\MPpagebackground{\overlaywidth}{\overlayheight}{.25cm}]

\setupbackgrounds
  [page]
  [background=PageBackground]

%D \macros
%D   {setuptexttexts}
%D
%D By clicking on the text area, one goes to the next page. 
%D We could also have use the overlay mechanism, but this 
%D takes more settings and requires to generate the backgounds 
%D each page. 
%D 
%D \starttypen 
%D \defineoverlay[BackgroundButton][\BackgroundButton]
%D \setupbackgrounds[text][text][background=BackgroundButton]
%D \setupbackgrounds[state=repeat]
%D \stoptypen
%D
%D We go for the fast one. 

\def\MPtextbackground#1#2#3%
  {\startreusableMPgraphic{tx:#1#2#3}
     delta := #3 ;
     draw unitsquare
       xscaled #1 yscaled #2
       withpen pencircle scaled .5delta
       withcolor \MPcolor{Backgrounds} ; 
   \stopreusableMPgraphic
   \reuseMPgraphic{tx:#1#2#3}}

\defineoverlay
  [TextBackground]
  [\MPtextbackground{\overlaywidth}{\overlayheight}{.125cm}]

\def\BackgroundButton%
  {\button
     [background=TextBackground,
      backgroundoffset=.25cm,
      frame=off,
      height=\vsize,
      width=\hsize]{}[nextpage]}

\setuptexttexts
  [\BackgroundButton]
  []

%D \macros
%D   {setupbottomtexts}
%D
%D At the bottom of the screen, we show three buttons. These
%D direct us to the previous or next jump or exit the document.

% much too complicated, can be done in a few lines nowadays

\def\MPright#1#2%
  {\startreusableMPgraphic{rt:#1#2}
     z1 = ( 0,    0) ;
     z2 = (#1, .5y3) ;
     z3 = ( 0,   #2) ;
     path p ; p := z1--z2--z3--cycle;
     lw := #2/5 ; pickup pencircle scaled lw;
     filldraw p withcolor \MPcolor{Backgrounds} ;
     set_outer_boundingbox currentpicture;
   \stopreusableMPgraphic
   \reuseMPgraphic{rt:#1#2}}

\def\MPleft#1#2%
  {\startreusableMPgraphic{lt:#1#2}
     z1 = (#1,    0) ;
     z2 = ( 0, .5y3) ;
     z3 = (#1,   #2) ;
     path p ; p := z1--z2--z3--cycle;
     lw := #2/5 ; pickup pencircle scaled lw;
     filldraw p withcolor \MPcolor{Backgrounds} ;
     set_outer_boundingbox currentpicture;
   \stopreusableMPgraphic
   \reuseMPgraphic{lt:#1#2}}

\def\MPcircle#1#2%
  {\startreusableMPgraphic{ci:#1#2}
     rx := #1; ry := #2; 
     path p ; p := fullcircle xscaled rx yscaled ry;
     lw := #2/5 ; pickup pencircle scaled lw;
     filldraw p withcolor \MPcolor{Backgrounds} ;
     set_outer_boundingbox currentpicture;
   \stopreusableMPgraphic
   \reuseMPgraphic{ci:#1#2}}

\def\MPwhat#1#2%
  {\startreusableMPgraphic{wh:#1#2} % could be rotated unitsquare
     z1 = (   0,  0) ;
     z2 = (  #1,  0) ;
     z3 = (.5x2, #2) ;
     path p ; p := z1--z2--z3--cycle;
     lw := #2/5 ; pickup pencircle scaled lw;
     filldraw p withcolor \MPcolor{Backgrounds} ;
     set_outer_boundingbox currentpicture;
   \stopreusableMPgraphic
   \reuseMPgraphic{wh:#1#2}}

\unexpanded\def\BackwardButton%
  {\gotobox{\MPleft{\the\bottomheight}{\the\bottomheight}}[previouspage]}

\unexpanded\def\ForwardButton%
  {\gotobox{\MPright{\the\bottomheight}{\the\bottomheight}}[nextpage]}

\unexpanded\def\ExitButton%
  {\gotobox{\MPcircle{\the\bottomheight}{\the\bottomheight}}[CloseDocument]}

\unexpanded\def\WhateverButton%
  {\doifreferencefoundelse{Whatever}
     {\gotobox{\MPwhat{\the\bottomheight}{\the\bottomheight}}[Whatever]}
     {}}

\def\BottomButton%
  {\button
     [frame=off,
      height=\bottomheight,
      width=.3\textwidth]{}[Topics]}

\setupbottomtexts
  [\BottomButton
   \hfill
   \WhateverButton
   \kern2\bottomheight
   \BackwardButton
   \kern.5\bottomheight
   \ExitButton 
   \kern.5\bottomheight
   \ForwardButton
   \kern.5\bottomheight]
  [] % redundant

% an alternative way: 
%
% \setupinteractionmenu[bottom][state=start]
%
% \setupinteraction[menu=on]
%
% \startinteractionmenu[bottom]
%   \button[frame=off,height=\bottomheight,width=.3\textwidth]{}[Topics]
%   \hfill
%   \doifreferencefoundelse{Whatever}
%     {\gotobox{\MPwhat{\the\bottomheight}{\the\bottomheight}}[Whatever]}{}%
%   \kern2\bottomheight
%   \gotobox{\MPleft{\the\bottomheight}{\the\bottomheight}}[previouspage]%
%   \kern.5\bottomheight
%   \gotobox{\MPcircle{\the\bottomheight}{\the\bottomheight}}[CloseDocument]%
%   \kern.5\bottomheight
%   \gotobox{\MPright{\the\bottomheight}{\the\bottomheight}}[nextpage]%
%   \kern.5\bottomheight
% \stopinteractionmenu

%D \macros
%D   {TitlePage, Topics, Topic, Subject}
%D
%D A presentation after loading this module looks like:
%D
%D \starttypen
%D \TitlePage {About Whatever\\Topics}
%D
%D \Topics {Todays Talk}
%D
%D \Topic {Some topic}
%D
%D .....
%D
%D \Topic {Next Topic}
%D
%D .....
%D \stoptypen

%D \macros
%D  {StartTitlePage, TitlePage}
%D
%D The titlepage is rather simple and can be typeset in two 
%D ways: 
%D
%D \starttypen
%D \StartTitlePage
%D text \\ text \\ text 
%D \StopTitlepage
%D \stoptypen
%D
%D or as one||liner:
%D
%D \starttypen
%D \TitlePage{text\\text\\text} 
%D \stoptypen
%D 
%D The first alternative can be used for more complicated 
%D title pages.  

\def\StartTitlePage%
  {\startstandardmakeup
   \bfd\setupinterlinespace
   \setupalign[middle]
   \vfil
   \let\\=\vfil}

\def\StopTitlePage%
  {\vfil\vfil\vfil
   \stopstandardmakeup}

\def\TitlePage#1%
  {\StartTitlePage#1\StopTitlePage}

%D \macros
%D   {definehead}
%D
%D The commands \type{\Topic} and \type{\Subject} are defined
%D as copies of head. We use \type{\Nopic} for internal 
%D purposes. 

\definehead [Topic]   [chapter]
\definehead [Subject] [section]

\definehead [Nopic]   [title]

%D \macros
%D   {setuphead}
%D
%D We use our own command for typesetting the titles. We hide
%D sectionnumbers from viewing. Each topic is followed by a
%D list of subjects that belong to the topic.

\setuphead
  [Topic, Nopic]
  [after={\blank[3*medium]},
   number=no,
   style=\tfb,
   page=yes,
   alternative=middle]

\setuphead
  [Subject]
  [after=\blank,
   number=no,
   page=yes,
   continue=no,
   style=\tfa]

%D The next piece of code is new. 

%D \macros
%D   {setuplist}
%D
%D When found, the subject list is automatically placed 
%D after the topic head. 

\setuplist
  [Topic,Subject]
  [alternative=g,
   interaction=all,
   before=,
   after=]

\def\Topics#1% 
  {\determinelistcharacteristics[Topic]
   \ifcase\utilitylistlength\else
     \Nopic[Topics]{#1}
     \startcolumns
     \placelist[Topic][criterium=all]
     \stopcolumns
   \fi}

\def\Subjects% 
  {\determinelistcharacteristics[Subject][criterium=Topic]
   \ifcase\utilitylistlength\else
     \placelist[Subject][criterium=Topic]
   \fi}

\setuphead
  [Topic]
  [after={\blank[3*medium]\Subjects}]

\endinput
