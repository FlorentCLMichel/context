%D \module
%D   [      file=x-fo,
%D        version=2004.03.12, % based on earlier experiments
%D          title=\FOXET,
%D       subtitle=Formatting Objects,
%D         author=Hans Hagen,
%D           date=\currentdate,
%D      copyright={PRAGMA ADE / Hans Hagen \& Ton Otten}]
%C
%C This module is part of the \CONTEXT\ macro||package and is
%C therefore copyrighted by \PRAGMA. See mreadme.pdf for
%C details.

% \showframe

% This is a first implementation, maybe I will write another one with mixed
% element indifferent vars and something 'when set, act upon it, and forget',
% for instance: in each element check if font set, if so, change font and
% reset font attributes. I'm not sure if this is wise.

% todo: global assignment in order to limit restore
% todo: combine mp graphics (see end) saves 30%
% todo: using contants and variables (for internal use)

% todo: language at more levels

% beware: aftergroup vs egroup/endgroup

\useXMLfilter[prs,run]

% \input xtag-run

\unprotect

% syst-new.tex

\long\def\unstringed#1% " ' space
  {\csname\ifcsname @u@s@#1\endcsname @u@s@#1\else\s!empty\fi\endcsname#1}

\long\setvalue{@u@s@"}#1#2"{#2}
\long\setvalue{@u@s@'}#1#2'{#2}
\long\setvalue{@u@s@ }#1#2 {#2}

% xtag-ini

\def\letXMLpar #1#2{\@EA \let\csname\@@XMLvariable:#1:#2\endcsname}
\def\setXMLpar #1#2{\@EA \def\csname\@@XMLvariable:#1:#2\endcsname}
\def\setXMLepar#1#2{\@EA\edef\csname\@@XMLvariable:#1:#2\endcsname}

\protect

%D Most time went into figuring out the specifications, especially
%D because there are no examples included. Samples that circulate on the
%D web are often border cases and torture test and don't have much to do
%D with real live. Another complication lays in the inheritance model:
%D some of the attributes are inherited. This also  leaves some room for
%D interpretation, for instance do values that  are used at a certain
%D point migrate downwards or not.
%D
%D The \CONTEXT\ \XML handler can deal with attributes in several ways
%D and for this purpose I have played with a few experimental mechanisms
%D just to end up with the existing begin/end mechanism combined with
%D a recursive attribute resolver which means that one has to implicitly
%D ask for an inherited attributes. This approach is probably one of the
%D most efficient ways of dealing with formatting objects in \CONTEXT,
%D unless of course I start adding rather specific support to the kernel.
%D
%D This module is rather experimental. More information about its usage
%D can be found in the \FOXET\ manual.

%D Since we're not dealing with the fine points of typesetting here, we
%D can safely ignore \TEX's warnings about overful or underful boxes.

\dontcomplain

%D We will use fonts that have the characters in the normal (ascii)
%D slots. We will also use the stupid verbatim handler.

\chardef\XMLtokensreduction = 2
\chardef\XMLcdatamethod     = 2

%D For the purpose of testing.

\startmode[fo-verbose]
  \def\writeFOstatus{\writestatus{XML-FO}}
\stopmode

\startnotmode[fo-verbose]
  \let\writeFOstatus\gobbleoneargument
\stopnotmode

%D For the moment we stick to utf-8.

\useXMLfilter[utf]

%D This will be sorted out later (esp in relation to mathml).

\setupbodyfont[pos,10pt]

%D There are a couple of predefined colors. Don't ask me why, but
%D formatting objects are not a fresh start but a mix of existing
%D technologies. Color support is poluted by cascading stylesheets.
%D
%D Because hexadecimal color specifications are not enabled by
%D default, this feature has to be enables by loading the appropriate
%D color module. Here we define colors in \RGB\ values because we
%D don't want to loose accuracy.

\setupcolors[state=start] \setupcolor[hex]

\definecolor [black]   [s=0]       % [h=000000]
\definecolor [gray]    [s=.5]      % [h=808080]
\definecolor [silver]  [s=.75]     % [h=C0C0C0]
\definecolor [white]   [s=1]       % [h=FFFFFF]
\definecolor [maroon]  [r=.5]      % [h=800000]
\definecolor [red]     [r=1]       % [h=FF0000]
\definecolor [purple]  [r=.5,b=.5] % [h=800080]
\definecolor [fuchsia] [r=1,b=1]   % [h=FF00FF]
\definecolor [green]   [g=.5]      % [h=008000]
\definecolor [lime]    [g=1]       % [h=00FF00]
\definecolor [olive]   [r=.5,g=.5] % [h=808000]
\definecolor [yellow]  [r=1,g=1]   % [h=FFFF00]
\definecolor [navy]    [r=1,g=1]   % [h=000080]
\definecolor [blue]    [b=1]       % [h=0000FF]
\definecolor [teal]    [g=.5,b=.5] % [h=008080]
\definecolor [aqua]    [g=1,b=1]   % [h=00FFFF]

%D The layout is rather basic. Of the 25 available areas we
%D only use the text area. Maybe some day I will plug in a
%D more dedicated page builder.

\setuplayout
  [backspace=0pt,
   topspace=0pt,
   header=0pt,
   footer=0pt,
   width=middle,
   height=middle,
   % marking=on,
   location=middle]

\setuppagenumbering
  [alternative={doublesided,singlesided}, % sic
   location=]

\setuptolerance
  [verytolerant,stretch]

%D We will position the regions using layers.

\definelayer[regions][width=\paperwidth,height=\paperheight]

\definelayer[xsl-region-before]
\definelayer[xsl-region-after]
\definelayer[xsl-region-start]
\definelayer[xsl-region-end]
\definelayer[xsl-region-body]

\setupbackgrounds[page][background=regions]

%D We now enter the part of this module where the action takes
%D place. As usual we provide some tracing options.

\newif\iftracingFO \readsysfile{page-run}\donothing\donothing

%D We will organize the attribute definitions in a similar fashion  as in
%D the specification. Unfortunately there are more sets defined  in there
%D than are actually used, so the definitions later on will  look a bit
%D messy.
%D
%D Quite some attributes can be inherited, which means that they can
%D end up in all elements and influence those way down the tree.

\defineXMLattributeset
  [fo:inherited]

%D The properties:

% \defineXMLattributeset
%   [fe:tracing]
%   [tracing=]

\defineXMLattributeset
  [fo:accessibility]
  [source-document=none,
   role=none]

\defineXMLattributeset
  [fo:absolute-position]
  [absolute-position=auto,
   top=auto,
   right=auto,
   bottom=auto,
   left=auto]

% \defineXMLattributeset
%   [fo:aural]
%   [azitmuth=,
%    cue-after=,
%    cue-before=,
%    elevation=,
%    pause-after=,
%    pause-before=,
%    pitch=,
%    pitch-range=,
%    play-during=,
%    richness=,
%    speak=,
%    speak-header=,
%    speak-numeral=,
%    speak-punctuation=,
%    speech-rate=,
%    stress=,
%    voice-family=,
%    volume=]

\defineXMLattributeset
  [fo:border-padding-background]
  [background-attachment=scroll,
   background-color=transparent,
   background-image=none,
   background-repeat=repeat,
   background-position-horizontal=left,
   background-position-vertical=top,
   border-color=transparent,
   border-style=none,
   border-width=medium,
   background-position=,
   border-top=,
   border-bottom=,
   border-left=,
   border-right=,
   border-before-color=\XMLop{border-color},
   border-before-style=\XMLop{border-style},
   border-before-width=\XMLop{border-width},
   border-after-color=\XMLop{border-color},
   border-after-style=\XMLop{border-style},
   border-after-width=\XMLop{border-width},
   border-start-color=\XMLop{border-color},
   border-start-style=\XMLop{border-style},
   border-start-width=\XMLop{border-width},
   border-end-color=\XMLop{border-color},
   border-end-style=\XMLop{border-style},
   border-end-width=\XMLop{border-width},
   border-top-color=\XMLop{border-before-color},
   border-top-style=\XMLop{border-before-style},
   border-top-width=\XMLop{border-before-width},
   border-bottom-color=\XMLop{border-after-color},
   border-bottom-style=\XMLop{border-after-style},
   border-bottom-width=\XMLop{border-after-width},
   border-left-color=\XMLop{border-start-color},
   border-left-style=\XMLop{border-start-style},
   border-left-width=\XMLop{border-start-width},
   border-right-color=\XMLop{border-end-color},
   border-right-style=\XMLop{border-end-style},
   border-right-width=\XMLop{border-end-width},
   padding=,% 0pt,
   padding-before=0pt,%\XMLop{padding},
   padding-after=0pt,%\XMLop{padding},
   padding-start=0pt,%\XMLop{padding},
   padding-end=0pt,%\XMLop{padding},
   padding-top=\XMLop{padding-before},
   padding-bottom=\XMLop{padding-after},
   padding-left=\XMLop{padding-start},
   padding-right=\XMLop{padding-end}]

\extendXMLattributeset
  [fo:border-padding-background]
  [fe:background-height=,
   fe:background-width=]

\defineXMLattributeset
  [fo:font]
  []

\extendXMLattributeset
  [fo:inherited]
  [font=,
   font-family=,% Times,
   font-selection-strategy=,
   font-size=,% 12pt,
   font-size-adjust=, % 1,
   font-style=, % normal,
   font-variant=, % normal,
   font-weight=] % normal]

\defineXMLattributeset
  [fo:hyphenation]
  []

\extendXMLattributeset
  [fo:inherited]
  [country=,
   language=,
   script=,
   hyphenate=,
   hyphenation-character=,
   hyphenation-push-character-count=,
   hyphenation-remain-character-count=]

\defineXMLattributeset
  [fo:margin-block]
  [margin=, % 0pt,
   margin-top=0pt,% \XMLop{margin},
   margin-bottom=0pt,% \XMLop{margin},
   margin-left=0pt,% \XMLop{margin},
   margin-right=0pt,% \XMLop{margin},
   space-before=0pt,
   space-after=0pt,
   space-before.precedence=,
   space-before.conditionality=,
   space-before.minimum=,
   space-before.optimum=,
   space-before.maximum=,
   space-after.precedence=,
   space-after.conditionality=,
   space-after.minimum=,
   space-after.optimum=,
   space-after.maximum=]

\extendXMLattributeset
  [fo:inherited]
  [start-indent=,
   end-indent=]

\defineXMLattributeset
  [fo:margin-inline]
  [space-start=,
   space-end=]

\defineXMLattributeset
  [fo:relative-position]
  [relative-position=,
   top=auto,
   right=auto,
   bottom=auto,
   left=auto]

\defineXMLattributeset
  [fo:area-alignment]
  [alignment-adjust=,
   alignment-baseline=,
   baseline-shift=,
   dominant-baseline=]

\extendXMLattributeset
  [fo:inherited]
  [display-align=,
   relative-align=]

\defineXMLattributeset
  [fo:area-dimension]
  [block-progression-dimension=,
   inline-progression-dimension=,
   content-height=,
   content-width=,
   height=,
   width=,
   max-height=,
   max-width=,
   min-height=,
   min-width=,
   scaling=,
   scaling-method=]

\defineXMLattributeset
  [fo:block-and-line]
  []

\extendXMLattributeset
  [fo:inherited]
  [hyphenation-keep=,
   hyphenation-ladder-count=,
   last-line-end-indent=,
   line-height=,
   line-height-shift-adjustment=,
   line-stacking-strategy=,
   linefeed-treatment=,
   white-space-treatment=,
   text-align=,
   text-align-last=,
   text-indent=,
   white-space-collapse=,
   wrap-option=]

\defineXMLattributeset
  [fo:character]
  [character=,
   suppress-at-line-break=,
   text-decoration=,
   text-shadow=,
   treat-as-word-space=]

\extendXMLattributeset
  [fo:inherited]
  [letter-spacing=,
   text-transform=,
   word-spacing=]

\defineXMLattributeset
  [fo:color]
  [color-profile-name=,
   rendering-intent=]

\extendXMLattributeset
  [fo:inherited]
  [color=]

\defineXMLattributeset
  [fo:float]
  [clear=,
   float=]

\extendXMLattributeset
  [fo:inherited]
  [intrusion-displace=]

\defineXMLattributeset
  [fo:keeps-and-breaks]
  [break-after=,
   break-before=,
   keep-with-next.within-line=,
   keep-with-next.within-column=,
   keep-with-next.within-page=,
   keep-with-previous.within-line=,
   keep-with-previous.within-column=,
   keep-with-previous.within-page=]

\extendXMLattributeset
  [fo:inherited]
  [keep-together.within-line=,
   keep-together.within-column=,
   keep-together.within-page=,
   orphans=,
   widows=]

\defineXMLattributeset
  [fo:layout]
  [clip=,
   overflow=
   span=]

\extendXMLattributeset
  [fo:inherited]
  [reference-orientation=]

\defineXMLattributeset
  [fo:leader-and-rule]
  []

\extendXMLattributeset
  [fo:inherited]
  [leader-alignment=,
   leader-pattern=,
   leader-pattern-width=,
   leader-pattern-width=,
   leader-length=,
   rule-style=,
   rule-thickness=]

\defineXMLattributeset
  [fo:dynamic-effects]
  [active-state=,
   case-name=,
   case-title=,
   destination-placement-offset=,
   external-destination=,
   indicate-destination=,
   internal-destination=,
   show-destination=,
   starting-state=,
   switch-to=,
   target-presentation-context=,
   target-processing-context=,
   target-stylesheet=]

\extendXMLattributeset
  [fo:inherited]
  [auto-restore=]

\defineXMLattributeset
  [fo:markers]
  [marker-class-name=,
   retrieve-class-name=,
   retrieve-position=,
   retrieve-boundary=]

\defineXMLattributeset
  [fo:number-to-string]
  [country=,
   language=,
   format=,
   grouping-separator=,
   grouping-size=,
   letter-value=]

% \defineXMLattributeset
%   [fo:pagination-and-layout]
%   [black-or-not-blank=,
%    column-count=1,
%    column-gap=12pt,
%    extent=,
%    flow-name=,
%    force-page-count=,
%    initial-page-number=,
%    master-name=,
%    master-reference=,
%    maximum-repeats=,
%    media-usage=,
%    odd-or-even=,
%    page-height=,
%    page-position=,
%    page-width=,
%    precedence=,
%    region-name=]

\defineXMLattributeset
  [fo:table]
  [border-after-precedence=,
   border-before-precedence=,
   border-end-precedence=,
   border-start-precedence=,
   column-number=,
   column-width=,
   ends-row=,
   number-columns-repeated=,
   number-columns-spanned=,
   number-rows-spanned=,
   starts-row=,
   table-layout=,
   table-omit-footer-at-break=,
   table-omit-header-at-break=]

\extendXMLattributeset
  [fo:inherited]
  [border-collapse=,
   border-separation=,
   caption-side=,
   empty-cells=]

\defineXMLattributeset
  [fo:writing-mode]
  [%text-altitude=,
   %text-depth=,
   unicode-bidi=]

\extendXMLattributeset % for practical reasons we inherit along the whole chain
  [fo:inherited]       % unless we implement relax skipping
  [text-altitude=,
   text-depth=]

\extendXMLattributeset
  [fo:inherited]
  [direction=,
   glyph-orientation-horizontal=,
   glyph-orientation-vertical=,
   writing-mode=]

\defineXMLattributeset
  [fo:list-block]
  []

\extendXMLattributeset
  [fo:inherited]
  [provisional-label-separation=,
   provisional-distance-between-starts=]

% \starttext
% \setuplayout[topspace=1cm,height=middle,header=0pt,footer=0pt]
% \setupbodyfont[small,tt]
% \expanded{\processcommalist[\XMLattributeset{fo:inherited}]}\endgraf
% \stoptext

% content-type
% id

% ref-id
% score-spaces % inherited
% src
% visibility % inherited
% z-index

% shorthands:
%
% background
% background-position
% border
% border-bottom
% border-left
% border-top
% border-right
% border-style
% border-color
% border-spacing  % inherited
% border-width
% cue
% font % inherited
% margin
% padding
% page-break-after
% page-break-before
% page-break-inside % inherited
% pause
% position
% size
% vertical-align
% white-space % inherited

%D We will speed up the process of setting up attributes by compiling the
%D definitions. Sometimes we need to access attributes explicitly by
%D element (for instance when handling regions). We also need to deal
%D with nested elements  (for instance blocks) or a sequence of similar
%D ones, while we may not always want to use grouping. As a result, the
%D next series of definitions and macros are quite ugly. The begin|/|end
%D is needed in order to comfortably fetch attribute values from
%D ancestors.

\startXMLcompiling[inherit]

%D Element: fo:root

%D todo: set defaults here

\defineXMLenvironment
  [fo:root]
  [\XMLattributeset{fo:inherited},
leader-pattern=spaces,
leader-pattern-width=12pt,
   media-usage=]
  {\directsetup{fo:root:start}}
  {\directsetup{fo:root:stop}}

\startsetups fo:root:start
  \starttext \beginXMLelement \startXMLignore
  \doifsomething{\XMLop{language}}{\mainlanguage[\XMLop{language}]}% todo, everywhere
\stopsetups

\startsetups fo:root:stop
  \stopXMLignore \endXMLelement \stoptext
\stopsetups

%D Element: fo:declarations

\defineXMLprocess
  [fo:declarations]

%D Element: fo:color-profile

\defineXMLignore
  [fo:color-profile]
  [src=,
   color-profile-name=,
   rendering-intent=]

%D Element: fo:page-sequence

% master-name and master-reference are often mixed up in examples

\defineXMLenvironment
  [fo:page-sequence]
  [\XMLattributeset{fo:inherited},
   id=,
   \XMLattributeset{fe:tracing},
   \XMLattributeset{fo:number-to-string},
   initial-page-number=auto,
   master-reference=any,
   force-page-count=auto]
  {\beginXMLelement\directsetup{fo:page-sequence:start}}
  {\directsetup{fo:page-sequence:stop}\endXMLelement}

\mapXMLvalue {fo:page-initial} {auto}        {\relax}
\mapXMLvalue {fo:page-initial} {auto-odd}    {\ifodd\pageno     \expanded{\setuppagenumber[number=\the\dimexpr(\pageno+1)]}\fi}
\mapXMLvalue {fo:page-initial} {auto-even}   {\ifodd\pageno\else\expanded{\setuppagenumber[number=\the\dimexpr(\pageno+1)]}\fi}

% todo: blokkeer left/right/etc in geval van blank

\mapXMLvalue {fo:page-start}   {auto}        {\page}
\mapXMLvalue {fo:page-start}   {even}        {\page\setuplayout[blank]\page[even]}
\mapXMLvalue {fo:page-start}   {odd}         {\page\setuplayout[blank]\page[odd]}
\mapXMLvalue {fo:page-end}     {end-on-even} {\page\setuplayout[blank]\page[even]}
\mapXMLvalue {fo:page-end}     {end-on-odd}  {\page\setuplayout[blank]\page[odd]}
\mapXMLvalue {fo:page-start}   {no-force}    {\page}

\startsetups fo:page-sequence:start

  % we're still in the previous page-sequence

  \XMLval{fo:page-start}{\XMLop{force-page-count}}{\page} \begingroup

  \directsetup{fe:setup}

  \iftracingFO \showframe[text] \fi

  \xdef\SavedPageNumber{\the\realpageno}

  % now we enter the new page-sequence

  % todo: check on number

  \doifXMLvalelse{fo:page-initial}{\XMLop{initial-page-number}}
    {\XMLval{fo:page-initial}{\XMLop{initial-page-number}}{}}
    {\doifnot{\XMLop{initial-page-number}}{auto}
       {\expanded{\setuppagenumber[number=\XMLop{initial-page-number}]}}}

  \doifsomething{\XMLpar{fo:page-sequence}{master-reference}{}}
    {\doifelseXMLelement{fo:page-sequence-master:\XMLpar{fo:page-sequence}{master-reference}{}}
       {%[starting page sequence master: \XMLpar{fo:page-sequence}{master-reference}{}]\endgraf
        \flushXMLelement{fo:page-sequence-master:\XMLpar{fo:page-sequence}{master-reference}{}}}
       {%[starting simple page master: \XMLpar{fo:page-sequence}{master-reference}{any}]\endgraf
        \flushXMLelement{fo:simple-page-master:\XMLpar{fo:page-sequence}{master-reference}{any}}}}

  \doif{\XMLpar{fo:simple-page-master-do}{fe:option}{}}{fit}
    {\directsetup{fe:page:option:fit:start}}

  \doifnot{\XMLpar{fo:region-body}{column-count}{1}}{1}
    {\directsetup{fo:columns:start}}

\stopsetups

\startsetups fo:page-sequence:stop

  \doifnot{\XMLpar{fo:region-body}{column-count}{1}}{1}
    {\directsetup{fo:columns:stop}}

  \doif{\XMLpar{fo:simple-page-master-do}{fe:option}{}}{fit}
    {\directsetup{fe:page:option:fit:stop}}

  % \XMLval{fo:page-end}{\XMLop{force-page-count}}{\page} \endgroup

  \ifnum\SavedPageNumber=\realpageno

    \ifdim\pagetotal<.5\textheight \null \vfill \fi % force a page with only containers

  \fi

  \XMLval{fo:page-end}{\XMLpar{fo:page-sequence}{force-page-count}{}}{\page} \endgroup

\stopsetups

\newdimen\FOcolumngap

\startsetups fo:columns:start

  \FOcolumngap\textwidth

  \setpercentdimen\FOcolumngap{\XMLpar{fo:region-body}{column-gap}{12pt}}

  % we need to freeze the lineheight here

  \expanded{\definecolumnset
    [fo:set]
    [n=\XMLpar{fo:region-body}{column-count}{1},
     distance=\FOcolumngap]}

  \expanded{\definecolumnsetspan
    [fo:set]
    [n=\XMLpar{fo:region-body}{column-count}{1}]}

  \directsetup{fo:font:setup} % else problems

  \directsetup{fo:line-height:setup}

%   \parseXMLattributes{fo:flow}{line-height='normal'}

  \startcolumnset[fo:set]

  % \startcolumns[\XMLpar{fo:region-body}{column-count}{1}]

\stopsetups

\startsetups fo:columns:stop

  % \stopcolumns

  \stopcolumnset

\stopsetups

%D Element: fo:layout-master-set

\defineXMLprocess
  [fo:layout-master-set]

%D Element: fo:page-sequence-master

\defineXMLenvironmentsave
  [fo:page-sequence-master]
  [\XMLattributeset{fo:inherited},
   master-name=any]
  {}
  {%[saved page sequence master: \XMLop{master-name}]\endgraf
   \gsaveXMLdatainelement
     {fo:page-sequence-master:\XMLop{master-name}}
     {fo:page-sequence-master-do}
     {fo:page-sequence-master}}

\defineXMLprocess
  [fo:page-sequence-master-do]
  [\XMLattributeset{fo:inherited}]

%D Element: fo:single-page-master-reference

% makeup - one page

\defineXMLcommand
  [fo:single-page-master-reference]
  [master-reference=any]
  {\flushXMLelement{fo:simple-page-master:\XMLop{master-reference}}}

%D Element: fo:repeatable-page-master-reference

\defineXMLcommand
  [fo:repeatable-page-master-reference]
  [master-reference=any,
   maximum-repeats=]
  {\flushXMLelement{fo:simple-page-master:\XMLop{master-reference}}}

%D Element: fo:repeatable-page-master-alternatives

\defineXMLprocess
  [fo:repeatable-page-master-alternatives]
  [maximum-repeats=]

%D Element: fo:conditional-page-master-reference

% page-position : first last rest any
% odd-or-even : odd even any
% blank-or-not-blank : blank not-blank

% The page-position default is needed (else possible loops)

\defineXMLcommand
  [fo:conditional-page-master-reference]
  [master-reference=any,
   page-position=\XMLpar{fo:conditional-page-master-reference}{master-reference}{any},
   blank-or-not-blank=,
   odd-or-even=]
  {\flushXMLelement{fo:simple-page-master:\XMLpar{fo:conditional-page-master-reference}{master-reference}{}}}

%D Element: fo:simple-page-master

% first-page left-page right-page blank-page

% default dimensions

\defineXMLenvironmentsave
  [fo:simple-page-master]
  [master-name=any]
  {}
  {%[saved simple page master: \XMLop{master-name}]\endgraf
   \gsaveXMLdatainelement
     {fo:simple-page-master:\XMLop{master-name}}
     {fo:simple-page-master-do}
     {fo:simple-page-master}}

% reference-orientation=0deg,
% writing-mode=

\defineXMLenvironment
  [fo:simple-page-master-do]
  [\XMLattributeset{fo:inherited}, % added
   \XMLattributeset{fo:margin-block},
   page-height=29.7cm,
   page-width=21cm]
  {\directsetup{fo:simple-page-master:start}}
  {\directsetup{fo:simple-page-master:stop}}

% not needed any more:

\mapXMLvalue {fo:reference-orientation}    {0deg}    {0}
\mapXMLvalue {fo:reference-orientation}   {90deg}   {90}
\mapXMLvalue {fo:reference-orientation}  {180deg}  {180}
\mapXMLvalue {fo:reference-orientation}  {270deg}  {270}
\mapXMLvalue {fo:reference-orientation}  {-90deg}  {270}
\mapXMLvalue {fo:reference-orientation} {-180deg}  {180}
\mapXMLvalue {fo:reference-orientation} {-270deg}   {90}

\startsetups fo:simple-page-master:start

  % nothing

\stopsetups

% can be low level tex

\startsetups fo:simple-page-master:stop

  \writeFOstatus{defining papersize '\directsetup{fo:layout:kind}'}

  \expanded
    {\definepapersize
       [\directsetup{fo:layout:kind}]
       [width=\XMLop{page-width},
        height=\XMLop{page-height}]}

  \checkFOpadding {fo:region-body}
  \checkFOmargin  {fo:region-body}
  \checkFOmargin  {fo:simple-page-master-do}

  \writeFOstatus{defining layout '\directsetup{fo:layout:kind}'}

  \FOscratchMT\paperheight \setpercentdimen\FOscratchMT{\XMLpar{fo:simple-page-master-do}{margin-top}   \zeropoint}
  \FOscratchMB\paperheight \setpercentdimen\FOscratchMB{\XMLpar{fo:simple-page-master-do}{margin-bottom}\zeropoint}
  \FOscratchML\paperwidth  \setpercentdimen\FOscratchML{\XMLpar{fo:simple-page-master-do}{margin-left}  \zeropoint}
  \FOscratchMR\paperwidth  \setpercentdimen\FOscratchMR{\XMLpar{fo:simple-page-master-do}{margin-right} \zeropoint}

  \FOscratchRB\paperheight \setpercentdimen\FOscratchRB{\XMLpar{fo:region-body}{margin-top}   \zeropoint}
  \FOscratchRA\paperheight \setpercentdimen\FOscratchRA{\XMLpar{fo:region-body}{margin-bottom}\zeropoint}
  \FOscratchRS\paperwidth  \setpercentdimen\FOscratchRS{\XMLpar{fo:region-body}{margin-left}  \zeropoint}
  \FOscratchRE\paperwidth  \setpercentdimen\FOscratchRE{\XMLpar{fo:region-body}{margin-right} \zeropoint}

  \FOscratchPB\paperheight \setpercentdimen\FOscratchPB{\XMLpar{fo:region-body}{padding-top}   \zeropoint}
  \FOscratchPA\paperheight \setpercentdimen\FOscratchPA{\XMLpar{fo:region-body}{padding-bottom}\zeropoint}
  \FOscratchPS\paperwidth  \setpercentdimen\FOscratchPS{\XMLpar{fo:region-body}{padding-left}  \zeropoint}
  \FOscratchPE\paperwidth  \setpercentdimen\FOscratchPE{\XMLpar{fo:region-body}{padding-right} \zeropoint}

  \expanded
    {\definelayout
       [\directsetup{fo:layout:kind}]
       [       page={\directsetup{fo:layout:kind},\XMLval{fo:reference-orientation}{\XMLop{reference-orientation}}{}},
              paper=\directsetup{fo:layout:kind},
          backspace=\the\dimexpr(\FOscratchML+\FOscratchPS+\FOscratchRS),
           cutspace=\the\dimexpr(\FOscratchMR+\FOscratchPE+\FOscratchRE),
           topspace=\the\dimexpr(\FOscratchMT+\FOscratchPB+\FOscratchRB),
        bottomspace=\the\dimexpr(\FOscratchMB+\FOscratchPA+\FOscratchRA)]}

  \expanded{\setuplayout[\directsetup{fo:layout:kind}]}

  % this is a nasty bit of code: this local setup stores some data that
  % needs to be used later

  \startexpanded
    \noexpand \startlocalsetups[layout:\directsetup{fo:layout:kind}]
    \noexpand   \writeFOstatus{processing simple page master '\XMLpar{fo:simple-page-master-do}{master-name}{any}'}
    \noexpand   \resetsetups[fo:simple-page-master:start]
    \noexpand   \resetsetups[fo:simple-page-master:stop]
    \noexpand   \flushXMLelement{fo:simple-page-master:\XMLpar{fo:simple-page-master-do}{master-name}{any}}
    \noexpand \stoplocalsetups
  \stopexpanded

\stopsetups

% \defineXMLcommand[fo:simple-page-master-do-do]
%   {\writeFOstatus{setting up master \XMLop{master-name} in page body}}

%D Element: fo:region-body

% display-align=,
% reference-orientation=,
% writing-mode=,

\defineXMLcommand % or process
  [fo:region-body]
  [\XMLattributeset{fo:border-padding-background},
   \XMLattributeset{fo:margin-block},
   fe:z-order=above,
   clip=,
   column-count=1,
   column-gap=12pt,
   overflow=,
   region-name=]
  {\directsetup{fo:region-body:process}}

\startsetups fo:region-body:process
  \writeFOstatus{refreshing region-body parameters}
\stopsetups

% todo: naar realfolio handelen ipv folio

\mapXMLvalue {fo:odd-or-even}   {odd}   {odd}
\mapXMLvalue {fo:odd-or-even}   {even}  {even}

\mapXMLvalue {fo:page-position} {any}   {rest}    % todo
\mapXMLvalue {fo:page-position} {first} {current}
\mapXMLvalue {fo:page-position} {last}  {last}    % todo
\mapXMLvalue {fo:page-position} {rest}  {rest}    % todo

%mapXMLvalue {fo:blank-or-not-blank} {any}       {} % todo
%mapXMLvalue {fo:blank-or-not-blank} {not-blank} {} % todo
%mapXMLvalue {fo:blank-or-not-blank} {blank}     {} % todo

% check this one esp default value

\startsetups fo:layout:kind

  \XMLpav
    {fo:odd-or-even}
    {fo:conditional-page-master-reference}
    {odd-or-even}
    {\XMLpav
       {fo:page-position}
       {fo:conditional-page-master-reference}
       {page-position}
       {\XMLpar{fo:page-sequence-master}{master-reference}{any}}}

\stopsetups

% common border things

% clip
% display-align
% extent
% overflow
% precedence
% region-name
% reference-orientation
% writing-mode

%D Element: fo:region-before fo:region-after fo:region-start fo:region-end

% border-before-color : <color> | inherit
% border-before-style : <border style> | inherit
% border-before-width.length|conditional : <border width> | <length conditional> | inherit
%
% style: none hidden dotted dashed solid double groove ridge inset outset
% width: thin medium thick length

\mapXMLvalue {fo:display-align} {auto}   {before} % todo: related to relative-align
\mapXMLvalue {fo:display-align} {before} {high}
\mapXMLvalue {fo:display-align} {after}  {low}
\mapXMLvalue {fo:display-align} {center} {lohi}

% display-align=,
% reference-orientation=,
% writing-mode=,

\defineXMLcommand % will become process when stable
  [fo:region-before]
  [\XMLattributeset{fo:border-padding-background},
   clip=,
   extent=,
   overflow=,
   precedence=,
   region-name=]
  {\directsetup{fo:region-before:process}}

\startsetups fo:region-before:process
  \writeFOstatus{refreshing region-before parameters}
\stopsetups

\defineXMLprocess
  [fo:region-after]
  [\XMLattributeset{fo:border-padding-background},
   clip=,
   extent=,
   overflow=,
   precedence=,
   region-name=]

\defineXMLprocess
  [fo:region-start]
  [\XMLattributeset{fo:border-padding-background},
   clip=,
   extent=,
   overflow=,
   region-name=]

\defineXMLprocess
  [fo:region-end]
  [\XMLattributeset{fo:border-padding-background},
   clip=,
   extent=,
   overflow=,
   region-name=]

\mapXMLvalue {fo:border-style} {none}   {0}
\mapXMLvalue {fo:border-style} {hidden} {1}
\mapXMLvalue {fo:border-style} {dotted} {2}
\mapXMLvalue {fo:border-style} {dashed} {3}
\mapXMLvalue {fo:border-style} {solid}  {4}
\mapXMLvalue {fo:border-style} {double} {5}
\mapXMLvalue {fo:border-style} {groove} {6}
\mapXMLvalue {fo:border-style} {ridge}  {7}
\mapXMLvalue {fo:border-style} {inset}  {8}
\mapXMLvalue {fo:border-style} {outset} {9}

\mapXMLvalue {fo:border-width} {thin}   {.25pt}
\mapXMLvalue {fo:border-width} {medium} {.5pt}
\mapXMLvalue {fo:border-width} {thick}  {1pt}

\startsetups fo:regions:check

  \startprocesscommalist[body,before,after,start,end]

    \checkFOborder{fo:region-\currentcommalistitem}{bottom}
    \checkFOborder{fo:region-\currentcommalistitem}{top}
    \checkFOborder{fo:region-\currentcommalistitem}{left}
    \checkFOborder{fo:region-\currentcommalistitem}{right}

    \checkhexcolor[\XMLpar{fo:region-\currentcommalistitem}{border-bottom-color}{}]
    \checkhexcolor[\XMLpar{fo:region-\currentcommalistitem}{border-top-color}{}]
    \checkhexcolor[\XMLpar{fo:region-\currentcommalistitem}{border-left-color}{}]
    \checkhexcolor[\XMLpar{fo:region-\currentcommalistitem}{border-right-color}{}]

    \checkhexcolor[\XMLpar{fo:region-\currentcommalistitem}{background-color}{}]

    \checkFOposition{fo:region-\currentcommalistitem}{background}
    \checkFOpadding {fo:region-\currentcommalistitem}
    \checkFOmargin  {fo:region-\currentcommalistitem}

  \stopprocesscommalist

\stopsetups

%D Element: fo:flow

\defineXMLenvironment
  [fo:flow]
  [\XMLattributeset{fo:inherited},
   flow-name=unknown]
  {\beginXMLelement\directsetup{fo:flow:start}}
  {\directsetup{fo:flow:stop}\endXMLelement}

\startsetups fo:flow:start
    \begingroup
\stopsetups

\startsetups fo:flow:stop
    \endgroup
\stopsetups

%D Element: fo:static-content

% \beginXMLelement \endXMLelement - maybe save with attributes

\defineXMLenvironmentsave
  [fo:static-content]
  [flow-name=unknown]
  {}
  {\directsetup{fo:static-content:process}}

\startsetups fo:static-content:process

  \gsaveXMLdata{fo:static-content:\XMLop{flow-name}}{fo:static-content}

\stopsetups

\newdimen\FOscratchML \newdimen\FOscratchMR \newdimen\FOscratchMT \newdimen\FOscratchMB
\newdimen\FOscratchPB \newdimen\FOscratchPA \newdimen\FOscratchPS \newdimen\FOscratchPE
\newdimen\FOscratchRB \newdimen\FOscratchRA \newdimen\FOscratchRS \newdimen\FOscratchRE

\mapXMLvalue {fo:background-repeat} {no-repeat} {0}
\mapXMLvalue {fo:background-repeat} {repeat}    {1}
\mapXMLvalue {fo:background-repeat} {repeat-x}  {2}
\mapXMLvalue {fo:background-repeat} {repeat-y}  {3}

\expanded {\mapXMLvalue {fo:background-location} {left}     {0\letterpercent}}
\expanded {\mapXMLvalue {fo:background-location} {right}  {100\letterpercent}}
\expanded {\mapXMLvalue {fo:background-location} {top}      {0\letterpercent}}
\expanded {\mapXMLvalue {fo:background-location} {bottom} {100\letterpercent}}
\expanded {\mapXMLvalue {fo:background-location} {center}  {50\letterpercent}}

\newdimen\FObgpositionH
\newdimen\FObgpositionV

\def\FObackgroundimage#1#2#3%
  {\doifnot{\XMLpar{fo:#1}{background-image}{none}}{none}
     {\setFOimagename{\XMLpar{fo:#1}{background-image}{dummy}}%
      \FObgpositionH#2%
      \setpercentdimen\FObgpositionH{\XMLpav
        {fo:background-location}
        {fo:#1}
        {background-position-horizontal}
        {\XMLpar{fo:#1}{background-position-horizontal}{}}}%
      \ifpercentdimendone
        \skip0\zeropoint plus \FObgpositionH
        \skip2\zeropoint plus \dimexpr(#2-\FObgpositionH)%
      \else
        \skip0\FObgpositionH
        \skip2\zeropoint plus 1fill\relax
      \fi
      \FObgpositionV#3%
      \setpercentdimen\FObgpositionV{\XMLpav
        {fo:background-location}{fo:#1}{background-position-vertical}
        {\XMLpar{fo:#1}{background-position-vertical}{}}}%
      \ifpercentdimendone
        \skip4\zeropoint plus \FObgpositionV
        \skip6\zeropoint plus \dimexpr(#3-\FObgpositionV)%
      \else
        \skip4\FObgpositionV
        \skip6\zeropoint plus 1fill\relax
      \fi
      \vbox to #3 \bgroup
        \vskip\skip4\relax
        \hbox to #2 \bgroup
          \hskip\skip0\relax
          \backgroundimage
            {\XMLpav{fo:background-repeat}{fo:#1}{background-repeat}{}}{#2}{#3}%
            {\externalfigure
               [\FOimagename]
               [width=\XMLpar{fo:#1}{fe:background-width}{},
                height=\XMLpar{fo:#1}{fe:background-height}{}]}%
          \hskip\skip2\relax
        \egroup
        \vskip\skip6\relax
      \egroup}}

\def\checkFOclipping#1%
  {\doifsomething{\XMLpar{#1}{clip}{}}
     {\analyzefunction{\XMLpar{#1}{clip}{}}%
      \doif\functionname{rect}
        {\def\postprocessframebox##1%
           {\edef\next{\dimen0=\the\wd##1\dimen2=\the\ht##1\dimen4=\the\dp##1}%
            \setbox##1\hbox
              {\clip % expanded?
                 [topoffset=-\functionA,
                  bottomoffset=-\functionC,
                  leftoffset=-\functionD,
                  rightoffset=-\functionB]
                 {\box##1}}%
            \next}}}}

\def\clipFOarea#1%
  {\doifsomething{\XMLpar{#1}{clip}{}}%
     {\analyzefunction{\XMLpar{#1}{clip}{}}%
      \doif\functionname{rect}
        {\setbox\scratchbox\hbox{\foregroundbox}%
         \edef\next{\dimen0=\the\wd\scratchbox\dimen2=\the\ht\scratchbox\dimen4=\the\dp\scratchbox}%
         \setbox\scratchbox\hbox
           {\clip % expanded?
              [topoffset=-\functionA,
               bottomoffset=-\functionC,
               leftoffset=-\functionD,
               rightoffset=-\functionB]
              {\box\scratchbox}}%
         \next
         \box\scratchbox}}}

\defineoverlay
  [text]
  [\clipFOarea{fo:region-body}]

\setupbackgrounds
  [text]
  [background=text]

\def\FOregionbuilder#1#2#3#4#5#6% #1=location #2=preset #3=x #4=y #5=width #6=height
  {\writeFOstatus{building region #1}%
   \defineoverlay
     [image]
     [{\framed
         [frame=off,
          orientation=\XMLpav{fo:reference-orientation}{fo:#1}{reference-orientation}{0},
          offset=overlay,
          height=\overlayheight,
          width=\overlaywidth]
         {\FObackgroundimage{#1}\hsize\vsize}}]%
   \setlayerframed
     [regions]
     [preset=#2,x=\dimexpr(#3),y=\dimexpr(#4)]
     [frame=off,
      offset=overlay,background={#1-graphic,image,xsl-#1},
      width=\dimexpr(#5),height=\dimexpr(#6)]
     {\lrtbbox
        {\XMLpar{fo:#1}{padding-left}\zeropoint}%
        {\XMLpar{fo:#1}{padding-right}\zeropoint}%
        {\XMLpar{fo:#1}{padding-top}\zeropoint}%
        {\XMLpar{fo:#1}{padding-bottom}\zeropoint}%
        {\checkFOclipping{fo:#1}%
         \framed
           [frame=off,
            offset=overlay,
            orientation=\XMLpav{fo:reference-orientation}{fo:#1}{reference-orientation}{0},
            align={\XMLpav{fo:display-align}{fo:#1}{display-align}{high},\XMLpav{fo:align-key}{fo:#1}{text-align}{normal}},
            width=\hsize,height=\vsize]
           {\doFObeforeskip{fo:#1}%
            \flushXMLelement{fo:static-content:\XMLpar{fo:#1}{region-name}{xsl-#1}}}%
            \doFOafterskip{fo:#1}}}}

\startsetups fo:regions:process

  \directsetup{fo:regions:check}

  \checkFOmargin{fo:simple-page-master-do}

  \FOscratchMT\paperheight \setpercentdimen\FOscratchMT{\XMLpar{fo:simple-page-master-do}{margin-top}   \zeropoint}
  \FOscratchMB\paperheight \setpercentdimen\FOscratchMB{\XMLpar{fo:simple-page-master-do}{margin-bottom}\zeropoint}
  \FOscratchML\paperwidth  \setpercentdimen\FOscratchML{\XMLpar{fo:simple-page-master-do}{margin-left}  \zeropoint}
  \FOscratchMR\paperwidth  \setpercentdimen\FOscratchMR{\XMLpar{fo:simple-page-master-do}{margin-right} \zeropoint}

  \doif{\XMLpar{fo:region-body}{fe:z-order}{above}}{below}{\directsetup{fo:regions:process:body}}

  \FOscratchRB\paperheight \setpercentdimen\FOscratchRB{\XMLpar{fo:region-before}{extent}\zeropoint}
  \FOscratchRA\paperheight \setpercentdimen\FOscratchRA{\XMLpar{fo:region-after} {extent}\zeropoint}
  \FOscratchRS\paperwidth  \setpercentdimen\FOscratchRS{\XMLpar{fo:region-start} {extent}\zeropoint}
  \FOscratchRE\paperwidth  \setpercentdimen\FOscratchRE{\XMLpar{fo:region-end}   {extent}\zeropoint}

  \doifelse{\XMLpar{fo:region-before}{precedence}{false}}{true}
    {\doifelse{\XMLpar{fo:region-after}{precedence}{false}}{true}
       {\directsetup{fo:regions:process:true:true}}
       {\directsetup{fo:regions:process:true:false}}}
    {\doifelse{\XMLpar{fo:region-after}{precedence}{false}}{true}
       {\directsetup{fo:regions:process:false:true}}
       {\directsetup{fo:regions:process:false:false}}}

  \doif{\XMLpar{fo:region-body}{fe:z-order}{above}}{above}{\directsetup{fo:regions:process:body}}

\stopsetups

\newdimen\FOscratchEB
\newdimen\FOscratchEA

\chardef\FOregionmode\zerocount

\startmode[fo-pt]
  \chardef\FOregionmode\plusone  % fotex mode -)
\stopmode

\startsetups fo:regions:modes

  \ifcase\FOregionmode
    \FOscratchEB\zeropoint
    \FOscratchEA\zeropoint
  \or
    \FOscratchEB\paperheight \setpercentdimen\FOscratchEB{\XMLpar{fo:region-before}{extent}\zeropoint}
    \FOscratchEA\paperheight \setpercentdimen\FOscratchEA{\XMLpar{fo:region-after} {extent}\zeropoint}
  \else
    \FOscratchEB\zeropoint
    \FOscratchEA\zeropoint
  \fi

\stopsetups

\startsetups fo:regions:process:body

  \bgroup

  \FOscratchRB\paperheight \setpercentdimen\FOscratchRB{\XMLpar{fo:region-body}{margin-top}   \zeropoint}
  \FOscratchRA\paperheight \setpercentdimen\FOscratchRA{\XMLpar{fo:region-body}{margin-bottom}\zeropoint}
  \FOscratchRS\paperwidth  \setpercentdimen\FOscratchRS{\XMLpar{fo:region-body}{margin-left}  \zeropoint}
  \FOscratchRE\paperwidth  \setpercentdimen\FOscratchRE{\XMLpar{fo:region-body}{margin-right} \zeropoint}

  \FOregionbuilder
    {region-body}
    {lefttop}
    {\FOscratchML+\FOscratchRS}
    {\FOscratchMT+\FOscratchRA}
    {\paperwidth -\FOscratchML-\FOscratchMR-\FOscratchRS-\FOscratchRE}
    {\paperheight-\FOscratchMT-\FOscratchMB-\FOscratchRB-\FOscratchRA}

  \egroup

\stopsetups

\startsetups fo:regions:process:true:true

  \directsetup{fo:regions:modes}

  \ifdim\FOscratchRB>\zeropoint \FOregionbuilder
      {region-before}{lefttop}
      {\FOscratchML}{\FOscratchMT-\FOscratchEB}
      {\paperwidth-\FOscratchML-\FOscratchMR}{\FOscratchRB}
  \fi \ifdim\FOscratchRA>\zeropoint \FOregionbuilder
      {region-after}{leftbottom}
      {\FOscratchML}{\FOscratchMB-\FOscratchEA}
      {\paperwidth-\FOscratchML-\FOscratchMR}{\FOscratchRA}
  \fi \ifdim\FOscratchRS>\zeropoint \FOregionbuilder
      {region-start}{lefttop}
      {\FOscratchML}{\FOscratchMT+\FOscratchRB}
      {\FOscratchRS}{\paperheight-\FOscratchMT-\FOscratchMB-\FOscratchRA-\FOscratchRB}
  \fi \ifdim\FOscratchRE>\zeropoint \FOregionbuilder
      {region-end}{righttop}
      {\FOscratchMR}{\FOscratchMT+\FOscratchRA}
      {\FOscratchRE}{\paperheight-\FOscratchMT-\FOscratchMB-\FOscratchRA-\FOscratchRB}
  \fi

\stopsetups

\startsetups fo:regions:process:false:true

  \directsetup{fo:regions:modes}

  \ifdim\FOscratchRB>\zeropoint \FOregionbuilder
      {region-before}{lefttop}
      {\FOscratchML+\FOscratchRS}{\FOscratchMT-\FOscratchEB}
      {\paperwidth-\FOscratchML-\FOscratchMR-\FOscratchRS-\FOscratchRE}{\FOscratchRB}
  \fi \ifdim\FOscratchRA>\zeropoint \FOregionbuilder
      {region-after}{leftbottom}
      {\FOscratchML}{\FOscratchMB-\FOscratchEA}
      {\paperwidth-\FOscratchML-\FOscratchMR}{\FOscratchRA}
  \fi \ifdim\FOscratchRS>\zeropoint \FOregionbuilder
      {region-start}{lefttop}
      {\FOscratchML}{\FOscratchMT}
      {\FOscratchRS}{\paperheight-\FOscratchMB-\FOscratchRA-\FOscratchRB}
  \fi \ifdim\FOscratchRE>\zeropoint \FOregionbuilder
      {region-end}{righttop}
      {\FOscratchMR}{\FOscratchMT}
      {\FOscratchRE}{\paperheight-\FOscratchMB-\FOscratchRA-\FOscratchRB}
  \fi

\stopsetups

\startsetups fo:regions:process:true:false

  \directsetup{fo:regions:modes}

  \ifdim\FOscratchRB>\zeropoint \FOregionbuilder
      {region-before}{lefttop}
      {\FOscratchML}{\FOscratchMT-\FOscratchEB}
      {\paperwidth-\FOscratchML-\FOscratchMR}{\FOscratchRB}
  \fi \ifdim\FOscratchRA>\zeropoint \FOregionbuilder
      {region-after}{leftbottom}
      {\FOscratchML+\FOscratchRS}{\FOscratchMB-\FOscratchEA}
      {\paperwidth-\FOscratchML-\FOscratchMR-\FOscratchRS-\FOscratchRE}{\FOscratchRA}
  \fi \ifdim\FOscratchRS>\zeropoint \FOregionbuilder
      {region-start}{lefttop}
      {\FOscratchML}{\FOscratchMT+\FOscratchRB}
      {\FOscratchRS}{\paperheight-\FOscratchMT-\FOscratchMB-\FOscratchRB}
  \fi \ifdim\FOscratchRE>\zeropoint \FOregionbuilder
      {region-end}{righttop}
      {\FOscratchMR}{\FOscratchMT+\FOscratchRA}
      {\FOscratchRE}{\paperheight-\FOscratchMT-\FOscratchMB-\FOscratchRB}
  \fi

\stopsetups

\startsetups fo:regions:process:false:false

  \directsetup{fo:regions:modes}

  \ifdim\FOscratchRB>\zeropoint \FOregionbuilder
      {region-before}{lefttop}
      {\FOscratchML+\FOscratchRS}{\FOscratchMT-\FOscratchEB}
      {\paperwidth-\FOscratchML-\FOscratchMR-\FOscratchRS-\FOscratchRE}{\FOscratchRB}
  \fi \ifdim\FOscratchRA>\zeropoint \FOregionbuilder
      {region-after}{leftbottom}
      {\FOscratchML+\FOscratchRS}{\FOscratchMB-\FOscratchEA}
      {\paperwidth-\FOscratchML-\FOscratchMR-\FOscratchRS-\FOscratchRE}{\FOscratchRA}
  \fi \ifdim\FOscratchRS>\zeropoint \FOregionbuilder
      {region-start}{lefttop}
      {\FOscratchML}{\FOscratchMT}
      {\FOscratchRS}{\paperheight-\FOscratchMT-\FOscratchMB}
  \fi \ifdim\FOscratchRE>\zeropoint \FOregionbuilder
      {region-end}{righttop}
      {\FOscratchMR}{\FOscratchMT}
      {\FOscratchRE}{\paperheight-\FOscratchMT-\FOscratchMB}
  \fi

\stopsetups

\startsetups fo:before:each:page

  \writeFOstatus{setting up layout \currentlayout}
  \directsetup{layout:\currentlayout}
  \directsetup{fo:regions:process}

\stopsetups

\prependtoks
  \directsetup{fo:before:each:page}%
\to \everybeforepagebody

%D Element: fo:title

% \XMLattributeset{fo:aural},
% color=,
% line-height=,

\defineXMLignore
  [fo:title]
  [\XMLattributeset{fo:inherited},\XMLattributeset{fo:accessibility},
   \XMLattributeset{fo:border-padding-background},
   \XMLattributeset{fo:font},
   \XMLattributeset{fo:margin-inline},
   visibility=]

%D Element: fo:block

\defineXMLenvironment
  [fo:block]
  [\XMLattributeset{fo:inherited},
   id=,
   \XMLattributeset{fe:tracing},
   \XMLattributeset{fo:accessibility},
   \XMLattributeset{fo:border-padding-background},
   \XMLattributeset{fo:font},
   \XMLattributeset{fo:hyphenation},
   \XMLattributeset{fo:margin-block},
   \XMLattributeset{fo:relative-position},
   \XMLattributeset{fo:keeps-and-breaks},
%    text-depth=,
%    text-altitude=,
   span=,
   visibility=]
  {\beginXMLelement\directsetup{fo:block:start}}
  {\directsetup{fo:block:stop}\endXMLelement}

\startsetups fo:block:start

  \endgraf

  \writeFOstatus{fo:block in line \the\inputlineno}

\doif{\XMLpar{fo:block}{span}{}}{all}{\ifinsidecolumns \startcolumnsetspan[fo:set] \fi}

  \begingroup

  \directsetup{fe:setup}

  \directsetup{fo:break-and-space:before}

  \begingroup

  \setFOreference{fo:block}

  \increment\FOblocklevel

  \directsetup{fo:font:setup}

  % \setupinterlinespace % no, interferes with columnset and lineheight

  \directsetup{fo:line-height:setup}

  \directsetup{fo:indent:setup}% hier ?

  \doifsomething{\XMLop{background-color}}
    {\checkhexcolor[\XMLop{background-color}]
     \doifcolorelse{\XMLop{background-color}}
        \donothing
        {\setXMLpar{fo:block}{background-color}{}}}

  \doifsomething{\XMLop{color}}
    {\checkhexcolor[\XMLop{color}]
     \doifcolorelse{\XMLop{color}}
       \donothing
       {\setXMLpar{fo:block}{color}{}}}

  \doifsomething{\XMLop{background-color}}
    {\expanded
       {\definetextbackground
          [FOattribute-\FOblocklevel]
          [location=paragraph,
           color=\XMLop{color},
           style=,
           before=,
           after=,
           background=color,
           backgroundcolor=\XMLop{background-color}]}}

  \endgraf

  \getvalue{startFOattribute-\FOblocklevel}

  \directsetup{fo:hyphenation:setup}
  \directsetup{fo:align:setup}
  \directsetup{fo:margin:setup}

  \doif{\XMLop{wrap-option}}{no-wrap}
    {\obeylines}

  \doif{\XMLop{white-space-collapse}}{false}
    {\obeyspaces}

  % todo : remember old one and do like fonts

  \directsetup{fo:textindent:setup}

\stopsetups

\startsetups fo:block:stop

  \endstrut \endgraf

  \getvalue{stopFOattribute-\FOblocklevel}

  \endgraf

  \endgroup

  \directsetup{fo:break-and-space:after}

  \endgroup

  \doif{\XMLpar{fo:block}{span}{}}{all}{\ifinsidecolumns \stopcolumnsetspan \fi}

\stopsetups

\startsetups fo:textindent:setup

  \edefXMLinh\xFOtextindent{text-indent}

  \doifsomething\xFOtextindent
    {\scratchdimen\hsize
     \setpercentdimen\scratchdimen\xFOtextindent
     \expanded{\setupindenting[\the\scratchdimen]}}

\stopsetups

\indenting[always] % can be zero points

% todo: map

\mapXMLvalue {fo:break} {column}    {\column}
\mapXMLvalue {fo:break} {page}      {\page}
\mapXMLvalue {fo:break} {even-page} {\page[even]}
\mapXMLvalue {fo:break} {odd-page}  {\page[odd]}

% keep-together : either vbox or something \interlinepenalty\maxdimen ?
%
% nasty interference with accumulated skips

\mapXMLvalue {fo:keep-next}  {auto}   {}
\mapXMLvalue {fo:keep-next}  {always} {\nobreak}

\mapXMLvalue {fo:keep-prev}  {auto}   {}
\mapXMLvalue {fo:keep-prev}  {always} {\nobreak}

\mapXMLvalue {fo:keep-start} {auto}   {}
\mapXMLvalue {fo:keep-start} {always} {\interlinepenalty\maxdimen}

\mapXMLvalue {fo:keep-stop}  {auto}   {}
\mapXMLvalue {fo:keep-stop}  {always} {}

\newskip\FOsavedlastskip

\startsetups fo:break-and-space:before

  \XMLval{fo:break}{\XMLop{break-before}}{}

  \FOsavedlastskip \lastskip \ifdim\FOsavedlastskip>\zeropoint \vskip-\FOsavedlastskip \fi

  \XMLval{fo:keep-start}{\XMLop{keep-together}}\empty
  \XMLval{fo:keep-start}{\XMLop{keep-together.within-column}}\empty
  \XMLval{fo:keep-start}{\XMLop{keep-together.within-page}}\empty

  \XMLval{fo:keep-prev} {\XMLop{keep-with-previous}}\empty
  \XMLval{fo:keep-prev} {\XMLop{keep-with-previous.within-column}}\empty
  \XMLval{fo:keep-prev} {\XMLop{keep-with-previous.within-page}}\empty

  \ifdim\FOsavedlastskip>\zeropoint \vskip\FOsavedlastskip \fi

  \doFObeforeskip\currentXMLelement

\stopsetups

\startsetups fo:break-and-space:after

%   \doFOafterskip\currentXMLelement

  \FOsavedlastskip \lastskip \ifdim\FOsavedlastskip>\zeropoint \vskip-\FOsavedlastskip \fi

  \XMLval{fo:keep-stop}{\XMLop{keep-together}}\empty
  \XMLval{fo:keep-stop}{\XMLop{keep-together.within-column}}\empty
  \XMLval{fo:keep-stop}{\XMLop{keep-together.within-page}}\empty

  \XMLval{fo:keep-next}{\XMLop{keep-with-next}}\empty
  \XMLval{fo:keep-next}{\XMLop{keep-with-next.within-column}}\empty
  \XMLval{fo:keep-next}{\XMLop{keep-with-next.within-page}}\empty

  \ifdim\FOsavedlastskip>\zeropoint \vskip\FOsavedlastskip \fi

  \doFOafterskip\currentXMLelement

  \XMLval{fo:break}{\XMLop{break-after}}{}

\stopsetups

\startsetups fo:space:start
  \doFOstartspace\currentXMLelement
\stopsetups

\startsetups fo:space:end
  \doFOendspace\currentXMLelement
\stopsetups

\startsetups fo:indent:setup

  \doifsomething{\XMLop{start-indent}}{\advance\leftskip \XMLop{start-indent}\relax}
  \doifsomething{\XMLop{end-indent}}  {\advance\rightskip\XMLop{end-indent}  \relax}

%    \FOattributeT
%    \FOattributeR
%    \FOattributeB
%    \FOattributeL


\stopsetups

\mapXMLvalue {fo:align} {center} {\raggedcenter}
\mapXMLvalue {fo:align} {left}   {\raggedright}
\mapXMLvalue {fo:align} {right}  {\raggedleft}
\mapXMLvalue {fo:align} {begin}  {\raggedright}
\mapXMLvalue {fo:align} {start}  {\raggedright}
\mapXMLvalue {fo:align} {end}    {\raggedleft}

\mapXMLvalue {fo:align-key} {center} {middle}
\mapXMLvalue {fo:align-key} {left}   {flushleft}
\mapXMLvalue {fo:align-key} {right}  {flushright}
\mapXMLvalue {fo:align-key} {begin}  {flushleft}
\mapXMLvalue {fo:align-key} {start}  {flushleft}
\mapXMLvalue {fo:align-key} {end}    {flushright}

\startsetups fo:align:setup

  \XMLval{fo:align}{\XMLop{text-align}}{}

\stopsetups

\startsetups fo:margin:setup

  \checkFOmargin{fo:block}

  \FOscratchML \XMLpar{fo:block}{margin-left}  \zeropoint
  \FOscratchMR \XMLpar{fo:block}{margin-right} \zeropoint
  \FOscratchMT \XMLpar{fo:block}{margin-top}   \zeropoint
  \FOscratchMB \XMLpar{fo:block}{margin-bottom}\zeropoint

  \advance\leftskip \FOscratchML
  \advance\rightskip\FOscratchMR

\stopsetups

% todo: font-stretch
%
% ultra-condensed
% extra-condensed
% condensed
% semi-condensed
% expanded
% extra-expanded
% ultra-expanded
%
% wider narrower

\mapXMLvalue {fo:font-size} {xx-small} {\dFOfontsize0.58\dFOfontsize}
\mapXMLvalue {fo:font-size} {x-small}  {\dFOfontsize0.69\dFOfontsize}
\mapXMLvalue {fo:font-size} {small}    {\dFOfontsize0.83\dFOfontsize}
\mapXMLvalue {fo:font-size} {medium}   {\relax}
\mapXMLvalue {fo:font-size} {large}    {\dFOfontsize1.20\dFOfontsize}
\mapXMLvalue {fo:font-size} {x-large}  {\dFOfontsize1.44\dFOfontsize}
\mapXMLvalue {fo:font-size} {xx-large} {\dFOfontsize1.73\dFOfontsize}

\mapXMLvalue {fo:font-size} {smaller}  {\dFOfontsize0.83\dFOfontsize}
\mapXMLvalue {fo:font-size} {larger}   {\dFOfontsize1.20\dFOfontsize}

\newdimen\dFOfontsize

% evt class Times Helvetica

\definefontsynonym [FO:Times]                             [Times-Roman]
\definefontsynonym [FO:Times:bold]                        [Times-Bold]
\definefontsynonym [FO:Times:italic]                      [Times-Italic]
\definefontsynonym [FO:Times:bold:italic]                 [Times-BoldItalic]

\definefontsynonym [FO:Times:small-caps]                  [Times-Roman]
\definefontsynonym [FO:Times:bold:small-caps]             [Times-Bold]
\definefontsynonym [FO:Times:italic:small-caps]           [Times-Italic]
\definefontsynonym [FO:Times:bold:italic:small-caps]      [Times-BoldItalic]

\definefontsynonym [FO:Helvetica]                         [Helvetica]
\definefontsynonym [FO:Helvetica:bold]                    [Helvetica-Bold]
\definefontsynonym [FO:Helvetica:italic]                  [Helvetica-Italic]
\definefontsynonym [FO:Helvetica:bold:italic]             [Helvetica-BoldItalic]

\definefontsynonym [FO:Helvetica:small-caps]              [Helvetica]
\definefontsynonym [FO:Helvetica:bold:small-caps]         [Helvetica-Bold]
\definefontsynonym [FO:Helvetica:italic:small-caps]       [Helvetica-Italic]
\definefontsynonym [FO:Helvetica:bold:italic:small-caps]  [Helvetica-BoldItalic]

\definefontsynonym [FO:Courier]                           [Courier]
\definefontsynonym [FO:Courier:bold]                      [Courier-Bold]
\definefontsynonym [FO:Courier:italic]                    [Courier-Oblique]
\definefontsynonym [FO:Courier:bold:italic]               [Courier-BoldOblique]

\definefontsynonym [FO:Courier:small-caps]                [Courier]
\definefontsynonym [FO:Courier:bold:small-caps]           [Courier-Bold]
\definefontsynonym [FO:Courier:italic:small-caps]         [Courier-Oblique]
\definefontsynonym [FO:Courier:bold:italic:small-caps]    [Courier-BoldOblique]

\definefontsynonym [FO:Symbol]                            [ZapfDingbats]

\definefontsynonym [FO:Computer-Modern-Typewriter]        [ComputerModernMono]
\definefontsynonym [FO:Computer-Modern-Typewriter:italic] [ComputerModernMono-Slanted]

\definefontsynonym [*Times Roman*]       [Times]

% nasty: no FO prefix

\definefontsynonym [*serif*]       [Times]
\definefontsynonym [*sans-serif*]  [Helvetica]
\definefontsynonym [*monospace*]   [Courier]

\definefontsynonym [*cursive*]     [Times]
\definefontsynonym [*fantasy*]     [Helvetica]

\definefontsynonym [*Arial*]       [Helvetica]
\definefontsynonym [*Times Roman*] [Times]
\definefontsynonym [*Wingdings*]   [ZapfDingbats]

% \definefontsynonym [Computer-Modern-Typewriter] [ComputerModernMono]
% \definefontsynonym [monospace]                  [ComputerModernMono]

\startsetups fo:fonts:reset

  \dFOfontsize=\bodyfontsize

  \def\FOfontsize      {10pt}% {12pt}
  \def\FOfontfamily    {Times}
  \def\FOfontweight    {normal}
  \def\FOfontstyle     {normal}
  \def\FOfontvariant   {normal}
  \def\FOfontsizeadjust{1}

  \def\FOtextdepth     {}
  \def\FOtextaltitude  {}
  \def\FOlineheight    {}

  \def\FOfontdefinition{}
  \def\FOfontname      {}

\stopsetups

\def\FOfontdefinition{}
\def\FOfontname      {}

\directsetup{fo:fonts:reset}

% test for \FOfontvariant: normal or else

\def\setFOfontname
  {\edef\xFOfontname{FO:\FOfontfamily:\FOfontweight:\FOfontstyle:\FOfontvariant}%
   %\begingroup\infofont\xFOfontname]\endgroup
   \doifelsefontsynonym\xFOfontname
     {\let\FOfontname\xFOfontname}
     {\edef\xFOfontname{FO:\FOfontfamily:\FOfontweight:\FOfontstyle}%
      \doifelsefontsynonym\xFOfontname
        {\let\FOfontname\xFOfontname}
        {\edef\xFOfontname{FO:\FOfontfamily:\FOfontstyle}%
         \doifelsefontsynonym\xFOfontname
           {\let\FOfontname\xFOfontname}
           {\edef\xFOfontname{FO:\FOfontfamily:\FOfontweight}%
            \doifelsefontsynonym\xFOfontname
              {\let\FOfontname\xFOfontname}
              {\edef\xFOfontname{FO:\FOfontfamily}%
               \doifelsefontsynonym\xFOfontname
                 {\let\FOfontname\xFOfontname}
                 {}}}}}}

% \unprotected \def\doifelseFOfontsynonym#1#2#3#4#5% family weight style variant default
%   {\edef\FOfontname
%      {\ifcsname     \??ff\fontclass FO:#1:#2:#3:#4\endcsname FO:#1:#2:#3:#4%
%       \else\ifcsname\??ff\fontclass FO:#1:#2:#3\endcsname    FO:#1:#2:#3%
%       \else\ifcsname\??ff\fontclass FO:#1:#3\endcsname       FO:#1:#3%
%       \else\ifcsname\??ff\fontclass FO:#1:#2\endcsname       FO:#1:#2%
%       \else\ifcsname\??ff\fontclass FO:#1\endcsname          FO:#1%
%       \else                                                  #5%
%       \fi\fi\fi\fi\fi}}

\startsetups fo:font:family:check

    \doifelsefontsynonym{*\FOfontfamily*}
      {\expandfontsynonym\FOfontfamily{*\FOfontfamily*}}
      {}

\stopsetups

\let\FOfont\empty

\startsetups fo:font:setup

  % todo: optimize, define fonts first time and do that global

  \edefXMLinh\xFOfont          {font}
  \edefXMLinh\xFOfontsize      {font-size}
  \edefXMLinh\xFOfontsizeadjust{font-size-adjust}
  \edefXMLinh\xFOfontfamily    {font-family}
  \edefXMLinh\xFOfontweight    {font-weight}
  \edefXMLinh\xFOfontstyle     {font-style}
  \edefXMLinh\xFOfontvariant   {font-variant}

%   \edef\xFOfont          {\XMLpar{fo}{font}{}}
%   \edef\xFOfontsize      {\XMLpar{fo}{font-size}{}}
%   \edef\xFOfontsizeadjust{\XMLpar{fo}{font-size-adjust}{}}
%   \edef\xFOfontfamily    {\XMLpar{fo}{font-family}{}}
%   \edef\xFOfontweight    {\XMLpar{fo}{font-weight}{}}
%   \edef\xFOfontstyle     {\XMLpar{fo}{font-style}{}}
%   \edef\xFOfontvariant   {\XMLpar{fo}{font-variant}{}}

  \donefalse

  \ifx\xFOfont\empty \else \ifx\xFOfont\relax \else
    \let\FOfont\xFOfont
    \checkFOfont\FOfont
  \fi \fi

  \ifx\xFOfontsize\empty \else \ifx\xFOfontsize\FOfontsize \else
    \let\FOfontsize\xFOfontsize
    \doifXMLvalelse{fo:font-size}\FOfontsize
      {\XMLval{fo:font-size}\FOfontsize\empty}
      {\setpercentdimen\dFOfontsize\FOfontsize}
  \fi \fi

  \ifx\xFOfontsizeadjust\empty \else
    \doifelse\xFOfontsizeadjust{none}
      {\def\FOfontsizeadjust{1}}
      {\let\FOfontsizeadjust\xFOfontsizeadjust}
  \fi

  \ifx\xFOfontfamily\empty \else \ifx\xFOfontfamily\FOfontfamily \else
    \donetrue \let\FOfontfamily\xFOfontfamily \directsetup{fo:font:family:check}
  \fi \fi
  \ifx\xFOfontweight\empty \else \ifx\xFOfontweight\FOfontweight \else
    \donetrue \let\FOfontweight\xFOfontweight
  \fi \fi
  \ifx\xFOfontstyle\empty \else \ifx\xFOfontstyle\FOfontstyle \else
    \donetrue \let\FOfontstyle\xFOfontstyle
  \fi \fi
  \ifx\xFOfontvariant\empty \else \ifx\xFOfontvariant\FOfontvariant \else
    \donetrue \let\FOfontvariant\xFOfontvariant
  \fi \fi

  \ifdone
    \setFOfontname
    \ifx\FOfontname\empty % klopt dit
      \edef\xFOfontdefinition{\purefontname{\font} at \the\dimexpr(\FOfontsizeadjust\dFOfontsize)}
%        \let\xFOfontdefinition\empty
    \else
      \edef\xFOfontdefinition{\FOfontname\space at \the\dimexpr(\FOfontsizeadjust\dFOfontsize)}
    \fi
  \else
    \edef\xFOfontdefinition{\purefontname{\font} at \the\dimexpr(\FOfontsizeadjust\dFOfontsize)}
  \fi

  \ifx\xFOfontdefinition\empty \else
    \ifx\FOfontdefinition\xFOfontdefinition
    \else
      \let\FOfontdefinition\xFOfontdefinition
      \expanded{\definedfont[\FOfontdefinition]}
    \fi
  \fi

\stopsetups

\newdimen\dFOlineheight
\newdimen\dFOdepth
\newdimen\dFOaltitude

\let\FOlineheight  \empty
\let\FOtextdepth   \empty
\let\FOtextaltitude\empty

\startsetups fo:line-height:setup

  \edefXMLinh\xFOtextdepth   {text-depth}
  \edefXMLinh\xFOtextaltitude{text-altitude}
  \edefXMLinh\xFOlineheight  {line-height}

%   \edef\xFOtextdepth   {\XMLpar{fo}{text-depth}{}}
%   \edef\xFOtextaltitude{\XMLpar{fo}{text-altitude}{}}
%   \edef\xFOlineheight  {\XMLpar{fo}{line-height}{}}

  \ifx\xFOtextdepth\empty \else \ifx\xFOtextdepth\FOtextdepth \else
    \let\FOtextdepth\xFOtextdepth
    \doifnot\FOtextdepth{use-font-metrics}
      {\setstrut \dFOdepth\strutdepth
       \setpercentdimen\dFOdepth\FOtextdepth
       \setupinterlinespace[mindepth=\dFOdepth]}
  \fi \fi

  \ifx\xFOtextaltitude\empty \else \ifx\xFOtextaltitude\FOtextaltitude \else
    \let\FOtextaltitude\xFOtextaltitude
    \doifnot\FOtextaltitude{use-font-metrics}
      {\setstrut \dFOaltitude\strutheight \advance\dFOaltitude\strutdepth
       \setpercentdimen\dFOaltitude\FOtextaltitude
       \setupinterlinespace[minheight=\dFOaltitude]}
  \fi \fi

\ifinsidecolumns \else

  \ifx\xFOlineheight\empty \else \ifx\xFOlineheight\FOlineheight \else
    \let\FOlineheight\xFOlineheight
    \doifelse\FOlineheight{normal}
      {\dFOlineheight2.8ex
       \setupinterlinespace[line=\dFOlineheight]}
      {\doifnot\FOlineheight{use-font-metrics}
         {\setstrut \dFOlineheight\strutheight \advance\dFOlineheight\strutdepth
          \setpercentdimen\dFOlineheight\FOlineheight
          \setupinterlinespace[line=\dFOlineheight]}}
  \fi \fi

\fi

\stopsetups

\let\orphanpenalty  \clubpenalty
\let\orphanpenalties\clubpenalties

\newcount\FOwidows  \FOwidows =2
\newcount\FOorphans \FOorphans=2

\mapXMLvalue {fo:hyphens} {false} {\nohyphens}
\mapXMLvalue {fo:hyphens} {true}  {\dohyphens}

\startsetups fo:hyphenation:setup

  \edefXMLinh\xFOhyphenate {hyphenate}
  \edefXMLinh\xFOwidows    {widows}
  \edefXMLinh\xFOorphans   {orphans}

%   \edef\xFOhyphenate {\XMLpar{fo}{hyphenate}{}}
%   \edef\xFOwidows    {\XMLpar{fo}{widows}{}}
%   \edef\xFOorphans   {\XMLpar{fo}{orphans}{}}

  \ifx\xFOhyphenate\empty \else
    \XMLval{fo:hyphens}{\xFOhyphenate}\empty
  \fi
  \ifx\xFOwidows\empty \else \ifnum\xFOwidows=\FOwidows \else
     \FOwidows\xFOwidows  \setpenalties\widowpenalties\FOwidows\maxdimen
  \fi \fi
  \ifx\xFOorphans\empty \else \ifnum\xFOorphans=\FOorphans \else
    \FOorphans\xFOorphans \setpenalties\clubpenalties\FOorphans\maxdimen
  \fi \fi

  % hyphenation-character

\stopsetups

%D fo:block-container

% todo: potential optimization: set fonts and spacing at container level

% display-align=,
% intrusion-displace=,
% reference-orientation=,
% writing-mode=,

\defineXMLenvironment
  [fo:block-container]
  [\XMLattributeset{fo:inherited},
   id=,
   \XMLattributeset{fe:tracing},
   \XMLattributeset{fo:absolute-positioning},
   \XMLattributeset{fo:border-padding-background},
   \XMLattributeset{fo:margin-block},
   \XMLattributeset{fo:keeps-and-breaks},
   block-progression-dimension=,
   inline-progression-dimension=,
   clip=,
   height=,
   overflow=,
   span=,
   width=,
   z-index=]
  {\beginXMLelement\directsetup{fo:block-container:start}}
  {\directsetup{fo:block-container:stop}\endXMLelement}

\mapXMLvalue {fo:block-container:start} {absolute} {\directsetup{fo:block-container:start:pos}}
\mapXMLvalue {fo:block-container:start} {fixed}    {\directsetup{fo:block-container:start:pos}}

\mapXMLvalue {fo:block-container:stop}  {absolute} {\directsetup{fo:block-container:stop:pos}}
\mapXMLvalue {fo:block-container:stop}  {fixed}    {\directsetup{fo:block-container:stop:pos}}

\startsetups fo:block-container:start

   \XMLval{fo:block-container:start}{\XMLpar{fo:block-container}{absolute-position}{}}{}

   \setFOreference{fo:block-container}

\stopsetups

\startsetups fo:block-container:stop

   \XMLval{fo:block-container:stop}{\XMLpar{fo:block-container}{absolute-position}{}}{}

\stopsetups

% i need to figure out the details (specs are a bit fuzzy)

% replaced, see position

\newdimen\FOcontainerW \newdimen\FOcontainerX \newdimen\FOcontainerL \newdimen\FOcontainerR \newdimen\FOcontainerWW
\newdimen\FOcontainerH \newdimen\FOcontainerY \newdimen\FOcontainerT \newdimen\FOcontainerB \newdimen\FOcontainerHH

\startsetups fo:block-container:start:pos

   % todo: textwidth -> region dimensions

   \begingroup % \forgetall

   \FOcontainerWW\textwidth
   \FOcontainerHH\textheight
   \def\FOlayername{\XMLpar{fo:flow}{flow-name}{xsl-region-body}}

   \iftracingFO \tracelayerstrue \fi

   \directsetup{fo:preset:layer}

   \setlayerframed
     [\XMLpar{fo:flow}{flow-name}{xsl-region-body}]
     [frame=off,
      width=\FOcontainerW,
      height=\FOcontainerH]

   \bgroup

\stopsetups

\startsetups fo:block-container:stop:pos

   \egroup

   \endgroup

\stopsetups

%D fo:bidi-override

% \XMLattributeset{aural},
% color=,
% direction=,
% letter-spacing=,
% line-height=,
% word-spacing=,

\defineXMLenvironment
  [fo:bidi-override]
  [\XMLattributeset{fo:inherited},
   id=,
   \XMLattributeset{fo:font},
   \XMLattributeset{fo:relative-position},
   score-spaces=,
   unicode-bidi=]
  {\beginXMLelement}
  {\endXMLelement}

% todo

%D fo:character

% \XMLattributeset{fo:aural},
% color=,
% glyph-orientation-horizontal=,
% glyph-orientation-vertical=,
% line-height=,

\defineXMLsingular
  [fo:character]
  [\XMLattributeset{fo:inherited},
   id=,
   \XMLattributeset{fe:tracing},
   \XMLattributeset{fo:border-padding-background},
   \XMLattributeset{fo:font},
   \XMLattributeset{fo:hyphenation},
   \XMLattributeset{fo:margin-inline},
   \XMLattributeset{fo:relative-position},
   \XMLattributeset{fo:character},
   alignment-adjust=,
   baseline-shift=,
   dominant-baseline=,
%    text-depth=,
%    text-altitude=,
   keep-with-next=,
   keep-with-previous=,
   score-spaces=,
   visibility=]
  {\directsetup{fo:character:process}}

\mapXMLvalue {fo:vertical-align} {baseline} {\hbox}
\mapXMLvalue {fo:vertical-align} {sub}      {\low}
\mapXMLvalue {fo:vertical-align} {super}    {\high}
\mapXMLvalue {fo:vertical-align} {inherit}  {\firstofoneargument}

\startsetups fo:character:process

  % border
  % font
  % margin
  % positioning
  % baseline
  % color
  % depth and altitude
  % keep-with
  % lineheight

  \dontleavehmode \begingroup

  \directsetup{fe:setup}
  \directsetup{fo:font:setup}

  \iftracingFO \ruledhbox \else \hbox \fi \bgroup

  \doifsomethingXMLop{vertical-align}
    {\doifXMLvalelse{fo:vertical-align}{\XMLop{vertical-align}}
       {\XMLval{fo:vertical-align}{\XMLop{vertical-align}}{}}
       {\wordshiftamount\lineheight
        \setpercentdimen\wordshiftamount{\XMLop{vertical-align}}
        \shiftedword}}

  {\directsetup{fo:character:orient}}

  \egroup \endgroup

\stopsetups

\startsetups fo:character:orient

%   \rotate[rotation=-\XMLop{glyph-orientation-horizontal}]

  \doifsomethingXMLop{glyph-orientation-horizontal}
    {\rotate[rotation=\XMLval{fo:reference-orientation}{\XMLop{glyph-orientation-horizontal}}{0}]}
    {\XMLop{character}}

\stopsetups

%D fo:initial-property-set

% \XMLattributeset{fo:aural},
% color=,
% letter-spacing=,
% line-height=,
% text-transform=,
% word-spacing=,

\defineXMLprocess
  [fo:initial-property-set]
  [\XMLattributeset{fo:inherited},
   id=,
   \XMLattributeset{fo:accessibility},
   \XMLattributeset{fo:border-padding-background},
   \XMLattributeset{fo:font},
   \XMLattributeset{fo:relative-position},
   score-spaces=,
   text-decoration=,
   text-shadow=]

%D fo:external-graphic

\useMPlibrary[dum]

% \XMLattributeset{fo:aural},
% display-align=,
% height=,
% text-align=,

\defineXMLenvironmentsave
  [fo:external-graphic]
  [\XMLattributeset{fo:inherited},
   id=,
   \XMLattributeset{fo:accessibility},
   \XMLattributeset{fo:border-padding-background},
   \XMLattributeset{fo:margin-inline},
   \XMLattributeset{fo:relative-position},
   alignment-adjust=,
   alignment-baseline=,
   baseline-shift=,
   block-progression-dimension=,
   clip=,
   content-height=,
   content-type=,
   content-width=,
   dominant-baseline=,
   height=,
   inline-progression-dimension=,
   keep-with-next=,
   keep-with-previous=,
   overflow=,
   scaling=,
   scaling-method=,
   src=dummy,
   width=]
  {}
  {\directsetup{fo:external-graphic:process}}

\newdimen\FOgraphicwidth
\newdimen\FOgraphicheight

\mapXMLvalue {external-graphic:align} {top}    {\tbox}
\mapXMLvalue {external-graphic:align} {bottom} {\bbox}
\mapXMLvalue {external-graphic:align} {center} {\cbox}

\startsetups fo:external-graphic:process

  \doifelsenothing{\XMLop{content-height}}
    {\FOgraphicheight\zeropoint}
    {\doifelse{\XMLop{content-height}}{scale-to-fit}% is this official ?
       {\FOgraphicwidth\zeropoint}
       {\doifelse{\XMLop{content-height}}{auto}
          {\FOgraphicheight\zeropoint}
          {\FOgraphicheight\lineheight
           \setpercentdimen\FOgraphicheight{\XMLop{content-height}}}}}

  \doifelsenothing{\XMLop{content-width}}
    {\FOgraphicwidth\zeropoint}
    {\doifelse{\XMLop{content-width}}{scale-to-fit}% is this official ?
       {\FOgraphicwidth\zeropoint}
       {\doifelse{\XMLop{content-width}}{auto}
          {\FOgraphicwidth\zeropoint}
          {\FOgraphicwidth1em
           \setpercentdimen\FOgraphicwidth {\XMLop{content-width}}}}}

  % leeg maken vars gaat ook goed, dan een \externalfigure

  % todo : height/width scale-to-fit: factor=...

  \setbox\scratchbox\hbox
    {\setFOimagename{\XMLpar{fo:external-graphic}{src}{dummy}}
     \ifdim\FOgraphicheight>\zeropoint
       \ifdim\FOgraphicwidth>\zeropoint
         \externalfigure[\FOimagename][height=\FOgraphicheight,width=\FOgraphicwidth]
       \else
         \externalfigure[\FOimagename][height=\FOgraphicheight]
       \fi
     \else
       \ifdim\FOgraphicwidth>\zeropoint
         \externalfigure[\FOimagename][width=\FOgraphicwidth]
       \else
         \externalfigure[\FOimagename]
       \fi
     \fi}

  \dontleavehmode \XMLval{external-graphic:align}{\XMLop{vertical-align}}{}{\box\scratchbox}

\stopsetups

%D fo:instream-foreign-object

% like external-graphic, only no src

% \XMLattributeset{fo:aural},
% display-align=,
% line-height=,
% text-align=,

\defineXMLprocess
  [fo:instream-foreign-object]
  [\XMLattributeset{fo:inherited},
   id=,
   \XMLattributeset{fo:accessibility},
   \XMLattributeset{fo:margin-inline},
   \XMLattributeset{fo:border-padding-background},
   \XMLattributeset{fo:relative-position},
   alignment-adjust=,
   alignment-baseline=,
   baseline-shift=,
   block-progression-dimension=,
   clip=,
   content-height=,
   content-type=,
   content-width=,
   dominant-baseline=,
   height=,
   inline-progression-dimension=,
   keep-with-next=,
   keep-with-previous=,
   overflow=,
   scaling=,
   scaling-method=,
   width=]

%D Element: fo:inline

% \XMLattributeset{fo:aural},
% line-height=,
% wrap-option=,
% color=,
% keep-together=,

\defineXMLnestedenvironmentsave
  [fo:inline]
  [\XMLattributeset{fo:inherited},
   id=,
   \XMLattributeset{fo:accessibility},
   \XMLattributeset{fo:border-padding-background},
   \XMLattributeset{fo:font},
   \XMLattributeset{fo:margin-inline},
   \XMLattributeset{fo:relative-position},
   alignment-adjust=,
   alignment-baseline=,
   baseline-shift=,
   block-progression-dimension=,
   dominant-baseline=,
   height=,
   inline-progression-dimension=,
   keep-with-next=,
   keep-with-previous=,
   text-decoration=,
   visibility=,
   width=]
  {\beginXMLelement}
  {\directsetup{fo:inline:process}\endXMLelement}

% baseline-shift: baseline sub super % dimen inherit

\chardef\isolatedwordsmode=1

\newdimen\wordshiftamount

\def\shiftedword{\raise\wordshiftamount\hbox}

\def\shiftedwords#1{\processisolatedwords{#1}\shiftedword}
\def\normalwords #1{\processisolatedwords{#1}\hbox}
\def\highwords   #1{\processisolatedwords{#1}\high}
\def\lowwords    #1{\processisolatedwords{#1}\low}

\mapXMLvalue {fo:baseline-shift} {baseline} {\normalwords}
\mapXMLvalue {fo:baseline-shift} {sub}      {\lowwords}
\mapXMLvalue {fo:baseline-shift} {super}    {\highwords}
\mapXMLvalue {fo:baseline-shift} {inherit}  {\firstofoneargument}

\startsetups fo:inline:process

  \directsetup{fo:position:start}

  \dontleavehmode

  \doFOreference{fo:inline}

  \begingroup

  \directsetup{fe:setup}
  \directsetup{fo:space:start}

  \begingroup

  \directsetup{fo:hyphenation:setup}
  \directsetup{fo:font:setup}

  \doifelsenothing{\XMLop{baseline-shift}}
    {\XMLflushself}
    {\doifXMLvalelse{fo:baseline-shift}{\XMLop{baseline-shift}}
       {\XMLval{fo:baseline-shift}{\XMLop{baseline-shift}}{}{\XMLflushself}}
       {\wordshiftamount\lineheight
        \setpercentdimen\wordshiftamount{\XMLop{baseline-shift}}
        \shiftedwords{\XMLflushself}}}

  \endgroup

  \directsetup{fo:space:end}

  \endgroup

  \directsetup{fo:position:stop}

\stopsetups

\startsetups fo:position:start
  \begingroup
  \directsetup{fo:position:\XMLop{position}:start}
  \begingroup
\stopsetups

\startsetups fo:position:stop
  \endgroup
  \directsetup{fo:position:\XMLop{position}:stop}
  \endgroup
\stopsetups

\startsetups fo:position:static:start
\stopsetups

\startsetups fo:position:static:stop
\stopsetups

\startsetups fo:position:fixed:start
  \FOcontainerWW\paperwidth
  \FOcontainerHH\paperheight
  \def\FOlayername{regions}
  \directsetup{fo:preset:layer}
  \setlayer[regions]{\vbox \bgroup \setlocalhsize}
\stopsetups

\startsetups fo:position:fixed:stop
  \egroup
\stopsetups

\positioningpartrue \positioningtrue

\startsetups fo:position:absolute:start
  \setbox\FOpositionbox\hbox\bgroup
\stopsetups

\startsetups fo:position:absolute:stop
  \egroup
  % evt uitstellen tot otr, zodat text/realfolio is solved
  \edef\FOpartag{p:\number\parposcounter}
  \edef\FOtxttag{text:\realfolio}
  \FOcontainerWW\MPplus\FOpartag{1}{0pt}
  \FOcontainerHH\zeropoint % todo: add anchors to each 'object'
  \directsetup{fo:preset:position}
  \setlayer
    [xsl-region-body]
    [preset=lefttop,
     hoffset=\dimexpr(\MPx\FOtxttag-\MPx\FOpartag),
     voffset=\dimexpr(\MPy\FOtxttag+\MPh\FOtxttag-\MPy\FOpartag-\MPh\FOpartag)]
    {\iftracingFO \ruledhbox \bgroup \fi
     \offset
       [method=fixed,
        leftoffset=\FOcontainerL,
        rightoffset=\FOcontainerR,
        topoffset=\FOcontainerT,
        bottomoffset=\FOcontainerB]
       {\box\FOpositionbox}
     \iftracingFO \egroup \fi}
\stopsetups

\newbox\FOpositionbox

\startsetups fo:position:relative:start
  \setbox\FOpositionbox\hbox\bgroup
\stopsetups

\startsetups fo:position:relative:stop
  \egroup
  \FOcontainerWW\wd\FOpositionbox
  \FOcontainerHH\dimexpr(\ht\FOpositionbox+\dp\FOpositionbox)
  \directsetup{fo:preset:position}
  \iftracingFO \ruledhbox \bgroup \fi
  \offset
    [method=fixed,
     leftoffset=\FOcontainerL,
     rightoffset=\FOcontainerR,
     topoffset=\FOcontainerT,
     bottomoffset=\FOcontainerB]
    {\box\FOpositionbox}
  \iftracingFO \egroup \fi
\stopsetups

%

\startsetups fo:preset:position

   \FOcontainerW\zeropoint \FOcontainerL\zeropoint \FOcontainerR\zeropoint
   \FOcontainerH\zeropoint \FOcontainerT\zeropoint \FOcontainerB\zeropoint

   \doifnot{\XMLop{left}}  {auto}{\FOcontainerL\FOcontainerWW\setpercentdimen\FOcontainerL{\XMLop{left}}}
   \doifnot{\XMLop{right}} {auto}{\FOcontainerR\FOcontainerWW\setpercentdimen\FOcontainerR{\XMLop{right}}}
   \doifnot{\XMLop{top}}   {auto}{\FOcontainerT\FOcontainerHH\setpercentdimen\FOcontainerT{\XMLop{top}}}
   \doifnot{\XMLop{bottom}}{auto}{\FOcontainerB\FOcontainerHH\setpercentdimen\FOcontainerB{\XMLop{bottom}}}

   \doifnot{\XMLop{width}} {auto}{\FOcontainerW\FOcontainerWW\setpercentdimen\FOcontainerW{\XMLop{width}}}
   \doifnot{\XMLop{height}}{auto}{\FOcontainerH\FOcontainerHH\setpercentdimen\FOcontainerH{\XMLop{height}}}

\stopsetups

\startsetups fo:preset:layer

   \directsetup{fo:preset:position}

   \setuplayer
     [\FOlayername]
     [width=\FOcontainerWW,
      height=\FOcontainerHH]

   \ifzeropt\FOcontainerW
     \FOcontainerW\dimexpr(\FOcontainerWW-\FOcontainerL-\FOcontainerR)
   \fi
   \ifzeropt\FOcontainerH
     \FOcontainerH\dimexpr(\FOcontainerHH-\FOcontainerT-\FOcontainerB)
   \fi

   \ifzeropt\FOcontainerB
     \ifzeropt\FOcontainerL
       \setuplayer[\FOlayername][preset=righttop,   x=\FOcontainerR,y=\FOcontainerT]
     \else
       \setuplayer[\FOlayername][preset=lefttop,    x=\FOcontainerL,y=\FOcontainerT]
     \fi
   \else
     \ifzeropt\FOcontainerL
       \setuplayer[\FOlayername][preset=rightbottom,x=\FOcontainerR,y=\FOcontainerB]
     \else
       \setuplayer[\FOlayername][preset=leftbottom, x=\FOcontainerL,y=\FOcontainerB]
     \fi
   \fi

\stopsetups

%D Element: fo:inline-container

% display-align=,
% line-height=,
% reference-orientation=,
% writing-mode=,
% keep-together=,

\defineXMLenvironment
  [fo:inline-container]
  [\XMLattributeset{fo:inherited},
   id=,
   \XMLattributeset{fo:border-padding-background},
   \XMLattributeset{fo:margin-inline},
   \XMLattributeset{fo:relative-position},
   alignment-adjust=,
   alignment-baseline=,
   baseline-shift=,
   block-progression-dimension=,
   clip=,
   dominant-baseline=,
   height=,
   inline-progression-dimension=,
   keep-with-next=,
   keep-with-previous=,
   overflow=,
   width=]
  {\beginXMLelement\begingroup}
  {\endgroup\endXMLelement}

%D Element: fo:leader

% also a kind of fake fill

% \XMLattributeset{fo:aural},
% color=,
% line-height=,
% word-spacing=,

\defineXMLenvironmentsave
  [fo:leader]
  [\XMLattributeset{fo:inherited},
   id=,
   \XMLattributeset{fo:accessibility},
   \XMLattributeset{fo:border-padding-background},
   \XMLattributeset{fo:font},
   \XMLattributeset{fo:margin-inline},
   \XMLattributeset{fo:relative-position},
   \XMLattributeset{fo:leader-and-rule},
   alignment-adjust=,
   alignment-baseline=,
   baseline-shift=,
   dominant-baseline=,
%    text-depth=,
%    text-altitude=,
   keep-with-next=,
   keep-with-previous=,
   letter-spacing=,
   text-shadow=,
   visibility=]
  {\beginXMLelement}
  {\directsetup{fo:leader:process}\endXMLelement}

\mapXMLvalue {fo:leader-pattern} {space}       {\hfill}
\mapXMLvalue {fo:leader-pattern} {dots}        {.}
\mapXMLvalue {fo:leader-pattern} {rule}        {\hrulefill}
\mapXMLvalue {fo:leader-pattern} {use-content} {\XMLflushself}

% todo: speed up

\startsetups fo:leader:process

%   \tracebackXMLattribute{leader-pattern-width}

  \strut \leaders

%   \edefXMLinh \FOlepatwd {leader-pattern-width}

    \hbox to \XMLinh{leader-pattern-width}
      {\hss\XMLval{fo:leader-pattern}{\XMLinh{leader-pattern}}{\hfill}\hss}

  \hfill \strut

\stopsetups

%D Element: fo:pagenumber

% \XMLattributeset{fo:aural},
% line-height=,
% wrap-option=,
% letter-spacing=,
% text-transform=,
% word-spacing=,

\defineXMLsingular
  [fo:page-number]
  [\XMLattributeset{fo:inherited},
   id=,
   \XMLattributeset{fo:accessibility},
   \XMLattributeset{fo:border-padding-background},
   \XMLattributeset{fo:font},
   \XMLattributeset{fo:margin-inline},
   \XMLattributeset{fo:relative-position},
   alignment-adjust=,
   alignment-baseline=,
   baseline-shift=,
   dominant-baseline=,
   keep-with-next=,
   keep-with-previous=,
   score-spaces=,
%    text-altitude=,
   text-decoration=,
%    text-depth=,
   text-shadow=,
   visibility=]
  {\directsetup{fo:page-number:process}}

\newcount\FOpnrefcounter

\startsetups fo:page-number:process

  \doifelsenothing{\XMLpar{fo:page-sequence}{format}{}}
    {\pagenumber}
    {\ifinotr
       \globallet\FOpnrefnumber\folio
     \else
        \global\advance\FOpnrefcounter\plusone
        \pagereference[pnref:\the\FOpnrefcounter]
        \doifreferencefoundelse{pnref:\the\FOpnrefcounter}
          {\globallet\FOpnrefnumber\currentfolioreference}
          {\globallet\FOpnrefnumber\folio}
     \fi
     \expanded{\handletokens\XMLpar{fo:page-sequence}{format}{}}\with{\handleFOformat{\FOpnrefnumber}}}

\stopsetups

\defineconversion[1][\numbers]

\def\handleFOformat#1#2%
  {\defconvertedargument\ascii{#2}%
   \doifconversiondefinedelse\ascii{\convertnumber\ascii{#1}}{#2}}

%D Element: fo:pagenumber-citation

% same as page-number

% \XMLattributeset{fo:aural},
% line-height=,
% wrap-option=,
% letter-spacing=,
% text-transform=,
% word-spacing=,

\defineXMLsingular
  [fo:page-number-citation]
  [\XMLattributeset{fo:inherited},
   id=,
   ref-id=,
   \XMLattributeset{fo:accessibility},
   \XMLattributeset{fo:border-padding-background},
   \XMLattributeset{fo:font},
   \XMLattributeset{fo:margin-inline},
   \XMLattributeset{fo:relative-position},
   alignment-adjust=,
   alignment-baseline=,
   baseline-shift=,
   dominant-baseline=,
   keep-with-next=,
   keep-with-previous=,
   score-spaces=,
%    text-altitude=,
   text-decoration=,
%    text-depth=,
   text-shadow=,
   visibility=]
  {\directsetup{fo:page-number-citation:process}}

\startsetups fo:page-number-citation:process

  \doifreferencefoundelse{\XMLop{ref-id}}
    {\globallet\FOpnrefnumber\currentfolioreference
     \globallet\FOpnrefformat\currenttextreference}
    {\gdef\FOpnrefnumber{?}
     \gdef\FOpnrefformat{}}

  \doifelsenothing{\FOpnrefformat}
    {\FOpnrefnumber}
    {\expanded{\handletokens\FOpnrefformat}\with{\handleFOformat{\FOpnrefnumber}}}

\stopsetups

%D Element: fo:table-and-caption

% \XMLattributeset{fo:aural},
% text-align=,
% caption-side=,
% intrusion-displace=,
% keep-together=,

\defineXMLenvironment
  [fo:table-and-caption]
  [\XMLattributeset{fo:inherited},
   id=,
   \XMLattributeset{fo:accessibility},
   \XMLattributeset{fo:border-padding-background},
   \XMLattributeset{fo:margin-block},
   \XMLattributeset{fo:relative-position},
   break-after=,
   break-before=,
   keep-with-next=,
   keep-with-previous=]
  {\beginXMLelement}
  {\endXMLelement}

%D Element: fo:table fo:table-caption fo:table-header fo:table-footer
%D          to:table-column fo:table-body fo:table-row fo:table-cell

% \XMLattributeset{fo:aural},
% border-collapse=,
% border-separation=,
% intrusion-displace=,
% keep-together=,
% writing-mode=,

\defineXMLenvironment
  [fo:table]
  [\XMLattributeset{fo:inherited},
   id=,
   \XMLattributeset{fo:accessibility},
   \XMLattributeset{fo:border-padding-background},
   \XMLattributeset{fo:margin-block},
   \XMLattributeset{fo:relative-position},
   block-progression-dimension=,
   border-after-precedence=,
   border-before-precedence=,
   border-start-precedence=,
   border-end-precedence=,
   break-after=,
   break-before=,
   inline-progression-dimension=,
   height=,
   keep-with-next=,
   keep-with-previous=,
   table-layout=,
   table-omit-footer-at-break=,
   table-omit-header-at-break=,
% text-indent=0pt, % yes or no?
   width=]
  {\beginXMLelement
   \bTABLE % [option=stretch] %
   \newcounter\FOtablecolumn}
  {\eTABLE
   \endXMLelement}

\newdimen\FOtableW
\newdimen\FOtableH

\defineXMLsingular
  [fo:table-column]
  [\XMLattributeset{fo:inherited},
   \XMLattributeset{fo:border-padding-background}, % only background, not the rest, make subset
   border-after-precedence=,
   border-before-precedence=,
   border-end-precedence=,
   border-start-precedence=,
   column-width=,
   column-number=,
   number-columns-repeated=,
   number-columns-spanned=,
   visibility=]
  {\directsetup{fo:table-column:action}}

% \XMLattributeset{fo:aural},
% intrusion-displace=,
% keep-together=,

\defineXMLprocess
  [fo:table-caption]
  [\XMLattributeset{fo:inherited},
   id=,
   \XMLattributeset{fo:accessibility},
   \XMLattributeset{fo:border-padding-background},
   \XMLattributeset{fo:relative-position},
   block-progression-dimension=,
   height=,
   inline-progression-dimension=,
   width=]

% \XMLattributeset{fo:aural},

\defineXMLnested
  [fo:table-header]
  [\XMLattributeset{fo:inherited},
   id=,
   \XMLattributeset{fo:accessibility},
   \XMLattributeset{fo:border-padding-background},
   \XMLattributeset{fo:relative-position},
   border-after-precedence=,
   border-before-precedence=,
   border-end-precedence=,
   border-start-precedence=,
   visibility=]
  {\beginXMLelement\bTABLEhead}
  {\eTABLEhead\endXMLelement}

% \XMLattributeset{fo:aural},

\defineXMLnested
  [fo:table-footer]
  [\XMLattributeset{fo:inherited},
   id=,
   \XMLattributeset{fo:accessibility},
   \XMLattributeset{fo:border-padding-background},
   \XMLattributeset{fo:relative-position},
   border-after-precedence=,
   border-before-precedence=,
   border-end-precedence=,
   border-start-precedence=,
   visibility=]
  {\beginXMLelement\bTABLEfoot}
  {\eTABLEfoot\endXMLelement}

% \XMLattributeset{fo:aural},

\defineXMLnested
  [fo:table-body]
  [\XMLattributeset{fo:inherited},
   id=,
   \XMLattributeset{fo:accessibility},
   \XMLattributeset{fo:border-padding-background},
   \XMLattributeset{fo:relative-position},
   border-after-precedence=,
   border-before-precedence=,
   border-end-precedence=,
   border-start-precedence=,
   visibility=]
  {\beginXMLelement\bTABLEbody}
  {\eTABLEbody\endXMLelement}

% TODO: when stretch and when not

% \XMLattributeset{fo:aural},
% keep-together=,

\defineXMLnested
  [fo:table-row]
  [\XMLattributeset{fo:inherited},
   id=,
   \XMLattributeset{fo:accessibility},
   \XMLattributeset{fo:border-padding-background},
   \XMLattributeset{fo:relative-position},
   border-after-precedence=,
   border-before-precedence=,
   border-end-precedence=,
   border-start-precedence=,
   break-after=,
   break-before=,
   height=,
   keep-with-next=,
   keep-with-previous=,
   visibility=]
  {\beginXMLelement
   \directsetup{fo:table-row:start}%
   \expanded{\bTR[\the\scratchtoks]}%
   \beginXMLelement}
  {\endXMLelement
   \eTR
   \directsetup{fo:table-row:stop}
   \endXMLelement}

\startsetups fo:table-row:start

   \inTABLErowtrue

   \scratchtoks\emptytoks

   \doifsomething{\XMLop{height}}
     {\FOtableH\textheight
      \setpercentdimen\FOtableH{\XMLop{height}}
      \appendetoks
        height=\the\FOtableH
      \to \scratchtoks}

   \appendetoks
     ,extras={\rescanXMLattributes{fo:table-row}}
   \to\scratchtoks

\stopsetups

\startsetups fo:table-row:stop

   \inTABLErowfalse

\stopsetups

% \XMLattributeset{fo:aural},
% display-align=,
% relative-align=,
% empty-cells=,

\newif\ifinTABLErow
\newdimen\FOtablecellwidth
\newdimen\FOtablecellheight

\defineXMLnested
  [fo:table-cell]
  [\XMLattributeset{fo:inherited},
   id=,
   \XMLattributeset{fo:accessibility},
   \XMLattributeset{fo:border-padding-background},
   \XMLattributeset{fo:relative-position},
   border-after-precedence=,
   border-before-precedence=,
   border-end-precedence=,
   border-start-precedence=,
   column-number=,
   ends-row=,
   height=,
   inline-progression-dimension=,
   number-columns-spanned=1,
   number-rows-spanned=1,
   starts-row=,
   width=]
  {\directsetup{fo:table-cell:start}%
   \expanded{\bTD[\the\scratchtoks]}%
   \beginXMLelement}
  {\endXMLelement
   \eTD
   \directsetup{fo:table-cell:stop}}

\startsetups fo:table-column:action

  \doifelsenothing{\XMLop{column-number}}
    {\increment\FOtablecolumn}
    {\edef\FOtablecolumn{\XMLop{column-number}}
     \expanded{\setupTABLE[column][\FOtablecolumn][n=\FOtablecolumn]}}

  \doifsomething{\XMLop{column-width}}
    {%\setlocalhsize
     %\FOtableW\localhsize
     \analyzefunction{\XMLop{column-width}}%
     % hm, we need to set localhsize earlier
     \doifelse\functionname{proportional-column-width}
       {\FOtableW\functionA\textwidth}
       {\FOtableW\textwidth
        \setpercentdimen\FOtableW{\XMLop{column-width}}}%
     \expanded{\setupTABLE[column][\FOtablecolumn][width=\the\FOtableW]}}

  \doif{\XMLop{border-style}}{none}
    {\expanded{\setupTABLE[column][\FOtablecolumn][frame=off]}}

  \doifelsenothing{\XMLop{display-align}}
    {\doifsomething{\XMLop{text-align}}
       {\expanded{\setupTABLE[column][\FOtablecolumn]
            [align=\XMLpav{fo:align-key}{fo:table-column}{text-align}{normal}]}}}
    {\doifsomething{\XMLop{text-align}}
        {\expanded{\setupTABLE[column][\FOtablecolumn]
            [align={\XMLpav{fo:display-align}{fo:table-column}{display-align}{high},\XMLpav{fo:align-key}{fo:table-column}{text-align}{normal}}]}}
        {\expanded{\setupTABLE[column][\FOtablecolumn]
            [align=\XMLpav{fo:display-align}{fo:table-column}{display-align}{high}]}}}

\expanded{\setupTABLE[column][\FOtablecolumn][extras={\rescanXMLattributes{fo:table-column}}]}

\stopsetups

\startsetups fo:table-cell:start

  \doif{\XMLop{starts-row}}{true}{\ifinTABLErow\eTR\inTABLErowfalse\fi}

  \ifinTABLErow\else\bTR\inTABLErowtrue\fi

  \doifelsenothing{\XMLop{background-color}}
    {\let\FoTableBG\empty}
    {\checkhexcolor[\XMLop{background-color}]
     \doifcolorelse{\XMLop{background-color}}
       {\def\FoTableBG{color}}
       {\setXMLpar{fo:table-cell}{background-color}{}
        \let\FoTableBG\empty}}

%   \doifelse{\XMLpar{fo:table-cell}{width}{}}{}
%     {\def\pFOtablewidth{fit}}
%     {\FOtablecellwidth\textwidth % probably must be localhsize or frozen at an outer level
%      \setpercentdimen\FOtablecellwidth{\XMLpar{fo:table-cell}{width}{0pt}}%
%      \edef\pFOtablewidth{\the\FOtablecellwidth}}%

%   \doifelse{\XMLpar{fo:table-cell}{height}{}}{}
%     {\def\pFOtableheight{fit}}
%     {\FOtablecellheight\textheight % probably must be localhsize or frozen at an outer level
%      \setpercentdimen\FOtablecellheight{\XMLpar{fo:table-cell}{height}{0pt}}%
%      \edef\pFOtableheight{\the\FOtablecellheight}}%

  \scratchtoks\emptytoks \appendetoks
%     style=\noexpand\directsetup{fo:font:setup}, % else not expanded
    nx=\XMLop{number-columns-spanned},
    ny=\XMLop{number-rows-spanned},
    n=\XMLop{column-number},
    background=\FoTableBG
  \to \scratchtoks

  \doifnot{\XMLop{border-style}}{none}
    {\appendetoks
       ,frame=on
     \to\scratchtoks}

  \doifsomething{\XMLop{background-color}}
    {\appendetoks
       ,backgroundcolor=\XMLop{background-color}
     \to \scratchtoks}

  % todo : padding

  \doifsomething{\XMLop{padding}}
    {\appendetoks
       ,offset=\XMLop{padding}
     \to \scratchtoks}

  % todo: interference with presets in column (outer level) -> \setupcolumn[column] ...;
  % misschien meerdere align switches

  \doifelsenothing{\XMLop{display-align}}
    {\doifsomething{\XMLop{text-align}}
       {\appendetoks
          ,align=\XMLpav{fo:align-key}{fo:table-cell}{text-align}{normal}
        \to \scratchtoks}}
    {\doifsomething{\XMLop{text-align}}
       {\appendetoks
          ,align={\XMLpav{fo:display-align}{fo:table-cell}{display-align}{high},\XMLpav{fo:align-key}{fo:table-cell}{text-align}{normal}},
        \to \scratchtoks}
       {\appendetoks
          ,align=\XMLpav{fo:display-align}{fo:table-cell}{display-align}{high}
        \to \scratchtoks}}

   \appendetoks
     ,extras={\rescanXMLattributes{fo:table-cell}}
   \to\scratchtoks

\stopsetups

% \startsetups fo:table-cell:setup

%   [\XMLpar{fo:table-cell}{text-indent}{}]

%   \edefXMLinhpar\xFOtextindent{fo:table-cell}{text-indent}

%   \doifsomething\xFOtextindent
%     {\scratchdimen\hsize
%      \setpercentdimen\scratchdimen\xFOtextindent
%      \expanded{\setupindenting[\the\scratchdimen]}}

% \stopsetups

\startsetups fo:table-cell:stop

  \doif{\XMLop{ends-row}}{true}{\eTR\inTABLErowfalse}

\stopsetups

%D Element: fo:list-block fo:list-item fo:list-body fo:list-item-label

% \XMLattributeset{fo:aural},
% intrusion-displace=,
% keep-together=,
% provisional-distance-between-starts=24pt,
% provisional-label-separation=6pt,

\defineXMLenvironment
  [fo:list-block]
  [\XMLattributeset{fo:inherited},
   id=,
   \XMLattributeset{fe:tracing},
   \XMLattributeset{fo:accessibility},
   \XMLattributeset{fo:border-padding-background},
   \XMLattributeset{fo:margin-block},
   \XMLattributeset{fo:relative-position},
   break-after=,
   break-before=,
   keep-with-next=,
%  space-between-list-rows=, % ? mentioned in bradley
text-indent=0pt, % yes
   keep-with-previous=]
  {\beginXMLelement\directsetup{fo:list:start}}
  {\directsetup{fo:list:stop}\endXMLelement}

\startsetups fo:list:start
  \endgraf
  \begingroup
  \directsetup{fe:setup}
  \disablemode[fo:in-list]
  % \forgetall, no!
%   \directsetup{fo:break-and-space:before}
  \directsetup{fo:indent:setup}
  \begingroup
\stopsetups

\startsetups fo:list:stop
  \endgraf
  \endgroup
%   \directsetup{fo:break-and-space:after}
  \endgroup
\stopsetups

% \XMLattributeset{fo:aural},
% relative-align=,
% intrusion-displace=,
% keep-together=,

% The list model is plain stupid. Instead of just defining a few mechanism
% or using some kind of type attribute, a strange mechanism of functions is
% used. Why on the one hand introduce redundant attributes and on the other
% hand safe a few elements. A proper segmentation of the problem would have
% brought better solutions.

\defineXMLenvironment
  [fo:list-item]
  [\XMLattributeset{fo:inherited},
   id=,
   \XMLattributeset{fe:tracing},
   \XMLattributeset{fo:accessibility},
   \XMLattributeset{fo:border-padding-background},
   \XMLattributeset{fo:margin-block},
   \XMLattributeset{fo:relative-position},
   break-after=,
   break-before=,
   keep-with-next=,
   keep-with-previous=]
  {\beginXMLelement\directsetup{fo:list-item:start}}
  {\directsetup{fo:list-item:stop}\endXMLelement}

% check what is needed

\newdimen\FOlistitemlabelhsize      \newdimen\FOlistitembodyhsize     \newdimen\FOlistitemdistance
\newdimen\FOlistitemlabelleftskip   \newdimen\FOlistitembodyleftskip
\newdimen\FOlistitemlabelrightskip  \newdimen\FOlistitembodyrightskip

\defineXMLnestedsave
  [fo:list-item-body]
  [\XMLattributeset{fo:inherited},
   id=, % keep-together=,
   \XMLattributeset{fo:accessibility}]

\defineXMLnestedsave
  [fo:list-item-label]
  [\XMLattributeset{fo:inherited},
   id=, % keep-together=,
   \XMLattributeset{fo:accessibility}]

\startsetups fo:list-item:start

  \bgroup

%   \startmode[fo:in-list]
%    \doifsomething{\XMLpar{fo:list-block}{space-between-list-rows}{}}
%      {\vskip\XMLpar{fo:list-block}{space-between-list-rows}{}} % todo ! ! ! !
%   \stopmode

  \enablemode[fo:in-list]

\stopsetups

% todo : relative-align in list item

\newif\ifFOlabelend
\newif\ifFObodystart

\startsetups fo:list-item:stop

  % 24pt en 6pt in fo:root instellen

  % \tracebackXMLattribute{provisional-distance-between-starts}

  \edefXMLinh \FOprodis {provisional-distance-between-starts}
  \edefXMLinh \FOprolab {provisional-label-separation}

%   \edef\FOprodis{\XMLpar{fo}{provisional-distance-between-starts}{}}
%   \edef\FOprolab{\XMLpar{fo}{provisional-label-separation}{}}

  \edef\FOprodis{\ifx\FOprodis\empty24pt\else\FOprodis\fi}
  \edef\FOprolab{\ifx\FOprolab\empty 6pt\else\FOprolab\fi}

  \setlocalhsize

  \FOlistitemlabelleftskip \zeropoint
  \FOlistitemlabelrightskip\zeropoint
  \FOlistitembodyleftskip  \zeropoint
  \FOlistitembodyrightskip \zeropoint

  \doifelse{\XMLpar{fo:list-item-label}{end-indent}  {}}{label-end()} \FOlabelendtrue \FOlabelendfalse
  \doifelse{\XMLpar{fo:list-item-body} {start-indent}{}}{body-start()}\FObodystarttrue\FObodystartfalse

  \setpercentdimen\FOlistitemlabelleftskip {\XMLpar{fo:list-item-label}{start-indent}{0pt}}
  \setpercentdimen\FOlistitembodyrightskip {\XMLpar{fo:list-item-body} {end-indent}  {0pt}}

  % maybe i need to implement something configurable

  \ifFObodystart
    \ifFOlabelend
      \FOlistitemlabelrightskip\dimexpr(\localhsize-\FOlistitemlabelleftskip-\FOprodis+\FOprolab)
      \FOlistitembodyleftskip\dimexpr(\FOlistitemlabelleftskip+\FOprodis)
      \FOlistitemlabelhsize\dimexpr(\FOprodis-\FOprolab)
    \else
      \setpercentdimen\FOlistitemlabelrightskip{\XMLpar{fo:list-item-label}{end-indent}{0pt}}
      \FOlistitemlabelhsize\dimexpr(\localhsize-\FOlistitemlabelleftskip-\FOlistitemlabelrightskip)
      \FOlistitembodyleftskip\dimexpr(\FOlistitemlabelleftskip+\FOlistitemlabelhsize+\FOprolab)
    \fi
    \FOlistitemdistance  \dimexpr(\FOprolab)
  \else
    \setpercentdimen\FOlistitembodyleftskip{\XMLpar{fo:list-item-body}{start-indent}{0pt}}
    \ifFOlabelend
      \FOlistitemlabelrightskip\dimexpr(\localhsize-\FOlistitembodyleftskip+\FOprolab)
      \FOlistitemlabelhsize\dimexpr(\localhsize-\FOlistitemlabelleftskip-\FOlistitemlabelrightskip)
      \FOlistitemdistance  \dimexpr(\FOprolab)
    \else
      \setpercentdimen\FOlistitemlabelrightskip{\XMLpar{fo:list-item-label}{end-indent}{0pt}}
      \FOlistitemlabelhsize\dimexpr(\localhsize-\FOlistitemlabelleftskip-\FOlistitemlabelrightskip)
      \FOlistitemdistance  \dimexpr(\FOlistitembodyleftskip-\FOlistitemlabelleftskip-\FOlistitemlabelhsize)
    \fi
  \fi

  % is this fall back permitted ?

  \ifzeropt\FOlistitemlabelleftskip \ifzeropt\FOlistitemlabelrightskip
    \FOlistitembodyleftskip\FOprodis
    \FOlistitemdistance\FOprolab
    \FOlistitemlabelhsize\dimexpr(\FOlistitembodyleftskip-\FOlistitemdistance)
  \fi \fi

  %

  \FOlistitembodyhsize\localhsize

  \advance\FOlistitembodyhsize-\FOlistitembodyleftskip
  \advance\FOlistitembodyhsize-\FOlistitembodyrightskip

  \doifelse{\XMLpar{fo:list-item}{display-align}{}}{center}
    {\directsetup{fo:list-item:display}}
    {\directsetup{fo:list-item:text}}

  \egroup

\stopsetups

% todo: textindent

\startsetups fo:list-item:display

  \endgraf

  \advance\leftskip \FOlistitemlabelleftskip
  \advance\rightskip\FOlistitembodyrightskip

  \dontleavehmode \valign\bgroup\forgetall\vss##\vss\cr
    \iftracingFO\ruledvtop\else\vbox\fi{\hsize\FOlistitemlabelhsize\directsetup{fo:list-item-label:setup}\XMLflush{fo:list-item-label}}\cr
    \iftracingFO\ruledvtop\else\vbox\fi{\hsize\FOlistitembodyhsize \directsetup{fo:list-item-body:setup}\XMLflush{fo:list-item-body}}\cr
  \egroup

% \dontleavehmode \placesidebyside % or maybe paired boxes (legends)
%   {\ruledvtop{\forgetall\hsize\FOlistitemlabelhsize\XMLflush{fo:list-item-label}}}
%   {\ruledvtop{\forgetall\hsize\FOlistitembodyhsize \XMLflush{fo:list-item-body}}}

  \endgraf

\stopsetups

\newtoks\savedeverypar \savedeverypar\everypar

\startsetups fo:list-item:text

  \everypar\savedeverypar % \appendtoksonce\insertparagraphintro\to\everypar % hack, binnen footnote ...

  \advance\leftskip \FOlistitembodyleftskip
  \advance\rightskip\FOlistitembodyrightskip

  \setupparagraphintro[first][\directsetup{fo:list-item-label:process}]
  \setupparagraphintro[next] [\begstrut\resetpenalties\clubpenalties]
  \directsetup{fo:list-item-body:setup}
  \XMLflush{fo:list-item-body}\endstrut

\stopsetups

\startsetups fo:list-item-label:setups

  \edefXMLinhpar\xFOtextindent{fo:item-label}{text-indent}

  \doifsomething\xFOtextindent
    {\scratchdimen\hsize
     \setpercentdimen\scratchdimen\xFOtextindent
     \expanded{\setupindenting[\the\scratchdimen]}}

\stopsetups

\startsetups fo:list-item-body:setups

  \edefXMLinh\xFOtextindent{fo:item-body}{text-indent}

  \doifsomething\xFOtextindent
    {\scratchdimen\hsize
     \setpercentdimen\scratchdimen\xFOtextindent
     \expanded{\setupindenting[\the\scratchdimen]}}

\stopsetups

\newbox\FOitembox

\startsetups fo:list-item-label:process

  \setbox \FOitembox \iftracingFO \ruledvtop \else \vtop \fi \bgroup
    \forgetall
    \postponenotes
    \hsize\FOlistitemlabelhsize
    \directsetup{fo:list-item-label:setup}
    \XMLflush{fo:list-item-label}
  \egroup
  \getnoflines{\dimexpr(\ht\FOitembox+\dp\FOitembox)}
  \setpenalties\clubpenalties\noflines\maxdimen
  \strut\llap{\box\FOitembox\hskip\FOlistitemdistance}

\stopsetups

% \setlocalhsize \hsize\localhsize

%D Element: fo:basic-link

% \XMLattributeset{fo:aural},
% keep-together=,
% line-height=,

\defineXMLenvironmentsave
  [fo:basic-link]
  [\XMLattributeset{fo:inherited},
   id=,
   \XMLattributeset{fo:accessibility},
   \XMLattributeset{fo:border-padding-background},
   \XMLattributeset{fo:margin-inline},
   \XMLattributeset{fo:relative-position},
   alignment-adjust=,
   alignment-baseline=,
   baseline-shift=,
   destination-placement-offset=,
   dominant-baseline=,
   external-destination=,
   indicate-destination=,
   internal-destination=,
   keep-with-next=,
   keep-with-previous=,
   show-destination=,
   target-processing-context=,
   target-presentation-context=,
   target-stylesheet=]
  {}
  {\directsetup{fo:basic-link}}

\startsetups fo:basic-link

   \goto{\XMLflushself}[unknown]

\stopsetups

%D Element: fo:multi-switch fo:multi-case fo:multi-toggle fo:multi-properties fo:multi-property-set

\defineXMLprocess[fo:multi-switch]
\defineXMLprocess[fo:multi-case]
\defineXMLprocess[fo:multi-toggle]
\defineXMLprocess[fo:multi-properties]
\defineXMLprocess[fo:multi-property-set]

%D Element: fo:float

\defineXMLenvironmentsave
  [fo:float]
  [\XMLattributeset{fo:inherited},
   float=before,
   clear=]
  {}
  {\directsetup{fo:float:process}}

% clear: start end left right both none inherit
% float: before start end left right none

\mapXMLvalue {fo:float-position} {before} {here}   % todo
\mapXMLvalue {fo:float-position}  {start} {here}   % todo
\mapXMLvalue {fo:float-position}    {end} {here}   % todo
\mapXMLvalue {fo:float-position}   {left} {left}
\mapXMLvalue {fo:float-position}  {right} {right}
\mapXMLvalue {fo:float-position}   {none} {here}   % todo

\startsetups fo:float:process

  \placefigure
    [\XMLval{fo:float-position}{\XMLop{float}},none]
    {}
    {\XMLflushself}

\stopsetups

%D Element: fo:footnote fo:footnote-body

% Let's assume that 'whatever' contains the number or footnote marker.
%
% <fo:footnote>whatever<fo:footnote-body>note</fo:footnote-body></fo:footnote>

% todo xsl-footnote area

\defineXMLprocess
  [fo:footnote]
  [\XMLattributeset{fo:accessibility}]

\defineXMLargument
  [fo:footnote-body]
  [\XMLattributeset{fo:accessibility}]
  {\footnote[-]}

%D Element: fo:wrapper

\defineXMLenvironment % todo: all inheritable
  [fo:wrapper]
  [\XMLattributeset{fo:inherited},
   \XMLattributeset{fe:tracing},
   \XMLattributeset{fo:fonts},
   \XMLattributeset{fo:hyphenation}]
  {\beginXMLelement\begingroup\directsetup{fo:wrapper}}
  {\endgroup\endXMLelement}

\startsetups fo:wrapper

  \directsetup{fe:setup}
  \directsetup{fo:hyphenation:setup}
  \directsetup{fo:font:setup}

\stopsetups

%D Element: fo:marker fo:retrieve-marker

% In order to support 'retrieve-boundary' (page, page-sequence,
% document) I need to extend the context mark handler.

% This object will probably interfere with a too spacy layout since
% it is unaware if its surrounding.

\defineXMLenvironmentsave
  [fo:marker]
  [marker-class-name=unknown]
  {}
  {\directsetup{fo:marker:process}}

\startsetups fo:marker:process

  \doifelsemarking{fo:\XMLop{marker-class-name}}
    {} {\definerawmarking[fo:\XMLop{marker-class-name}]}

  \expanded{\marking[fo:\XMLop{marker-class-name}]{\XMLflushself}}

\stopsetups

\defineXMLcommand
  [fo:retrieve-marker]
  [retrieve-class-name=unknown,
   retrieve-position=first-starting-within-page,
   retrieve-boundary=]
  {\directsetup{fo:retrieve-marker:process}}

\mapXMLvalue {fo:marker-position} {first-starting-within-page} {first}    % first mark
\mapXMLvalue {fo:marker-position} {first-including-carryover}  {previous} % top mark
\mapXMLvalue {fo:marker-position} {last-starting-within-page}  {first}    % dunno
\mapXMLvalue {fo:marker-position} {last-ending-within-page}    {last}     % bot mark

\startsetups fo:retrieve-marker:process

   \expanded{\getmarking
     [fo:\XMLop{retrieve-class-name}]
     [\XMLval{fo:marker-position}{\XMLop{retrieve-position}}{first}]}

\stopsetups

%D Auxiliary macros

\unprotect

\def\noFOchecks#1\od{}

\def\FOassignskip#1#2#3%
  {\edef\!!stringa{\XMLpar{#1}{#2}\empty}%
   \edef\!!stringb{\XMLpar{#1}{#2.optimum}\empty}%
   \edef\!!stringc{\XMLpar{#1}{#2.minimum}\empty}%
   \edef\!!stringd{\XMLpar{#1}{#2.maximum}\empty}%
   \dimen0=\ifx\!!stringa\empty\zeropoint\else\!!stringa\fi
   \dimen2=\ifx\!!stringb\empty\dimen0   \else\!!stringb\fi
   \dimen4=\dimexpr(\ifx\!!stringd\empty\dimen0 \else\!!stringd\fi-\dimen2)\relax
   \dimen6=\dimexpr(\ifx\!!stringc\empty\dimen0 \else\!!stringc\fi-\dimen2)\relax
   #3=\dimen2 \ifzeropt\dimen4 \else\!!plus\dimen4 \fi\ifzeropt\dimen6 \else\!!minus\dimen6 \fi\relax}

\mapXMLvalue{fo:space:conditionality} {retain}  {\let\next\retainedskip }
\mapXMLvalue{fo:space:conditionality} {discard} {\let\next\discardedskip}
\mapXMLvalue{fo:space:conditionality} {}        {\let\next\discardedskip}

\mapXMLvalue{fo:space:precedence}     {force}   {\let\next\forcedskip}

\def\FOdoskip#1#2%
  {\begingroup
   \iftracingFO\showskips\fi
   \FOassignskip{#1}{#2}\scratchskip
   \XMLval{fo:space:conditionality}{\XMLpar{#1}{#2.conditionality}\empty}\empty
   \XMLval{fo:space:precedence}{\XMLpar{#1}{#2.precedence}\empty}\empty
   \ifdim\scratchskip=\zeropoint
     \ifdim\gluestretch\scratchskip=\zeropoint
       \ifdim\glueshrink\scratchskip=\zeropoint
         \let\next\gobbleoneargument
       \fi
     \fi
   \fi
   \next\scratchskip
   \endgroup}

\def\doFObeforeskip#1{\FOdoskip{#1}{space-before}}
\def\doFOafterskip #1{\FOdoskip{#1}{space-after}}

\def\FOassignspace#1#2#3%
  {\edef\!!stringa{\XMLpar{#1}{#2}\empty}%
   \ifx\!!stringa\empty
     #3=\zeropoint
   \else
     #3=1em% ?
     \setpercentdimen#3\!!stringa
   \fi
   \relax}

\def\FOdospace#1#2%
  {\begingroup
   \iftracingFO\showskips\fi
   \FOassignspace{#1}{#2}\scratchskip
   \ifdim\scratchskip=\zeropoint \else
     \hskip\scratchskip
   \fi
   \endgroup}

\def\doFOstartspace#1{\FOdospace{#1}{space-start}}
\def\doFOendspace  #1{\FOdospace{#1}{space-end}}

\def\checkFOborder#1#2%
  {\edef\FOattribute{\XMLpar{#1}{border-#2}\empty}%
   \ifx\FOattribute\empty\else
     \edef\FOtag{#1}%
     \edef\FOatt{border-#2}%
     \expanded{\docheckFOborder\FOattribute\space\relax\space\relax}\od
   \fi}

\def\docheckFOborder#1#2 #3%
  {\ifx#1\relax
     \expandafter\noFOchecks
   \else
     \doifhexcolorelse{#1#2}
       {\setXMLpar\FOtag{\FOatt-color}{#1#2}}
       {\doifelsenothing{\XMLval{fo:border-style}{#1#2}\empty}
          {\doifcolorelse{#1#2}
             {\setXMLpar\FOtag{\FOatt-color}{#1#2}}
             {\setXMLpar\FOtag{\FOatt-width}{#1#2}}}
          {\setXMLpar\FOtag{\FOatt-style}{#1#2}}}%
     \expandafter\docheckFOborder
   \fi#3}

\def\checkFOposition#1#2%
  {\edef\FOattribute{\XMLpar{#1}{#2-position}\empty}%
   \ifx\FOattribute\empty\else
     \edef\FOtag{#1}%
     \edef\FOatt{#2-position}%
     \scratchcounter\zerocount
     \expanded{\docheckFOposition\FOattribute\space\relax\space\relax}\od
   \fi}

\def\docheckFOposition#1#2 #3%
  {\ifx#1\relax
     \expandafter\noFOchecks
   \else
     \advance\scratchcounter\plusone
     \ifcase\scratchcounter
     \or
       \setXMLpar\FOtag{\FOatt-vertical}{#1#2}%
     \or
       \setXMLpar\FOtag{\FOatt-horizontal}{#1#2}%
     \fi
     \expandafter\docheckFOposition
   \fi#3}

\def\checkFOpadding{\def\FOatt{padding}\checkFOquadruple}
\def\checkFOmargin {\def\FOatt{margin}\checkFOquadruple}

\def\checkFOquadruple#1%
  {\edef\FOattribute{\XMLpar{#1}\FOatt\empty}%
   \ifx\FOattribute\empty\else
     \edef\FOtag{#1}%
     \scratchcounter\zerocount
     \expanded{\docheckFOquadruple\FOattribute\space\relax\space\relax}\od
     \ifcase\scratchcounter
       \let\FOattributeT\FOattribute
       \let\FOattributeR\FOattribute
       \let\FOattributeB\FOattribute
       \let\FOattributeL\FOattribute
     \or % (tblr)
       \let\FOattributeT\FOattribute
       \let\FOattributeR\FOattribute
       \let\FOattributeB\FOattribute
       \let\FOattributeL\FOattribute
     \or % (tb)(lr)
       \let\FOattributeB\FOattributeT
       \let\FOattributeL\FOattributeR
     \or % (t)(lr)(b)
       \let\FOattributeL\FOattributeR
     \or % (t)(r)(b)(l)
       % already ok
     \fi
     \letXMLpar\FOtag{\FOatt-top}\FOattributeT
     \letXMLpar\FOtag{\FOatt-right}\FOattributeR
     \letXMLpar\FOtag{\FOatt-bottom}\FOattributeB
     \letXMLpar\FOtag{\FOatt-left}\FOattributeL
   \fi}

\def\docheckFOquadruple#1#2 #3%
  {\ifx#1\relax
     \expandafter\noFOchecks
   \else
     \advance\scratchcounter\plusone
     \ifcase\scratchcounter
     \or
      \edef\FOattributeT{#1#2}%
     \or
      \edef\FOattributeR{#1#2}%
     \or
      \edef\FOattributeB{#1#2}%
     \or
      \edef\FOattributeL{#1#2}%
     \fi
     \expandafter\docheckFOquadruple
   \fi#3}

% \def\setFOimagename#1%
%   {\edef\FOimagename{#1}%
%    \aftersplitstring \FOimagename\at url('\to\xFOimagename
%    \ifx\xFOimagename\empty \else
%      \beforesplitstring\xFOimagename\at ')\to\FOimagename
%    \fi
%    \aftersplitstring \FOimagename\at url("\to\xFOimagename
%    \ifx\xFOimagename\empty \else
%      \beforesplitstring\xFOimagename\at ")\to\FOimagename
%    \fi}
%
% let's overkill:

\def\setFOimagename#1%
  {\analyzefunction{#1}%
   \doifelse\functionname{url}
     {\edef\FOimagename{\@EA\unstringed\functionA}}
     {\ifx\functionname\empty
        \def\FOimagename{#1}%
      \else
        \def\FOimagename{dummy}%
      \fi}}

% font

\mapXMLvalue {fo:weight}  {normal}  {}
\mapXMLvalue {fo:weight}  {bold}    {bold}
\mapXMLvalue {fo:weight}  {bolder}  {bold}
\mapXMLvalue {fo:weight}  {lighter} {normal}
\mapXMLvalue {fo:weight}  {100}     {normal}
\mapXMLvalue {fo:weight}  {200}     {normal}
\mapXMLvalue {fo:weight}  {300}     {normal}
\mapXMLvalue {fo:weight}  {400}     {normal}
\mapXMLvalue {fo:weight}  {500}     {normal}
\mapXMLvalue {fo:weight}  {600}     {normal}
\mapXMLvalue {fo:weight}  {700}     {normal}
\mapXMLvalue {fo:weight}  {800}     {normal}
\mapXMLvalue {fo:weight}  {900}     {normal}

\mapXMLvalue {fo:variant} {normal}     {}
\mapXMLvalue {fo:variant} {small-caps} {small-caps}

\mapXMLvalue {fo:style}   {normal}    {normal}
\mapXMLvalue {fo:style}   {italic}    {italic}
\mapXMLvalue {fo:style}   {oblique}   {oblique}
\mapXMLvalue {fo:style}   {backslant} {normal}

% we can get crap like: 10pt/1.5 bold "Times Roman" ; i'm really puzzled why an
% otherwise rather verbose coding occasionally packs attributes; a design flaw

\unprotect

\newtoks\FOfonttoks

\def\checkFOfontSS#1'{}
\def\checkFOfontDD#1"{}
\def\checkFOfontII#1 {}

\bgroup
\catcode`\'=\active
\catcode`\"=\active
\catcode`\/=\active
\gdef\setcheckFOfontX
  {\catcode`\'=\active
   \catcode`\"=\active
   \catcode`\/=\active
   \def'##1'{\global\FOfonttoks\expandafter{\the\FOfonttoks\def\FOfontfamily{##1}}}%
   \def"##1"{\global\FOfonttoks\expandafter{\the\FOfonttoks\def\FOfontfamily{##1}}}%
   \def/##1 {}}% todo linespacing
\gdef\setcheckFOfontXX
  {\catcode`\'=\active
   \catcode`\"=\active
   \catcode`\/=\active
   \def'##1'{}%
   \def"##1"{}%
   \def/##1 {}}% todo linespacing
\egroup

\globallet\xFOattribute\empty

\def\checkFOfont#1%
  {\FOfonttoks\emptytoks
   \bgroup
   \catcode`\\=\@@escape
   \catcode`\{=\@@begingroup
   \catcode`\}=\@@endgroup
% \catcode`\:=\@@letter
% \catcode`\-=\@@letter
   \setcheckFOfontX
   \xdef\xFOattribute{#1 }%
   \setbox\scratchbox\hbox{\scantokens\@EA{\xFOattribute}}%
   \setcheckFOfontXX
   \scantokens\@EA{\@EA\xdef\@EA\xFOattribute\@EA{\xFOattribute}}%
   \egroup
   \the\FOfonttoks
   \ifx\xFOattribute\empty\else
      \expanded{\docheckFOfont\xFOattribute\space\relax\space\relax}\od
   \fi
   \directsetup{fo:font:family:check}}

\def\docheckFOfont#1#2 #3%
  {\ifx#1\relax
     \expandafter\noFOchecks
   \else
     \directsetup{fo:fonts:reset}%
     \doifelsefontsynonym{#1#2}
       {\def\FOfontfamily{#1#2}}
       {\doifelsenothing{\XMLval{fo:weight}{#1#2}{}}
          {\doifelsenothing{\XMLval{fo:variant}{#1#2}{}}
             {\doifelsenothing{\XMLval{fo:style}{#1#2}{}}
                {\setpercentdimen\dFOfontsize{#1#2}}
                {\edef\FOfontstyle{\XMLval{fo:style}{#1#2}{}}}}
             {\edef\FOfontvariant{\XMLval{fo:variant}{#1#2}{}}}}
          {\edef\FOfontweight{\XMLval{fo:weight}{#1#2}{}}}}%
     \expandafter\docheckFOfont
   \fi#3}

\protect

\newtoks\FOreferences

\def\setFOreference#1%
  {\doifsomething{\XMLpar{#1}{id}{}}
     {\expanded{\appendtoks
        \noexpand\reference[\XMLpar{#1}{id}{}]{\XMLpar{fo:page-sequence}{format}{}}}%
      \to\FOreferences}}

\def\flushFOreferences
  {\the\FOreferences
   \global\FOreferences\emptytoks}

\def\doFOreference#1%
  {\doifsomething{\XMLpar{#1}{id}{}}
     {\expanded{\reference[\XMLpar{#1}{id}{}]{\XMLpar{fo:page-sequence}{format}{}}}}}

\appendtoks \flushFOreferences \to \everypar
\appendtoks \flushFOreferences \to \neverypar % check !

\protect

%D Graphics: static frames

\startMPinclusions
  input mp-fobg.mp ;
\stopMPinclusions

\def\unknownMPcolor{FoNoColor}

% todo: combine into one en alleen tweede run, immers toch geen invloed; is
% aangezien de referentiepunten vast liggen

\def\FoRegionWidth#1%
  {\XMLpav
     {fo:border-width}
     {fo:region-\MPvar{location}}
     {border-#1-width}
     {FoMedium}}

\def\FoRegionStyle#1%
  {\XMLpav
     {fo:border-style}
     {fo:region-\MPvar{location}}
     {border-#1-style}
     {FoNone}}

\def\FoRegionColor#1%
  {\MPcolor{\XMLpar
     {fo:region-\MPvar{location}}
     {border-#1-color}
     {black}}}

\def\FoRegionBackgroundColor
  {\MPcolor{\XMLpar
     {fo:region-\MPvar{location}}
     {background-color}
     {FoNoColor}}}

% todo: when connected and same color : one draw

\startuseMPgraphic{region-do}
  FoBackgroundColor     := \FoRegionBackgroundColor ;
  FoLineColor[FoTop]    := \FoRegionColor{top} ;
  FoLineColor[FoBottom] := \FoRegionColor{bottom} ;
  FoLineColor[FoLeft]   := \FoRegionColor{left} ;
  FoLineColor[FoRight]  := \FoRegionColor{right} ;
  FoLineStyle[FoTop]    := \FoRegionStyle{top} ;
  FoLineStyle[FoBottom] := \FoRegionStyle{bottom} ;
  FoLineStyle[FoLeft]   := \FoRegionStyle{left} ;
  FoLineStyle[FoRight]  := \FoRegionStyle{right} ;
  FoLineWidth[FoTop]    := \FoRegionWidth{top} ;
  FoLineWidth[FoBottom] := \FoRegionWidth{bottom} ;
  FoLineWidth[FoLeft]   := \FoRegionWidth{left} ;
  FoLineWidth[FoRight]  := \FoRegionWidth{right} ;
  if FoBackgroundColor <> FoNoColor :
    fill OverlayBox
      withcolor FoBackgroundColor ;
  fi ;
  path OverlayFrameBox ;
  interim linecap := butt ;
  OverlayFrameBox := OverlayBox
      topenlarged    -.5FoLineWidth[FoTop]
      bottomenlarged -.5FoLineWidth[FoBottom]
      leftenlarged   -.5FoLineWidth[FoLeft]
      rightenlarged  -.5FoLineWidth[FoRight] ;
  DrawFoFrame(FoTop,    topboundary    OverlayFrameBox) ;
  DrawFoFrame(FoBottom, bottomboundary OverlayFrameBox) ;
  DrawFoFrame(FoLeft,   leftboundary   OverlayFrameBox) ;
  DrawFoFrame(FoRight,  rightboundary  OverlayFrameBox) ;
\stopuseMPgraphic

\startuseMPgraphic{region-body}   \includeMPgraphic{region-do} \stopuseMPgraphic
\startuseMPgraphic{region-before} \includeMPgraphic{region-do} \stopuseMPgraphic
\startuseMPgraphic{region-after}  \includeMPgraphic{region-do} \stopuseMPgraphic
\startuseMPgraphic{region-start}  \includeMPgraphic{region-do} \stopuseMPgraphic
\startuseMPgraphic{region-end}    \includeMPgraphic{region-do} \stopuseMPgraphic

\startnotmode[fo-no-bg]

\defineoverlay[region-body-graphic]  [\useMPgraphic{region-body}{location=body}]
\defineoverlay[region-before-graphic][\useMPgraphic{region-before}{location=before}]
\defineoverlay[region-after-graphic] [\useMPgraphic{region-after}{location=after}]
\defineoverlay[region-start-graphic] [\useMPgraphic{region-start}{location=start}]
\defineoverlay[region-end-graphic]   [\useMPgraphic{region-end}{location=end}]

\stopnotmode

% more efficient: todo: relocate and move to page background

% \def\FoRegionWidth#1#2%
%   {\XMLpav
%      {fo:border-width}
%      {fo:region-#2}
%      {border-#1-width}
%      {FoMedium}}

% \def\FoRegionStyle#1#2%
%   {\XMLpav
%      {fo:border-style}
%      {fo:region-#2}
%      {border-#1-style}
%      {FoNone}}

% \def\FoRegionColor#1#2%
%   {\MPcolor{\XMLpar
%      {fo:region-#2}
%      {border-#1-color}
%      {black}}}

% \def\FoRegionBackgroundColor#1%
%   {\MPcolor{\XMLpar
%      {fo:region-#1}
%      {background-color}
%      {FoNoColor}}}

% \def\combinedFOgraphic#1%
%  {FoBackgroundColor     := \FoRegionBackgroundColor{#1} ;
%   FoLineColor[FoTop]    := \FoRegionColor{top}{#1} ;
%   FoLineColor[FoBottom] := \FoRegionColor{bottom}{#1} ;
%   FoLineColor[FoLeft]   := \FoRegionColor{left}{#1} ;
%   FoLineColor[FoRight]  := \FoRegionColor{right}{#1} ;
%   FoLineStyle[FoTop]    := \FoRegionStyle{top}{#1} ;
%   FoLineStyle[FoBottom] := \FoRegionStyle{bottom}{#1} ;
%   FoLineStyle[FoLeft]   := \FoRegionStyle{left}{#1} ;
%   FoLineStyle[FoRight]  := \FoRegionStyle{right}{#1} ;
%   FoLineWidth[FoTop]    := \FoRegionWidth{top}{#1} ;
%   FoLineWidth[FoBottom] := \FoRegionWidth{bottom}{#1} ;
%   FoLineWidth[FoLeft]   := \FoRegionWidth{left}{#1} ;
%   FoLineWidth[FoRight]  := \FoRegionWidth{right}{#1} ;
%   if FoBackgroundColor <> FoNoColor :
%     fill OverlayBox
%       withcolor FoBackgroundColor ;
%   fi ;
%   path OverlayFrameBox ;
%   interim linecap := butt ;
%   OverlayFrameBox := OverlayBox
%       topenlarged    -.5FoLineWidth[FoTop]
%       bottomenlarged -.5FoLineWidth[FoBottom]
%       leftenlarged   -.5FoLineWidth[FoLeft]
%       rightenlarged  -.5FoLineWidth[FoRight] ;
%   DrawFoFrame(FoTop,    topboundary    OverlayFrameBox) ;
%   DrawFoFrame(FoBottom, bottomboundary OverlayFrameBox) ;
%   DrawFoFrame(FoLeft,   leftboundary   OverlayFrameBox) ;
%   DrawFoFrame(FoRight,  rightboundary  OverlayFrameBox) ;}

% \startuseMPgraphic{region-body}
%   \combinedFOgraphic{before}
%   \combinedFOgraphic{body}
%   \combinedFOgraphic{after}
%   \combinedFOgraphic{start}
%   \combinedFOgraphic{end}
% \stopuseMPgraphic

% \defineoverlay[region-body-graphic]  [\useMPgraphic{region-body}{location=body}]
% \defineoverlay[region-before-graphic][]
% \defineoverlay[region-after-graphic] []
% \defineoverlay[region-start-graphic] []
% \defineoverlay[region-end-graphic]   []

\stopXMLcompiling

\protect  \endinput

% we can follow two approaches: set the attributes global, using
%
% \defineXML...[tag][prefix][empty]
%
% in that case we trust the fo-file to be correct, i.e. the xslt style
% sheets should not put attributes in the wrong places; however, we need
% to do that with care, since for instance the attributes of some objects
% (regions) are used mixed
%
% \defineXMLprocess [fo:root] [XMLFO] [test=unset]
%
% \defineXMLenvironment [fo:block-container] [XMLFO]
%   {\begingroup}
%   {\endgroup}
%
% \defineXMLenvironment [fo:block] [XMLFO]
%   {\begingroup\getXMLparameters[XMLFO]\begingroup}
%   {\endgroup\XMLFOtest\endgraf\endgroup}
%
% \startXMLdata
% <fo:root>
%     <fo:block-container test='first'><fo:block test='second'>second:</fo:block></fo:block-container>
%     <fo:block>unset:</fo:block>
%     <fo:block test='outer'><fo:block test='nested'>nested:</fo:block>outer:</fo:block>
%     <fo:block test='last'>last:</fo:block>
% </fo:root>
% \stopXMLdata
%
% the other approach is to set the attributes explicitly for each
% element, which is slower but more robust
%
% A mix is:
%
% \defineXMLenvironment
%   [fo:root]
%   [test=unset]
%   {\beginXMLelement}
%   {\endXMLelement}
%
% \defineXMLenvironment
%   [fo:block-container]
%   [test=\XMLpar\XMLpureparent{test}{}]
%   {\beginXMLelement}
%   {\endXMLelement}
%
% \defineXMLenvironment
%   [fo:block]
%   [test=\XMLpar\XMLpureparent{test}{}]
%   {\beginXMLelement
%    \begingroup}
%   {\endgroup
%    \XMLpar{fo:block}{test}{}
%    \endXMLelement}
%
% \startXMLdata
% <fo:root>
%     <fo:block-container test='first'><fo:block test='second'>second:</fo:block></fo:block-container>
%     <fo:block>unset:</fo:block>
%     <fo:block test='outer'><fo:block test='nested'>nested:</fo:block>outer:</fo:block>
%     <fo:block test='last'>last:</fo:block>
% </fo:root>
% \stopXMLdata
