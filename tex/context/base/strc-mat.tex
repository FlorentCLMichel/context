%D \module
%D   [       file=strc-mat,
%D        version=2008.10.20,
%D          title=\CONTEXT\ Structure Macros,
%D       subtitle=Math Numbering,
%D         author=Hans Hagen,
%D           date=\currentdate,
%D      copyright=PRAGMA-ADE / Hans Hagen]
%C
%C This module is part of the \CONTEXT\ macro||package and is
%C therefore copyrighted by \PRAGMA. See mreadme.pdf for
%C details.

\writestatus{loading}{ConTeXt Structure Macros / Math Numbering}

\registerctxluafile{strc-mat}{1.001}

\unprotect

\definestructureconversionset[\v!formula][number,characters]

\setupformulas
  [%\c!way=\@@nrway,
   %\c!blockway=,
   %\c!sectionnumber=\@@nrsectionnumber,
   %\c!conversion=\v!numbers,
   \c!location=\v!right,
   \c!left=(,
   \c!right=),
   %\c!numberstyle=,
   %\c!numbercolor=,
   %\c!numbercommand=,
   \c!spacebefore=\v!big,
   \c!spaceafter=\formulaparameter\c!spacebefore,
   \c!leftmargin=\!!zeropoint,
   \c!rightmargin=\!!zeropoint,
   %\c!margin=,
   \c!indentnext=\v!no,
   \c!alternative=\s!default,
   %\c!align=,
   \c!strut=\v!no,
   %\c!separator=\@@koseparator,
   %\c!grid=,
   \c!distance=1em]

\definestructurecounter
  [\v!formula]

\setupstructurecounter
  [\v!formula]
  [\c!numberconversionset=\v!formula]

\def\storecurrentformulanumber#1#2#3% ref, todo:str, \sync % todo: title etc (like float)
  {\dostructurecountercomponent
     {formula}%
     \getfloatparameters
     \formulaparameter
     \detokenizedformulaparameter
     \relax
     \relax
     \relax
     [\c!name=\currentformula,\s!counter=\currentformula,%
      \s!hascaption=\v!yes,\s!hasnumber=\v!yes,\s!hastitle=\v!yes,%
      \c!reference=#1,\c!title=,\c!bookmark=]%
     [#2]%
   \globallet\currentformulanumber\laststructurecounternumber
   \globallet#3\laststructurecountersynchronize}

\def\thecurrentformulanumber
  {%\ifnoformulacaption \else \ifnoformulanumber \else
     \labeltexts\currentformula{\convertedstructurecounter[formula]}% ! ! todo: use a lua call instead
   }%\fi \fi}

\def\placecurrentformulanumber
  {\currentformulassynchronize
   \currentformulasynchronize
   \currentsubformulasynchronize
   \thecurrentformulanumber} %\convertedstructurecounter[\v!formula]\relax}

\def\doformulareference#1#2%
  {\doifsomething{#1}{\doifnotinset{#1}{+,-}{\rawreference\s!for{#1}{#2}}}}

\def\doformulanumber
  {\dotripleempty\dodoformulanumber}

\def\dodoformulanumber[#1][#2][#3]%
  {\doquadruplegroupempty\dododoformulanumber{#1}{#2}{#3}}

\let\subformulasreference\empty % temp hack

\let\currentformulasynchronize \relax
\let\currentformulassynchronize\relax

\def\dododoformulanumber#1#2#3#4% (#1,#2)=outer(ref,sub) (#3,#4)=inner(ref,sub)
  {\hbox\bgroup
   \ifconditional\handleformulanumber
     \ifconditional\incrementformulanumber
       \ifconditional\insidesubformulas
         \incrementsubstructurecounter[\v!formula][2]%
       \else
         \incrementstructurecounter[\v!formula]%
       \fi
     \fi
     %
     % main counter
     \setbox0\hbox{\ignorespaces#2\unskip}%
     \ifdim\wd0>\zeropoint
       \setsubstructurecounterown[\v!formula][2]{#2}% \detokenize?
     \fi
     \edef\currentformulareference{#1}%
     \ifx\currentformulareference\empty
       \glet\currentformulasynchronize\relax
     \else
       \storecurrentformulanumber\currentformulareference\empty\currentformulasynchronize
     \fi
     % subcounter
     \setbox0\hbox{\ignorespaces#4\unskip}%
     \ifdim\wd0>\zeropoint
       \setsubstructurecounterown[\v!formula][2]{#4}% \detokenize?
     \fi
     \edef\currentsubformulareference{#3}%
     \ifx\currentsubformulareference\empty
       \glet\currentsubformulasynchronize\relax
     \else
       \storecurrentformulanumber\currentsubformulareference\empty\currentsubformulasynchronize
     \fi
     %
     \rm % nodig ?
     \doif{\formulaparameter\c!location}\v!right{\hskip\formulaparameter\c!distance}%
     \formulaparameter\c!numbercommand
       {\dosetformulaattributes\c!numberstyle\c!numbercolor
        \strut
        \formulaparameter\c!left
        \labeltexts\v!formula{\ignorespaces\placecurrentformulanumber\unskip}%
        \formulaparameter\c!right}%
      \doif{\formulaparameter\c!location}\v!left{\hskip\formulaparameter\c!distance}%
   \fi
   \egroup}

\let\donestedformulanumber\gobbletwoarguments

\definelist[\v!formula]

\global\let\doflushformulalistentry\gobbleoneargument

\def\setformulalistentry#1%
  {\gdef\doflushformulalistentry##1%
     {\normalexpanded{\noexpand\writetolist[\v!formula]{##1}}{#1}%
      \global\let\doflushformulalistentry\gobbleoneargument}}

\newconditional\handleformulanumber
\newconditional\incrementformulanumber
\newconditional\insidesubformulas

\newif\ifinformula

\let\doplaceformulanumber\empty

%D We need a hook into the plain math alignment macros
%D
%D \starttyping
%D \displaylines
%D \eqalignno
%D \eqalignno
%D \stoptyping
%D
%D Otherwise we get a missing \type {$$} error reported.

\def\resetdisplaymatheq
  {\let\normalleqno\relax \let\leqno\relax
   \let\normalreqno\relax \let\eqno \relax
   \let\doplaceformulanumber\empty}

%D

\def\defineformula
  {\dodoubleempty\dodefineformula}

\def\dodefineformula[#1][#2]%
  {\doifsomething{#1}
     {\getparameters[\??fm#1][\s!parent=\??fm,#2]%
      \definelist[#1]%
      \setvalue{\e!start#1\v!formula}{\dostartformula{#1}}%
      \setvalue{\e!stop #1\v!formula}{\dostopformula}}}

\def\defineformulaalternative
  {\dotripleargument\dodefineformulaalternative}

\def\dodefineformulaalternative[#1][#2][#3]%
  {\setvalue{\e!start#1\v!formula}{#2}%
   \setvalue{\e!stop #1\v!formula}{#3}}

% sp = single line paragraph  sd = single line display
% mp = multi  line paragraph  md = multy  line display

\defineformulaalternative[\s!default][\startdisplaymath][\stopdisplaymath]
\defineformulaalternative[\s!single] [\startdisplaymath][\stopdisplaymath]
\defineformulaalternative[\s!multi]  [\startdisplaymath][\stopdisplaymath]

\defineformula
  [sp]
  [\c!spacebefore=\v!none,
   \c!spaceafter=\v!none,
   \c!indentnext=\v!no,
   \c!alternative=\s!single]

\defineformula
  [sd]
  [\c!spacebefore=\v!none,
   \c!spaceafter=\v!none,
   \c!indentnext=\v!yes,
   \c!alternative=\s!single]

\defineformula
  [mp]
  [\c!indentnext=\v!no,
   \c!alternative=\s!multi]

\defineformula
  [md]
  [\c!indentnext=\v!yes,
   \c!alternative=\s!multi]

%D \macros
%D   {setupsubformulas, startsubformulas}

\def\subformulaparameter#1{\ifcname\??fn#1\endcsname\cname\??fn#1\endcsname\fi}

\def\setupsubformulas
  {\dodoubleargument\getparameters[\??fn]}

\setupsubformulas
  [\c!indentnext=\formulaparameter\c!indentnext]

% \setupsubformulas[conversion=romannumerals]
%
% \placeformula
% \startsubformulas[Maxwell]
%     \startformulas
%      \startformula \startalign
%        \NC \nabla\cdot\bf E  \NC = \frac{\rho}{\varepsilon_0} \NR[Maxwell 1]
%        \NC \nabla\times\bf E \NC = - \frac{\partial\bf B}{\partial t} \NR[Maxwell II]
%        \stopalign \stopformula
%        \startformula \startalign
%          \NC \nabla\cdot \bf B \NC = 0 \NR[Maxwell III]
%          \NC \nabla\times\bf B \NC = \mu_0{\bf j}+\varepsilon_0\mu_0\frac{\partial\bf E}{\partial t} \NR[Maxwell IV]
%        \stopalign \stopformula
%    \stopformulas
% \stopsubformulas
%
% Maxwell : \in [Maxwell] and II : \in [Maxwell II]

%D Tricky stuff:

\newdimen\lastlinewidth

% test \par \dorecurse{10}{test } \moveformula \startformula test \stopformula test \endgraf
% test \par \dorecurse{10}{test } \startformula test \stopformula test \endgraf
% \dorecurse{30}{\bpar \dorecurse\recurselevel{test } \epar \startformula formula \stopformula}

\def\setlastlinewidth
  {\resetlastlinewidth
   \ifoptimizedisplayspacing\ifmmode\else\ifhmode
     \bgroup
     \forgetdisplayskips
     \displaywidowpenalty\widowpenalty % brrr, else widowpenalty does not work
     \everymath   \emptytoks
     \everydisplay\emptytoks
     $$\strut\global\lastlinewidth\predisplaysize$$
     \vskip-\lineheight
     \vskip\zeropoint
     \egroup
   \fi\fi\fi}

\def\resetlastlinewidth
  {\global\lastlinewidth\zeropoint\relax}

\abovedisplayskip      \zeropoint
\abovedisplayshortskip \zeropoint % evt. 0pt minus 3pt
\belowdisplayskip      \zeropoint
\belowdisplayshortskip \zeropoint % evt. 0pt minus 3pt

\predisplaypenalty     \zerocount
\postdisplaypenalty    \zerocount % -5000 gaat mis, zie penalty bij \paragraaf

% we don't use the skip's

\def\displayskipsize#1#2% obsolete
  {\ifdim\ctxparskip>\zeropoint
     #1\ctxparskip\!!plus#2\ctxparskip\!!minus#2\ctxparskip\relax
   \else
     #1\lineheight\!!plus#2\lineheight\!!minus#2\lineheight\relax
   \fi}

\def\forgetdisplayskips % to do
  {\abovedisplayskip     \zeropoint
   \belowdisplayskip     \zeropoint
   \abovedisplayshortskip\zeropoint
   \belowdisplayshortskip\zeropoint}

\setvalue{\e!start\v!formula}{\dostartformula{}}
\setvalue{\e!stop \v!formula}{\dostopformula}

\def\predisplaysizethreshhold{2em} % was 3em

\def\leftdisplayskip    {\leftskip}
\def\rightdisplayskip   {\rightskip}
\def\leftdisplaymargin  {\formulaparameter\c!leftmargin}
\def\rightdisplaymargin {\formulaparameter\c!rightmargin}
\def\displaygridsnapping{\formulaparameter\c!grid}

\def\beforedisplayspace
  {\doifnot{\formulaparameter\c!spacebefore}\v!none{\blank[\formulaparameter\c!spacebefore]}}

\def\afterdisplayspace
  {\doifnot{\formulaparameter\c!spaceafter }\v!none{\blank[\formulaparameter\c!spaceafter ]}}

\def\setpredisplaysize#1%
  {\predisplaysize#1\relax
   \ifdim\predisplaysize<\maxdimen
     \ifdim\predisplaysize>\zeropoint
       \advance\predisplaysize \predisplaysizethreshhold
     \fi
     \advance\predisplaysize \displayindent % needed ?
     \ifdim\predisplaysize>\hsize
       \predisplaysize\hsize
     \fi
   \else
     \predisplaysize\zeropoint
   \fi}

\def\setdisplaydimensions
  {\displayindent\leftdisplayskip
   \advance\displayindent\leftdisplaymargin
   \displaywidth\hsize
% \setlocalhsize
% \displaywidth\localhsize
   \ifdim\hangindent>\zeropoint
     \advance\displayindent\hangindent
   \else
     \advance\displaywidth\hangindent
   \fi
   \advance\displaywidth\dimexpr-\displayindent-\rightdisplayskip-\rightdisplaymargin\relax
   \hsize\displaywidth} % new, else overfull in itemize

\newif\ifoptimizedisplayspacing

\def\dostartformula#1%
  {\dodoubleempty\dodostartformula[#1]}

\newskip\formulaparskip
\newskip\formulastrutht
\newskip\formulastrutdp

% hm, invoke otr in hmode in order to move skips to mvl, could be an option

%D \startbuffer
%D \startformula[9pt] x = 1 \stopformula
%D \startformula[7pt] x = 1 \stopformula
%D \stopbuffer
%D
%D \typebuffer \getbuffer

\def\dodostartformula[#1][#2]% setting leftskip adaption is slow !
  {% todo: test first
   %
   % \ifdim\lastskip>\zeropoint
   %   \resetlastlinewidth % else problems with in between stuff without \epar
   % \fi
   \bgroup % HERE
   \def\currentformula{#1}%
   \the\everybeforedisplayformula
   \formulaparskip\parskip
   \formulastrutdp\strutdepth
   \formulastrutht\strutheight
   \doifsomething{#2}{\switchtoformulabodyfont[#2]}%
   \parskip\formulaparskip
   % may look better in itemizations
   \doif{\formulaparameter\c!option}\v!middle
     {\def\leftdisplayskip{\zeropoint}%
      \def\rightdisplayskip{\zeropoint}}%
   % this was an experiment
   \doifsomething{\formulaparameter\c!margin}% so we test first
     {\dosetleftskipadaption{\formulaparameter\c!margin}%
      \edef\leftdisplaymargin{\the\leftskipadaption}}% overloaded
   \long\def\dostartformula##1{\bgroup\let\dostopformula\egroup}%
   \freezedimenmacro\leftdisplayskip
   \freezedimenmacro\rightdisplayskip
   \freezedimenmacro\leftdisplaymargin
   \freezedimenmacro\rightdisplaymargin
   \freezedimenmacro\predisplaysizethreshhold
   \forgetdisplayskips
   \ifoptimizedisplayspacing
     \ifdim\lastlinewidth>\zeropoint
       \abovedisplayshortskip-\strutht\relax
     \fi
   \else
     \resetlastlinewidth
   \fi
   \getvalue{\e!start\formulaparameter\c!alternative\v!formula}}

\def\switchtoformulabodyfont{\switchtobodyfont}

\setvalue{\v!formula}{\dosingleempty\doformula}

\def\doformula[#1]#2%
  {\begingroup
   \doifsomething{#1}{\switchtoformulabodyfont[#1]}%
   % not : \def\doformula[##1]##2{\mathematics{##2}}%
   \mathematics{#2}%
   \endgroup}

\def\dostopformula
  {\doplaceformulanumber
   \getvalue{\e!stop\formulaparameter\c!alternative\v!formula}%
   \resetlastlinewidth
   \nonoindentation
   \checknextindentation[\formulaparameter\c!indentnext]%
   \egroup
   \hangafter\minusone   % added for side floats
   \hangindent\zeropoint % added for side floats
   \setfalse\handleformulanumber
   \dorechecknextindentation} % here ?

\def\startdisplaymath
  {\ifgridsnapping
     \beforedisplayspace
     \snapmathtogrid\vbox
     \bgroup
     \informulatrue
    %\forgetall % breaks side floats
   \else
     \bgroup
     \parskip\formulaparskip % ! !
     \informulatrue
    %\forgetall % otherwise backgrounds fail
     \ifdim\lastskip<\zeropoint\else
       \par
       \ifvmode \ifdim\parskip>\zeropoint\relax
         \whitespace \vskip-\parskip % kind of forces and cancels again
       \fi \fi
     \fi
     \doif\displaygridcorrection{-\v!top}{\kern-\strutht}% new, currently only option/default
     \beforedisplayspace
     \par
     \ifvmode
        \prevdepth-\maxdimen % texbook pagina 79-80
        % otherwise problems at the top of a page
        \verticalstrut
        \vskip-\struttotal
        \vskip-\baselineskip
     \fi
   \fi
   $$\setdisplaydimensions
   \setpredisplaysize\lastlinewidth
   \startinnermath}

\def\stopdisplaymath
  {\stopinnermath
   $$%
   \ifgridsnapping
     \egroup
     \afterdisplayspace
   \else
     \par\ifvmode\ifdim\parskip>\zeropoint\whitespace\vskip-\parskip\fi\fi
     \afterdisplayspace
     \egroup
   \fi
   \globallet\displaylinecorrection\empty
   \gdef\displaygridcorrection{\displaygridsnapping}}

\newif\ifclipdisplaymath \clipdisplaymathtrue
\def\displaymathclipfactor{1.1}

\def\snapmathtogrid % to do \dp
  {\dowithnextbox
     {\bgroup
      \donefalse
      \ifclipdisplaymath
        \ifdim\nextboxht<\displaymathclipfactor\lineheight
          \donetrue
        \fi
      \fi
      \ifdone
        \nextboxht\lineheight
      \else
        \getnoflines\nextboxht
        \setbox\nextbox\vbox to \noflines\lineheight{\vfill\flushnextbox\vfill}%
        \setbox\nextbox\hbox{\lower\strutdepth\flushnextbox}%
      \fi
      \snaptogrid[\displaygridcorrection]\hbox{\flushnextbox}%
      \egroup}}

\def\displaygridcorrection{\displaygridsnapping}
\let\displaygridcorrection\empty

\def\moveformula
  {\dosingleempty\domoveformula}

\def\domoveformula[#1]% brr gaat mogelijk fout
  {\iffirstargument
     \xdef\displaygridcorrection{#1}%
   \else
     \gdef\displaygridcorrection{-\v!top}% handy with short preline
   \fi
   \globallet\displaylinecorrection\displaygridcorrection}

\let\startinnermath\empty
\let\stopinnermath \empty

% \defineformulaalternative[multi][\begindmath][\enddmath]
%
% \fakewords{20}{40}\epar
% \placeformula {a} $$              \fakespacingformula $$
% \fakewords{20}{40}\epar
% \placeformula {b} \startformule   \fakespacingformula \stopformule
% \placeformula {b} \startformule   \fakespacingformula \stopformule
% \fakewords{20}{40}\epar
% \placeformula {c} \startmdformule \fakespacingformula \stopmdformule
% \placeformula {c} \startmdformule \fakespacingformula \stopmdformule
% \fakewords{20}{40}\epar
% \placeformula {d} \startmpformule \fakespacingformula \stopmpformule
% \placeformula {d} \startmpformule \fakespacingformula \stopmpformule
% \fakewords{20}{40}\epar
% \placeformula {e} \startsdformule \fakespacingformula \stopsdformule
% \placeformula {e} \startsdformule \fakespacingformula \stopsdformule
% \fakewords{20}{40}\epar
% \placeformula {f} \startspformule \fakespacingformula \stopspformule
% \placeformula {f} \startspformule \fakespacingformula \stopspformule
% \fakewords{20}{40}

\def\placeformula
  {\settrue\incrementformulanumber
   \dodoubleempty\doplaceformula}

\def\placesubformula
  {\setfalse\incrementformulanumber
   \dodoubleempty\doplaceformula}

\def\startsubformulas
  {\dosingleempty\dostartsubformulas}

\def\dostartsubformulas[#1]%
  {\ifconditional\incrementformulanumber
     \incrementstructurecounter[\v!formula]%
     \edef\subformulasreference{#1}% messy
     \ifx\subformulasreference\empty
       \glet\currentformulassynchronize\relax
     \else
       \storecurrentformulanumber\subformulasreference\empty\currentformulassynchronize
     \fi
   \fi
   \settrue\insidesubformulas}

\def\stopsubformulas
  {\setfalse\insidesubformulas
   \resetlastlinewidth
   \nonoindentation
   \checknextindentation[\formulaparameter\c!indentnext]%
   \dorechecknextindentation} % here ?

%D Named subformulas

\def\startnamedsubformulas
  {\dosingleempty\dostartnamedsubformulas}

\def\dostartnamedsubformulas[#1]#2%
  {\setformulalistentry{#2}%
   \startsubformulas[#1]}

\def\stopnamedsubformulas
  {\stopsubformulas}

%D Experimental goodie:
%D
%D \startbuffer
%D \placelist[formula][criterium=text] \blank[2*big]
%D \placenamedformula[one]{first}  \startformula a = 1 \stopformula \endgraf
%D \placeformula                   \startformula a = 2 \stopformula \endgraf
%D \placenamedformula     {second} \startformula a = 3 \stopformula \endgraf
%D \stopbuffer
%D
%D \typebuffer \getbuffer

\def\placenamedformula
  {\dosingleempty\doplacenamedformula}

\def\doplacenamedformula[#1]#2%
  {\iffirstargument
     \def\next{\placeformula[#1]}%
   \else
     \let\next\placeformula
   \fi
   \setformulalistentry{#2}%
   \next}

%D The implementation of placement is a bit ugly:

\def\doplaceformula[#1][#2]% #2 = dummy, gobbles spaces
  {\def\redoplaceformula
     {\bgroup
      \ifx\next\bgroup
        \egroup \@EA\moreplaceformula % [ref]{}
      \else
        \let\nextnext$% no def
        \ifx\next\nextnext
          \egroup \@EAEAEA\dispplaceformula % [ref]$$
        \else
          \egroup \@EAEAEA\dodoplaceformula % [ref]\start
        \fi
      \fi[#1]{}}%
   \futurelet\next\redoplaceformula}

\long\def\moreplaceformula[#1]#2#3#4% #2 dummy #4 gobbles spaces
  {\def\redoplaceformula
     {\bgroup
      \let\nextnext$% no def
      \ifx\next\nextnext
        \egroup \@EA\dispplaceformula % [ref]$$
      \else
        \egroup \@EA\dodoplaceformula % [ref]\start
      \fi
      [#1]{#3}}%
   \futurelet\next\redoplaceformula#4}

\let\startplaceformula\placeformula
\let\stopplaceformula \relax

\def\startformulas#1\stopformulas % new / to be internationalized
  {\bgroup
   \let\currentformula\empty
   \forgetdisplayskips
   \startdisplaymath
   \setlocalhsize
   \long\def\startformula##1\stopformula
     {\advance\scratchcounter\plusone}%
   \scratchcounter\zerocount
   #1% preroll
   \ifcase\scratchcounter\else
     \divide \hsize \scratchcounter
   \fi
   \hbox to \localhsize \bgroup
     \hss
     \def\normalstartformula{\vskip-\strutdepth$$}% i hate this
     \def\normalstopformula {$$}%
     \def\startformula      {$\vcenter\bgroup\normalstartformula}%
     \def\stopformula       {\normalstopformula\egroup$\hss}%
     #1%
     \egroup
   \stopdisplaymath
   \egroup
   \hangafter\minusone    % added for side floats
   \hangindent\zeropoint} % added for side floats

\def\dispplaceformula[#1]#2$$#3$$%
  {\dodoplaceformula[#1]{#2}\dostartformula{}#3\dostopformula}

\def\dodoplaceformula[#1]#2% messy, needs a clean up
  {\doifelse{#1}{-}
     {\setfalse\handleformulanumber}
     {\doifelse{#2}{-}
        {\setfalse\handleformulanumber}
        {\settrue\handleformulanumber}}%
   \ifconditional\handleformulanumber
     \def\formulanumber
       {%\global\let\subformulanumber\doformulanumber % no, bug
        \doformulanumber[#1][#2]}%
     \def\donestedformulanumber##1##2%
       {\doifsomething{##1}
          {\doifelse{##1}{+}{\doformulanumber[#1]}{\doformulanumber[##1]}[##2][]{}}}%
     \def\subformulanumber
       {\setfalse\incrementformulanumber
        \formulanumber}%
     \gdef\doplaceformulanumber
       {\global\let\doplaceformulanumber\empty
        \doifelse{\formulaparameter\c!location}\v!left
          {\normalleqno{\doformulanumber[#1][#2][]{}}}
          {\normalreqno{\doformulanumber[#1][#2][]{}}}}%
   \else
     \def\formulanumber{\doformulanumber[#1][#2]}%
     \let\donestedformulanumber\gobbletwoarguments
     \let\subformulanumber\doformulanumber % was \global
     \global\let\doplaceformulanumber\empty
   \fi}

%D Here we implement a basic math alignment mechanism. Numbers
%D are also handled. The macros \type {\startinnermath} and
%D \type {\stopinnermath} can be overloaded in specialized
%D modules.

\def\startinnermath
  {\getvalue{\e!start\??fm\formulaparameter\c!align}}

\def\stopinnermath
  {\getvalue{\e!stop \??fm\formulaparameter\c!align}}

\def\mathinnerstrut
  {\doif{\formulaparameter\c!strut}\v!yes\strut}

\long\def\defineinnermathhandler#1#2#3%
  {\setvalue{\e!start\??fm#1}{#2}%
   \setvalue{\e!stop \??fm#1}{#3}}

\newif\iftracemath

\def\mathhbox
  {\iftracemath\ruledhbox\else\hbox\fi}

\chardef\mathraggedstatus=0 % normal left center  right
\chardef\mathnumberstatus=0 % nothing normal shift_right
\let\mathnumbercorrection\!!zeropoint

\def\startmathbox#1%
  {\hsize\displaywidth
   \global\chardef\mathnumberstatus\plusone
   \chardef\mathraggedstatus#1\relax
   \let\mathnumbercorrection\!!zeropoint
   \global\let\@eqno \empty \def\eqno {\gdef\@eqno }%
   \global\let\@leqno\empty \def\leqno{\gdef\@leqno}%
   % added
   \let\normalreqno\eqno
   \let\normalleqno\leqno
   % added
   \doplaceformulanumber
   \setbox\scratchbox\mathhbox to \displaywidth\bgroup
     \mathinnerstrut
     $%
     \displaystyle
     \ifcase\mathraggedstatus\or\hfill\or\hfill\fi}

\def\llappedmathno
  {\ifcase\mathraggedstatus\or
     \@eqno
   \or
     \llap{\@eqno}%
   \or
     \llap{\@eqno}%
   \fi}

\def\rlappedmathno
  {\ifcase\mathraggedstatus\or
     \rlap{\@leqno}%
   \or
     \rlap{\@leqno}%
   \or
     \@leqno
   \fi}

\def\stopmathbox
  {$%
   \ifcase\mathraggedstatus\or\or\hfill\or\hfill\fi
   \egroup
   \setbox0\hbox{\unhcopy\scratchbox}%
   \scratchdimen\wd0
   \ifdim\scratchdimen>\displaywidth
     \donetrue
   \else
     \donefalse
   \fi
   \hbox to \displaywidth\bgroup
     \ifcase\mathnumberstatus
       \box\scratchbox
     \or
       \ifx\@leqno\empty
         \ifx\@eqno\empty
           \box\scratchbox
         \else
           \ifdone
             \vbox{\box\scratchbox\hbox to \displaywidth{\hss\llappedmathno}}%
           \else
             \hss\box\scratchbox\llappedmathno % hss makes room for number
           \fi
         \fi
       \else
         \ifdone
           \vbox{\hbox to \displaywidth{\rlappedmathno\hss}\box\scratchbox}%
         \else
           \rlappedmathno\box\scratchbox\hss % hss makes room for number
         \fi
       \fi
     \or
        \hskip\mathnumbercorrection
        \box\scratchbox
        \hss
     \else
        \box\scratchbox
     \fi
   \egroup}

\defineinnermathhandler\v!left      {\startmathbox\plusone  }{\stopmathbox}
\defineinnermathhandler\v!middle    {\startmathbox\plustwo  }{\stopmathbox}
\defineinnermathhandler\v!right     {\startmathbox\plusthree}{\stopmathbox}
\defineinnermathhandler\v!flushleft {\startmathbox\plusthree}{\stopmathbox}
\defineinnermathhandler\v!center    {\startmathbox\plustwo  }{\stopmathbox}
\defineinnermathhandler\v!flushright{\startmathbox\plusone  }{\stopmathbox}

%D [The examples below are in english and don't process in the
%D  documentation style, which will be english some day.]
%D
%D Normally a formula is centered, but in case you want to
%D align it left or right, you can set up formulas to behave
%D that way. Normally a formula will adapt is left indentation
%D to the environment:
%D
%D \startbuffer
%D \fakewords{20}{40}\epar
%D \startitemize
%D \item \fakewords{20}{40}\epar
%D       \placeformula \startformula \fakeformula \stopformula
%D \item \fakewords{20}{40}\epar
%D \stopitemize
%D \fakewords{20}{40}\epar
%D \stopbuffer
%D
%D % \getbuffer
%D
%D In the next examples we explicitly align formulas to the
%D left (\type {\raggedleft}), center and right (\type
%D {\raggedright}):
%D
%D \startbuffer
%D \setupformulas[align=left]
%D \startformula\fakeformula\stopformula
%D \setupformulas[align=middle]
%D \startformula\fakeformula\stopformula
%D \setupformulas[align=right]
%D \startformula\fakeformula\stopformula
%D \stopbuffer
%D
%D \typebuffer
%D
%D Or in print:
%D
%D % {\getbuffer}
%D
%D With formula numbers these formulas look as follows:
%D
%D \startbuffer
%D \setupformulas[align=left]
%D \placeformula \startformula\fakeformula\stopformula
%D \setupformulas[align=middle]
%D \placeformula \startformula\fakeformula\stopformula
%D \setupformulas[align=right]
%D \placeformula \startformula\fakeformula\stopformula
%D \stopbuffer
%D
%D % {\getbuffer}
%D
%D This was keyed in as:
%D
%D \typebuffer
%D
%D When tracing is turned on (\type {\tracemathtrue}) you can
%D visualize the bounding box of the formula,
%D
%D % {\tracemathtrue\getbuffer}
%D
%D As you can see, the dimensions are the natural ones, but if
%D needed you can force a normalized line:
%D
%D \startbuffer
%D \setupformulas[strut=yes]
%D \placeformula \startformula \fakeformula \stopformula
%D \stopbuffer
%D
%D \typebuffer
%D
%D This time we get a more spacy result.
%D
%D % {\tracemathtrue\getbuffer}
%D
%D We will now show a couple of more settings and combinations
%D of settings. In centered formulas, the number takes no space
%D
%D \startbuffer
%D \setupformulas[align=middle]
%D \startformula \fakeformula \stopformula
%D \placeformula \startformula \fakeformula \stopformula
%D \stopbuffer
%D
%D \typebuffer % {\tracemathtrue\getbuffer}
%D
%D You can influence the placement of the whole box with the
%D parameters \type {leftmargin} and \type {rightmargin}.
%D
%D \startbuffer
%D \setupformulas[align=right,leftmargin=3em]
%D \startformula \fakeformula \stopformula
%D \placeformula \startformula \fakeformula \stopformula
%D
%D \setupformulas[align=left,rightmargin=1em]
%D \startformula \fakeformula \stopformula
%D \placeformula \startformula \fakeformula \stopformula
%D \stopbuffer
%D
%D \typebuffer % {\tracemathtrue\getbuffer}
%D
%D You can also inherit the margin from the environment.
%D
%D \startbuffer
%D \setupformulas[align=right,margin=standard]
%D \startformula \fakeformula \stopformula
%D \placeformula \startformula \fakeformula \stopformula
%D \stopbuffer
%D
%D \typebuffer % {\tracemathtrue\getbuffer}
%D
%D The distance between the formula and the number is only
%D applied when the formula is left or right aligned.
%D
%D \startbuffer
%D \setupformulas[align=left,distance=2em]
%D \startformula \fakeformula \stopformula
%D \placeformula \startformula \fakeformula \stopformula
%D \stopbuffer
%D
%D \typebuffer % {\tracemathtrue\getbuffer}

\protect \endinput

% \abovedisplayshortskip0pt \belowdisplayshortskip0pt \abovedisplayskip0pt \belowdisplayskip0pt \forgetall
%
% test       \par                      $$xxxxxxxxxxxxxxxxxxxxxxxxxxxxxxxxxxxxxxxxxxxxxxxxxxxxxxxxxxxxxx$$ \par test \par
% test       \par                      $$xxxxxxxxxxxxxxxxxxxxxxxxxxxxxxxxxxxxxxxxxxxxxxxxxxxxxxxxxxxxxx$$ \par test \par
% test plus  \par \prevdepth \maxdimen $$xxxxxxxxxxxxxxxxxxxxxxxxxxxxxxxxxxxxxxxxxxxxxxxxxxxxxxxxxxxxxx$$ \par test \par
% test minus \par \prevdepth-\maxdimen $$xxxxxxxxxxxxxxxxxxxxxxxxxxxxxxxxxxxxxxxxxxxxxxxxxxxxxxxxxxxxxx$$ \par test \par
%
% \parskip\baselineskip
%
% test       \par                      $$xxxxxxxxxxxxxxxxxxxxxxxxxxxxxxxxxxxxxxxxxxxxxxxxxxxxxxxxxxxxxx$$ \par test \par
% test       \par                      $$xxxxxxxxxxxxxxxxxxxxxxxxxxxxxxxxxxxxxxxxxxxxxxxxxxxxxxxxxxxxxx$$ \par test \par
% test plus  \par \prevdepth \maxdimen $$xxxxxxxxxxxxxxxxxxxxxxxxxxxxxxxxxxxxxxxxxxxxxxxxxxxxxxxxxxxxxx$$ \par test \par
% test minus \par \prevdepth-\maxdimen $$xxxxxxxxxxxxxxxxxxxxxxxxxxxxxxxxxxxxxxxxxxxxxxxxxxxxxxxxxxxxxx$$ \par test \par
