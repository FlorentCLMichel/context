%D \module
%D   [       file=spec-ini,
%D        version=1996.01.25,
%D          title=\CONTEXT\ Special Macros,
%D       subtitle=Initialization,
%D         author=Hans Hagen,
%D           date=\currentdate,
%D      copyright={PRAGMA / Hans Hagen \& Ton Otten}]
%C
%C This module is part of the \CONTEXT\ macro||package and is
%C therefore copyrighted by \PRAGMA. Non||commercial use is 
%C granted. 

%D Specials are \TEX's channel to the outside world. They make
%D \TEX\ even more platform independant and permit easy
%D adaption to new developments. One major drawback of specials
%D is that they have to be supported by printer drivers. We've
%D tried to overcome this problem by implementinmg specials as
%D a sort of drivers themselves.

\writestatus{loading}{Context Special Macros / Initialization}

\unprotect

\startmessages  dutch  library: specials
  title: specials
      1: -- geladen
      2: verdere nesting is niet toegestaan --
      3: -- gereset
      4: commando -- bestaat niet
      5: definitiefile -- wordt geladen
      6: nesting is niet toegestaan
\stopmessages

\startmessages  english  library: specials
  title: specials
      1: -- loaded
      2: no deeper nesting is permitted --
      3: -- is reset
      4: command -- does not exist
      5: loading definition file --
      6: nesting is not permitted
\stopmessages

\startmessages  german  library: specials
  title: spezielles
      1: -- geladen
      2: Keine tiefere Verschachtelung erlaubt --
      3: -- ist zurueckgesetzt
      4: Befehl -- existiert nicht
      5: Lade Definitionsdatei --
      6: Verschachtelung nicht erlaubt
\stopmessages

\startmessages  dutch  library: interactions
     21: -- code tussengevoegd
\stopmessages

\startmessages  english  library: interactions
     21: -- code inserted
\stopmessages

\startmessages  german  library: interactions
     21: -- Code eingefuegt
\stopmessages

%D Because there is no standardization in the use of specials,
%D more than one driver or program can be supported. The
%D specials are grouped in libraries. Some of these are
%D general, such as the \type{postscript} library, some are
%D tuned to a special kind of program, like the \type{pdf}
%D ones, and some support a specific driver, as we can see in
%D the \type{yandy} library. A library is build with the
%D commands:
%D
%D \starttypen
%D \startspecials[name][inheritance]
%D
%D \definespecial\none{...}
%D \definespecial\onlyone#1{...}
%D \definespecial\alot#1#2#3#4{...}
%D
%D \stopspecials
%D \stoptypen
%D
%D Because drivers show some overlap in their support of
%D specials, a mechanism of inheritance is implemented. The
%D predefined libraries show this feature.
%D
%D Every special has to be predefined first. We  do this with
%D the command:
%D
%D \starttypen
%D \installspecial [\none]    [and] [0]
%D \installspecial [\onlyone] [and] [1]
%D \installspecial [\alot]    [or]  [4]
%D \stoptypen
%D
%D This means as much as: there is a special names
%D \type{\none} which has no arguments and has more than one
%D appearance. The special \type{\alot} on the other hand has
%D four arguments and is only defined once. Every instance in
%D the libraries of a special of category \type{and} is
%D executed when called upon, but only one special of
%D category \type{or} can be active. Most of the
%D \type{postscript}||specials are of category \type{or},
%D because they tend to interfere with driver specific ones.
%D The interactive specials of \type{dviwindo} and \type{pdf}
%D are an example of specials that can be called both.
%D
%D A library is defined in a file with the name
%D \type{spec-...}. We load a library with the command:
%D
%D \starttypen
%D \usespecials [list]
%D \stoptypen
%D
%D where the list can contain one or more file tags, the
%D \type{...} in the filename. The keyword \type{reset}
%D resets all loaded specials. This is equivalent to
%D \type{\resetspecials}.

%D Although a mechanism of nesting can be implemented, we
%D prefer to use a inheritance mechanism as mentioned. Calls
%D upon \type{\usespecials} within a \type{\startspecials}
%D would lead to confusion and errors.

\newif\ifinheritspecials

%D We define some local constants and variables. They look a
%D bit horrible but we don't want conflicts.

\def\@@specfil@@{@@specfil@@}
\def\@@speclst@@{@@speclst@@}
\def\@@speccat@@{@@speccat@@}
\def\@@specarg@@{@@specarg@@}
\def\@@specexc@@{@@specexc@@}

\def\currentspecial      {}
\def\currentspecialfile  {}
\def\preloadedspecials   {}

%D \macros
%D   {preloadspecials}
%D   {}
%D
%D The following command can be used to show the loaded list
%D of specials.

\def\preloadspecials%
  {\doifsomething{\preloadedspecials}
     {\showmessage{\m!specials}{1}{\preloadedspecials}}}

%D \macros
%D   {startspecials}
%D   {}
%D
%D Every library has a unique name, which is given as the first
%D argument to \type{\startspecials}. When another library is
%D defined with the same name, previous specials can be
%D overruled. The name may differ from the file||tag.
%D
%D The optional second argument can consist of a list of
%D libraries that are to be loaded first. This list can contain
%D file||tags or names of libraries. Names are often more
%D meaningful.

\def\dostartspecials[#1][#2]%
  {\let\mainspecialfile=\currentspecialfile
   \doifelsenothing{#2}
     {\inheritspecialsfalse}
     {\ifinheritspecials
        \showmessage{\m!specials}{2}{(#2)}%
      \else
        \inheritspecialstrue
        \processcommalist[#2]\dousespecials
        \inheritspecialsfalse
      \fi}%
   \doifelsenothing{#1}
     {\def\currentspecial{\s!unknown}}
     {\def\currentspecial{#1}}%
   \let\currentspecialfile=\mainspecialfile
   \setevalue{\@@specfil@@\currentspecial}{\currentspecialfile}%
   \unprotect
   \addtocommalist{\currentspecial}\preloadedspecials}

\def\startspecials%
  {\dodoubleempty\dostartspecials}

\def\stopspecials%
  {\def\currentspecial{}%
   \protect}

%D \macros
%D   {installspecial,
%D    resetspecials}
%D   {}
%D
%D We have to install specials before we can define and use
%D them. The command itself is defined as a call to another
%D command that executes one or more user||defined specials,
%D depending of it's category: \type{or} versus \type{and}.
%D
%D The command \type{\installspecial} takes three
%D (non||optional) arguments: the name of the command, the
%D category it belongs to and the number of arguments it
%D takes.
%D
%D With \type{\resetspecials} we can unload the predefined
%D specials.

\def\@@allspecials{}

\def\doinstallspecial[#1][#2][#3]%
  {\setvalue{\@@speclst@@\string#1}{}%
   \setvalue{\@@speccat@@\string#1}{#2}%
   \setvalue{\@@specarg@@\string#1}{#3}%
   \addtocommalist{\string#1}\@@allspecials
   \def#1{\executespecial#1}}

\def\installspecial%
  {\dotripleargument\doinstallspecial}

\def\resetspecials%
  {\def\docommando##1%
     {\setvalue{\@@speclst@@##1}{}}%
   \processcommacommand[\@@allspecials]\docommando
   \showmessage{\m!specials}{3}{\preloadedspecials}%
   \def\preloadedspecials{}%
   \def\@@allspecials{}}

%D \macros
%D   {definespecial}
%D   {}
%D
%D The command \type{\definespecial} take the place of
%D \type{\def} in the definition of a special. Just to be
%D sure, we first check if the command is permitted, i.e.
%D installed. If not, we give a warning and gobble the
%D illegal command in an quite elegant way.
%D
%D If the command can be combined (\type{and}) with others,
%D we append it to a list, otherwise (\type{or}) it becomes
%D the only item in the list.

\def\definespecial#1%
  {\ifx#1\undefined
     \showmessage{\m!specials}{4}{\string#1}%
     \def\next%
       {\def\@@illegalspecial@@}%
   \else
     \def\next%
       {\doifelse{\getvalue{\@@speccat@@\string#1}}{or}
          {\edef\@@newspeclst@@{\currentspecial}}
          {\edef\@@newspeclst@@{\getvalue{\@@speclst@@\string#1}}%
           \addtocommalist{\currentspecial}\@@newspeclst@@}%
        \setevalue{\@@speclst@@\string#1}{\@@newspeclst@@}%
        \setvalue{\currentspecial\string#1}}%
   \fi
   \next}

%D \macros
%D   {usespecials}
%D   {}
%D
%D We use \type{\usespecials} to load a specific library.
%D This command is only permitted outside de definition part.

\def\dousespecials#1%
  {\doifelse{#1}{\v!reset}
     {\resetspecials}
     {\doifdefinedelse{\@@specfil@@#1}
        {\edef\currentspecialfile{\getvalue{\@@specfil@@#1}}}
        {\edef\currentspecialfile{#1}}%
      \makeshortfilename[\f!specialprefix\currentspecialfile]%
      \showmessage{\m!specials}{5}{\currentspecialfile}%
      \startreadingfile
      \readsysfile{\shortfilename}{}{}%
      \stopreadingfile
      \showmessage{\m!specials}{1}{\preloadedspecials}}}

\def\usespecials[#1]%
  {\doifelsenothing{\currentspecial}
     {\processcommalist[#1]\dousespecials}
     {\showmessage{\m!specials}{6}{}}}

%D \macros
%D   {executespecials}
%D   {}
%D
%D The command \type{\executespecials} is used to execute the
%D defined specials. Once a special is installed, the special
%D itself calls for this command, so it's not needed outside
%D this module. One can use it if wanted.
%D 
%D A former implementation grouped the execution. Recent 
%D additions however |<|like the specials that implement object 
%D handling|>| asked for non||grouped execution. 

\def\executespecials#1#2%
  {\def\doonespecial##1%
     {\getvalue{##1\string#1}#2\relax}%
   \processcommacommand
     [\getvalue{\@@speclst@@\string#1}]\doonespecial}

\def\executespecial#1%
  {\expandafter\ifcase\getvalue{\@@specarg@@\string#1}\relax
     \def\next%
       {\executespecials#1{}}%
   \or
     \def\next##1%
       {\executespecials#1{{##1}}}%
   \or
     \def\next##1##2%
       {\executespecials#1{{##1}{##2}}}%
   \or
     \def\next##1##2##3%
       {\executespecials#1{{##1}{##2}{##3}}}%
   \or
     \def\next##1##2##3##4%
       {\executespecials#1{{##1}{##2}{##3}{##4}}}%
   \or
     \def\next##1##2##3##4##5%
       {\executespecials#1{{##1}{##2}{##3}{##4}{##5}}}%
   \or
     \def\next##1##2##3##4##5##6%
       {\executespecials#1{{##1}{##2}{##3}{##4}{##5}{##6}}}%
   \or
     \def\next##1##2##3##4##5##6##7%
       {\executespecials#1{{##1}{##2}{##3}{##4}{##5}{##6}{##7}}}%
   \or
     \def\next##1##2##3##4##5##6##7##8%
       {\executespecials#1{{##1}{##2}{##3}{##4}{##5}{##6}{##7}{##8}}}%
   \or
     \def\next##1##2##3##4##5##6##7##8##9%
       {\executespecials#1{{##1}{##2}{##3}{##4}{##5}{##6}{##7}{##8}{##9}}}% 
   \else
     \def\next%
       {\message{illegal special: \string#1}}%
   \fi
   \next}

%D The \type{{{...}}} are needed because we pass all those 
%D arguments to the specials support macro. 

\let\openspecialfile  = \relax
\let\closespecialfile = \relax

% %D This is some new, experimental code, used for testing some 
% %D proposals of Laurent Siebenmann on behalf of the 
% %D \kap{EMJ} discussionlist. 
% 
% \newif\ifexternalspecials  \externalspecialsfalse
% \newif\ifspecialstatus     \specialstatustrue
% 
% \newwrite\specialfile
% 
% \def\openspecialfile%
%   {\immediate\openout\specialfile=\jobname.etc\relax}
% 
% \def\closespecialfile%
%   {\immediate\closeout\specialfile}
% 
% \let\internalspecial=\special
% 
% \def\externalspecial#1%
%   {\internalspecial{}%
%    \immediate\write\specialfile{\currentspecialdriver\space: #1}}
% 
% \def\doinstallspecial[#1][#2][#3]%
%   {\setvalue{\@@specexc@@\string#1}{}%
%    \setvalue{\@@speclst@@\string#1}{}%
%    \setvalue{\@@speccat@@\string#1}{#2}%
%    \setvalue{\@@specarg@@\string#1}{#3}%
%    \addtocommalist{\string#1}\@@allspecials
%    \def#1{\executespecial#1}}
% 
% \def\resetspecials%
%   {\def\docommando##1%
%      {\setvalue{\@@specexc@@##1}{}%
%       \setvalue{\@@speclst@@##1}{}}%
%    \processcommacommand[\@@allspecials]\docommando
%    \showmessage{\m!specials}{3}{\preloadedspecials}}
% 
% \def\executespecials#1#2%
%   {\edef\supportedspecials{\getvalue{\@@speclst@@\string#1}}%
%    \def\doonespecial##1%
%      {\doifdefined{##1\string#1}
%         {\def\currentspecialdriver{##1}%
%          \getvalue{##1\string#1}#2\relax}}%
%    \ifexternalspecials
%      \let\special=\externalspecial
%      \doifelse{\supportedspecials}{}
%        {\ifspecialstatus
%           \immediate\write\specialfile{}%
%           \immediate\write\specialfile{skipping  : \string#1}
%           \immediate\write\specialfile{}%
%         \fi}
%        {\ifspecialstatus
%           \immediate\write\specialfile{}%
%           \immediate\write\specialfile{executing : \string#1}%
%           \immediate\write\specialfile{supported : \supportedspecials}%
%         \fi
%         \immediate\write\specialfile{}%
%         \processcommacommand[\supportedspecials]\doonespecial}%
%    \else
%      \let\special=\internalspecial
%      \doifelse{\getvalue{\@@speccat@@\string#1}}{or}
%        {\doonespecial{\getvalue{\@@specexc@@\string#1}}}
%        {\processcommacommand[\supportedspecials]\doonespecial}%
%    \fi}
% 
% \def\definespecial#1%
%   {\ifx#1\undefined
%      \showmessage{\m!specials}{4}{\string#1}%
%      \def\next%
%        {\def\@@illegalspecial@@}%
%    \else
%      \def\next%
%        {\edef\@@newspeclst@@{\getvalue{\@@speclst@@\string#1}}%
%         \addtocommalist{\currentspecial}\@@newspeclst@@
%         \setevalue{\@@speclst@@\string#1}{\@@newspeclst@@}%
%         \setevalue{\@@specexc@@\string#1}{\currentspecial}%
%         \setvalue{\currentspecial\string#1}}%
%    \fi
%    \next}
% 
% %D So far for the experiment. 

\protect

%D The following libraries are defined. Two postscript
%D drivers are supported, as well as two mechanisms for
%D interactive texts.
%D
%D \startregelcorrectie
%D \starttabel[|l|l|l|l|l|]
%D \HL
%D \NC \bf file             \NC
%D     \bf name             \NC
%D     \bf calls            \NC
%D     \bf support          \NC
%D     \bf program / driver \NC\SR
%D \HL
%D \NC \tttf spec-tex       \NC
%D     \tttf tex            \NC
%D                          \NC
%D     Generic \TEX\ (\DVI) \NC
%D     (default)            \NC\FR
%D \NC \tttf spec-ps        \NC
%D     \tttf postscript     \NC
%D                          \NC
%D     Adobe PostScript     \NC
%D     (default)            \NC\MR
%D \NC \tttf spec-tr        \NC
%D     \tttf rokicky        \NC
%D     \tttf postscript     \NC
%D     Thomas Rokicky       \NC
%D     (dvips)              \NC\MR
%D \NC \tttf spec-yy        \NC
%D     \tttf yandy          \NC
%D     \tttf postscript     \NC
%D     YandY                \NC
%D     (dvipsone, dviwindo) \NC\MR
%D \NC \tttf spec-pdf       \NC
%D     \tttf pdf            \NC
%D                          \NC
%D     Adobe PDF V2.1       \NC
%D     (Acrobat)            \NC\MR
%D \NC \tttf spec-win       \NC
%D     \tttf dviwindo       \NC
%D     YandY                \NC
%D     (dviwindo)           \NC\MR
%D \NC \tttf spec-1p0       \NC
%D     \tttf pdf            \NC
%D                          \NC
%D     Adobe PDF V 1.0      \NC
%D     (Acrobat)            \NC\MR
%D \NC \tttf spec-2p0       \NC
%D     \tttf pdf            \NC
%D                          \NC
%D     Adobe PDF V 2.0      \NC
%D     (Acrobat)            \NC\MR
%D \NC \tttf spec-htm       \NC
%D     \tttf html           \NC
%D                          \NC
%D     HTML V 2.0           \NC
%D     (dvips)              \NC\LR
%D \HL
%D \stoptabel
%D \stopregelcorrectie

%D \macros
%D   {dostartgraymode,dostopgraymode,
%D    dostartrgbcolormode,dostartcmykcolormode,dostartgraycolormode,dostopcolormode}
%D   {}
%D
%D We start with the installation of color and grayscale
%D specials. The values are in the range 0..1 (e.g. 0.25).
%D
%D \starttypen
%D \dostartgraymode      {gray} ... \dostopgraymode
%D \dostartrgbcolormode  {red} {green} {blue} ... \dostopcolormode
%D \dostartcmykcolormode {cyan} {magenta} {yellow} {black} ... \dostopcolormode
%D \dostartgraycolormode {gray} ... \dostopcolormode
%D \stoptypen
%D
%D Because we can expect conflicts between drivers, we
%D implement them as category \type{or}. In previous versions
%D of \DVIPSONE\ the use of their color||specials did not
%D interfere with the PostScript ones, but recent versions do.

\installspecial [\dostartgraymode]      [or] [1]
\installspecial [\dostopgraymode]       [or] [0]

\installspecial [\dostartrgbcolormode]  [or] [3]
\installspecial [\dostartcmykcolormode] [or] [4]
\installspecial [\dostartgraycolormode] [or] [1]
\installspecial [\dostopcolormode]      [or] [0]

%D \macros
%D   {doinsertfile}
%D   {}
%D
%D Probably the most problematic special is the following
%D one. Because we want to be able to support different
%D schemes, we pass a lot of data to it.
%D
%D \starttypen
%D \doinsertfile {type,method} {file} 
%D               {xscale} {yscale} {x} {y} {w} {h} 
%D               {options}
%D \stoptypen
%D
%D The scale is given percents, the other values are base 
%D points.
%D
%D The special is implemented as \type{or}. Because
%D \DVIPSONE\ understands them all, a chain of alternatives
%D would generate multiple courrences of the same
%D illustration.
%D
%D When option 1 is passed, the viewers is asked to present a
%D preview, like the first frame of a movie. 

\installspecial [\doinsertfile] [or] [9]

%D \macros
%D   {dostartrotation,
%D    dostoprotation}
%D   {}
%D
%D We support rotation with the special:
%D
%D \starttypen
%D \dostartrotation {angle} ... \dostoprotation
%D \stoptypen
%D
%D For the moment these specials are installed as
%D category \type{or}.

\installspecial [\dostartrotation] [or] [1]
\installspecial [\dostoprotation]  [or] [0]

%D \macros
%D   {dostartscaling,
%D    dostopscaling}
%D   {}
%D
%D Scaling is also supported:
%D
%D \starttypen
%D \dostartscaling {x} {y} ... \dostopscaling
%D \stoptypen
%D
%D Like the previous one, these specials are of category 
%D \type{or}.

\installspecial [\dostartscaling] [or] [2]
\installspecial [\dostopscaling]  [or] [0]

%D \macros
%D   {dostartmirroring,
%D    dostopmirroring}
%D   {}
%D
%D And indeed, mirroring is there too:
%D
%D \starttypen
%D \dostartmirroring {x} {y} ... \dostopmirroring
%D \stoptypen
%D
%D Again these specials are installed as category \type{or}.

\installspecial [\dostartmirroring] [or] [0]
\installspecial [\dostopmirroring]  [or] [0]

%D \macros
%D   {dostartnegative,
%D    dostopnegative}
%D   {}
%D
%D When producing output for an image setter, negating the 
%D page comes into view. Here are the tools:

\installspecial [\dostartnegative] [or] [0]
\installspecial [\dostopnegative]  [or] [0]

%D \macros
%D   {doselectfirstpaperbin,
%D    doselectsecondpaperbin}
%D   {}
%D
%D Here are some very printer||specific ones. No further
%D comment.

\installspecial [\doselectfirstpaperbin]  [or]  [0]
\installspecial [\doselectsecondpaperbin] [or]  [0]

%D \macros
%D   {doovalbox}
%D   {}
%D
%D When we look at the implementation, this is a complicated
%D one. There are seven arguments.
%D
%D \starttypen
%D \doovalbox {w} {h} {d} {linewidth} {radius} {stroke} {fill}
%D \stoptypen
%D
%D This command has to return a \type{\vbox} which can be used
%D to lay over another one (with text). The radius is in
%D degrees, the stroke and fill are~\type{1} (true) of~\type{0}
%D (false). 

\installspecial [\doovalbox] [or] [7]

%D \macros
%D   {dosetupidentity}
%D   {}
%D
%D We can declare some characteristics of the document with
%D
%D \starttypen 
%D \dosetupidentity {title} {subject} {author} {creator} {date} 
%D \stoptypen
%D
%D All data is in string format. 

\installspecial [\dosetupidentity] [and] [5]

%D \macros
%D   {dosetuppaper}
%D   {}
%D
%D This special can be used to tell the driver what page size
%D to use. The special takes three arguments. 
%D
%D \starttypen
%D \dosetuppaper {type} {width} {height}
%D \stoptypen
%D
%D The type is one of the common identifiers, like A4, A5 or 
%D B2. 

\installspecial [\dosetuppaper] [and] [3]

%D \macros
%D   {dosetupprinter}
%D   {}
%D
%D Some drivers enable the user to specify the paper type 
%D used and/or page dimensions to be taken into account. 
%D
%D \starttypen
%D \dosetupprinter {type} {hoffset} {voffset} {width} {height}
%D \stoptypen
%D
%D The first argument is one of \type{letter}, \type{legal}, 
%D \type{A4}, \type{A5} etc. The dimensions are in 
%D basepoints. 

\installspecial [\dosetupprinter] [and] [5]

%D \macros
%D   {dosetuppage,
%D    dosetupinteraction,
%D    dosetupscreen}
%D   {}
%D
%D Here come some obscure interactive commands. Probably the
%D specs will change with the development of the macros that
%D use them.
%D
%D The first ones can be used to set up the interaction.
%D
%D \starttypen
%D \dosetupinteraction
%D \stoptypen
%D
%D Normally this command does nothing but giving a message
%D that some scheme is supported. Postscript prolog files
%D can best be loaded by the printer driver program.
%D
%D The second one sets up the screen. It takes three
%D arguments:
%D
%D \starttypen
%D \dosetupscreen {hoffset} {voffset} {width} {height} {options}
%D \stoptypen
%D
%D The first four arguments are in scaled points. Option~1 
%D results in a full screen launch.

\installspecial [\dosetupinteraction] [and] [0]
\installspecial [\dosetupscreen]      [and] [5]

%D \macros
%D   {dostarthide,
%D    dostophide}
%D   {}
%D
%D Not every part of the screen is suitable for paper. Menus
%D for instance have no meaning on an non||interactive medium.
%D These elements are hidden by means of:
%D
%D \starttypen
%D \dostarthide
%D \dostophide
%D \stoptypen

\installspecial [\dostarthide]        [or]  [0]
\installspecial [\dostophide]         [or]  [0]

%D \macros
%D   {dostartgotolocation, dostopgotolocation,
%D    dostartgotorealpage, dostopgotorealpage}
%D   {}
%D
%D The interactive real work is done by the following four
%D specials. The reason for providing the first one with both
%D a label and a number, is a result of the quite poor
%D implementation of \type{pdfmarks} in version 1.0 of
%D Acrobat. Because only pagenumbers were supported as
%D destination, we had to provide both labels (\DVIWINDO) and
%D pagenumbers (\PDF). Some drivers use start stop pairs. 
%D
%D \starttypen
%D \dostartgotolocation {w} {h} {url} {file} {label} {page}
%D \dostartgotorealpage {w} {h} {url} {file}         {page}
%D \stoptypen
%D
%D Their counterparts are:
%D
%D \starttypen
%D \dostopgotolocation  
%D \dostopgotorealpage  
%D \stoptypen
%D
%D The internal alternative is used for system||generated
%D links, the external one for user||generated links. The 
%D Uniform Resource Locator can be used to let the reader 
%D surf the net. 

\installspecial [\dostartgotolocation] [and] [6]
\installspecial [\dostopgotolocation]  [and] [0]
\installspecial [\dostartgotorealpage] [and] [5]
\installspecial [\dostopgotorealpage]  [and] [0]

%D \macros
%D  {dostartgotoJS, doflushJSpreamble}
%D   
%D
%D Rather special is the option to include and execute 
%D JavaScript code. This is a typical \PDF\ option.
%D
%D \starttypen
%D \dostartgotoJS {w} {h} {script}
%D \stoptypen
%D
%D This not so standard \TEX\ feature should be used with 
%D care. Preamble scripts are flushed by
%D
%D \doflushJSpreamble {script}

\installspecial [\dostartgotoJS]    [and] [3]
\installspecial [\dostopgotoJS]     [and] [0]
\installspecial [\doflushJSpreamble][and] [1]

%D \macros
%D   {dostartthisislocation, dostopthisislocation,
%D    dostartthisisrealpage, dostopthisisrealpage}
%D   {}
%D
%D The opposite commands of \type{\dogotosomething} have only
%D one argument:
%D
%D \starttypen
%D \dostartthisislocation {label}
%D \dostartthisisrealpage {page}
%D \stoptypen
%D
%D These commands are accompanied by: 
%D
%D \starttypen
%D \dostopthisislocation  
%D \dostopthisisrealpage 
%D \stoptypen
%D
%D As with all interactive commands's they are installed as
%D \type{and} category specials.

\installspecial [\dostartthisislocation] [and] [1]
\installspecial [\dostopthisislocation]  [and] [0]
\installspecial [\dostartthisisrealpage] [and] [1]
\installspecial [\dostopthisisrealpage]  [and] [0]

%D \macros
%D   {dostartexecutecommand, dostopexecutecommand}
%D   {}
%D
%D The actual behavior of the next pair of commands depends 
%D much on the viewing engine. Therefore one cannot depend 
%D too much on their support. 
%D
%D \starttypen
%D \dostartexecutecommand {w} {h} {command} {options} 
%D \stoptypen
%D
%D
%D The next commands are supported: 
%D
%D \startregelcorrectie\steluitlijnenin[midden]\leavevmode
%D \starttabel[|l|l|] 
%D \HL
%D \NC \bf command  \NC \bf action                  \NC\SR 
%D \HL
%D \NC first        \NC go to the first page        \NC\FR
%D \NC previous     \NC go to the previous page     \NC\MR
%D \NC next         \NC go to the next page         \NC\MR
%D \NC last         \NC go to the last page         \NC\MR
%D \NC backward     \NC go back to the  link list   \NC\MR
%D \NC forward      \NC go forward in the link list \NC\MR
%D \NC print        \NC enter print mode            \NC\MR
%D \NC exit         \NC exit viewer                 \NC\MR
%D \NC close        \NC close document              \NC\MR
%D \NC enter        \NC enter viewer                \NC\MR
%D \NC help         \NC show help on the viewer     \NC\LR
%D \HL
%D \stoptabel
%D \stopregelcorrectie
%D
%D There are no options yet. Options are to be passed as a 
%D comma separated list of assignments.

\installspecial [\dostartexecutecommand] [and] [4]
\installspecial [\dostopexecutecommand]  [and] [0]

%D \macros
%D   {dostartobject, 
%D    dostopobject,
%D    doinsertobject}
%D
%D Reuse of object can reduce the output filesize 
%D considerably. Reusable objects are implemented with: 
%D
%D \starttypen
%D \dostartobject{name}{width}{height}{depth}
%D some typeset material
%D \dostopobject
%D \stoptypen
%D
%D \starttypen
%D \doinsertobject{name}
%D \stoptypen
%D
%D The savings can be huge in interactive texts. 

\installspecial [\dostartobject]  [or] [4]
\installspecial [\dostopobject]   [or] [0]
\installspecial [\doinsertobject] [or] [1]

% %D \macros
% %D   {dogetobjectreference} 
% %D
% %D For very special purposes, one can ask for the internal 
% %D reference to the object. Beware!
% 
% \installspecial [\dogetobjectreference] [or] [2]
% 
% %D The first argument is the name, the second a macro that 
% %D gets the assiciated value.

%D \macros
%D   {dostartrunprogram, dostoprunprogram,
%D    dostartgotoprofile, dostopgotoprofile,
%D    dobeginofprofile,
%D    doendofprofile}
%D   {}
%D
%D These specials are still experimental. They are not yet
%D supported by the programs the way they should be.
%D
%D {\em --- still undocumented ---}

\installspecial [\dostartrunprogram]  [and] [3]
\installspecial [\dostoprunprogram]   [and] [0]
\installspecial [\dostartgotoprofile] [and] [3]
\installspecial [\dostopgotoprofile]  [and] [0]
\installspecial [\dobeginofprofile]   [and] [4]
\installspecial [\doendofprofile]     [and] [0]

%D \macros
%D   {doinsertbookmark}
%D 
%D Bookmarks, that is viewer generated tables of contents, are
%D a strange phenomena, mainly because \TEX\ can provide
%D whatever kind of table in much better quality. 

\installspecial [\doinsertbookmark] [and] [5]

%D This special is called as:
%D 
%D \starttypen
%D \doinstallbookmark {level} {nofsubentries} {text} {page} {open}
%D \stoptypen
%D
%D This definition is very \PDF\ oriented, so for more 
%D information we kindly refer to the \PDF\ manuals. 

%D \macros
%D   {dosetpagetransition}
%D
%D In presentations, fancy page transitions can, at least for a
%D short moment, let the audience focus at the screen. Like the
%D previous one, this special is very \PDF. 
%D
%D \starttypen
%D \dosetpagetransition{dissolve}
%D \stoptypen
%D
%D Transitions have symbolic names, like dissolve, box, split,
%D blinds, wipe and glitter.

\installspecial [\dosetpagetransition] [or] [1]

%D So far for the installation. Finally we preload our
%D favorite set of specials.

\usespecials[ps,yy,win,pdf]

%D One can overrule this by for instance
%D 
%D \starttypen
%D \usespecials[reset,ps,tr,pdf]
%D \stoptypen

\endinput
