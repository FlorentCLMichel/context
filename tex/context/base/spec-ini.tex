%D \module
%D   [       file=spec-ini,
%D        version=1996.01.25,
%D          title=\CONTEXT\ Special Macros,
%D       subtitle=Initialization,
%D         author=Hans Hagen,
%D           date=\currentdate,
%D      copyright={PRAGMA / Hans Hagen \& Ton Otten}]
%C
%C This module is part of the \CONTEXT\ macro||package and is
%C therefore copyrighted by \PRAGMA. See mreadme.pdf for
%C details.

% todo: make this one more plain so that i can use it in plain jobs
% todo: no args, named vars instead
% maybe also drop multiple drivers and make simplify the default, then simplier defs

\newif\ifsimplifyspecials \simplifyspecialstrue % see later

%D We will forbid loading specials after the first page is
%D shipped out.

\ifx\realpageno\undefined
  \countdef\realpageno=0 \realpageno=1
\fi

%D Specials are \TEX's channel to the outside world. They make
%D \TEX\ even more platform independant and permit easy
%D adaption to new developments. One major drawback of specials
%D is that they have to be supported by printer drivers. We've
%D tried to overcome this problem by implementing specials as
%D a sort of drivers themselves.

\writestatus{loading}{Context Special Macros / Initialization}

\unprotect

\startmessages  dutch  library: specials
  title: specials
      1: -- geladen
      2: verdere nesting is niet toegestaan --
      3: -- gereset
      4: commando -- bestaat niet
      5: definitiefile -- wordt geladen
      6: nesting is niet toegestaan
      7: onbekende driver --
\stopmessages

\startmessages  english  library: specials
  title: specials
      1: -- loaded
      2: no deeper nesting is permitted --
      3: -- is reset
      4: command -- does not exist
      5: loading definition file --
      6: nesting is not permitted
      7: unknown driver --
\stopmessages

\startmessages  german  library: specials
  title: spezielles
      1: -- geladen
      2: keine tiefere Verschachtelung erlaubt --
      3: -- ist zurueckgesetzt
      4: Befehl -- existiert nicht
      5: lade Definitionsdatei --
      6: Verschachtelung nicht erlaubt
      7: unbekante Driver --
\stopmessages

\startmessages  czech  library: specials
  title: speciality
      1: -- nacteno
      2: neni dovoleno hlubsi zanoreni --
      3: -- je resetovano
      4: prikaz -- neexistuje
      5: nacita se definicni soubor --
      6: zanoreni neni dovoleno
      7: neznamy ovladac (driver) --
\stopmessages

\startmessages  italian  library: specials
  title: specialit�
      1: -- caricato
      2: non � permesso un annidamento maggiore --
      3: -- reimpostato
      4: il comando -- non esiste
      5: caricamento del file di definizione --
      6: annidamento non permesso
      7: driver sconosciuto --
\stopmessages

\startmessages  norwegian  library: specials
  title: specials
      1: -- er lest inn
      2: dypere 'nesting' er ikke tillatt --
      3: -- er tilbakestilt
      4: kommando -- eksisterer ikke
      5: leser inn definisjonsfil for --
      6: 'nesting' er ikke tillatt
      7: ukjent driver --
\stopmessages

\startmessages  romanian  library: specials
  title: specials
      1: -- incarcat
      2: nu este permis un nivel de imbricare mai mare --
      3: -- s-a resetat
      4: comanda -- nu exista
      5: se incarca fisierul de definitii --
      6: imbricarea nu este permisa
      7: driver necunoscut --
\stopmessages

\startmessages  dutch  library: interactions
     21: -- code tussengevoegd
\stopmessages

\startmessages  english  library: interactions
     21: -- code inserted
\stopmessages

\startmessages  german  library: interactions
     21: -- Code eingefuegt
\stopmessages

\startmessages  czech  library: interactions
     21: -- kod vlozen
\stopmessages

\startmessages  italian  library: interactions
     21: codice -- inserito
\stopmessages

\startmessages  norwegian  library: interactions
     21: -- kode satt inn / tilf�yd
\stopmessages

\startmessages  romanian  library: interactions
     21: -- cod inserat
\stopmessages

%D \TEX\ produces files in the \DVI\ format. This format is
%D well defined and stable. In this format one||byte commands
%D are used which can optionally be followed by length
%D specifiers and arguments. The \DVI||format incorporates a
%D channel to the outside world. This channel is activated by
%D the \TEX\ primitive \type {\special}. The sequence
%D
%D \starttyping
%D \special{Hello here I am.}
%D \stoptyping
%D
%D results in \DVI||codes:
%D
%D \starttyping
%D xxx1 16 Hello here I am.
%D \stoptyping
%D
%D The \type {xxx1} is represented in byte code 239 and the
%D number of following bytes in a~1, 2, 3 or~4 byte number. So
%D here we get $1+1+16$ bytes of code.
%D
%D Translating these codes is upto the \DVI\ driver. It's
%D common use to ignore specials that cannot be interpreted, so
%D the example string should have no consequences for the
%D output.

%D \macros
%D   {everyresetspecials}
%D
%D Now what will this one do? We'll see in a few lines.

\newevery \everyresetspecials \relax

\appendtoksonce
  \ifx\setjobsuffix\undefined\else\setjobsuffix{dvi}\fi
\to \everyresetspecials

%D A rather fundamental difference between special and direct
%D settings is that the latter don't interfere with typesetting
%D but must be set before the first shipout, while the specials
%D must be packaged in the shipped out box in such a way that
%D they don't interfere.

\newif\ifspecialbasedsettings \specialbasedsettingstrue

\appendtoksonce
  \specialbasedsettingstrue
\to \everyresetspecials

%D Because there is no standardization in the use of specials,
%D more than one driver or program can be supported. The
%D specials are grouped in libraries. Some of these are
%D general, such as the \type{postscript} library, some are
%D tuned to a special kind of program, like the \type{pdf}
%D ones, and some support a specific driver, as we can see in
%D the \type{yandy} library. A library is build with the
%D commands:
%D
%D \starttyping
%D \startspecials[name][inheritance]
%D
%D \definespecial\none{...}
%D \definespecial\onlyone#1{...}
%D \definespecial\alot#1#2#3#4{...}
%D
%D \stopspecials
%D \stoptyping
%D
%D Because drivers can have overlap in low level macros, a
%D mechanism of inheritance is implemented. The libraries
%D defined as second argument are loaded first.
%D
%D Every special has to be predefined first. We  do this with
%D the command:
%D
%D \starttyping
%D \installspecial [\none]    [and] [0]
%D \installspecial [\onlyone] [and] [1]
%D \installspecial [\alot]    [or]  [4]
%D \stoptyping
%D
%D This means as much as: there is a special names
%D \type{\none} which has no arguments and has more than one
%D appearance. The special \type{\alot} on the other hand has
%D four arguments and is only defined once. Every instance in
%D the libraries of a special of category \type{and} is
%D executed when called upon, but only one special of
%D category \type{or} can be active. Most of the
%D \type{postscript}||specials are of category \type{or},
%D because they tend to interfere with driver specific ones.
%D The interactive specials of \type{dviwindo} and \type{pdf}
%D are an example of specials that can be called both.
%D
%D A library is defined in a file with the name
%D \type{spec-...}. We load a library with the command:
%D
%D \starttyping
%D \usespecials [list]
%D \stoptyping
%D
%D where the list can contain one or more file tags, the
%D \type{...} in the filename. The keyword \type{reset}
%D resets all loaded specials. This is equivalent to
%D \type{\resetspecials}.

%D Although a mechanism of nesting can be implemented, we
%D prefer to use a inheritance mechanism as mentioned. Calls
%D upon \type{\usespecials} within a \type{\startspecials}
%D would lead to confusion and errors.

\newif\ifinheritspecials

%D We define some local constants and variables. They look a
%D bit horrible but we don't want conflicts.

\def\@@specfil@@{@@spcfil@@}
\def\@@speclst@@{@@spclst@@}
\def\@@speccat@@{@@spccat@@}
\def\@@specarg@@{@@spcarg@@}
\def\@@specexc@@{@@spcexc@@}

% not faster
%
% \def\@@specfil@@{@sp@f@}
% \def\@@speclst@@{@sp@l@}
% \def\@@speccat@@{@sp@c@}
% \def\@@specarg@@{@sp@a@}
% \def\@@specexc@@{@sp@e@}

\let\currentspecial    \empty
\let\currentspecialfile\empty
\let\preloadedspecials \empty

%D \macros
%D   {preloadspecials}
%D
%D The following command can be used to show the loaded list
%D of specials.

\def\preloadspecials
  {\doifsomething\preloadedspecials
     {\showmessage\m!specials1\preloadedspecials}}

%D \macros
%D   {startspecials}
%D
%D Every library has a unique name, which is given as the first
%D argument to \type{\startspecials}. When another library is
%D defined with the same name, previous specials can be
%D overruled. The name may differ from the file||tag.
%D
%D The optional second argument can consist of a list of
%D libraries that are to be loaded first.

\def\dostartspecials[#1][#2]%
  {\doifsomething{#2}
     {\processcommalist[#2]\dousespecials}%
   \doifelsenothing{#1}
     {\let\currentspecial\s!unknown}
     {\def\currentspecial{#1}}%
   \unprotect
   \addtocommalist\currentspecial\preloadedspecials}

\def\startspecials
  {\localpushmacro\currentspecial
   \dodoubleempty\dostartspecials}

\def\stopspecials
  {\localpopmacro\currentspecial
   \protect}

%D \macros
%D   {installspecial,
%D    resetspecials}
%D
%D We have to install specials before we can define and use
%D them. The command itself is defined as a call to another
%D command that executes one or more user||defined specials,
%D depending of it's category: \type{or} versus \type{and}.
%D
%D The command \type{\installspecial} takes three
%D (non||optional) arguments: the name of the command, the
%D category it belongs to and the number of arguments it
%D takes.
%D
%D With \type{\resetspecials} we can unload the predefined
%D specials. Special reset actions |<|look in \type{spec-mis}
%D for an example|>| can be assigned to the token register
%D \type{\everyresetspecials}.

\let\@@allspecials=\empty

\def\doinstallspecial[#1][#2][#3]%
  {\letvalue{\@@speclst@@\string#1}\empty
   \setvalue{\@@speccat@@\string#1}{#2}%
   \setvalue{\@@specarg@@\string#1}{#3}%
   \addtocommalist{\string#1}\@@allspecials
   \def#1{\executespecial#1}}

\def\installspecial
  {\dotripleargument\doinstallspecial}

\def\resetspecials
  {\the\everyresetspecials
   \def\docommando##1%
     {\letvalue{\@@speclst@@##1}\empty}%
   \processcommacommand[\@@allspecials]\docommando
   \ifx\preloadedspecials\empty \else
     \showmessage\m!specials3\preloadedspecials
     \let\preloadedspecials\empty
   \fi}

%D \macros
%D   {definespecial}
%D
%D The command \type{\definespecial} take the place of
%D \type{\def} in the definition of a special. Just to be
%D sure, we first check if the command is permitted, i.e.
%D installed. If not, we give a warning and gobble the
%D illegal command in an quite elegant way.
%D
%D If the command can be combined (\type{and}) with others,
%D we append it to a list, otherwise (\type{or}) it becomes
%D the only item in the list.

\def\definespecial#1%
  {\ifx#1\undefined
     \showmessage\m!specials4{\string#1}%
     \def\next
       {\def\@@illegalspecial@@}%
   \else
     \def\next
       {\doifelsevalue{\@@speccat@@\string#1}{or}
          {\edef\@@newspeclst@@{\currentspecial}}
          {\edef\@@newspeclst@@{\getvalue{\@@speclst@@\string#1}}%
           \addtocommalist\currentspecial\@@newspeclst@@}%
        \setevalue{\@@speclst@@\string#1}{\@@newspeclst@@}%
        \setvalue{\currentspecial\string#1}}%
   \fi
   \next}

%D \macros
%D   {usespecials}
%D
%D We use \type{\usespecials} to load a specific library.
%D This command is only permitted outside the definition part.

\def\dousespecials#1%
  {\doifelse{#1}\v!reset
     {\resetspecials}
     {\doifdefinedelse{\@@specfil@@#1}
        {\edef\currentspecialfile{\getvalue{\@@specfil@@#1}}}
        {\edef\currentspecialfile{#1}}%
      \makeshortfilename[\f!specialprefix\currentspecialfile]%
      \showmessage\m!specials5\currentspecialfile
      \startreadingfile
      \readsysfile\shortfilename\donothing\donothing
      \stopreadingfile
      \showmessage\m!specials1\preloadedspecials}}

\def\usespecials[#1]%
  {\ifnum\realpageno<2
     \doifelsenothing\currentspecial
       {\processcommalist[#1]\dousespecials}
       {\showmessage\m!specials6\empty}%
   \fi}

%D \macros
%D   {executespecials}
%D
%D The command \type{\executespecials} is used to execute the
%D defined specials. Once a special is installed, the special
%D itself calls for this command, so it's not needed outside
%D this module. One can use it if wanted.
%D
%D A former implementation grouped the execution. Recent
%D additions however |<|like the specials that implement object
%D handling|>| asked for non||grouped execution.

%D \starttyping
%D \def\executespecials#1#2%
%D   {\def\doonespecial##1%
%D      {\getvalue{##1\string#1}#2\relax}%
%D    \processcommacommand
%D      [\getvalue{\@@speclst@@\string#1}]\doonespecial}
%D
%D \def\executespecial#1%
%D   {\expandafter\ifcase\getvalue{\@@specarg@@\string#1}\relax
%D      \def\next%
%D        {\executespecials#1{}}%
%D    \or
%D      \def\next##1%
%D        {\executespecials#1{{##1}}}%
%D    \or
%D      \def\next##1##2%
%D        {\executespecials#1{{##1}{##2}}}%
%D    \or
%D      \def\next##1##2##3%
%D        {\executespecials#1{{##1}{##2}{##3}}}%
%D    \or
%D      \def\next##1##2##3##4%
%D        {\executespecials#1{{##1}{##2}{##3}{##4}}}%
%D    \or
%D      \def\next##1##2##3##4##5%
%D        {\executespecials#1{{##1}{##2}{##3}{##4}{##5}}}%
%D    \or
%D      \def\next##1##2##3##4##5##6%
%D        {\executespecials#1{{##1}{##2}{##3}{##4}{##5}{##6}}}%
%D    \or
%D      \def\next##1##2##3##4##5##6##7%
%D        {\executespecials#1{{##1}{##2}{##3}{##4}{##5}{##6}{##7}}}%
%D    \or
%D      \def\next##1##2##3##4##5##6##7##8%
%D        {\executespecials#1{{##1}{##2}{##3}{##4}{##5}{##6}{##7}{##8}}}%
%D    \or
%D      \def\next##1##2##3##4##5##6##7##8##9%
%D        {\executespecials#1{{##1}{##2}{##3}{##4}{##5}{##6}{##7}{##8}{##9}}}%
%D    \else
%D      \def\next%
%D        {\message{illegal special: \string#1}}%
%D    \fi
%D    \next}
%D \stoptyping
%D
%D Because specials happen quite often, we will use a bit more
%D brute force. Keep in mind that we have to collect the
%D arguments because we want to support more drivers at once.
%D
%D I tested this on the next test. Where the previous alternative
%D took about 32 seconds, the new alternative takes 25 seconds.
%D
%D \starttyping
%D \testfeature{10000}{\setbox0=\hbox{test \color[red]{oeps} test}}
%D \stoptyping

\def\@@exsp{exsp}

\setvalue{\@@exsp0}{{}}
\setvalue{\@@exsp1}#1{{{#1}}}
\setvalue{\@@exsp2}#1#2{{{#1}{#2}}}
\setvalue{\@@exsp3}#1#2#3{{{#1}{#2}{#3}}}
\setvalue{\@@exsp4}#1#2#3#4{{{#1}{#2}{#3}{#4}}}
\setvalue{\@@exsp5}#1#2#3#4#5{{{#1}{#2}{#3}{#4}{#5}}}
\setvalue{\@@exsp6}#1#2#3#4#5#6{{{#1}{#2}{#3}{#4}{#5}{#6}}}
\setvalue{\@@exsp7}#1#2#3#4#5#6#7{{{#1}{#2}{#3}{#4}{#5}{#6}{#7}}}
\setvalue{\@@exsp8}#1#2#3#4#5#6#7#8{{{#1}{#2}{#3}{#4}{#5}{#6}{#7}{#8}}}
\setvalue{\@@exsp9}#1#2#3#4#5#6#7#8#9{{{#1}{#2}{#3}{#4}{#5}{#6}{#7}{#8}{#9}}}

%D \starttyping
%D \def\executespecials#1%
%D   {\def\doonespecial##1%
%D      {\csname##1\xspecialcommand\endcsname#1\relax}%
%D    \@EA\rawprocesscommalist\@EA
%D      [\csname\@@speclst@@\xspecialcommand\endcsname]\doonespecial}
%D
%D \def\executespecial#1%
%D   {\def\xspecialcommand{\string#1}%
%D    \@EA\@EA\@EA\executespecials\csname\@@exsp\csname\@@specarg@@\xspecialcommand\endcsname\endcsname}
%D \stoptyping

%D Some more speed can be gained by using a dedicated string
%D processing routine. Now we can bring down the execution
%D time to 21 seconds, one third less than the original run time.

\def\executespecials#1%
  {\@EA\let\@EA\speciallist\csname\@@speclst@@\xspecialcommand\endcsname
   \ifx\speciallist\empty\else
     \def\doonespecial##1%
       {\csname##1\xspecialcommand\endcsname#1\relax}%
     \@EA\dodoonespecial\speciallist,\end,%
   \fi}

\def\executespecial#1%
  {\def\xspecialcommand{\string#1}%
   \@EA\@EA\@EA\executespecials\csname\@@exsp\csname\@@specarg@@\xspecialcommand\endcsname\endcsname}

\def\dodoonespecial#1,%
  {\ifx\end#1\else
     \doonespecial{#1}\expandafter\dodoonespecial
   \fi}

%D This kind of saving only shows up when making interative
%D documents with lots of color switches. In such documents
%D tens of thousands of special calls are rather normal.
%D On a 650 Mhz Pentium, the previous test takes 15 seconds
%D less (on about 65 seconds). When processing 2000 page
%D interactive documents this saving can be neglected.

%D In the previous macros, the \type{{{...}}} are needed
%D because we pass all those arguments to the specials support
%D macro.

\let\openspecialfile  \relax
\let\closespecialfile \relax

%D \macros
%D   {doifspecialavailableelse}
%D
%D For testing purposes (this was first needed when object
%D support was implemented) we have:
%D
%D \starttyping
%D \doifspecialavailableelse\specialcommand{true}{false}
%D \stoptyping
%D
%D e.g:
%D
%D \starttyping
%D \doifspecialavailableelse\doinsertobject{...}{...}
%D \stoptyping

\def\doifspecialavailableelse#1#2#3%
  {\doifelsevaluenothing{\@@speclst@@\string#1}{#3}{#2}}

%D So far for the macros that deal with installing specials.
%D In the file \type {spec-def} you will find the predefined
%D specials.

%D Now that we have seen the flexible way (permitting
%D special chains) we will implement a faster and flat
%D alternative. But only if flag si set.

\ifsimplifyspecials

  \def\doinstallspecial[#1][#2][#3]%
    {\appendtoks\forgetspecial#1{#3}\to\everyresetspecials
     \@EA\chardef\csname\@@speclst@@\string#1\endcsname\zerocount
     \forgetspecial#1{#3}}

  \def\forgetspecial#1#2%
    {\ifcase#2\relax
        \let#1\relax                \or
        \let#1\gobbleoneargument    \or
        \let#1\gobbletwoarguments   \or
        \let#1\gobblethreearguments \or
        \let#1\gobblefourarguments  \or
        \let#1\gobblefivearguments  \or
        \let#1\gobblesixarguments   \or
        \let#1\gobblesevenarguments \or
        \let#1\gobbleeightarguments \or
        \let#1\gobbleninearguments  \or
        \let#1\gobbletenarguments   \fi}

  \def\resetspecials
    {\the\everyresetspecials
     \let\preloadedspecials\empty}

  \def\definespecial#1%
    {\@EA\chardef\csname\@@speclst@@\string#1\endcsname=1
     \def#1}

  \def\doifspecialavailableelse#1%
    {\ifcase\csname\@@speclst@@\string#1\endcsname
       \expandafter\secondoftwoarguments
     \else
       \expandafter\firstoftwoarguments
     \fi}

\fi

%D For quite some time the \CONTEXT\ way of specifying the
%D output format has been:
%D
%D \starttyping
%D \usespecials[ps,yy,win,pdf]
%D \stoptyping
%D
%D Because at \PRAGMA\ we use \DVIPSONE, this was a suitable
%D setting, but with \CONTEXT\ going public, the next sequence
%D is more suitable for \DVIPS\ users:
%D
%D \starttyping
%D \usespecials[reset,ps,tr,pdf]
%D \stoptyping
%D
%D On the other hand, for \PDFTEX\ we needed:
%D
%D \starttyping
%D \usespecials[tpd]
%D \stoptyping
%D
%D To simplify things, I decided to provide a higher level
%D command.
%D
%D \starttyping
%D \defineoutput[name][specials]
%D \setupoutput[name,...]
%D \stoptyping
%D
%D In a \type {spec-def} you can find some examples.

\def\defineoutput
  {\dodoubleargument\dodefineoutput}

\def\dodefineoutput[#1][#2]%
  {\setvalue{\??ui#1}{#2}}

\def\dosetupoutput#1%
  {\doifdefinedelse{\??ui#1}
     {\processcommacommand[\getvalue{\??ui#1}]\dousespecials}
     {\doifdefinedelse{\@@specfil@@#1}
        {\dousespecials{#1}}
        {\showmessage\m!specials7{#1}}}}

\def\setupoutput[#1]%
  {\ifnum\realpageno<2 % new
     \resetspecials\processcommacommand[#1]\dosetupoutput
   \fi}

\protect \endinput
