%D \module
%D   [       file=lang-ctx,
%D        version=2005.02.12,
%D          title=\CONTEXT\ Language Macros,
%D       subtitle=Generic Patterns,
%D         author=Hans Hagen,
%D           date=\currentdate,
%D      copyright={PRAGMA / Hans Hagen \& Ton Otten}]
%C
%C This module is part of the \CONTEXT\ macro||package and is
%C therefore copyrighted by \PRAGMA. See mreadme.pdf for
%C details.

\writestatus{loading}{Context Language Macros / Generic Patterns}

\unprotect

%D The \CONTEXT\ specific patterns are more generic and
%D are more or less encoding independent. They are generated
%D from the ones shipped with distributions using:
%D
%D \starttyping
%D ctxtools --pattern --all
%D \stoptyping
%D
%D Later we will provide utf specific patterns.

%D Todo:

\definefilesynonym [lang-ru.pat]  [ruenhyph.tex]
\definefilesynonym [lang-ua.pat]  [ukrenhyp.tex]
\definefilesynonym [lang-af.pat]  [lang-nl.pat]

%D In order to get 8 bit characters hyphenated, we need to load
%D patterns under the right circumstances. In some countries, more
%D than one font encoding is in use. I can add more defaults here
%D if users let me know what encoding they use.

\installlanguage [\s!nl] [\s!mapping={texnansi,ec},\s!encoding={texnansi,ec}]
\installlanguage [\s!fr] [\s!mapping={texnansi,ec},\s!encoding={texnansi,ec}]
\installlanguage [\s!de] [\s!mapping={texnansi,ec},\s!encoding={texnansi,ec}]
\installlanguage [\s!it] [\s!mapping={texnansi,ec},\s!encoding={texnansi,ec}]

\installlanguage [\s!hr] [\s!mapping=ec,\s!encoding=ec] % no il2, misses cacute characters

\installlanguage [\s!pl] [\s!mapping={pl0,ec},\s!encoding={pl0,ec}]
\installlanguage [\s!cz] [\s!mapping={il2,ec},\s!encoding={il2,ec}]
\installlanguage [\s!sk] [\s!mapping={il2,ec},\s!encoding={il2,ec}]
\installlanguage [\s!sl] [\s!mapping={il2,ec},\s!encoding={il2,ec}]

\protect \endinput
