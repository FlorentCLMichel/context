%D \module
%D   [       file=core-01c,
%D        version=1997.03.31,
%D          title=\CONTEXT\ Core Macros,
%D       subtitle=1C (to be split),
%D         author=Hans Hagen,
%D           date=\currentdate,
%D      copyright={PRAGMA / Hans Hagen \& Ton Otten}]
%C
%C This module is part of the \CONTEXT\ macro||package and is
%C therefore copyrighted by \PRAGMA. See licen-en.pdf for 
%C details. 

\writestatus{loading}{Context Core Macros (c)}

\unprotect

\startmessages  dutch  library: systems
     41: externe file -- in groep -- bestaat niet
\stopmessages

\startmessages  english  library: systems
     41: external file -- in group -- does not exist
\stopmessages

\startmessages  german  library: systems
     41: Externe Datei -- in Gruppe -- existiert nicht
\stopmessages

%I n=Formules+
%I c=\startlegenda,\startgegeven
%I
%I Ten behoeve van een consistente toelichting op een
%I formule zijn er de volgende commando's:
%I
%I \startlegenda
%I   \leg symbool \\ betekenis \\ dimensie \\
%I   \leg symbool \\ betekenis \\ dimensie \\
%I \stoplegenda
%I
%I \startgegeven
%I   \geg betekenis \\ symbool \\ waarde \\
%I   \geg betekenis \\ symbool \\ waarde \\
%I \stopgegeven
%I
%I Ten behoeve van het zetten van ub- en superscripts zijn
%I er, naast \hbox, de commando's \xbox en \xxbox.
%P
%I Het onderstaande mag ook:
%I
%I \startlegenda[twee]
%I   \leg symbool \\ symbool \\ betekenis \\ dimensie \\
%I   \leg symbool \\ symbool \\ betekenis \\ dimensie \\
%I \stoplegenda

% \newif\ifdoublelegends
% 
% \def\legendaspacing%
%   {\hskip.5em}
% 
% \def\dostartlegenda[#1]%
%   {\witruimte
%    \blanko
%    \bgroup
%    \doifelse{#1}{\v!twee}
%      {\doublelegendstrue
%       \let\leg=\doubleleg}
%      {\doublelegendsfalse
%       \let\leg=\singleleg}%
%    \tabskip=\!!zeropoint
%    \halign
%    \bgroup
%    \hskip\leftskip
%    $##$\hfil
%    &\hfil\legendaspacing##\legendaspacing\hfil
%    &\ifdoublelegends$##$\hfil\fi
%    &\ifdoublelegends\hfil\legendaspacing##\legendaspacing\hfil\fi
%    &##\unskip\hfil\legendaspacing\legendaspacing
%    &$\rm##$\hfill\cr}
% 
% \def\singleleg#1\\#2\\#3\\%
%   {#1&\doifsomething{#1}{=}&
%    &&
%    #2\unskip&
%    #3\cr}
% 
% \def\doubleleg#1\\#2\\#3\\#4\\%
%   {#1&\doifsomething{#1}{\doifnot{#1}{ }{=}}&
%    #2&\doifsomething{#2}{\doifnot{#2}{ }{=}}&
%    #3\unskip&
%    #4\cr}
% 
% \def\startlegenda%
%   {\dosingleempty\dostartlegenda}
% 
% \def\stoplegenda%
%   {\egroup
%    \egroup
%    \blanko}
% 
% % tzt: \crlf == \\ \\ \leg \\ afh kolom
% 
% \def\startgegeven%
%   {\witruimte
%    \blanko
%    \bgroup
%    \tabskip=\!!zeropoint
%    \halign
%    \bgroup
%    \hskip\leftskip##\unskip\hfil
%    &\legendaspacing\legendaspacing\hfil$##$&\hfil\legendaspacing##\legendaspacing\hfil&
%    $\rm##$\hfil\cr}
% 
% \def\stopgegeven%
%   {\egroup
%    \egroup
%    \blanko}
% 
% \def\geg#1\\#2\\#3\\%
%   {#1&
%    #2&=&
%    #3\cr}

\definetabulate
  [\e!legenda]      
  [|emj1|i1|mR|]    

\setuptabulate 
  [\e!legenda]
  [\c!eenheid=.75em,\c!binnen=\setquicktabulate\leg,EQ={=}]

\definetabulate
  [\e!legenda][\v!twee]
  [|emj1|emk1|i1|mR|]

\definetabulate
  [\e!gegeven]      
  [|R|ecmj1|i1mR|]    

\setuptabulate 
  [\e!gegeven]
  [\c!eenheid=.75em,\c!binnen=\setquicktabulate\geg,EQ={=}]

\unexpanded\def\xbox%
  {\bgroup\aftergroup\egroup\hbox\bgroup\tx\let\next=}

\unexpanded\def\xxbox%
  {\bgroup\aftergroup\egroup\hbox\bgroup\txx\let\next=}

% \def\mrm#1%
%   {$\rm#1$}

%I n=Combinaties
%I c=\startcombinatie,\stelcombinatiesin
%I
%I Er kunnen meerdere tabellen, figuren enz. worden
%I gecombineerd. Dit gebeurt met het commando:
%I
%I   \startcombinatie[n*m]
%I      {inhoud 1}{ondertitel 1}
%I      {inhoud 2}{ondertitel 2}
%I      .....
%I   \stopcombinatie
%I
%I Eventueel kan volstaan worden met [n]. Vier inhouden
%I kunnen bijvoorbeeld worden gecombineerd als:
%I
%I   [4*1] of [4]     vier inhouden naast elkaar
%I   [1*4]            vier inhouden onder elkaar
%I   [2*2]            inhouden twee aan twee onder elkaar
%P
%I Dit commando is goed te combineren met \plaats-commando's:
%I
%I   \plaatsfiguur[][]{}
%I     \startcombinatie[2]
%I       {\legefiguur}{a}
%I       {\legefiguur}{b}
%I     \stopcombinatie
%I
%I Rond \start-\stopcombineer hoeven geen {} te worden
%I geplaatst.
%I
%I De afstanden tussen de inhouden en de titels kunnen worden
%I ingesteld met:
%I
%I   \stelcombinatiesin[voor=,na=,tussen=,afstand=,
%I     uitlijnen=,hoogte=,breedte=]
%I
%I Waarbij de afstand betrekking heeft op de horizontale
%I afstand (een maat dus) en voor, na en tussen commando's
%I zijn (bijvoorbeeld \blanko).

%T n=combinaties
%T m=com
%T a=c
%T
%T \startcombinatie
%T   {?} {}
%T   {} {}
%T \stopcombinatie

\newcount\horcombinatie  % counter
\newcount\totcombinatie

\def\stelcombinatiesin%
  {\dodoubleargument\getparameters[\??co]}

% \long\def\dodostartcombinatie[#1*#2*#3]%
%   {\stelfractiesin
%      [\c!n=\v!passend,
%       \c!afstand=\@@coafstand]%
%    \global\horcombinatie=#1\relax
%    \global\totcombinatie=#2\relax
%    \multiply\totcombinatie by \horcombinatie
%    \long\def\docombinatie##1##2##3%               % ##3 gobbles spaces.
%      {\vbox
%         {\setbox0=\vbox
%            {\hbox{##1}}%
%          \doifemptyelse{##2}                      % Dit moet per se
%            {\setbox2=\box\voidb@x}                % \doifempty zijn!
%            {\setbox2=\vtop
%               {\hsize\wd0
%                \forgetall
%                \steluitlijnenin[\@@couitlijnen]%  % \raggedcenter
%                \begstrut##2\endstrut}}%
%          \vbox
%            {\forgetall % \stelwitruimtein[\v!geen]%
%             \box0\relax
%             \ifvoid2\relax
%             \else
%               \@@cotussen
%               \nointerlineskip  % recently added
%               \box2\relax
%             \fi}}%
%       \ifnum\totcombinatie>1\relax
%         \global\advance\totcombinatie by -1\relax
%         \global\advance\horcombinatie by -1\relax
%         \ifnum\horcombinatie=0\relax
%           \def\next%
%             {\cr
%              \noalign
%                {\forgetall %\stelwitruimtein[\v!geen]%
%                 \nointerlineskip
%                 \@@cona
%                 \@@covoor
%                 \vss
%                 \nointerlineskip}%
%              \global\horcombinatie=#1\relax
%              \docombinatie{##3}}%
%         \else
%           \def\next%
%             {&&&\hskip\@@coafstand
%              &\docombinatie{##3}}%
%         \fi
%       \else
%         \def\next%
%           {\cr
%            \egroup
%            ##3}%
%       \fi
%       \next}%
%    \tabskip=\!!zeropoint
%    \doifelse{\@@cobreedte}{\v!passend}
%      {\halign}
%      {\halign to \@@cobreedte}%
%    \bgroup&\hfil##\hfil&\tabskip\!!zeropoint \!!plus 1fill##\cr
%    \docombinatie}

\long\def\dodostartcombinatie[#1*#2*#3]%
  {\stelfractiesin
     [\c!n=\v!passend,
      \c!afstand=\@@coafstand]%
   \global\horcombinatie=#1\relax
   \global\totcombinatie=#2\relax
   \xdef\maxhorcombinatie{\the\horcombinatie}%
   \multiply\totcombinatie by \horcombinatie
   \tabskip=\!!zeropoint
   \doifelse{\@@cobreedte}{\v!passend}
     {\halign}
     {\halign to \@@cobreedte}%
   \bgroup&\hfil##\hfil&\tabskip\!!zeropoint \!!plus 1fill##\cr
   \docombinatie}

\def\docombinatie%
  {\dowithnextbox
     {\setbox0=\box\nextbox
      \dowithnextbox
        {\setbox2=\box\nextbox
         \dodocombinatie}
     \hbox}
  \hbox}

\def\dodocombinatie%
  {\vbox
     {\forgetall % \stelwitruimtein[\v!geen]%
      \vbox
        {\copy0}%
      \ifdim\ht2>\!!zeropoint\relax % beter dan \wd2, nu \strut mogelijk
        \@@cotussen
        \vtop
          {\nointerlineskip  % recently added
           \hsize\wd0
           \steluitlijnenin[\@@couitlijnen]%  % \raggedcenter
           \begstrut\unhbox2\endstrut}%
      \fi}%
   \ifnum\totcombinatie>1
     \global\advance\totcombinatie by -1
     \global\advance\horcombinatie by -1
     \ifnum\horcombinatie=0
       \def\next%
         {\cr\noalign
            {\forgetall %\stelwitruimtein[\v!geen]%
             \nointerlineskip
             \@@cona
             \@@covoor
             \vss
             \nointerlineskip}%
          \global\horcombinatie=\maxhorcombinatie\relax
          \docombinatie}%
     \else
       \def\next%
         {&&&\hskip\@@coafstand&\docombinatie}%
     \fi
   \else
     \def\next%
       {\cr\egroup}%
   \fi
   \next}

\def\complexdostartcombinatie[#1]%
  {\dodostartcombinatie[#1*1*]}

\def\simpledostartcombinatie%
  {\complexdostartcombinatie[2]}

\def\startcombinatie%
  {\bgroup
   \forgetall
   \doifelse{\@@cohoogte}{\v!passend}
     {\vbox}
     {\vbox to \@@cohoogte}%
   \bgroup
   \complexorsimple\dostartcombinatie}

\def\stopcombinatie%
  {\egroup
   \egroup}

%\def\plaatsnaastelkaar%
%  {\bgroup
%   \dowithnextbox
%     {\dowithnextbox
%         {\valign{\vss########\vss\cr\box0\cr\box\nextbox\cr}%
%          \egroup}
%      \vbox}
%   \vbox}

\def\plaatsondernaastelkaar#1#2%
  {\bgroup
   \def\plaatsonderelkaar%
     {#2\cr\omit\bgroup#2%
      \aftergroup#2%
      \aftergroup\cr
      \aftergroup\egroup
      \aftergroup\egroup
      \let\next=}%
   #1\bgroup##\cr
   \omit\bgroup#2%
   \aftergroup\plaatsonderelkaar
   \let\next=}

\def\plaatsonderelkaar%
  {\plaatsondernaastelkaar\halign\hss}

\def\plaatsnaastelkaar%
  {\plaatsondernaastelkaar\valign\vss}

%I n=Files
%I c=\definieerfile
%I
%I De onderstaande commando's zijn beschikbaar, maar nog in
%I ontwikkeling.
%I
%I   \gebruikexternefile  [groep] [naam] [file] [instellingen]
%I
%I   \gebruikexternefiles [groep] [korps=,file=]
%I   \stelexternefilesin  [groep] [korps=,file=]
%I
%I   \naam{naam} of \naam
%I
%I Standaard zijn gedefinieerd:
%I
%I   \gebruikexternefiles[pictex][korps=klein,file=pictex]
%I   \gebruikexternefiles[table][file=table]

\def\dogebruikexternefiles[#1][#2]%
  {\getparameters
     [\??fi#1]
     [\c!file=,
      \c!korps=,
      \c!optie=,
      #2]}

\def\gebruikexternefiles%
  {\dodoubleargument\dogebruikexternefiles}

\def\dostelexternefilesin[#1][#2]%
  {\doifundefinedelse{\??fi#1\c!file}
     {\gebruikexternefiles[#1][#2]}
     {\getparameters[\??fi#1][#2]}}

\def\stelexternefilesin%
  {\dodoubleargument\dostelexternefilesin}

\def\verwerkexternefile#1#2#3%
  {\bgroup
   \getparameters[\??fi#1][\c!file=,#3]%
   \doinputonce{\getvalue{\??fi#1\c!file}}%
   \ExpandFirstAfter\switchtobodyfont[\getvalue{\??fi#1\c!korps}]%
   \readsysfile{#2}  % beter: loc of fix gebied
     {}
     {\showmessage{\m!systems}{41}{#2,#1}}%
   \egroup}

\def\dogebruikexternefile[#1][#2][#3][#4]%
  {\stelexternefilesin[#1][]%
   \doinputonce{\getvalue{\??fi#1\c!file}}%
   \doifelsenothing{#2}
     {\setvalue{#3}{\verwerkexternefile{#1}{#3}{#4}}}
     {\setvalue{#2}{\verwerkexternefile{#1}{#3}{#4}}}}

\def\gebruikexternefile%
  {\doquadrupleargument\dogebruikexternefile}

%I n=Roteren
%I c=\roteer
%I
%I Er is een rotatiecommando beschikbaar:
%I
%I   \roteer[rotatie=]{}
%I
%I Waarbij als rotatie 0, 90, 180 of 270 kan worden opgegeven.
%I Als extra instellingen kunnen de instellingen van
%I \omlijnd worden meegegeven. Er wordt gebruik gemaakt van
%I het \special commando en het rotatiemechanisme van
%I PostScript. Dit betekent dat in de previewer de tekst
%I niet (!) geroteerd wordt. Overigens draagt TeX zorg voor
%I de exacte plaatsing, uitlijnen enz. De afhankelijkheid
%I van PostScript is dus tot een minimum beperkt.
%I
%I Verder zijn dezelfde instellingen mogelijk als bij
%I \omlijnd.

\presetlocalframed[\??ro]

\def\stelroterenin%
  {\dodoubleargument\getparameters[\??ro]}

% \ht, \vfillvoor, \vfillna, \wd, \hfillvoor, \hfillna

\def\dodostoproteer#1#2#3#4#5#6%
  {\dontshowcomposition
   \vbox to #1\nextbox
     {#2\relax
      \hbox to #4\nextbox
        {#5\relax
         \doifelsenothing{\@@rorotatie}
           {\dostartrotation{90}}
           {\dostartrotation{\@@rorotatie}}%
         \wd\nextbox=\!!zeropoint
         \ht\nextbox=\!!zeropoint
         \box\nextbox
         \dostoprotation
         #6}
      #3}%
   \egroup}

\def\dostoproteer%
  {\!!counta=\@@rorotatie
   \divide\!!counta by 90
   \ifcase\!!counta
     \dodostoproteer\ht\relax\vfill\wd\relax\hfill
   \or
     \dodostoproteer\wd\vfill\relax\ht\relax\hfill
   \or
     \dodostoproteer\ht\vfill\relax\wd\hfill\relax
   \or
     \dodostoproteer\wd\relax\vfill\ht\hfill\relax
   \or
     \dodostoproteer\ht\relax\vfill\wd\relax\hfill
   \else
     \def\@@rotatie{90}%
     \dodostoproteer\ht\relax\vfill\wd\relax\hfill
   \fi}

\def\dorotatebox#1% {angle} \hbox/\vbox/\vtop
  {\bgroup
   \hbox\bgroup % compatibility hack
   \dowithnextbox
     {\edef\@@rorotatie{#1}%
      \setbox\nextbox=\vbox{\box\nextbox}%
      \dostoproteer
      \egroup}}

\def\complexroteer[#1]%
  {\dowithnextbox
     {\getparameters[\??ro][#1]%
      \dostoproteer}%
   \vbox\localframed[\??ro][#1]}

\def\roteer%
  {\bgroup     % \roteer kan argument zijn
   \complexorsimpleempty\roteer}

% schaal

\def\doschaal[#1]%
  {\bgroup
   \forgetall
   \getparameters[\??xy][\c!sx=1,\c!sy=1,#1]%
   \dowithnextbox
     {\dontshowcomposition
      \dimen0=\@@xysy\ht\nextbox
      \dimen2=\@@xysy\dp\nextbox
      \dimen4=\@@xysx\wd\nextbox
      \dimen6=\dimen0\advance\dimen6 by \dimen2
      \setbox\nextbox=\vbox to \dimen6
        {\ht\nextbox=\!!zeropoint
         \dp\nextbox=\!!zeropoint
         \vfill % erbij
         \dostartscaling\@@xysx\@@xysy\box\nextbox\dostopscaling
         }% was \vfill}%
      \ht\nextbox=\dimen0
      \dp\nextbox=\dimen2
      \wd\nextbox=\dimen4
      \box\nextbox
      \egroup}
   \hbox} % was \vbox}

\def\schaal%
  {\dosingleempty\doschaal}

% mirror

\def\domirrorbox% \hbox/\vbox/\vtop
  {\bgroup
   \dowithnextbox
     {\dontshowcomposition
      \dimen0=\wd\nextbox
      \setbox\nextbox=\vbox
        {\dostartmirroring\hskip-\wd\nextbox\box\nextbox\dostopmirroring}%
      \wd\nextbox=\dimen0
      \box\nextbox
      \egroup}}

\def\spiegel%
  {\domirrorbox\hbox}

%\setbox0=\hbox{gans}
%
%\ruledhbox{\copy0 \schaal[sx=2,sy=2]{\copy0}}
%
%\spiegel{\ruledhbox{\copy0 \schaal{\box0}}}

% verdelen \hsize in fracties
%
% \fractie[n/m,elementen,afstand]
%
% \fractie[2/5,3,1em]
% \fractie[2/5,3,1em]
% \fractie[1/5,3,1em]
%
% \stelfractiesin[afstand=,aantal=]  (passend,passend)

\def\??fr{@@fr}

\def\stelfractiesin%
  {\dodoubleargument\getparameters[\??fr]}

\def\dodofractie[#1/#2,#3,#4,#5]%
  {\doifelsenothing{#3}
     {\doifelse{\@@frn}{\v!passend}
        {\!!counta=#2\relax}
        {\!!counta=\@@frn\relax}}
     {\!!counta=#3\relax}%
   \doifelsenothing{#4}
     {\doifelse{\@@frafstand}{\v!passend}
        {\!!widtha=\!!zeropoint}
        {\!!widtha=\@@frafstand}}
     {\!!widtha=#4}%
   \advance\!!counta by -1\relax
   \multiply\!!widtha by \!!counta
   \advance\hsize by -\!!widtha
   \divide\hsize by #2\relax
   \hsize=#1\hsize}

\def\dofractie[#1]%
  {\dodofractie[#1,,,,,,]}

\def\fractie%
  {\dosingleargument\dofractie}

\stelfractiesin
  [\c!afstand=\tfskipsize,
   \c!n=\v!passend]

% Standaardinstellingen

\stelroterenin
  [\c!rotatie=90,
   \c!breedte=\v!passend,
   \c!hoogte=\v!passend,
   \c!offset=\v!overlay,
   \c!kader=\v!uit]

\stelcombinatiesin
  [\c!breedte=\v!passend,
   \c!hoogte=\v!passend,
   \c!afstand=1em,
   \c!voor=\blanko,
   \c!tussen={\blanko[\v!middel]},
   \c!na=,
   \c!uitlijnen=\v!midden]

\gebruikexternefiles
  [pictex]
  [\c!korps=\v!klein,
   \c!file=pictex]

\gebruikexternefiles
  [table]
  [\c!file=table]

\protect

\endinput
