%D \module
%D   [       file=core-01c,
%D        version=1997.03.31,
%D          title=\CONTEXT\ Core Macros,
%D       subtitle=1C (to be split),
%D         author=Hans Hagen,
%D           date=\currentdate,
%D      copyright={PRAGMA / Hans Hagen \& Ton Otten}]
%C
%C This module is part of the \CONTEXT\ macro||package and is
%C therefore copyrighted by \PRAGMA. Non||commercial use is 
%C granted. 

\writestatus{loading}{Context Core Macros (c)}

\unprotect

\startmessages  dutch  library: figures
   title: figuren
      1: maten van -- geladen uit locale figuurfile
      2: maten van -- geladen uit globale figuurfile
      3: maten van -- geladen uit figuurfile zelf 
      4: figuur -- is niet te vinden
      5: er moet een nieuwe figuurfile worden aangemaakt
\stopmessages

\startmessages  english  library: figures
   title: figures
      1: dimensions of -- loaded from local figurefile
      2: dimensions of -- loaded from global figurefile
      3: dimensions of -- loaded from figurefile itself
      4: figure -- can not be found 
      5: you have to regenerate a new figure file 
\stopmessages

\startmessages  german  library: figures
   title: figures
      1: Dimensionen von -- geladen von lokaler Abbildungsdatei
      2: Dimensionen von -- geladen von globaler Abbildungsdatei
      3: Dimensionen von -- geladen von Abbildungsdatei selbst
      4: Abbildung -- kann nicht gefunden werden
      5: Sie haben eine neu Abbildungsdatei erstellt 
\stopmessages

\startmessages  dutch  library: systems
     41: externe file -- in groep -- bestaat niet
\stopmessages

\startmessages  english  library: systems
     41: external file -- in group -- does not exist 
\stopmessages

\startmessages  german  library: systems
     41: Externe Datei -- in Gruppe -- existiert nicht 
\stopmessages

% Een verdere 'besparing' kan worden gerealiseerd als de
% variabelen (ef#1...) lokaal blijven als een figuur wordt
% geladen. Ook kunnen in texutil.tuf \c!variabelen worden
% gebruikt.

%I n=Figuren+
%I c=\gebruikexternfiguur,\gebruikexternefiguren,\externfiguur
%I c=\figuurintekst
%I
%I Extern aangemaakte (postscript)-figuren kunnen in
%I de tekst worden opgenomen. De afmetingen van deze
%I figuren kunnen op verschillende manieren worden
%I ingesteld:
%I
%I   \gebruikexternfiguur[naam][file][schaal=,kader=]
%I
%I   \gebruikexternfiguur[naam][file][hfactor=,kader=]
%I   \gebruikexternfiguur[naam][file][bfactor=,kader=]
%I
%I   \gebruikexternfiguur[naam][file][hoogte=,kader=]
%I   \gebruikexternfiguur[naam][file][breedte=,kader=]
%I 
%I   \gebruikexternfiguur[eerste naam][file][....] 
%I   \gebruikexternfiguur[tweede naam][file][eerste naam]
%I
%I Externe figuren kunnen vervolgens worden opgeroepen op
%I naam:
%I
%I   \naam
%P
%I De schaal wordt opgegeven in tienden procenten. Dat
%I betekent dat 12.5 procent wordt opgegeven als 125.
%I
%I In plaats van een schaal kan ook een schaalfactor worden
%I opgegeven. Een factor 15 resulteert in een figuur met een
%I hoogte van 1.5 maal de korpsgrootte. In plaats van een
%I getal kan ook 'max', 'passend' of 'ruim' worden opgegeven.
%I
%I De breedte en hoogte worden opgegeven in de gebruikelijke
%I TeX-maten.
%I
%I Als men automatisch wil laten bepalen wat de maximale
%I hoogte of breedte is, dan kan men 'factor=max' meegeven.
%P
%I Externe figuren moeten worden voorgedefinieerd. Dit doet
%I men met het commando:
%I
%I   \presetfigure[file][type=,breedte=,hoogte=,
%I     omvang=,bron=,titel=]
%I
%I Met behulp van texutil kan een file 'texutil.tuf'
%I worden aangemaakt waarin de op het werkgebied beschikbare
%I figuren worden voorgedefinieerd. In dat geval kan het
%I commando \presetfigure achterwege blijven.
%I
%I De voorinstellingen worden geladen met het commando:
%I
%I   \gebruikexternefiguren[optie=,korps=,lokatie=]
%I
%I waarbij als opties kader en leeg mogelijk zijn. In
%I het eerste geval wordt een kader om de figuren
%I geplaatst, in het tweede geval worden alleen kenmerken
%I van de figuur weergegeven. Lokatie kan lokaal, globaal 
%I of {lokaal,globaal} (default) zijn. 
%P
%I Figuren kunnen op het actuele gebied staan, op een root
%I van het actuele gebied of op een nader in te stellen gebied.
%I Dit instellen gebeurt door aan 'gebied' een naam toe te
%I kennen.
%I
%I De in de file 'system.tex' opgenomen standaardinstelling
%I is (let op het gebruik van /):
%I
%I   \stelexternefigurenin[gebied=c:/figuren/eps,file=]
%P
%I Het eerder genoemde commando \gebruikexternfiguur ziet
%I er als volgt uit:
%I
%I   \gebruikexternfiguur[naam][file][instellingen]
%I
%I De naam is facultatief en is standaard gelijk aan de
%I filenaam. Als een (file)naam cijfers bevat, moet het
%I commando \naam{naam} worden gebruikt.
%I
%I Het is mogelijk in \plaatsfiguur direkt een externe
%I figuur op te roepen:
%I
%I   \plaatsfiguur{commentaar}{\naam}
%I
%I Ook is het mogelijk een figuur in de tekst op te nemen
%I met het commando:
%I
%I   \figuurintekst{\naam}

%T n=extern
%T m=ext
%T a=e
%T
%T \gebruikexternfiguur [] [?] [bfactor=]

%D The following macro's are derived from Thomas Rockicky's 
%D macro's. They are rewritten to a more compact form, made a 
%D bit more robust and also handle the \type{HiResBoundingBox} 
%D or \type{ExactBoundingBox} that sometimes is present.
%D
%D A bounding box has the form: 
%D
%D \starttypen
%D %%BoundingBox: llx lly urx ury 
%D \stoptypen
%D
%D Before we scan the file, we have to reset special 
%D characters and set some others. The percentage symbol also 
%D needs special treatment. When a bounding box is 
%D encountered, we keep on scanning until no more directives are
%D found, i.e. a line is found that does not start with a 
%D percentage symbol. We also abort scanning after finding a 
%D high resolution bounding box. 
%D
%D This method also works inside verbatim mode (like when we 
%D are typesettinmg sources and putting eps coded logos into 
%D a heading. Temporary restoring the \CATCODES\ is done in 
%D the calling routine.  

\def\dogetfiguresizeeps#1#2#3#4#5%
  {\bgroup
   \openjobin{\scratchread}{#1}%
   \ifeof\scratchread
     \doifsomething{\@@exgebied}
       {\closein\scratchread
        \openfixin{\scratchread}{\@@exgebied}{#1}}%
   \fi
   \global\chardef\epsffound=0
   \ifeof\scratchread
   \else
     \uncatcodespecials
     \def\next%
       {\read\scratchread to \epsffileline
        \ifeof\scratchread
          \let\next=\relax
        \else
          \expandafter\epsfaux\epsffileline:. \\%
        \fi
        \next}%
     \next
   \fi
   \closein\scratchread
   \egroup
   \ifnum\epsffound>0
     \dimen0=1bp\relax
     #2=\epsfllx\dimen0
     #4=\epsfurx\dimen0
     \advance#4 by -\epsfllx\dimen0
     #3=\epsflly\dimen0
     #5=\epsfury\dimen0
     \advance#5 by -\epsflly\dimen0
   \else
     #2=\!!zeropoint
     #3=\!!zeropoint
     #4=\!!zeropoint
     #5=\!!zeropoint
   \fi}

\bgroup
\catcode`\%=\@@other
\global\let\epsfpercent=%
\gdef\epsfboundingbox     {%BoundingBox}  
\gdef\epsfhiresboundingbox{%HiResBoundingBox}  
\gdef\epsfexactboundingbox{%ExactBoundingBox}  
\egroup  

\long\def\epsfaux#1#2:#3\\%
  {\ifx#1\epsfpercent
     \def\!!stringa{#2}%
     \ifx\!!stringa\epsfboundingbox
       \epsfgrab #3 . . . \\%
       \global\chardef\epsffound=1
     \else\ifx\!!stringa\epsfhiresboundingbox
       \epsfgrab #3 . . . \\%
       \global\chardef\epsffound=2
       \let\next=\relax
     \else\ifx\!!stringa\epsfexactboundingbox
       \epsfgrab #3 . . . \\%
       \global\chardef\epsffound=2
       \let\next=\relax
     \fi\fi\fi
   \else\ifnum\epsffound>0
     \let\next=\relax
   \fi\fi}

\def\epsfgrab #1 #2 #3 #4 #5\\%
  {\gdef\epsfllx{#1}%
   \ifx\epsfllx\empty
     \epsfgrab #2 #3 #4 #5 .\\%
   \else
     \gdef\epsflly{#2}%
     \gdef\epsfurx{#3}%
     \gdef\epsfury{#4}%
   \fi}

\edef\figureversion {1996.06.01}

\def\@@figureerrormessage%
  {\showmessage{\m!figures}{5}{}%
   \global\let\@@figureerrormessage=\relax}

\def\thisisfigureversion#1%
  {\doifnot{\figureversion}{#1}%
     {\@@figureerrormessage
      \endinput}}

\newcount \figxsca
\newcount \figysca

\newdimen \fighei
\newdimen \figwid

\newif    \iffigurefound
\newif    \iflocalfigures
\newif    \ifglobalfigures

\def\doreadlocalfigurefile%
  {\iflocalfigures
     \pushendofline
     \readlocfile{\@@exfile}{}{}%
     \popendofline
   \fi}   

\def\doreadglobalfigurefile%
  {\ifglobalfigures
     \doifsomething{\@@exgebied}
       {\pushendofline
        \readfixfile{\@@exgebied}{\@@exfile}{}{}%
        \popendofline}%
   \fi}

\def\complexgebruikexternefiguren[#1]%
  {\getparameters[\??ex][#1]%
   \ExpandFirstAfter\processallactionsinset
     [\@@exlokatie]
     [    \v!geen=>,
       \v!globaal=>\globalfigurestrue,
        \v!lokaal=>\localfigurestrue,
       \s!default=>\globalfigurestrue
                   \localfigurestrue]%
   \doifnothing{\@@exfile}
     {\globalfiguresfalse
      \localfiguresfalse}}

\def\gebruikexternefiguren%
  {\complexorsimpleempty{gebruikexternefiguren}}

\def\stelexternefigurenin%
  {\dodoubleargument\getparameters[\??ex]}

\presetlocalframed[\??ef]

\def\docalculatenorm#1#2#3#4#5%
  {\processaction
      [#2]
      [     \v!max=>#1=#4\relax,
        \v!passend=>#1=#5\relax,
           \v!ruim=>#1=#5\relax
                    \advance #1 by -4\@@exkorps\relax,
        \s!default=>\doifsomething{#3}{#1=#3\relax},
        \s!unknown=>#1=\@@exkorps\relax
                    \divide#1 by \!!ten\relax
                    \multiply#1 by #2\relax]}

\def\docalculatescales#1#2#3#4%
  {\dimen0=#1\relax                           % #1 = new 1-value
   \dimen2=#2\relax                           % #2 = old 1-value
   \divide\dimen2 by \!!thousand\relax
   \divide\dimen0 by \dimen2\relax
   \figxsca=\dimen0\relax                     %      x scale
   \figysca=\dimen0\relax                     %      y scale
   \dimen2=#4\relax                           % #4 = old 2-value
   \divide\dimen2 by \!!thousand\relax
   \multiply\dimen2 by \dimen0\relax
   #3=\dimen2\relax}                          % #3 = new 2-value

\def\docalculatescale#1#2#3%
  {\dimen0=#1\relax                           % #1 = new value
   \dimen2=#2\relax                           % #2 = old value
   \divide\dimen2 by \!!thousand\relax
   \divide\dimen0 by \dimen2\relax
   #3=\dimen0\relax}                          % #3 = schaal

\def\doapplyscale#1#2#3%
  {#1=#2\relax
   #3=\@@efschaal\relax
   \divide#1 by \!!thousand\relax
   \multiply#1 by #3\relax}

\def\dosetefsize%
  {\ifinner
     \teksthoogte=\vsize
     \dimen0=\teksthoogte
   \else
     \ifdim\pagegoal<\maxdimen
       \ifdim\pagetotal<\pagegoal
         \dimen0=\pagegoal
         \advance\dimen0 by -\pagetotal
       \else
         \dimen0=\teksthoogte
       \fi
     \else
       \dimen0=\teksthoogte
     \fi
   \fi  
   \doifelsenothing{\@@efhoogte}
     {\edef\@@efvsize{\the\dimen0}}
     {\let\@@efvsize=\@@efhoogte}%
   \doifelsenothing{\@@efbreedte}
     {\edef\@@efhsize{\the\hsize}}
     {\let\@@efhsize=\@@efbreedte}}

\def\berekenexternfiguur[#1][#2]%
  {\mindermeldingen
   \restorecatcodes  % recently added; we presume local use
   \beforesplitstring#2\at.\to\@@effilename
   \aftersplitstring #2\at.\to\@@efextension
   \getparameters
     [\??ep]
     [\c!e=\s!unknown,
      \c!w=15\korpsgrootte,
      \c!h=10\korpsgrootte,
      \c!x=\!!zeropoint,
      \c!y=\!!zeropoint,
      \c!t=,
      \c!s=,
      \c!a=,
      \c!f=\@@effilename]%
   \getparameters
     [\??ef]
     [\c!type=\s!unknown,
      \c!methode=\@@eftype,
      \c!preset=\v!ja,
      \c!preview=\v!nee,
      \c!schaal=\!!thousand,
      \c!factor=,
      \c!hfactor=,
      \c!bfactor=,
      \c!breedte=,
      \c!hoogte=,
      \c!achtergrond=,
      \c!achtergrondkleur=,
      \c!achtergrondraster=\@@rsraster,
      \c!hoek=,
      \c!straal=.5\korpsgrootte,
      \c!kader=\v!uit]%
   \getvalue{\??ef#1}%
   \def\@@effullname{#2.\@@eftype}% waarom per se nodig?
   \doifelse{\@@efpreset}{\v!nee}  % iets anders 
     {\figurefoundtrue}
     {\doifelse{#1}{\s!figurepreset}
        {\figurefoundtrue
         \let\@@eftype=\@@epe}
        {\figurefoundfalse
         \def\presetfigure[##1][##2]%
           {\iffigurefound
              \endinput
            \else
              \DOIF{#2}{##1}
                {\getparameters[\??ep][##2]%
                 \doifelse{\@@eftype}{\s!unknown}
                   {\let\@@eftype=\@@epe
                    \figurefoundtrue}
                   {\doif{\@@epe}{\@@eftype}
                      {\figurefoundtrue}}}%
            \fi}%
         \doreadlocalfigurefile
         \iffigurefound
           \showmessage{\m!figures}{1}{#2.\@@eftype}%
         \else
           \doreadglobalfigurefile
           \iffigurefound
             \showmessage{\m!figures}{2}{#2.\@@eftype}%
           \else
             \doif{\@@eftype}{\s!unknown}
               {\let\@@eftype=\c!eps}%
             \doifelse{\@@efextension}{}
               {\edef\@@effullname{\@@effilename.\@@eftype}}
               {\edef\@@effullname{\@@effilename.\@@efextension}}%
             \doiffileelse{\@@effullname}
               {\executeifdefined{dogetfiguresize\@@eftype}\gobblefivearguments
                  {\@@effullname}
                  {\!!widtha}
                  {\!!heighta}
                  {\!!widthb}
                  {\!!heightb}%
                \ifdim\!!widthb>\!!zeropoint\relax
                  \figurefoundtrue
                  \geteparameters % e !
                    [\??ep]
                    [\c!x=\the\!!widtha,
                     \c!y=\the\!!heighta,
                     \c!w=\the\!!widthb,
                     \c!h=\the\!!heightb]%
                \fi}
               {\figurefoundfalse}%
             \iffigurefound
               \showmessage{\m!figures}{3}{\@@effullname}%
             \else
               \showmessage{\m!figures}{4}{\@@effullname}%
               \def\@@efkader{\v!aan}%
             \fi
           \fi
         \fi}}%
   \let\@@epe=\@@eftype
   \doif{\@@exoptie}{\v!kader}
     {\doassign[\??ef][\c!kader=\v!aan]}%
   \figwid=\!!zeropoint
   \fighei=\!!zeropoint
   \doifinsetelse{\@@effactor}{\v!max,\v!passend,\v!ruim}
     {\dosetefsize
      \ifdim\@@epw>\@@eph\relax
        \docalculatenorm\figwid\@@effactor\@@efbreedte\hsize\@@efhsize
        \docalculatescales\figwid\@@epw\fighei\@@eph
      \else
        \docalculatenorm\fighei\@@effactor\@@efhoogte\teksthoogte\@@efvsize
        \docalculatescales\fighei\@@eph\figwid\@@epw
      \fi
      \!!doneatrue}
     {\doifinsetelse{\@@efhfactor}{\v!max,\v!passend,\v!ruim}
        {\dosetefsize
         \docalculatenorm\fighei\@@efhfactor\@@efhoogte\teksthoogte\@@efvsize
         \docalculatescales\fighei\@@eph\figwid\@@epw
         \!!doneatrue}
        {\doifinsetelse{\@@efbfactor}{\v!max,\v!passend,\v!ruim}
           {\dosetefsize
            \docalculatenorm\figwid\@@efbfactor\@@efbreedte\hsize\@@efhsize
            \docalculatescales\figwid\@@epw\fighei\@@eph
            \!!doneatrue}
           {\docalculatenorm\fighei\@@effactor\@@efhoogte\teksthoogte\@@efvsize
            \docalculatenorm\fighei\@@efhfactor\@@efhoogte\teksthoogte\@@efvsize
            \docalculatenorm\figwid\@@efbfactor\@@efbreedte\hsize\hsize
            \!!doneafalse}}}%
   \if!!donea
     \ifdim\figwid>\@@efhsize\relax
       \fighei=\!!zeropoint\relax
       \figwid=\@@efhsize\relax
     \else\ifdim\fighei>\@@efvsize\relax
       \fighei=\@@efvsize\relax
       \figwid=\!!zeropoint\relax
     \fi\fi
   \fi
   \ifdim\figwid>\!!zeropoint\relax
     \ifdim\fighei>\!!zeropoint\relax
       \docalculatescale\fighei\@@eph\figysca
       \docalculatescale\figwid\@@epw\figxsca
     \else
       \docalculatescales\figwid\@@epw\fighei\@@eph
     \fi
   \else
     \ifdim\fighei>\!!zeropoint\relax
       \docalculatescales\fighei\@@eph\figwid\@@epw
     \else
       \doapplyscale\figwid\@@epw\figxsca
       \doapplyscale\fighei\@@eph\figysca
     \fi
   \fi
   \doif{\@@exoptie}{\v!leeg}
     {\getparameters
        [\??ef]
        [\c!kader=\v!aan,
         \c!file=\v!leeg]}}

\def\convertinsertscale#1#2#3#4% 
  {\dimen0=#1\relax
   %\advance\dimen0 by .0005pt\relax
   \divide\dimen0 by \!!thousand
   \multiply\dimen0 by #3\relax
   \dimen0=-\dimen0  % beter hier - dan in driver 
   \edef#2{\number\dimen0}%
   \dimen0=#3pt\divide\dimen0 by \!!ten\relax
   \edef#4{\@EA\withoutpt\@EA{\the\dimen0}}}

\def\doplaatsexternfiguur[#1][#2]%
  {\gebruikexternefiguren
   \berekenexternfiguur[#1][#2]%
   \iffigurefound
     \localframed
       [\??ef]
       [\c!breedte=\figwid,
        \c!hoogte=\fighei,
        \c!offset=\v!overlay]
       {\vfilll
        \convertinsertscale\@@epx\figx\figxsca\scax
        \convertinsertscale\@@epy\figy\figysca\scay
        \doifelse{\@@efpreview}{\v!ja}
          {\def\@@efpreview{1}}
          {\def\@@efpreview{0}}%
        \doinsertfile
          {\@@eftype,\@@efmethode}{\@@effullname} %{\@@epf.\@@epe}
          {\scax}{\scay}
          {\figx}{\figy}
          {\number\figwid}{\number\fighei}
          {\@@efpreview}}%
   \else
     \localframed
       [\??ef]
       [\c!breedte=\figwid,
        \c!hoogte=\fighei,
        \c!kader=\v!aan]
       {#1 / #2}%
   \fi}

\def\dopresetfigure[#1][#2]%
  {\getparameters[\??ef][#1]%
   \getparameters[\??ep][#2]}

\def\doprecopfigure[#1][#2]%
  {\getvalue{\??ef#1}%
   \getparameters[\??ep][#2]}

\def\dosetgebruikexternfiguur[#1][#2][#3][#4]% 
  {\doifsomething{#3}
     {\doifinstringelse{=}{#3}
        {\setvalue{\??ef#1}{\dopresetfigure[#3][#4]}}
        {\setvalue{\??ef#1}{\doprecopfigure[#3][#4]}}}%
   \setvalue{#1}{\doplaatsexternfiguur[#1][#2]}}

\def\dogebruikexternfiguur[#1][#2][#3][#4]%
  {\doifelsenothing{#1}
     {\doifsomething{#2}
        {\dosetgebruikexternfiguur[#2][#2][#3][#4]}}
     {\doifelsenothing{#2}
        {\dosetgebruikexternfiguur[#1][#1][#3][#4]}
        {\dosetgebruikexternfiguur[#1][#2][#3][#4]}}}

\def\gebruikexternfiguur%
  {\doquadrupleempty\dogebruikexternfiguur}

\def\doexternalfigure[#1][#2]%
  {\bgroup
   \gebruikexternfiguur[#1][#1][#2]%
   \getvalue{#1}%
   \egroup}

\unexpanded\def\externalfigure%
  {\dodoubleempty\doexternalfigure}

\def\toonexternefiguren%
  {\bgroup
   \mindermeldingen
   \def\presetfigure[##1][##2]%
     {\gebruikexternfiguur
        [\s!figurepreset][##1]
        [\c!kader=\v!aan,\c!breedte=6cm][##2]%
      \startfiguurtekst[\v!links][]
        {\v!geen}
        {\hbox
           {\getvalue{\s!figurepreset}%
            \tfskip
            \framed[\c!breedte=\figwid,\c!hoogte=\fighei]{}}}%
        {\tfa ##1}%
        \blanko
        \tfx
        \def\docommando####1%
          {\beforesplitstring####1\at=\to\asciia
           \aftersplitstring####1\at=\to\asciib
           \doifsomething{\asciib}
             {\hsmash{\hbox to .75em{\asciia\hss}: \asciib}%
              \endgraf}}%
        \processcommalist[##2]\docommando
        \strut
        \endgraf
      \stopfiguurtekst}%
   \localfigurestrue
   \doreadlocalfigurefile
   \egroup}

%I n=Formules+
%I c=\startlegenda,\startgegeven
%I
%I Ten behoeve van een consistente toelichting op een
%I formule zijn er de volgende commando's:
%I
%I \startlegenda
%I   \leg symbool \\ betekenis \\ dimensie \\
%I   \leg symbool \\ betekenis \\ dimensie \\
%I \stoplegenda
%I
%I \startgegeven
%I   \geg betekenis \\ symbool \\ waarde \\
%I   \geg betekenis \\ symbool \\ waarde \\
%I \stopgegeven
%I
%I Ten behoeve van het zetten van ub- en superscripts zijn 
%I er, naast \hbox, de commando's \xbox en \xxbox. 
%P
%I Het onderstaande mag ook:
%I 
%I \startlegenda[twee]
%I   \leg symbool \\ symbool \\ betekenis \\ dimensie \\
%I   \leg symbool \\ symbool \\ betekenis \\ dimensie \\
%I \stoplegenda

\newif\ifdoublelegends

\def\dostartlegenda[#1]%
  {\witruimte
   \blanko
   \bgroup
   \doifelse{#1}{\v!twee}
     {\doublelegendstrue
      \let\leg=\doubleleg}
     {\doublelegendsfalse
      \let\leg=\singleleg}%
   \tabskip=\!!zeropoint
   \halign 
   \bgroup
   \hskip\leftskip
   $##$\hfil&\hfil~##~\hfil&
   \ifdoublelegends$##$\hfil\fi&\ifdoublelegends\hfil~##~\hfil\fi& 
   ##\unskip\hfil~~&
   $\rm##$\hfill\cr}

\def\singleleg#1\\#2\\#3\\%
  {#1&\doifsomething{#1}{=}&
   &&
   #2\unskip&
   #3\cr}

\def\doubleleg#1\\#2\\#3\\#4\\%
  {#1&\doifsomething{#1}{\doifnot{#1}{ }{=}}&
   #2&\doifsomething{#2}{\doifnot{#2}{ }{=}}&
   #3\unskip&
   #4\cr}

\def\startlegenda%
  {\dosingleempty\dostartlegenda}

\def\stoplegenda%
  {\egroup
   \egroup
   \blanko}

% tzt: \crlf == \\ \\ \leg \\ afh kolom 

\def\startgegeven%
  {\witruimte
   \blanko
   \bgroup
   \tabskip=\!!zeropoint
   \halign
   \bgroup
   \hskip\leftskip##\unskip\hfil&
   ~\hfil$##$&\hfil~##~\hfil&
   $\rm##$\hfil\cr}

\def\stopgegeven%
  {\egroup
   \egroup
   \blanko}

\def\geg#1\\#2\\#3\\%
  {#1&
   #2&=&
   #3\cr}

\unexpanded\def\xbox%
  {\bgroup\aftergroup\egroup\hbox\bgroup\tx\let\next=}

\unexpanded\def\xxbox%
  {\bgroup\aftergroup\egroup\hbox\bgroup\txx\let\next=}

% \def\mrm#1%
%   {$\rm#1$}

%I n=Combinaties
%I c=\startcombinatie,\stelcombinatiesin
%I
%I Er kunnen meerdere tabellen, figuren enz. worden
%I gecombineerd. Dit gebeurt met het commando:
%I
%I   \startcombinatie[n*m]
%I      {inhoud 1}{ondertitel 1}
%I      {inhoud 2}{ondertitel 2}
%I      .....
%I   \stopcombinatie
%I
%I Eventueel kan volstaan worden met [n]. Vier inhouden
%I kunnen bijvoorbeeld worden gecombineerd als:
%I
%I   [4*1] of [4]     vier inhouden naast elkaar
%I   [1*4]            vier inhouden onder elkaar
%I   [2*2]            inhouden twee aan twee onder elkaar
%P
%I Dit commando is goed te combineren met \plaats-commando's:
%I
%I   \plaatsfiguur[][]{}
%I     \startcombinatie[2]
%I       {\legefiguur}{a}
%I       {\legefiguur}{b}
%I     \stopcombinatie
%I
%I Rond \start-\stopcombineer hoeven geen {} te worden
%I geplaatst.
%I
%I De afstanden tussen de inhouden en de titels kunnen worden
%I ingesteld met:
%I
%I   \stelcombinatiesin[voor=,na=,tussen=,afstand=,
%I     uitlijnen=,hoogte=,breedte=]
%I
%I Waarbij de afstand betrekking heeft op de horizontale
%I afstand (een maat dus) en voor, na en tussen commando's
%I zijn (bijvoorbeeld \blanko).

%T n=combinaties
%T m=com
%T a=c
%T
%T \startcombinatie
%T   {?} {}
%T   {} {}
%T \stopcombinatie

\newcount\horcombinatie  % counter 
\newcount\totcombinatie

\def\stelcombinatiesin%
  {\dodoubleargument\getparameters[\??co]}

\long\def\dodostartcombinatie[#1*#2*#3]%
  {\stelfractiesin
     [\c!n=\v!passend,
      \c!afstand=\@@coafstand]%
   \global\horcombinatie=#1\relax
   \global\totcombinatie=#2\relax
   \multiply\totcombinatie by \horcombinatie
   \long\def\docombinatie##1##2##3%               % ##3 gobbles spaces.
     {\vbox
        {\setbox0=\vbox
           {\hbox{##1}}%
         \doifemptyelse{##2}                      % Dit moet per se 
           {\setbox2=\box\voidb@x}                % \doifempty zijn!
           {\setbox2=\vtop
              {\hsize\wd0
               \forgetall
               \steluitlijnenin[\@@couitlijnen]%  % \raggedcenter
               \begstrut##2\endstrut}}%
         \vbox
           {\forgetall % \stelwitruimtein[\v!geen]%
            \box0\relax
            \ifvoid2\relax  
            \else
              \@@cotussen
              \nointerlineskip  % recently added 
              \box2\relax
            \fi}}%
      \ifnum\totcombinatie>1\relax
        \global\advance\totcombinatie by -1\relax
        \global\advance\horcombinatie by -1\relax
        \ifnum\horcombinatie=0\relax
          \def\next%
            {\cr
             \noalign
               {\forgetall %\stelwitruimtein[\v!geen]%
                \nointerlineskip
                \@@cona
                \@@covoor
                \vss
                \nointerlineskip}%
             \global\horcombinatie=#1\relax
             \docombinatie{##3}}%
        \else
          \def\next%
            {&&&\hskip\@@coafstand
             &\docombinatie{##3}}% 
        \fi
      \else
        \def\next%
          {\cr
           \egroup
           ##3}%
      \fi 
      \next}%
   \tabskip=\!!zeropoint
   \doifelse{\@@cobreedte}{\v!passend}
     {\halign}
     {\halign to \@@cobreedte}% 
   \bgroup&\hfil##\hfil&\tabskip\!!zeropoint \!!plus 1fill##\cr
   \docombinatie}

\def\complexdostartcombinatie[#1]%
  {\dodostartcombinatie[#1*1*]}

\def\simpledostartcombinatie%
  {\complexdostartcombinatie[2]}

\def\startcombinatie%
  {\bgroup
   \forgetall
   \doifelse{\@@cohoogte}{\v!passend}
     {\vbox}
     {\vbox to \@@cohoogte}% 
   \bgroup
   \complexorsimple{dostartcombinatie}}

\def\stopcombinatie%
  {\egroup
   \egroup}

\def\plaatsonderelkaar%
  {\bgroup
   \dowithnextbox
     {\setbox0=\box\nextbox
      \dowithnextbox
         {\mindermeldingen
          \halign{\hss########\hss\cr\box0\cr\box\nextbox\cr}%
          \egroup}
      \hbox}
   \hbox}

\def\plaatsnaastelkaar%
  {\bgroup
   \dowithnextbox
     {\dowithnextbox
         {\valign{\vss########\vss\cr\box0\cr\box\nextbox\cr}% 
          \egroup}
      \vbox}
   \vbox}

%I n=Overlay
%I c=\startoverlay
%I
%I De onderstaande commando's zijn beschikbaar, maar nog in
%I ontwikkeling.
%I
%I \startoverlay
%I    {} {} {} 
%I \stopoverlay

\def\startoverlay%
  {\hbox\bgroup\futurelet\next\dogetoverlay}

\def\stopoverlay%
  {\unskip\egroup}

\def\dogetoverlay%
  {\ifx\next\bgroup
     \let\next=\dodogetoverlay
   \else
     \let\next=\relax
   \fi
   \next}

\def\dodogetoverlay%
  {\dowithnextbox{\unskip\copy\nextbox\hskip-\wd\nextbox}\hbox}

%I n=Files
%I c=\definieerfile
%I
%I De onderstaande commando's zijn beschikbaar, maar nog in
%I ontwikkeling.
%I
%I   \gebruikexternefile  [groep] [naam] [file] [instellingen]
%I
%I   \gebruikexternefiles [groep] [korps=,file=]
%I   \stelexternefilesin  [groep] [korps=,file=]
%I
%I   \naam{naam} of \naam
%I
%I Standaard zijn gedefinieerd:
%I
%I   \gebruikexternefiles[pictex][korps=klein,file=pictex]
%I   \gebruikexternefiles[table][file=table]

\def\dogebruikexternefiles[#1][#2]%
  {\getparameters
     [\??fi#1]
     [\c!file=,
      \c!korps=,
      \c!optie=,
      #2]}

\def\gebruikexternefiles%
  {\dodoubleargument\dogebruikexternefiles}

\def\dostelexternefilesin[#1][#2]%
  {\doifundefinedelse{\??fi#1\c!file}
     {\gebruikexternefiles[#1][#2]}
     {\getparameters[\??fi#1][#2]}}

\def\stelexternefilesin%
  {\dodoubleargument\dostelexternefilesin}

\def\verwerkexternefile#1#2#3%
  {\bgroup
   \getparameters[\??fi#1][\c!file=,#3]%
   \doinputonce{\getvalue{\??fi#1\c!file}}%
   \ExpandFirstAfter\switchtocorps[\getvalue{\??fi#1\c!korps}]%
   \readsysfile{#2}  % beter: loc of fix gebied 
     {}
     {\showmessage{\m!systems}{41}{#2,#1}}%
   \egroup}

\def\dogebruikexternefile[#1][#2][#3][#4]%
  {\stelexternefilesin[#1][]%
   \doinputonce{\getvalue{\??fi#1\c!file}}%
   \doifelsenothing{#2}
     {\setvalue{#3}{\verwerkexternefile{#1}{#3}{#4}}}
     {\setvalue{#2}{\verwerkexternefile{#1}{#3}{#4}}}}

\def\gebruikexternefile%
  {\doquadrupleargument\dogebruikexternefile}

%I n=Roteren
%I c=\roteer
%I
%I Er is een rotatiecommando beschikbaar:
%I
%I   \roteer[rotatie=]{}
%I
%I Waarbij als rotatie 0, 90, 180 of 270 kan worden opgegeven.
%I Als extra instellingen kunnen de instellingen van
%I \omlijnd worden meegegeven. Er wordt gebruik gemaakt van
%I het \special commando en het rotatiemechanisme van
%I PostScript. Dit betekent dat in de previewer de tekst
%I niet (!) geroteerd wordt. Overigens draagt TeX zorg voor
%I de exacte plaatsing, uitlijnen enz. De afhankelijkheid
%I van PostScript is dus tot een minimum beperkt.
%I
%I Verder zijn dezelfde instellingen mogelijk als bij 
%I \omlijnd. 

\presetlocalframed[\??ro]

\def\stelroterenin%
  {\dodoubleargument\getparameters[\??ro]}

% \ht, \vfillvoor, \vfillna, \wd, \hfillvoor, \hfillna 

\def\dodostoproteer#1#2#3#4#5#6%  
  {\vbox to #1\framebox
     {#2\relax
      \hbox to #4\framebox
        {#5\relax
         \doif{\@@rorotatie}{}%
           {\def\@@rorotatie{90}}%
         \dostartrotation{\@@rorotatie}%
         \wd\framebox=\!!zeropoint
         \ht\framebox=\!!zeropoint
         \box\framebox\relax
         \dostoprotation
         #6}%
      #3}}

\def\dostoproteer%
  {\egroup
   \!!counta=\@@rorotatie\relax
   \divide\!!counta by 90\relax
   \ifcase\!!counta
     \dodostoproteer\ht\relax\vfill\wd\relax\hfill
   \or
     \dodostoproteer\wd\vfill\relax\ht\relax\hfill
   \or
     \dodostoproteer\ht\vfill\relax\wd\hfill\relax
   \or
     \dodostoproteer\wd\relax\vfill\ht\hfill\relax
   \else
     \dodostoproteer\ht\relax\vfill\wd\relax\hfill
   \fi
   \egroup
   \egroup}

\def\complexroteer[#1]%
  {\bgroup 
   \getparameters[\??ro][#1]%
   \setbox\framebox=\vbox  
     \bgroup
     \localframed[\??ro][#1]%
     \bgroup
     \aftergroup\dostoproteer
     \let\next=}

\def\roteer%
  {\bgroup     % \roteer kan argument zijn 
   \complexorsimpleempty{roteer}}

% verdelen \hsize in fracties
%
% \fractie[n/m,elementen,afstand]
%
% \fractie[2/5,3,1em]
% \fractie[2/5,3,1em]
% \fractie[1/5,3,1em]
%
% \stelfractiesin[afstand=,aantal=]  (passend,passend)

\def\??fr{@@fr}

\def\stelfractiesin%
  {\dodoubleargument\getparameters[\??fr]}

\def\dodofractie[#1/#2,#3,#4,#5]%
  {\doifelsenothing{#3}
     {\doifelse{\@@frn}{\v!passend}
        {\!!counta=#2\relax}
        {\!!counta=\@@frn\relax}}
     {\!!counta=#3\relax}%
   \doifelsenothing{#4}
     {\doifelse{\@@frafstand}{\v!passend}
        {\!!widtha=\!!zeropoint}
        {\!!widtha=\@@frafstand}}
     {\!!widtha=#4}%
   \advance\!!counta by -1\relax
   \multiply\!!widtha by \!!counta
   \advance\hsize by -\!!widtha
   \divide\hsize by #2\relax
   \hsize=#1\hsize}

\def\dofractie[#1]%
  {\dodofractie[#1,,,,,,]}

\def\fractie%
  {\dosingleargument\dofractie}

\stelfractiesin
  [\c!afstand=\tfskipsize,
   \c!n=\v!passend]

% Standaardinstellingen

\stelexternefigurenin
  [\c!optie=,
   \c!korps=\korpsgrootte,
   \c!gebied=,
   \c!file=\f!utilityfilename.\f!figureextension,
   \c!straal=.5\korpsgrootte,
   \c!hoek=\v!recht,
   \c!lokatie=]

\stelroterenin
  [\c!rotatie=90,
   \c!breedte=\v!passend,
   \c!hoogte=\v!passend,
   \c!offset=\v!geen,
   \c!kader=\v!uit]

\stelcombinatiesin
  [\c!breedte=\v!passend,
   \c!hoogte=\v!passend,
   \c!afstand=1em,
   \c!voor=\blanko,
   \c!tussen={\blanko[\v!middel]},
   \c!na=,
   \c!uitlijnen=\v!midden]

\gebruikexternefiles
  [pictex]
  [\c!korps=\v!klein,
   \c!file=pictex]

\gebruikexternefiles
  [table]
  [\c!file=table]

\protect

\endinput
