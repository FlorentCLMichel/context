%D \module
%D   [       file=core-gen,
%D        version=1995.10.10,
%D          title=\CONTEXT\ Core Macros,
%D       subtitle=General,
%D         author=Hans Hagen,
%D           date=\currentdate,
%D      copyright={PRAGMA / Hans Hagen \& Ton Otten}]
%C
%C This module is part of the \CONTEXT\ macro||package and is
%C therefore copyrighted by \PRAGMA. See mreadme.pdf for 
%C details. 

\writestatus{loading}{Context Core Macros / General}

\unprotect

%D \macros
%D   {assigndimension,assignalfadimension}
%D
%D Hieronder worden enkele commando's gedefinieerd rond
%D toekenningen. Allereerst een commando om waarden aan
%D een \DIMENSION\ toe te kennen:
%D
%D \starttypen
%D \assigndimension
%D   {<waarde>|klein|middel|groot|-klein|-middel|-groot|geen}
%D   {\dimensie}
%D   {waarde klein}
%D   {waarde middel}
%D   {waarde groot}
%D \stoptypen
%D
%D Hierbij krijgt de \DIMENSION\ \type{\dimensie} een waarde 
%D afhankelijk van het meegegeven trefwoord.
%D
%D \startsmaller
%D \startregels
%D \type{(-)klein }\qquad (--) waarde klein
%D \type{(-)middel}\qquad (--) waarde middel
%D \type{(-)groot }\qquad (--) waarde groot
%D \type{geen     }\qquad 0pt
%D \type{waarde   }\qquad waarde
%D \stopregels
%D \stopsmaller
%D
%D Een trefwoord mag worden voorafgegaan door een \type{-}.
%D Deze macro toont een voorbeeld van het gebruik van
%D \type{\processaction} en constanten.
%D
%D Analoog aan het bovenstaande commando kennen we een
%D commando om waarden toe te kennen aan een macro:
%D
%D \starttypen
%D \assignalfadimension
%D   {<waarde>|klein|middel|groot|geen}
%D   {\macro}
%D   {waarde klein}
%D   {waarde middel}
%D   {waarde groot}
%D \stoptypen

\def\assigndimension#1#2#3#4#5%
  {\processaction
     [#1]
     [   \v!klein=>#2=#3,
        \v!middel=>#2=#4,
         \v!groot=>#2=#5,
          \v!geen=>#2=\zeropoint,
        -\v!klein=>#2=-#3,
       -\v!middel=>#2=-#4,
        -\v!groot=>#2=-#5,
       \s!unknown=>#2=#1]}

\def\assignalfadimension#1#2#3#4#5%
  {\processaction
     [#1]
     [   \v!klein=>\edef#2{#3},
        \v!middel=>\edef#2{#4},
         \v!groot=>\edef#2{#5},
          \v!geen=>\edef#2{0},
       \s!unknown=>\edef#2{#1}]}

%D De onderstaande implementatie is veel sneller, maar
%D tegelijkertijd ook veel lelijker. Omdat we deze macro
%D relatief weinig aanroepen laten we deze optimalisatie maar
%D achterwege. Bovendien kunnen oplossingen als deze de
%D hash||table aardig uitputten (\type{\doifdefined}).
%D
%D \starttypen
%D \edef\@@dimension{@@dim}
%D \edef\@@negdimension{\@@dimension-}
%D
%D \def\assigndimension#1#2#3#4#5%
%D   {\setvalue{\@@dimension\v!klein}{#3}%
%D    \setvalue{\@@dimension\v!middel}{#4}%
%D    \setvalue{\@@dimension\v!groot}{#5}%
%D    \setvalue{\@@dimension\v!geen}{\!!zeropoint}%
%D    \setvalue{\@@negdimension\v!klein}{-#3}%
%D    \setvalue{\@@negdimension\v!middel}{-#4}%
%D    \setvalue{\@@negdimension\v!groot}{-#5}%
%D    \setvalue{\@@negdimension\v!geen}{\!!zeropoint}%
%D    \doifdefinedelse{\@@dimension#1}
%D      {#2=\getvalue{\@@dimension#1}}
%D      {#2=#1}}
%D \stoptypen
%D
%D Let's give this a try: 

\let\nopv!geen     \v!geen
\let\posv!groot    \v!groot 
\let\posv!middel   \v!middel
\let\posv!klein    \v!klein 
\edef\negv!groot {-\v!groot }
\edef\negv!middel{-\v!middel}
\edef\negv!klein {-\v!klein }

%\def\assigndimension#1#2#3#4#5%
%  {\edef\!!stringa{#1}%
%   \ifx\!!stringa\nopv!geen  #2=\zeropoint\else
%   \ifx\!!stringa\posv!groot #2=#5\else
%   \ifx\!!stringa\posv!middel#2=#4\else
%   \ifx\!!stringa\posv!klein #2=#3\else
%   \ifx\!!stringa\negv!groot #2=-#5\else
%   \ifx\!!stringa\negv!middel#2=-#4\else
%   \ifx\!!stringa\negv!klein #2=-#3\else
%                             #2=#1\fi\fi\fi\fi\fi\fi\fi}
%
%\def\assignalfadimension#1#2#3#4#5%
%  {\edef\!!stringa{#1}%
%   \ifx\!!stringa\posv!groot \edef#2{#5}\else
%   \ifx\!!stringa\posv!middel\edef#2{#4}\else
%   \ifx\!!stringa\posv!klein \edef#2{#3}\else
%   \ifx\!!stringa\nopv!geen  \edef#2{0}\else
%                             \edef#2{#1}\fi\fi\fi\fi}

\def\assigndimension#1#2#3#4#5%
  {\edef\!!stringa{#1}%
   #2=\ifx\!!stringa\nopv!geen   \zeropoint\else
      \ifx\!!stringa\posv!groot  #5\else
      \ifx\!!stringa\posv!middel #4\else
      \ifx\!!stringa\posv!klein  #3\else
      \ifx\!!stringa\negv!groot -#5\else
      \ifx\!!stringa\negv!middel-#4\else
      \ifx\!!stringa\negv!klein -#3\else
                                 #1\fi\fi\fi\fi\fi\fi\fi}

\def\assignalfadimension#1#2#3#4#5%
  {\edef\!!stringa{#1}%
   \edef#2{\ifx\!!stringa\posv!groot #5\else
           \ifx\!!stringa\posv!middel#4\else
           \ifx\!!stringa\posv!klein #3\else
           \ifx\!!stringa\nopv!geen   0\else
                                     #1\fi\fi\fi\fi}}

%D \macros
%D   {assignvalue}
%D
%D Een variant hierop is het commando:
%D
%D \starttypen
%D \assignvalue
%D   {<waarde>|klein|middel|groot}
%D   {\macro}
%D   {waarde klein }
%D   {waarde middel}
%D   {waarde groot}
%D \stoptypen
%D
%D Hierbij krijgt \type{\macro} een waarde afhankelijk van
%D het meegegeven trefwoord:
%D
%D \startsmaller
%D \startregels
%D \type{klein }\qquad waarde klein
%D \type{middel}\qquad waarde middel
%D \type{groot }\qquad waarde groot
%D \type{waarde}\qquad waarde
%D \stopregels
%D \stopsmaller
%D
%D Hier doet \type{geen} dus niet mee.

\def\assignvalue#1#2#3#4#5%
  {\processaction
     [#1]
     [   \v!klein=>\edef#2{#3},
        \v!middel=>\edef#2{#4},
         \v!groot=>\edef#2{#5},
       \s!unknown=>\edef#2{#1}]}

%D Or faster: 

%\def\assignvalue#1#2#3#4#5%
%  {\edef\!!stringa{#1}%
%   \ifx\!!stringa\posv!groot \edef#2{#5}\else
%   \ifx\!!stringa\posv!middel\edef#2{#4}\else
%   \ifx\!!stringa\posv!klein \edef#2{#3}\else
%                             \edef#2{#1}\fi\fi\fi}

\def\assignvalue#1#2#3#4#5%
  {\edef\!!stringa{#1}%
   \edef#2{\ifx\!!stringa\posv!groot #5\else
           \ifx\!!stringa\posv!middel#4\else
           \ifx\!!stringa\posv!klein #3\else
                                     #1\fi\fi\fi}}

%D \macros
%D   {assignwidth}
%D
%D Een breedte van een opgegeven tekst kan worden berekend en
%D toegekend aan een \DIMENSION\ met:
%D
%D \starttypen
%D \assignwidth
%D   {\dimensie}
%D   {<waarde>|passend|ruim}
%D   {tekst}
%D \stoptypen
%D
%D Dit commando sluit, evenals de bovenstaande
%D \type{\assign}||commando's, aan op de wijze waarop
%D in de andere \CONTEXT||modules toekenningen
%D plaatsvinden. Bij \type{ruim} wordt de gemeten breedte
%D met 1~em vermeerderd.

\def\assignwidth#1#2#3#4%
  {\doifelse{#2}{}
     {\setbox0=\hbox{#3}%
      #1=\wd0}
     {\doifinsetelse{#2}{\v!passend,\v!ruim}
        {\setbox0=\hbox{#3}%
         #1=\wd0
         \doif{#2}{\v!ruim}
           {\advance#1 by #4}}%
        {#1=#2}}}%

\protect

\endinput
