%D \module
%D   [       file=core-tab,
%D        version=1997.10.10,
%D          title=\CONTEXT\ Core Macros,
%D       subtitle=\TABLE\ Embedding,
%D         author=Hans Hagen,
%D           date=\currentdate,
%D      copyright={PRAGMA / Hans Hagen \& Ton Otten}]
%C
%C This module is part of the \CONTEXT\ macro||package and is
%C therefore copyrighted by \PRAGMA. See mreadme.pdf for
%C details.

\writestatus{loading}{Context Core Macros / TaBlE Embedding}

% By now it makes more sense to merge the patches into the original
% and clean that one up too.

% Don't change the splitter:
%
% ... \NR
% \TABLEnoalign{\page}\TABLEhead
% \NC ...

% e-tex: reverse rows or vadjust or ... in tables
% \ifalign
% \xhrule : calls for 'special' with width
% BUG:
%
% \starttable[|l|l|]
% \HL
% \RL\FR \VL Head 1 \VL Head 2 \VL\FR
% \RL\LR \VL Head A \VL Head B \VL\LR  % niet grijs ??
% \HL
% \VL 1 \VL 2 \VL\FR
% \VL a \VL b \VL\LR
% \HL
% \stoptable

% melden als in kleur conflict, uitgestelde test op \SR\SR

% verengelsen
% interface

% footnotes flushen
% \......TABLE........ namen
% kolommen testen
% unbreakable kop definieren
% voetnoten
% meldingen
% als direct \use{max} dan fout
% \BREAKPOINT
% breedte lijn telt
% errors: ook gray in handle

% \AR -> als in DL dan \DR

% nieuw:
%
% \NL / \NL[blanko] is skip, nog default?
% geen \HL in a row
% \HL[n]
% \VL[n] + remembers
% c{colorspec} key
% \HC[color][width]
% \VC[color]
% meldingen row, column, use, advise
% \AR: UITSTELLEN / EXPERIMENTEEL

% WAARDELOZE ERROR HANDLER
% THIS RENEWED MODULE WORKS OK BUT STILL LOOKS BAD

%D We felt no need to write our own table building macros,
%D simply because Michael Wichura made a terrific one. This
%D package is quite complete and well documented. In \CONTEXT\
%D we provide a shell for consistent spacing as well as color
%D support. Implementing these features without adapting the
%D original macros is not trivial. One easilly gets conflicts
%D with \type{\omit}, \type{\span} and \type{\noalign}, which
%D means that we end up postponing and overloading macros,
%D mostly global. Now, let's start with loading the main
%D macros:

\doifundefined{BeginTable}{\doinputonce{table.tex}}

\unprotect

%D \macros
%D   {inintable, ifsplittables}
%D
%D First we declare some variables. These show a bit what we
%D are dealing with. First we introdoce some booleans that
%D enable us, inside as well as outside this module, to
%D determine in what mode we are.

\newif\ifintable
\newif\ifsplittables

%D \macros
%D   {tracetablestrue}
%D
%D When I documented this module, I felt the need for tracing
%D options. After implementing this feature, I also added
%D warnings, error recovery and automatic spacing.

\newif\iftracetables

%D We show this feature in an eample that also shows some of
%D the basic table typesetting commands.
%D
%D \startbuffer
%D \starttable[|||]
%D \HL
%D \VL first \VL second \VL\AR
%D \HL
%D \VL alfa  \VL 1      \VL\AR
%D \VL beta  \VL 2      \VL\AR
%D \VL gamma \VL 3      \VL\AR
%D \HL
%D \stoptable
%D \stopbuffer
%D
%D \startcombination
%D   {\tracetablesfalse\getbuffer} {\type{\tracetablesfalse}}
%D   {\tracetablestrue\getbuffer}  {\type{\tracetablestrue}}
%D \stopcombination
%D
%D This table is specified as:
%D
%D \typebuffer
%D
%D This examples shows about the minimum of commands needed to
%D typeset such a table. In this table, the \type {\AR} is
%D automatically translated into the more primitive (but more
%D verbose) commands \type {\SR}, \type {\FR}, \type {\MR} and
%D \type {\LR} commands.
%D
%D \startbuffer
%D \starttables[|||]
%D \HL
%D \VL first \VL second \VL\AR
%D \HL
%D \VL alfa  \VL 1      \VL\AR
%D \VL beta  \VL 2      \VL\AR
%D \VL gamma \VL 3      \VL\AR
%D \HL
%D \stoptable
%D \stopbuffer
%D
%D When we use the split table feature, we get a bit more
%D information.
%D
%D {\tracetablesfalse\getbuffer}
%D
%D Sometimes in tables information shows up that is not typed
%D in by the user. These messages give a cue in what aspect a
%D table definition is wrong.
%D
%D \startbuffer
%D \starttable[||||]
%D \HL
%D \VL first     second    \VL third \VL\AR
%D \HL
%D \VL alfa  \VL 1         \VL a     \VL\AR
%D \VL beta  \VL 2         \VL b     \VL
%D \VL gamma \VL \THREE{3}     c     \VL\AR
%D \HL
%D \stoptable
%D \stopbuffer
%D
%D \typebuffer
%D
%D Those terrible table has three errors, which all show up in
%D typeset messages. Errors cannot always recovered 100\% and
%D therefore can result in two or more succesive messages, like
%D in the last row.
%D
%D \getbuffer

%D Bringing color into tables is complicated by the mere fact
%D that color is not part of \TEX. The main complication is
%D that we don't know in advance how wide a column will be. I
%D implemented color support in tables in the early 90's
%D because I needed it for some articles on color. I have to
%D admit that I seldom use the mechanism.
%D
%D Most color support in \CONTEXT\ makes use of colored rules.
%D At first sight, one is tempted to implement colors in tables
%D in a similar way, but as said, we don't know the dimensions
%D in advance. It turns out however that we don't have to,
%D simply because alignments take care of stretching rules to
%D the appropritate dimensions. This means that we can provide
%D backgrounds by coloring rules with the height of a row,
%D skipping upwards and finally drawing the content, like in:
%D
%D \gdef\ShowExample
%D   {\startfiguretext
%D      {none}
%D      {\getbuffer}
%D    \typebuffer
%D    \stopfiguretext}
%D
%D \startbuffer
%D \starttable[|c|c|]
%D \HL
%D \BL[2]               \SR
%D \VL test \VL test \VL\SR
%D \HL
%D \VL test \VL test \VL\FR
%D \VL test \VL test \VL\MR
%D \VL test \VL test \VL\LR
%D \HL
%D \stoptable
%D \stopbuffer
%D
%D \ShowExample
%D
%D Just to be complete we show how the other columns can be
%D given a background. Later we will provide more details over
%D the commands used.
%D
%D \startbuffer
%D \starttable[|c|c|c|]
%D \HL
%D \BL[3]                        \SR
%D \VL test \VL test \VL test \VL\SR
%D \HL
%D \stoptable
%D \stopbuffer
%D
%D \ShowExample
%D
%D \startbuffer
%D \starttable[|c|c|c|]
%D \HL
%D \BC      \BL[2]               \SR
%D \VL test \VL test \VL test \VL\SR
%D \HL
%D \stoptable
%D \stopbuffer
%D
%D \ShowExample
%D
%D \startbuffer
%D \starttable[|c|c|c|]
%D \HL
%D \BC      \BC      \BL         \SR
%D \VL test \VL test \VL test \VL\SR
%D \HL
%D \stoptable
%D \stopbuffer
%D
%D \ShowExample
%D
%D \startbuffer
%D \starttable[|c|c|c|]
%D \HL
%D \BC      \BL                  \SR
%D \VL test \VL test \VL test \VL\SR
%D \HL
%D \stoptable
%D \stopbuffer
%D
%D \ShowExample
%D
%D \startbuffer
%D \starttable[|c|c|c|]
%D \BL               \BL         \SR
%D \HL
%D \VL test \VL test \VL test \VL\SR
%D \HL
%D \stoptable
%D \stopbuffer
%D
%D \ShowExample

%D In these examples we can clearly see that for being a real
%D background, the color or gray specification has to precede
%D the content. Just to keep things simple, we can recall this
%D specification later on:
%D
%D \startbuffer
%D \starttable[|c|c|c|]
%D \BC      \BL                  \SR
%D \HL
%D \VL test \VL test \VL test \VL\SR
%D \HL
%D                            \BR\FR
%D \VL test \VL test \VL test \VL\FR
%D                            \BR\MR
%D \VL test \VL test \VL test \VL\MR
%D                            \BR\LR
%D \VL test \VL test \VL test \VL\LR
%D \HL
%D \stoptable
%D \stopbuffer
%D
%D \ShowExample
%D
%D Close study learns that we can put the specification
%D before or after the \type{\HL}, whatever suits best. Keeping
%D track of these specifications is taken care of by the next
%D variables:

\newif   \ifTABLEgrayline    % executing gray line
\newif   \ifTABLEgraydone    % gray line executed
\newtoks \TABLEgraytoks      % gray line specification

\newif\ifTABLEinbreak

%D Nog vervangen:

\def\c!Table{Table}
\def\m!TABLE{TABLE}

%D We already saw that the table macros report errors and
%D provide automatic spacing. These features can only be
%D implemented by keeping track of the state, often the last
%D command on a row.

\chardef\TABLEunknown          = 0

\chardef\TABLEseparaterow      = 1
\chardef\TABLEfirstrow         = 2
\chardef\TABLEmidrow           = 3
\chardef\TABLElastrow          = 4
\chardef\TABLErule             = 5
\chardef\TABLEskip             = 6
\chardef\TABLEautorow          = 7

\chardef\TABLEforcefirstrow    = 1
\chardef\TABLEforcelastrow     = 2

\chardef\TABLEmissingrow       = 1
\chardef\TABLEmissingcolumn    = 2
\chardef\TABLEspanoverflow     = 3
\chardef\TABLEdivisionoverflow = 4

%D We store these states using efficient \type {\chardef}'s.
%D Like most variables, these are global ones. When needed,
%D especially when we flush the backgrounds, we can temporary
%D disable the assignment.

\newif\ifsetTABLEaction

\def\setTABLEaction#1%
  {\ifsetTABLEaction\global\chardef\TABLEaction#1\fi}

\def\setTABLEforce#1%
  {\ifsetTABLEaction\global\chardef\TABLEforce#1\fi}

\def\setTABLEerror#1%
  {\global\chardef\TABLEerror#1}

%D Before we come to using these variables, we redefine and/or
%D adapt some \TABLE\ macros. Within \TABLE's the \type{|} and
%D \type{"} have special meanings in templates and are active
%D during. Their meaning can therefore conflict with those
%D elsewhere defined. To be compatible with traditional \TABLE\
%D as well as \CONTEXT's \type{||} and the active \type{"}
%D extensions for my german friends, we do some catcode magic.

\newif\ifForgetTableBarAndQuote \ForgetTableBarAndQuotetrue

% \bgroup

% \catcode`\|=\@@active
% \catcode`\"=\@@active
%
% \gdef\pushouterbarandquote
%    {\ifForgetTableBarAndQuote
%       \ifnum\catcode`\|=\@@active \let\outertablebar  |\else\let\outertablebar  \relax\fi
%       \ifnum\catcode`\"=\@@active \let\outertablequote"\else\let\outertablequote\relax\fi
%       \let|\letterbar
%       \let"\letterdoublequote
%     \fi}
%
% \gdef\popouterbarandquote
%   {\ifForgetTableBarAndQuote
%      \ifx\outertablebar  \relax\else\let|\outertablebar  \fi
%      \ifx\outertablequote\relax\else\let"\outertablequote\fi
%    \else
%      \redefinetablebarandquote
%    \fi}
%
% \egroup
%
% \def\ObeyTableBarAndQuote
%   {\ForgetTableBarAndQuotefalse
%    \ifintable
%      \redefinetablebarandquote
%    \fi}

\let\ActivateBarAndQuote \relax
\let\ObeyTableBarAndQuote\relax
\let\pushouterbarandquote\relax
\let\popouterbarandquote \relax

%D \macros
%D   {ObeyTableBarAndQuote}
%D
%D As said, the \type{|} and \type{"} active characters are
%D often used for other purposes. By default, the outside
%D meanings are therefore preserved and available inside
%D tables. If for some reason one wants to use the \TABLE\
%D primitives, one can say:
%D
%D \starttyping
%D \ObeyTableBarAndQuote
%D \stoptyping
%D
%D To keep things verbose, as well as to show what \TABLE\
%D commands we affect, we show some meanings.

\def\normalTABLEshortrule  {\!ttShortHrule} % \-
\def\normalTABLElongrule   {\!ttLongHrule} % \=
\def\normalTABLEfullrule   {\!ttFullHrule} % \_
\def\normalTABLEendofrow   {\!ttEndOfRow} % \\
\def\normalTABLEsimplebar  {\unskip\!ttRightGlue&&} % |
\def\normalTABLEcomplexbar {\unskip\!ttRightGlue&\omit\!ttAlternateVrule} % \|
\def\normalTABLEquote      {\unskip\!ttRightGlue&\omit&} % "
\def\normalTABLElineformat {\normalTABLEendofrow+}
\def\normalTABLElineending {\normalTABLEendofrow0 }
\def\normalTABLEsinglerule {&\normalTABLElongrule&}
\def\normalTABLEmultirule#1{&\use{#1}\normalTABLElongrule&}

%D The next hack is dedicated to Tobias, who found out that
%D paragraph entries don't break well.

\def\TABLEhack{\hskip\zeropoint}

%D The first attemp to solve this problem was:
%D
%D \starttyping
%D \def\normalTABLEquote%
%D   {\unskip\TABLEhack\!ttRightGlue&\omit&\TABLEhack}
%D \stoptyping
%D
%D But, as usual, this interfered with \type {\omit}.
%D
%D The next attempt is redefining some core \TABLE\ macro:.
%D This works ok, but breaks for instance the~\type{b}
%D key handling.
%D
%D \starttyping
%D \def\!tfAdjoinPriorColumn%
%D   {\ifnum\!taColumnNumber=0
%D      \!taPreamble=\!taRuleColumnTemplate
%D    ...
%D    \if!taOnceOnlyTabskip
%D      \!thToksEdef\!taDataColumnTemplate=
%D        {\TABLEhack####\TABLEhack\tabskip\the\!taLastRegularTabskip}
%D    \else
%D      \!taDataColumnTemplate{\TABLEhack##\TABLEhack}%
%D    \fi
%D    ...
%D    \ReadFormatKeys}
%D \stoptyping

% \newdimen\TABLEparheight

\def\BeginTableParBox#1%
  {\setbox\scratchbox\vtop\bgroup % \setbox added
     \hsize#1\relax
     \normalbaselines
     \let~\!ttTie
     \let\-\!ttDH
     \blank[\v!disable]% % added
     \the\EveryTableParBox}

\def\EndTableParBox
  {\removelastskip                                   % itemize or so
   \endgraf
   \ifnum\prevgraf>\zerocount                        % we want at least
     \verticalstrut \nowhitespace \vskip-\struttotal% one line of text
     \egroup
     \ifdim\dp\scratchbox>\lineheight                % see (*) for an
       \getnoflines{\dp\scratchbox}%                 % example of where
       \dp\scratchbox\zeropoint                      % saving can go
       \setbox\scratchbox                            % terrible wrong
         \vtop to \noflines\lineheight{\box\scratchbox}%
     \fi                                             % esp between rows
   \else                                             % of paragraphs
     \egroup
   \fi
%    \getboxheight\scratchdimen\of\box\scratchbox\relax% compensate for
%    \ifdim\scratchdimen>\TABLEparheight               % funny depth of
%      \global\TABLEparheight\scratchdimen             % multi-line box
%    \fi                                               % i.e. vtop
   \box\scratchbox}

% We also need to patch away the interfering math switch:

% \mathpunctuationtrue

% test, test
% \starttable[|c|]
% \NC1,,10\NC\AR
% \stoptable
% test, test

\def\!ttBeginTableA[#1]{%
  \if #1u%                  % "unboxed" table
    \ifmmode
      \def\!ttEndTable{%    % user had better be in display math mode
        \relax}%            %   and have only one table at the outer level
    \else                   % user had better be in vertical mode
      \bgroup
      \def\!ttEndTable{%
        \egroup}%
    \fi
  \else
    %\hbox\bgroup $
    %\def\!ttEndTable{%
    %  \egroup %   for the \vtop, \vbox, or \vcenter, yet to come
    %  $%          for math mode
    %  \egroup}%   for the \hbox
    %\if #1t%
    %  \vtop
    %\else
    %  \if #1b%
    %    \vbox
    %  \else
    %    \vcenter % math mode was essential for this
    %  \fi
    %\fi
    %
    \hbox\bgroup
    \def\!ttEndTable{\egroup\egroup}%
    \if#1t%
      \vtop
    \else\if#1b%
      \vbox
    \else
      \def\!ttEndTable{\egroup$\egroup}%
      %$\vcenter
      \scratchtoks\everymath\everymath\emptytoks$\everymath\scratchtoks\vcenter
    \fi\fi
    %
    \bgroup % for the \vtop, \vbox, or \vcenter
  \fi
  \advance\!taRecursionLevel 1 % RecursionLevel governs initialization
  \let\!ttRightGlue=\relax  % This may be changed by \JustCenter, etc
  \everycr\emptytoks % ={}
  \ifnum \!taRecursionLevel=1
    \!ttInitializeTable
  \fi}

%D The next redefinition is more robust than the original:

\def\SetTableToWidth#1%
  {\doifelsenothing{#1}{\!taTableSpread\emptytoks}{\!taTableSpread{to #1}}}

% (*) Try this one with \type {direction} and {girection};
% the \PPCHTEX\ manual is a nice testcase.
%
% \startoverlay
%   {\starttable[ | l w(2cm) | w(8cm) | ]
%    \HL
% \VL direction \VL \showbaselines \dorecurse{3}{direction }\VL \FR
% \VL direction \VL \showbaselines \dorecurse{3}{direction }\VL \MR
% \VL direction \VL \showbaselines \dorecurse{3}{direction }\VL \LR
%    \HL
%    \stoptable}
%   {\starttable[ | l w(2cm) | p(8cm) | ]
%    \HL
% \VL direction \VL \showbaselines \dorecurse{3}{direction }\VL \FR
% \VL direction \VL \showbaselines \dorecurse{3}{direction }\VL \MR
% \VL direction \VL \showbaselines \dorecurse{3}{direction }\VL \LR
%    \HL
%    \stoptable}
% \stopoverlay
% \vskip2cm
% \starttable[ | l w(2cm) | p(8cm) | ]
% \HL
% \VL direction \VL \showbaselines \dorecurse{3}{direction }\VL \FR
% \VL direction \VL \showbaselines \dorecurse{8}{direction }\VL \LR
% \HL
% \stoptable
% \vskip2cm
% \starttable[ | l w(2cm) | p(8cm) | ]
% \HL
% \VL direction \VL \showbaselines \dorecurse{8}{direction }\VL \FR
% \VL direction \VL \showbaselines \dorecurse{8}{direction }\VL \LR
% \HL
% \stoptable

%D To give an impression of what the (well documented) source
%D of \TABLE\ looks like, we first implement an alternative for
%D the numeric keys. The quantity keys (\type{q} and \type{Q})
%D support the more european way of writing numbers:
%D
%D \startnarrower
%D 100.000.000,00 instead of 100,000,000.00
%D \stopnarrower
%D
%D The next table shows how to use these keys. We use braces
%D instead of brackets because we need brackets to specify the
%D format.
%D
%D \startbuffer
%D \starttable{|q[00,000]|Q[00,00]|}
%D \HL
%D \VL -1,2   \VL 12,35 \VL\FR
%D \VL 11,203 \VL  2,4  \VL\LR
%D \HL
%D \stoptable
%D \stopbuffer
%D
%D \ShowExample
%D
%D Although a more efficient implementation is possible |<|we
%D can for instance share common macros|>| we just adapt a copy
%D of the numeric ones. To permit double loading of this
%D module, we check for the existence of one of the macros.

\letvalue{!tk<\string q>}=\undefined
\letvalue{!tk<\string Q>}=\undefined

%D We just copy the original {\em comments}.
%D
%D \em Key \type{q}: quantity item, non||math mode.

\NewFormatKey q%
  {\letempty\!tqStyle
   \futurelet\!tnext\!tqTestForBracket}

%D \em Key \type{Q}: quantity item, math mode.

\NewFormatKey Q%
  {\def\!tqStyle{$}%
   \futurelet\!tnext\!tqTestForBracket}

%D \em Note: the space between a quantity entry and the
%D following \type{|}, \type{"}, or \type{\|} is mandatory.
%D empty quantity entries are not allowed: use \type{{}} or
%D \type{\omit} instead.
%D
%D \em Test for bracket: invoked by the keys \type{q} and
%D \type{Q}.

\def\!tqTestForBracket
  {\ifx[\!tnext
     \!thx\!tqGetArgument
   \else
     \!thx\!tqGetCode
   \fi}

%D \em Get code: e.g. \type{4}, or \type{4,0}, \type{0,4}, or
%D \type{10,2}.

\def\!tqGetCode#1 % note the blank
  {\!tqConvertCode #1,,!}

%D \em Convert code: e.g. converts above to \type{[0000]},
%D \type{[0000,]}, \type{[,0000]}, \type{[0000000000,00]}.

\def\!tqConvertCode #1,#2,#3!%
  {\begingroup
   \aftergroup\edef
   \aftergroup\!ttemp
   \aftergroup{%
   \aftergroup[%
   \!taCountA #1
   \!thLoop
     \ifnum \!taCountA>\zerocount
       \advance\!taCountA \minusone
       \aftergroup0
   \repeat
   \def\!ttemp{#3}%
   \ifx\!ttemp\empty
   \else
     \aftergroup,
     \!taCountA #2
     \!thLoop
       \ifnum\!taCountA>\zerocount
         \advance\!taCountA \minusone
         \aftergroup0
     \repeat
   \fi
   \aftergroup]\aftergroup}%
   \endgroup\relax
   \!thx\!tqGetArgument\!ttemp}

%D \em Get argument:
%D
%D \starttyping
%D <sample left field> <optional, sample right field>
%D \stoptyping

\def\!tqGetArgument[#1]%
  {\!tqMakeQuantityTemplate\!tqStyle#1,,!}

%D \em Make quantity template.

\def\!tqMakeQuantityTemplate#1#2,#3,#4!%  #1=<empty> or $
  {\def\!ttemp{#4}%
   \ifx\!ttemp\empty
     \!taDimenC\zeropoint
   \else
     \setbox0\hbox{\m@th #1,#3#1}%
     \!taDimenC\wd0
   \fi
   \setbox0\hbox{\m@th #1#2#1}%
   \!thToksEdef\!taDataColumnTemplate
     ={\noexpand\!tqSetQuantityItem{\the\wd0 }{\the\!taDimenC}{#1}%
       \the\!taDataColumnTemplate}%
  \ReadFormatKeys}

%D \em Set numeric item.

\def\!tqSetQuantityItem #1#2#3#4 %
  {\!tqSetQuantityItemA{#1}{#2}{#3}#4,,!}

\def\!tqSetQuantityItemA #1#2#3#4,#5,#6!%
  {\def\!ttemp{#6}%
   \hbox to #1{\hss\m@th#3#4#3}%
   \hbox to #2{\ifx\!ttemp\empty\else\m@th#3,#5#3\fi\hss}}

%D Here ends the Q||extension. Did you watch the clever use
%D of aftergroup in \type{\!tqConvertCode}.

% %D We also (have to) define a key for \type{\cap}:
%
% \letvalue{!tk<\string K>}=\undefined
%
% \NewFormatKey K%
%   {\ReadFormatKeys b\smallcapped}

%D A few pages back we saw backgrounds, further on we will see
%D colored rules, and here we provide a means to color the
%D entries in a column. (We can of course always use the normal
%D color commands for individual entries.) We could not use the
%D lowercase~\type{c}, because that one is used to force {\em
%D centering}.
%D
%D \startbuffer
%D \starttable[|C{red}|C{green}|C{blue}|]
%D \VL R(ed) \VL G(reen) \VL B(lue) \VL\SR
%D \stoptable
%D \stopbuffer
%D
%D \ShowExample

\letvalue{!tk<\string C>}=\undefined

\NewFormatKey C#1%
  {\ReadFormatKeys b{\localstartcolor[#1]} a{\localstopcolor}}

%D So now we have three new keys:
%D
%D \starttable[|||]
%D \HL
%D \NC \bf key       \NC \bf meaning                 \NC\AR
%D \HL
%D \NC Q[x,y]        \NC math mode formatted numbers \NC\AR
%D \NC q[x,y]        \NC text mode formatted numbers \NC\AR
%D \NC C{identifier} \NC column entry color          \NC\AR
%D \HL
%D \stoptable

%D To be compatible with the tabulate environment, we also
%D support the \type {l}, \type {c} and \type {r} keys for
%D paragraph entries.

\letvalue{!tk<\string l>}=\undefined
\letvalue{!tk<\string c>}=\undefined
\letvalue{!tk<\string r>}=\undefined
\letvalue{!tk<\string x>}=\undefined % not that needed

\NewFormatKey c%
  {\prependtoks\raggedcenter\to\!taDataColumnTemplate
   \ReadFormatKeys \LeftGlue\hfil \RightGlue\hfil}

\NewFormatKey l%
  {\prependtoks\raggedright\to\!taDataColumnTemplate
   \ReadFormatKeys \LeftGlue\empty \RightGlue\hfil}

\NewFormatKey r%
  {\prependtoks\raggedleft\to\!taDataColumnTemplate
   \ReadFormatKeys \LeftGlue\hfil \RightGlue\empty}

\NewFormatKey x%
  {\prependtoks\notragged\to\!taDataColumnTemplate
   \ReadFormatKeys \LeftGlue\hfil \RightGlue\empty}

\appendtoks \TABLEparalignment \to \EveryTableParBox

\def\!tfReFormat#1%
  {\the \!taLeftGlue
   \vbox{\forgetall\ialign{\span\the\!taDataColumnTemplate\cr#1\cr}}%
   \the \!taRightGlue
   \kern\zeropoint} % prevents \unskip / really needed

%D Later on, we're going to implement multiple page table
%D support, therefore the next \TABLE\ macro needs to be
%D slightly adapted, i.c. the penalty is removed. We also
%D add basic color support.

\def\!ttFullHruleA
  {\!ttGetHalfRuleThickness
   \startglobalTABLEcolor % added
   \hrule\!thHeight\dimen0\!thDepth\dimen0
   \stopglobalTABLEcolor  % added
   %\penalty0 % removed
   \egroup}

%D We'll see that when we want to give a vertical rule a color,
%D we have to set and reset states. After heavy testing it
%D proved most useful to extend a \TABLE\ primitive with some
%D hooks. One thing to keep in mind is that \type{&} keeps
%D assignments local. Again, we add basic color support.

\let\TABLEbeforebar\empty
\let\TABLEafterbar \empty

\def\@VLn{1}
\def\@VLd{.125em}

\def\do!ttInsertVrule % will be merged in 2005
  {\vrule \!thWidth
   \ifnum\!tgCode=\plusone
     \ifx\!tgValue\empty
       \LineThicknessFactor
     \else
       \!tgValue
     \fi
     \LineThicknessUnit
   \else
     \!tgValue
   \fi
   \hskip\@VLd}

\def\!ttInsertVrule
  {\hfil
   \TABLEbeforebar % added
   \startglobalTABLEcolor % added
   % we could do without this speedup, some day merge 'm
   \ifcase\@VLn\or
     \do!ttInsertVrule
     \unskip
   \else
     \dorecurse\@VLn\do!ttInsertVrule
     \gdef\@VLn{1}%
     \unskip
   \fi
   \stopglobalTABLEcolor % added
   \TABLEafterbar % added
   \hfil
   &}

%D The next two macros are only adapted to basis rule
%D color support.

\def\!tfSetVrule
  {\!thToksEdef\!taRuleColumnTemplate=
      {\noexpand\hfil
       \noexpand\startglobalTABLEcolor % added
       \noexpand\vrule
       \noexpand\!thWidth
       \ifnum\!tgCode=\plusone
         \ifx\!tgValue\empty
           \the\LineThicknessFactor
         \else
           \!tgValue
         \fi
         \!taLTU
       \else
         \!tgValue
       \fi
       ####%
       \noexpand\hfil
       \noexpand\stopglobalTABLEcolor % added
       \the\!taRuleColumnTemplate}%
  \!tfAdjoinPriorColumn}

\def\!ttShortHruleA
  {\!ttGetHalfRuleThickness
   \startglobalTABLEcolor % added
   \leaders\hrule\!thHeight\dimen0\!thDepth\dimen0\hfill
   \stopglobalTABLEcolor % added
   \null
   \ignorespaces}

%D We already showed the next one, but here we slightly adapt
%D the macro by adding an \type{\expandafter}. The space after
%D \type{#1} is crucial!

\def\normalTABLEcomplexbar#1%
  {\unskip\!ttRightGlue&\omit\expandafter\!ttAlternateVrule#1 }

%D To get rid of interfering \type{\omit}'s when we are
%D checking the number of columns and reporting problems. The
%D extensions concern the second level check, the first
%D subbranch and advancing the column.

\ifx\mscount\undefined \newcount\mscount \fi

\def\!ttuse#1%
  {\ifnum#1>\plusone
     \omit
     \global\TABLEdivisionfalse
     \scratchcounter\currentTABLEcolumn                % added
     \advance\scratchcounter #1%                       % added
     \advance\scratchcounter \minusone                 % added
     \ifnum\scratchcounter>\maxTABLEcolumn             % added
       \def\next                                       % added
         {\setTABLEerror\TABLEspanoverflow             % added
          \handleTABLEerror}%                          % added
     \else                                             % added
       \def\next                                       % added
         {\global\advance\currentTABLEcolumn #1%       % added
          \global\advance\currentTABLEcolumn \minusone % added
          \mscount#1%   \mscount is in Plain
          \advance\mscount \minusone
          \advance\mscount \mscount
          \!thLoop
            \ifnum\mscount>\plusone
              \sp@n % from Plain (\span\omit \advance\mscount\m@ne)
          \repeat
          \span}%
     \fi                                                 % added
   \else                                                 % added
     \def\next % conflicts with possible next \omit      % added
       {\global\advance\currentTABLEcolumn \plusone}%    % added
   \fi
   \next}                                                % added

% \starttable[|c|c|c|c|]
% \HL
% \VL     {test}  \VL \TWO{}     \VL test \VL\FR
% \DL             \DC                     \DL\DR
% \VL     {test}  \VL \TWO{}     \VL test \VL\LR
% \HL
% \stoptable

%D All commands that are executed between rows are to be put in
%D \type {\noalign}. We can however not verify if we (that is
%D \TABLE) does or did not enter this mode. A moderate dirty
%D but useful trick is using our own alternative:\footnote{Once
%D one has entered the stage of redefining \TEX\ primitives,
%D such hacks become a second nature. However, redefining \type
%D {\omit} and \type{\span} is not that easy.}

\def\TABLEnoalign
  {\noalign\bgroup\let\noalign\relax\let\next=}

%D \macros
%D   {starttable}
%D
%D The rest of this module is not easy to comprehend, mainly
%D because we have to take care of:
%D
%D \startitemize[packed]
%D \item  \type{\startitemize[template]}
%D \item  \type{\startitemize{template}}
%D \item  \type{\startitemize[predefined]}
%D \stopitemize
%D
%D as well as:
%D
%D \startitemize[continue]
%D \item  restart after table break
%D \stopitemize
%D
%D The official specification of the start command is:
%D
%D \showsetup{starttable}

\newconditional\tablerepeathead
\newconditional\tablerepeattail

\def\starttable
  {\bgroup
   \doif\@@tisplit\v!auto
     {\ifinsidesplitfloat\let\@@tisplit\v!yes\fi}%
   \doifinsetelse\@@tisplit{\v!yes,\v!repeat}
     {\def\stoptable{\stoptables\egroup}%
      \starttables}
     {\doifelsenothing\@@tiframe
        {\ifinsidefloat\else\startbaselinecorrection\fi}
        {\startframedcontent[\@@tiframe]}%
      \postponefootnotes
      \firststagestartTABLE}}

\def\stoptable
  {\chuckTABLEautorow % before the tail, else noalign problem
   \insertTABLEtail
   \TABLEnoalign{\globalletempty\@@TABLEhead}%
   \TABLEnoalign{\globalletempty\@@TABLEtail}%
   \finishTABLE
   \doifelsenothing\@@tiframe
     {\ifinsidefloat\else
        \stopbaselinecorrection
        \goodbreak % compensates all the nobreaks
      \fi}
     \stopframedcontent
   \egroup}

%D Before we can grab the argument, we have to make sure that
%D the \CATCODES\ are set. The first stage takes care of that.

\def\firststagestartTABLE
  {\bgroup                   % kan-ie weg?
   \global\intabletrue
   \pushouterbarandquote
   %catcode`\|=\@@other
   \complexorsimple\secondstagestartTABLE}

\def\simplesecondstagestartTABLE#1%
  {\complexsecondstagestartTABLE[{#1}]}

%D \macros
%D   {definetabletemplate}
%D
%D The complex (and main) start macro first takes care of the
%D predefined case. Such a predefined setup looks like:
%D
%D \starttyping
%D \definetabletemplate[test][|||]
%D
%D \starttable[test]
%D \VL test \VL test \VL\AR
%D \VL test \VL test \VL\AR
%D \VL test \VL test \VL\AR
%D \stoptable
%D \stoptyping
%D
%D The implementation of the definition macro is not that
%D complicated:

\def\definetabletemplate % to be redone
  {\bgroup
   \catcode`\|=\@@other
   \doquadrupleempty\dodefinetabletemplate}

\def\dodefinetabletemplate[#1][#2][#3][#4]%
  {\ifsecondargument
     \setgvalue{\c!Table#1}{\douseTABLEtemplate{#2}{#3}{#4}}%
   \fi
   \egroup}

\def\douseTABLEtemplate#1#2#3%
  {\gdef\TABLEhead{\getvalue{@@TABLEhead#2}}%
   \gdef\TABLEtail{\getvalue{@@TABLEtail#3}}%
   \complexsecondstagestartTABLE[#1]}

%D The optional third and fourth arguments define which table
%D head and tail to use.
%D
%D \starttyping
%D \definetabletemplate[test][|||][before][after]
%D \stoptyping
%D
%D This also means that one can define table heads and tails
%D by name!
%D
%D \starttyping
%D \starttablehead[before]
%D \HL \VL first \VL second \VL \SR \HL
%D \stoptablehead
%D \stoptyping
%D
%D Templates defined this way get protected names, that cannot
%D conflict with existing commands.
%D
%D \showsetup{definetabletemplate}
%D
%D The second half of the next macro prepares table
%D splitting.

\def\insertTABLEhead
  {\TABLEnoalign{\global\settrue \preventTABLEbreak \global\setfalse\someTABLEhead}%
   \TABLEhead
   \TABLEnoalign{\global\setfalse\preventTABLEbreak}}

\def\insertTABLEtail
  {\TABLEnoalign{\global\settrue \preventTABLEbreak \global\setfalse\someTABLEtail}%
   \TABLEtail
   \TABLEnoalign{\global\setfalse\preventTABLEbreak}}

% \def\dorestartTABLE#1%
%   {\gdef\restartTABLE{#1}%
%    \restartTABLE
%    \insertTABLEhead
%    \ifsplittables \ifconditional \tablerepeattail
%      \TABLEnoalign{\goodbreak}%
%      \insertTABLEtail
%      \TABLEnoalign{\goodbreak}%
%    \fi \fi}

\def\verysimpleTableHL
  {\TABLEnoalign{\expandafter\normalTABLEfullrule\@@tiHLheight}}

\def\dorestartTABLE#1%
  {\gdef\restartTABLE{#1}%
   \restartTABLE
   \TABLEnoalign{\globalpushmacro\simpleTableHL\global\let\simpleTableHL\verysimpleTableHL}%
   \insertTABLEhead
   \ifsplittables \ifconditional \tablerepeattail
     \TABLEnoalign{\goodbreak}%
     \insertTABLEtail
     \TABLEnoalign{\goodbreak}%
   \fi \fi
   \TABLEnoalign{\globalpopmacro\simpleTableHL}}

% \def\complexsecondstagestartTABLE#1[#2]%
%   {\convertargument|\to\asciia
%    \convertargument#2\to\asciib
%    \global\setfalse\someTABLEhead
%    \global\setfalse\someTABLEtail
%    \doifinstringelse\asciia\asciib
%      {\gdef\restartTABLE{\dorestartTABLE{\thirdstagestartTABLE{#2}}}}
%      {\doifdefinedelse{\c!Table#2}
%         {\gdef\restartTABLE{\getvalue{\c!Table#2}}}
%         {\gdef\restartTABLE{\dorestartTABLE{\getvalue{#2}}}}}%
%    \restartTABLE}

\bgroup \catcode`|=\@@other \catcode`"=\@@other

\gdef\complexsecondstagestartTABLE#1[#2]% brr nested mess
  {\bgroup
   \@@useotherbar
   \@@useotherquote
   \global\setfalse\someTABLEhead
   \global\setfalse\someTABLEtail
   \expanded{\doifinstringelse{|}{#2}}
     {\xdef\restartTABLE{\noexpand\dorestartTABLE{\noexpand\thirdstagestartTABLE{#2}}}}
     {\doifdefinedelse{\c!Table#2}
        {\gdef\restartTABLE{\getvalue{\c!Table#2}}}
        {\gdef\restartTABLE{\dorestartTABLE{\getvalue{#2}}}}}%
   \egroup
   \restartTABLE}

\egroup

%D The third stage involves a lot of (re)sets, which we will
%D explain later.

%D The next definition is convenient and more in tune with
%D \CONTEXT.

\let \everytable \EveryTable

%D We immediately use this register:

\appendtoks
    \fixedspaces
    \let\_\normalunderscore
\to \everytable

%D Now we can start the table.

\def\thirdstagestartTABLE#1%
  {\global\setTABLEactiontrue
   \setTABLEaction\TABLEunknown
   \setTABLEforce\TABLEunknown
   \setTABLEerror\TABLEunknown
   \global\TABLEgraylinefalse
   \global\TABLEgraydonefalse
   \globalletempty\TABLEgrayline
   \globalletempty\nextTABLEgrayline
   \globalletempty\TABLEgraylineerror
   \globalletempty\TABLEgraylinestatus
   \resetVLvalues
   \appendtoks\popouterbarandquote\to\EveryTable
   \appendtoks\localTABLEsetup\to\EveryTable
   \BeginTable[\ifsplittables u\else b\fi]%
   \defineTABLEunits
   \defineTABLEsteps
   \defineTABLErules
   \defineTABLEdivisions
   \defineTABLEshorthands
   \defineTABLEbackgrounds
   \defineTABLEendings
   \forgetall % added
   \doifsomething{#1}
     {\def\TABLEformat{#1}%
      \getTABLEnofcolumns\TABLEformat
      % more modern is to use catcode tables
      \expandafter\BeginFormat\TABLEformat\EndFormat}}

\def\finishTABLE
  {\chuckTABLEautorow
   \unskip\crcr
   \EndTable
   \global\intablefalse
   \egroup}

%D \macros
%D   {starttables}
%D
%D Split tables are specified using the plural form of the
%D start and stop commands.
%D
%D \showsetup{starttables}
%D
%D For example:
%D
%D \starttyping
%D \starttables[|||]
%D \HL
%D \VL element \VL atom weight \VL\AR
%D \HL
%D \VL ....... \VL ........... \VL\AR
%D \VL ....... \VL ........... \VL\AR
%D \HL
%D \stoptables
%D \stoptyping

\newbox\tablecontentbox

\def\starttables
  {\bgroup
   \splittablestrue
   \doifelse\@@tisplit\v!repeat
     {\settrue \tablerepeathead\settrue \tablerepeattail}
     {\setfalse\tablerepeathead\setfalse\tablerepeattail}%
   \flushnotes
   \setbox\tablecontentbox\vbox\bgroup
   \forgetall
   \global\TABLEinbreakfalse
   \firststagestartTABLE}

% \def\stoptables
%   {\ifconditional\tablerepeattail\else\insertTABLEtail\fi
%    \finishTABLE
%    \egroup
%    \dosplittablebox\tablecontentbox
%    \flushnotes
%    \egroup}

\def\stoptables
   {\chuckTABLEautorow % AM: before the tail, else noalign problem
    \ifconditional\tablerepeattail\else\insertTABLEtail\fi
    \finishTABLE
    \egroup
    \dosplittablebox\tablecontentbox
    \flushnotes
    \egroup}

\newdimen\TABLEcaptionheight % obsolete

\def\dosplittablebox#1%
  {\resettsplit
   \def\tsplitminimumfreelines{2}%
   \def\tsplitminimumfreespace{\TABLEcaptionheight}%
   \setbox\tsplitcontent\box#1%
   \ifconditional\tablerepeathead \ifconditional\someTABLEhead
     \setbox\tsplithead\vsplit\tsplitcontent to \lineheight
     \setbox\tsplithead\vbox{\unvbox\tsplithead}%
   \fi \fi
   \ifconditional\tablerepeattail \ifconditional\someTABLEtail
     \setbox\tsplittail\vsplit\tsplitcontent to \lineheight
     \setbox\tsplittail\vbox{\unvbox\tsplittail}%
   \fi \fi
   \ifinsidefloat\else
     \def\tsplitbeforeresult{\startbaselinecorrection}%
     \def\tsplitafterresult {\stopbaselinecorrection}%
   \fi
   \handletsplit}

%D When the table in the previous example is split across
%D pages, only the first gets a head. We could have said
%D something like:
%D
%D \starttyping
%D \starttablekop
%D \HL
%D \VL element \VL atom weight \VL\AR
%D \HL
%D \stoptablekop
%D
%D \starttablestaart
%D \HL
%D \stoptablestaart
%D
%D \starttables[|||]
%D \VL ....... \VL ........... \VL\AR
%D \VL ....... \VL ........... \VL\AR
%D \stoptables
%D \stoptyping
%D
%D This time each split table gets a head line and ends with
%D a rule. Keep in mind that such heads also apply to the
%D unbroken ones and should be defined local (grouped) if
%D needed. The rather complicated definition below is due to
%D the fact that the stopcondition is interface language
%D dependant.

\let\@@TABLEhead\empty  \def\TABLEhead{\@@TABLEhead}
\let\@@TABLEtail\empty  \def\TABLEtail{\@@TABLEtail}

\letvalue{\e!start\v!tablehead}=\undefined
\letvalue{\e!stop \v!tablehead}=\undefined
\letvalue{\e!start\v!tabletail}=\undefined
\letvalue{\e!stop \v!tabletail}=\undefined

\expanded
  {\def\csname\e!start\v!tablehead\endcsname##1\csname\e!stop\v!tablehead\endcsname%
     {\noexpand\setTABLEhead##1\noexpand\end}}

\expanded
  {\def\csname\e!start\v!tabletail\endcsname##1\csname\e!stop\v!tabletail\endcsname%
     {\noexpand\setTABLEtail##1\noexpand\end}}

%D The second argument is a dummy one, by scanning for it, we
%D get rid of interfering spaces.

\def\setTABLEhead{\dodoubleempty\dosetTABLEhead}
\def\setTABLEtail{\dodoubleempty\dosetTABLEtail}

\newconditional\preventTABLEbreak
\newconditional\someTABLEhead

\def\dosetTABLEhead[#1][#2]#3\end{\setvalue{@@TABLEhead#1}{\TABLEnoalign{\global\settrue\someTABLEhead}#3}}
\def\dosetTABLEtail[#1][#2]#3\end{\setvalue{@@TABLEtail#1}{\TABLEnoalign{\global\settrue\someTABLEtail}#3}}

%D Redudant \type{\HL}'s are removed automatically, so
%D mid||lines can be used without problems.

%D We need an alternative for the normal complex or simple
%D commands, because assignments in these system commands
%D conflict with \type{\noalign}. This alternative is about
%D as efficient as possible.

\def\complexorsimpleTable#1#2%
  {\csname\if[\noexpand#2\s!complex\else\s!simple\fi\c!Table#1\endcsname#2}

%D The next one is used in \type{\VL} cum suis and honours
%D the next grouping.

\def\docomplexorsimpleTable#1#2%
  {\ifx\next\bgroup\@EA#2\else\@EA\dodocomplexorsimpleTable\@EA#1\@EA#2\fi}

\def\dodocomplexorsimpleTable#1#2#3%
  {\if[\noexpand#3\@EA#1\else\@EA#2\fi#3}

%D The order of the next macros is more or less random. First
%D we implement error recovery. Errors are reported to the
%D screen and log file as well as visualized in the table in
%D teletype.

\def\handleTABLEerror
  {\ifTABLEgrayline \else
     \ifnum\TABLEerror=\TABLEunknown \else
       \setTABLEaction\TABLEunknown
       \globalletempty\checkTABLEautorow
       \globalletempty\chuckTABLEautorow
     \fi
     \ifcase\TABLEerror
       % no error
     \or
       % \TABLEmissingrow
       \tttf [missing row]%
       \writestatus\m!TABLE{missing row}%
       \SR
     \or
       % \TABLEmissingcolumn
       \fillTABLEcolumns
       \tttf [missing column]%
       \writestatus\m!TABLE{missing column}%
       \SR
     \or
       % \TABLEspanoverflow
       \fillTABLEcolumns
       \tttf [columnspan too large]%
       \writestatus\m!TABLE{columnspan too large}%
       \SR
     \or
       % \TABLEdivisionoverflow
       \fillTABLEcolumns
       \tttf [division line too long]%
       \writestatus\m!TABLE{division line too long}%
       \SR
     \fi
   \fi
   \ifnum\TABLEerror=\TABLEunknown \else
     \finishTABLErow
   \fi}

\def\finishTABLErow
  {\crcr
   \TABLEnoalign
     {\nobreak
      \setTABLEaction\TABLEunknown
      \setTABLEerror\TABLEunknown
      \globalletempty\checkTABLEautorow
      \globalletempty\chuckTABLEautorow
      \global\currentTABLEcolumn\zerocount}}

\def\fillTABLEcolumns
  {\ifnum\currentTABLEcolumn>\maxTABLEcolumn \else
     \global\advance\currentTABLEcolumn \plusone
     \normalTABLEquote
     \expandafter\fillTABLEcolumns
   \fi}

%D Next we enter the more complicated area of column and row
%D switching. I won't go into much detail from now on, but just
%D mention the general principles.
%D
%D \startitemize[3*ruim]
%D \sym{\type{\SR}} end a separate row (between rules)
%D \sym{\type{\FR}} end a first row (after a rule)
%D \sym{\type{\MR}} end a mid row (between text lines)
%D \sym{\type{\LR}} end a last row (before a rule)
%D \stopitemize
%D
%D and best of all:
%D
%D \startitemize[continue]
%D \sym{\type{\AR}} end a row with automatic spacing
%D \stopitemize
%D
%D As far as possible, we report confusing situations. In
%D most cases one can use \type{\AR}, which transfigurates
%D itself into one of the other types.
%D
%D \starttyping
%D \starttable[||]
%D \HL
%D \VL a separate row \VL\SR
%D \HL
%D \VL a first row    \VL\FR
%D \VL a mid row      \VL\MR
%D \VL a last row     \VL\LR
%D \HL
%D \stoptable
%D \stoptyping
%D
%D In this example we could have used \type{\AR} without
%D problems.
%D
%D Color or gray scale backgrounds precede the content. They
%D are passed over horizontal (division) lines when needed.
%D Errors in the color template are traced elsewhere. Here we
%D only check for inconsistent spacing. Due to the way \TEX\
%D handles alignments, we cannot automate spacing for colored
%D rows and columns.

\chardef\TABLErowzero=0

\def\checkTABLErow#1% pure for message purposes
  {\unskip % added
   \ifTABLEgraydone
     \convertargument         #1\to\asciia
     \convertcommand\TABLEendBCL\to\asciib
     \ifx\asciia\asciib \else
       \writestatus\m!TABLE{confusing \asciia\space and \asciib}%
       \gdef\TABLEgraylineerror%
         {\globalletempty\TABLEgraylineerror
          [\asciia\unskip<->\asciib\unskip]}%
     \fi
     \global\TABLEgraydonefalse
   \fi}

\def\defineTABLEendings
  {\let\SR\TableSR
   \let\FR\TableFR
   \let\MR\TableMR
   \let\LR\TableLR
   \let\AR\TableAR}

\def\TableSR
  {\ifTABLEgrayline \else
     \ifnum\TABLEaction=\TABLEfirstrow
       \writestatus\m!TABLE{change \string\SR\space into \string\MR/\string\LR}%
     \else\ifnum\TABLEaction=\TABLEmidrow
       \writestatus\m!TABLE{change \string\SR\space into \string\MR/\string\LR}%
     \else\ifnum\TABLEaction=\TABLEmidrow
       \writestatus\m!TABLE{change \string\SR\space into \string\MR/\string\LR}%
     \fi\fi\fi
   \fi
   \checkTABLErow\SR
   \endTABLErow\TABLEseparaterow\TABLErowfactor\TABLErowfactor}

\def\TableFR
  {\ifTABLEgrayline \else
     \ifnum\TABLEaction=\TABLEmidrow
       \writestatus\m!TABLE{change \string\FR\space into \string\MR/\string\LR}%
     \else\ifnum\TABLEaction=\TABLElastrow
       \writestatus\m!TABLE{change \string\FR\space into \string\MR/\string\LR}%
     \fi\fi
   \fi
   \checkTABLErow\FR
   \endTABLErow\TABLEfirstrow\TABLErowfactor\TABLErowzero}

\def\TableMR
  {\ifTABLEgrayline \else
     \ifnum\TABLEaction=\TABLErule
       \writestatus\m!TABLE{change \string\MR\space into \string\FR/\string\SR}%
     \else\ifnum\TABLEaction=\TABLElastrow
       \writestatus\m!TABLE{change \string\MR\space into \string\FR}%
     \fi\fi
   \fi
   \checkTABLErow\MR
   \endTABLErow\TABLEmidrow00}

\def\TableLR
  {\ifTABLEgrayline \else
     \ifnum\TABLEaction=\TABLErule
       \writestatus\m!TABLE{change \string\LR\space into \string\FR/\string\SR}%
     \fi
   \fi
   \checkTABLErow\LR
   \endTABLErow\TABLElastrow\TABLErowzero\TABLErowfactor}

%D \macros
%D   {ifcheckTABLEcolums}
%D
%D
%D The next macros handle the actual row ending. This macro
%D also take care of space corrections due to table splitting
%D when \type{\MR} and collegues are used. When tracing is
%D enabled, the corrections as well as the values used to
%D determine the available space are shown (in color). By default
%D checking is off.

\newif\ifcheckTABLEcolumns

\let\beforeTABLEline\empty
\let\afterTABLEline \empty

\def\doendTABLErow#1#2#3%
  {\handleTABLEbreak#2#3%
   \beforeTABLEline
   \ifcase#1\relax
     % unknown
   \or
     \endofTABLEline[blue][\SR->\SR]\TABLErowfactor\TABLErowfactor
   \or
     \endofTABLEline[red][\FR->\FR]\TABLErowfactor\TABLErowzero
   \or
     \ifnum\TABLEforce=\TABLEforcelastrow
       \endofTABLEline[red][\MR->\LR]\TABLErowzero\TABLErowfactor
     \else\ifnum\TABLEforce=\TABLEforcefirstrow
       \endofTABLEline[red][\MR->\FR]\TABLErowfactor\TABLErowzero
     \else
       \endofTABLEline[green][\MR->\MR]\TABLErowzero\TABLErowzero
     \fi\fi
   \or
     \endofTABLEline[red][\LR->\LR]\TABLErowzero\TABLErowfactor
   \fi
   \TABLEnoalign
     {\setTABLEforce\TABLEunknown
      \global\currentTABLEcolumn\zerocount}%
   \afterTABLEline}

\def\endTABLErow#1#2#3%
  {\setTABLEaction#1%
   \ifTABLEgrayline
     \finishTABLErow
   \else
     \ifnum\currentTABLEcolumn>\maxTABLEcolumn
       \doendTABLErow{#1}{#2}{#3}%
     \else\ifcheckTABLEcolumns
       \setTABLEerror\TABLEmissingcolumn
       \handleTABLEerror
     \else
       \doendTABLErow{#1}{#2}{#3}%
     \fi\fi
   \fi}

%D Handling \type{\AR} is postponed till the next row. The
%D check takes care of the first and mid rows, the chuck macro
%D |<|how about that name|>| handles the last row.

\def\TableAR
  {\ifTABLEgraydone
     \globalletempty\checkTABLEautorow
     \globalletempty\chuckTABLEautorow
     \global\TABLEgraydonefalse
     \TABLEendBCL
   \else
     \globallet\checkTABLEautorow\docheckTABLEautorow
     \globallet\chuckTABLEautorow\dochuckTABLEautorow
   \fi}

\let\checkTABLEautorow\empty
\let\chuckTABLEautorow\empty

\def\docheckTABLEautorow
  {\globallet\checkTABLEautorow\empty
   \ifnum\TABLEaction=\TABLErule          \FR
   \else\ifnum\TABLEaction=\TABLEunknown  \FR
   \else                                  \MR
   \fi\fi}

\def\dochuckTABLEautorow
  {\globalletempty\checkTABLEautorow
   \globalletempty\chuckTABLEautorow
   \ifnum\TABLEaction=\TABLErule          \SR
   \else\ifnum\TABLEaction=\TABLEunknown  \SR
   \else                                  \LR
   \fi\fi}

%D When a table is split, we also add a tail and when present
%D we repeat the table head.

\def\handleTABLEbreak#1#2%
  {\globalletempty\beforeTABLEline
   \gdef\afterTABLEline{\TABLEnoalign{\ifconditional\preventTABLEbreak\nobreak\else\goodbreak\fi}}}

%D When tables are split, the spacing before and after a
%D horizontal rule is corrected according to what we expect.

\def\endofTABLEline[#1][#2->#3]#4#5%
  {\ifx#2#3\else
     \writestatus\m!TABLE{\string#2\space changed into \string#3}%
   \fi
   \iftracetables
     \bgroup
     \tttf\space
     \ifnum\TABLEerror=\TABLEunknown
       \ifx#2#3\else\string#2->\fi
     \else
       ->%
     \fi
     \color[#1]{\string#3}%
     \ifx\TABLEgraylineerror\empty
       \space\TABLEgraylinestatus
     \else
       \space\TABLEgraylineerror
     \fi
     \egroup
   \else\ifx\TABLEgraylineerror\empty \else
  %   \bgroup
  %   \tttf\space\TABLEgraylineerror
  %   \egroup
   \fi\fi
   \globalletempty\TABLEgraylinestatus
   \globalletempty\TABLEgraylineerror
   \expandafter\normalTABLElineformat#4#5\crcr % \crcr nodig ?
   \TABLEnoalign{\nobreak\global\setTABLEactiontrue}}

%D In order to prevent (as good as possible) alignment overflow
%D and therefore \TEX\ error messages, we check the maximum
%D number of columns. We keep track of the current column and
%D maximum column by means of two \COUNTERS. Keep in mind that
%D the number of \type{|}'s and \type{\VL}'s or alike is always
%D one more than the number of columns.

\newcount\currentTABLEcolumn
\newcount\maxTABLEcolumn

%D While defining this macro we change the \CATCODE\ of
%D \type{|}. When counting the bars, we use a non active
%D representation of the bar, simply because we cannot be sure
%D if the bar is active or not.\footnote{Normally it is, but
%D \TABLE\ changes the catcode when needed.}

% \bgroup
% \catcode`\|=\@@active
% \gdef\protectTABLEbar{\let|\letterbar}
% \catcode`\|=\@@other
% \gdef\getTABLEnofcolumns#1%
%   {\bgroup
%    \cleanupfeatures % needed !
%    \protectTABLEbar % is still a feature
%    \expanded{\convertargument#1}\to\ascii
%    \@EA\doglobal\@EA\counttoken\@EA|\@EA\in\ascii\to\maxTABLEcolumn
%    \global\advance\maxTABLEcolumn \minusone
%    % in case of & counting, divide by 2
%    \egroup}
% \egroup

\bgroup
    \catcode`\|=\@@other  \gdef\@@otherbar     {|}
    \catcode`\"=\@@other  \gdef\@@otherquite   {"}
    \catcode`\|=\@@active \gdef\@@useotherbar  {\let|\@@otherbar}
    \catcode`\"=\@@active \gdef\@@useotherquote{\let"\@@otherquote}
\egroup

\bgroup \catcode`\|=\@@other

\gdef\getTABLEnofcolumns#1%
  {\bgroup
   \cleanupfeatures % needed !
   \@@useotherbar
   \@@useotherquote
   \expanded{\convertargument#1}\to\ascii
   \@EA\doglobal\@EA\counttoken\@EA|\@EA\in\ascii\to\maxTABLEcolumn
   \global\advance\maxTABLEcolumn \minusone
   % in case of & counting, divide by 2
   \egroup}

\egroup

\def\!ttDoHalign
  {\baselineskip \zeropoint
   \lineskiplimit\zeropoint
   \lineskip     \zeropoint
   \tabskip      \zeropoint
   % does not work in normal tex
   % \expanded{\getTABLEnofcolumns{\the\!taPreamble}}% added
   \halign \the\!taTableSpread \bgroup
   \span\the\!taPreamble
   \ifx \!tfRowOfWidths \empty \else \!tfRowOfWidths \cr \fi}

%D \startitemize[3*ruim]
%D \sym{\type{\VL}} a vertical line
%D \sym{\type{\VC}} a vertical colored line
%D \sym{\type{\HL}} a horizontal line
%D \sym{\type{\HC}} a horizontal colored line
%D \stopitemize

% \def\defineTABLErules
%   {\let\VL\TableVL
%    \let\VC\TableVC
%    \let\HL\TableHL
%    \let\HC\TableHC}

\def\defineTABLErules
  {\let\VL\TableVL
   \let\VC\TableVC
   \let\HL\TableHL
   \let\HC\TableHC
   \let\VS\TableVS
   \let\VD\TableVD
   \let\VT\TableVT
   \let\VN\TableVN}

\def\TableVL
  {\checkTABLEautorow
   \nextTABLEgrayline
   \ifnum\currentTABLEcolumn>\maxTABLEcolumn
     \setTABLEerror\TABLEmissingrow
     \handleTABLEerror
   \else
     \global\advance\currentTABLEcolumn \plusone
     \expandafter\doTableVL
   \fi}

\def\doTableVL
  {\futurelet\next\dodoTableVL}

\def\dodoTableVL
  {\docomplexorsimpleTable\complexTableVL\simpleTableVL}

\def\complexTableVL[#1]%
  {\scratchcounter=0#1%
   \multiply\scratchcounter \@@tiVLwidth
   \setxvalue{wVL\the\currentTABLEcolumn}{\the\scratchcounter}%
   \simpleTableVL}

\def\simpleTableVL
  {\doifundefined{wVL\the\currentTABLEcolumn}%
     {\setgvalue{wVL\the\currentTABLEcolumn}{\@@tiVLwidth}}%
   \gdef\TABLEbeforebar
     {\getvalue{bVL\the\currentTABLEcolumn}%
      \letgvalueempty{bVL\the\currentTABLEcolumn}}%
   \gdef\TABLEafterbar
     {\getvalue{eVL\the\currentTABLEcolumn}%
      \letgvalueempty{eVL\the\currentTABLEcolumn}}%
   \edef\@@tiVLwidth{\getvalue{wVL\the\currentTABLEcolumn}}%
   \expanded{\normalTABLEcomplexbar\@@tiVLwidth\space}}% \relax breaks \use

% \starttable[|||]
% \HL
% \VL test \VS test \VL \FR
% \VL test \VD test \VL \MR
% \VL test \VT test \VL \LR
% \HL
% \stoptable

\def\TableVS  {\VN1}
\def\TableVD  {\VN2}
\def\TableVT  {\VN3}
\def\TableVN#1{\gdef\@VLn{#1}\VL}

\def\resetVLvalues
  {\dostepwiserecurse\zerocount\maxTABLEcolumn\plusone
     {\setgvalue{wVL\recurselevel}{\@@tiVLwidth}%
      \letgvalueempty{bVL\recurselevel}%
      \letgvalueempty{eVL\recurselevel}}%
   \global\currentTABLEcolumn\zerocount}

\def\TableVC
  {\checkTABLEautorow
   \nextTABLEgrayline
   \ifnum\currentTABLEcolumn>\maxTABLEcolumn
     \setTABLEerror\TABLEmissingrow
     \handleTABLEerror
   \else
     \global\advance\currentTABLEcolumn \plusone
     \expandafter\doTableVC
   \fi}

\def\doTableVC
  {\futurelet\next\dodoTableVC}

\def\dodoTableVC
  {\docomplexorsimpleTable\complexTableVC\simpleTableVC}

\def\complexTableVC[#1]%
  {\global\setvalue{bVC\the\currentTABLEcolumn}{\localstartcolor[#1]}%
   \global\setvalue{eVC\the\currentTABLEcolumn}{\localstopcolor}%
   \simpleTableVC}

\def\simpleTableVC
  {\global\setvalue{bVL\the\currentTABLEcolumn}%
     {\getvalue{bVC\the\currentTABLEcolumn}}%
   \global\setvalue{eVL\the\currentTABLEcolumn}%
     {\getvalue{eVC\the\currentTABLEcolumn}}%
   \doTableVL}

\def\TableHL
  {\ifnum\currentTABLEcolumn>\maxTABLEcolumn
     \chuckTABLEautorow
   \else\ifnum\currentTABLEcolumn=\zerocount
    %\chuckTABLEautorow
     \TABLEnoalign
       {\globalletempty\checkTABLEautorow
        \globalletempty\chuckTABLEautorow}%
   \else
     \setTABLEerror\TABLEmissingcolumn
     \handleTABLEerror
   \fi\fi
   \complexorsimpleTable{HL}}

\def\complexTableHL[#1]%
  {\TABLEnoalign
     {\scratchcounter0#1%
      \multiply\scratchcounter \@@tiHLheight
      \edef\@@tiHLheight{\the\scratchcounter}%
      \simpleTableHL}}

\def\simpleTableHL
  {\TABLEnoalign
     {\nobreak
      \ifnum\TABLEaction=\TABLErule
        \writestatus\m!TABLE{skipping \string\HL}% \statusmessage
      \else
        \ifnum\TABLEaction=\TABLEmidrow
          \writestatus\m!TABLE{change \string\MR\space into \string\LR/\string\SR}%
        \else\ifnum\TABLEaction=\TABLEfirstrow
          \writestatus\m!TABLE{change \string\MR\space into \string\SR}%
        \fi\fi
        \startHLcommand
        \expandafter\normalTABLEfullrule\@@tiHLheight
        \stopHLcommand
        \globalletempty\startHLcommand
        \globalletempty\stopHLcommand
        \accountTABLElinewidth
      \fi
      \setTABLEaction\TABLErule
      \nobreak}}

\let\startHLcommand\empty
\let\stopHLcommand \empty

\def\TableHC
  {\complexorsimpleTable{HC}}

\def\complexTableHC[#1]%
  {\TABLEnoalign
     {\gdef\startHCcommand{\localstartcolor[#1]}%
      \gdef\stopHCcommand {\localstopcolor}}%
   \simpleTableHC}

\def\simpleTableHC
  {\TABLEnoalign
     {\globallet\startHLcommand\startHCcommand
      \globallet\stopHLcommand \stopHCcommand}%
   \HL}

%D \startitemize[3*ruim]
%D \sym{\type{\NL}} a vertical skip
%D \sym{\type{\NR}} goto the next row
%D \sym{\type{\NC}} goto the next column
%D \sym{\type{\FC}} a first column
%D \sym{\type{\MC}} a mid column
%D \sym{\type{\LC}} a last column
%D \stopitemize

% n+1 uitleggen

\def\defineTABLEsteps
  {\let\NL\TableNL
   \let\NR\TableNR
   \let\NC\TableNC
   \let\FC\TableNC
   \let\MC\TableNC
   \let\LC\TableNC}

\def\TableNL
  {\complexorsimpleTable{NL}}

\def\complexTableNL[#1]%
  {\TABLEnoalign
     {\edef\@@tiNL{#1}%
      \simpleTableNL}}%

\def\simpleTableNL
  {\TABLEnoalign
     {\nobreak
      \setbox0\vbox{\blank[\@@tiNL]}%
      \vskip\ht0
      \nobreak}}

\def\TableNR
  {\ifnum\currentTABLEcolumn>\maxTABLEcolumn
     \global\currentTABLEcolumn\zerocount
     \normalTABLElineending
   \else
     \setTABLEerror\TABLEmissingcolumn
     \handleTABLEerror
   \fi
   \TABLEnoalign
     {\nobreak
      \setTABLEaction\TABLEunknown}}

\def\TableNC
  {\checkTABLEautorow
   \nextTABLEgrayline
   \ifnum\currentTABLEcolumn>\maxTABLEcolumn
     \setTABLEerror\TABLEmissingrow
     \handleTABLEerror
   \else
     \global\advance\currentTABLEcolumn \plusone
     \normalTABLEquote
   \fi}

% \bgroup
% \catcode`\|=\@@active
% \catcode`\"=\@@active
% \gdef\redefinetablebarandquote
%   {\def|{\VL}%           % \normalTABLEsimplebar
%    \def\|##1{\VL[##1]}%  % \normalTABLEcomplexbar
%    \def"{\NC}}           % \normalTABLEquote
% \egroup

\let\redefinetablebarandquote\relax

%D \startitemize[3*ruim]
%D \sym{\type{\DL}}
%D \sym{\type{\DV}} (\type{\VD})
%D \sym{\type{\DC}}
%D \sym{\type{\DR}}
%D \stopitemize

\newif\ifTABLEdivision

% \def\defineTABLEdivisions
%   {\global\TABLEdivisionfalse % in start
%    \let\DL\TableDL
%    \let\DC\TableDC
%    \let\DV\TableDV
%    \let\VD\TableDV
%    \let\DR\TableDR}

\def\defineTABLEdivisions
  {\global\TABLEdivisionfalse % in start
   \let\DL\TableDL
   \let\DC\TableDC
   \let\DV\TableDV
   \let\DR\TableDR}

\def\checkTABLEdivision
  {\ifTABLEdivision \else
     \chuckTABLEautorow
     \global\currentTABLEcolumn\zerocount
     \global\TABLEdivisiontrue
   \fi}

\def\TableDL
  {\checkTABLEdivision
   \complexorsimpleTable{DL}}

\def\simpleTableDL
  {\complexTableDL[1]}

\def\complexTableDL[#1]%
  {\ifnum\TABLEaction=\TABLErule
     \writestatus\m!TABLE{skipping \string\DL}%
   \else
     \ifnum\TABLEaction=\TABLEmidrow
       \writestatus\m!TABLE{change \string\MR\space into \string\LR/\string\SR}%
     \else\ifnum\TABLEaction=\TABLEfirstrow
       \writestatus\m!TABLE{change \string\MR\space into \string\SR}%
     \fi\fi
     \setTABLEaction=\TABLEunknown
     \ifnum\currentTABLEcolumn>\maxTABLEcolumn
       \setTABLEerror\TABLEmissingrow
       \handleTABLEerror
     \fi
     %\startHLcommand
     \ifnum#1=\plusone
       \global\advance\currentTABLEcolumn \plustwo
       \let\next\normalTABLEsinglerule
     \else
       \ifnum#1<\maxTABLEcolumn
         \global\advance\currentTABLEcolumn \plusone
         \def\next{\normalTABLEmultirule{#1}}%
       \else
         \setTABLEerror\TABLEdivisionoverflow
         \let\next\handleTABLEerror
       \fi
     \fi
     \next
     %\stopHLcommand
     %\globalletempty\startHLcommand
     %\globalletempty\stopHLcommand
   \fi}

\def\TableDV
  {\TableDCV\normalTABLEsimplebar}

\def\TableDC
  {\TableDCV\normalTABLEquote}

\def\TableDCV#1%
  {\checkTABLEdivision
   \checkTABLEautorow
   \ifnum\currentTABLEcolumn>\maxTABLEcolumn
     \setTABLEerror\TABLEmissingrow
     \handleTABLEerror
   \else
     \global\advance\currentTABLEcolumn \plusone
     #1%
   \fi}

\def\TableDR
  {\ifnum\currentTABLEcolumn<\maxTABLEcolumn % silent recovery
     %\setTABLEerror\TABLEmissingcolumn      % some day warning
     %\handleTABLEerror
     \finishTABLErow
   \else
     \global\currentTABLEcolumn\zerocount    % nog check
     \normalTABLElineending
   \fi
   \TABLEnoalign
     {\nobreak
      \global\TABLEdivisionfalse
      \accountTABLElinewidth % temporary solution
      \setTABLEaction\TABLErule}}

\def\accountTABLElinewidth
  {\scratchdimen\LineThicknessUnit}

%D \startitemize[3*ruim]
%D \sym{\type{\BC}}
%D \sym{\type{\BR}}
%D \sym{\type{\BACKGROUND}}
%D \sym{\type{\CL}}
%D \sym{\type{\RL}}
%D \sym{\type{\BL}}
%D \sym{\type{\RASTER}}
%D \sym{\type{\COLOR}}
%D \stopitemize

% definieer: \BC \BL
% herhaal: \BR
% definieer: \CL \RL (eerste \CL[green] = hele row! / \CL[1,green])
% dus: \CL en \RL mix tussen \HL en \BL

\def\defineTABLEbackgrounds
  {\let\BC        \TableBC
   \let\BL        \TableBL
   \let\BR        \TableBR
   \let\BACKGROUND\TableBR
   \let\CL        \TableCL
   \let\RL        \TableRL
   \let\COLOR     \TableCOLOR
   \let\RASTER    \TableRASTER
   \globallet\lastTABLEc\@@tibackgroundcolor
   \globallet\lastTABLEr\@@tibackgroundscreen
   \doifinsetelse\@@tibackground{c,color} % \v!color
     {\global\chardef\TABLEcr\plusone}
     {\global\chardef\TABLEcr\plustwo}}

\def\TableBC
  {\ifTABLEgrayline
     \normalTABLEquote
   \else
     \TABLEnoalign\bgroup
       \globallet\nextTABLEgrayline\executeTABLEgrayline
       \globalletempty\TABLEgrayline % new
       \let\BL\doTableBL
       \let\BC\doTableBC
       \expandafter\doTableBC
   \fi}

\def\doTableBC
  {\addtoTABLEgrayline{\BC}%
   \gobbleTableBCL}

\def\TableBL
  {\TABLEnoalign\bgroup
     \globallet\nextTABLEgrayline\executeTABLEgrayline
     \globalletempty\TABLEgrayline % new
     \let\BL\doTableBL
     \let\CL\doTableCL
     \let\RL\doTableRL
     \let\BC\doTableBC
     \doTableBL}

\def\doTableBL
  {\complexorsimpleTable{BL}}

\def\simpleTableBL
  {\complexTableBL[,]}

\def\complexTableBL[#1]%
  {\analyzeTABLEcr[#1]%
   \handleTABLEcr}

\def\TableBR#1%
  {\TABLEnoalign
     {\globallet\nextTABLEgrayline\executeTABLEgrayline
      \checkTABLEgrayline#1\BR
      \global\TABLEgraylinetrue}}

\def\analyzeTABLEcr[#1]%
  {\doanalyzeTABLEcr[#1,,]}

\def\doanalyzeTABLEcr[#1,#2,#3]%
  {\doifnumberelse{#1x}  % Is the x still needed here?
     {\dodoanalyzeTABLEcr[#1,#2,#3]}
     {\dodoanalyzeTABLEcr[1,#1,#2]}}

\def\dodoanalyzeTABLEcr[#1,#2,#3]%
  {\global\chardef\TABLEn#1\relax
   \processaction
     [#2]
     [     c=>\global\chardef\TABLEcr1,%
       color=>\global\chardef\TABLEcr1,%
           r=>\global\chardef\TABLEcr2,%
      raster=>\global\chardef\TABLEcr2]%
   \ifcase\TABLEcr \or
     \doifsomething{#3}{\xdef\lastTABLEc{#3}}%
   \or
     \doifsomething{#3}{\xdef\lastTABLEr{#3}}%
   \fi}

\def\handleTABLEcr
  {\relax % else funny side effect
   \ifcase\TABLEcr
     % Can't happen!
   \or
     \addtoTABLEgrayline{\complexTableCOLOR[\the\TABLEn,\lastTABLEc]}%
   \else
     \addtoTABLEgrayline{\complexTableRASTER[\the\TABLEn,\lastTABLEr]}%
   \fi
   \gobbleTableBCL}

\def\analyzeTABLEcrl#1[#2]%
  {\doanalyzeTABLEcrl#1[#2,,]}

\def\doanalyzeTABLEcrl#1[#2,#3,#4]%
  {\doifnumberelse{#2x} % x ????????????????????
     {\dodoanalyzeTABLEcr[#2,#1,#3]}
     {\dodoanalyzeTABLEcr[\ifTABLEgrayline1\else\maxTABLEcolumn\fi,#1,#2]}}

\def\TableCL
  {\TABLEnoalign\bgroup
     \globallet\nextTABLEgrayline\executeTABLEgrayline
     \globalletempty\TABLEgrayline % new
     \let\BL\doTableBL
     \let\CL\doTableCL
     \let\RL\doTableRL
     \let\BC\doTableBC
     \doTableCL}

\def\doTableCL
  {\complexorsimpleTable{CL}}

\def\simpleTableCL%  nog eens \'e\'en lijn van maken
  {\BL[\the\maxTABLEcolumn,c,\lastTABLEc]}

\def\complexTableCL[#1]%
  {\analyzeTABLEcrl{c}[#1]%
   \handleTABLEcr}

\def\TableRL
  {\TABLEnoalign\bgroup
     \globallet\nextTABLEgrayline\executeTABLEgrayline
     \globalletempty\TABLEgrayline % new
     \let\BL\doTableBL
     \let\CL\doTableCL
     \let\RL\doTableRL
     \let\BC\doTableBC
     \doTableRL}

\def\doTableRL
  {\complexorsimpleTable{RL}}

\def\simpleTableRL
  {\BL[\the\maxTABLEcolumn,r,\lastTABLEr]}

\def\complexTableRL[#1]%
  {\analyzeTABLEcrl{r}[#1]%
   \handleTABLEcr}

\def\checkTABLEgrayline#1#2%
  {\!!doneatrue
   \ifx#1\AR
     \!!doneafalse
   \else\ifx#1\SR\else\ifx#1\FR\else\ifx#1\MR\else\ifx#1\LR\else
     \!!doneafalse
   \fi\fi\fi\fi\fi
   \if!!donea
     \gdef\TABLEgraylinestatus
       {[\string#1]}%
     \gdef\TABLEendBCL
       {#1}%
   \else
     \gdef\TABLEgraylineerror
       {[\string#2\string#1->\string#2\string\SR]}%
     \gdef\TABLEendBCL
       {\SR}%
   \fi}

\def\endTABLErowGL#1#2#3%
  {\ifcase#1\relax
     % unknown
   \or
     \doPreTableGL\TABLErowfactor\TABLErowfactor
   \or
     \doPreTableGL\TABLErowfactor\TABLErowzero
   \or
     \ifnum\TABLEforce=\TABLEforcelastrow
       \doPreTableGL\TABLErowzero\TABLErowfactor
     \else\ifnum\TABLEforce=\TABLEforcefirstrow
       \doPreTableGL\TABLErowfactor\TABLErowzero
     \else
       \doPreTableGL\TABLErowzero\TABLErowzero
     \fi\fi
   \or
     \doPreTableGL\TABLErowzero\TABLErowfactor
   \fi}

\def\doPreTableGL#1#2%  betere namen
  {\xdef\OldLineThicknessFactor{\the\LineThicknessFactor}%
   \xdef\OldLineThicknessUnit{\the\LineThicknessUnit}%
   \global\LineThicknessFactor\plusone
   \setbox0\hbox{\AugmentedTableStrut{#1}{#2}}%
   \getboxheight\dimen0\of\box0\relax
   \xdef\TABLEgraylineHeight{\the\dimen0}%
   \global\LineThicknessUnit\TABLEgraylineHeight}

\def\doPostTableGL
  {\global\LineThicknessFactor\OldLineThicknessFactor
   \global\LineThicknessUnit  \OldLineThicknessUnit}

% kan simpeler

\def\docomplexTableCOLOR[#1]%
  {\dodocomplexTableGL\localstartcolor \localstopcolor [#1,\lastTABLEc,,]}

\gdef\docomplexTableRASTER[#1]%
  {\dodocomplexTableGL\localstartraster\localstopraster[#1,\lastTABLEr,,]}

\def\dodocomplexTableGL#1#2[#3,#4,#5,#6]%
  {\doifelsenothing{#4}{#1[#5]}{#1[#4]}%
   \doPreTableGL\TABLEendofrowheight\TABLEendofrowdepth
   \ifnum#3=\plusone % else conflict with \omit in \=
     \let\next\normalTABLEsinglerule
   \else
     \def\next{\normalTABLEmultirule{#3}}%
   \fi
   \next
   \doPostTableGL
   #2}

\def\TableBACKGROUND
  {\TableBR}

\def\simpleTableRASTER   #1{\docomplexTableRASTER[1]#1}
\def\complexTableRASTER[#1]{\docomplexTableRASTER[#1]}
\def\simpleTableCOLOR      {\docomplexTableCOLOR [1]}
\def\complexTableCOLOR [#1]{\docomplexTableCOLOR [#1]}

\def\TableRASTER{\complexorsimpleTable{RASTER}}
\def\TableCOLOR {\complexorsimpleTable{COLOR}}

\def\addtoTABLEgrayline#1%
  {\TABLEgraytoks\expandafter{\TABLEgrayline}%
   \xdef\TABLEgrayline{\the\TABLEgraytoks\noexpand#1}}

\def\setTableBCL#1#2%
  {\ifx#1#2%
     \gdef\TABLEgraylinestatus{[\string#1]}%
     \gdef\TABLEendBCL{#1}%
     \addtoTABLEgrayline{#1}%
   \else
     \gdef\TABLEgraylineerror{[\string#1->\string#2]}%
     \gdef\TABLEendBCL{#2}%
     \addtoTABLEgrayline{#2}%
   \fi}

\def\gobbleTableBCL#1%
  {\ifx#1\BC                    \let\next\doTableBC \else
   \ifx#1\BL                    \let\next\doTableBL \else
   \ifx#1\SR \setTableBCL\SR\SR \let\next\egroup    \else
   \ifx#1\FR \setTableBCL\FR\FR \let\next\egroup    \else
   \ifx#1\MR \setTableBCL\MR\MR \let\next\egroup    \else
   \ifx#1\LR \setTableBCL\LR\LR \let\next\egroup    \else
             \setTableBCL #1\SR \let\next\egroup
   \fi\fi\fi\fi\fi\fi
   \next}

\def\executeTABLEgrayline
  {\TABLEnoalign
     {\def\BC
        {\advance\currentTABLEcolumn \plusone}%
      \def\dodocomplexTableGL##1##2[##3,##4,##5,##6]%
        {\BC\advance\currentTABLEcolumn ##3 }%
      \let\endTABLErow\endTABLEgrayrow
      \currentTABLEcolumn\zerocount
      \TABLEgrayline\TABLEendBCL % determine n of columns and height
      \advance\currentTABLEcolumn \minusone
      \ifnum\currentTABLEcolumn>\maxTABLEcolumn
        % error message too long line
        \globalletempty\TABLEgrayline
      \else
        % \message{n of color columns: \the\currentTABLEcolumn}\wait
        \global\TABLEgraylinetrue  % vanaf hier nog checken
      \fi
      \global\currentTABLEcolumn\zerocount}%
   \unskip\TABLEgrayline\TABLEendBCL
   \TABLEnoalign
     {\nobreak
      \vskip-\TABLEgraylineHeight
      \nobreak
      \global\setTABLEactiontrue
      \global\currentTABLEcolumn\zerocount
      \globalletempty\nextTABLEgrayline
      \global\TABLEgraydonetrue
      \global\TABLEgraylinefalse}}

\def\endTABLEgrayrow#1#2#3%
  {\ifcase#1\relax
     \global\chardef\TABLEendofrowheight\TABLErowfactor
     \global\chardef\TABLEendofrowdepth \TABLErowfactor
   \or
     \global\chardef\TABLEendofrowheight\TABLErowfactor
     \global\chardef\TABLEendofrowdepth \TABLErowfactor
   \or
     \global\chardef\TABLEendofrowheight\TABLErowfactor
     \global\chardef\TABLEendofrowdepth \TABLErowzero
   \or
     \ifnum\TABLEforce=\TABLEforcelastrow
       \global\chardef\TABLEendofrowheight\TABLErowzero
       \global\chardef\TABLEendofrowdepth \TABLErowfactor
     \else\ifnum\TABLEforce=\TABLEforcefirstrow
       \global\chardef\TABLEendofrowheight\TABLErowfactor
       \global\chardef\TABLEendofrowdepth \TABLErowzero
     \else
       \global\chardef\TABLEendofrowheight\TABLErowzero
       \global\chardef\TABLEendofrowdepth \TABLErowzero
     \fi\fi
   \or
     \global\chardef\TABLEendofrowheight\TABLErowzero
     \global\chardef\TABLEendofrowdepth \TABLErowfactor
   \fi}

\def\defineTABLEshorthands%
  {\def\SPAN##1{\use{##1}}%
   \def\TWO    {\use2}%
   \def\THREE  {\use3}%
   \def\FOUR   {\use4}%
   \def\FIVE   {\use5}%
   \def\SIX    {\use6}%
   \def\REF    {\ReFormat}}

\def\defineTABLEunits
  {\processaction
     [\@@tidistance]
     [  \v!none=>\OpenUp00\def\LOW{\Lower6 },
       \v!small=>\OpenUp00\def\LOW{\Lower6 }, % == baseline
      \v!medium=>\OpenUp11\def\LOW{\Lower7 },
         \v!big=>\OpenUp22\def\LOW{\Lower8 }]%
   \doifelse\@@tidistance\v!none
     {\chardef\TABLErowfactor\zerocount}
     {\chardef\TABLErowfactor\plustwo  }}

\def\dohandlebar % here ?
  {\ifmmode
     \@EA\domathmodebar
   \else\ifintable
     \@EAEAEA\domathmodebar
   \else
     \@EAEAEA\dotextmodebar
   \fi\fi}

% De macro's t.b.v. instellingen.

\def\setuptables
  {\dosingleargument\dosetuptables}

\def\dosetuptables[#1]%
  {\getparameters[\??ti][#1]%
   \processaction
     [\@@tialign]
     [  \v!right=>\def\TABLEparalignment{\raggedright},
         \v!left=>\def\TABLEparalignment{\raggedleft},
       \v!middle=>\def\TABLEparalignment{\raggedcenter},
      \s!default=>\def\TABLEparalignment{\notragged},
      \s!unknown=>\def\TABLEparalignment{\notragged}]%
   \assignalfadimension\@@tiVL\@@tiVLwidth 246%
   \assignalfadimension\@@tiHL\@@tiHLheight246}

\def\localTABLEsetup
  {\@@ticommands\relax
   \expanded{\switchtobodyfont[\@@tibodyfont]}%
   \StrutHeightFactor  8
   \StrutDepthFactor   4
   \LineThicknessFactor4
   \NormalTLTU         {.1pt}%
   \NormalTSU          {\normalbaselineskip\divide\StrutUnit 12 }%
   \NormalTableUnits}

%D And then I wrote the tabulate environment. That
%D alternative supports setting the rule thickness and color,
%D so here is the table alternative.

\let\startglobalTABLEcolor\empty
\let\stopglobalTABLEcolor \empty

\def\localTABLEsetup
  {\@@ticommands\relax
   % bodyfont
   \expanded{\switchtobodyfont[\@@tibodyfont]}%
   % linecolor
   \doifsomething\@@tirulecolor
     {\def\startglobalTABLEcolor{\localstartcolor[\@@tirulecolor]}%
      \def\stopglobalTABLEcolor {\localstopcolor}}%
   % linethickness
   \LineThicknessFactor4
   \scratchdimen\@@tirulethickness
   \divide\scratchdimen \LineThicknessFactor
   \expanded{\NormalTLTU{\the\scratchdimen}}%
   % spacing, was depth=4 height=8 (counters, sigh, now macros)
   \doifelse\@@tiheight\v!strut
     {\let\StrutHeightFactor\@@itheight}
     {\let\StrutHeightFactor\@@tiheight}%
   \doifelse\@@tidepth\v!strut
     {\let\StrutDepthFactor\@@itdepth}
     {\let\StrutDepthFactor\@@tidepth}%
   \scratchdimen\StrutHeightFactor\points \multiply\scratchdimen 10%
   \edef\StrutHeightFactor{\withoutpt\the\scratchdimen}%
   \scratchdimen\StrutDepthFactor \points \multiply\scratchdimen 10%
   \edef\StrutDepthFactor{\withoutpt\the\scratchdimen}%
   % units
   \NormalTSU{\normalbaselineskip\divide\StrutUnit 12 }%
   \NormalTableUnits}

\def\OpenUp#1#2%
  {\scratchdimen\StrutHeightFactor \points \advance\scratchdimen #1\points
   \edef\StrutHeightFactor{\withoutpt\the\scratchdimen}%
   \scratchdimen\StrutDepthFactor  \points \advance\scratchdimen #2\points
   \edef\StrutDepthFactor{\withoutpt\the\scratchdimen}}

%D As one can see, we didn't only add color, but also more
%D control over spacing.
%D
%D \startbuffer[a]
%D \starttable[|c|]
%D \HL
%D \VL \strut test \VL \FR
%D \VL \strut test \VL \MR
%D \VL \strut test \VL \MR
%D \VL \strut test \VL \LR
%D \HL
%D \stoptable
%D \stopbuffer
%D
%D \startbuffer[b]
%D \starttabulate[|c|]
%D \HL
%D \NC test \NC \NR
%D \NC test \NC \NR
%D \NC test \NC \NR
%D \NC test \NC \NR
%D \HL
%D \stoptabulate
%D \stopbuffer
%D
%D In the next example, the first table is defined as:
%D
%D \typebuffer[a]
%D
%D and the second one as:
%D
%D \typebuffer[b]
%D
%D The first table is typeset using the default height and
%D depth factors .8 and .4. The second table has both factors
%D set to \type {strut}, and the third table shows what
%D happens when we set the values to zero. The rightmost table
%D is typeset using the tabulate environment.
%D
%D \startcombination[4*1]
%D   {$\vcenter{\getbuffer[a]}$}
%D     {\hbox{h=.8 d=.4}}
%D   {\setuptables[height=strut,depth=strut]$\vcenter{\getbuffer[a]}$}
%D     {\hbox{h=d=\type{strut}}}
%D   {\setuptables[height=0,depth=0]$\vcenter{\getbuffer[a]}$}
%D     {\hbox{h=d=0}}
%D   {$\vcenter{\getbuffer[b]}$}
%D     {\hbox{tabulate}}
%D \stopcombination

\setuptables
  [HL=\v!medium,
   VL=\v!medium,
   NL=\v!small,
   \c!frame=,
   \c!align=\v!right,
   \c!depth=.40, % \v!strut
   \c!height=.80, % \v!strut
   \c!rulethickness=\linewidth,
   \c!rulecolor=,
   \c!distance=\v!medium,
   \c!bodyfont=\the\bodyfontsize,
   \c!commands=,
   \c!background=\v!screen,
   \c!backgroundscreen=\@@rsscreen,
   \c!backgroundcolor=,
   \c!split=\v!auto]

\def\ifintabel{\ifintable} % upward compatible

\protect \endinput
