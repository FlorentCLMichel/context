%D \module
%D   [       file=supp-lan,
%D        version=1998.05.15,
%D          title=\CONTEXT\ Support Macros,
%D       subtitle=Number (Digit) Handling, 
%D         author=Hans Hagen,
%D           date=\currentdate,
%D      copyright={PRAGMA / Hans Hagen \& Ton Otten}]
%C
%C This module is part of the \CONTEXT\ macro||package and is
%C therefore copyrighted by \PRAGMA. See mreadme.pdf for 
%C details. 

\writestatus{loading}{Context Support Macros / Number (Digit) Handling}

\unprotect

%D \macros
%D   {digitmode, digits} 
%D
%D Depending on \type{\digitmode} the command \type{\digits} 
%D normalizes number patterns depending on the language set.
%D
%D \starttypen
%D This will never be a \digits{1.000.000} seller. 
%D \stoptypen
%D
%D or
%D
%D \starttypen
%D I will never grow longer than \digits 1.86 \Meter. 
%D \stoptypen
%D
%D The different modes are shown in:
%D
%D \startbuffer
%D / \setdigitmode 1 / \digits 12.345,90 / \digits 12.345.000 / \digits 1,23 /
%D / \setdigitmode 2 / \digits 12.345,90 / \digits 12.345.000 / \digits 1,23 /
%D / \setdigitmode 3 / \digits 12.345,90 / \digits 12.345.000 / \digits 1,23 /
%D / \setdigitmode 4 / \digits 12.345,90 / \digits 12.345.000 / \digits 1,23 /
%D / \setdigitmode 5 / \digits 12.345,90 / \digits 12.345.000 / \digits 1,23 /
%D / \setdigitmode 6 / \digits 12.345,90 / \digits 12.345.000 / \digits 1,23 /
%D \stopbuffer
%D
%D \typebuffer
%D
%D This is typset as: 
%D
%D \startregels
%D \haalbuffer
%D \stopregels

\chardef\digitinputmode =1
\chardef\digitoutputmode=1 

\def\setdigitmode#1%
  {\chardef\digitoutputmode#1}

%D The digit modes are:
%D 
%D \startopsomming[opelkaar]
%D \som periods      \& comma
%D \som commas       \& period 
%D \som thinmuskips  \& comma
%D \som thinmuskips  \& period
%D \som thickmuskips \& comma
%D \som thickmuskips \& period
%D \stopopsomming

\let\collecteddigits \empty
\let\saveddigits     \empty
\let\savedpowerdigits\empty
\chardef\powerdigits=0

\def\digits%
  {\bgroup\grabdigit}

\def\grabdigit%
  {\futurelet\next\scandigit}

\let\normalmath$

\def\scandigit%
  {\ifx\next\blankspace 
     \let\next\handledigits
   \else\ifx\next\bgroup
     \let\next\handledigits
   \else\ifx\next\egroup
     \let\next\handledigits
   \else\ifx\next\normalmath
     \let\next\handledigits
   \else
     \let\next\collectdigit
   \fi\fi\fi\fi
   \next}

\long\def\collectdigit#1%
  {\if\noexpand#1\relax
     \let\grabdigit\handledigits
   \else\ifcase\powerdigits
     \if#1^%
       \chardef\powerdigits=1
     \else
       \doifnumberelse{#1}
         {\edef\collecteddigits{\collecteddigits\saveddigits#1}%
          \let\saveddigits\empty}
         {\def\saveddigits{#1}}%
     \fi
   \else
     \doifnumberelse{#1}
       {\edef\savedpowerdigits{\savedpowerdigits\saveddigits#1}%
        \let\saveddigits\empty}
       {\def\saveddigits{#1}}%
   \fi\fi
   \grabdigit}
  
\def\handledigits%  
  {\ifmmode
     \dohandledigits
   \else
     \dontleavehmode\hbox{$\dohandledigits$}%
   \fi
   \saveddigits
   \egroup}

\def\dohandledigits%
  {\mathcode`\,="013B
   \mathcode`\.="013A
   \expandafter\handletokens\collecteddigits\with\scandigits
   \ifcase\powerdigits\else\digitpowerseperator^{\savedpowerdigits}\fi}

\def\scandigits#1%
  {\if#1.\doscandigit1\else
   \if#1,\doscandigit2\else
   #1\fi\fi}

\def\doscandigit#1%
  {\ifnum\digitinputmode=#1\relax
     \ifcase\digitoutputmode
     \or .%
     \or ,%
     \or \mskip\thinmuskip
     \or \mskip\thinmuskip
     \or \mskip\thickmuskip
     \or \mskip\thickmuskip
     \fi
   \else
     \ifodd\digitoutputmode,\else.\fi
   \fi}

%D We also permit:

\let\Digits=\digits

%D These macros are complicated by the fact that we also 
%D have to support cases like:
%D
%D \starttypen
%D {\digits1234}
%D \digits{1234}
%D \digits 1234\whatever
%D $\digits 123.222,00$
%D \digits 123.222,00.
%D \stoptypen
%D
%D The latter case shows us that trailing non digits are to
%D be passed untreated. 
%D
%D Another interesting case is: 
%D
%D \starttypen
%D \digits 123.222,00^10
%D \stoptypen
%D
%D The seperator is defined as:

\def\digitpowerseperator%
  {\cdot10}

%D \macros
%D   {digittemplate}
%D
%D Users can specify the way they enter those digits by sayon
%D something like:
%D
%D \starttypen
%D \digittemplate 12.000.000,00 % \digittemplate .,
%D \stoptypen

\def\digittemplate #1 %
  {\chardef\digitinputmode=0
   \handletokens#1\with\scandigittemplate}  

\def\scandigittemplate#1%
  {\if     #1.\ifcase\digitinputmode\chardef\digitinputmode=1 \fi% period
   \else\if#1,\ifcase\digitinputmode\chardef\digitinputmode=2 \fi% comma
   \fi\fi}

\protect

\endinput
