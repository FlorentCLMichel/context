%D \module
%D   [       file=core-nte,
%D        version=2009.03.08,
%D          title=\CONTEXT\ Table Macros,
%D       subtitle=Natural Tables Extensions,
%D         author=Hans Hagen \& Wolfgang Schuster,
%D           date=\currentdate,
%D      copyright=PRAGMA]
%C
%C This module is part of the \CONTEXT\ macro||package and is
%C therefore copyrighted by \PRAGMA. See mreadme.pdf for
%C details.

\writestatus{loading}{ConTeXt Table Module / Natural Tables Extensions}

\unprotect

%D This module is suggested by Wolfgang Schuster who also prototyped
%D it and came up with the rationale:
%D
%D This module provides an easy way to use natural in a similiar
%D way as the older table module (based on the \TABLE\ macros) and
%D the newer tabulate module.
%D
%D You can see the advantage in the following table, once created
%D with the new macros and once with the normal macros provided
%D with the natural table module.
%D
%D Let us start with the original macros:
%D
%D \starttyping
%D \bTABLE
%D   \bTR
%D     \bTD Text 1 \eTD
%D     \bTD Text 2 \eTD
%D   \eTR
%D   \bTR
%D     \bTD Text 3 \eTD
%D     \bTD Text 4 \eTD
%D   \eTR
%D \eTABLE
%D \stoptyping
%D
%D Watch how the new macros use less code:
%D
%D \starttyping
%D \startTABLE
%D \NC Text 1 \NC Text 2 \NC\NR
%D \NC Text 3 \NC Text 4 \NC\NR
%D \stopTABLE
%D \stoptyping
%D
%D The actual code differs from the prototype that it does not need
%D to collect whole rows and parse them but looks ahead instead.

\def\startTABLE
  {\dosingleempty\dostartTABLE}

\def\dostartTABLE[#1]%
  {\bgroup
   \bTABLE[#1]%
   \let\NC\doTABLENC
   \let\NR\doTABLENR
   \let\bTR\relax
   \let\bTD\relax
   \let\bTH\relax
   \let\bTN\relax}

\def\stopTABLE
  {\eTABLE
   \egroup}

\newconditional\inTABLEnc

\unexpanded\def\doTABLENR
  {\eTR
   \setfalse\inTABLEnc}

\unexpanded\def\doTABLENC
  {\futurelet\next\dodoTABLENC}

\def\dodoTABLENC
  {\ifx\next\doTABLENR \else
     \expandafter\dododoTABLENC
   \fi}

% \long\def\dododoTABLENC#1\NC
%   {\ifconditional\inTABLEnc\else\settrue\inTABLEnc\parseTR[][]\fi
%    \parseTD[][]#1\eTD\NC}

\long\def\dododoTABLENC#1\NC
 {\ifconditional\inTABLEnc\else\settrue\inTABLEnc\parseTR[][]\fi
  \dodoubleempty\parseTD#1\eTD\NC}

%D The related structure commands are also available:

\unexpanded\def\startTABLEhead{\dosingleempty\dostartTABLEhead} \let\stopTABLEhead\relax
\unexpanded\def\startTABLEnext{\dosingleempty\dostartTABLEnext} \let\stopTABLEnext\relax
\unexpanded\def\startTABLEbody{\dosingleempty\dostartTABLEbody} \let\stopTABLEbody\relax
\unexpanded\def\startTABLEfoot{\dosingleempty\dostartTABLEfoot} \let\stopTABLEfoot\relax

\long\def\dostartTABLEhead[#1]#2\stopTABLEhead{\appendtoks\doTABLEsection[#1]{#2}\to\TBLhead}
\long\def\dostartTABLEnext[#1]#2\stopTABLEnext{\appendtoks\doTABLEsection[#1]{#2}\to\TBLnext}
\long\def\dostartTABLEbody[#1]#2\stopTABLEbody{\appendtoks\doTABLEsection[#1]{#2}\to\TBLbody}
\long\def\dostartTABLEfoot[#1]#2\stopTABLEfoot{\appendtoks\doTABLEsection[#1]{#2}\to\TBLfoot}

\protect \endinput
