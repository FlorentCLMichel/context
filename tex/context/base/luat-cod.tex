%D \module
%D   [       file=luat-cod,
%D        version=2005.05.26,
%D          title=\CONTEXT\ Lua Macros,
%D       subtitle=Code,
%D         author=Hans Hagen,
%D           date=\currentdate,
%D      copyright=PRAGMA]
%C
%C This module is part of the \CONTEXT\ macro||package and is
%C therefore copyrighted by \PRAGMA. See mreadme.pdf for
%C details.

% \writestatus{loading}{ConTeXt Lua Macros / Code}

%D Originally we compiled the lua files externally and loaded
%D then at runtime, but when the amount grew, we realized that
%D we needed away to store them in the format, which is what
%D bytecode arrays do. And so the following is obsolete:
%D
%D \starttyping
%D \chardef\ctxluaembeddingmode \plusone
%D
%D 0 = external compilation and loading
%D 1 = runtime compilation and embedding
%D \stoptyping
%D
%D Allocation of \LUA\ engines has changed too. The original idea
%D was to have multiple \LUA\ instances and it worked that way for
%D several years. Hoewver in practice we used only one engine because
%D scripts need to share data anyway. So eventually \LUATEX\ got only
%D one instance. Because each call is reentrant there is not much
%D danger for crashes.

\def\ctxdirectlua{\directlua\zerocount}
\def\ctxlatelua  {\latelua  \zerocount}

%D Take your choice \unknown

\let\ctxlua       \ctxdirectlua
\let\luacode      \ctxdirectlua
\let\lateluacode  \ctxlatelua
\let\directluacode\ctxdirectlua

%D Reporting the version of \LUA\ that we use is done as follows:

\edef\luaversion{\ctxlua{tex.print(_VERSION)}}

%D We want to define \LUA\ related things in the format but
%D need to reload code because \LUA\ instances themselves are
%D not dumped into the format.

\newtoks\everyloadluacode
\newtoks\everyfinalizeluacode

\normaleveryjob{\the\everyloadluacode\the\everyfinalizeluacode\the\everyjob}

\newif\ifproductionrun

%D Here we operate in the \TEX\ catcode regime as we haven't yet defined
%D catcode regimes. A chicken or egg problem.

\long\def\startruntimeluacode#1\stopruntimeluacode % only simple code (load +init)
  {\ifproductionrun
     \global\let\startruntimeluacode\relax
     \global\let\stopruntimeluacode \relax
   \else
     \global\everyloadluacode\expandafter{\the\everyloadluacode#1}%
   \fi
   #1} % maybe no interference

\long\def\startruntimectxluacode#1\stopruntimectxluacode
  {\startruntimeluacode\ctxlua{#1}\stopruntimeluacode}

%D Next we load the initialization code.

\startruntimectxluacode
    environment = environment or { }
    environment.jobname = "\jobname"                            % tex.jobname
    environment.initex  = \ifproductionrun false \else true \fi % tex.formatname == ""
    environment.version = "\fmtversion"
\stopruntimectxluacode

% we start at 500, below this, we store predefined data (dumps)

\newcount\luabytecodecounter \luabytecodecounter=500

\startruntimectxluacode
    lua.bytedata = lua.bytedata or { }
\stopruntimectxluacode

%D Handy when we expand:

\let\stopruntimeluacode   \relax
\let\stopruntimectxluacode\relax

\long\def\lastexpanded{} % todo: elsewhere we use \@@expanded

\long\def\expanded#1{\long\xdef\lastexpanded{\noexpand#1}\lastexpanded}

%D More code:

% \def\ctxluabytecode#1% executes an already loaded chunk
%   {\ctxlua {
%       local str = ''
%       if lua.bytedata[#1] then
%           str = " from file " .. lua.bytedata[#1][1] .. " version " .. lua.bytedata[#1][2]
%       end
%       if lua.bytecode[#1] then
%           if environment.initex then
%               texio.write_nl("bytecode: executing blob " .. "#1" .. str)
%               assert(lua.bytecode[#1])()
%           else
%               texio.write_nl("bytecode: initializing blob " .. "#1" .. str)
%               assert(lua.bytecode[#1])()
%               lua.bytecode[#1] = nil
%           end
%       else
%          texio.write_nl("bytecode: invalid blob " .. "#1" .. str)
%       end
%   }}

\def\ctxluabytecode#1% executes an already loaded chunk
  {\ctxlua {
      local lbc = lua.bytecode
      if lbc[#1] then
          assert(lbc[#1])()
          if not environment.initex then
              lbc[#1] = nil
          end
      end
  }}

\def\ctxluabyteload#1#2% registers and compiles chunk
  {\global\advance\luabytecodecounter \plusone
   \expanded{\startruntimectxluacode
     lua.bytedata[\the\luabytecodecounter] = { "#1", "#2" }
   \stopruntimectxluacode}%
   \ctxlua {
     lua.bytedata[\the\luabytecodecounter] = { "#1", "#2" }
     lua.bytecode[\the\luabytecodecounter] = environment.luafilechunk("#1")
    }}

\def\ctxloadluafile#1#2% load a (either not compiled) chunk at runtime
  {\doifelsenothing{#2}
     {\ctxlua{environment.loadluafile("#1")}}
     {\ctxlua{environment.loadluafile("#1",#2)}}}

\def\registerctxluafile#1#2% name version
  {\ifproductionrun
     \ctxloadluafile{#1}{#2}%
   \else
     \ctxluabyteload{#1}{#2}% can go away
   \fi
   \global\everyloadluacode\expandafter\expandafter\expandafter{\expandafter\the\expandafter\everyloadluacode
      \expandafter\ctxluabytecode\expandafter{\the\luabytecodecounter}}%
   \ctxluabytecode{\the\luabytecodecounter}}

\everydump\expandafter{\the\everydump\ctxlua{luatex.dumpstate(environment.jobname..".lui",501)}}

\endinput
