%D \module
%D   [       file=catc-xml,
%D        version=2006.09.18,
%D          title=\CONTEXT\ Catcode Macros,
%D       subtitle=\XML\ Catcode Tables,
%D         author=Hans Hagen,
%D           date=\currentdate,
%D      copyright={PRAGMA / Hans Hagen \& Ton Otten}]
%C
%C This module is part of the \CONTEXT\ macro||package and is
%C therefore copyrighted by \PRAGMA. See mreadme.pdf for
%C details.

\writestatus{loading}{ConTeXt Catcode Regimes / XML}

\ifdefined \xmlcatcodesn \else \newcatcodetable \xmlcatcodesn \fi % normal
\ifdefined \xmlcatcodese \else \newcatcodetable \xmlcatcodese \fi % entitle
\ifdefined \xmlcatcodesr \else \newcatcodetable \xmlcatcodesr \fi % reduce

\startcatcodetable \xmlcatcodesn
    \catcode\tabasciicode      \spacecatcode
    \catcode\endoflineasciicode\endoflinecatcode
    \catcode\formfeedasciicode \endoflinecatcode
    \catcode\spaceasciicode    \spacecatcode
    \catcode\endoffileasciicode\ignorecatcode
    \catcode`\&                \activecatcode
    \catcode`\<                \activecatcode
    \catcode`\>                \othercatcode
    \catcode`\"                \othercatcode % probably not needed any more
    \catcode`\/                \othercatcode % probably not needed any more
    \catcode`\'                \othercatcode % probably not needed any more
    \catcode`\~                \othercatcode % probably not needed any more
    \catcode`\#                \othercatcode % probably not needed any more
    \catcode`\\                \othercatcode % probably not needed any more
\stopcatcodetable

\startcatcodetable \xmlcatcodese
    \catcode\tabasciicode      \spacecatcode
    \catcode\endoflineasciicode\endoflinecatcode
    \catcode\formfeedasciicode \endoflinecatcode
    \catcode\spaceasciicode    \spacecatcode
    \catcode\endoffileasciicode\ignorecatcode
    \catcode`\&                \activecatcode
    \catcode`\<                \activecatcode
    \catcode`\>                \activecatcode
    \catcode`\#                \activecatcode
    \catcode`\$                \activecatcode
    \catcode`\%                \activecatcode
    \catcode`\\                \activecatcode
    \catcode`\^                \activecatcode
    \catcode`\_                \activecatcode
    \catcode`\{                \activecatcode
    \catcode`\}                \activecatcode
    \catcode`\|                \activecatcode
    \catcode`\~                \activecatcode
\stopcatcodetable

\startcatcodetable \xmlcatcodesr
    \catcode\tabasciicode      \spacecatcode
    \catcode\endoflineasciicode\endoflinecatcode
    \catcode\formfeedasciicode \endoflinecatcode
    \catcode\spaceasciicode    \spacecatcode
    \catcode\endoffileasciicode\ignorecatcode
    \catcode`\&                \activecatcode
    \catcode`\<                \activecatcode
    \catcode`\>                \activecatcode
    \catcode`\#                \activecatcode
    \catcode`\$                \activecatcode
    \catcode`\%                \activecatcode
    \catcode`\\                \activecatcode
    \catcode`\^                \activecatcode
    \catcode`\_                \activecatcode
    \catcode`\{                \activecatcode
    \catcode`\}                \activecatcode
    \catcode`\|                \activecatcode
    \catcode`\~                \activecatcode
\stopcatcodetable

%D Next we hook in some active character definitions.

\letcatcodecommand \xmlcatcodesn `\&   \relax
\letcatcodecommand \xmlcatcodesn `\<   \relax

\letcatcodecommand \xmlcatcodese `\&   \relax
\letcatcodecommand \xmlcatcodese `\<   \relax

\letcatcodecommand \xmlcatcodesr `\&   \relax
\letcatcodecommand \xmlcatcodesr `\<   \relax

\letcatcodecommand \xmlcatcodese `\#   \relax
\letcatcodecommand \xmlcatcodese `\$   \relax
\letcatcodecommand \xmlcatcodese `\%   \relax
\letcatcodecommand \xmlcatcodese `\\   \relax
\letcatcodecommand \xmlcatcodese `\^   \relax
\letcatcodecommand \xmlcatcodese `\_   \relax
\letcatcodecommand \xmlcatcodese `\{   \relax
\letcatcodecommand \xmlcatcodese `\}   \relax
\letcatcodecommand \xmlcatcodese `\|   \relax
\letcatcodecommand \xmlcatcodese `\~   \relax

\letcatcodecommand \xmlcatcodesr `\#   \relax
\letcatcodecommand \xmlcatcodesr `\$   \relax
\letcatcodecommand \xmlcatcodesr `\%   \relax
\letcatcodecommand \xmlcatcodesr `\\   \relax
\letcatcodecommand \xmlcatcodesr `\^   \relax
\letcatcodecommand \xmlcatcodesr `\_   \relax
\letcatcodecommand \xmlcatcodesr `\{   \relax
\letcatcodecommand \xmlcatcodesr `\}   \relax
\letcatcodecommand \xmlcatcodesr `\|   \relax
\letcatcodecommand \xmlcatcodesr `\~   \relax

\let\xmlcatcodes   \xmlcatcodesn % beware, in mkiv we use \notcatcodes

%D We register the catcodetables at the \LUA\ end where some further
%D initializations take place.

\ifnum\texengine=\luatexengine

    \ctxlua {
        characters.define(
            {   % letter catcodes
                \number\xmlcatcodesn,
                \number\xmlcatcodese,
                \number\xmlcatcodesr,
            },
            {   % activate catcodes
                \number\xmlcatcodesn,
                \number\xmlcatcodese,
                \number\xmlcatcodesr,
            }
        )
       catcodes.register("xmlcatcodes",\number\xmlcatcodes)
    }

\fi

\endinput
