%D \module
%D   [       file=colo-ini,
%D        version=1997.4.1,
%D          title=\CONTEXT\ Color Macros,
%D       subtitle=Initialization,
%D         author=Hans Hagen,
%D           date=\currentdate,
%D      copyright={PRAGMA / Hans Hagen \& Ton Otten}]
%C
%C This module is part of the \CONTEXT\ macro||package and is
%C therefore copyrighted by \PRAGMA. See mreadme.pdf for 
%C details. 

\writestatus{loading}{Context Color Macros}

%D To do: stroke versus fill color 

\unprotect

%D Color support is not present in \TEX. Colorful output can 
%D however be accomplished by using specials. This also means 
%D that this support depends on the \DVI\ driver used. At the 
%D moment this module was written, still no decent standard on 
%D color specials has been agreed upon. We therefore decided to 
%D implement a mechanism that is as independant as possible of 
%D drivers. 
%D 
%D Color support shares with fonts that is must be implemented
%D in a way that permits processing of individual \DVI\ pages.
%D Furthermore it should honour grouping. The first condition
%D forces us to use a scheme that keeps track of colors at
%D page boundaries. This can be done by means of \TEX's
%D marking mechanism (\type{\mark}). 
%D 
%D When building pages, \TEX\ periodically looks at the
%D accumulated typeset contents and breaks the page when
%D suitable. At that moment, control is transfered to the
%D output routine. This routine takes care of building the
%D pagebody and for instance adds headers and footers. The page
%D can be broken in the middle of some colored text, but
%D headers and footers are often in black upon white or
%D background. If colors are applied there, they definitely
%D are used local, which means that they don't cross page
%D borders. 
%D 
%D Boxes are handled as a whole, which means that when we 
%D apply colors inside a box, those colors don't cross page 
%D boundaries, unless of course boxes are split or unboxed. 
%D Especially in interactive texts, colors are often used in 
%D such a local way: in boxes (buttons and navigational tools) 
%D or in the pagebody (backgrounds). 
%D 
%D So we can distinguish local colors, that don't cross
%D pages from global colors, of which we can end many pages 
%D later. The color macros will treat both types in a different 
%D way, thus gaining some speed.   
%D 
%D This module also deals with gray scales. Because similar
%D colors can end up in the same gray scale when printed in
%D black and white, we also implement a palet system that deals
%D with these matters. Because of fundamental differences
%D between color and gray scale printing, in \CONTEXT\ we also
%D differ between these. For historic reasons |<|we first
%D implemented gray scales using patterns of tiny periods|>| 
%D and therefore called them {\em rasters}. So don't be 
%D surprised if this term shows up.

\startmessages  dutch  library: colors
  title: kleur
      1: systeem -- is globaal actief
      2: systeem -- is lokaal actief
      3: -- is niet gedefinieerd
      4: systeem -- wordt geladen
      5: onbekend systeem --
      6: palet -- is beschikbaar
      7: palet -- is niet beschikbaar
      8: specificatie -- bij -- wordt zwart
      9: -- kleurruimte wordt niet ondersteund
     10: -- kleurruimte wordt ondersteund
     11: kleur wordt vertaald in grijs
\stopmessages

\startmessages  english  library: colors
  title: color
      1: system -- is global activated 
      2: system -- is local activated 
      3: -- is not defined
      4: system -- is loaded
      5: unknown system --
      6: palette -- is available
      7: palette -- is not available 
      8: specification -- at color -- becomes black
      9: -- color space is not supported 
     10: -- color space is supported 
     11: color is converted to gray
\stopmessages

\startmessages  german  library: colors
  title: farbe
      1: system -- ist global aktiviert
      2: system -- ist lokal aktiviert
      3: -- ist undefiniert
      4: system -- ist geladen
      5: unbekanntes System --
      6: palette -- ist verfuegbar
      7: palette -- ist nicht verfuegbar
      8: Spezifikation -- bei Farbe -- wird schwarz
      9: -- Farbraum wird nicht unterstuetzt
     10: -- Farbraum wird unterstuetzt 
     11: Farbe wird in Grau umgewandelt
\stopmessages

\startmessages  czech  library: colors
  title: barva
      1: system -- je globalne aktivovana
      2: system -- je lokalne activovana
      3: -- neni definovana
      4: system -- je nacten
      5: neznamy system --
      6: palette -- je k dispozici
      7: palette -- neni k dispozici
      8: specifikace -- v barve -- bude cerna
      9: -- prostor barev neni podporovan
     10: -- prostor barev je podporovan
     11: barva je prevedena na sed
\stopmessages

\startmessages  italian  library: colors
  title: colore
      1: sistema -- attivato globalmente
      2: sistema -- attivato localmente
      3: -- non definito
      4: sistema -- caricato
      5: sistema -- sconosciuto 
      6: tavolozza -- resa disponibile
      7: tavolozza -- non disponibile
      8: specifica -- del colore -- convertita in nero
      9: spazio dei colori -- non supportato
     10: spazio dei colori -- supportato
     11: il colore � convertito in grigio
\stopmessages

\startmessages  norwegian  library: colors
  title: farge
      1: system -- er aktivert globalt
      2: system -- er aktivert lokalt
      3: -- er udefinert
      4: system -- er lest inn 
      5: ukjent system --
      6: palett -- er tilgjengelig
      7: palett -- er ikke tilgjengelig
      8: spesifikasjon -- for farge -- gir kun svart
      9: -- fargerom er ikke st�ttet
     10: -- fargerom er st�ttet 
     11: fargen vil bli vist som gr�
\stopmessages

\startmessages  romanian  library: colors
  title: culori
      1: sistem -- este activata global 
      2: sistem -- este activata local
      3: -- nu este definita
      4: sistem -- este incarcata
      5: sistem -- necunoscuta
      6: paleta -- este disponibila
      7: palette -- nu este disponibila 
      8: specificatia -- la culoarea -- devine neagra
      9: spatiul de culoare -- nu este suportat
     10: spatiul de culoare -- este suportat 
     11: culoarea este convertita la gri
\stopmessages

%D \macros
%D   {definecolor}
%D
%D We will enable users to specify colors in \kap{RGB} and 
%D \kap{CMYK} color spaces or gray scales using 
%D
%D \showsetup{\y!definecolor}
%D
%D For example:
%D
%D \starttypen
%D \definecolor [SomeKindOfRed] [r=.8,g=.05,b=.05]
%D \stoptypen
%D 
%D Such color specifications are saved in a macro in the 
%D following way:
%D 
%D \starttypen 
%D \setvalue{\??cr name}{R:r:g:b}
%D \setvalue{\??cr name}{C:c:m:y:k}
%D \setvalue{\??cr name}{S:s}
%D \stoptypen
%D
%D Gray scales are specified with the \type{s} parameter, 
%D where the is derived from screen.

\def\colorlist{}

\def\@@cl@@z{0} 
\def\@@cl@@o{1} 

\def\definecolor%
  {\dodoubleargument\dodefinecolor}

\def\dodefinecolor[#1][#2]%
  {% beware
\saveglobal
     \addtocommalist{#1}\colorlist
\restoreglobal
   \doifassignmentelse{#2}
     {\let\@@cl@@r=\@@cl@@z\let\@@cl@@g=\@@cl@@z\let\@@cl@@b=\@@cl@@z
      \let\@@cl@@c=\@@cl@@z\let\@@cl@@m=\@@cl@@z\let\@@cl@@y=\@@cl@@z
      \let\@@cl@@k=\@@cl@@z\let\@@cl@@s=\@@cl@@z
      \getparameters[\??cl @@][#2]%
      \doifelse{\@@cl@@r\@@cl@@g\@@cl@@b}{\@@cl@@z\@@cl@@z\@@cl@@z}
        {\doifelse{\@@cl@@c\@@cl@@m\@@cl@@y\@@cl@@k}{\@@cl@@z\@@cl@@z\@@cl@@z\@@cl@@z}
           {\doifelse{\@@cl@@s}{\@@cl@@z}
              {\showmessage{\m!colors}{8}{{[#2]},#1}%
               \redoglobal\setevalue{\??cr#1}{S:\@@cl@@z}}
              {\redoglobal\setevalue{\??cr#1}{S:\@@cl@@s}}}
           {\redoglobal\setevalue{\??cr#1}{C:\@@cl@@c:\@@cl@@m:\@@cl@@y:\@@cl@@k}}}
        {\redoglobal\setevalue{\??cr#1}{R:\@@cl@@r:\@@cl@@g:\@@cl@@b}}}
     {\doifdefinedelse{\??cr#2}
        {\redoglobal\setevalue{\??cr#1}{\getvalue{\??cr#2}}}
        {\showmessage{\m!colors}{3}{#1}}}%
   \dodoglobal\setvalue{#1}{\color[#1]}}

%D The names of colors are stored in a comma separated list 
%D only for the purpose of showing them with \type{\showcolor}.
%D 
%D Colors can be defined global by using \type{\doglobal}, 
%D like in 
%D 
%D \startbuffer
%D \doglobal\definecolor [SomeKindOfRed] [r=.8,g=.05,b=.05]
%D \stopbuffer
%D
%D \typebuffer
%D
%D \haalbuffer
%D 
%D This color shows up as \color[SomeKindOfRed]{some kind 
%D of red}.

%D \macros
%D   {setupcolor}
%D
%D Color definitions can be grouped in files with the name: 
%D
%D \starttypen
%D \f!colorprefix-identifier.tex
%D \stoptypen
%D
%D where \type{\f!colorprefix} is \unprotect {\tttf \f!colorprefix}. 
%D Loading such a file is done by \protect
%D
%D \showsetup{\y!setupcolor} 
%D
%D Some default colors are specified in \type{colo-rgb.tex},
%D which is loaded into the format by: 
%D
%D \starttypen
%D \setupcolor[rgb]
%D \stoptypen
 
\def\colorstyle{}

\def\setupcolor%
  {\dosingleargument\dosetupcolor}

\def\dosetupcolor[#1]%
  {\doifnot{#1}{\colorstyle}
     {\def\colorstyle{#1}%
      \def\dodosetupcolor##1%
        {\makeshortfilename[\f!colorprefix\truefilename{##1}]%
         \readsysfile{\shortfilename}%
           {\showmessage{\m!colors}{4}{\colorstyle}}
           {\showmessage{\m!colors}{5}{\colorstyle}}}%
      \processcommalist[#1]\dodosetupcolor}}

%D When typesetting for paper, we prefer using the \kap{CMYK} 
%D color space, but for on||screen viewing we prefer \kap{RGB} 
%D (the previous implementation supported only this scheme). 
%D Independant of such specifications, we support some automatic 
%D conversions: 
%D 
%D \startopsomming[opelkaar]
%D \som  convert all colors to \kap{RGB}
%D \som  convert all colors to \kap{CMYK}
%D \som  convert all colors to gray scales
%D \stopopsomming
%D 
%D We also support optimization of colors to gray scales. 
%D 
%D \startopsomming[verder]
%D \som  reduce gray colors to gray scales
%D \som  reduce \kap{CMY} components to \kap{K}
%D \stopopsomming
%D 
%D These options are communicated by means of: 

\newif\ifRGBsupported
\newif\ifCMYKsupported
\newif\ifconverttoGRAY % obsolete
\newif\ifpreferGRAY
\newif\ifGRAYprefered
\newif\ifreduceCMYK

%D needs: 

% \newif\ifMPgraphics
% \newif\ifinpagebody

%D The last boolean controls reduction of \kap{CMYK} to 
%D \kap{CMY} colors. When set to true, the black component 
%D is added to the other ones. 

%D Color modes are entered using the next set of commands. 
%D The \type{\stop} alternatives are implemented in a way 
%D that permits non||grouped use. 
%D
%D The, for this module redundant, check if we are in color 
%D mode is needed when we use these macros in other modules.  

\def\dostartcolormodeR#1:#2:#3\od%
  {\bgroup
   %\def\@@cl@@r{#1}\def \@@cl@@g{#2}\def \@@cl@@b{#3}%
   \edef\@@cl@@r{#1}\edef\@@cl@@g{#2}\edef\@@cl@@b{#3}%
   \ifpreferGRAY\ifx\@@cl@@r\@@cr@@g\ifx\@@cl@@r\@@cl@@b
     \GRAYpreferedtrue
   \fi\fi\fi
   \ifincolor\else\RGBsupportedfalse\CMYKsupportedfalse\fi
   \ifGRAYprefered
     \dostartgraycolormode\@@cl@@r
   \else\ifRGBsupported
     \dostartrgbcolormode\@@cl@@r\@@cl@@g\@@cl@@b
   \else\ifCMYKsupported
     \convertRGBtoCMYK\@@cl@@r\@@cl@@g\@@cl@@b
     \dostartcmykcolormode\@@cl@@c\@@cl@@m\@@cl@@y\@@cl@@k
   \else
     \convertRGBtoGRAY\@@cl@@r\@@cl@@g\@@cl@@b
     \dostartgraycolormode\@@cl@@s
   \fi\fi\fi
   \egroup}

\def\dostartcolormodeC#1:#2:#3:#4\od%
  {\bgroup
   %\def\@@cl@@c{#1}\def \@@cl@@m{#2}\def \@@cl@@y{#3}\def \@@cl@@k{#4}%
   \edef\@@cl@@c{#1}\edef\@@cl@@m{#2}\edef\@@cl@@y{#3}\edef\@@cl@@k{#4}%
   \ifpreferGRAY\ifx\@@cl@@k\@@cl@@z\ifx\@@cl@@c\@@cr@@m\ifx\@@cl@@c\@@cl@@y
     \GRAYpreferedtrue
   \fi\fi\fi\fi
   \ifincolor\else\RGBsupportedfalse\CMYKsupportedfalse\fi 
   \ifGRAYprefered
     \dostartgraycolormode\@@cl@@c
   \else\ifCMYKsupported
     \ifreduceCMYK
       \convertCMYKtoCMY\@@cl@@c\@@cl@@m\@@cl@@y\@@cl@@k
       \dostartcmykcolormode\@@cl@@c\@@cl@@m\@@cl@@y\@@cl@@k
     \else
       \dostartcmykcolormode\@@cl@@c\@@cl@@m\@@cl@@y\@@cl@@k
     \fi
   \else\ifRGBsupported
     \convertCMYKtoRGB\@@cl@@c\@@cl@@m\@@cl@@y\@@cl@@k
     \dostartrgbcolormode\@@cl@@r\@@cl@@g\@@cl@@b
   \else
     \convertCMYKtoGRAY\@@cl@@c\@@cl@@m\@@cl@@y\@@cl@@k
     \dostartgraycolormode\@@cl@@s
   \fi\fi\fi
   \egroup}

\def\dostartcolormodeS#1\od%
  {\dostartgraycolormode{#1}}

%D Prefering gray is not the same as converting to gray. Conversion treats 
%D each color components  in a different way, while prefering is just a 
%D reduction and thus a space||saving option.

%D \macros
%D   {startcolormode,stopcolormode,permitcolormode}
%D 
%D We use \type{\stopcolormode} to reset the color in 
%D whatever color space and do so by calling the corresponding
%D special. Both commands can be used for fast color 
%D switching, like in colored verbatim,

\newif\ifpermitcolormode \permitcolormodetrue

\def\dostartcolormode#1:%
  {\getvalue{dostartcolormode#1}}  

\def\startcolormode#1% 
  {\ifincolor\ifpermitcolormode 
     \doifcolorelse{#1}
       {\getcurrentcolorspecs{#1}%
        \expandafter\dostartcolormode\currentcolorspecs\od}
       {\dostopcolormode}%
   \fi\fi}

\def\stopcolormode%
  {\ifincolor\ifpermitcolormode\dostopcolormode\fi\fi}

%D \macros
%D   {startregistercolor,stopregistercolor,permitcolormode}
%D
%D If you only want to register a color, the switch \type 
%D {\ifpermitcolormode} can be used. That way the nested 
%D colors know where to go back to. 

\def\startregistercolor[#1]%
  {\permitcolormodefalse\startcolor[#1]\permitcolormodetrue}

\def\stopregistercolor%
  {\permitcolormodefalse\stopcolor\permitcolormodetrue}

%D We use some reserved names for local color components. 
%D Consistent use of these scratch variables saves us 
%D unneccessary hash entries. 
%D 
%D \starttypen 
%D \@@cl@@r \@@cl@@g \@@cl@@b
%D \@@cl@@c \@@cl@@m \@@cl@@y \@@cl@@k
%D \@@cl@@s
%D \stoptypen
%D 
%D We implement several conversion routines. 
%D
%D \starttypen
%D \convertRGBtoCMYK  {r} {g} {b} 
%D \convertRGBtoGRAY  {r} {g} {b} 
%D \convertCMYKtoRGB  {c} {m} {y} {k} 
%D \convertCMYKtoGRAY {c} {m} {y} {k}
%D \convertCMYKtoCMY  {c} {m} {y} {k}
%D \stoptypen
%D 
%D The relation between \kap{Gray}, \kap{RGB} and \kap{CMYK} 
%D is: 
%D 
%D \plaatsformule[-]
%D   \startformule
%D   G = .30r + .59g + .11b 
%D     = 1.0 - \min(1.0,\ .30c + .59m + .11y + k)
%D   \stopformule
%D 
%D When converting from \kap{CMYK} to \kap{RGB} we use the 
%D formula: 
%D 
%D \plaatsformule[-]
%D   \startformule
%D   \eqalign
%D     {r &= 1.0 - \min(1.0,\ c+k) \cr
%D      g &= 1.0 - \min(1.0,\ m+k) \cr
%D      b &= 1.0 - \min(1.0,\ y+k)}
%D   \stopformule
%D 
%D In the conversion routine the color components are calculated 
%D in three digits precision. 

\def\realcolorvalue#1%  
  {\ifnum     #1<10   0.00\the#1%
   \else\ifnum#1<100  0.0\the#1%
   \else\ifnum#1<1000 0.\the#1%
   \else              1\fi\fi\fi}

\def\doconvertCMYKtoRGB#1\k#2\to#3%
  {\scratchdimen=#1\s!pt
   \advance\scratchdimen by #2\s!pt\relax
   \ifdim\scratchdimen>1\s!pt
     \scratchdimen=-1\s!pt
   \else
     \scratchdimen=-\scratchdimen
   \fi
   \advance\scratchdimen by 1\s!pt
   \multiply\scratchdimen by 1000
   \scratchcounter=\scratchdimen
   \advance\scratchcounter by \!!medcard
   \divide\scratchcounter by \!!maxcard
   \edef#3{\realcolorvalue\scratchcounter}}

\def\convertCMYKtoRGB#1#2#3#4%
  {\doconvertCMYKtoRGB#1\k#4\to\@@cl@@r
   \doconvertCMYKtoRGB#2\k#4\to\@@cl@@g
   \doconvertCMYKtoRGB#3\k#4\to\@@cl@@b}

\def\doconvertRGBtoCMYK#1\to#2%
  {\scratchdimen=#1\s!pt
   \multiply\scratchdimen by 1000
   \scratchcounter=\scratchdimen
   \advance\scratchcounter by \!!medcard
   \divide\scratchcounter by \!!maxcard
   \scratchcounter=-\scratchcounter
   \advance\scratchcounter by 1000
   \edef#2{\realcolorvalue\scratchcounter}}

\def\convertRGBtoCMYK#1#2#3%
  {\doconvertRGBtoCMYK#1\to\@@cl@@c
   \doconvertRGBtoCMYK#2\to\@@cl@@m
   \doconvertRGBtoCMYK#3\to\@@cl@@y
   \let\@@cl@@k=\@@cl@@z}

%D The following switch is mainly meant for (hidden) 
%D documentation purposes. 

\newif\ifweightGRAY \weightGRAYtrue

\def\nGRAYfactor{333.333}
\def\rGRAYfactor{\ifweightGRAY300\else\nGRAYfactor\fi}
\def\gGRAYfactor{\ifweightGRAY590\else\nGRAYfactor\fi}
\def\bGRAYfactor{\ifweightGRAY110\else\nGRAYfactor\fi}

\def\convertRGBtoGRAY#1#2#3%  
  {\scratchdimen=#1\s!pt
   \scratchdimen=\rGRAYfactor\scratchdimen
   \scratchcounter=\scratchdimen
   \scratchdimen=#2\s!pt
   \scratchdimen=\gGRAYfactor\scratchdimen
   \advance\scratchcounter by \scratchdimen
   \scratchdimen=#3\s!pt
   \scratchdimen=\bGRAYfactor\scratchdimen
   \advance\scratchcounter by \scratchdimen
   \advance\scratchcounter by \!!medcard
   \divide\scratchcounter by \!!maxcard 
   \edef\@@cl@@s{\realcolorvalue\scratchcounter}}

\def\convertCMYKtoGRAY#1#2#3#4%  
  {\convertCMYKtoRGB{#1}{#2}{#3}{#4}%
   \convertRGBtoGRAY\@@cl@@r\@@cl@@g\@@cl@@b}

\def\doconvertCMYKtoCMY#1\k#2\to#3%
  {\scratchdimen=#1\s!pt
   \advance\scratchdimen by #2\s!pt\relax
   \ifdim\scratchdimen>1\s!pt
     \scratchdimen=1\s!pt
   \else
     \scratchdimen=\scratchdimen
   \fi
   \multiply\scratchdimen by 1000
   \scratchcounter=\scratchdimen
   \advance\scratchcounter by \!!medcard
   \divide\scratchcounter by \!!maxcard
   \edef#3{\realcolorvalue\scratchcounter}}

\def\convertCMYKtoCMY#1#2#3#4%
  {\doconvertCMYKtoCMY#1\k#4\to\@@cl@@c
   \doconvertCMYKtoCMY#2\k#4\to\@@cl@@m
   \doconvertCMYKtoCMY#3\k#4\to\@@cl@@y
   \let\@@cl@@k=\@@cl@@z}

%D Before we present the color macros, we first define the 
%D setup command. This command takes care of setting up the 
%D booleans that control local and global behavior (more on 
%D that later) and conversion to other color spaces. 

\newif\ifincolor
\newif\iflocalcolor

\def\setupcolors%
  {\dosingleargument\dosetupcolors}

\def\dosetupcolors[#1]%
 {\getparameters[\??cl][#1]%
  \doifelse{\@@clreductie}{\v!ja}
    {\reduceCMYKtrue}
    {\reduceCMYKfalse}% 
  \doifelse{\@@clrgb}{\v!nee}
    {\showmessage{\m!colors}{9}{\v!rgb}\RGBsupportedfalse} 
    {\showmessage{\m!colors}{10}{\v!rgb}\RGBsupportedtrue}%
  \doifelse{\@@clcmyk}{\v!nee}
    {\showmessage{\m!colors}{9}{\v!cmyk}\CMYKsupportedfalse}
    {\showmessage{\m!colors}{10}{\v!cmyk}\CMYKsupportedtrue}%
  \processaction
    [\@@clconversie]
    [    \v!ja=>\preferGRAYtrue,
        \v!nee=>\preferGRAYfalse,
     \v!altijd=>\preferGRAYtrue
                \RGBsupportedfalse
                \CMYKsupportedfalse,
    \s!default=>\preferGRAYfalse,
    \s!unknown=>\preferGRAYfalse]%
  \ifRGBsupported 
    \converttoGRAYfalse
    \forcegrayMPcolorsfalse
  \else\ifCMYKsupported
    \converttoGRAYfalse
    \forcegrayMPcolorsfalse
    \convertMPcolorstrue
    \ifreduceCMYK
      \reduceMPcolorstrue
    \fi
  \else
    \converttoGRAYtrue
    \forcegrayMPcolorstrue
    \convertMPcolorsfalse
    \reduceMPcolorsfalse
    \showmessage{\m!colors}{11}{}%
  \fi\fi
  \processaction
    [\@@clstatus]
    [\v!globaal=>\incolortrue\localcolorfalse
                 \showmessage{\m!colors}{1}{\colorstyle}, 
      \v!lokaal=>\incolortrue\localcolortrue
                 \showmessage{\m!colors}{2}{\colorstyle}, 
       \v!start=>\let\@@clstatus=\v!globaal
                 \incolortrue\localcolorfalse
                 \showmessage{\m!colors}{1}{\colorstyle}, 
        \v!stop=>\incolorfalse\localcolorfalse
                 \forcegrayMPcolorstrue]}

%D \macros
%D   {doifcolorelse}
%D
%D Switching to a color is done by means of the following 
%D command. Later on we will explain the use of palets.  We 
%D define ourselves a color conditional first.  

\def\currentpalet{}

\def\doifcolorelse#1%
  {\doifdefinedelse{\??cr\currentpalet#1}}

\def\getcurrentcolorspecs#1%
  {\edef\currentcolorspecs{\getvalue{\??cr\currentpalet#1}}}

%D \macros
%D   {localstartcolor,localstopcolor}
%D
%D Simple color support, that is without nesting, is provided 
%D by:

\def\localstartcolor[#1]% 
  {\ifincolor
     \localcolortrue
     \doglobalstartcolor[#1]%
   \fi}

\def\localstopcolor%
  {\ifincolor
     \doglobalstopcolor
   \fi}

%D \macros
%D   {startcolor,stopcolor}
%D
%D The more save method, the one that saves the current color 
%D state and returns to this state afterward, is activated by: 
%D
%D \showsetup{\y!startcolor}

\def\startcolor[#1]%
  {\ifincolor
     \doglobalstartcolor[#1]%
   \fi}

\def\stopcolor%
  {\ifincolor
     \doglobalstopcolor
   \fi}

%D This macros call the global color switching ones. Starting 
%D a global, i.e. a possible page boundary crossing, color 
%D mode also sets a \type{\mark} in \TEX's internal list. 

\newcount\colorlevel

\setvalue{\??cl0C}{}  % saved color 
\setvalue{\??cl0S}{}  % stop command

%D We keep a positive color stack for foreground colors, and 
%D a negative one for backgrounds. Not that brilliant a 
%D solution, but it suits. The signs are swapped when the 
%D page ornaments are typeset. 

\def\@@colorplus {+}  
\def\@@colorminus{-}

\def\dodoglobalstartcolor[#1]%
  {\xdef\currentcolor{\getvalue{\??cl\the\colorlevel C}}%
   \global\advance\colorlevel by \@@colorplus1
   \setxvalue{\??cl\the\colorlevel C}{#1}%
  %\debuggerinfo{\m!colors}
  %  {start #1 at level \the\colorlevel}%
   \doifelsenothing{#1}
     {\setxvalue{\??cl\the\colorlevel C}{\currentcolor}%
      \setgvalue{\??cl\the\colorlevel S}{\donoglobalstopcolor}}
     {\doifelse{#1}{\currentcolor}
        {\setgvalue{\??cl\the\colorlevel S}{\donoglobalstopcolor}}
        {\doifcolorelse{#1}
           {\docolormark{#1}%
            \setgvalue{\??cl\the\colorlevel S}{\dodoglobalstopcolor}%
            \startcolormode{#1}}
           {\setgvalue{\??cl\the\colorlevel S}{\donoglobalstopcolor}%
            \showmessage{\m!colors}{3}{#1}}}}}

\def\doglobalstartcolor[#1]%
  {\ifnum\colorlevel=0
     \doifelsenothing{#1}
       {\setgvalue{\??cl\the\colorlevel S}{}}
       {\dodoglobalstartcolor[#1]}%
   \else
     \dodoglobalstartcolor[#1]%
   \fi
   \ignorespaces}

\def\donoglobalstopcolor%
  {\ifnum\colorlevel=0 \else
     \xdef\currentcolor{\getvalue{\??cl\the\colorlevel C}}%
    %\debuggerinfo{\m!colors}
    %  {stop \currentcolor\normalspace at level \the\colorlevel}%
     \global\advance\colorlevel by \@@colorminus1
   \fi}

\def\dodoglobalstopcolor%
  {\ifnum\colorlevel=0 \else
     \donoglobalstopcolor
     \xdef\previouscolor{\getvalue{\??cl\the\colorlevel C}}%
     \ifnum\colorlevel=0
       \docolormark{}%
       \stopcolormode
     \else % let's do a bit redundant testing here
       \docolormark{\previouscolor}% 
       \doifelsenothing{\previouscolor}
         {\dostopcolormode}
         {\doifcolorelse{\previouscolor} 
            {\doifnot{\currentcolor}{\previouscolor}
               {\startcolormode{\previouscolor}}} 
            {\dostopcolormode}}%
     \fi
   \fi}

\def\doglobalstopcolor%
  {\getvalue{\??cl\the\colorlevel S}}

%D We don't use grouping and save each stop alternative. This 
%D permits be especially useful in for instance local color 
%D support in verbatim. Using \type{\bgroup}||\type{\egroup} 
%D pairs could interfere with calling commands 

%D This color mechanism takes care of nested colors, like in:
%D
%D \startbuffer
%D \kleur[groen]{groen \kleur[groen]{groen \kleur[rood]{rood}} groen}
%D \kleur[groen]{groen \kleur[]{groen \kleur[rood]{rood}} groen}
%D \kleur[groen]{groen \kleur[rood]{rood \kleur[rood]{rood}} groen}
%D \kleur[groen]{groen \kleur[groen]{groen \kleur[]{groen}} groen}
%D \kleur[groen]{groen \kleur[rood]{rood} groen}
%D \kleur[groen]{groen \kleur[]{groen} groen}
%D \kleur[]{zwart \kleur[rood]{rood} zwart}
%D \kleur[]{zwart}
%D \stopbuffer
%D 
%D \typebuffer
%D 
%D or 
%D 
%D \startvoorbeeld
%D \startregels
%D \haalbuffer
%D \stopregels
%D \stopvoorbeeld
%D
%D Crossing page boundaries is of course also handled. 
%D Undefined or empty color specifications are treated as 
%D efficient as possible.  
%D 
%D \startbuffer
%D \startkleur[groen]
%D   [groen] \input tufte [groen] \par 
%D   \startkleur[]
%D     [groen] \input knuth [groen] \par 
%D     \startkleur[rood]
%D       [rood] \input tufte [rood] \par 
%D       \startkleur[geel]
%D         [geel] \input knuth [geel] \par 
%D       \stopkleur
%D       [rood] \input tufte [rood] \par 
%D     \stopkleur
%D     [groen] \input knuth [groen] \par 
%D   \stopkleur
%D   [groen] \input tufte [groen] \par 
%D \stopkleur
%D \stopbuffer
%D 
%D \startopelkaar
%D \haalbuffer
%D \stopopelkaar
%D 
%D These quotes are typeset by saying:
%D 
%D \typebuffer
 
%D We already mentioned that colors interfere with building
%D the pagebody. This means that when the page is composed,
%D the colors temporary have to be reset. After the page is
%D shipped out, we have to revive the current color. 
%D 
%D We use \type{\marks} to keep track of colors across page 
%D boundaries. Unfortunately standard \TEX\ supports only one mark, 
%D and using this one for color support only would be a waste. 
%D We therefore use an adapted version of J.~Fox's multiple mark
%D mechanism as (re|)|implemented in \module{supp-mrk}. 

\doifdefinedelse{newmark}
  {\newmark\colormark}  
  {\def\colormark#1{}}

%D Using this mark mechanism with lots of colors has one 
%D major drawback: \TEX's memory tends to overflow when 
%D very colorful text is stored in a global box. Even worse is that
%D the processing time grows considerably. We therefore support
%D local as well as global color switching. 
%D 
%D Of the next macros, \type{\popcolor} is to be used after
%D the actual \type{\shipout} and \type{\startcolorpage} and
%D \type{\stopcolorpage} are called when entering and leaving
%D the \type{\pagebody} builder. In case of emergencies
%D \type{\pushcolor} can be used to undo the current color, 
%D for instance when insertions are appended to the page.  
%D
%D Out of efficiency we only use marks when needed. The next 
%D macro tries to find out if indeed a mark should be set. 
%D This macro uses the boolean \type{\ifinpagebody}, which can 
%D be defined and set in the module that handles the pagebody. 

\def\docolormark#1%
  {\iflocalcolor \else \ifinpagebody \else \ifinframed \else 
     \dodocolormark{#1}%
   \fi \fi \fi}

\let\lastcolormark=\empty

\def\dodocolormark#1%
  {\doifnot{#1}{\lastcolormark}
     {\xdef\lastcolormark{#1}%
      \expandafter\setmark\expandafter\colormark\expandafter{\lastcolormark}}}

%D \macros
%D   {pushcolor, popcolor}
%D
%D Pushing the current state in the output routine simply comes 
%D to resetting the color to black, while popping restores the 
%D color state to that of before the break. 

\def\pushcolor%
  {\stopcolormode} 

\def\popcolor%
  {\doifsomething{\getbotmark\colormark}
     {%\debuggerinfo{\m!colors}{popping \getbotmark\colormark}%
      \startcolormode{\getbotmark\colormark}}}

\def\popsplitcolor% 
  {\getsplitmarks\colormark  % hier wel 
   \doifsomething{\getsplitbotmark\colormark}
     {%\debuggerinfo{\m!colors}{split popping \getsplitbotmark\colormark}%
      \startcolormode{\getsplitbotmark\colormark}}}

%D \macros
%D   {startcolorpage, stopcolorpage}
%D
%D Local use can be forced with the next two macros. Nesting 
%D is still supported but colors are no longer marked. 

%D \starttypen
%D \def\startcolorpage%    
%D   {\bgroup
%D    \let\docolormark=\gobbleoneargument
%D    \edef\savedcolorlevel{\the\colorlevel}%
%D    \ifnum\colorlevel>0
%D      \stopcolormode
%D    \fi
%D    \global\colorlevel=1000\relax} % a save offset 
%D 
%D \def\stopcolorpage%
%D   {\global\colorlevel=\savedcolorlevel\relax
%D    \egroup}
%D \stoptypen

%D The next alternative makes (simple) color separation more 
%D easy. It also supports nested colors in page backgrounds 
%D and texts. 

\def\startcolorpage%    
  {\bgroup
   \def\@@colorplus{-}%
   \def\@@colorminus{+}%
   \let\docolormark=\gobbleoneargument
   \edef\savedcolorlevel{\the\colorlevel}%
   \global\colorlevel=0       % before \localstartcolor of  
   \localstartcolor[black]}   % course, ugly bug removed

\def\stopcolorpage%
  {\localstopcolor
   \global\colorlevel=\savedcolorlevel\relax
   \egroup}

%D \macros
%D   {color,graycolor}
%D 
%D This leaves the simple color command: 
%D
%D \showsetup{\y!color}
%D \showsetup{\y!graycolor}
%D
%D Which can be used straightforward: \color[groen]{green as gras}. 
%D We want color support to be similar to font support and 
%D therefore implement \type{\color} as:

\unexpanded\def\color[#1]%
  {\groupedcommand
     {\startcolor[#1]} 
     {\stopcolor}}

\unexpanded\def\graycolor[#1]% not \gray because this is a color
  {\groupedcommand
     {\RGBsupportedfalse\CMYKsupportedfalse\startcolor[#1]}
     {\stopcolor}}

\let\grey\graycolor

%D This implementation enables use of defined colors like:
%D
%D \starttypen
%D Look at the {\brightgreen bright} side of life and get 
%D yourself no \red{red} head!
%D \stoptypen

%D \macros 
%D   {colorvalue, grayvalue}
%D 
%D We can typeset the color components using \type{\colorvalue} and 
%D \type{\grayvalue}. The commands: 
%D
%D \startbuffer
%D color value of SomeKindOfRed: \colorvalue{SomeKindOfRed} \crlf
%D gray value of SomeKindOfRed: \grayvalue{SomeKindOfRed}
%D \stopbuffer
%D
%D \typebuffer
%D
%D show us: 
%D
%D \startvoorbeeld
%D \haalbuffer
%D \stopvoorbeeld

\def\realcolorformat#1% 
  {\ifnum#1<10        0.00\the#1%
   \else\ifnum#1<100  0.0\the#1%
   \else\ifnum#1<1000 0.\the#1%
   \else              1.000\fi\fi\fi}

\def\colorformatseparator{ }

\def\dodoformatcolor#1%
  {\scratchdimen=#1\s!pt\relax
   \ifdim\scratchdimen>1\s!pt
     \scratchdimen=1\s!pt
   \fi
   \multiply\scratchdimen by 1000
   \scratchcounter=\scratchdimen
   \advance\scratchcounter by \!!medcard
   \divide\scratchcounter by \!!maxcard \relax
   \realcolorformat\scratchcounter}

\def\doformatcolorR#1:#2:#3\od%
  {\dodoformatcolor{#1}\colorformatseparator
   \dodoformatcolor{#2}\colorformatseparator 
   \dodoformatcolor{#3}}

\def\doformatcolorC#1:#2:#3:#4\od%
  {\dodoformatcolor{#1}\colorformatseparator 
   \dodoformatcolor{#2}\colorformatseparator 
   \dodoformatcolor{#3}\colorformatseparator 
   \dodoformatcolor{#4}}

\def\doformatcolorS#1\od%
  {\dodoformatcolor{#1}}

\def\doformatcolor#1:%
  {\getvalue{doformatcolor#1}}  

\def\colorvalue#1% 
  {\doifcolorelse{#1}
     {\getcurrentcolorspecs{#1}%
      \expandafter\doformatcolor\currentcolorspecs\od}
     {}}

\def\doformatgrayR#1:#2:#3\od%
  {\convertRGBtoGRAY{#1}{#2}{#3}%
   \dodoformatcolor\@@cl@@s}

\def\doformatgrayC#1:#2:#3:#4\od%
  {\convertCMYKtoGRAY{#1}{#2}{#3}{#4}%
   \dodoformatcolor\@@cl@@s}

\def\doformatgrayS#1\od%
  {\dodoformatcolor{#1}}

\def\doformatgray#1:%
  {\getvalue{doformatgray#1}}  

\def\grayvalue#1% 
  {\doifcolorelse{#1}
     {\getcurrentcolorspecs{#1}%
      \expandafter\doformatgray\currentcolorspecs\od}
     {}}

%D \macros
%D   {localstartraster,localstopraster,
%D    startraster,stopraster}   
%D 
%D The previous conversions are not linear and treat each color
%D component according to human perception curves. Pure gray
%D (we call them rasters) has equal color components. In
%D \CONTEXT\ rasters are only used as backgrounds and these
%D don't cross page boundaries in the way color does. Therefore
%D we don't need stacks and marks. Just to be compatible with
%D color support we offer both 'global' and 'local' commands.  

\def\localstartraster[#1]%       
 %{\doifinstringelse{.}{#1} % no, we also want 0 and 1  
  {\doifelsenothing{#1}
     {\dostartgraymode{\@@rsraster}}
     {\dostartgraymode{#1}}}

\def\localstopraster%        
  {\dostopgraymode}

\def\startraster%
  {\localstartraster}

\def\stopraster%
  {\localstopraster}

%D In this documentation we will not go into too much details
%D on palets. Curious users can find more information on this
%D topic in \uit[use of color]. 
%D 
%D At the moment we implemented color in \CONTEXT\ color
%D printing was not yet on the desktop. In spite of this lack our
%D graphics designer made colorfull illustrations. When printed
%D on a black and white printer, distinctive colors can come
%D out equally gray. We therefore decided to use only colors
%D that were distinctive in colors as well as in black and
%D white print. 
%D 
%D Although none of the graphic packages we used supported
%D logical colors and global color redefition, we build this
%D support into \CONTEXT. This enabled us to experiment and
%D also prepared us for the future. 

%D \macros
%D   {definepalet}
%D 
%D Colors are grouped in palets. The colors in such a palet can
%D have colorful names, but best is to use names that specify
%D their use, like {\em important} or {\em danger}. As a sort
%D of example \CONTEXT\ has some palets predefined, 
%D like:\voetnoot{At the time I wrote the palet support, I was 
%D reading 'A hort history of time' of S.~Hawkins, so that's
%D why we stuck to quarks.} 
%D 
%D \starttypen
%D \definepalet    
%D   [alfa]
%D   [     top=rood:7,
%D      bottom=groen:6,
%D          up=blauw:5,
%D        down=cyaan:4,
%D     strange=magenta:3,
%D       charm=geel:2]
%D \stoptypen
%D
%D It's formal definition is: 
%D 
%D \showsetup{\y!definepalet}
%D
%D Visualized, such a palet looks like: 
%D 
%D \startbuffer[palet]
%D \showpalet [alfa] [horizontaal,naam,nummer,waarde]
%D \stopbuffer
%D 
%D \startregelcorrectie
%D \haalbuffer[palet]
%D \stopregelcorrectie
%D
%D This bar shows both the color and gray alternatives of the 
%D palet components (not visible in black and white print).
%D
%D When needed, one can copy a palet by saying:
%D
%D \starttypen
%D \definepalet [TEXcolorpretty] [colorpretty]
%D \stoptypen
%D
%D This saves us some typing in for instance the modules that
%D deal with pretty verbatim typesetting.

\def\definepalet%
  {\dodoubleargument\dodefinepalet}

\def\dodefinepalet[#1][#2]%
  {\doifassignmentelse{#2}
     {\setvalue{\??pa#1}{}%
      \setevalue{\??pa\??pa#1}{#2}%
      \showmessage{\m!colors}{6}{#1}%
      \def\dodododefinepalet[##1=##2]%
        {\doifvaluesomething{\??pa#1}
           {\setevalue{\??pa#1}{\getvalue{\??pa#1},}}%
         \setevalue{\??pa#1}{\getvalue{\??pa#1}##1}%
         \doifdefinedelse{\??cr##2}
           {\setevalue{\??cr#1:##1}{\getvalue{\??cr##2}}}
           {\setevalue{\??cr#1:##1}{S:0}}}%
      \def\dododefinepalet##1%
        {\dodododefinepalet[##1]}%
      \processcommalist[#2]\dododefinepalet}
     {\doifdefined{\??pa#2}
        {\expanded{\dodefinepalet[#1][\getvalue{\??pa\??pa#2}]}}}}

%D \macros 
%D   {setuppalet}
%D
%D Colors are taken from the current palet, if defined. 
%D Setting the current palet is done by:
%D
%D \showsetup{\y!setuppalet}

\def\currentpalet{}

\def\setuppalet%
  {\dosingleempty\dosetuppalet}

\def\dosetuppalet[#1]%
  {\doifelsenothing{#1}
     {\def\currentpalet{}}
     {\doifelsevaluenothing{\??pa#1}
        {\showmessage{\m!colors}{7}{#1}%
         \def\currentpalet{}}
        {\def\currentpalet{#1:}}}}

%D \macros 
%D   {showpalet}
%D 
%D The previous visualization was typeset with: 
%D 
%D \typebuffer[palet]
%D
%D This commands is defined as: 
%D
%D \showsetup{\y!showpalet}

\def\showpalet%
  {\dodoubleargument\doshowpalet}

\def\doshowpalet[#1][#2]%
  {\doifdefined{\??pa#1}
     {\doifinsetelse{\v!vertikaal}{#2}
        {\showverticalpalet[#1][#2]}
        {\showhorizontalpalet[#1][#2]}}}

\def\showverticalpalet[#1][#2]%
  {\localvbox
     {\offinterlineskip
      \setuppalet[#1]
      \def\rule%
        {\vrule\!!width3em\!!height\ht\strutbox\!!depth\dp\strutbox}
      \tabskip\!!zeropoint
      \def\doshowpalet##1%
        {\doifinsetelse{\v!nummer}{#2}{##1\hskip.5em}{}&
         \color[##1]{\rule}\graycolor[##1]{\rule}&
         \doifinset{\v!waarde}{#2}{\hskip.5em\colorvalue{##1}}\crcr}
      \halign
        {\hss##&\hss##\hss&##\cr
         &\doifinset{\v!naam}{#2}{\strut#1}&\cr%
         \processpalet[#1]\doshowpalet\crcr}}}

\def\showhorizontalpalet[#1][#2]%
  {\localvbox
     {\offinterlineskip
      \setuppalet[#1]
      \tabskip\!!zeropoint
      \!!widtha=\!!zeropoint
      \doifinset{\v!nummer}{#2}
        {\def\doshowpalet##1%
           {\setbox0=\hbox{##1}%
            \ifdim\!!widtha<\wd0\!!widtha=\wd0\fi}%
         \processpalet[#1]\doshowpalet}%
      \advance\!!widtha by 1em
      \ifdim\!!widtha<5em
        \!!widtha=5em
      \fi
      \halign
        {##&&\hbox to \!!widtha{\hss##\hss}\cr
         \doifinset{\v!nummer}{#2}
           {\def\doshowpalet##1{&\strut##1}%
            \processpalet[#1]\doshowpalet}\cr
         \doifinset{\v!naam}{#2}{#1\hskip.5em}% 
         \def\doshowpalet##1%
           {&\color[##1]{\vrule\!!width\!!widtha\!!height\ht\strutbox\!!depth\!!zeropoint}}%
         \processpalet[#1]\doshowpalet\crcr
         \def\doshowpalet##1%
           {&\graycolor[##1]{\vrule\!!width\!!widtha\!!height\!!zeropoint\!!depth\dp\strutbox}}%
         \processpalet[#1]\doshowpalet\crcr
         \doifinset{\v!waarde}{#2}
           {\def\doshowpalet##1%
             {&\vbox
                {\hsize\!!widtha
                 \vskip.25ex
                 \everypar{\strut}
                 \veryraggedcenter
                 \let\colorformatseparator=\endgraf
                 \colorvalue{##1}}}%
            \processpalet[#1]\doshowpalet}%
         \crcr}}}

\def\processpalet[#1]%
  {\expanded{\globalprocesscommalist[\getvalue{\??pa#1}]}}

%D \macros
%D   {definecolorgroup}
%D
%D The naming of the colors in this palet suggests some 
%D ordening, which in turn is suported by color grouping. 
%D 
%D \starttypen
%D \definecolorgroup
%D   [rood]
%D   [1.00:0.90:0.90,
%D    1.00:0.80:0.80,
%D    1.00:0.70:0.70,
%D    1.00:0.55:0.55,
%D    1.00:0.40:0.40,
%D    1.00:0.25:0.25,
%D    1.00:0.15:0.15,
%D    0.90:0.00:0.00]
%D \stoptypen
%D 
%D In such a color group colors are numbered from~$1$ to~$n$. 
%D
%D \showsetup{\y!definecolorgroup}
%D
%D This kind of specification is not only more compact than 
%D defining each color separate, it also loads faster and takes 
%D less bytes. 

\def\definecolorgroup%
  {\dotripleempty\dodefinecolorgroup}

\def\dodefinecolorgroup[#1][#2][#3]%
  {\ifthirdargument
     \processaction
       [#2]
       [    \v!cmyk=>\edef\currentcolorspace{C},
             \v!rgb=>\edef\currentcolorspace{R},
            \v!gray=>\edef\currentcolorspace{S},
               \v!s=>\edef\currentcolorspace{S},
         \s!unknown=>\edef\currentcolorspace{R}]%
     \scratchcounter=0
     \def\dododefinecolorgroup##1%
       {\advance\scratchcounter by 1
        \setevalue{\??cr#1:\the\scratchcounter}{\currentcolorspace:##1}}%
     \processcommalist[#3]\dododefinecolorgroup
   \else
     \doifinstringelse{:}{#2}
       {\definecolorgroup[#1][\v!rgb][#2]}
       {\doloop
          {\doifdefinedelse{\??cr#2:\recurselevel}
             {\setevalue{\??cr#1:\recurselevel}%
                {\getvalue{\??cr#2:\recurselevel}}}
             {\exitloop}}}%
   \fi}

%D \macros
%D   {showcolorgroup}
%D 
%D We can show the group by:
%D 
%D \startbuffer
%D \showcolorgroup [blauw] [horizontaal,naam,nummer,waarde]
%D \stopbuffer
%D 
%D \typebuffer
%D 
%D or in color: 
%D 
%D \startregelcorrectie
%D \haalbuffer
%D \stopregelcorrectie
%D
%D which uses: 
%D
%D \showsetup{\y!showcolorgroup}

\def\showcolorgroup%
  {\dodoubleargument\doshowcolorgroup} 

\def\doshowcolorgroup[#1][#2]%
  {\doifdefined{\??cr#1:1}
     {\doifinsetelse{\v!vertikaal}{#2}
        {\showverticalcolorgroup[#1][#2]}
        {\showhorizontalcolorgroup[#1][#2]}}}

\def\showhorizontalcolorgroup[#1][#2]%
  {\localvbox
     {\offinterlineskip
      \setuppalet
      \tabskip\!!zeropoint
      \def\rule%
        {\vrule\!!width4em\!!height\ht\strutbox\!!depth\dp\strutbox}
      \def\colorformatseparator{\strut\cr}
      \def\dodoshowgroup##1%
        {\halign
           {\hss####\hss\cr
            \doifinset{\v!nummer}{#2}{\strut##1}\cr
            \color[#1:##1]{\vrule\!!width4em\!!height\ht\strutbox\!!depth\!!zeropoint}\cr
            \graycolor[#1:##1]{\vrule\!!width4em\!!height\!!zeropoint\!!depth\dp\strutbox}\cr
            \doifinset{\v!waarde}{#2}{\colorvalue{#1:##1}\strut}\crcr}}
      \def\doshowgroup##1%
        {\doifdefined{\??cr#1:##1}
           {\vbox{\dodoshowgroup{##1}}}}
      \hbox 
        {\doifinset{\v!naam}{#2}
           {\strut
            \doifinsetelse{\v!waarde}{#2}
              {\raise3\lineheight\hbox{#1\hskip.5em}}
              {#1}%
            \hskip.5em}%
         \doshowgroup1\doshowgroup2\doshowgroup3\doshowgroup4%
         \doshowgroup5\doshowgroup6\doshowgroup7\doshowgroup8}}}

\def\showverticalcolorgroup[#1][#2]%
  {\localvbox
     {\offinterlineskip
      \setuppalet
      \tabskip\!!zeropoint
      \def\rule%
        {\vrule\!!width2.5em\!!height\ht\strutbox\!!depth\dp\strutbox}
      \def\doshowgroup##1%
        {\doifdefined{\??cr#1:##1}
           {\doifinset{\v!nummer}{#2}{##1\hskip.5em}&
            \color[#1:##1]{\rule}\graycolor[#1:##1]{\rule}&
            \doifinset{\v!waarde}{#2}{\hskip.5em\colorvalue{#1:##1}}\crcr}}
      \halign
        {\hss##&\hss##\hss&##\hss\cr
         &\doifinset{\v!naam}{#2}{\strut#1}&\crcr   
         \doshowgroup1\doshowgroup2\doshowgroup3\doshowgroup4%
         \doshowgroup5\doshowgroup6\doshowgroup7\doshowgroup8}}}

%D There are ten predefined color groups, like
%D \color[groen]{\em groen}, \color[rood]{\em rood},
%D \color[blauw]{\em blauw}, \color[cyaan]{\em cyaan},
%D \color[magenta]{\em magenta} and \color[geel]{\em geel}. 
%D 
%D \startregelcorrectie
%D \hbox to \hsize
%D   {\hss
%D    \showcolorgroup [rood]    [vertikaal,naam,nummer]\hss
%D    \showcolorgroup [groen]   [vertikaal,naam]\hss
%D    \showcolorgroup [blauw]   [vertikaal,naam]\hss
%D    \showcolorgroup [cyaan]   [vertikaal,naam]\hss
%D    \showcolorgroup [magenta] [vertikaal,naam]\hss
%D    \showcolorgroup [geel]    [vertikaal,naam]\hss}
%D \stopregelcorrectie
%D 
%D These groups are used to define palets {\em alfa} upto {\em
%D zeta}. As long as we don't use colors from the same row, we
%D get ourselves distinctive palets. By activating such a palet
%D one gains access to its members {\em top} to {\em charm} (of
%D course one should use more suitable names than these). 
%D 
%D \startregelcorrectie
%D \hbox to \hsize
%D   {\showpalet [alfa]    [vertikaal,naam,nummer]\hss
%D    \showpalet [beta]    [vertikaal,naam]\hss
%D    \showpalet [gamma]   [vertikaal,naam]\hss
%D    \showpalet [delta]   [vertikaal,naam]\hss
%D    \showpalet [epsilon] [vertikaal,naam]\hss
%D    \showpalet [zeta]    [vertikaal,naam]}
%D \stopregelcorrectie
%D  
%D By using the keyword \type{\v!waarde} the individual color
%D components are shown too. When printed in color, these
%D showcases show both the colors and the gray value. 

%D \macros
%D   {comparepalet}
%D 
%D There are some more testing macros available:
%D 
%D \startbuffer
%D \comparepalet [alfa]
%D \stopbuffer
%D 
%D \typebuffer
%D 
%D shows the palet colors against a background:
%D 
%D \startregelcorrectie
%D \haalbuffer
%D \stopregelcorrectie
%D
%D The formal definition is: 
%D 
%D \showsetup{\y!comparepalet}

\def\comparepalet%
  {\dosingleargument\docomparepalet}

\def\docomparepalet[#1]%
  {\doifdefined{\??pa#1}
     {\hbox
        {\dodocomparepalet\color[#1]%
         \quad
         \dodocomparepalet\graycolor[#1]}}}

\def\dodocomparepalet#1[#2]%
  {\localvbox
     {\offinterlineskip
      \setuppalet[#2]
      \getcommacommandsize[\getvalue{\??pa#2}]
      \!!widtha=2em\relax
      \hsize=\commalistsize\!!widtha
      \def\rule%
        {\vrule\!!width.5\!!widtha\!!height2.25ex\!!depth-.75ex}
      \def\dododocomparepalet##1%
        {\hbox
           {\setbox0=\hbox
              {#1[##1]{\vrule\!!width\hsize\!!height3ex}}%
            \wd0=\!!zeropoint\box0
            \hbox to \hsize
              {\def\dododocomparepalet####1%
                 {\hbox to \!!widtha
                    {\hss#1[####1]{\rule}\hss}}%
               \processcommacommand[\getvalue{\??pa#2}]\dododocomparepalet}}
         \endgraf}
      \processcommacommand[\getvalue{\??pa#2}]\dododocomparepalet}}

%D \macros
%D   {comparecolorgroup}
%D
%D The similar command: 
%D
%D \startbuffer
%D \comparecolorgroup [blauw]
%D \stopbuffer
%D 
%D \typebuffer
%D 
%D shows color groups:
%D 
%D \startregelcorrectie
%D \haalbuffer
%D \stopregelcorrectie
%D
%D this commands are defined as: 
%D
%D \showsetup{\y!comparecolorgroup}

\def\comparecolorgroup%
  {\dosingleargument\docomparecolorgroup}

\def\docomparecolorgroup[#1]%
  {\doifdefined{\??cr#1:1}
     {\hbox
        {\dodocomparecolorgroup\color[#1]%
         \quad
         \dodocomparecolorgroup\graycolor[#1]}}}

\def\dodocomparecolorgroup#1[#2]%
  {\localvbox
     {\!!counta=0
      \dorecurse{15}
        {\doifdefined{\??cr#2:\recurselevel}{\advance\!!counta by 1}}
      \!!widtha=2em\relax
      \hsize=\!!counta\!!widtha
      \def\rule%
        {\vrule\!!width.5\!!widtha\!!height2.25ex\!!depth-.75ex}
      \def\dododocomparecolorgroup##1%
        {\hbox to \hsize
           {\setbox0=\hbox
              {#1[#2:##1]{\vrule\!!width\hsize\!!height3ex}}%
            \wd0=\!!zeropoint\box0
            \hbox to \hsize
              {\hss\dorecurse{\!!counta}{#1[#2:\recurselevel]{\rule}\hss}}}
         \endgraf}
      \dorecurse{\!!counta}{\dododocomparecolorgroup\recurselevel}}}

%D \macros 
%D   {showcolor}
%D 
%D But let's not forget that we also have the more traditional 
%D non||related colors. These show up after: 
%D 
%D \starttypen
%D \showcolor [name]
%D \stoptypen
%D 
%D Where \type{name} for instance can be \type{rgb}. 
%D
%D \showsetup{\y!showcolor}

\def\showcolor[#1]%
  {\bgroup
     \setupcolor[#1]
     \stelwitruimtein[\v!geen]
     \def\rule%
       {\vrule\!!width4em\!!height\ht\strutbox\!!depth\dp\strutbox}
     \def\docommand##1%
       {\hbox 
          {\graycolor[##1]{\rule}\quad
           \color[##1]{\rule}\quad
           \grayvalue{##1}\quad
           \hbox to 12em{\colorvalue{##1}\hss}%
           \strut##1}
        \endgraf}
     \processcommacommand[\colorlist]\docommand
   \egroup}

%D \macros
%D   {negativecolorbox}
%D
%D Sometimes, especially when we deal with typesetting 
%D devices, we want to reverse the color scheme. Instead of 
%D recalculating all those colors, we use a quick and dirty 
%D approach: 
%D 
%D \starttypen
%D \negativecolorbox0
%D \stoptypen
%D 
%D will negate the colors in box zero. 

\def\negatecolorbox#1%
  {\setbox#1=\hbox
     {\dostartnegative  
      \localstartcolor[white]%
      \vrule\!!height\ht#1\!!depth\dp#1\!!width\wd#1%
      \localstopcolor
      \hskip-\wd#1%
      \box#1\dostopnegative}}

%D \macros 
%D   {ifMPgraphics, MPcolor}
%D 
%D A very special macro is \type{\MPcolor}. This one can be 
%D used to pass a \CONTEXT\ color to \METAPOST. 
%D
%D \starttypen
%D \MPcolor{my own red}
%D \stoptypen
%D
%D This macro returns a \METAPOST\ triplet \type{(R,G,B)}. 
%D Only \kap{RGB} colors and gray scales are supported. 

\def\MPcolor#1%
  {\ifMPgraphics
     \@EA\@EA\@EA\doMPcolor\csname\??cr#1\endcsname:::::\end
   \else
     #1%
   \fi}

\def\doMPcolor#1:#2:#3:#4:#5:#6\end%
  {\if     #1R(#2,#3,#4)%
   \else\if#1C(1-#2-#5,1-#3-#5,1-#4-#5)%
%   \else\if#1C(1-#2+#5,1-#3+#5,1-#4+#5)%
   \else\if#1S(#2,#2,#2)% 
   \else      (0,0,0)%
   \fi\fi\fi}
 
%D \macros 
%D   {PDFcolor,FDFcolor}
%D 
%D Similar alternatives are avaliable for \PDF:

\def\PDFcolor#1%
  {\@EA\@EA\@EA\doPDFcolor\csname\??cr#1\endcsname:::::\end}
   
\def\doPDFcolor#1:#2:#3:#4:#5:#6\end%
  {\if     #1R#2 #3 #4 rg%
   \else\if#1C#2 #3 #4 #5 k%
   \else\if#1S#2 g%
   \else      0 g%
   \fi\fi\fi}

\def\FDFcolor#1%
  {\@EA\@EA\@EA\doFDFcolor\csname\??cr#1\endcsname:::::\end}
   
\def\doFDFcolor#1:#2:#3:#4:#5:#6\end%
  {[\if     #1R#2 #3 #4%
    \else\if#1C#2 #3 #4 #5%
    \else\if#1S#2%
    \else      0%
    \fi\fi\fi]}

%D We default to the colors defined in \module{colo-rgb} and
%D support both \kap{RGB} and \kap{CMYK} output. 

\setupcolors
  [\c!status=\v!stop,
   \c!conversie=\v!ja,
   \c!reductie=\v!nee,
   \c!rgb=\v!ja,
   \c!cmyk=\v!ja]

\setupcolor
  [\v!rgb]

%D As we can see, color support is turned off by default. 
%D Reduction of gray colors to gray scales is turned on. 

\protect

\endinput
