%D \module
%D   [       file=font-eul,
%D        version=1995.1.1,
%D          title=\CONTEXT\ Font Macros,
%D       subtitle=Euler,
%D         author=Hans Hagen,
%D           date=\currentdate,
%D      copyright={PRAGMA / Hans Hagen \& Ton Otten}]
%C
%C This module is part of the \CONTEXT\ macro||package and is
%C therefore copyrighted by \PRAGMA. See licen-en.pdf for 
%C details. 

%D The Euler Fonts are designed by Herman Zapf and can be 
%D used with the Concrete Fonts defined elsewhere. 

\definebodyfont [12pt] [mm]     % scaled \magstep1
  [mi=eurm10 at 12pt,
   ex=euex10 at 12pt,
   ma=euex10 at 12pt,
   mb=eusm10 at 12pt,
   mc=eufm10 at 12pt]

\definebodyfont [11pt] [mm]     % scaled \magstephalf
  [mi=eurm10 at 11pt,
   ex=euex10 at 11pt,
   ma=euex10 at 11pt,
   mb=eusm10 at 11pt,
   mc=eufm10 at 11pt]

\definebodyfont [10pt] [mm]
  [mi=eurm10,
   ex=euex10,
   ma=euex10,
   mb=eusm10,
   mc=eufm10]

\definebodyfont [9pt] [mm]
  [mi=eurm10 at 9pt,
   ex=euex10 at 9pt,
   ma=euex10 at 9pt,
   mb=eusm10 at 9pt,
   mc=eufm10 at 9pt]

\definebodyfont [8pt] [mm]
  [mi=eurm7  at 8pt,
   ex=euex10 at 8pt,
   ma=euex10 at 8pt,
   mb=eusm7  at 8pt,
   mc=eufm7  at 8pt]

\definebodyfont [7pt] [mm]
  [mi=eurm7,
   ex=euex10 at 7pt,
   ma=euex10 at 7pt,
   mb=eusm7,
   mc=eufm7]

\definebodyfont [6pt] [mm]
  [mi=eurm7  at 6pt,
   ex=euex10 at 6pt,
   ma=euex10 at 6pt,
   mb=eusm7  at 6pt,
   mc=eufm7  at 6pt]

\definebodyfont [5pt] [mm]
  [mi=eurm5,
   ex=euex10 at 5pt,
   ma=euex10 at 5pt,
   mb=eusm5,
   mc=eufm5]

%D Here we copy part of the files that are distributed along 
%D with these fonts, but first we define some extra families.

\let\exfam=\mafam   % was A 
\let\smfam=\mbfam   % was 8 
\let\fmfam=\mcfam   % was 9 

\let\hexexfam=\hexmafam
\let\hexsmfam=\hexmbfam
\let\hexfmfam=\hexmcfam

%D Now we're up to the redefinitions. 

\mathcode`0="7130
\mathcode`1="7131
\mathcode`2="7132
\mathcode`3="7133
\mathcode`4="7134
\mathcode`5="7135
\mathcode`6="7136
\mathcode`7="7137
\mathcode`8="7138
\mathcode`9="7139

\mathchardef\intop          ="1\hexexfam 52
\mathchardef\ointop         ="1\hexexfam 48
\mathchardef\coprod         ="1\hexexfam 60
\mathchardef\prod           ="1\hexexfam 51
\mathchardef\sum            ="1\hexexfam 50
\mathchardef\braceld        ="\hexexfam 7A 
\mathchardef\bracerd        ="\hexexfam 7B
\mathchardef\bracelu        ="\hexexfam 7C 
\mathchardef\braceru        ="\hexexfam 7D
\mathchardef\infty          ="0\hexexfam 31

\mathchardef\nearrow        ="3\hexexfam 25
\mathchardef\searrow        ="3\hexexfam 26
\mathchardef\nwarrow        ="3\hexexfam 2D
\mathchardef\swarrow        ="3\hexexfam 2E
\mathchardef\Leftrightarrow ="3\hexexfam 2C
\mathchardef\Leftarrow      ="3\hexexfam 28
\mathchardef\Rightarrow     ="3\hexexfam 29
\mathchardef\leftrightarrow ="3\hexexfam 24 
\mathchardef\leftarrow      ="3\hexexfam 20 
\mathchardef\rightarrow     ="3\hexexfam 21

\let\gets =\leftarrow 
\let\to   =\rightarrow  

\mathcode`\^^W              ="3\hexexfam 24
\mathcode`\^^X              ="3\hexexfam 20
\mathcode`\^^Y              ="3\hexexfam 21
\mathcode`\^^K              ="3\hexexfam 22
\mathcode`\^^A              ="3\hexexfam 23

\def\uparrow                {\delimiter"3\hexexfam 22378 } 
\def\downarrow              {\delimiter"3\hexexfam 23379 } 
\def\updownarrow            {\delimiter"3\hexexfam 6C33F }
\def\Uparrow                {\delimiter"3\hexexfam 2A37E }
\def\Downarrow              {\delimiter"3\hexexfam 2B37F }
\def\Updownarrow            {\delimiter"3\hexexfam 6D377 }

\mathchardef\leftharpoonup    ="3\hexexfam 18
\mathchardef\leftharpoondown  ="3\hexexfam 19
\mathchardef\rightharpoonup   ="3\hexexfam 1A
\mathchardef\rightharpoondown ="3\hexexfam 1B

\mathcode`+="2\hexfmfam 2B
\mathcode`-="2\hexfmfam 2D
\mathcode`!="0\hexfmfam 21
\mathcode`(="4\hexfmfam 28    \delcode`(="\hexfmfam 28300
\mathcode`)="5\hexfmfam 29    \delcode`)="\hexfmfam 29301
\mathcode`[="4\hexfmfam 5B    \delcode`[="\hexfmfam 5B302
\mathcode`]="5\hexfmfam 5D    \delcode`]="\hexfmfam 5D303
\mathcode`=="3\hexfmfam 3D

\mathchardef\Relbar  ="303D % we need the old = to match \Arrows
\mathchardef\Gamma   ="7100
\mathchardef\Delta   ="7101
\mathchardef\Theta   ="7102
\mathchardef\Lambda  ="7103
\mathchardef\Xi      ="7104
\mathchardef\Pi      ="7105
\mathchardef\Sigma   ="7106
\mathchardef\Upsilon ="7107
\mathchardef\Phi     ="7108
\mathchardef\Psi     ="7109
\mathchardef\Omega   ="710A

\let\varsigma        =\sigma % Euler doesn't have these
\let\varrho          =\rho   % Euler doesn't have these
\mathchardef\aleph   ="0D40

\def\rbrace          {\delimiter"5\hexsmfam 67A09 } \let\}=\rbrace
\def\lbrace          {\delimiter"4\hexsmfam 66A08 } \let\{=\lbrace

\mathchardef\leq     ="3\hexsmfam 14 \let\le=\leq
\mathchardef\geq     ="3\hexsmfam 15 \let\ge=\geq
\mathchardef\Re      ="0\hexsmfam 3C
\mathchardef\Im      ="0\hexsmfam 3D

\def\vert            {\delimiter"\hexsmfam 6A30C }
\def\backslash       {\delimiter"\hexsmfam 6E30F }

\endinput
