%D \module
%D   [       file=cont-new,
%D        version=1995.10.10,
%D          title=\CONTEXT\ Miscellaneous Macros,
%D       subtitle=New Macros,
%D         author=Hans Hagen,
%D           date=\currentdate,
%D      copyright={PRAGMA / Hans Hagen \& Ton Otten}]
%C
%C This module is part of the \CONTEXT\ macro||package and is
%C therefore copyrighted by \PRAGMA. See mreadme.pdf for
%C details.

% manual : offsetbox alignbox 

\unprotect

\def\savefont % do we also need to store the encoding ? 
  {\edef\savedfont{\the\font}%
   \pushmacro\savedfont
   \pushmacro\characterregime
   \pushmacro\charactermapping
   \pushmacro\characterencoding}

\def\restorefont%
  {\popmacro\characterencoding
   \popmacro\charactermapping
   \popmacro\characterregime
   \popmacro\savedfont
   \savedfont}

%D This file is loaded at runtime, thereby providing an
%D excellent place for hacks and new features.

\writestatus{\m!systems}{beware: some patches loaded from cont-new.tex!}

\def\setraggedparagraphmode#1#2%
  {\ifinpagebody
     \ifdubbelzijdig
       \ifodd\realfolio#1\else#2\fi
     \else
       #2\relax
     \fi
   \else\ifinner
     \ifdubbelzijdig
       \gettwopassdata{\s!paragraph}%
       \iftwopassdatafound
         \ifodd\twopassdata#1\else#2\fi
       \else
         \ifodd\realfolio#1\else#2\fi
       \fi
       \doparagraphreference
     \else
       #2\relax
     \fi
   \else
     #2\relax
   \fi\fi}

% no, too buggy, leads to top of page crap  
%
%\def\flushsidefloats%
%  {\par
%   \dochecksidefloat
%   \scratchcounter=-\hangafter 
%   \dorecurse{\scratchcounter}{\strut\hfill\strut\par}}

\def\flushsidefloats%
  {\par
   \!!heighta=\sidefloatvsize
   \advance\!!heighta by -\pagetotal
   \ifdim\!!heighta>\zeropoint
     % to be checked for interference 
     \witruimte 
     % will be option 
     \getnoflines\!!heighta 
     \!!heighta=\noflines\lineheight
     % so far for option
     \kern\!!heighta
   \fi
   \global\sidefloatvsize=\nofloatvsize
   \global\floatflagfalse}

\def\thinrule%
  {\strut
   \bgroup
   \chardef\ruletype=1
   \processaction
     [\@@dlvariant]
     [ \v!a=>\chardef\ruletype=0,% no line 
      %\v!b=>\chardef\ruletype=1,% height/depth
       \v!c=>\chardef\ruletype=2,% topheight/botdepth 
      %  11=>\chardef\ruletype=1,% fallback for  backgrounds
          0=>\chardef\ruletype=0,% compatible with backgrounds
      %   1=>\chardef\ruletype=1,% compatible with backgrounds
          2=>\chardef\ruletype=2]% compatible with backgrounds
   \doifsomething{\@@dllijndikte}
     {\linewidth=\@@dllijndikte}%
   \ifdim\linewidth=\zeropoint  
     \chardef\ruletype=0 
   \else
     \doifnot{\@@dlkader}{\v!aan}{\chardef\ruletype=0\relax}%
   \fi
   \ifnum\ruletype=1
     \doif{\@@dlhoogte}{\v!max}{\def\@@dlhoogte{1}}%
     \doif{\@@dldiepte}{\v!max}{\def\@@dldiepte{1}}%
   \else
     \def\@@dlhoogte{1}%
     \def\@@dldiepte{1}%
   \fi
   \freezedimensionwithunit\@@dlhoogte{\ht\strutbox}%
   \freezedimensionwithunit\@@dldiepte{\dp\strutbox}%
   \divide\linewidth 2 
   \doifelse{\@@dlachtergrond}{\v!kleur}
     {\startcolor[\@@dlachtergrondkleur]%
      \dimen0=\@@dlhoogte
      \dimen2=\@@dldiepte
      \ifnum\ruletype=2 % prevent overshoot due to rounding 
        \advance\dimen0 by -.5\linewidth
        \advance\dimen2 by -.5\linewidth
      \fi
      \leaders\hrule\!!height\dimen0\!!depth\dimen2\hfill
      \stopcolor
      \ifcase\ruletype 
        % no rule 
      \or
        \startcolor[\@@dlkleur]%
        \hfillneg
        \leaders\hrule\!!height\linewidth\!!depth\linewidth\hfill
        \stopcolor
      \or
        \startcolor[\@@dlkleur]%
        \dimen2=\@@dldiepte\dimen0=-\dimen2 \advance\dimen0 \linewidth
        \hfillneg\leaders\hrule\!!height\dimen0\!!depth\dimen2\hfill
        \dimen2=\@@dlhoogte\dimen0=-\dimen2 \advance\dimen0 \linewidth
        \hfillneg\leaders\hrule\!!height\dimen2\!!depth\dimen0\hfill
        \stopcolor
      \fi}
     {\ifcase\ruletype \else
        \startcolor[\@@dlkleur]%
        \leaders\hrule\!!height\@@dlhoogte\!!depth\@@dldiepte\hfill
        \stopcolor
      \fi}%
   \strut
   \carryoverpar\egroup}

\setupthinrules
  [\c!kader=\v!aan, % compatible with textbackgrounds  
   \c!variant=\v!b,
   \c!achtergrondkleur=,
   \c!achtergrond=,
   \c!lijndikte=]

% \thinruled[n=3,alternative=a]
% \thinruled[n=3,alternative=b]
% \thinruled[n=3,alternative=c]
% \thinruled[n=3,alternative=a,background=color]
% \thinruled[n=3,alternative=b,background=color]
% \thinruled[n=3,alternative=c,background=color]

\def\dothinrules[#1]%
  {\bgroup
   \dosetupthinrules[#1]%
   \@@dlvoor
   \assignvalue{\@@dlinterlinie}{\@@dlinterlinie}{1.0}{1.5}{2.0}%
   \spacing\@@dlinterlinie
   \dorecurse
     {\@@dln}
     {\ifnum\recurselevel=\@@dln \dothinrulesnobreak \else
      \ifnum\recurselevel=2      \dothinrulesnobreak \fi\fi
                                 \thinrule
     %\ifnum\recurselevel<\@@dln \endgraf \fi}%
      \ifnum\recurselevel<\@@dln \endgraf \geenwitruimte \@@dltussen \fi}%
%  \@@dlna
%  \egroup}
   \doifelsenothing{\@@dlna}
     {\carryoverpar\egroup}
     {\@@dlna\egroup}}

\def\dodousemodules#1#2%
  {\setfalse\moduleisloaded 
   \doifelsenothing{#1}
     {\def\next
        {\dododousemodules\f!moduleprefix {#2}% 
         \dododousemodules\f!privateprefix{#2}%
         \dododousemodules\f!styleprefix  {#2}%
         \dododousemodules\f!xstyleprefix {#2}%
         \dododousemodules\f!thirdprefix  {#2}}}
     {\def\next
        {\dododousemodules{#1-}{#2}}}%
   \next
   \ifconditional\moduleisloaded\else
     \showmessage{\m!systems}{6}{#2}%
   \fi}

\def\dousemodules[#1][#2]%
  {\ifsecondargument
     \doifelsenothing{#2}
       {\let\next\relax}
       {\def\next{\processcommalist[#2]{\dodousemodules{#1}}}}%
   \else
     \def\next{\usemodules[][#1]}%
   \fi
   \next}

\def\usemodules%
  {\dodoubleempty\dousemodules}

\let\usemodule\usemodules

% \usemodule[t][speech]

\def\complexTableTB[#1]{\TABLEnoalign{\blanko[#1]}}
\def\simpleTableTB     {\TABLEnoalign{\blanko}}

\def\TabulateTB
  {\complexorsimpleTable{TB}}

\def\doTableinterline% #1 
  {\ifnum\currentTABLEcolumn>\maxTABLEcolumn
     \chuckTABLEautorow
   \else\ifnum\currentTABLEcolumn=0
     \TABLEnoalign
       {\global\let\checkTABLEautorow=\empty
        \global\let\chuckTABLEautorow=\empty}%
   \else
     \setTABLEerror\TABLEmissingcolumn
     \handleTABLEerror
   \fi\fi
   \complexorsimpleTable} % {#1} 

\def\TableHL{\doTableinterline{HL}}
\def\TableTB{\doTableinterline{TB}}

\appendtoks\let\TB\TableTB   \to\everytable
\appendtoks\let\TB\TabulateTB\to\everytabulate

% \starttabulate
% \NC text \NC text \NC \NR 
% \TB[small]
% \NC text \NC text \NC \NR 
% \TB[4*big]
% \NC text \NC text \NC \NR 
% \stoptabulate
% 
% \starttable[|||]
% \VL text \VL text \VL \AR 
% \TB[small]
% \VL text \VL text \VL \AR 
% \TB[4*big]
% \VL text \VL text \VL \AR 
% \stoptable

% Quite experimental !

% the split is needed when for instance the float goes into 
% a multi page field and the list of figs becomes larger than 
% one page: cycle between 'only flush when object ref ok' 
% and 'one/many page fig list'; see "uguide finometer"  

\def\placefloatcaption
  {\dodoubleempty\doplacefloatcaption}

\def\doplacefloatcaption[#1][#2]#3%
  {\setfloatcaption[#1][#2]{#3}% 
   \placefloatcaptiontext[#1]% 
   \placefloatcaptionreference[#1]}

\def\setfloatcaption 
  {\dodoubleempty\dodosetfloatcaption} % beware, name clash 

\def\dodosetfloatcaption[#1][#2]#3% to do namespace for number/ascii
  {\doifelsevalue{\??kj#1\c!nummer}{\v!ja} % also handle trialtypesetting
     {\verhoognummer[#1]%
      \maakhetnummer[#1]%
      \global\let\flhetnummer\hetnummer
      \setgvalue{@fl@r@#1}%
        {\dofloatreference
         \redofloatorder{#1}%
         \doschrijfnaarlijst{#1}{\flhetnummer}{#3}{#1}%
         \doglobal\convertargument#3\to\flasciititle % \asciititle is global 
         \doifsomething{#2}{\rawreference{\s!flt}{#2}{{\flhetnummer}{\flasciititle}}}%
         \global\letvalue{@fl@r@#1}\relax}% nills 
      \setgvalue{@fl@t@#1}%
        {\doattributes{\??kj#1}\c!kopletter\c!kopkleur
           {\labeltexts{#1}{\flhetnummer}}%
         \doattributes{\??kj#1}\c!letter\c!kleur
           {\tfskip#3}}}
     {\global\letvalue{@fl@r@#1}\relax
      \global\letvalue{@fl@t@#1}\relax}}

\def\placefloatcaptiontext     [#1]{\getvalue{@fl@t@#1}}
\def\placefloatcaptionreference[#1]{\getvalue{@fl@r@#1}}

% still needed for uguide 

\let\placefloatlabel          \placefloatcaption
\let\placefloatlabeltext      \placefloatcaptiontext
\let\placefloatlabelreference \placefloatcaptionreference

\def\checkframedtext%
  {\ifinsidefloat
     \localhsize\hsize
   \else\ifdim\sidefloatvsize>\zeropoint % will be proper handle 
   %  \strut            % rather clean way to invoke the sidefloat OTR
   %  \setbox0=\lastbox % and get the widths set, so from now on we
   %  \setlocalhsize    % can have framed texts alongside sidefloats
   \checksidefloat
   \setlocalhsize
     \advance\localhsize-\hangindent
   \else
     \localhsize\hsize
   \fi \fi}

\long\def\parseTR[#1][#2]#3\eTR% [#2] is dummy that kills spaces
  {\def\currentcol{0}\increment\maximumrow
   \setupTABLE[\v!rij][\maximumrow][#1]#3}

\def\obeyfollowingtoken{{}}  % end \cs scanning 

\def\gobbleparameters{\doquadrupleempty\dogobbleparameters}
\def\dogobbleparameters[#1][#2][#3][#4]{}

% faster, and looks okay

\dostepwiserecurse{0}{255}{1}
  {\@EA\chardef\csname-\recurselevel\endcsname=\recurselevel}

\newtoks\withminorcharacters
\newtoks\withlowercharacters
\newtoks\withuppercharacters

% \thewithcharacter#1 % self 

\dostepwiserecurse{0}{31}{1}
  {\expanded
     {\appendtoks\noexpand\withcharacter\csname-\recurselevel\endcsname
        \noexpand\to\withminorcharacters}}

\dostepwiserecurse{32}{127}{1}
  {\expanded
     {\appendtoks\noexpand\withcharacter\csname-\recurselevel\endcsname
        \noexpand\to\withlowercharacters}}

\dostepwiserecurse{128}{255}{1}
  {\expanded
     {\appendtoks\noexpand\withcharacter\csname-\recurselevel\endcsname
        \noexpand\to\withuppercharacters}}

\def\doassigncatcodes#1%
  {\def\withcharacter##1{\catcode##1#1}%
   \the\withminorcharacters
   \the\withlowercharacters
   \ifeightbitcharacters\the\withuppercharacters\fi}

\def\makeallother%
  {\doassigncatcodes\@@other}

\makeallothertoks\emptytoks
 
\chardef\obeyedlccode=`. % so <32 and >127 chars become . 

\def\obeylccodes%
  {\def\withcharacter##1{\lccode##1##1}%
   \the\withlowercharacters
   \def\withcharacter##1{\lccode##1\obeyedlccode}%
   \the\withminorcharacters
   \ifeightbitcharacters\the\withuppercharacters\fi}

\definesystemvariable{en}

\def\setupenv{\dodoubleargument\rawgetparameters[\??en]}

\def\doifenvelse#1{\doifdefinedelse{\??en#1}}

% \def\envvar#1#2{\ifundefined{\??en#1}#2\else\getvalue{\??en}\fi}

\def\env#1{\getvalue{\??en#1}}

\beginTEX

\def\envvar#1#2%
  {\@EA\ifx\csname\??en#1\endcsname\relax
     #2\else\csname\??en#1\endcsname
   \fi}

\endTEX

\beginETEX \ifcsname

\def\envvar#1#2%
  {\ifcsname\??en#1\endcsname
     \csname\??en#1\endcsname\else#2%
   \fi}

\endETEX

\def\setvariables%
  {\dodoubleargument\dosetvariables}

\def\dosetvariables[#1][#2]%
  {\def\currentvariableclass{#1}%
   \getparameters[vars:#1:][#2]}

\def\getvariable#1#2%
  {\ifundefined{vars:#1:#2}\else\getvalue{vars:#1:#2}\fi}

\let\currentvariableclass\empty

% in both (otr) modules ! 

\def\doifrightpageelse#1#2%
  {\ifdubbelzijdig
     \gettwopassdata{\s!paragraph}%
     \iftwopassdatafound
       \ifodd\twopassdata#1\else#2\fi
     \else
       \ifodd\realfolio#1\else#2\fi
     \fi
   \else
     #1% was #2 
   \fi}

\def\signalrightpage%
  {\ifdubbelzijdig
     \doparagraphreference
   \fi}

%D To be documented, \type {\includemenu[menu]}. 
%D To be documented, \type {\emphbf} cum suis.

%D A prelude to strategies. Note for myself: overloads
%D previous stuff from local pragma test files.

\def\s!strategy{strategy}

\def\currentstrategypass    {1}
\def\currentstrategyvariable{0}
\def\maximumstrategypass    {8}

\newconditional\strategypassneeded
\newconditional\strategypassforced

\definetwopasslist{\s!strategy}

\def\registerstrategypass%
  {\ifnum\currentstrategypass>\maximumstrategypass \else
     \ifconditional\strategypassforced
       \doglobal\increment\currentstrategypass
     \else%\ifconditional\strategypassneeded
       %\doglobal\increment\currentstrategypass
     \fi%\fi
   \fi
   \savecurrentvalue\currentstrategypass{\currentstrategypass}}

\appendtoks \registerstrategypass \to \everybye % \everylastshipout

\def\setstrategyvariable#1#2% key value
  {%\doifnotstrategyvariable{#1}{\global\settrue\strategypassneeded}%
   \doglobal\increment\currentstrategyvariable
   \expanded{\immediatewriteutilitycommand{\noexpand
     \twopassentry{\s!strategy}{\currentstrategyvariable}{#1::#2}}}}

\def\doifstrategyvariableelse#1#2#3%
  {\getstrategyvariable{#1}\iftwopassdatafound#2\else#3\fi}

\def\getstrategyvariable#1% key
  {\findtwopassdata{\s!strategy}{#1::}%
   \setxvalue{\s!strategy:#1}{\twopassdata}}

\def\retainstrategyvariable#1% key
  {\expanded{\setstrategyvariable{#1}{\strategyvariable{#1}}}}

\def\strategyvariable#1% key
  {\csname\s!strategy:#1\endcsname}

\let\stratvar\strategyvariable

\def\forcestrategy{\global\settrue \strategypassforced}
\def\abortstrategy{\global\setfalse\strategypassforced}

\def\doifstrategyvariableelse#1#2#3%
  {\getstrategyvariable{#1}\iftwopassdatafound#2\else#3\fi}

\def\doifstrategyvariable   #1#2{\doifstrategyvariableelse{#1}{#2}{}}
\def\doifnotstrategyvariable#1#2{\doifstrategyvariableelse{#1}{}{#2}}

%D New: only at start of columns; may change ! Rather 
%D interwoven and therefore to be integrated when the multi 
%D column modules are merged. 

%  already taken care of: \definesystemvariable{ks}

\def\setupcolumnspan[#1]%
  {\getparameters[\??ks][#1]}

\presetlocalframed
  [\??ks]

\setupcolumnspan
  [\c!n=2,
   \c!offset=\v!overlay,
   \c!kader=\v!uit]

\newbox\columnspanbox \let\postprocesscolumnspanbox\gobbleoneargument

\def\dostartcolumnspan[#1]%
  {\bgroup
   \setupcolumnspan[#1]%
   \forgetall
   \ifbinnenkolommen
     \advance\hsize by \intercolumnwidth
     \hsize=\@@ksn\hsize
     \advance\hsize by -\intercolumnwidth
   \fi
   \dowithnextbox
     {\setbox\columnspanbox=\box\nextbox
      \ifbinnenkolommen\wd\columnspanbox=\hsize\fi
      \postprocesscolumnspanbox\columnspanbox
      \scratchdimen=\ht\columnspanbox
      \setbox\columnspanbox=\hbox % depth to be checked, probably option!
        {\localframed[\??ks][\c!offset=\v!overlay]{\box\columnspanbox}}%
      \ht\columnspanbox=\scratchdimen
      \dp\columnspanbox=\dp\strutbox
      \wd\columnspanbox=\hsize
      \ifbinnenkolommen
        \ifnum\@@ksn>1
          \setvsize
          \dohandleallcolumns
            {\ifnum\currentcolumn>\@@ksn\else
               \global\setbox\currenttopcolumnbox=\vbox
                 {\ifnum\currentcolumn=1
                    \snaptogrid\vbox{\copy\columnspanbox}
                  \else
                    \snaptogrid\vbox{\vphantom{\copy\columnspanbox}}
                  \fi}%
               \wd\currenttopcolumnbox=\hsize
               \global\advance\vsize by -\ht\currenttopcolumnbox
             \fi}
          \global\pagegoal=\vsize
        \else
          \snaptogrid\vbox{\box\columnspanbox}
        \fi
      \else
        \snaptogrid\vbox{\box\columnspanbox}
      \fi
      \prevdepth\dp\strutbox
      \egroup}
     \vbox\bgroup
      %\topskipcorrection % becomes an option ! 
       \EveryPar{\begstrut\EveryPar{}}} % also ! 

\def\startcolumnspan%
  {\dosingleempty\dostartcolumnspan}

\def\stopcolumnspan%
  {\egroup}

%D For Ton. Do be documented.

\def\plaatsexterndocument[#1]%
  {\def\doexternaldocument[##1][##2][##3]%
     {\readlocfile{##2}{}{}}%
   \getvalue{\v!file:::#1}}

%D Far from complete. 

\def\startgeheel%
  {\startregelcorrectie
   \insidefloattrue}

\def\stopgeheel
  {\stopregelcorrectie}

%D No more news. 

\protect

%D A few local optimizations and new features, if defined: 

\readfile {cont-loc} {} {}

\endinput
