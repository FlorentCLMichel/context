%D \module
%D   [       file=cont-new,
%D        version=1995.10.10,
%D          title=\CONTEXT\ Miscellaneous Macros,
%D       subtitle=New Macros,
%D         author=Hans Hagen,
%D           date=\currentdate,
%D      copyright={PRAGMA / Hans Hagen \& Ton Otten}]
%C
%C This module is part of the \CONTEXT\ macro||package and is
%C therefore copyrighted by \PRAGMA. See mreadme.pdf for
%C details.

%D This file is loaded at runtime, thereby providing an
%D excellent place for hacks, patches, extensions and new
%D features.

% todo: mp-new
\unprotect

\writestatus{\m!systems}{beware: some patches loaded from cont-new.tex!}

% todo : test low level translation (nl->en) and optimize script

% todo : share symbols

% \definecolor[rollover:n][red]
% \definecolor[rollover:r][green]
% \definecolor[rollover:d][blue]

\definepalet
  [rollover]
  [n=red,
   r=green,
   d=blue]

% \newcounter\nofrollovers
%
% \def\dorollbutton[#1][#2]#3[#4]%
%   {\dontleavehmode
%    \bgroup
%    \doglobal\increment\nofrollovers
%    \unexpanded\def\dosetlocationbox[##1]##2[##3]%
%      {\getparameters[##1][##3]%
%       \definecolor[rollover][rollover:##2]%
%       \let\next\hbox
%       \doif{##2}{n}
%         {\doifvalue{##1\c!variant}\v!verborgen{\let\next\phantom}}%
%       \next
%         {\localframed[##1]
%            [\c!kaderkleur=rollover,\c!achtergrondkleur=rollover,\c!kleur=rollover]%
%            {\dolocationattributes{##1}\c!letter\c!kleur{#3}}}}%
%    \iffirstargument
%      \ifsecondargument
%        \def\setlocationbox##1{\dosetlocationbox[\??am#1]{##1}[#2]}%
%      \else
%        \doifassignmentelse{#1}
%          {\def\setlocationbox##1{\dosetlocationbox[\??bt]{##1}[#1]}}
%          {\def\setlocationbox##1{\dosetlocationbox[\??am#1]{##1}[]}}%
%      \fi
%    \else
%      \def\setlocationbox##1{\dosetlocationbox[\??bt]{##1}[]}%
%    \fi
%    % todo: share symbols
%    \definesymbol[rsym:\nofrollovers:n][\setlocationbox n]%
%    \definesymbol[rsym:\nofrollovers:r][\setlocationbox r]%
%    \definesymbol[rsym:\nofrollovers:d][\setlocationbox d]%
%    \nextsystemfield
%    \setupfield
%      [rollbutton]
%      [\c!kader=\v!uit,\c!offset=\v!overlay,\c!klikuit={#4}]%
%    \definefield
%      [\currentsystemfield][push][rollbutton]
%      [rsym:\nofrollovers:n,%
%       rsym:\nofrollovers:r,%
%       rsym:\nofrollovers:d]%
%    \fitfield[\currentsystemfield]%
%    \egroup}

\newcounter\nofrollovers
\newcounter\nofrollbuttons

\def\dorollbutton[#1][#2]#3[#4]%
  {\dontleavehmode
   \bgroup
   \doglobal\increment\nofrollovers
   \doglobal\increment\nofrollbuttons
   \unexpanded\def\dosetlocationbox[##1]##2[##3]%
     {\getparameters[##1][##3]%
      \definecolor[rollover][rollover:##2]%
      \doifelse{##2}{n}{\doifelsevalue{##1\c!variant}\v!verborgen\phantom\hbox}\hbox
        {\localframed[##1]
           [\c!kaderkleur=rollover,\c!achtergrondkleur=rollover,\c!kleur=rollover]%
           {\dolocationattributes{##1}\c!letter\c!kleur{#3}}}}%
   \iffirstargument
     \ifsecondargument
       \def\setlocationbox##1{\dosetlocationbox[\??am#1]{##1}[#2]}%
     \else
       \doifassignmentelse{#1}
         {\def\setlocationbox##1{\dosetlocationbox[\??bt]{##1}[#1]}}
         {\def\setlocationbox##1{\dosetlocationbox[\??am#1]{##1}[]}}%
     \fi
   \else
     \def\setlocationbox##1{\dosetlocationbox[\??bt]{##1}[]}%
   \fi
   % todo: share symbols, tricky since different dimensions
   \definesymbol[rsym:\nofrollovers:n][\setlocationbox n]%
   \definesymbol[rsym:\nofrollovers:r][\setlocationbox r]%
   \definesymbol[rsym:\nofrollovers:d][\setlocationbox d]%
   \setupfield
     [rollbutton]
     [\c!kader=\v!uit,
      \c!offset=\v!overlay,
      \c!klikuit={#4}]%
   \definefield
     [roll:\nofrollbuttons][push][rollbutton]
     [rsym:\nofrollovers:n,%
      rsym:\nofrollovers:r,%
      rsym:\nofrollovers:d]%
   \fitfield[roll:\nofrollbuttons]%
   \egroup}

\unexpanded\def\rollbutton
  {\dodoubleempty\dorollbutton}

% \def\do@@amrob[#1]#2\\%
%   {\txt\rollbutton[\currentmenu]{\ignorespaces#2\unskip}[#1]\\}%

% \appendtoks \let\rob\do@@amrob \to \everysetmenucommands

\def\menu@rob[#1]#2\\%
  {\@@amboxcommand\rollbutton[\currentmenu]{\ignorespaces#2\unskip}[#1]\\}%

\appendtoks \let\rob\menu@rob \to \everysetmenucommands

% calls:
%              {..} [JS..]
% [left]       {..} [JS..]
%        [a=b] {..} [JS..]
% [left] [a=b] {..} [JS..]
%
% \setupbuttons[offset=0pt,frame=off] % alternative=hidden
%
% \rollbutton {Manuals}       [JS(Goto_File{show-man.pdf})]
% \rollbutton {Articles}      [JS(Goto_File{show-art.pdf})]
% \rollbutton {Papers}        [JS(Goto_File{show-pap.pdf})]
% \rollbutton {Presentations} [JS(Goto_File{show-pre.pdf})]
% \rollbutton {Resources}     [JS(Goto_File{show-res.pdf})]
%
% \rob [JS(...)] bla bla \\

\unexpanded\def\overlayrollbutton
  {\dodoubleargument\dooverlayrollbutton}

\def\dooverlayrollbutton[#1][#2]%
  {\bgroup
   \nextsystemfield
   \setupfield
     [overlayrollbutton]
     [\c!kader=\v!uit,\c!offset=\v!overlay,\c!gebiedin={#1},\c!gebieduit={#2}]%
   \definesymbol
     [\currentsystemfield]
     [{\framed[\c!kader=\v!uit,\c!breedte=\overlaywidth,\c!hoogte=\overlayheight]{}}]%
   \definefield
     [\currentsystemfield][push][overlayrollbutton][\currentsystemfield][\currentsystemfield]%
   \fitfield[\currentsystemfield]%
   \egroup}

% \defineoverlay
%   [ShowMenu]
%   [{\overlayrollbutton[VideLayer{navigation}][HideLayer{navigation}]}]

\def\dodoflushlayer#1#2#3%
  {\ifundefined{\@@layerbox#3}%
     \ifcase#1\else\writestatus{layer}{unknown layer #3}\fi
   \else
     \bgroup
     \forgetall
     \offinterlineskip
     % needed because we need to handle method
\executeifdefined{\??ll\??ll\getvalue{\??ll#2\c!preset}}\gobbletwoarguments{#2}{}%
     %
     \doifvalue{\??ll#2\c!optie}\v!test\tracelayerstrue
     \iftracelayers\traceboxplacementtrue\fi
     \!!doneafalse
     \!!donebfalse
     \doifvalue{\??ll#2\c!methode}\v!overlay\!!doneatrue
     \doifvalue{\??ll#2\c!methode}\v!passend\!!donebtrue
     \!!donectrue
     \ifcase#1\else
       \doifnotvalue{\??ll#2\c!positie}\v!ja
         {\doifvalue{\??ll#2\c!herhaal}\v!ja\!!donecfalse
          \doifvalue{\??ll#2\c!status}\v!herhaal\!!donecfalse}%
     \fi
     \chardef\layerbox\csname\@@layerbox#3\endcsname
     % we need to copy in order to retain the negative offsets for a next
     % stage of additions, i.e. llx/lly accumulate in repeat mode and the
     % compensation may differ each flush depending on added content
     \setbox\nextbox \if!!doneb
       \vbox
         {\scratchdimen\getboxlly\layerbox
          \vskip-\scratchdimen
          \scratchdimen\getboxllx\layerbox
          \hskip-\scratchdimen
          \advance\scratchdimen-\wd\layerbox
          \hsize-\scratchdimen
          \if!!donec\box\else\copy\fi\layerbox}%
     \else
        \if!!donec\box\else\copy\fi\layerbox % sorry for the delay due to copying
     \fi
     \iftracelayers \ruledvbox \else \vbox \fi \if!!donea to \overlayheight \fi
       {\hbox \if!!donea to \overlaywidth \fi
          {% klopt dit? #3 en niet #2 ?
           \doifvalue{\??ll#3\realfolio\c!positie}\v!ja     {\xypos{lyr:#3:\realfolio}}%
           \doifoverlayelse{#3}
             {\box\nextbox}
             {\startlayoutcomponent{l:#3}{layer #3}\box\nextbox\stoplayoutcomponent}%
           \hss}%
        \vss}%
     \if!!donec
       \gsetboxllx\layerbox\zeropoint
       \gsetboxlly\layerbox\zeropoint
     \fi
     \egroup
   \fi}

\def\tightlayer[#1]%
  {\hbox
     {\def\currentlayer{#1}% todo: left/right
      \setbox\nextbox\emptybox        % hoogte/breedte are \wd\nextbox/\ht\nextbox
      \hsize\layerparameter\c!breedte % \overlaywidth   = \hsize
      \vsize\layerparameter\c!hoogte  % \overlaywheight = \vsize
      \composedlayer{#1}}}

\def\inlinedbox
  {\bgroup
   \dowithnextbox
     {\scratchdimen\nextboxht
      \advance\scratchdimen\nextboxdp
      \advance\scratchdimen-\lineheight
      \divide\scratchdimen\plustwo
      \advance\scratchdimen\strutdepth
      \setbox\nextbox\hbox{\lower\scratchdimen\flushnextbox}%
      \nextboxht\strutht
      \nextboxdp\strutdp
      \flushnextbox
      \egroup}%
     \hbox}

\def\dosetups#1% the grid option will be extended to other main modes
  {\executeifdefined{\??su\ifgridsnapping\v!grid\fi:#1}{\csname\??su:#1\endcsname}\empty}

\let\directsetup\dosetups

% \readfile{cont-exp}\donothing\donothing % speed up (5-20%)

\def\dimenratio#1#2% etex only
  {\withoutpt\the\dimexpr(2\dimexpr(#1)/(\dimexpr(#2)/32768))}

% in page-lyr

\definelayer[OTRTEXT] \setuplayer[OTRTEXT][\c!breedte=\zetbreedte,\c!hoogte=\teksthoogte]

\let\normalsettextpagecontent\settextpagecontent

\def\settextpagecontent#1#2#3% #2 and #3 will disappear
  {\doifelselayerdata{OTRTEXT}
     {\setbox#1\hbox to \zetbreedte
        {\startoverlay
           {\normalsettextpagecontent{#1}{#2}{#3}\box#1}
           {\tightlayer[OTRTEXT]}
         \stopoverlay}%
      \dp#1\zeropoint}%
     {\normalsettextpagecontent{#1}{#2}{#3}}}

% in page-set

\def\OTRSETdofinaloutput
  {\ifdim\ht\OTRfinalpagebox=\teksthoogte
    \bgroup % \let\OTRSETsetvsize\relax % prevents useless search for gap
     \ifcase\OTRSETbalancemethod
       \finaloutput\box\OTRfinalpagebox
     \else\ifdim\OTRSETbalht>\zeropoint
       % catch a bordercase
       \scratchdimen\OTRSETbalht
       \advance\scratchdimen\lineheight\relax
       \ifdim\scratchdimen>\teksthoogte
         % full page
         \finaloutput\box\OTRfinalpagebox
       \else
         % same page
         \global\setbox\OTRfinalpagebox \iftracecolumnset\ruledvbox\else\vbox\fi to \OTRSETbalht
           {\box\OTRfinalpagebox\vss}%
         \setlayer[OTRTEXT]{\box\OTRfinalpagebox}%
         \snaptogrid\vbox{\vskip\OTRSETbalht}% hack
       \fi
     \else
       \finaloutput\box\OTRfinalpagebox
     \fi \fi
     \globallet\OTRSETbalht\zeropoint
    \egroup
   \fi}

\def\doxprecurse#1#2%
  {\ifnum#1=\zerocount % no \ifcase
     \expandafter\gobblethreearguments
   \else
     #2\expandafter\expandafter\expandafter\doxprecurse\expandafter
   \fi\expandafter{\the\numexpr(#1-1)}{#2}}

\def\buttonframed{\dodoubleempty\localframed[\??bt]} % goodie

\unexpanded\def\asciistr#1{\convertargument#1\to\ascii{\verbatimfont\ascii}}

\prependtoks \setnormalcatcodes  \to \everyTEXinputmode
\appendtoks  \processingXMLfalse \to \everyTEXinputmode

\let\normalenableXML\enableXML % some day we move the normal \enableXML into the toks

\prependtoks \normalenableXML    \to \everyXMLinputmode
\appendtoks  \processingXMLtrue  \to \everyXMLinputmode

\def\enableXML {\setinputmode[XML]}
\def\disableXML{\setinputmode[TEX]}

\beginTEX

  % else the skip aborts the reshape process

  \def\shapefill{\vskip\onepoint\!!plus\lineheight\!!minus\lineheight\relax}

\endTEX

\beginETEX

  \def\shapefill{\vskip\zeropoint\!!plus\lineheight\!!minus\lineheight\relax}

\endETEX

% Currently there is a bug in \lastnodetype, so we will enable this
% feature when the bugfix is widespread.

% \beginETEX \lastnodetype
%
% \def\dodoreshapebox#1#2#3#4% \shapebox, \shapepenalty, \shapekern, \shapeskip
%   {\ifnum\lastnodetype=\@@gluenode % \ifcase\lastskip % \ifdim\lastskip=\zeropoint\relax
%      \shapeskip\lastskip
%      \global\setbox\tmpshapebox\normalvbox{#4\unvbox\tmpshapebox}%
%      \unskip
%    \else\ifnum\lastnodetype=\@@kernnode % \ifcase\lastkern % \ifdim\lastkern=\zeropoint\relax
%      \shapekern\lastkern
%      \global\setbox\tmpshapebox\normalvbox{#3\unvbox\tmpshapebox}%
%      \unkern
%    \else\ifnum\lastnodetype=\@@penaltynode % \ifcase\lastpenalty % \ifnum\lastpenalty=\zerocount
%      \shapepenalty\lastpenalty
%      \global\setbox\tmpshapebox\normalvbox{#2\unvbox\tmpshapebox}%
%      \unpenalty
%    \else
%      \setbox\shapebox\lastbox
%      \ifvoid\shapebox
%        \unskip\unpenalty\unkern
%      \else
%        \ifdim\wd\shapebox=\shapesignal\relax
%          \exitloop
%        \else
%          \shapecounter\zerocount
%          \global\setbox\tmpshapebox\normalvbox{#1\unvbox\tmpshapebox}%
%        \fi
%      \fi
%    \fi\fi\fi
%    \ifnum\shapecounter>100 % can be less
%      \message{<<forced exit from shapebox>>}%
%      \global\setbox\tmpshapebox\copy\oldshapebox
%      \exitloop
%    \else
%      \advance\shapecounter \plusone
%    \fi}
%
% \endETEX

\let\normaltype\type

\beginTEX

  \unexpanded\def\retype#1{\bgroup\convertargument#1\to\ascii\@EA\normaltype\@EA{\ascii}\egroup}

\endTEX

\beginETEX

  \unexpanded\def\retype#1{\scantokens{\normaltype{#1}}}

\endETEX

\def\simplifytype{\let\type\retype}

% \ruledhbox
%   {\startignorespaces
%      \def\oeps{a}
%      \startignorespaces
%        \def\oeps{a}
%      \stopignorespaces
%      \def\oeps{a}
%    \stopignorespaces
%    \oeps}

\newsignal\boissignal
\newcount \boislevel

\long\def\startignorespaces
  {\advance\boislevel\plusone
   \ifcase\boislevel\or \ifhmode
     \hskip\boissignal
   \fi \fi
   \ignorespaces}

\long\def\stopignorespaces
  {\ifcase\boislevel\or \ifhmode
    \doloop
      {\ifdim\lastskip=\zeropoint
         \exitloop
       \else\ifdim\lastskip=\boissignal
         \unskip
         \exitloop
       \else
         \unskip
       \fi\fi}%
   \fi \fi
   \advance\boislevel\minusone}

\defineblankmethod [\v!synchroniseer] {\verticalstrut\vskip-2\lineheight\verticalstrut}

% \vtop{\blank[synchronize]\blank[line]test}

\def\minimalhbox#1#%
  {\dowithnextbox
     {\bgroup
      \setbox\scratchbox\hbox#1{\hss}%
      \ifdim\nextboxwd<\wd\scratchbox\nextboxwd\wd\scratchbox\fi
      \flushnextbox
      \egroup}
     \hbox}

% manual
%
% externfiguur     -> grid        =ja|hoogte|diepte|halveregel|passend -> helemaal in details
% stelplaatsblokin -> zijuitlijnen=hoogte|diepte|regel|halveregel|grid -> halveregel in 'details'

% TODO: TEST FIRST, NO CORRECTION NEEDED IN GRID MODE, EVT OPTION

\def\OTRONEsomeherefloat[#1]% spacing between two successive must be better
  {\baselinecorrection                           % not really needed in grid mode:
  %\ifgridsnapping \else \baselinecorrection \fi % ! ! ! test test test ! ! ! !
   \doplacefloatbox
   \doinsertfloatinfo
   \dochecknextindentation\??bk}

% todo: switch koppelen aan par scheelt pos

% to be documented: \startspread .. \stopspread

% to be documented primarydef p crossed d
% to be documented PlainTextArea

% manual
%
% Sometimes the demands are getting pretty weird:
%
% \startitemize
%   \item test
%   \item test
%   \headsym{xx} test \par test
% \stopitemize

% wait till bugfix in etex is widespead
%
% \beginETEX \lastnodetype
%
% \def\removeunwantedspaces
%   {\ifhmode
%      \doloop{\ifnum\lastnodetype=\@@gluenode\unskip\else\exitloop\fi}%
%    \fi}
%
% \endETEX

% \def\dodimchoice#1#2#3%
%   {\ifx#3\relax
%      #1\@EA\gobbleuntilrelax
%    \else\ifdim#1#2%
%      #3\@EAEAEA\gobbleuntilrelax
%    \else
%      \@EAEAEA\dodimchoice
%    \fi\fi{#1}}

% \def\donumchoice#1#2#3%
%   {\ifx#3\relax
%      #1\@EA\gobbleuntilrelax
%    \else\ifnum#1#2%
%      #3\@EAEAEA\gobbleuntilrelax
%    \else
%      \@EAEAEA\dodimchoice
%    \fi\fi{#1}}

% \def\dimchoice#1#2{\dodimchoice{#1}#2\empty\relax}
% \def\numchoice#1#2{\donumchoice{#1}#2\empty\relax}

\def\gobbleuntilempty#1\empty{}

\def\dodimchoice#1#2#3%
  {\ifdim#1#2%
     #3\@EA\gobbleuntilempty
   \else
     \@EA\dodimchoice
   \fi{#1}}

\def\donumchoice#1#2#3%
  {\ifnum#1#2%
     #3\@EA\gobbleuntilempty
   \else
     \@EA\dodimchoice
   \fi{#1}}

\def\dimchoice#1#2{\dodimchoice{#1}#2{=#1}{#1}\empty}
\def\numchoice#1#2{\donumchoice{#1}#2{=#1}{#1}\empty}

% \the\dimexpr(\dimchoice {7pt}{{<10pt}{8pt}{<12pt}{9pt}{<15pt}{10pt}{=11pt}{12pt}})
% \the\dimexpr(\dimchoice{11pt}{{<10pt}{8pt}{<12pt}{9pt}{<15pt}{10pt}{=11pt}{12pt}})
% \the\dimexpr(\dimchoice{14pt}{{<10pt}{8pt}{<12pt}{9pt}{<15pt}{10pt}{=11pt}{12pt}})

\def\showsetupsdefinition[#1]{\showvalue{\??su:#1}} % temp hack for debugging

% documentation : \setupregister[alternative=a|b|A|B]

\def\pushXMLmeaning#1%
  {\@EA\pushmacro\csname\@@XMLelement:#1/\endcsname
   \@EA\pushmacro\csname\@@XMLelement:#1\endcsname
   \@EA\pushmacro\csname\@@XMLelement:/#1\endcsname}

\def\popXMLmeaning#1%
  {\@EA\popmacro\csname\@@XMLelement:#1/\endcsname
   \@EA\popmacro\csname\@@XMLelement:#1\endcsname
   \@EA\popmacro\csname\@@XMLelement:/#1\endcsname}

\def\defineXMLstore {\doquadrupleargument\dodefineXMLstore[\saveXMLasdata]}
\def\defineXMLgstore{\doquadrupleargument\dodefineXMLstore[\gsaveXMLasdata]}

\def\dodefineXMLstore[#1][#2][#3][#4]% element attribute prefix % will become faster
  {\defineXMLargument[#2][#3=\s!dummy]{#1{#4:\XMLop{#3}}}}

\def\countXMLchildren[#1]#2%
  {\startnointerference
     \doglobal\newcounter\nofXMLchildren
     \defineXMLargument[#1]{\doglobal\increment\nofXMLchildren}%
     \startXMLignore
       #2%
     \stopXMLignore
   \stopnointerference}

\unprotected \def\traceposstring#1#2#3%
  {\iftracepositions
     \smashedhbox%
       {#1{\infofont#2#3}%
        \scratchdimen.5\points
        \kern-2\scratchdimen
        \vrule\!!width4\scratchdimen\!!height\scratchdimen\!!depth\scratchdimen}%
   \fi}

% It took quite a while to figure this out (using the preliminary 1.5
% spec). There are still a lot of things to be implemented. This is
% the third alternative.

% todo: multiple instances, dus indirect

\let\currentrendering\empty

\definereference[StartCurrentRendering] [\v!StartRendering{\currentrendering}]
\definereference[StopCurrentRendering]  [\v!StopRendering {\currentrendering}]
\definereference[PauseCurrentRendering] [\v!PauseRendering{\currentrendering}]
\definereference[ResumeCurrentRendering][\v!ResumeRendering{\currentrendering}]

\newcounter\nofexternalrenderings

\def\useexternalrendering{\doquadrupleempty\douseexternalrendering}
\def\setinternalrendering{\dodoubleempty   \dosetinternalrendering}

\def\douseexternalrendering[#1][#2][#3][#4]% tag mime file options
  {\setgvalue{\??rd:#1}{\plusone{#1}{#2}{#3}{#4}}}

\def\dosetinternalrendering[#1][#2]% tag options {content}
  {\bgroup
   \dowithnextbox
     {\setgvalue{\??rd:#1}{\plustwo{#1}{IRO}{#1}{#2}}%
      \let\objectoffset\zeropoint
      \setobject{IRO}{#1}\hbox{\box\nextbox}%
      \egroup}%
     \hbox}

\def\checkrendering#1% let's hope that \next is not used
  {\iflocation
    \doifsomething{#1}%
      {\doifdefined{\??rd:#1}%
         {\expanded{\getvalue{\??rd::\number\renderingtype{#1}}%
            {\filterfromvalue{\??rd:#1}52}{\filterfromvalue{\??rd:#1}53}%
            {\filterfromvalue{\??rd:#1}54}{\filterfromvalue{\??rd:#1}55}}}}%
   \fi}

\setvalue{\??rd::1}{\doinsertrendering}
\setvalue{\??rd::2}{\doinsertrenderingobject}

\def\renderingtype   #1{\filterfromvalue{\??rd:#1}51}
\def\renderingoptions#1{\filterfromvalue{\??rd:#1}55}

\setexecutecommandcheck {startrendering}  \checkrendering
\setexecutecommandcheck {stoprendering}   \checkrendering
\setexecutecommandcheck {pauserendering}  \checkrendering
\setexecutecommandcheck {resumerendering} \checkrendering

% by using a nice trick (used in other places of context as well) we
% can easily overload the default size to match the opbject size

\def\renderingwidth {8cm}
\def\renderingheight{6cm}

\def\definerenderingwindow
  {\dodoubleempty\dodefinerenderingwindow}

\def\dodefinerenderingwindow[#1][#2]%
  {\presetlocalframed[\??rw#1]%
   \getparameters%
     [\??rw#1]%
     [\c!openpaginaactie=,\c!sluitpaginaactie=,%
      \c!breedte=\renderingwidth,\c!hoogte=\renderingheight,%
      #2]}

\def\setuprenderingwindow
  {\dodoubleargument\dosetuprenderingwindow}

\def\dosetuprenderingwindow[#1]%
  {\getparameters[\??rw#1]}

\def\placerenderingwindow
  {\dodoubleempty\doplacerenderingwindow}

\def\doplacerenderingwindow[#1][#2]%
  {\bgroup
   \edef\currentrendering{\ifsecondargument#2\else#1\fi}%
   \ifcase\renderingtype\currentrendering\or
     % a file
   \or
     % an object
     \getobjectdimensions{IRO}\currentrendering
     \scratchdimen\objectheight
     \advance\scratchdimen\objectdepth
     \edef\renderingheight{\the\scratchdimen}%
     \edef\renderingwidth{\objectwidth}%
   \fi
   % create fall back if needed
   \doifdefinedelse{\??rw#1\c!breedte}
     {\def\currentrenderingwindow{#1}}
     {\let\currentrenderingwindow\s!default
      \definerenderingwindow[\currentrenderingwindow]}%
   \checkrendering\currentrendering
   \handlereferenceactions{\getvalue{\??rw\currentrenderingwindow\c!openpaginaactie }}\dosetuprenderingopenpageaction
   \handlereferenceactions{\getvalue{\??rw\currentrenderingwindow\c!sluitpaginaactie}}\dosetuprenderingclosepageaction
   \localframed
     [\??rw\currentrenderingwindow][\c!offset=\v!overlay]%
     {\expanded{\doinsertrenderingwindow
        \noexpand\currentrendering\hsize\vsize{\renderingoptions\currentrendering}}}%
   \egroup}

% todo:
%
% \setinternalrendering[example-1][options]{}

% test file:
%
% \definerenderingwindow
%   [example]
%   [width=320pt,height=150pt,frame=off,
%    background=color,backgroundcolor=gray,
%    openpageaction=StartCurrentRendering,
%    closepageaction=NextPage]% StopCurrentRendering]
%
% \useexternalrendering[example-1][audio/mpeg]                   [eldorado.mp3]
% \useexternalrendering[example-2][audio/mpeg]                   [myst-12.mp3]
% \useexternalrendering[example-3][application/x-shockwave-flash][http://localhost/mb.swf] [auto]
% \useexternalrendering[example-4][application/x-shockwave-flash][celebration.swf]
% \useexternalrendering[example-5][video/quicktime]              [p1000726.mov]
% \useexternalrendering[example-6][application/smil]             [quadratic_map.smi]
%
% \def\renderingmenu[#1]%
%   {\hbox
%      {\setupbuttons[width=2.5em]%
%       \button{\symbol[StartRendering]} [StartRendering{#1}]\enspace
%       \button{\symbol[StopRendering]}  [StopRendering{#1}]\enspace
%       \button{\symbol[PauseRendering]} [PauseRendering{#1}]\enspace
%       \button{\symbol[ResumeRendering]}[ResumeRendering{#1}]}}
%
% \renderingmenu[example-1]\blank
% \renderingmenu[example-2]\blank
% \renderingmenu[example-3]\blank
% \renderingmenu[example-4] \placefigure{A ShockWave}{\placerenderingwindow[example][example-4]} \page
% \renderingmenu[example-5] \placefigure{A Movie}{\placerenderingwindow[example][example-5]} \page
% \renderingmenu[example-6] \placefigure{A Smile}{\placerenderingwindow[example][example-6]}

% will be a MyWay
%
% \setuplayout[grid=yes] \setupcaption[figure][inbetween=] \useMPlibrary[dum] \setupcolors[state=start]
%
% \starttext \showgrid \showstruts
%
% \input ward \placefigure{}{\externalfigure[dummy][width=.5\hsize,lines=1.4,grid=yes]}
% \input ward \placefigure{}{\externalfigure[dummy][width=.5\hsize,lines=1.4,grid=fit]}
% \input ward \placefigure{}{\externalfigure[dummy][width=.5\hsize,lines=1.4,grid=height]}
% \input ward
% \page
% \input ward \placefigure{}{\externalfigure[dummy][width=.5\hsize,lines=1.5,grid=yes]}
% \input ward \placefigure{}{\externalfigure[dummy][width=.5\hsize,lines=1.5,grid=fit]}
% \input ward \placefigure{}{\externalfigure[dummy][width=.5\hsize,lines=1.5,grid=height]}
% \input ward
% \page
% \input ward \placefigure{}{\externalfigure[dummy][width=.5\hsize,lines=1.6,grid=yes]}
% \input ward \placefigure{}{\externalfigure[dummy][width=.5\hsize,lines=1.6,grid=fit]}
% \input ward \placefigure{}{\externalfigure[dummy][width=.5\hsize,lines=1.6,grid=height]}
% \input ward
% \page
% \input ward \placefigure[none]{}{\externalfigure[dummy][width=.5\hsize,lines=1.4,grid=yes]}
% \input ward \placefigure[none]{}{\externalfigure[dummy][width=.5\hsize,lines=1.4,grid=fit]}
% \input ward \placefigure[none]{}{\externalfigure[dummy][width=.5\hsize,lines=1.4,grid=height]}
% \input ward
% \page
% \input ward \placefigure[none]{}{\externalfigure[dummy][width=.5\hsize,lines=1.5,grid=yes]}
% \input ward \placefigure[none]{}{\externalfigure[dummy][width=.5\hsize,lines=1.5,grid=fit]}
% \input ward \placefigure[none]{}{\externalfigure[dummy][width=.5\hsize,lines=1.5,grid=height]}
% \input ward
% \page
% \input ward \placefigure[none]{}{\externalfigure[dummy][width=.5\hsize,lines=1.6,grid=yes]}
% \input ward \placefigure[none]{}{\externalfigure[dummy][width=.5\hsize,lines=1.6,grid=fit]}
% \input ward \placefigure[none]{}{\externalfigure[dummy][width=.5\hsize,lines=1.6,grid=height]}
% \input ward
%
% \stoptext

% funny, as field action with e.g. dissolve ... only the field dissolves, bug?

\setglobalsystemreference\rt!exec{Transition}{transition}

%def\PDFexecutetransition {/Trans /Trans <</Type /Trans \executeifdefined{PDFpage\argumentA}\PDFpagereplace>>}
\def\PDFexecutetransition {/Trans /Trans <<\executeifdefined{PDFpage\argumentA}\PDFpagereplace>>}

% new, continuous blocks, \som \par \startdoorlopendblok ...

% \startitemize
%   \item                                      bagger
%   \item                                      bagger
%   \item          \startdoorlopendblok        bagger \stopdoorlopendblok
%   \item \endgraf \startdoorlopendblok        bagger \stopdoorlopendblok
%   \item \endgraf \startdoorlopendblok \strut bagger \stopdoorlopendblok
%   \item \startdoorlopendblok
%         \starttabulate
%         \NC test \NC test \NC \NR
%         \NC test \NC test \NC \NR
%         \NC test \NC test \NC \NR
%         \stoptabulate
%         \stopdoorlopendblok
%   \item test
% \stopitemize

\def\startdoorlopendblok % for special cases, don't change it too much and don't rely on it
  {\ifhmode\endgraf\nobreak\fi % don't remove the \nobreak
   \dowithnextboxcontent
     {\setlocalhsize \hsize\localhsize \forgetall}
     {\bgroup
      \forgeteverypar
      \forgetparskip
      \scratchdimen\nextboxht
      \advance\scratchdimen\nextboxdp
      \getnoflines\scratchdimen
      \advance\scratchdimen-\strutheight
      \setbox\nextbox\hbox{\lower\scratchdimen\box\nextbox}%
      \ht\nextbox\strutheight
      \dp\nextbox\strutdepth
      \setbox\nextbox\vbox
        {\indent\box\nextbox
         \endgraf
         \nobreak
         \advance\noflines\minusone
         \dorecurse\noflines{\crlf\nobreak}}%
      \verticalstrut
      \endgraf
      \nobreak
      \offinterlineskip
      \kern-2\lineheight % 2\lineheight when no vertical struts in main \vbox
      \nobreak
      \unvbox\nextbox
      \prevdepth\strutdepth
      % evt (eerst testen) een signal zodat een direct volgend blok goed gaat)
      \egroup}
     \vbox\bgroup
       \vskip-\lineheight \verticalstrut\endgraf
       \insidefloattrue
       \doinhibitblank} % beware, no \inhibitblank ! ! ! ! ! !

\def\stopdoorlopendblok
  {\endgraf\verticalstrut\endgraf\kern-2\lineheight
   \egroup}

% Just a simple and fast hanger, for usage in macros.

\def\setuphanging
  {\dodoubleempty\getparameters[\??ha]}

\setuphanging
  [\c!afstand=.5em]

\def\starthanging
  {\noindent\bgroup
   \dowithnextbox
     {\setbox\nextbox\hbox{\flushnextbox\hskip\@@haafstand}%
      \hangindent\nextboxwd
      \hangafter\plusone
      \flushnextbox\ignorespaces}
   \hbox}

\def\stophanging
  {\endgraf
   \egroup}

\def\definepushbutton % name optional setup
  {\dodoubleempty\dodefinepushbutton}

\def\dodefinepushbutton[#1][#2]% name setup
  {\dododefinepushbutton{#1}{n}{push}%
   \dododefinepushbutton{#1}{r}{\symbol[psym:#1:n]}%
   \dododefinepushbutton{#1}{d}{\symbol[psym:#1:r]}%
   \setvalue{pushbutton:#1}{\dohandlepushbutton{#1}{#2}}}

\def\dododefinepushbutton#1#2#3%
  {\doifsymboldefinedelse{psym:#1:#2}%
     \donothing{\definesymbol[psym:#1:#2][{#3}]}}

\def\definepushsymbol
  {\dotripleargument\dodefinepushsymbol}

\def\dodefinepushsymbol[#1][#2]% [#3]
  {\definesymbol[psym:#1:#2]}

\def\dopushbutton[#1][#2]%
  {\executeifdefined{pushbutton:#1}\gobbleoneargument{#2}}

\def\pushbutton
  {\dodoubleargument\dopushbutton}

\def\dohandlepushbutton#1#2#3% identifier setup script
  {\bgroup
   \nextsystemfield
   \setupfield
     [pushbutton]
     [\c!kader=\v!overlay,
      \c!offset=\v!overlay,
      \c!klikuit=#3,#2]%
   \definefield
     [\currentsystemfield]
     [push]
     [pushbutton]
     [psym:#1:n,psym:#1:r,psym:#1:d]%
   \fitfield
     [\currentsystemfield]%
   \egroup}

% \def\do@@ampsh
%   {\dodoubleargument\dodo@@ampsh}
%
% \def\dodo@@ampsh[#1][#2]#3\\%
%   {\txt\pushbutton[#1][#2]\\}%
%
%\appendtoks \let\psh\do@@ampsh \to \everysetmenucommands

\def\@@ampsh{\txt\pushbutton}

\appendtoks \let\psh\@@ampsh \to \everysetmenucommands

% \definepushbutton [reset]
%
% \definepushsymbol [reset] [n] [\uniqueMPgraphic{whatever}{color=green}]
% \definepushsymbol [reset] [r] [\uniqueMPgraphic{whatever}{color=white}]
%
% \startinteractionmenu[bottom]
%   \psh [reset] [JS(reset_something)] \\
% \stopinteractionmenu

\def\tabulaterule % to be redone, not correct
  {\dotabulaterule
     {\hrule\!!height.5\scratchdimen\!!depth.5\scratchdimen\relax
      \doifvalue{\??tt\currenttabulate\c!afstand}\v!grid
        {\kern-\scratchdimen}}} % experimental tm-prikkels

% todo: \setupinterlinespace[\c!regel=\v!vast] => ==\the\baselineskip

%%%%%%%% todo: \chardef\snapstruts=1 => d=l-h

\def\useMPvariables
  {\dodoubleargument\douseMPvariables}

\def\douseMPvariables[#1][#2]%
  {\def\@@meta{#1:}%
   \prepareMPvariables{#2}}

\def\processlinetableXMLfile#1%
  {\bgroup
   \let\startlinetable\donothing
   \let\stoplinetable \donothing
   \startlinetableanalysis\processXMLfile{#1}\stoplinetableanalysis
   \startlinetablerun     \processXMLfile{#1}\stoplinetablerun
   \egroup}

% experimental: \synchronizegrid bla bla bla

\newcounter\currentgridsync

\def\gridsynctag{grs:\currentgridsync}

\def\synchronizegrid
  {\doglobal\increment\currentgridsync
   \par\prevdepth\zeropoint
   \nointerlineskip
   \hpos\gridsynctag{\strut}\par
   \vskip-\lineheight
   \nointerlineskip
   % top of text
   \scratchdimen\MPy{\v!tekst:\MPp\gridsynctag}%
   \advance\scratchdimen\MPh{\v!tekst:\MPp\gridsynctag}%
   % move to first baseline
   \advance\scratchdimen-\topskip
   % subtract wrong baseline
   \advance\scratchdimen-\MPy\gridsynctag
   % get minimal number of lines
   \advance\scratchdimen\lineheight
   \getnoflines\scratchdimen
   % calculate difference
   \advance\scratchdimen-\noflines\lineheight\relax
   \scratchdimen-\scratchdimen\relax
   \ifdim\scratchdimen>\zeropoint
     \nointerlineskip
     \advance\scratchdimen-\lineheight
     \vskip\scratchdimen \dontleavehmode \quad \strut
     \par
  %\else
  %  \message{no grid correction: \the\scratchdimen}\wait
   \fi}

% needed for extreme

\definesystemvariable{ie}

% \def\definetest[#1]#2%
%   {\long\setvalue{\??ie#1}{#2}}

\def\definetest
  {\dodoubleempty\dodefinetest}

\def\dodefinetest[#1][#2]#3%
  {\setgvalue{\??ie#1}{#3}%
   \ifsecondargument
      \processaction
        [#2]
        [% first test true, rest depends
         \v!volgende=>\setgvalue{\??ie#1}{\setgvalue{\??ie#1}{#3}\firstoftwoarguments},
         % rest true if first true
         % \v!eerste=>\setgvalue{\??ie#1}{#3{\letgvalue{\??ie#1}%
         %              \firstoftwoarguments\firstoftwoarguments}%
         %              \secondoftwoarguments},
         % always true
               \v!ja=>\letgvalue{\??ie#1}\firstoftwoarguments,
         % always false
              \v!nee=>\letgvalue{\??ie#1}\secondoftwoarguments]%
   \fi}

\def\doperformtest#1%
  {\executeifdefined{\??ie#1}\secondoftwoarguments}

\def\definecolumnsethsize#1#2#3#4% will be improved/speed up
  {\bgroup
   \def\OTRSETidentifier{#1}%
   \ifcase\columnsetlevel\relax
     \mofcolumns\plusone
     \OTRSETinitializecolumns
     \OTRSETassignwidths
     \OTRSETsethsize
   \fi
   \!!counta#2\!!countb#3\docalculatecolumnsetspan
   \expandafter\egroup\expandafter\edef\expandafter
     #4\expandafter{\the\!!widtha}}

% so far

% between alignment lines certain rules apply, and even a
% simple test can mess up a table, which is why we have a
% special test facility
%
% \ruledvbox
%   {\starttabulate[|l|p|]
%    \NC 1test \NC test \NC \NR
%    \tableifelse{\doifelse{a}{a}}{\NC Xtest \NC test \NC \NR}{}%
%    \stoptabulate}

\long \def\tableifelse#1%
  {\TABLEnoalign{#1%
     {\aftergroup \firstoftwoarguments}%
     {\aftergroup\secondoftwoarguments}}}

% \long \def\tableif#1% whow, this is real ugly
%   {\TABLEnoalign{\let\gnext\gobbleoneargument#1%
%      {\let\gnext\firstofoneargument}}\gnext}

\long \def\tableiftextelse#1{\tableifelse{\doiftextelse{#1}}}

\def\overloaded#1#2%
  {\appendtoks
     \writestatus\m!systems{overloaded: \string#2}%
   \to \everybye
   #1#2}

\def\expandifnonempty#1%
  {\@EA\ifx\csname#1\endcsname\empty
     \expandafter\secondoftwoarguments
   \else
     \expandafter\firstoftwoarguments
   \fi
   {\csname#1\endcsname}}

\def\@@sectiekoppeling#1%
  {\expandifnonempty{\??ko#1\c!koppeling}{#1}}

\def\@@sectiesectie#1%
  {\expandifnonempty{\??ko#1\c!sectie}{\@@sectiekoppeling{#1}}}

\def\sectioncountervalue#1%
  {\@@sectionvalue{\@@sectiesectie{#1}}}

% todo namespace \@@meta:#1:... ! ! ! ! ! !

\def\presetMPvariable
  {\dodoubleargument\dopresetMPvariable}

\def\dopresetMPvariable[#1][#2=#3]%
  {\doifundefined{#1:#2}{\setvalue{#1:#2}{#3}}}

% experiment, not yet to be used

\def\displaybreak
  {\ifhmode
     \removeunwantedspaces
     \ifcase\raggedstatus\hfill\fi
     \strut\penalty-9999 % \break fails on case (3)
   \fi}

\def\startdisplay{\displaybreak\ignorespaces\startopelkaar}
\def\stopdisplay {\stopopelkaar\displaybreak\ignorespaces}

\def\tightvbox
  {\dowithnextbox{\nextboxdp\zeropoint\flushnextbox}\vbox}

\def\tightvtop
  {\dowithnextbox{\nextboxht\zeropoint\flushnextbox}\vtop}

% pretty important (esp since we now ignore shipouts)
%
% actually we should nil all writes, marks, specials

\appendtoks \globallet\popcolor\relax \to \everylastshipout

\def\incrementvalue#1{\expandafter\increment\csname#1\endcsname}
\def\decrementvalue#1{\expandafter\decrement\csname#1\endcsname}

% \translateMPinput{il2-pl}
%
% \startMPenvironment[global]
%   \setupbodyfont[plr]
% \stopMPenvironment
%
% \TeX: � �
%
% \startMPcode
% draw btex MetaPost: � � etex scaled 5 ;
% \stopMPcode

% \startcolumnset[two] \input tufte
% \startcolumnsetspan[two] \input tufte \stopcolumnsetspan
% \input tufte \stopcolumnset

% now in cont-loc.tex, for the sake of testing.
%
% %D When \type {\somecolor} is issued, we can savely assume
% %D grouping. Using \type {\groupedcommand} here (i.e.\ the
% %D definition of \type {\color}) is unsafe because in
% %D interferes with for instance switching attributes.
%
% \def\switchtocolor[#1]%
%   {\bgroup\startcolor[#1]
%    \aftergroup\stopcolor
%    \aftergroup\egroup}

% what is this stupid macro meant for:

\def\hyphenationpoint
  {\hskip\zeropoint}

\def\hyphenated#1%
  {\bgroup
   \!!counta\zerocount
   \def\hyphenated##1{\advance\!!counta\plusone}%
   \handletokens#1\with\hyphenated
   \!!countb\plusone
   \def\hyphenated##1%
     {##1%
      \advance\!!countb\plusone\relax
      \ifnum\!!countb>2 \ifnum\!!countb<\!!counta
        \hyphenationpoint
      \fi\fi}%
   \handletokens#1\with\hyphenated
   \egroup}

\def\obeysupersubletters
  {\let\super\normalsuper
   \let\suber\normalsuber
   \let\normalsuper\letterhat
   \let\normalsuber\letterunderscore
   \enablesupersub}

\def\obeysupersubmath
  {\let\normalsuper\letterhat
   \let\normalsuber\letterunderscore
   \enablesupersub}

%\let\normaltype\type
%
%\def\type#1%
%  {\expanded{\normaltype{\detokenize{#1}}}}

% {x123 \os x123} {\tfa x123 \os x123} {x123 \tx x123 \os x123}
% \definefontsynonym[OldStyle][Serif]
% {x123 \os x123} {\tfa x123 \os x123} {x123 \tx x123 \os x123}

% testen :
%
% \appendtoks
%   \let\registerparoptions\relax
% \to \everyforgetall

\def\startgridcorrection
  {\dosingleempty\dostartgridcorrection}

\def\dostartgridcorrection[#1]%
  {\ifgridsnapping
     \iffirstargument\doifsomething{#1}{\verplaatsopgrid[#1]}\fi
     \snaptogrid\vbox\bgroup
   \else
     \startbaselinecorrection
   \fi}

\def\stopgridcorrection
  {\ifgridsnapping
     \egroup
   \else
     \stopbaselinecorrection
   \fi}

\def\checkgridsnapping
  {\lineskip\ifgridsnapping\zeropoint\else\normallineskip\fi}

\def\startplaatsen
  {\dosingleempty\dostartplaatsen}

\def\dostartplaatsen[#1]% tzt n*links etc
  {\endgraf
   \noindent\bgroup
   \setlocalhsize
   \hbox to \localhsize\bgroup
     \doifnot{#1}\v!links\hss
     \def\stopplaatsen
       {\unskip\unskip\unskip
        \doifnot{#1}\v!rechts\hss
        \egroup
        \egroup
        \endgraf}%
     \gobblespacetokens}

% \startplaatsen[links] bla \stopplaatsen

% we don't register the paragraph characteristics, only the
% width

\appendtoks
  \setinnerparpositions % see "techniek" for application
\to \everytabulate

\appendtoks \checkcurrentlayout \to \everystarttext

\def\flushfootnotes  {\flushnotes}
\def\doflushfootnotes{\doflushnotes}

%D This alternative is slower, since it works on top of the
%D color (stack) mechanism, but it does provide nesting.

\def\dosetrastercolor#1%
  {\edef\@@cl@@s{#1}%
   \ifx\@@cl@@s\empty
     \let\@@cl@@s\@@rsraster
   \fi
   \setevalue{\??cr\??rs}{\colorSpattern}}

% beware, don't add extra grouping, else color in tables
% fails

\def\localstartraster[#1]%
  {\ifincolor\dosetrastercolor{#1}\localstartcolor[\??rs]\fi}

\def\startraster[#1]%
  {\ifincolor\dosetrastercolor{#1}\startcolor[\??rs]\fi}

\def\localstopraster{\ifincolor\localstopcolor\fi}
\def\stopraster     {\ifincolor\stopcolor\fi}

\def\fontclassname#1#2%
  {\ifcsname\??ff#1#2\endcsname
     \fontclassname{#1}{\csname\??ff#1#2\endcsname}%
   \else\ifcsname\??ff#2\endcsname
     \fontclassname{#1}{\csname\??ff#2\endcsname}%
   \else
     #2%
   \fi\fi}

\def\defineclassfontsynonym
  {\dotripleargument\dodefineclassfontsynonym}

\def\dodefineclassfontsynonym[#1][#2][#3]%
  {\definefontsynonym[#1][\fontclassname{#2}{#3}]}

%\definefontsynonym [KopFont] [\fontclassname{officina}{SerifBold}]
%
%\defineclassfontsynonym [KopFont] [officina] [SerifBold]

\def\startcolumnmakeup % don't change
  {\bgroup
   \getrawnoflines\teksthoogte % teksthoogte kan topskip hebben, dus raw
   \scratchdimen\noflines\lineheight
   \advance\scratchdimen-\lineheight
   \advance\scratchdimen\topskip
   \setbox\scratchbox
   \ifcase\showgridstate\vbox\else\ruledvbox\fi to \scratchdimen\bgroup
   \forgetall} % ! don't change

\def\stopcolumnmakeup
  {\egroup
   \dp\scratchbox\zeropoint
   \wd\scratchbox\tekstbreedte
   \box\scratchbox
   \egroup
   \synchronizehsize}

% todo : hoe komt box er uit

\long\def\startexternalfigure
  {\dotripleempty\dostartexternalfigure}

\long\def\dostartexternalfigure[#1][#2][#3]#4\stopexternalfigure
  {\gdef\figuredescription{#4}%
   \externalfigure[#1][#2][#3]%
   \globallet\figuredescription\empty}

\let\figuredescription\empty

% beware, changing this will break some code (like pos/backgrounds)

\newtoks\everyfirstparagraphintro
\newtoks\everynextparagraphintro

\chardef\everyparagraphintro\zerocount

\def\setupparagraphintro
  {\dodoubleempty\dosetupparagraphintro}

\def\dosetupparagraphintro[#1][#2]%
  {\processallactionsinset
     [#1]
     [   \v!reset=>\global\chardef\everyparagraphintro\zerocount
                   \global\everyfirstparagraphintro\emptytoks
                   \global\everynextparagraphintro \emptytoks,
        \v!eerste=>\global\chardef\everyparagraphintro\plusone
                   \doglobal\appendtoks#2\to\everyfirstparagraphintro,
      \v!volgende=>\ifcase\everyparagraphintro\global\chardef\everyparagraphintro\plusone\fi
                   \doglobal\appendtoks#2\to\everynextparagraphintro,
           \v!elk=>\ifcase\everyparagraphintro\global\chardef\everyparagraphintro\plustwo\fi
                   \doglobal\appendtoks#2\to\everyfirstparagraphintro
                   \doglobal\appendtoks#2\to\everynextparagraphintro]}

\def\doinsertparagraphintro
  {\ifcase\everyparagraphintro\relax
     % no data
   \or
     % first data
     \global\chardef\everyparagraphintro\plustwo
     \scratchtoks\everyfirstparagraphintro
     \global\everyfirstparagraphintro\emptytoks
   \or
     % next data
     \scratchtoks\everynextparagraphintro
   \fi
   \the\scratchtoks}

\def\insertparagraphintro
  {\ifcase\everyparagraphintro\else\@EA\doinsertparagraphintro\fi}

\appendtoks\insertparagraphintro\to\everypar

%D \starttypen
%D \setupparagraphintro[first][\hbox to 3.5em{\tt FIRST \hss}]
%D \setupparagraphintro[first][\hbox to 3.5em{\tt TSRIF \hss}]
%D \setupparagraphintro[next] [\hbox to 3.5em{\tt NEXT  \hss}]
%D \setupparagraphintro[next] [\hbox to 3.5em{\tt TXEN  \hss}]
%D \setupparagraphintro[each] [\hbox to 3.0em{\tt EACH  \hss}]
%D \setupparagraphintro[each] [\hbox to 3.0em{\tt HCEA  \hss}]
%D
%D some paragraph \par
%D some paragraph \par
%D some paragraph \par
%D
%D \definelabel[parnumber]
%D
%D \setupparagraphintro[reset,each][\inleft{\slxx\parnumber}]
%D
%D some paragraph \par
%D some paragraph \par
%D some paragraph \par
%D \stoptypen

% wrong names

\newif\ifpagechanged \let\lastchangedpage\empty

\def\checkpagechange#1%
  {\gettwopassdata\s!paragraph
   \pagechangedfalse
   \iftwopassdatafound
     \ifnum\twopassdata>0\getvalue{\s!paragraph:p:#1}\relax
       \pagechangedtrue
     \fi
   \fi
   \ifpagechanged
     \letgvalue{\s!paragraph:p:#1}\twopassdata
     \globallet\lastchangedpage\twopassdata
   \else
     \globallet\lastchangedpage\realfolio
   \fi
   \doparagraphreference}

\def\changedpage#1%
  {\getvalue{\s!paragraph:p:#1}}

% incomplete, will be a special case of float placement

\def\startfixed{\dosingleempty\dostartfixed}

\long\def\dostartfixed[#1]%
  {\expanded{\dowithnextbox{\noexpand\dodofixed{\ifhmode0\else1\fi}{#1}}}%
   \vbox\bgroup
   \setlocalhsize}

\def\stopfixed
  {\egroup}

\def\dodofixed#1#2%
  {\ifcase#1\relax
     \processaction
       [#2]
       [   \v!hoog=>\bbox   {\flushnextbox},
           \v!laag=>\tbox   {\flushnextbox},
         \v!midden=>\vcenter{\flushnextbox},
           \v!laho=>\vcenter{\flushnextbox},
        \s!unknown=>\tbox   {\flushnextbox},
        \s!default=>\tbox   {\flushnextbox}]%
   \else
     \startbaselinecorrection
     \noindent\flushnextbox
     \stopbaselinecorrection
   \fi}

% \startitemize
%
% \item \externalfigure[koe][height=2cm]
% \item \externalfigure[koe][height=2cm]
% \item \externalfigure[koe][height=2cm]
% \item \externalfigure[koe][height=2cm]
%
% \page
%
% \item \startfixed      \externalfigure[koe][height=2cm]\stopfixed
% \item \startfixed[high]\externalfigure[koe][height=2cm]\stopfixed
% \item \startfixed[low] \externalfigure[koe][height=2cm]\stopfixed
% \item \startfixed[lohi]\externalfigure[koe][height=2cm]\stopfixed
%
% \page
%
% \item test \startfixed      \externalfigure[koe][height=2cm]\stopfixed
% \item test \startfixed[high]\externalfigure[koe][height=2cm]\stopfixed
% \item test \startfixed[low] \externalfigure[koe][height=2cm]\stopfixed
% \item test \startfixed[lohi]\externalfigure[koe][height=2cm]\stopfixed
%
% \page
%
% \item test \par \startfixed      \externalfigure[koe][height=2cm]\stopfixed
% \item test \par \startfixed[high]\externalfigure[koe][height=2cm]\stopfixed
% \item test \par \startfixed[low] \externalfigure[koe][height=2cm]\stopfixed
% \item test \par \startfixed[lohi]\externalfigure[koe][height=2cm]\stopfixed
%
% \stopitemize

% \def\docalculatefigurenorm#1#2%
%   {\dodocalculatefigurenorm{#1}[#2\empty\empty]}
%
% \def\dodocalculatefigurenorm#1[#2#3#4]#5#6#7%
%   {\ExpandFirstAfter\processaction
%       [#2#3#4]
%       [     \v!max=>\global#1=#6\relax,
%           \v!kolom=>\global#1=#6\relax,
%           \v!tekst=>\global#1=#6\relax,
%         \v!passend=>\global#1=#7\relax,
%            \v!ruim=>\global#1=#7\relax
%                     \global\advance #1 -4\@@exkorps\relax,
%        #2*\v!kolom=>\global#1=#6\relax
%                     \ifbinnenkolommen
%                       \global\advance#1 \intercolumnwidth
%                       \global\multiply#1 #2\relax
%                       \global\advance#1 -\intercolumnwidth
%                     \fi,
%        #2*\v!tekst=>\global#1=\zetbreedte
%                     \global\advance#1 \papierbreedte,
%         \s!default=>\doifsomething{#5}{\global#1=#5\relax},
%         \s!unknown=>\global#1=\@@exkorps\relax
%                     \global\divide#1 \!!ten\relax
%                     \global\multiply#1 #2#3#4\relax]}

% still needed for uguide

\let\placefloatlabel          \placefloatcaption
\let\placefloatlabeltext      \placefloatcaptiontext
\let\placefloatlabelreference \placefloatcaptionreference

\def\obeyfollowingtoken{{}}  % end \cs scanning

\def\gobbleparameters{\doquadrupleempty\dogobbleparameters}
\def\dogobbleparameters[#1][#2][#3][#4]{}

% documentation

% \starttable[|||]
% \HL
% \VL test \VS test \VL \FR
% \VL test \VD test \VL \MR
% \VL test \VT test \VL \LR
% \HL
% \stoptable

%D To be documented, \type {\includemenu[menu]}.
%D To be documented, \type {\emphbf} cum suis.

%D A prelude to strategies. Note for myself: overloads
%D previous stuff from local pragma test files.

\def\s!strategy{strategy}

\def\currentstrategypass    {1}
\def\currentstrategyvariable{0}
\def\maximumstrategypass    {8}

\newconditional\strategypassneeded
\newconditional\strategypassforced

\definetwopasslist{\s!strategy}

\def\registerstrategypass%
  {\ifnum\currentstrategypass>\maximumstrategypass \else
     \ifconditional\strategypassforced
       \doglobal\increment\currentstrategypass
     \else%\ifconditional\strategypassneeded
       %\doglobal\increment\currentstrategypass
     \fi%\fi
   \fi
   \savecurrentvalue\currentstrategypass{\currentstrategypass}}

\appendtoks \registerstrategypass \to \everybye % \everylastshipout

\def\setstrategyvariable#1#2% key value
  {%\doifnotstrategyvariable{#1}{\global\settrue\strategypassneeded}%
   \doglobal\increment\currentstrategyvariable
   \expanded{\immediatewriteutilitycommand{\noexpand
     \twopassentry{\s!strategy}{\currentstrategyvariable}{#1::#2}}}}

\def\doifstrategyvariableelse#1#2#3%
  {\getstrategyvariable{#1}\iftwopassdatafound#2\else#3\fi}

\def\getstrategyvariable#1% key
  {\findtwopassdata{\s!strategy}{#1::}%
   \setxvalue{\s!strategy:#1}{\twopassdata}}

\def\retainstrategyvariable#1% key
  {\expanded{\setstrategyvariable{#1}{\strategyvariable{#1}}}}

\def\strategyvariable#1% key
  {\csname\s!strategy:#1\endcsname}

\let\stratvar\strategyvariable

\def\forcestrategy{\global\settrue \strategypassforced}
\def\abortstrategy{\global\setfalse\strategypassforced}

\def\doifstrategyvariableelse#1#2#3%
  {\getstrategyvariable{#1}\iftwopassdatafound#2\else#3\fi}

\def\doifstrategyvariable   #1#2{\doifstrategyvariableelse{#1}{#2}{}}
\def\doifnotstrategyvariable#1#2{\doifstrategyvariableelse{#1}{}{#2}}

%D New: only at start of columns; may change ! Rather
%D interwoven and therefore to be integrated when the multi
%D column modules are merged.

%  already taken care of: \definesystemvariable{ks}

% is buggy now and does not work any longer

\def\setupcolumnspan[#1]%
  {\getparameters[\??ks][#1]}

\presetlocalframed
  [\??ks]

\setupcolumnspan
  [\c!n=2,
   \c!offset=\v!overlay,
   \c!kader=\v!uit]

\newbox\columnspanbox \let\postprocesscolumnspanbox\gobbleoneargument

\def\dostartcolumnspan[#1]%
  {\bgroup
   \setupcolumnspan[#1]%
   \forgetall
   \ifbinnenkolommen
     \advance\hsize \intercolumnwidth
     \hsize\@@ksn\hsize
     \advance\hsize -\intercolumnwidth
   \fi
   \dowithnextbox
     {\setbox\columnspanbox\flushnextbox
      \ifbinnenkolommen\wd\columnspanbox\hsize\fi
      \postprocesscolumnspanbox\columnspanbox
      \scratchdimen\ht\columnspanbox
      \setbox\columnspanbox\hbox % depth to be checked, probably option!
        {\localframed[\??ks][\c!offset=\v!overlay]{\box\columnspanbox}}%
      \ht\columnspanbox\scratchdimen
      \dp\columnspanbox\strutdp
      \wd\columnspanbox\hsize
      \ifbinnenkolommen
        \ifnum\@@ksn>1
          \setvsize
          \dohandleallcolumns
            {\ifnum\currentcolumn>\@@ksn\else
               \global\setbox\currenttopcolumnbox=\vbox
                 {\ifnum\currentcolumn=1
                    \snaptogrid\vbox{\copy\columnspanbox}
                  \else
                    \snaptogrid\vbox{\vphantom{\copy\columnspanbox}}
                  \fi}%
               \wd\currenttopcolumnbox\hsize
               \global\advance\vsize -\ht\currenttopcolumnbox
             \fi}
          \global\pagegoal\vsize
        \else
          \snaptogrid\vbox{\box\columnspanbox}
        \fi
      \else
        \snaptogrid\vbox{\box\columnspanbox}
      \fi
      \endgraf
      \prevdepth\strutdp
      \egroup}
     \vbox\bgroup
      %\topskipcorrection % becomes an option !
       \EveryPar{\begstrut\EveryPar{}}} % also !

\def\startcolumnspan
  {\dosingleempty\dostartcolumnspan}

\def\stopcolumnspan
  {\egroup}

%D For Ton. To be documented.

\def\plaatsexterndocument[#1]%
  {\def\doexternaldocument[##1][##2][##3]%
     {\readlocfile{##2}\donothing\donothing}%
   \getvalue{\v!file:::#1}}

%D Far from complete.

\def\startgeheel
  {\startregelcorrectie
   \insidefloattrue}

\def\stopgeheel
  {\stopregelcorrectie}

%D No more news.

\protect

%D A few local optimizations and new features, if defined:

\readfile {cont-loc} {} {}

\endinput