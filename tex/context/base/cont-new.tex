%D \module
%D   [       file=cont-new,
%D        version=1995.10.10,
%D          title=\CONTEXT\ Miscellaneous Macros,
%D       subtitle=New Macros,
%D         author=Hans Hagen,
%D           date=\currentdate,
%D      copyright={PRAGMA / Hans Hagen \& Ton Otten}]
%C
%C This module is part of the \CONTEXT\ macro||package and is
%C therefore copyrighted by \PRAGMA. See mreadme.pdf for
%C details.

% manual : offsetbox alignbox
% todo achtergronden in kolommen

%D This file is loaded at runtime, thereby providing an
%D excellent place for hacks and new features.

\unprotect

\setupclipping
  [\c!linkeroffset=\zeropoint,
   \c!rechteroffset=\zeropoint,
   \c!bovenoffset=\zeropoint,
   \c!onderoffset=\zeropoint]

\def\doclip[#1]% nb top->bottom left->right
  {\bgroup
   \getparameters[\??cp][#1]%
   \dowithnextbox
     {\ifdim\@@cpbreedte>\zeropoint
        \dimen0=\@@cpbreedte
        \dimen4=\@@cphoffset
      \else
        \dimen0=\wd\nextbox
        \divide\dimen0 \@@cpnx
        \dimen4=\@@cpx\dimen0
        \advance\dimen4 -\dimen0
        \dimen0=\@@cpsx\dimen0
      \fi
      \relax % sure
      \ifdim\@@cphoogte>\zeropoint
        \dimen2=\@@cphoogte
        \dimen6=\ht\nextbox
        \advance\dimen6 -\@@cpvoffset
        \advance\dimen6 -\dimen2
      \else
        \dimen2=\ht\nextbox
        \divide\dimen2 \@@cpny
        \dimen6=-\@@cpy\dimen2
        \advance\dimen6 -\@@cpsy\dimen2
        \advance\dimen6 \dimen2
        \dimen2=\@@cpsy\dimen2
        \advance\dimen6 \ht\nextbox
      \fi
      \setbox\nextbox\hbox                       % old 
        {\advance\dimen4 -\@@cplinkeroffset      % new !
         \advance\dimen6  \@@cpbovenoffset       % new !
         \hskip-\dimen4\lower\dimen6\box\nextbox}% old 
      \wd\nextbox\zeropoint
      \ht\nextbox\zeropoint
      \dp\nextbox\zeropoint
      \setbox\nextbox\hbox
        {\advance\dimen0 \@@cplinkeroffset        % new !
         \advance\dimen0 \@@cprechteroffset       % new !
         \advance\dimen2 \@@cpbovenoffset         % new !
         \advance\dimen2 \@@cponderoffset         % new !
         \dostartclipping\@@cpmp{\dimen0}{\dimen2}% old
           \box\nextbox
         \dostopclipping}%
      \setbox\nextbox\hbox                      % new !
        {\dimen0-\@@cplinkeroffset              % new !
         \dimen2-\@@cpbovenoffset               % new !
         \hskip\dimen0\lower\dimen2\box\nextbox}% new !
      \wd\nextbox\dimen0
      \ht\nextbox\dimen2
      \dp\nextbox\zeropoint
      \box\nextbox
      \egroup}%
   \hbox}

% beware, we have clipping offsets of 2\lineheight by default 

\def\columntextareaparameter#1%
  {\csname\??mt\currentcolumntestarea#1\endcsname}

\def\dodoplacecolumntextareas#1#2%
  {\def\currentcolumntestarea{#1#2}%
   \!!counta\columntextareaparameter\c!x
   \!!countb\columntextareaparameter\c!nx
   \docalculatecolumnsetspan
   \!!heighta\columntextareaparameter\c!ny\lineheight
   % to do: met/zonder ht/dp
   \ifnum\columntextareaparameter\c!y=\zerocount
     \advance\!!heighta -\lineheight
     \advance\!!heighta \topskip
   \fi
   \advance\!!heighta -\lineheight % option
   \setbox\scratchbox\vbox
     {\donetrue\localframed
        [\??mt\currentcolumntestarea] 
        [\c!breedte=\!!widtha,\c!hoogte=\!!heighta,\c!regels=]
        {\columntextareaparameter\empty}}%
   \!!counta\columntextareaparameter\c!x
   \!!countb\columntextareaparameter\c!y
   \advance\!!countb \columntextareaparameter\c!ny
   \advance\!!countb \minusone
   \OTRSETsetgridcell
     \!!counta\!!countb
     \hbox{\clip
       [\c!bovenoffset=\columntextareaparameter\c!clipoffset,%
        \c!onderoffset=\columntextareaparameter\c!clipoffset,%
        \c!linkeroffset=\columntextareaparameter\c!clipoffset,%
        \c!breedte=\!!widthb,%
        \c!hoogte=\!!heighta]%
       {\copy\scratchbox}}%
   \ifcase\!!countc\else
     \advance\!!counta \columntextareaparameter\c!nx
     \advance\!!counta -\!!countc
     \advance\!!widtha -\!!widthb
     \OTRSETsetgridcell
       \!!counta\!!countb
       \hbox
         {\hskip-\namedlayoutparameter\v!oneven\c!rugwit
          \clip
            [\c!bovenoffset=\columntextareaparameter\c!clipoffset,%
             \c!onderoffset=\columntextareaparameter\c!clipoffset,%
             \c!rechteroffset=\columntextareaparameter\c!clipoffset,%
             \c!breedte=\!!widtha,%
             \c!hoogte=\!!heighta,%
             \c!hoffset=\!!widthb]%
            {\copy\scratchbox}}%
   \fi}

\def\NormalizeFontSize#1#2#3#4#5% the normal struggle with accuracy 
  {\bgroup
   \dimen0=#4% #4 can be \ht0 or so
   \setbox0\hbox{\definedfont[#5 at 5pt]#3}% 10pt 
   \ifdim\wd0>\zeropoint
     \dimen2=#10 % #1 is \wd or \ht
     \dimen4=\maxdimen % 10000pt
     \divide\dimen4 \dimen2
     \divide\dimen0 1638 % 1000
     \dimen0=\number\dimen4\dimen0
     \divide \dimen0 \plustwo % ... 
     \xdef\TheNormalizedFontSize{\the\dimen0}%
   \else
     \dimen0\bodyfontsize
   \fi
   \definedfont[#5 at \the\dimen0]%
   \expandafter
   \egroup
   \expandafter\font\expandafter#2\fontname\font\relax}



% todo namespace \@@meta:#1:... ! ! ! ! ! !

\def\presetMPvariable  
  {\dodoubleargument\dopresetMPvariable}

\def\dopresetMPvariable[#1][#2=#3]% 
  {\doifundefined{#1:#2}{\setvalue{#1:#2}{#3}}}

\def\complexcolumnbreak[#1]% if empty, do nothing and avoid processing
  {\doifsomething{#1}{\executecolumnbreakhandlers{#1}}}   

\def\OTRSETsethsize
  {%\OTRSETassignwidth\OTRSETidentifier\to\localcolumnwidth
   \localcolumnwidth\OTRSETlocalwidth\mofcolumns
   \tekstbreedte\localcolumnwidth
   \hsize\localcolumnwidth}

\def\OTRSETsynchronizehsize  
  {\ifcase0\getvalue{\??mc\??mc\c!breedte}\else % some width set  
     \bgroup
     \scratchdimen\OTRSETlocalwidth\mofcolumns
     \ifdim\scratchdimen=\tekstbreedte
       \egroup 
     \else
       \egroup \OTRSETsethsize % only if change in width and \column/\break 
     \fi
   \fi } 

\def\filluparrangedpages % beware: \realpageno is 1 ahead
  {\ifarrangingpages
     \scratchcounter\realpageno 
     \advance\scratchcounter \minusone
     \dosetmodulo\scratchcounter\arrangedpageT\scratchcounter 
     \ifcase\scratchcounter\else
       \advance\scratchcounter \plusone
       \dostepwiserecurse\scratchcounter\arrangedpageT\plusone
         {\noheaderandfooterlines\ejectdummypage}%
     \fi
   \fi}

\def\substituteincommalist#1#2#3% old, new, list (slooow) 
  {\edef\!!stringb{#1}%
   \edef\!!stringd{#2}%
   \let\!!stringa#3%
   \let#3\empty
   \def\dosubstituteincommalist##1%
     {\edef\!!stringc{##1}%
      \ifx\!!stringb\!!stringc
        \ifx\!!stringd\empty\else
          \edef#3{#3\ifx#3\empty\else,\fi\!!stringd}%
        \fi
        \def\docommando####1{\edef#3{#3,####1}}%
      \else
        \edef#3{#3\ifx#3\empty\else,\fi##1}%
      \fi}%
   \@EA\rawprocesscommacommand\@EA[\!!stringa]\dosubstituteincommalist}

% experiment, not yet to be used 

\def\displaybreak
  {\ifhmode
     \removeunwantedspaces
     \ifcase\raggedstatus\hfill\fi
     \strut\penalty-9999 % \break fails on case (3) 
   \fi}

\def\startdisplay{\displaybreak\ignorespaces\startopelkaar}
\def\stopdisplay {\stopopelkaar\displaybreak\ignorespaces}

\def\tightvbox
  {\dowithnextbox{\dp\nextbox\zeropoint\box\nextbox}\vbox}
\def\tightvtop
  {\dowithnextbox{\ht\nextbox\zeropoint\box\nextbox}\vtop}

% cleaner 

\def\@@nmpre#1{\doiftext{#1}{{#1}\tfskip}}
\def\@@nmpos#1{\doiftext{#1}{\tfskip{#1}}}

% newer 

\def\@@nmprepos#1#2#3#4#5%
  {\doifelsenothing\@@nmbreedte
     {\doiftext{#5}{#1{#5}#2}}
     {\doiftext{#5}{\hbox to \@@nmbreedte{#3{#5}#4}}}}

\def\@@nmpre{\@@nmprepos\empty\tfskip\relax\hss}
\def\@@nmpos{\@@nmprepos\tfskip\empty\hss\relax}

\def\startpagefigure
  {\dodoubleempty\dostartpagefigure}

\def\dostartpagefigure[#1][#2]%
  {\bgroup
   \getparameters[\??ex][\c!offset=\!!zeropoint,#2]%
   \startTEXpage[\c!offset=\@@exoffset]%
     \externalfigure[#1][#2]\ignorespaces}

\def\stoppagefigure
  {\stopTEXpage
   \egroup}

\def\pagefigure
  {\dodoubleempty\dopagefigure}

\def\dopagefigure[#1][#2]%
  {\dostartpagefigure[#1][#2]\stoppagefigure}

\def\doprocesstabskipline#1%
  {\bgroup
   \scratchcounter\plusone
   \dodoprocesstabskipline#1\relax 
   \egroup}
      
\def\dodoprocesstabskipline#1%
  {\ifnum\scratchcounter>\spacespertab\relax
     \donetrue \else \donefalse \advance
   \fi \scratchcounter \plusone
   \ifx#1\relax \else
     \ifcase\tabskipmode
       % can't happen 
     \or 
       % go on 
     \else\ifnum`#1<128
       % ok, no special character  
     \else\ifnum\catcode`#1=\active
       % quits parsing, else utf lookahead problems  
       \chardef\tabskipmode\zerocount
     \fi\fi\fi
     \ifcase\tabskipmode 
       \@EAEAEA#1%
     \else
       \@EAEAEA#1\@EAEAEA\dodoprocesstabskipline
     \fi
   \fi}

\setuptyping[\c!tab=\s!ascii] % better default than \v!yes 

% obey eigennummer

\def\doresetsectioncounters#1%
  {\resetcounter{\??se#1}%
   \letgvalue{\??se#1\c!eigennummer}\relax
   \donexttracklevel{#1}}

\def\@@shortsectionnumber#1%
  {\@EA\ifx\csname\??se#1\c!eigennummer\endcsname\relax
     \@EA\ifx\csname\??se#1\@@sectieblok\c!conversie\endcsname\relax
       \@EA\ifx\csname\??se#1\c!conversie\endcsname\relax
         \@@sectionvalue{#1}%
       \else
         \@@sectionconversion{#1}{\@@sectionvalue{#1}}%
       \fi
     \else
       \@@sectionconversion{#1}{\@@sectionvalue{#1}}%
     \fi
   \else
     \csname\??se#1\c!eigennummer\endcsname
   \fi}

\def\setsomeheadconversion#1#2%
   {\someheadconversionfalse
    \doifelsevalue{\??ko#1\c!eigennummer}\v!ja
      {\setgvalue{\??se\@@sectie\c!eigennummer}{#2}%
       \def\someheadconversion{#2}}
      {\letgvalue{\??se\@@sectie\c!eigennummer}\relax
       \bepaalkopnummer[#1]%
       \@EA\ifx\csname\??se\@@sectie\@@sectieblok\c!kopconversie\endcsname\relax
         \@EA\ifx\csname\??se\@@sectie\c!kopconversie\endcsname\relax
           \def\someheadconversion{#2}%
         \else
           \@EA\ifx\csname\??se\@@sectie\c!kopconversie\endcsname\empty
             \def\someheadconversion{#2}%
           \else
             \someheadconversiontrue
             \def\someheadconversion%
               {\fullsectionnumber{#1}{\getvalue{\??se\@@sectie\c!kopconversie}}{#2}}%
           \fi
         \fi
       \else
         \@EA\ifx\csname\??se\@@sectie\@@sectieblok\c!kopconversie\endcsname\empty
           \def\someheadconversion{#2}%
         \else
           \someheadconversiontrue
           \def\someheadconversion%
             {\fullsectionnumber{#1}{\getvalue{\??se\@@sectie\@@sectieblok\c!kopconversie}}{#2}}%
         \fi
       \fi}}

% pretty important (esp since we now ignore shipouts) 
%
% actually we should nil all writes, marks, specials 

\appendtoks \globallet\popcolor\relax \to \everylastshipout

\def\doscalelikeafigure % quite dirty and potential interference possible
  {\doifsomething{\@@xyfactor\@@xyhfactor\@@xybfactor\@@xyschaal
                  \@@xybreedte\@@xyhoogte\@@xyregels}
     {\let \@@efschaal \@@xyschaal
      \let \@@effactor \@@xyfactor
      \let \@@efbfactor\@@xybfactor
      \let \@@efhfactor\@@xyhfactor
      \let \@@efbreedte\@@xybreedte
      \let \@@efhoogte \@@xyhoogte
      \let \@@efregels \@@xyregels
      \let \@@epx      \!!zeropoint
      \let \@@epy      \!!zeropoint
      \edef\@@epw     {\the\wd\nextbox}%
      \edef\@@eph     {\the\ht\nextbox}%
      \figwid\zeropoint \figxsca\plusone % see note * (core-fig)
      \fighei\zeropoint \figysca\plusone % see note * (core-fig)
      \checkfiguresettings
      \setfactorfiguresize
      \setscalefiguresize
      \setdimensionfiguresize
      \convertfigureinsertscale\@@epx\figx\figxsca\scax
      \convertfigureinsertscale\@@epy\figy\figysca\scay
      \scratchdimen\scax\s!pt \divide\scratchdimen 100
      \edef\@@xysx{\withoutpt\the\scratchdimen}%
      \scratchdimen\scay\s!pt \divide\scratchdimen 100
      \edef\@@xysy{\withoutpt\the\scratchdimen}}}

\def\doschaal[#1]%
  {\bgroup
   \forgetall
   \getparameters
     [\??xy]
     [\c!schaal=,\c!breedte=,\c!hoogte=,\c!regels=,
      \c!factor=,\c!hfactor=,\c!bfactor=,
      \c!sx=1,\c!sy=1,#1]%
   \dowithnextbox
     {\dontshowcomposition
      \ifdim\ht\nextbox>\zeropoint \ifdim\wd\nextbox>\zeropoint
        \doscalelikeafigure
        \dimen0=\@@xysy\ht\nextbox
        \dimen2=\@@xysy\dp\nextbox
        \dimen4=\@@xysx\wd\nextbox
        \dimen6=\dimen0\advance\dimen6 \dimen2
%        \setbox\nextbox\vbox to \dimen6
%          {\ht\nextbox\zeropoint
%           \dp\nextbox\zeropoint
%           \vfill % erbij
%           \dostartscaling\@@xysx\@@xysy\box\nextbox\dostopscaling}%
        \setbox\nextbox\hbox
          {\smashbox\nextbox
           \dostartscaling\@@xysx\@@xysy\box\nextbox\dostopscaling}%
        \ht\nextbox\dimen0
        \dp\nextbox\dimen2
        \wd\nextbox\dimen4
      \fi \fi
      \box\nextbox
      \egroup}
   \hbox}

\def\incrementvalue#1%
  {\expandafter\increment\csname#1\endcsname}

\def\decrementvalue#1%
  {\expandafter\decrement\csname#1\endcsname}

% \translateMPinput{il2-pl}
%
% \startMPenvironment[global] 
%   \setupbodyfont[plr] 
% \stopMPenvironment 
%
% \TeX: � � 
%
% \startMPcode
% draw btex MetaPost: � � etex scaled 5 ;  
% \stopMPcode

\def\doMPpositiongraphic#1#2% tag setups
  {\bgroup
   \def\@@meta{#1:}%
   \setupMPvariables[#2]%
   \prepareMPpositionvariables
   \MPshiftdrawingtrue
   \def\doMPpositiongraphic##1##2%
     {{% new, see (techniek) 
       \def\@@meta{##1:}%
       \setupMPvariables[#2,##2]%
       \prepareMPpositionvariables
       % and needed 
       \getvalue{MPG:##1}}}% temp hack
   \setbox\positiongraphicbox\hbox
     {\ignorespaces 
      \executeifdefined{MPM:#1}{\executeifdefined{MPG:#1}\donothing}%
      \removelastspace}%
   \smashbox\positiongraphicbox
   \box\positiongraphicbox
   \egroup}

\writestatus{\m!systems}{beware: some patches loaded from cont-new.tex!}

\def\columnhbreak
  {\ifhmode
     \bgroup
     \removeunwantedspaces
     \parfillskip\zeropoint
     \OTRSETcolumnseparator
     \par
     \egroup
   \fi}

\installcolumnbreakhandler {SET} \v!lokaal
  {\columnhbreak
   \ejectinsert
   \ejectpage % brrr
   \OTRSETsynchronizehsize} % no \OTRSETsethsize, can be mid smaller (like tabulate)

% We need to make sure that we really leave the column; mid 
% column we may end up in an empty gap, and we don;t want to 
% stay there (basically such a gap is a small empty page 
% then).  

\installcolumnbreakhandler {SET} \v!ja
  {\columnhbreak
   \edef\savedmofcolumns{\the\mofcolumns}%
   \edef\savedrealpageno{\the\realpageno}%
   \ejectinsert
   \ejectpage % brrr
   \doloop
     {\ifnum\savedmofcolumns=\mofcolumns
        \ifnum\savedrealpageno=\realpageno
          \OTRSETdummycolumn
        \else
          \exitloop
        \fi 
      \else      
        \exitloop
      \fi}%
   \OTRSETsynchronizehsize} 

% testcase : pascal demo-bbi, paragraaf/aanduiding koppen 

\ifx\lastskipinotr\undefined \newskip\lastskipinotr \fi 

\installoutput\OTRSETflushpreposttext
  {\global\setbox\precolumnbox\vbox
     {\unvbox\normalpagebox
      \global\lastskipinotr\lastskip}%
   \ifdim\lastskipinotr>\zeropoint
     \global\setbox\precolumnbox\hbox
       {\lower\strutdepth\box\precolumnbox}%
   \fi 
   \global\dp\precolumnbox\strutdepth
   \ifcarryoverfootnotes \else
     \global\setbox\postcolumnbox\vbox{\placebottomnotes}%
   \fi}

\def\dopositionaction#1% test saves hash entry in etex
  {\ifundefined{\POSactionprefix#1::}\else
     \ifnum\MPp{#1}>\zerocount % new 
\setbox\scratchbox\hbox
       \bgroup
       \traceposstring\clap\red{<#1>}%
       \the\everyinsertpositionaction
       \the\everypositionaction
       \ifcollectMPpositiongraphics 
         % can save a lot of run time 
         \pushMPdrawing
         \MPshiftdrawingtrue
         \resetMPdrawing
         \getvalue{\POSactionprefix#1::}%
         \ifMPdrawingdone
           \getMPdrawing
         \fi
         \resetMPdrawing
         \popMPdrawing
       \else
         \getvalue{\POSactionprefix#1::}%
       \fi
       \cleanuppositionaction{#1}%
       \egroup
\smashedbox\scratchbox
     \else
       % shouldn't happen too often 
       \traceposstring\clap\cyan{<#1>}%
     \fi
   \fi}

\def\MPspacechar{\char32\relax} % old solution does not work with math 

\def\begintbl
  {\doglobal\newcounter\colTBL
   \doglobal\newcounter\rowTBL
   \doglobal\decrement\rowTBL
   \tabskip\zeropoint
   \halign\bgroup
\registerparoptions
   \ignorespaces##\unskip&&\ignorespaces##\unskip\cr}

% \startcolumnset[two] \input tufte
% \startcolumnsetspan[two] \input tufte \stopcolumnsetspan
% \input tufte \stopcolumnset

%D Well, here comes some real trickery. When we have the 100\%
%D spot color or black color, we don't want to erase the
%D background. So, instead we hide the content by giving it
%D zero transparency.

% todo : #1#2#3 met #2 > of < and #3 een threshold

% \newif\ifhidesplitcolor \hidesplitcolortrue

\def\dohidecolor#1#2%
  {\ifhidesplitcolor
     \ifx#1#2%
       \dostartgraycolormode\@@cl@@o
     \else
       \fullytransparentcolor
     \fi
   \else
     \dostartgraycolormode\@@cl@@o
   \fi}

\def\dovidecolor#1#2%
  {\ifhidesplitcolor
     \ifx#1#2%
       \fullytransparentcolor
     \else
       \dostartgraycolormode\@@cl@@o
     \fi
   \else
     \dostartgraycolormode\@@cl@@o
   \fi}

\def\fullytransparentcolor
  {\dostartgraycolormode\@@cl@@o % better than z
  %\global\@EA\chardef\csname\@@currenttransparent\endcsname\plusone
  %\global\intransparenttrue 
   \dostarttransparency10}

\def\noexeccolorS#1:#2\od
  {\edef\@@cl@@s{#1}%
   \dohidecolor\@@cl@@s\@@cl@@o}

\def\noexeccolorP#1:#2:#3\od
  {\edef\@@cl@@p{#2}%
   \dohidecolor\@@cl@@p\@@cl@@z}

\def\doexeccolorPP#1:#2:%
  {\edef\@@cl@@n{#1}%
   \edef\@@cl@@p{#2}%
   \registerusedspotcolor\@@cl@@n
   \ifx\@@cl@@n\currentspotcolor
     \normalizeSPOT
     \dostartgraycolormode\@@cl@@p % was spotcolormode
   \else
     \dovidecolor\@@cl@@p\@@cl@@o
   \fi
   \exectransparency}

% now in cont-loc.tex, for the sake of testing.
%
% %D When \type {\somecolor} is issued, we can savely assume
% %D grouping. Using \type {\groupedcommand} here (i.e.\ the
% %D definition of \type {\color}) is unsafe because in
% %D interferes with for instance switching attributes.
%
% \def\switchtocolor[#1]%
%   {\bgroup\startcolor[#1]
%    \aftergroup\stopcolor
%    \aftergroup\egroup}

% this supports:
%
% \starttypen
% \placelist[section][criterium=chapter,number=1] \blank
% \placelist[section][criterium=chapter,number=2] \blank
% \placelist[section][criterium=chapter,number=3] \blank
%
% \chapter{first}  \section{AA} \section{BB}
% \chapter{second} \section{CC} \section{DD}
% \chapter{third}  \section{EE} \section{FF}
% \stoptypen

\def\dosettoclevel#1#2%
  {\ifundefined{#1#2\c!nummer}%
     \dosetfilterlevel{\getvalue{#1#2\c!criterium}}\empty
   \else
     \doifelsevaluenothing{#1#2\c!nummer}%
       {\dosetfilterlevel{\getvalue{#1#2\c!criterium}}\empty}
       {\setsectieenkoppeling{\getvalue{#1#2\c!criterium}}%
        \dosetfilterlevel
          {\previoussection\@@sectie}%
          {\getvalue{#1#2\c!nummer}}}%
   \fi}

\def\GetPar
  {\expanded
     {\dowithpar
        {\the\BeforePar
         \BeforePar\emptytoks}
        {\the\AfterPar
         \BeforePar\emptytoks
         \AfterPar\emptytoks}}}

\def\GotoPar
  {\expanded
     {\dogotopar
        {\the\BeforePar
         \BeforePar\emptytoks}}}

\def\@@somedefinitie#1[#2]#3%
  {\dowithpar
     {\bgroup\executedoordefinitie{#1}[#2]{#3}}%
     {\@@stopdefinitie{#1}}}

% test this for a long time, esp since from now on, by default
% \commands are not expanded

\setupreferencing
  [\c!expansie=\v!nee]

\def\dotextreference[#1]#2%
  {\bgroup
   \def\asciia{#1}%
   \convertexpanded\??rf{#2}\asciib
   \@EA\rawtextreference\@EA\s!txt\@EA\asciia\@EA{\asciib}%
   \egroup}

\def\dopagereference[#1]%
  {\rawpagereference\s!pag{#1}}

\def\doreference[#1]#2%
  {\bgroup
   \def\asciia{#1}%
   \convertexpanded\??rf{#2}\asciib
   \@EA\rawreference\@EA\s!ref\@EA\asciia\@EA{\asciib}%
   \egroup}

\def\hyphenationpoint
  {\hskip\zeropoint}

\def\hyphenated#1%
  {\bgroup
   \!!counta\zerocount
   \def\hyphenated##1{\advance\!!counta\plusone}%
   \handletokens#1\with\hyphenated
   \!!countb\plusone
   \def\hyphenated##1%
     {##1%
      \advance\!!countb\plusone\relax
      \ifnum\!!countb>2 \ifnum\!!countb<\!!counta
        \hyphenationpoint
      \fi\fi}%
   \handletokens#1\with\hyphenated
   \egroup}

\def\obeysupersubletters
  {\let\super\normalsuper
   \let\suber\normalsuber
   \let\normalsuper\letterhat
   \let\normalsuber\letterunderscore
   \enablesupersub}

\def\obeysupersubmath
  {\let\normalsuper\letterhat
   \let\normalsuber\letterunderscore
   \enablesupersub}

%\let\normaltype\type
%
%\def\type#1%
%  {\expanded{\normaltype{\detokenize{#1}}}}

% {x123 \os x123} {\tfa x123 \os x123} {x123 \tx x123 \os x123}
% \definefontsynonym[OldStyle][Serif]
% {x123 \os x123} {\tfa x123 \os x123} {x123 \tx x123 \os x123}

% testen :
%
% \appendtoks
%   \let\registerparoptions\relax
% \to \everyforgetall

\newsignal\noblanksignal

\def\docomplexdoblanko[#1]% pas op \relax's zijn nodig ivm volgende \if
  {\global\blankoresetfalse
   \global\blankoblokkeerfalse
   \global\blankogeenwitfalse
   \global\lokaalblankoflexibelfalse
   \global\lokaalblankovastfalse
   \global\blankoskip\zeropoint
   \global\blankoforceerfalse
   \blankobuitenfalse
   \expanded{\rawprocesscommalist[#1]}\doblanko
   \ifdim\blankoskip=\zeropoint\relax
     \iflokaalblankoflexibel
       \doglobal\advance\blankoskip \currentblanko
     \else\iflokaalblankovast
       \doglobal\advance\blankoskip \currentblanko
     \fi\fi
   \fi
   \ifblankobuiten
   \else
     \par
     \ifvmode          %in pos fonts gaat dit mis
       \ifblankoforceer%\ifdim\prevdepth>\zeropoint\else
         % -1000pt signals top of page or column (\ejectcolumn)
         \vbox{\strut}\kern-\lineheight
       \fi
       \ifblankoblokkeer
         \global\doeblankofalse
         \ifgridsnapping
           \ifdim\prevdepth<\zeropoint
             % brrr
           \else
             % dirty trick: smaller blanks are ignored after
             % a larger one, so 10 lines is probably safe; first make
             % sure that we honor penalties
             \scratchcounter\lastpenalty
             % now comes the trick (cross our fingers that this works
             % well in multi columns; maybe an ifinner test is needed
             % \vskip-10\lineheight
             %    \ifnum\scratchcounter=\zerocount \else \penalty\lastpenalty \fi
             %    \vskip 10\lineheight
             % allas, this leads to overfull pages, so we try this:
             \kern-\noblanksignal
             \ifnum\scratchcounter=\zerocount
             \else
               \penalty\lastpenalty
             \fi
             \kern\noblanksignal
             % end-of-dirty-trick
           \fi
         \else
           \ifdim\prevdepth<\zeropoint
             % brrr
           \else
             % ensure at least a proper prevdepth, this should be
             % an option
             \vskip-\prevdepth
             \vskip\strutdepth
             \prevdepth\strutdepth
           \fi
           % the old crappy piece of code
           \edef\oldprevdepth{\the\prevdepth}%
           \prevdepth\newprevdepth
         \fi
       \else
         \global\doeblankotrue
       \fi
       \ifblankoreset
         \global\doeblankotrue
         \ifgridsnapping
            % let's play safe and not fool around with the depth, if
            % only because it took a lot of effort to sort out the grid
            % stuff in the first place
         \else
           \ifdim\prevdepth=\newprevdepth
             \prevdepth\oldprevdepth
           \fi
         \fi
       \fi
       \ifdoeblanko
         \ifdim1\lastskip<1\blankoskip\relax
           % else when \blanko[2*groot] + \blanko[3*groot] with parskip
           % equaling 1*groot, gives a groot=\parskip so adding a small
           % value makes it distinguishable; can also be done at parskip
           % setting time (better)
           \global\advance\blankoskip \mindimen\relax % = skip
           % test this on 2* + 3* and parskip groot
           \ifblankogeenwit
             \global\advance\blankoskip -\parskip
           \else
             \ifdim\lastskip=\parskip
             \else  % force this due to previous comment
               \ifdim\parskip>\zeropoint\relax
                 \ifdim\blankoskip<\parskip\relax
                   \global\blankoskip\zeropoint
                 \else
                   \global\advance\blankoskip -\parskip
                 \fi
               \fi
             \fi
           \fi
           \ifblankoflexibel \else
             \blankoskip1\blankoskip
           \fi
           \iflokaalblankovast
             \blankoskip1\blankoskip
           \fi
           \iflokaalblankoflexibel
             \blankoskip1\blankoskip
               \!!plus\skipgluefactor\blankoskip
               \!!minus\skipgluefactor\blankoskip
           \fi
           \ifgridsnapping
             \ifdim\lastkern=\noblanksignal
               \global\doeblankofalse
             \fi
           \else
             \ifdim\prevdepth=\newprevdepth
               \global\doeblankofalse
             \fi
           \fi
           \ifdoeblanko
             \iffuzzyvskip
               \removelastfuzzyvskip
               \fuzzyvskip\blankoskip\relax
             \else
               \removelastskip
               \vskip\blankoskip\relax
             \fi
           \fi
         \else
           \iffuzzyvskip
             \removelastfuzzyvskip
             \fuzzyvskip\blankoskip\relax
           \else
             % new, test this on pascal
             \ifdim\blankoskip<\zeropoint
               \advance\blankoskip-\lastskip
               \removelastskip
               \ifdim\blankoskip>\zeropoint
                 \vskip\blankoskip
               \else
                 \vskip\zeropoint
               \fi
             \else
               % also new 
               \ifdim\blankoskip=\zeropoint
                 \ifblankogeenwit
                   \geenwitruimte 
                 \fi
               \fi
             \fi
           \fi
         \fi
       \fi
     \fi
   \fi
   \global\fuzzyvskipfalse
   \presetindentation}

\def\processfilelinesverbatim#1#2#3%
  {\bgroup
   \let\saveddoflushverbatimline\doflushverbatimline
   \let\saveddoemptyverbatimline\doemptyverbatimline
   \def\checkverbatimfileline##1%
     {\ifnum\verbatimlinenumber<#2\else
      \ifnum\verbatimlinenumber>#3\else
        ##1%
      \fi\fi}%
   \def\doflushverbatimline
     {\checkverbatimfileline\saveddoflushverbatimline}%
   \def\doemptyverbatimline
     {\checkverbatimfileline\saveddoemptyverbatimline}%
   \processfileverbatim{#1}%
   \egroup}

\def\typefile
  {\dodoubleempty\dotypefile}

\def\dotypefile[#1][#2]#3%
  {\ifsecondargument
     \dodotypefile[#1][#2]{#3}%
   \else\iffirstargument
     \doifassignmentelse{#1}
       {\dodotypefile[\v!file][#1]{#3}}
       {\dodotypefile[#1][]{#3}}%
   \else
     \dodotypefile[\v!file][]{#3}%
   \fi\fi}

\def\dosetuptypelinenumbering#1#2%
  {\setuptyping[#1][\c!start=,\c!stop=,\c!stap=,\c!nregels=,#2]%
   \doifelsevalue{\??tp#1\c!nummeren}\v!file
     {\stelregelnummerenin[\c!methode=\v!file]%
      \donetrue}
     {\doifelsevalue{\??tp#1\c!nummeren}\v!regel
        {% \stelregelnummerenin defaults start/step to 1/1, so we need
         \doifvaluenothing{\??tp#1\c!start}{\setvalue{\??tp#1\c!start}{1}}%
         \doifvaluenothing{\??tp#1\c!stap }{\setvalue{\??tp#1\c!stap }{1}}%
         \stelregelnummerenin
           [\c!methode=\v!type,
            \c!start=\getvalue{\??tp#1\c!start},
            \c!stap=\getvalue{\??tp#1\c!stap}]%
         \donetrue}
        {\donefalse}}%
   \ifdone
     \def\beginofverbatimlines{\startregelnummeren}%
     \def\endofverbatimlines  {\stopregelnummeren }%
   \fi}

\def\dodotypefile[#1][#2]#3%
  {\getvalue{\??tp#1\c!voor}%
   \doiflocfileelse{#3}
     {\startopelkaar % includes \bgroup
      \dosetuptypelinenumbering{#1}{#2}%
      \doifinset{\getvalue{\??tp#1\c!optie}}{\v!commandos,\v!schuin,\v!normaal}
        {\setuptyping[#1][\c!optie=\v!geen]}%
      \doifvalue{\??tp#1\c!optie}\v!kleur
        {\expandafter\aftersplitstring#3\at.\to\prettyidentifier
         \letvalue{\??tp#1\c!optie}\prettyidentifier}%
      \initializetyping{#1}%
      \startverbatimcolor
      \makelocreadfilename{#3}%
      \doifundefinedelse{\??tp#3\v!globaal\c!start}
        {\scratchcounter\zerocount}
        {\scratchcounter\getvalue{\??tp#3\v!globaal\c!start}}%
      \advance\scratchcounter\plusone
      \setxvalue{\??tp#3\v!globaal\c!start}{\the\scratchcounter}%
      \doifelsevaluenothing{\??tp#1\c!start}
        {\processfileverbatim\readfilename}
        {\doifvalue{\??tp#1\c!start}\v!verder
           {\setevalue{\??tp#1\c!start}%
              {\getvalue{\??tp#3\v!globaal\c!start}}}%
         \doifelsevaluenothing{\??tp#1\c!stop}
           {\doifelsevaluenothing{\??tp#1\c!nregels}
              {\processfileverbatim\readfilename}
              {\scratchcounter\getvalue{\??tp#1\c!start}%
               \advance\scratchcounter\getvalue{\??tp#1\c!nregels}%
               \advance\scratchcounter\minusone
               \setxvalue{\??tp#3\v!globaal\c!start}%
                 {\the\scratchcounter}%
               \processfilelinesverbatim\readfilename
                 {\getvalue{\??tp#1\c!start}}
                 {\getvalue{\??tp#3\v!globaal\c!start}}}}%
           {\processfilelinesverbatim\readfilename
              {\getvalue{\??tp#1\c!start}}
              {\getvalue{\??tp#1\c!stop }}}}%
      \stopverbatimcolor
      \stopopelkaar}  % includes \egroup
     {\bgroup
      \expanded{\convertargument#3}\to\ascii
      \tttf[\makemessage\m!verbatims1\ascii]%
      \showmessage\m!verbatims1\ascii
      \egroup}%
   \getvalue{\??tp#1\c!na}}

% \setuptyping[file][numbering=file]
%
% \typefile[start=2,nlines=3]{zapf}
% \typefile[start=continue,nlines=13]{zapf}
% \typefile{zapf}
%
% \setuptyping[file][numbering=line]
%
% \typefile[start=4,step=3]{zapf}
% \typefile{zapf}

\def\startgridcorrection
  {\dosingleempty\dostartgridcorrection}

\def\dostartgridcorrection[#1]%
  {\ifgridsnapping
     \iffirstargument\doifsomething{#1}{\verplaatsopgrid[#1]}\fi
     \snaptogrid\vbox\bgroup
   \else
     \startbaselinecorrection
   \fi}

\def\stopgridcorrection
  {\ifgridsnapping
     \egroup
   \else
     \stopbaselinecorrection
   \fi}

\def\checkgridsnapping
  {\lineskip\ifgridsnapping\zeropoint\else\normallineskip\fi}

\def\splittblbox#1% #1 <> 0/2
  {\ifinsidefloat
     \unvbox#1%
   \else
     % spacing between rows gets lost in split
     \setbox4\vbox
       {\doifsomething\tbltbltussenwit{\blank[\tbltbltussenwit]}}%
     \setbox2\vbox
       {}%
     \dorecurse\noftblheadlines
       {\setbox0\vsplit#1 to \lineheight
        \setbox2\vbox{\unvcopy2\unvcopy0}}%
     \ifcase\noftblheadlines\else\unvcopy2\fi
     \donefalse
     \doloop
       {\setbox0\vsplit#1 to \lineheight
        \ifdim\pagegoal<\maxdimen
          \setbox0\vbox{\unvbox0}%
          \dimen0\ht0
          \advance\dimen0\ht4
          \advance\dimen0\pagetotal
          \ifdim\dimen0>\pagegoal
            \bgroup \pagina \egroup % make sure that local vars are kept
            \ifcase\noftblheadlines\else\unvcopy2\fi
          \fi
        \fi
        \ifdone
          \doifsomething\tbltbltussenwit{\blank[\tbltbltussenwit]}%
        \fi
        \unvbox0
        \allowbreak
        \ifvoid#1 \exitloop \else \donetrue \fi}%
   \fi}

\def\startplaatsen
  {\dosingleempty\dostartplaatsen}

\def\dostartplaatsen[#1]% tzt n*links etc
  {\endgraf
   \noindent\bgroup
   \setlocalhsize
   \hbox to \localhsize\bgroup
     \doifnot{#1}\v!links\hss
     \def\stopplaatsen
       {\unskip\unskip\unskip
        \doifnot{#1}\v!rechts\hss
        \egroup
        \egroup
        \endgraf}%
     \gobblespacetokens}

% \startplaatsen[links] bla \stopplaatsen

% we don't register the paragraph characteristics, only the
% width

\appendtoks
  \setinnerparpositions % see "techniek" for application
\to \everytabulate

\appendtoks \checkcurrentlayout \to \everystarttext

\def\flushfootnotes  {\flushnotes}
\def\doflushfootnotes{\doflushnotes}

%D This alternative is slower, since it works on top of the
%D color (stack) mechanism, but it does provide nesting.

\def\dosetrastercolor#1%
  {\edef\@@cl@@s{#1}%
   \ifx\@@cl@@s\empty
     \let\@@cl@@s\@@rsraster
   \fi
   \setevalue{\??cr\??rs}{\colorSpattern}}

% beware, don't add extra grouping, else color in tables
% fails

\def\localstartraster[#1]%
  {\ifincolor\dosetrastercolor{#1}\localstartcolor[\??rs]\fi}

\def\startraster[#1]%
  {\ifincolor\dosetrastercolor{#1}\startcolor[\??rs]\fi}

\def\localstopraster{\ifincolor\localstopcolor\fi}
\def\stopraster     {\ifincolor\stopcolor\fi}

\def\fontclassname#1#2%
  {\ifcsname\??ff#1#2\endcsname
     \fontclassname{#1}{\csname\??ff#1#2\endcsname}%
   \else\ifcsname\??ff#2\endcsname
     \fontclassname{#1}{\csname\??ff#2\endcsname}%
   \else
     #2%
   \fi\fi}

\def\defineclassfontsynonym
  {\dotripleargument\dodefineclassfontsynonym}

\def\dodefineclassfontsynonym[#1][#2][#3]%
  {\definefontsynonym[#1][\fontclassname{#2}{#3}]}

%\definefontsynonym [KopFont] [\fontclassname{officina}{SerifBold}]
%
%\defineclassfontsynonym [KopFont] [officina] [SerifBold]

\def\woordrechts % zie naw
  {\groupedcommand
     {\removeunwantedspaces
      \hfill
      \hskip\zeropoint % permit break, \allowbreak fails here
      \strut
      \hfill
      \quad % decent spacing
      \hbox}
     {\parfillskip\zeropoint
      \par}}

\def\startkolomopmaak % don't change
  {\bgroup
   \getrawnoflines\teksthoogte % teksthoogte kan topskip hebben, dus raw
   \scratchdimen\noflines\lineheight
   \advance\scratchdimen-\lineheight
   \advance\scratchdimen\topskip
   \setbox\scratchbox
   \ifcase\showgridstate\vbox\else\ruledvbox\fi to \scratchdimen\bgroup}

\def\stopkolomopmaak
  {\egroup
   \dp\scratchbox\zeropoint
   \wd\scratchbox\tekstbreedte
   \box\scratchbox
   \egroup}

% todo : hoe komt box er uit

\long\def\startexternalfigure
  {\dotripleempty\dostartexternalfigure}

\long\def\dostartexternalfigure[#1][#2][#3]#4\stopexternalfigure
  {\gdef\figuredescription{#4}%
   \externalfigure[#1][#2][#3]%
   \globallet\figuredescription\empty}

\let\figuredescription\empty

% very experimental

\def\redoanalyzefigurefiles#1%
  {\ifcase\figurestatus
     \def\@@efcurrenttype{#1}%
     \dododoanalyzefigurefiles\empty
   \fi}

\def\analyzefigurefiles
  {\let\externalfigurelog\empty
   \let\@@efcurrenttype\empty
   \let\@@efcurrentpath\empty
   \let\@@efcurrentfile\empty
   \doanalyzefigurefiles\doanalyzefigurefilesA
   \doanalyzefigurefiles\doanalyzefigurefilesB
   \doanalyzefigurefiles\doanalyzefigurefilesC
   % new, permits rather raw names like e:/....
   \let\dodoanalyzefigurefiles\redoanalyzefigurefiles
   \doanalyzefigurefiles\doanalyzefigurefilesA
   \doanalyzefigurefiles\doanalyzefigurefilesB
   \doanalyzefigurefiles\doanalyzefigurefilesC}

\def\phantombox[#1]%
  {\hbox\bgroup
   \getparameters
     [\??ol]
     [\c!breedte=\zeropoint,\c!hoogte=\zeropoint,\c!diepte=\zeropoint,#1]%
   \setbox\scratchbox\null
   \wd\scratchbox\@@olbreedte
   \ht\scratchbox\@@olhoogte
   \dp\scratchbox\@@oldiepte
   \box\scratchbox
   \egroup}

\long\@EA\def\csname\e!start\e!instellingen\endcsname#1 %
  {\bgroup
   \catcode`\^^M=\@@ignore
   \xdostartsetups{#1}}

\expanded
  {\long\noexpand\def\noexpand\xdostartsetups##1##2\csname\e!stop\e!instellingen\endcsname%
     {\egroup
      \long\noexpand\setvalue{\??su##1}{##2}}}

%\def\startsetups % for international purposes
%  {\bgroup
%   \doifnextcharelse[\startsetupsA\startsetupsB}
%
%\def\startsetupsA[#1]%
%  {\catcode`\^^M=\@@ignore
%   \dostartsetups{#1}}
%
%\def\startsetupsB#1 % space delimited
%  {\catcode`\^^M=\@@ignore
%   \dostartsetups{#1}}
%
%\long\def\dostartsetups#1#2\stopsetups
%  {\egroup
%   \long\setvalue{\??su#1}{#2}}

\def\startsetups % for international purposes
  {\bgroup\doifnextcharelse[{\startsetupsA\stopsetups}%
                            {\startsetupsB\stopsetups}}
 
\def\startlocalsetups % for nested purposes
  {\bgroup\doifnextcharelse[{\startsetupsA\stoplocalsetups}%
                            {\startsetupsB\stoplocalsetups}}

\def\startsetupsA#1[#2]%
  {\catcode`\^^M=\@@ignore
   \dostartsetups#1{#2}}

\def\startsetupsB#1#2 % space delimited
  {\startsetupsA#1[#2]}%

\long\def\dostartsetups#1#2% watch out: not \grabuntil 
  {\dograbuntil#1{\egroup\long\setvalue{\??su#2}}}

\newtoks\everyfirstparagraphintro
\newtoks\everynextparagraphintro

\chardef\everyparagraphintro=0

\def\setupparagraphintro
  {\dodoubleempty\dosetupparagraphintro}

\def\dosetupparagraphintro[#1][#2]%
  {\processallactionsinset
     [#1]
     [   \v!reset=>\global\chardef\everyparagraphintro0
                   \global\everyfirstparagraphintro\emptytoks
                   \global\everynextparagraphintro \emptytoks,
        \v!eerste=>\global\chardef\everyparagraphintro1
                   \doglobal\appendtoks#2\to\everyfirstparagraphintro,
      \v!volgende=>\ifcase\everyparagraphintro\global\chardef\everyparagraphintro=2\fi
                   \doglobal\appendtoks#2\to\everynextparagraphintro,
           \v!elk=>\ifcase\everyparagraphintro\global\chardef\everyparagraphintro=2\fi
                   \doglobal\appendtoks#2\to\everyfirstparagraphintro
                   \doglobal\appendtoks#2\to\everynextparagraphintro]}

\def\doinsertparagraphintro
  {\ifcase\everyparagraphintro\relax
     % no data
   \or
     % first data
     \global\chardef\everyparagraphintro2
     \scratchtoks\everyfirstparagraphintro
     \global\everyfirstparagraphintro\emptytoks
   \or
     % next data
     \scratchtoks\everynextparagraphintro
   \fi
   \the\scratchtoks}

\def\insertparagraphintro
  {\ifcase\everyparagraphintro\else\@EA\doinsertparagraphintro\fi}

\appendtoks\insertparagraphintro\to\everypar

%D \starttypen
%D \setupparagraphintro[first][\hbox to 3.5em{\tt FIRST \hss}]
%D \setupparagraphintro[first][\hbox to 3.5em{\tt TSRIF \hss}]
%D \setupparagraphintro[next] [\hbox to 3.5em{\tt NEXT  \hss}]
%D \setupparagraphintro[next] [\hbox to 3.5em{\tt TXEN  \hss}]
%D \setupparagraphintro[each] [\hbox to 3.0em{\tt EACH  \hss}]
%D \setupparagraphintro[each] [\hbox to 3.0em{\tt HCEA  \hss}]
%D
%D some paragraph \par
%D some paragraph \par
%D some paragraph \par
%D
%D \definelabel[parnumber]
%D
%D \setupparagraphintro[reset,each][\inleft{\slxx\parnumber}]
%D
%D some paragraph \par
%D some paragraph \par
%D some paragraph \par
%D \stoptypen

% wrong names

\newif\ifpagechanged \let\lastchangedpage\empty

\def\checkpagechange#1%
  {\gettwopassdata\s!paragraph
   \pagechangedfalse
   \iftwopassdatafound
     \ifnum\twopassdata>0\getvalue{\s!paragraph:p:#1}\relax
       \pagechangedtrue
     \fi
   \fi
   \ifpagechanged
     \letgvalue{\s!paragraph:p:#1}\twopassdata
     \globallet\lastchangedpage\twopassdata
   \else
     \globallet\lastchangedpage\realfolio
   \fi
   \doparagraphreference}

\def\changedpage#1%
  {\getvalue{\s!paragraph:p:#1}}

\def\startfixed{\dosingleempty\dostartfixed}

\long\def\dostartfixed[#1]%
  {\expanded{\dowithnextbox{\noexpand\dodofixed{\ifhmode0\else1\fi}{#1}}}%
   \vbox\bgroup
   \setlocalhsize}

\def\stopfixed%
  {\egroup}

\def\dodofixed#1#2%
  {\ifcase#1\relax
     \processaction
       [#2]
       [   \v!hoog=>\bbox   {\flushnextbox},
           \v!laag=>\tbox   {\flushnextbox},
         \v!midden=>\vcenter{\flushnextbox},
           \v!laho=>\vcenter{\flushnextbox},
        \s!unknown=>\tbox   {\flushnextbox},
        \s!default=>\tbox   {\flushnextbox}]%
   \else
     \startbaselinecorrection
     \noindent\flushnextbox
     \stopbaselinecorrection
   \fi}

% \startitemize
%
% \item \externalfigure[koe][height=2cm]
% \item \externalfigure[koe][height=2cm]
% \item \externalfigure[koe][height=2cm]
% \item \externalfigure[koe][height=2cm]
%
% \page
%
% \item \startfixed      \externalfigure[koe][height=2cm]\stopfixed
% \item \startfixed[high]\externalfigure[koe][height=2cm]\stopfixed
% \item \startfixed[low] \externalfigure[koe][height=2cm]\stopfixed
% \item \startfixed[lohi]\externalfigure[koe][height=2cm]\stopfixed
%
% \page
%
% \item test \startfixed      \externalfigure[koe][height=2cm]\stopfixed
% \item test \startfixed[high]\externalfigure[koe][height=2cm]\stopfixed
% \item test \startfixed[low] \externalfigure[koe][height=2cm]\stopfixed
% \item test \startfixed[lohi]\externalfigure[koe][height=2cm]\stopfixed
%
% \page
%
% \item test \par \startfixed      \externalfigure[koe][height=2cm]\stopfixed
% \item test \par \startfixed[high]\externalfigure[koe][height=2cm]\stopfixed
% \item test \par \startfixed[low] \externalfigure[koe][height=2cm]\stopfixed
% \item test \par \startfixed[lohi]\externalfigure[koe][height=2cm]\stopfixed
%
% \stopitemize

% \def\docalculatefigurenorm#1#2%
%   {\dodocalculatefigurenorm{#1}[#2\empty\empty]}
%
% \def\dodocalculatefigurenorm#1[#2#3#4]#5#6#7%
%   {\ExpandFirstAfter\processaction
%       [#2#3#4]
%       [     \v!max=>\global#1=#6\relax,
%           \v!kolom=>\global#1=#6\relax,
%           \v!tekst=>\global#1=#6\relax,
%         \v!passend=>\global#1=#7\relax,
%            \v!ruim=>\global#1=#7\relax
%                     \global\advance #1 -4\@@exkorps\relax,
%        #2*\v!kolom=>\global#1=#6\relax
%                     \ifbinnenkolommen
%                       \global\advance#1 \intercolumnwidth
%                       \global\multiply#1 #2\relax
%                       \global\advance#1 -\intercolumnwidth
%                     \fi,
%        #2*\v!tekst=>\global#1=\zetbreedte
%                     \global\advance#1 \papierbreedte,
%         \s!default=>\doifsomething{#5}{\global#1=#5\relax},
%         \s!unknown=>\global#1=\@@exkorps\relax
%                     \global\divide#1 \!!ten\relax
%                     \global\multiply#1 #2#3#4\relax]}

\def\complexTableTB[#1]{\TABLEnoalign{\blanko[#1]}}
\def\simpleTableTB     {\TABLEnoalign{\blanko}}

\def\TabulateTB
  {\complexorsimpleTable{TB}}

\def\doTableinterline% #1
  {\ifnum\currentTABLEcolumn>\maxTABLEcolumn
     \chuckTABLEautorow
   \else\ifnum\currentTABLEcolumn=\zerocount
     \TABLEnoalign
       {\globalletempty\checkTABLEautorow
        \globalletempty\chuckTABLEautorow}%
   \else
     \setTABLEerror\TABLEmissingcolumn
     \handleTABLEerror
   \fi\fi
   \complexorsimpleTable} % {#1}

\def\TableHL{\doTableinterline{HL}}
\def\TableTB{\doTableinterline{TB}}

\appendtoks\let\TB\TableTB   \to\everytable
\appendtoks\let\TB\TabulateTB\to\everytabulate

% \starttabulate
% \NC text \NC text \NC \NR
% \TB[small]
% \NC text \NC text \NC \NR
% \TB[4*big]
% \NC text \NC text \NC \NR
% \stoptabulate
%
% \starttable[|||]
% \VL text \VL text \VL \AR
% \TB[small]
% \VL text \VL text \VL \AR
% \TB[4*big]
% \VL text \VL text \VL \AR
% \stoptable

% still needed for uguide

\let\placefloatlabel          \placefloatcaption
\let\placefloatlabeltext      \placefloatcaptiontext
\let\placefloatlabelreference \placefloatcaptionreference

\def\obeyfollowingtoken{{}}  % end \cs scanning

\def\gobbleparameters{\doquadrupleempty\dogobbleparameters}
\def\dogobbleparameters[#1][#2][#3][#4]{}

% \setvariables[xx][titel=]
% \setvariables[xx][titel=test test]
% \setvariables[xx][titel=test $x=1$ test]   % fatal error reported
% \setvariables[xx][titel=test {$x=1$} test]
% \setvariables[xx][titel]                   % fatal error reported
% \setvariables[xx][titel=e]

\def\??vars{@@vars}

\def\setvariables
  {\dotripleargument\dosetvariables[\getrawparameters]}

\def\globalsetvariables
  {\dotripleargument\dosetvariables[\globalgetrawparameters]}

\def\dosetvariables[#1][#2][#3]%
  {\errorisfataltrue
   \def\currentvariableclass{#2}%
   #1[\??vars:#2:][#3]%
   \errorisfatalfalse}

\beginTEX

\def\getvariable#1#2% to be sped up
  {\csname
     \ifundefined{\??vars:#1:#2}\s!empty\else\??vars:#1:#2\fi
   \endcsname}

\endTEX

\beginETEX \ifcsname

\def\getvariable#1#2% to be sped up
  {\csname
     \ifcsname\??vars:#1:#2\endcsname\??vars:#1:#2\else\s!empty\fi
   \endcsname}

\endETEX

\def\showvariable#1#2%
  {\showvalue{\ifundefined{\??vars:#1:#2}\s!empty\else\??vars:#1:#2\fi}}

\let\currentvariableclass\empty

% Let's see how fast Mr Bigfoot aka GB tracks down this new
% feature -)

\def\defineTABLEdivisions
  {\global\TABLEdivisionfalse % in start
   \let\DL\TableDL
   \let\DC\TableDC
   \let\DV\TableDV
   \let\DR\TableDR}

\def\defineTABLErules
  {\let\VL\TableVL
   \let\VC\TableVC
   \let\HL\TableHL
   \let\HC\TableHC
   \let\VS\TableVS
   \let\VD\TableVD
   \let\VT\TableVT}

\def\TableVS{\gdef\@VLn{1}\VL}
\def\TableVD{\gdef\@VLn{2}\VL}
\def\TableVT{\gdef\@VLn{3}\VL}

\def\@VLn{1}
\def\@VLd{.125em}

\def\do!ttInsertVrule % will be merged in 2005
  {\vrule \!thWidth
   \ifnum\!tgCode=1
     \ifx\!tgValue\empty
       \LineThicknessFactor
     \else
       \!tgValue
     \fi
     \LineThicknessUnit
   \else
     \!tgValue
   \fi
   \hskip\@VLd}

\def\!ttInsertVrule%
  {\hfil
   \TABLEbeforebar % added
   \startglobalTABLEcolor % added
   % we could do without this speedup, some day merge 'm
   \ifcase\@VLn\or
     \do!ttInsertVrule
     \unskip
   \else
     \dorecurse\@VLn\do!ttInsertVrule
     \gdef\@VLn{1}%
     \unskip
   \fi
   \stopglobalTABLEcolor % added
   \TABLEafterbar % added
   \hfil
   &}

% \starttable[|||]
% \HL
% \VL test \VS test \VL \FR
% \VL test \VD test \VL \MR
% \VL test \VT test \VL \LR
% \HL
% \stoptable

%D To be documented, \type {\includemenu[menu]}.
%D To be documented, \type {\emphbf} cum suis.

%D A prelude to strategies. Note for myself: overloads
%D previous stuff from local pragma test files.

\def\s!strategy{strategy}

\def\currentstrategypass    {1}
\def\currentstrategyvariable{0}
\def\maximumstrategypass    {8}

\newconditional\strategypassneeded
\newconditional\strategypassforced

\definetwopasslist{\s!strategy}

\def\registerstrategypass%
  {\ifnum\currentstrategypass>\maximumstrategypass \else
     \ifconditional\strategypassforced
       \doglobal\increment\currentstrategypass
     \else%\ifconditional\strategypassneeded
       %\doglobal\increment\currentstrategypass
     \fi%\fi
   \fi
   \savecurrentvalue\currentstrategypass{\currentstrategypass}}

\appendtoks \registerstrategypass \to \everybye % \everylastshipout

\def\setstrategyvariable#1#2% key value
  {%\doifnotstrategyvariable{#1}{\global\settrue\strategypassneeded}%
   \doglobal\increment\currentstrategyvariable
   \expanded{\immediatewriteutilitycommand{\noexpand
     \twopassentry{\s!strategy}{\currentstrategyvariable}{#1::#2}}}}

\def\doifstrategyvariableelse#1#2#3%
  {\getstrategyvariable{#1}\iftwopassdatafound#2\else#3\fi}

\def\getstrategyvariable#1% key
  {\findtwopassdata{\s!strategy}{#1::}%
   \setxvalue{\s!strategy:#1}{\twopassdata}}

\def\retainstrategyvariable#1% key
  {\expanded{\setstrategyvariable{#1}{\strategyvariable{#1}}}}

\def\strategyvariable#1% key
  {\csname\s!strategy:#1\endcsname}

\let\stratvar\strategyvariable

\def\forcestrategy{\global\settrue \strategypassforced}
\def\abortstrategy{\global\setfalse\strategypassforced}

\def\doifstrategyvariableelse#1#2#3%
  {\getstrategyvariable{#1}\iftwopassdatafound#2\else#3\fi}

\def\doifstrategyvariable   #1#2{\doifstrategyvariableelse{#1}{#2}{}}
\def\doifnotstrategyvariable#1#2{\doifstrategyvariableelse{#1}{}{#2}}

%D New: only at start of columns; may change ! Rather
%D interwoven and therefore to be integrated when the multi
%D column modules are merged.

%  already taken care of: \definesystemvariable{ks}

% is buggy now and does not work any longer

\def\setupcolumnspan[#1]%
  {\getparameters[\??ks][#1]}

\presetlocalframed
  [\??ks]

\setupcolumnspan
  [\c!n=2,
   \c!offset=\v!overlay,
   \c!kader=\v!uit]

\newbox\columnspanbox \let\postprocesscolumnspanbox\gobbleoneargument

\def\dostartcolumnspan[#1]%
  {\bgroup
   \setupcolumnspan[#1]%
   \forgetall
   \ifbinnenkolommen
     \advance\hsize \intercolumnwidth
     \hsize\@@ksn\hsize
     \advance\hsize -\intercolumnwidth
   \fi
   \dowithnextbox
     {\setbox\columnspanbox\flushnextbox
      \ifbinnenkolommen\wd\columnspanbox\hsize\fi
      \postprocesscolumnspanbox\columnspanbox
      \scratchdimen\ht\columnspanbox
      \setbox\columnspanbox\hbox % depth to be checked, probably option!
        {\localframed[\??ks][\c!offset=\v!overlay]{\box\columnspanbox}}%
      \ht\columnspanbox\scratchdimen
      \dp\columnspanbox\dp\strutbox
      \wd\columnspanbox\hsize
      \ifbinnenkolommen
        \ifnum\@@ksn>1
          \setvsize
          \dohandleallcolumns
            {\ifnum\currentcolumn>\@@ksn\else
               \global\setbox\currenttopcolumnbox=\vbox
                 {\ifnum\currentcolumn=1
                    \snaptogrid\vbox{\copy\columnspanbox}
                  \else
                    \snaptogrid\vbox{\vphantom{\copy\columnspanbox}}
                  \fi}%
               \wd\currenttopcolumnbox\hsize
               \global\advance\vsize -\ht\currenttopcolumnbox
             \fi}
          \global\pagegoal\vsize
        \else
          \snaptogrid\vbox{\box\columnspanbox}
        \fi
      \else
        \snaptogrid\vbox{\box\columnspanbox}
      \fi
      \endgraf
      \prevdepth\dp\strutbox
      \egroup}
     \vbox\bgroup
      %\topskipcorrection % becomes an option !
       \EveryPar{\begstrut\EveryPar{}}} % also !

\def\startcolumnspan%
  {\dosingleempty\dostartcolumnspan}

\def\stopcolumnspan%
  {\egroup}

\def\backgroundline[#1]%
 %{\doifsomething{#1}{\dobackgroundline{#1}}\hbox}
  {\doifcolorelse{#1}{\dobackgroundline{#1}\hbox}\hbox}

\def\dobackgroundline#1%
  {\dowithnextbox
     {\hbox
        {\localcolortrue
         \startcolor[#1]%
         \vrule
           \!!width \wd\nextbox
           \!!height\ht\nextbox
           \!!depth \dp\nextbox
       \stopcolor
       \hskip-\wd\nextbox
       \box\nextbox}}}

%D For Ton. To be documented.

\def\plaatsexterndocument[#1]%
  {\def\doexternaldocument[##1][##2][##3]%
     {\readlocfile{##2}\donothing\donothing}%
   \getvalue{\v!file:::#1}}

%D Far from complete.

\def\startgeheel
  {\startregelcorrectie
   \insidefloattrue}

\def\stopgeheel
  {\stopregelcorrectie}

%D No more news.

\protect

%D A few local optimizations and new features, if defined:

\readfile {cont-loc} {} {}

\endinput
