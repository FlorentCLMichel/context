% This is an old version, I still have to patch the latest 
% changes into this file. 

%D \module
%D   [       file=s-map-01,
%D        version=1998.05.05,
%D          title=\CONTEXT\ Style File,
%D       subtitle=\MAPS\ basis stijl,
%D         author={Taco Hoekwater, Siep Kroonenberg \& Hans Hagen},
%D           date=\currentdate,
%D      copyright={NTG / MAPS}]
%C
%C This module is part of the \CONTEXT\ macro||package and is
%C therefore copyrighted by \PRAGMA. See mreadme.pdf for 
%C details. 

%D The \MAPS\ layout is designed by Taco Hoekwater and Siep 
%D Kroonenberg, who on behalf of the \MAPS/\NTG\ own the 
%D copyright to the design. 

%D In deze file wordt zowel de dubbelzijdig als de enkelzijdige
%D layout van de \MAPS\ gedefinieerd. Naast deze file
%D zijn|/|komen wat aanvullende stijlen voor specifieke
%D doeleinden beschikbaar.

%D Normaal gesproken wordt met \type{\stelinterliniein} de
%D interlinie voor het gehele document ingesteld op een
%D zodanige wijze dat deze voor elk korps vergelijkbaar is. In
%D de \MAPS\ gebruiken we echter per korps een verschillende
%D specificatie. De daadwerkelijke definities gebeuren in de
%D file \type{font-map.tex}.

\unprotect

\definebodyfontenvironment
  [7pt]
  [\c!interlinie=8pt,
   \c!groot=8pt,
   \c!klein=6pt]

\definebodyfontenvironment
  [8pt]
  [\c!interlinie=9pt,
   \c!groot=9pt,
   \c!klein=7pt]

\definebodyfontenvironment
  [9pt]
  [\c!interlinie=11pt,
   \c!groot=10pt,
   \c!klein=8pt]

\definebodyfontenvironment
  [22pt]
  [\c!interlinie=22pt,
   \c!groot=22pt,
   \c!klein=17pt,
   \c!x=17pt]

%D Met de instellingen \type{groot} en \type{klein} leggen we
%D vast naar welk korps wordt overgegaan als we aan
%D \type{\switchnaarkorps} de trefwoorden \type{klein} of
%D \type{groot} meegeven. Met \type{x} geven we de maat aan van
%D de pseudo klein||kapitalen. De \MAPS\ wordt gezet in 9pt.

%\setupbodyfont[map,9pt]
%
%\startnotmode[localfonts]
%
%  \setupbodyfont[mty]
%
%\stopnotmode

%D We gebruiken echte small caps: 

\stelsorterenin
  [logo]
  [\c!letter=\v!smallcaps]

\setupcapitals
  [\c!titel=\v!nee]

%D De \MAPS\ heeft de breedte van een A4||tje maar is iets
%D minder hoog. We definieren daarom een wat afwijkend
%D papierformaat en projecteren dat op A4.

\definepapersize
  [maps]
  [\c!breedte=21cm,
   \c!hoogte=26.4cm]

\setuppapersize
  [maps]
  [A4]

%D Artikelen in de \MAPS\ worden breed danwel in twee kolommen
%D gezet. Vooralsnog beperken we de definitie van de layout tot
%D de een||koloms opmaak.

\setuplayout
  [\c!breedte=28pc,
   \c!hoogte=\v!midden,
   \c!marge=9pc,
   \c!margeafstand=1pc,
   \c!rugwit=1in,
   \c!kopwit=1.8cm,
   \c!regels=53,
   \c!hoofd=35pt,
   \c!voet=35pt]

\stelkolommenin
  [\c!afstand=1pc]

%D Later zullen we zonodig nog instellen dat de \MAPS\ op een
%D grid wordt gezet. Het zetten op een grid is voor \TEX\ niet
%D triviaal, en de ondersteuning in \CONTEXT\ is dan ook
%D enigzins experimenteel. De \MAPS\ is in die zin een soort
%D continue testcase.

%D De \MAPS\ wordt vanzelfsprekend dubbelzijdig gezet. Verderop
%D gaan het paginanummer expliciet plaatsen, vandaar dat we
%D hier de plaats niet specificeren.

\setuppagenumbering
  [\c!plaats=,
   \c!variant=\v!dubbelzijdig]

%D De hoofd- en voetregels worden gezet in een sans serif.
%D Het instellen op layout niveau is efficienter dan op elk
%D regel niveau.

\setuplayout
  [\c!letter=\ss]

%D In principe hebben we 4 verschillende soorten witruimte,
%D afhankelijk van het al dan niet op grid en/of in
%D kolommen zetten. We definieren daarom drie soorten wit:

\definieerblanko [mapsvoor]   [groot]
\definieerblanko [mapstussen] [groot]
\definieerblanko [mapsna]     [mapsvoor]

%D Waar nodig springen we in. We willen echter niet inspringen
%D na een witruimte, vandaar het trefwoord \type{volgende}.

\setupindenting
  [\v!volgende,9.5pt]

%D Hoewel niet strikt noodzakelijk, definities en dergelijke
%D worden namelijk omgeven door witruimte, stellen we ook hier
%D in dat we na zo'n definitie niet willen inspringen.

\steldoordefinierenin
  [\c!springvolgendein=\v!nee]

\steldoornummerenin
  [\c!springvolgendein=\v!nee]

%D We lijnen netjes uit, behalve in opsommingen, waar we
%D rechts raffelen. Ook hier springen we niet in na een
%D opsomming.

\stelopsommingin
  [\v!elk]
  [\c!springvolgendein=\v!nee,
   \c!uitlijnen=\v!rechts]

%D Nu we toch met opsommingen bezig zijn, introduceren we
%D meteen een nieuw symbool, een vierkantje. We tekenen dit
%D symbool met behulp van de macros van de visuele debugger.

\def\MapsSymbool%
  {\hbox
     {\boxrulewidth=.4pt
      \raise.2ex\ruledvbox
        {\phantom{\vrule width .75ex height .75ex}}}}

%D Vervolgens definieren we dit stukje zetwerk als symbool:

\definesymbol
  [MapsSymbool]
  [\MapsSymbool]

%D en koppelen het aan de opsommingen op het eerste niveau.

\stelopsommingin
  [1]
  [\c!symbool=MapsSymbool]

%D Ook passen we de breedte aan het inspringen aan en
%D plaatsen we de verschillende items op elkaar. De afstand
%D speelt vooral een rol als we ruimer willen zetten.

\stelopsommingin
  [\v!elk]
  [\c!breedte=9.5pt,
   \c!afstand=0pt]

\stelopsommingin
  [1]
  [\v!opelkaar]

%M % Nodig voor documenteren:
%M
%M \unprotect
%M
%M \doordefinieren
%M   [MapsBegrip]
%M   [\c!plaats=\v!hangend,
%M    \c!afstand=.75em,
%M    \c!marge=\v!standaard]
%M
%M \protect

%D Hoewel auteurs zelf nieuwe begrippen en dergelijke kunnen
%D definieren, introduceren we een klasse standaard begrippen,
%D namelijk \type{\MapsBegrip}. Men kan dus een begrip als
%D volgt definieren:
%D
%D \startbuffer
%D \MapsBegrip {MAPS} Het periodiek van de \NTG\ dat twee maal
%D per jaar verschijnt.
%D \stopbuffer
%D
%D \typebuffer
%D
%D en krijgt dan:
%D
%D \haalbuffer

\doordefinieren
  [MapsBegrip]
  [\c!plaats=\v!hangend,
   \c!afstand=.75em,
   \c!marge=\v!standaard]

%D Om ruimte te sparen, omringen we verbatim niet met hele lege
%D regels, maar met halve.

\setuptyping
  [\c!optie=\v!kleur,
   \c!blanko=mapstussen,
   \c!voor={\blanko[mapsvoor]},
   \c!na={\blanko[mapsna]}]

%D Speciaal voor erg brede verbatim teksten, definieren we
%D een brede variant. Deze steekt alleen op de linkerbladzijden
%D in de marge.

\definetyping
  [breedtypen]

\setuptyping
  [breedtypen]
  [\c!onevenmarge=-6pc]

%D En dan nu wat echte macro definities. Samenvattingen en
%D trefwoorden worden links danwel rechts uitgelijnd gezet.
%D Omdat sommige artikelen ook nog andere introducerende toeters
%D en bellen hebben, definieren we een algemene macro.

\def\startWhatever#1%
  {\witruimte
   \snaptogrid\vbox\bgroup
     \forgetall
     \setupalign[\v!rechts]
     \parfillskip 0pt plus 1 fill
     \setuptolerance[\v!zeersoepel]
     \setupindenting[\v!geen]
     {\ssbf#1}\par
     \switchtobodyfont[8pt]
     \ss\tf
     \ignorespaces}

\def\stopWhatever%
  {\par
   \egroup
   \blanko[\v!regel]}

%D De macro \type{\snaptogrid} is hier essentieel en zorgt
%D ervoor dat de tekst op het grid wordt gezet, ondanks de
%D afwijkende spatiering.

\def\Whatever#1#2%
  {\startWhatever{#1}#2\stopWhatever}

\def\defineWhatever#1#2%
  {\setvalue{\e!start#1}{\startWhatever{#2}}
   \setvalue {\e!stop#1}{\stopWhatever}
   \setvalue        {#1}{\Whatever{#2}}}

\defineWhatever{Abstract}{abstract}
\defineWhatever{Keywords}{keywords}

%D Een samenvatting ziet er nu als volgt uit:
%D
%D \starttypen
%D \startAbstract ..... \stopAbstract
%D \stoptypen
%D
%D of korter:
%D
%D \starttypen
%D \Abstract{...}
%D \stoptypen
%D
%D Ook hebben we de beschikking over:
%D
%D \starttypen
%D \startWhatever{....} ..... \stopWhatever
%D \stoptypen

%D We zullen de artikelen per stuk verwerken. Elk artikel heeft
%D z'n eigen kenmerken, die, zoals we later zullen zien, worden
%D ingesteld aan heb begin van een artikel:
%D
%D \starttypen
%D \startBijdrage[instellingen]
%D .....
%D .....
%D \stopBijdrage
%D \stoptypen
%D
%D Hieronder is te zien wat de standaard instellingen zijn:

\def\MapsTypering[#1]%
  {\getparameters
     [Maps]
     [Jaar=1999,
      Periode=Voorjaar,
      Categorie=Bijlage,
      Nummer=F,
      Pagina=43,
      Titel=Publish or Perish,
      Subtitel=,
      Auteur=M.A.P.S. Auteur,
      Adres=,
      Email=maps@ntg.nl,
      Kolommen=\v!nee,
      Grid=\v!ja,
      #1]}

%D In het adres en bij de auteur kan men \type{\\} gebruiken
%D om naar een nieuwe regel over te gaan. Voor de zekerheid
%D initialiseren we deze instellingen met:

\MapsTypering[Categorie=Voorbeeld]

%D De instellingen zijn beschikbaar als \type{\MapsVariabele}.
%D In de onderstaande definities van de hoofd- en voetregels
%D zien we de instellingen terug (de prefix \type{\unexpanded}
%D zorgt ervoor dat we straffeloos kunnen testen):

\unexpanded\def\LinkerKolomTekst#1%
  {\rlap{\hbox to 10pc{\hfill#1}}}

\unexpanded\def\RechterKolomTekst#1%
  {\llap{\hbox to 10pc{#1\hfill}}}

\edef\AuteurScheider{, }

\unexpanded\def\AuteurNamen#1%
  {{\let\\=\AuteurScheider#1}}

\setupheadertexts
  [\MapsTitel]
  [\LinkerKolomTekst{\MapsCategorie\ \MapsNummer}]
  [\RechterKolomTekst{\MapsCategorie\ \MapsNummer}]
  [\AuteurNamen{\MapsAuteur}]

\setupfootertexts
  [\MapsPeriode\ \MapsJaar]
  [\LinkerKolomTekst{\pagenumber}]
  [\RechterKolomTekst{\pagenumber}]
  [MAPS]

%D Voetnoten worden onderaan de bladzijde gezet, of, in geval
%D van kolommen, in de tweede kolom. We definieren een wat
%D afwijkende lijn als scheider:

\def\VoetNootLijn%
  {\strut\vrule height .4pt depth 0pt width 9.25pc
   \vskip0pt}

\setupfootnotes
  [\c!korps=8pt,
   \c!plaats=\v!kolommen,
   \c!lijn=\VoetNootLijn,
   \c!voor=\blanko,
   \c!letter=\v!schuin,
   \c!nummercommando=]

%D De voetnoot zelf stellen we in met:

\setupfootnotedefinition
  [\c!plaats=\v!aansluitend,
   \c!breedte=\v!passend,
   \c!kopletter=\v!normaal,
   \c!afstand=.5em]

%D We komen nu aan de wat lastiger macros. Het eerste dat een
%D bijdrage definitie doet, is nagaan of we in kolommen zetten.
%D Afhankelijk van de situatie, laden we wat aanvullende
%D definities. De titelpgina heeft geen hoofdregels en de
%D titel wordt automatisch gezet.

\def\Bericht{Bericht}

\newif\ifMapsInKolommen

\def\dostartBijdrage[#1]%
  {\pagina
   \MapsTypering[#1]
   \doifelse{\MapsKolommen}{\v!nee}
     {\MapsInKolommenfalse}
     {\MapsInKolommentrue}
   \ifMapsInKolommen
     \haalbuffer[s-maps-1] % see later on
   \fi
   \setupheader[\c!status=\v!leeg]
   \stelpaginanummerin[\c!nummer=\MapsPagina]
   \setuplayout[\c!grid=\MapsGrid]
   \ifgridsnapping                           % nog controleren
     \setupblank[\v!regel]
     \definieerblanko[mapsvoor][\v!halveregel]
     \definieerblanko[mapstussen][\v!regel]
   \else
     \setupblank[\v!halveregel]
     \definieerblanko[\v!middel][\v!halveregel]
     \definieerblanko[mapsvoor][\v!halveregel]
     \definieerblanko[mapstussen][\v!halveregel]
   \fi
   \snaptogrid\vbox
     \bgroup
       \switchtobodyfont[ss,22pt]%\ss
       \bgroup
         \topskip 0pt
         \forgetall
         \ifMapsInKolommen \setupalign[\v!links] \fi
         \noindent\tf\MapsAffiliatieA
         \ifx\MapsCategorie\Bericht
            ~\strut
         \else
            \MapsCategorie~\MapsNummer
         \fi
         \par \kern -.5pt
       \egroup
       \bgroup
         \forgetall
         \kern 7.5pt
         \ifMapsInKolommen 
           \setupalign[\v!links] 
         \else
           \setupalign[\v!rechts] 
         \fi
         {\bf\MapsTitel\par}
         {\bfx\MapsSubtitel\par}
         \blanko[\v!regel]
         \kern 3.5pt
       \egroup
     \egroup
   \ifMapsInKolommen\startkolommen\fi
   \MapsAffiliatieB}

\def\startBijdrage%
  {\starttekst
   \dosingleempty\dostartBijdrage}

\def\stopBijdrage%
  {\ifMapsInKolommen\stopkolommen\fi
   \stoptekst}

%D De affiliatie wordt in de marge gezet. De wat gecompliceerde
%D definitie zorgt er voor dat de bovenkant van de eerste
%D regels uitlijnen. Met \type{\getpagestatus} kunnen we
%D vaststellen of we op een rechterbladzijde zitten.

\unexpanded\def\MapsAffiliatieA%
  {\ifMapsInKolommen \else
     \snaptogrid\vbox\bgroup
     \setbox0=\hbox{X}%
     \dimen0=\ht0
     \switchtobodyfont[9pt,\v!reset]%
     \setbox0=\hbox{X}%
     \advance\dimen0 by -\ht0
     \getpagestatus
     \ifodd\MapsPagina\relax \rightpagetrue \fi
     \setbox0=\vtop
       {\hsize\margebreedte
        \forgetall
        \let\\=\par
        \ifrightpage\raggedright\else\raggedleft\fi
        \parfillskip 0pt plus \margebreedte
        \strut\MapsAuteur\\
        \MapsAdres\\ % kan leeg zijn
        \strut\tttf\MapsEmail}%
     \ht0=\ht\strutbox
     \dp0=\dp\strutbox
     \ifrightpage
       \rlap{\kern\zetbreedte\kern\margeafstand\raise\dimen0\box0}%
     \else
       \llap{\raise\dimen0\box0\kern\margeafstand}%
     \fi
     \global\let\Affiliatie=\relax
     \egroup
   \fi}

\unexpanded\def\MapsAffiliatieB%
  {\ifMapsInKolommen
     \snaptogrid\vbox\bgroup
     \forgetall
     \ss
     \let\\=\par
     \strut\MapsAuteur\\
     \MapsAdres\\ % kan leeg zijn
     \strut\tttf\MapsEmail\\
     \egroup
   \fi}

%D De verschillende koppen worden zo efficient mogelijk gezet.
%D Let ook hier weer op de halve regels, waardoor gridsnapping
%D eenvoudiger is.

\stelkopin
  [\v!paragraaf]
  [\c!letter=\bfa,
   \c!voor={\blanko[\v!halveregel]},
   \c!na={\blanko[\v!halveregel]}]

\stelkopin
  [\v!sub\v!paragraaf]
  [\c!letter=\bf,
   \c!voor=\blanko,
   \c!na=]

\stelkopin
  [\v!sub\v!sub\v!paragraaf]
  [\c!letter=\bf,
   \c!variant=\v!tekst,
   \c!voor=\blanko,
   \c!na=]

%D De plaats van de figuren en het lettertype waarin de
%D bijschriften worden gezet stellen we in met:

\stelplaatsblokkenin
  [\c!plaats=\v!links]

\stelblokkopjesin
  [\c!letter=\sstf,
   \c!tussen=]

%D Gezien het brede karakter van de \MAPS\ gebruiken we andere
%D waarden voor links geraffeld uitlijnen. The low level macro 
%D used here is subjected to changes!

\def\raggedleft%
  {\setraggedness\leftraggedness
   \setraggedskips{1}{1fill}{0em}{.3333em}{.5em}{1fil}{\parindent}}

%D Standaard gebruikt \CONTEXT\ kleur bij pretty verbatim
%D printing. In the \MAPS\ gebruiken we echter fonts:

\def\beginofpretty[#1]%
  {\bgroup
   \def\COMMONprettyone  {\ttsl}%
   \def\COMMONprettytwo  {\ttbf}%
   \def\COMMONprettythree{}%
   \def\COMMONprettyfour {}%
   \getvalue{COMMON#1}}

\def\endofpretty%
  {\egroup}

%D We zijn (heel) wat toleranter in het uitlijnen:

\steltolerantiein
  [\v!zeersoepel]

%D We misbruiken het buffer mechanisme om enkele eerder
%D gedefinieerde instellingen te veranderen in geval van zetten
%D in kolommen.

\startbuffer[s-maps-1]

%D In kolom||mode gebruiken we geen marge en hebben dus wat
%D meer breedte ter beschikking.

\setuplayout
  [\c!breedte=38pc]

\setupheadertexts
  [\MapsTitel]
  [\MapsCategorie\ \MapsNummer]
  [\MapsCategorie\ \MapsNummer]
  [\AuteurNamen{\MapsAuteur}]

\setupfootertexts
  [\MapsPeriode\ \MapsJaar]
  [\pagenumber]
  [\pagenumber]
  [MAPS]

%D In kolom||mode plaatsen we de inleidingen rechts met een
%D raffelige linkerkantlijn.

\def\startWhatever#1%
  {\witruimte
   \snaptogrid\vbox\bgroup
   \forgetall
   \setupalign[\v!links]
   \steltolerantiein[\v!zeersoepel]
   \setupindenting[\v!geen]
   \switchtobodyfont[8pt]
   \noindent{\ssbf#1}\par
   \ss\tf
   \ignorespaces}

\def\stopWhatever
  {\par
   \egroup
   \verticalstrut
   \noindentation}

%D Tot zover de twee||koloms instelingen.

\stopbuffer

%D Enkele \LATEX\ conversie macro's:

\let\verb=\type

\def\mailadres#1{{\tt#1}}

\endinput
