%D \module
%D   [       file=core-fig,
%D        version=1997.03.31,
%D          title=\CONTEXT\ Core Macros,
%D       subtitle=Figure Inclusion,
%D         author=Hans Hagen,
%D           date=\currentdate,
%D      copyright={PRAGMA / Hans Hagen \& Ton Otten}]
%C
%C This module is part of the \CONTEXT\ macro||package and is
%C therefore copyrighted by \PRAGMA. See mreadme.pdf for
%C details.

% error in calculations : .25% (too much: 1.5pt over full page)
%
% this module will be reimplemented(read: cleaned up)

\writestatus{loading}{Context Core Macros / Figure Inclusion}

\unprotect

% tex, tmp, mov and avi will become part of the fuzzy
% graphics and also behandled by special drivers; the
% current support is hackery

% figurefilemode checken
% zowieso alles checken
% movie scanner

%D Scanning for illustrations is automated to the max. Right
%D from the beginning \CONTEXT\ supported figure inclusion
%D using a dedicated figure directory file. Apart from the fact
%D that such a file enables us to include graphics that cannot
%D be parsed by \TEX\ for dimensions, by using this file we can
%D also quite easily generate figure directories. Only when
%D \PDFTEX\ started offering \PDF\ inclusion, I felt the need
%D to automate dimension detection to a higher degree.
%D Fortunately \TEXUTIL\ can scan more types now as well as
%D that we can run \TEXUTIL\ from within \TEX.

\startmessages  dutch  library: figures
   title: figuren
       1: figuur -- is niet te vinden
       2: figuur -- wordt niet preset
       3: maten van figuur -- geleend van --
       4: maten van -- geladen uit figuurfile zelf
       5: maten van -- geladen uit figuurfile --
       6: maten van -- berekend door TeXUtil
       7: figuurfile -- moet opnieuw worden aangemaakt
       8: figuurobject -- wordt opnieuw gebruikt
       9: figuur -- wordt niet afgehandeld
      10: figuur -- heeft geen afmetingen
      11: invoegen bron --
\stopmessages

\startmessages  english  library: figures
   title: figures
       1: figure -- can not be found
       2: figure -- is not preset
       3: dimensions of figure -- borrowed from --
       4: dimensions of -- loaded from figurefile itself
       5: dimensions of -- loaded from figurefile --
       6: dimensions of -- calculated by TeXUtil
       7: you have to regenerate figure file --
       8: figureobject -- is reused
       9: figure -- is not handled
      10: figure -- has zero dimensions
      11: including resource --
\stopmessages

\startmessages  german  library: figures
   title: Abbildungen
       1: Abbildung -- kann nicht gefunden werden
       2: Abbildung -- wird nicht erstellt
       3: Dimensionen von -- uebernommen von --
       4: Dimensionen von -- geladen aus der Abbildungsdatei selbst
       5: Dimensionen von -- geladen aus Abbildungsdatei --
       6: Dimensionen von -- ausgerechnet durch TeXUtil
       7: Sie muessen eine neue Abbildungsdatei -- erstellen
       8: Abbildungobjekt -- wurde wiederverwandt
       9: Abbildung -- wird nicht unterstuetzt
      10: figure -- has zero dimensions
      11: including resource --
\stopmessages

\startmessages  czech  library: figures
   title: obrazy
       1: obraz -- nelze nalezt
       2: obraz -- nepritomen
       3: dimenze obrazu -- vypujceny od --
       4: dimenze obrazu -- nacteny primo z jeho souboru
       5: dimenze obrazu -- nacteny ze souboru obrazu --
       6: dimenze obrazu -- spocteny programem TeXUtil
       7: musite znovu vygenerovat soubor obrazu --
       8: obrazovy objekt -- je znovu pouzit
       9: figure -- is not handled
      10: figure -- has zero dimensions
      11: including resource --
\stopmessages

\startmessages  italian  library: figures
   title: figure
       1: figura -- non trovata
       2: la figura -- non � preimpostata
       3: dimensioni della figura -- prese da --
       4: dimensioni di -- caricate dal file di immagini stesso
       5: dimensioni di -- caricate dal file di immagini --
       6: dimensioni di -- calcolate da TeXUtil
       7: bisogna rigenerare il file di immagini --
       8: oggetto-figura -- riutilizzato
       9: figura -- non gestita
      10: la figura -- ha dimensioni nulle
      11: including resource --
\stopmessages

\startmessages  romanian  library: figures
   title: figuri
       1: figura -- nu poate fi gasita
       2: figura -- nu este presetata
       3: dimensiunea figurii -- se imprumuta din --
       4: dimensiunea figurii -- se incarca din fisierul insusi
       5: dimensiunea figurii -- se incarca din fisierul --
       6: dimensiunea figurii -- este calculata de TeXutil
       7: trebuie sa refaceti fisierul imagine --
       8: obiectul figura -- este refolosit
       9: sufixul -- din figura -- nu este folosit
      10: figura -- are dimensiuni nule
      11: including resource --
\stopmessages

%D Due to the mere fact that \DVI|/|\PDF\ drivers differ in their
%D needs for figure dimensions, we have to provide the width,
%D height, horizontal and vertical scale. Also we want to
%D specify at the user level either width and|/|or height, scale,
%D or a factor related to the current document bodyfont size.
%D Even better: we can also specify isometric scaling and
%D automatically let \CONTEXT\ calculate the maximum possible
%D dimensions. Whatever we calculate, the results will come
%D available in the next registers.

\newcount \figxsca
\newcount \figysca
\newdimen \fighei
\newdimen \figwid

%D Because looking for dimensions can take many steps (locating
%D the figure, maybe on more directories, scanning the figure
%D on dimension, or when not found, trying to find them in the
%D utility file, and again when not found, trying to generate
%D such a file, and, as a last resort, trying to use the
%D dimensions. Now when things do not work out the way we want,
%D we can set a switch and get some information on what takes
%D place.

\newif\iftraceexternalfigures % \traceexternalfigurestrue

\let\traceexternalfigures \traceexternalfigurestrue

%D Another switch tells \CONTEXT\ to locate and calculate a
%D figure, but does not actually insert it. Especially when we
%D use \PDFTEX\ this saves a lot of time on trialruns. (Keep
%D in mind that \PDFTEX\ is both a \TEX\ pre|| and postprocessor.)

\newif\ifskipexternalfigures  % \skipexternalfigurestrue

%newif\ifsplitexternalfigures

%D A last switch inhibits running \TEXUTIL. Lets do it when
%D possible.

\newif\ifrunutilityfile       % \runutilityfiletrue
\newif\ifconsultutilityfile     \consultutilityfiletrue

%D When I ever decide to change the format of the figure
%D directory file that \TEXUTIL\ produces, the next number
%D needs to be changed.

\edef\figureversion{1996.06.01}

%D We keep track of the current state by setting a variable
%D which value is related to the method that provided the
%D dimensions.

\chardef\figurefilemode=0

%D The next values are set:
%D
%D \startitemize[packed]
%D \sym  0  the dimensions are not found
%D \sym  1  the dimensions are not preset at all
%D \sym  2  the dimensions are taken from other
%D \sym  3  the dimensions are taken from figure
%D \sym  4  the dimensions are taken from texutil.tuf
%D \sym  5  the dimensions are generated by texutil.tmp
%D \stopitemize
%D
%D In our search for the right file, that is, when no
%D filetype is specified, we scan for the next set of files.
%D As one can see, we prefer outlines over bitmaps.

\def\figuretypes{\c!eps,\c!mps,\c!pdf,\c!svg,\c!svg z,\c!png,\c!jpg,\c!tif} % ,\c!tex,\c!tmp} % \c!mov

%D Instead of using a comma separated list, we could have use a
%D faster alternative, but the current implementation is not
%D that slow either.
%D
%D Sorry for those who want to understand every bit, but I
%D will only sparse comment on the next macros. These macros
%D evolved out of the original macros and thereby lost all of
%D their beauty.
%D
%D We save the progess state in a macro. The main reason for
%D this is that otherwise the log would end up intermingled
%D with \TEX's hard coded file loading messages and launching
%D \TEXUTIL.

\def\@@eftrace#1%
  {\iftraceexternalfigures
     \edef\externalfigurelog{\externalfigurelog[#1]\space}%
   \fi}

\let\@@efcurrenttype\empty
\let\@@efcurrentpath\empty
\let\@@efcurrentfile\empty

% \def\analyzefigurefiles
%   {\let\externalfigurelog\empty
%    \let\@@efcurrenttype\empty
%    \let\@@efcurrentpath\empty
%    \let\@@efcurrentfile\empty
%    \doanalyzefigurefiles\doanalyzefigurefilesA
%    \doanalyzefigurefiles\doanalyzefigurefilesB
%    \doanalyzefigurefiles\doanalyzefigurefilesC}
%
% but, we also want to support direct paths, like e:/....

\def\redoanalyzefigurefiles#1%
  {\ifcase\figurestatus
     \def\@@efcurrenttype{#1}%
     \dododoanalyzefigurefiles\empty
   \fi}

\def\analyzefigurefiles
  {\let\externalfigurelog\empty
   \let\@@efcurrenttype\empty
   \let\@@efcurrentpath\empty
   \let\@@efcurrentfile\empty
   % empty path list if hard coded path
   \doanalyzefigurefiles\doanalyzefigurefilesA
   \doanalyzefigurefiles\doanalyzefigurefilesB
   \doanalyzefigurefiles\doanalyzefigurefilesC
   % new, permits rather raw names like e:/....
\ifx\@@effilepath\empty
   % we don't want a global search (happens with empty path)
\else
   \@@eftrace{checking filepath "\@@effilepath"}%
   \let\dodoanalyzefigurefiles\redoanalyzefigurefiles
   \doanalyzefigurefiles\doanalyzefigurefilesA
   \doanalyzefigurefiles\doanalyzefigurefilesB
   \doanalyzefigurefiles\doanalyzefigurefilesC
\fi}

%D The previous macro suggests that there are three main
%D methods applied. First we pass over all types and
%D directories specified and as soon as we find a suitable
%D candidate, we try to find its dimensions. When we cannot in
%D any way find the dimensions, directly, using the utility
%D file, or using \TEXUTIL\ directly, we revert to the second
%D method, and make a pass over all utility files. The last
%D method scans the utility files for files with the same name,
%D but different type.

\let\figurepathlist\empty

\def\doanalyzefigurefiles#1%
  {\let\dodododoanalyzefigurefiles#1%
   \processcommacommand[\@@eftype]\dodoanalyzefigurefiles}

\def\dodoanalyzefigurefiles#1%
  {\ifcase\figurestatus
     \def\@@efcurrenttype{#1}%
     \processcommacommand[\figurepathlist]\dododoanalyzefigurefiles
   \fi}

\def\dododoanalyzefigurefiles#1% todo: use the \applied ones, less messy
  {\ifcase\figurestatus
     \def\@@efcurrentpath{#1}%
     \sanitizefilename#1\to\@@efcurrentpath
     \@@eftrace{checking path "\@@efcurrentpath"}%
     \doiffileinsertionsupportedelse\@@efcurrenttype
       {\assignfullfilename{\@@efcurrentpath}{\@@effilename.\figureextension{\@@efcurrenttype}}\to\@@efcurrentfile
        \dodododoanalyzefigurefiles}
       \donothing
   \fi}

%D Here is our first method: we scan the file directly, parse
%D the utility file next, and finaly run \TEXUTIL. The latter
%D two of course only take place when the first scan fails.

\def\doanalyzefigurefilesA
  {\ifcase\figurestatus
     \@@eftrace{locating \@@efcurrentfile\space as \@@efcurrenttype}%
     \doiffileelse\@@efcurrentfile
       {\getfiguredimensionsA
        \getfiguredimensionsB
        \getfiguredimensionsC}
       \donothing
   \fi}

%D It is possible to let \TEX\ determine the dimensions itself.
%D The next macro shows how such a test is implemented. The
%D special driver \type {supp-tpd} shows some more.

%D The check on extension prevents problems when drivers are
%D not loaded well, in which case the tex one comes first.
%D
%D Should be a special!

% never change the vsize / hsize here, is taken from env

\def\dogetfiguresizetex#1#2#3#4#5#6% file pagina ...
  {\doifinsetelse\@@effiletype{\c!tex,\c!tmp}
     {\ifx\@@efcurrentpath\empty\executedfalse\else\executedtrue\fi}
     \executedfalse
   \ifexecuted
     \global\setbox\foundexternalfigure\vbox
       {\insidefloattrue
        \forgetall
        \blank[\v!disable]% niet meer weg !
        \startreadingfile
        \readfile{#1}\donothing\donothing
        \stopreadingfile
        \endgraf
        \removelastskip}%
     \global\setbox\foundexternalfigure\hbox
       {\raise\dp\foundexternalfigure\box\foundexternalfigure}%
     #3\zeropoint
     #4\zeropoint
     #5\wd\foundexternalfigure
     #6\ht\foundexternalfigure
   \else
     \@@eftrace{ignored}%
   \fi}

\let\dogetfiguresizetmp\dogetfiguresizetex

%D Here we start scanning the other types:

\def\@@dogetfiguresize{dogetfiguresize}

\def\getfiguredimensionsA
  {\ifcase\figurestatus
     \@@eftrace{analyzing \@@efcurrentfile\space on \@@efcurrentpath\space as \@@efcurrenttype}%
     \!!widthb\zeropoint % ?
     \doifdefinedelse{\@@dogetfiguresize\@@efcurrenttype}
       {\executedtrue
        \getvalue{\@@dogetfiguresize\@@efcurrenttype}%
          \@@efcurrentfile\@@efpage
          \!!widtha\!!heighta\!!widthb\!!heightb}
       \executedfalse
     \ifexecuted
       \donetrue
       \ifdim\!!widtha=\zeropoint\relax\ifdim\!!heighta=\zeropoint\relax
         \ifdim\!!widthb=\zeropoint\relax\ifdim\!!heightb=\zeropoint\relax
           \showmessage\m!figures{10}\@@efcurrentfile
           \@@eftrace{zero}%
           \donefalse
         \fi\fi
       \fi\fi
       \doifelse\@@efcurrenttype\c!mps
         {\ifcase\EPScreator
            \executedfalse
          \else
            % zero width mp graphic can be useful -)
          \fi}
         {\ifdone
            % non zero dimensions
          \else
            % zero dimensions
            \executedfalse
          \fi}%
     \fi
     \ifexecuted
       \chardef\figurestatus\plusthree
       \doifelse\@@efcurrenttype\c!eps
         {\ifcase\EPScreator
            \@@eftrace{found}%
          \else
            \let\@@efcurrenttype\c!mps
            \@@eftrace{mps found}%
          \fi}
         {\@@eftrace{found}}%
       \geteparameters % e !
         [\??ep]
         [\c!x=\the\!!widtha,\c!y=\the\!!heighta,
          \c!w=\the\!!widthb,\c!h=\the\!!heightb]%
       \let\@@eftype\@@efcurrenttype
       \let\@@effullname\@@efcurrentfile
     \else
       \@@eftrace{not found}%
     \fi
   \fi}

\def\dogetfiguresizepdf#1#2#3#4#5#6%
  {\dogetPDFmediabox{#1}{#3}{#4}{#5}{#6}}

\def\dogetfiguresizeeps#1#2#3#4#5#6%
  {\dogetEPSboundingbox{#1}{#3}{#4}{#5}{#6}}

\def\dogetfiguresizemps
  {\dogetfiguresizeeps}

\def\dogetfiguresizesvg#1#2#3#4#5#6%
  {\doifelse\@@effiletype\c!svg\executedtrue\executedfalse
   \ifexecuted
     #3\zeropoint
     #4\zeropoint
     \startnointerference
       \startXMLignore
         \defineXMLcommand[svg][width=100,height=75]
           {\global\dimen1=\XMLop{width}\onebasepoint
            \global\dimen3=\XMLop{height}\onebasepoint
            \endinput}%
         \processXMLfilegrouped{#1}%
       \stopXMLignore
     \stopnointerference
     #5=\dimen1\relax
     #6=\dimen3\relax
   \else
     \@@eftrace{ignored}%
   \fi}

\def\getfiguredimensionsB
  {\ifcase\figurestatus\ifcase\figurefilemode\else
\doifsomething\@@efcurrentpath
{%
     \assignfullfilename\@@efcurrentpath\@@exfile\to\@@efloadname
     \edef\@@effilenametype{\@@effilename.\@@efcurrenttype}%
     \@@eftrace{analyzing \@@efloadname\space on \@@efcurrentpath\space for \@@effilenametype}%
     \pushendofline
     \startreadingfile
     \let\presetfigure\presetfigureA
     \readsetfile\@@efcurrentpath\@@exfile\donothing\donothing
     \stopreadingfile
     \popendofline
     \@@eftrace{\ifcase\figurestatus not \fi found}%
}%
   \fi\fi}

\def\presetfigureA[#1][#2]%
  {\ifcase\figurestatus
     \@EA\DOIF\@EA{\@@effilenametype}{#1} % hm, tzt ook nog eens met pad/naam
       {\getparameters[\??ep][#2]%
        \ExpandBothAfter\doif\@@epe\@@efcurrenttype
          {\chardef\figurestatus4
           \let\@@eftype\@@efcurrenttype
           \let\@@effullname\@@efcurrentfile}}%
   \else
     \endinput
   \fi}

\def\getfiguredimensionsC
  {\ifconsultutilityfile \ifrunutilityfile
     \ifcase\figurestatus\ifcase\figurefilemode\else
\doifsomething\@@efcurrentpath
{%
      \doifnotinset\@@effiletype{\c!tex,\c!tmp}
         {\doiffileelse\@@efcurrentfile
            {\edef\@@effilenametype{\@@effilename.\@@efcurrenttype}%
             \@@eftrace{running texutil on \@@effilenametype}%
             \def\@@efloadname{\f!utilityfilename.\f!temporaryextension}%
             \executesystemcommand{texutil --fig --out=\@@efloadname\space\@@effilenametype}%
             \@@eftrace{analyzing \@@efloadname\space on \@@effilenametype}%
             \pushendofline
             \startreadingfile
             \let\presetfigure\presetfigureB
             \readsetfile{.}\@@efloadname\donothing\donothing
             \stopreadingfile
             \popendofline
             \@@eftrace{\ifcase\figurestatus not \fi found}}
            {}}%
}%
     \fi\fi
   \fi\fi}

\def\presetfigureB[#1][#2]%
  {\getparameters[\??ep][#2]%
   \chardef\figurestatus=6    % ??????????????????
   \let\@@eftype\@@efcurrenttype
   \let\@@effullname\@@efcurrentfile}

%D The second pass over types and directories uses the
%D utilility files.

\def\doanalyzefigurefilesB
  {\ifconsultutilityfile\ifcase\figurestatus\ifcase\figurefilemode\else
\doifsomething\@@efcurrentpath
{%
     \assignfullfilename\@@efcurrentpath\@@exfile\to\@@efloadname
     \edef\@@effilenametype{\@@effilename.\figureextension{\@@efcurrenttype}}%
     \@@eftrace{analyzing \@@efloadname\space on \@@efcurrentpath\space for \@@effilenametype}%
     \pushendofline
     \startreadingfile
     \let\presetfigure\presetfigureC
     \readsetfile\@@efcurrentpath\@@exfile\donothing\donothing
     \stopreadingfile
     \popendofline
     \@@eftrace{\ifcase\figurestatus not \fi found}%
}%
   \fi\fi\fi}

\def\presetfigureC[#1][#2]%
  {\ifcase\figurestatus
     \@EA\DOIF\@EA{\@@effilenametype}{#1}
       {\getparameters[\??ep][#2]%
        \doif\@@epe\@@efcurrenttype
          {\chardef\figurestatus4
           \let\@@eftype\@@efcurrenttype
           \let\@@effullname\@@efcurrentfile}}%
   \else
     \endinput
   \fi}

%D The last and third pass mainly differs from the second in
%D being more tolerant.

\def\doanalyzefigurefilesC
  {\ifconsultutilityfile\ifcase\figurestatus\ifcase\figurefilemode\else
\doifsomething\@@efcurrentpath
{%
     \assignfullfilename\@@efcurrentpath\@@exfile\to\@@efloadname
     \@@eftrace{analyzing \@@efloadname\space on \@@efcurrentpath\space for \@@effilename.* surrogate}%
     \pushendofline
     \startreadingfile
     \let\presetfigure\presetfigureD
     \readsetfile\@@efcurrentpath\@@exfile\donothing\donothing
     \stopreadingfile
     \popendofline
     \@@eftrace{\ifcase\figurestatus not \fi found}%
}%
   \fi\fi\fi}

\def\presetfigureD[#1][#2]%
  {\ifcase\figurestatus
     \@EA\DOIFINSTRINGELSE\@EA{\@@effilename.}{#1}
       {\getparameters[\??ep][#2]%
        \ExpandBothAfter\doifinsetelse\@@epe\@@efcurrenttype
          {\chardef\figurestatus4
           \let\@@eftype\@@efcurrenttype
           \let\@@effullname\@@efcurrentfile}%
          \donothing}
       \donothing
   \else
     \endinput
   \fi}

%D While loading the utility file (often \type {texutil.tuf})
%D the next command (when present) aborts reading when the
%D versions don't match.

\def\thisisfigureversion#1%
  {\doifnot\figureversion{#1}
     {\showmessage\m!figures7\@@efloadname
      \endinput}}

%D Some files, take for instance movies, cannot easilly be
%D parsed on dimensions, that is, not yet. Although the current
%D mechanism has no problems with this, as long as the user
%D specified width and height reflect the right aspect ratio.
%D Nevertheless, when one does not want any scanning done, one
%D can disable \type{preset}. When no preset is needed, we only
%D locate the file.

\def\locatepresetfigurefiles
  {\processcommacommand[\@@eftype]\dolocatepresetfigurefiles}

\def\dolocatepresetfigurefiles#1%
  {\def\@@efcurrenttype{#1}%
   \processcommacommand[\figurepathlist]\dodolocatepresetfigurefiles}

\def\dodolocatepresetfigurefiles#1%
  {\ifcase\figurestatus
     \doiffileinsertionsupportedelse\@@efcurrenttype
       {\assignfullfilename{#1}{\@@effilename.\figureextension{\@@efcurrenttype}}\to\@@efcurrentfile
        \@@eftrace{only searching for \@@efcurrentfile}%
        \doiffileelse\@@efcurrentfile
          {\chardef\figurestatus\plusone
           \let\@@eftype\@@efcurrenttype
           \let\@@effullname\@@efcurrentfile}%
          \donothing}
       \donothing
   \fi}

%D Here we locate the to-be converted images (if needed).

\def\locatefigureconversionfile
  {\processcommacommand[\figurepathlist]\dolocatefigureconversionfile}

\def\dolocatefigureconversionfile#1%
  {\ifcase\figurestatus
     \sanitizefilename#1\to\@@efconversionpath
     \assignfullfilename{\@@efconversionpath}{\@@efconversionfile.\@@efconversiontype}\to\@@efconversionfull
     \@@eftrace{searching for original \@@efconversionfull}%
     \doiffileelse\@@efconversionfull{\chardef\figurestatus\plusone}\donothing
   \fi}

%D All these macros are in some way called by the macro \type
%D {\analyzefigurefiles}, which in turn is called by the next
%D macro.

% bools gebruiken

\def\setnaturalfiguresize
  {\doifsomething\@@efwidth
     {\global\figwid\@@efwidth}%
   \doifsomething\@@efheight
     {\global\fighei\@@efheight}%
   \doifsomething\@@efscale
     {\figxsca\@@efscale
      \figysca\@@efscale}}

\def\setfactorfiguresize
  {\doifinsetelse\@@effactor{\v!max,\v!fit,\v!broad}
     {\doapplyfiguresize
      \ifdim\@@epw>\@@eph\relax
        \docalculatefigurenorm\figwid\@@effactor\@@efmaxwidth\hsize\@@efhsize
        \docalculatefigurescales\figwid\@@epw\fighei\@@eph
      \else
        \docalculatefigurenorm\fighei\@@effactor\@@efmaxheight\figurevsize\@@efvsize
        \docalculatefigurescales\fighei\@@eph\figwid\@@epw
      \fi
      \!!doneatrue}
     {\doifinsetelse\@@efhfactor{\v!max,\v!fit,\v!broad}
        {\doapplyfiguresize
         \docalculatefigurenorm\fighei\@@efhfactor\@@efmaxheight\figurevsize\@@efvsize
         \docalculatefigurescales\fighei\@@eph\figwid\@@epw
         \!!doneatrue}
        {\doifinsetelse\@@efwfactor{\v!max,\v!fit,\v!broad}
           {\doapplyfiguresize
            \docalculatefigurenorm\figwid\@@efwfactor\@@efmaxwidth\hsize\@@efhsize
            \docalculatefigurescales\figwid\@@epw\fighei\@@eph
            \!!doneatrue}                            % max ???
           {\docalculatefigurenorm\fighei\@@effactor \@@efheight \textheight\@@efvsize
            \docalculatefigurenorm\fighei\@@efhfactor\@@efheight \textheight\@@efvsize
            \docalculatefigurenorm\figwid\@@efwfactor\@@efwidth\hsize\hsize
            \!!doneafalse}}}%
   \if!!donea
     \ifdim\figwid>\@@efhsize\relax
       \global\fighei\zeropoint
       \global\figwid\@@efhsize
     \else\ifdim\fighei>\@@efvsize\relax
       \global\fighei\@@efvsize
       \global\figwid\zeropoint
     \fi\fi
   \fi}

\def\setscalefiguresize
  {\doifsomething\@@efscale
     {\doapplyfigurescale\figwid\@@epw\figxsca
      \doapplyfigurescale\fighei\@@eph\figysca
      \global\figwid\zeropoint
      \global\fighei\zeropoint
      \doifelsenothing\@@efmaxwidth
        {\doifsomething\@@efmaxheight
           {\ifdim\@@eph>\@@efmaxheight
              \global\fighei\@@efmaxheight
            \fi}}
        {\ifdim\@@epw>\@@efmaxwidth
           \global\figwid\@@efmaxwidth
         \fi}}}

\let\@@efgrid\empty

\def\dosetdimensionfiguresize#1#2#3%
  {#1\relax
   \doifsomething\@@efmaxwidth {\ifdim\figwid>\@@efmaxwidth \global\figwid\@@efmaxwidth #2\relax\fi}%
   \doifsomething\@@efmaxheight{\ifdim\fighei>\@@efmaxheight\global\fighei\@@efmaxheight#3\relax\fi}}

\def\setdimensionfiguresize
  {\ifdim\figwid>\zeropoint\relax
     \ifdim\fighei>\zeropoint\relax
       \dosetdimensionfiguresize
         {\docalculatefigurescale\fighei\@@eph\figysca
          \docalculatefigurescale\figwid\@@epw\figxsca}%
         {\docalculatefigurescale\fighei\@@eph\figysca
          \docalculatefigurescale\figwid\@@epw\figxsca}%
         {\docalculatefigurescale\fighei\@@eph\figysca
          \docalculatefigurescale\figwid\@@epw\figxsca}%
     \else
       \dosetdimensionfiguresize
         {\docalculatefigurescales\figwid\@@epw\fighei\@@eph}%
         {\docalculatefigurescales\figwid\@@epw\fighei\@@eph}%
         {\docalculatefigurescales\figwid\@@epw\fighei\@@eph}%
     \fi
   \else
     \ifdim\fighei>\zeropoint\relax
       \dosetdimensionfiguresize
         {\docalculatefigurescales\fighei\@@eph\figwid\@@epw}%
         {\docalculatefigurescales\fighei\@@eph\figwid\@@epw}%
         {\docalculatefigurescales\fighei\@@eph\figwid\@@epw}%
     \else
       \dosetdimensionfiguresize
         {\doapplyfigurescale\figwid\@@epw\figxsca
          \doapplyfigurescale\fighei\@@eph\figysca}%
         {\docalculatefigurescales\figwid\@@epw\fighei\@@eph}%
         {\docalculatefigurescales\fighei\@@eph\figwid\@@epw}%
     \fi
   \fi}

\def\setupexternalfigures
  {\dosingleempty\dosetupexternalfigures}

\def\dosetupexternalfigures[#1]% needs a good clean up
  {\getparameters[\??ex][#1]%
   \getparameters[\??ef][#1]% dangerous for figs with backgrounds
   \checkfiguresettings
   \doifelsenothing\@@exlocation % fig file paths
     {\scratchcounter\plusthree}
     {\doifelsenothing\@@exfile % tuf file paths
        {\scratchcounter\plusthree}
        {\scratchcounter\zerocount
         \ExpandBothAfter\doifinset\v!local\@@exlocation
           {\advance\scratchcounter\plusone}%
         \ExpandBothAfter\doifinset\v!global\@@exlocation
           {\advance\scratchcounter\plustwo}}}%
   \chardef\figurefilemode\scratchcounter\relax
   \ifcase\figurefilemode
     \let\figurepathlist\f!currentpath
   \or % lokaal
     \let\figurepathlist\f!currentpath
   \or % globaal
     \let\figurepathlist\@@exdirectory
   \or % lokaal,globaal / non empty gebied
     \edef\figurepathlist{\f!currentpath\ifx\@@exdirectory\empty\else,\@@exdirectory\fi}%
   \fi
%
\doifnotmode{\systemmodeprefix\v!first}% first cache, else no lowres etc
  {\ifx\@@excache\empty\else\edef\figurepathlist{\@@excache,\figurepathlist}\fi}%
%
   \ExpandBothAfter\doifinset\v!default\@@exlocation
     {\edef\figurepathlist{\figurepathlist,}}% default tex path search
   \ifx\@@exfile\empty
     \chardef\figurefilemode\zerocount
   \fi}

%D The next one is for instance used in symbols:

\def\resetexternalfigures
  {\setupexternalfigures
     [\c!option=,\c!maxwidth=,\c!maxheight=,
     %\c!splitcolor=,% needed ?
      \c!foregroundcolor=,\c!color=,
     %\c!conversion=,\c!prefix=,
      \c!frame=\v!off,\c!background=]}

%D Since we only need to reset some parameters, we can
%D better use a faster alternative:

\def\resetexternalfigures
  {\getparameters[\??ef]
     [\c!option=,\c!maxwidth=,\c!maxheight=,
     %\c!splitcolor=,% needed ?
      \c!foregroundcolor=,\c!color=,
     %\c!conversion=,\c!prefix=,
      \c!frame=\v!off,\c!background=]}

%D This one dropped the runtime of the \MAPS\ bibliography
%D from over 110 seconds down to less than 105 seconds. The
%D tremendously faster (but uglier) implementation is:

\def\resetexternalfigures
  {\let\@@efoption         \empty
   \let\@@efmaxwidth       \empty
   \let\@@efmaxheight      \empty
   \let\@@efforegroundcolor\empty
   \let\@@efcolor          \empty
   \let\@@efconversion     \empty
  %\let\@@efprefix         \empty
  %\let\@@efcache          \empty
  %\let\@@efframe          \v!off
   \let\@@efbackground     \empty}

% The following code will move:

\appendtoks \resetexternalfigures \to \everyoverlay
\appendtoks \resetexternalfigures \to \everybeforepagebody % not really needed

%appendtoks \resetexternalfigures \to \everysymbol

\def\docalculatefigurenorm#1#2#3#4#5%
  {\processaction
      [#2]
      [     \v!max=>\global#1=#4\relax,
            \v!fit=>\global#1=#5\relax,
          \v!broad=>\global#1=#5\relax
                    \global\advance #1 -4\@@exbodyfont\relax,
        \s!default=>\doifsomething{#3}{\global#1=#3\relax},
        \s!unknown=>\global#1=\@@exbodyfont\relax
                    \global\divide#1 \!!ten\relax
                    \global\multiply#1 #2\relax]}

\beginTEX

  \def\docalculatefigurescales#1#2#3#4%
    {\dimen0=#1\relax                     % #1 = new 1-value
     \dimen2=#2\relax                     % #2 = old 1-value
     \divide\dimen2 \plusthousand
     \divide\dimen0 \dimen2
     \figxsca\dimen0                      %      x scale
     \figysca\dimen0                      %      y scale
     \dimen2=#4\relax                     % #4 = old 2-value
     \divide\dimen2 \plusthousand
     \multiply\dimen2 \dimen0
     #3=\dimen2 }                         % #3 = new 2-value

  \def\docalculatefigurescale#1#2#3%
    {\dimen0=#1\relax                     % #1 = new value
     \dimen2=#2\relax                     % #2 = old value
     \divide\dimen2 \plusthousand
     \divide\dimen0 \dimen2
     #3=\dimen0 }                         % #3 = schaal

  \def\doapplyfigurescale#1#2#3%
    {\global#1=#2\relax
     \ifcase0\@@efscale\relax % beter: doifnum...
       #3=\plusthousand
     \else
       #3=\@@efscale
     \fi\relax % important !
     \ifnum#3=\plusthousand\else
       \global\divide  #1 \plusthousand
       \global\multiply#1 #3\relax
     \fi}

\endTEX

% test extensively, more precise since one pass double precission

\beginETEX \dimexpr

  \def\docalculatefigurescales#1#2#3#4%
    {\scratchdimen\dimexpr(#1/\dimexpr(#2/\plusthousand))%
     \figxsca\scratchdimen
     \figysca\figxsca
     #3\dimexpr(\figxsca\dimexpr(#4/\plusthousand))}

  \def\docalculatefigurescale#1#2#3%
    {#3\dimexpr(#1/\dimexpr(#2/\plusthousand))}

  \def\doapplyfigurescale#1#2#3%
    {% #3=\ifnum0\number\@@efscale=\zerocount\plusthousand\else\@@efscale\fi
     \ifcase0\@@efscale\relax % @@efschaal kan empty zijn
       #3=\plusthousand
     \else
       #3=\@@efscale
     \fi\relax % important !
     \global#1\ifnum#3=\plusthousand#2\else\dimexpr(#3\dimexpr(#2/\plusthousand))\fi
     \relax}

\endETEX

\newdimen\figurevsize % we cannot manipulate any global vsize !

\def\doapplyfiguresize
  {\doifelsenothing\@@efmaxheight
     {\figurevsize\textheight
      \ifinner
        \figurevsize \vsize % \textheight =\vsize
        \scratchdimen\vsize % \scratchdimen=\textheight
      \else\ifinsidefloat
        \figurevsize \vsize % \textheight =\vsize
        \scratchdimen\vsize % \scratchdimen=\textheight
      \else\ifinpagebody
        \figurevsize \vsize % \textheight =\vsize
        \scratchdimen\vsize % \scratchdimen=\textheight
      \else % hm, there should be an option to force this
        \ifdim\pagegoal<\maxdimen
          \ifdim\pagetotal<\pagegoal
            \scratchdimen\pagegoal
            \advance\scratchdimen -\pagetotal
          \else
            \scratchdimen\figurevsize % \textheight
          \fi
        \else
          \scratchdimen\figurevsize % \textheight
        \fi
      \fi\fi\fi}
     {\figurevsize\@@efmaxheight}%
   \doifelsenothing\@@efheight
     {\edef\@@efvsize{\the\scratchdimen}}
     {\let\@@efvsize\@@efheight}%
   \doifelsenothing\@@efwidth
     {\edef\@@efhsize{\the\hsize}}
     {\let\@@efhsize\@@efwidth}}

% \def\convertfigureinsertscale#1#2#3#4%
%   {\scratchdimen#1\relax
%    \ifnum#3=\plusthousand\else % better 1000 100 10 ranges, evt round 2sp
%      \divide\scratchdimen \plusthousand
%      \multiply\scratchdimen #3\relax
%    \fi
%    \scratchdimen-\scratchdimen  % beter hier - dan in driver
%    \edef#2{\the\scratchdimen}% oeps, \the vergeten
%    \scratchdimen#3\points
%    \divide\scratchdimen \!!ten
%    \edef#4{\withoutpt\the\scratchdimen}}
%
% more obscure but better:

\def\convertfigureinsertscale#1#2#3#4%
  {\scratchdimen#1\relax
   \ifnum#3=\plusthousand
     % == scale 1
   \else
     % better 1000 100 10 ranges, evt round 2sp
     \divide\scratchdimen \plusthousand
     \multiply\scratchdimen #3\relax
   \fi
   \scratchdimen-\scratchdimen  % beter hier - dan in driver
   \edef#2{\the\scratchdimen}%
   \scratchcounter#3\relax
   \ifnum\scratchcounter>\plustenthousand
     \divide\scratchcounter \!!ten
     \scratchdimen\the\scratchcounter\points
   \else
     \scratchdimen\the\scratchcounter\points
     \divide\scratchdimen \!!ten
   \fi
   \edef#4{\withoutpt\the\scratchdimen}}

\newbox\foundexternalfigure

% \def\presetundefinedfigure#1%
%   {\let\@@eftype      #1%
%    \let\@@effiletype #1%
%    \let\@@efobject    \v!no
%    \let\@@efpreset    \v!no
%    \ifx\@@efwidth\empty
%      \def\@@efwidth{8\lineheight}% 4cm
%    \fi
%    \ifx\@@efheight\empty
%      \def\@@efheight{6\lineheight}%  3cm
%    \fi}

% \def\presetfiguremov{\presetundefinedfigure\c!mov}
% \def\presetfigureavi{\presetundefinedfigure\c!avi}

\def\presetundefinedfigure#1#2%
  {\let\@@eftype    #2%
   \let\@@effiletype#2%
   \let\@@efobject  \v!no
   \ifcase#1\or
     \let\@@efpreset\v!no
     \ifx\@@efwidth\empty\def\@@efwidth{8\lineheight}\fi
     \ifx\@@efwidth\empty\def\@@efwidth{6\lineheight}\fi
   \fi}

\def\presetfiguresvg{\presetundefinedfigure\zerocount\c!svg}
\def\presetfiguremov{\presetundefinedfigure\plusone  \c!mov}
\def\presetfigureavi{\presetundefinedfigure\plusone  \c!avi}

% The page number (frame) is passed as first option.

\newcounter\forcedMPSobject % better something \every<type>

%D We have arrived at one of the main macros, the one that
%D tries to analyze the figure, preloads it when possible, and
%D scales is according to the specifications. This macro is
%D quite unreadable, for which I appologize. The main
%D complication is that we have to catch all kind of border
%D cases, like \METAPOST\ graphics and buffers.

%  note * : this is needed because reusable graphics
%  combined with funny page aspect aspect ratio's can lead to
%  strange side effects of preceding factor=max specs. This
%  surfaced in the metafun manual, where the two side by
%  side clipped cow heads [the second one was a reused object]
%  where the second one inherited some characteristics from
%  the factor=max one some 30 pages back. Sigh.

%  This macro will be cleaned up when the tuf format has
%  become replaced by its xml counterpart; for that I first
%  need to patch texutil.

\def\checkfiguresettings
  {\doifsomething\@@eflines
     {\scratchdimen\@@eflines\lineheight
      \edef\@@efheight{\the\scratchdimen}}}

\chardef\splitexternalfigure=0 % 0 nosplit 1 split/yes 2 split/no

\newif\ifgridfigure

% the preset for mov/avi should move to the driver
%
% this whole mess needs a clean up anyway

\def\calculateexternalfigure[#1][#2][#3][#4][#5][#6]%
  {\dontcomplain
   \setupexternalfigures
   \the\externalfigureresets % hook, see resource libraries
   \global\figwid\zeropoint \figxsca\plusone % see note *
   \global\fighei\zeropoint \figysca\plusone % see note *
   \global\setbox\foundexternalfigure\box\voidb@x
   % get rid of active / and : as well as expand for [\get...] cases
   \sanitizefilename#3\to\expandedfigurename
   % nil path search in case of path spec
   \expanded{\checkfilename{\expandedfigurename}}%
   \ifcase\kindoffile\else \let\figurepathlist\empty \fi
%    \greedysplitstring\expandedfigurename\at.\to\@@effilename\and\@@effiletype
\splitfilename\expandedfigurename
\let\@@effilepath\splitoffpath
\let\@@effilename\splitoffname
\let\@@effiletype\splitofftype
\ifcase\splitoffkind\let\@@effilepath\empty\fi
   \doifelse\@@effilename{mprun}
     {\edef\@@effilepref{\bufferprefix}}
     {\let \@@effilepref\empty}%
   \edef\@@effilename{\@@effilepref\@@effilename}%
\ifx\@@effilepath\empty\else
% \addtocommalist\@@effilepath\figurepathlist
  \edef\@@effilename{\@@effilepath/\@@effilename}%
\fi
   \restorecatcodes  % recently added; we presume local use
   \def\@@eflabel{#2}%
   \global\let\externalfigurelog\empty
   \iftraceexternalfigures
     \showmessage\m!figures{11}\expandedfigurename
   \fi
   \getparameters
     [\??ep]
     [\c!e=\s!unknown,
     %\c!w=15\korpsgrootte,\c!h=10\korpsgrootte,
      \c!w=8\lineheight,\c!h=6\lineheight,
      \c!x=\!!zeropoint,\c!y=\!!zeropoint,
      \c!t=,\c!s=,\c!a=,\c!f=\@@effilename]%
   \getparameters
     [\??ef]
     [\c!type=\s!unknown,\c!method=\@@eftype,\c!symbol=\v!no,
      \c!object=\@@exobject,\c!preset=\v!yes,\c!split=,\c!size=,
      \c!page=0,\c!controls=\v!no,\c!preview=\v!no,\c!repeat=\v!no,
      \c!maxwidth=\@@exmaxwidth,\c!maxheight=\@@exmaxheight,
      \c!scale=,\c!width=,\c!height=,\c!display=,\c!lines=,\c!grid=,
     %\c!foregroundcolor=,
      \c!color=,\c!conversion=\@@exconversion,\c!prefix=\@@exprefix,\c!cache=\@@excache,
      \c!factor=,\c!hfactor=,\c!wfactor=]%
   % make into an installable list
   \doifnothing\@@efconversion
     {\doif\@@effiletype\c!mov\presetfiguremov
      \doif\@@effiletype\c!avi\presetfigureavi
      \doif\@@effiletype\c!svg\presetfiguresvg}%
   #1[#4][#5][#6]%
   % prefix in runs > 1
   % prefix in runs > 1
   \doifnotmode{\systemmodeprefix\v!first}
     {\doifsomething\@@efprefix{\edef\@@effilename{\@@efprefix\@@effilename}}%
      \let\@@effiletype\empty}% beware, walks over the whole suffixlist
   % new, somehow needed when \textwidth is used:
   \doifsomething\@@efmaxwidth {\freezedimenmacro\@@efmaxwidth }%
   \doifsomething\@@efmaxheight{\freezedimenmacro\@@efmaxheight}%
   % lines -> height
   \checkfiguresettings
   % new
   \doifinsetelse\@@efsize{mediabox,cropbox,artbox,bleedbox,trimbox}
     {\let\@@DriverImageBox\@@efsize}%
     {\let\@@DriverImageBox\empty}%
   % new, color separation
   \doifseparatingcolorselse
     {\let\@@efforegroundcolor\empty
      \doifelsenothing\@@efsplit
        {\chardef\splitexternalfigure0}
        {\doifcolorchannelelse\@@efsplit
           {\let\@@efobject\v!no % ?
            \chardef\splitexternalfigure1}
           {\chardef\splitexternalfigure2}}}
     {\chardef\splitexternalfigure0}%
   \relax % ends \chardef
   % new, fake color in gray bitmaps
   \doifsomething\@@efforegroundcolor
     {\getparameters[\??ef]
        [\c!background={\v!foreground,\v!color},
         \c!backgroundcolor=\@@efforegroundcolor]}%
   \doifsomething\@@efcolor
     {\doifcolorelse\@@efcolor
        {\checkpredefinedcolor[\@@efcolor]%
         \doregisterfigurecolor\@@efcolor}}%
        \donothing
   %
   \doif\@@efreset\v!yes\resetexternalfigures
   % todo: nop when conversion
   \doif\@@eftype \c!mov\presetfiguremov
   \doif\@@eftype \c!avi\presetfigureavi
   \doif\@@eftype \c!svg\presetfiguresvg
   % hack
   \doif\@@efmethod\c!mov{\doifsomething\@@effiletype{\presetundefinedfigure1\@@effiletype}}%
   \doif\@@efmethod\c!svg{\doifsomething\@@effiletype{\presetundefinedfigure0\@@effiletype}}%
   %
   \doif\@@eftype\v!buffer
     {\ifx\@@effiletype\empty
        \let\@@effiletype\c!tmp
      \fi
      \let\@@eftype\c!tex}%
   \@EA\doifnumberelse\@EA{\@@effiletype}% new, test first
     {\let\@@eftype\c!mps}
     {\processaction
       [\@@effiletype]
       [   \c!tex=>\let\@@eftype\c!tex,
           \c!tmp=>\let\@@eftype\c!tex
                   \edef\@@effilepref{\bufferprefix}%
                   \edef\@@effilename{\@@effilepref\@@effilename},
           % todo: nop when conversion
           \c!avi=>\presetfigureavi,
           \c!mov=>\presetfiguremov,
           \c!svg=>\presetfiguresvg]}%
   \edef\figuretypes{\figuretypes,\c!tex}%
   \ifx\@@eftype\c!tex
     % Since tex code can have positional stuff and worse,
     % we want to avoid interference with how objects end
     % up in files, therefore:
     \let\@@efobject\v!no
   \fi
   \edef\@@efobjectname{\@@effilename-\@@eftype-\@@effiletype-\@@efpage}%
   \doifelse\@@efobject\v!no
     \donefalse
     {\doifspecialavailableelse\dostartscaling
        {\doifobjectssupportedelse
           {\doifobjectfoundelse{FIG}\@@efobjectname\donetrue\donefalse}
           \donefalse}
        \donefalse}%
   % too strict:
   %    \doifparentfileelse\@@effilename
   %      {\@EA\removefromcommalist\@EA{\jobsuffix}\figuretypes
   %       \let\@@effiletype\empty
   %       \showmessage\m!figures9\@@effilename
   %       \donefalse}
   %      \donothing
   % ok, but catcode problems with jobfilesuffix
   \doifparentfileelse\@@effilename
     {\@EA\removefromcommalist\@EA{\jobsuffix    }\figuretypes
      \@EA\removefromcommalist\@EA{\jobfilesuffix}\figuretypes}%
     \donothing
   % so we get:
   \doifparentfileelse\@@effilename
     {\removefromcommalist{pdf}\figuretypes
      \removefromcommalist{tex}\figuretypes}%
     \donothing
   \ifdone
     \getobjectdimensions{FIG}\@@efobjectname
     \geteparameters % e !
       [\??ep]
       [\c!x=\!!zeropoint,\c!y=\!!zeropoint,
        \c!w=\objectwidth,\c!h=\objectheight]%
     \chardef\figurestatus=5
     \edef\@@effullname{\@@effilepref\expandedfigurename}%
   \else
     \doifelse{#2}\s!figurepreset
       {\def\figureextension##1{\@@effiletype}%
        \edef\@@effullname{\@@effilepref\expandedfigurename}}%
       {\ifx\@@effiletype\empty
          \dogetcommacommandelement1\from\@@eftype\to\commalistelement
          \edef\@@effullname{\@@effilename.\commalistelement}%
          \def\figureextension##1{##1}%
        \else
          \@EA\doifnumberelse\@EA{\@@effiletype}
            {\let\@@eftype\c!mps}\donothing
          \edef\@@effullname{\@@effilename.\@@effiletype}%
          \def\figureextension##1{\@@effiletype}%
        \fi}%
% begin prelocate to be converted image
\let\@@efconversionpath\empty
\let\@@efconversionfile\@@effilename
\let\@@efconversiontype\@@effiletype
\doifmodeelse{\systemmodeprefix\v!first}
  {\doifelsenothing\@@efconversion\donefalse\donetrue}%
  {\donefalse}%
\ifdone
   \chardef\figurestatus\zerocount
   \locatefigureconversionfile
   \chardef\figurestatus\zerocount
\fi
% end
     \doifelse\@@efpreset\v!no
       {\doifelse\@@eftype\s!unknown
          {\chardef\figurestatus\zerocount
           \let\@@eftype\figuretypes
           \locatepresetfigurefiles}
          {\chardef\figurestatus\plusone}}
       {\doifelse\@@eftype\s!unknown
          {\let\@@eftype\figuretypes}
          {\@EA\removefromcommalist\@EA{\@@eftype}\figuretypes
           \edef\@@eftype{\ifx\@@eftype\empty\else\@@eftype,\fi\figuretypes}}%
        \ifx\@@effiletype\empty\else
          \ExpandBothAfter\doifinsetelse\@@effiletype\@@eftype
            {\@EA\removefromcommalist\@EA{\@@effiletype}\@@eftype
             \edef\@@eftype{\@@effiletype,\@@eftype}}%
            \donothing
        \fi
        \doifelse{#2}\s!figurepreset
          {\chardef\figurestatus\plusfour
           \assignfullfilename\f!currentpath\@@exfile\to\@@efloadname
           \let\@@eftype\@@epe}
          {\chardef\figurestatus\zerocount
           \ifconditional\externalfigureflush
             \analyzefigurefiles
           \fi}}%
     \let\@@epe\@@eftype
     \edef\@@effiletype{\figureextension{\@@eftype}}% dirty trick
     \global\figwid\zeropoint \figxsca\plusone
     \global\fighei\zeropoint \figysca\plusone
     \doif\v!frame\@@exoption
       {\let\@@efframe\v!on}%
   \fi
   \ifcase\figurestatus
     \let\@@efframe\v!on
     \let\@@efobject\v!no
     \showmessage\m!figures1\@@effilename
   \or
     \showmessage\m!figures2\@@effullname
   \or
     \showmessage\m!figures3{\@@effullname,\@@eflenttype}%
   \or
     \showmessage\m!figures4\@@effullname
   \or
     \showmessage\m!figures5{\@@effullname,\@@efloadname}%
   \or % no message
     \doifnot\@@efsymbol\v!yes{\showmessage\m!figures8\@@effullname}%
   \fi
   \ifdim\@@epw=\zeropoint \chardef\figurestatus\plusone \fi
   \ifdim\@@eph=\zeropoint \chardef\figurestatus\plusone \fi
   \global\gridfigurefalse
   \ifnum\figurestatus=1 % unknown dimensions, take width or height or scale
     \setnaturalfiguresize
     \xdef\naturalfigurewidth {\the\figwid}%
     \xdef\naturalfigureheight{\the\fighei}%
     \let\@@efframe\v!off
   \else
     \global\let\naturalfigurewidth\@@epw
     \global\let\naturalfigureheight\@@eph
     \setfactorfiguresize
     \setscalefiguresize
     \setdimensionfiguresize
     \processaction
       [\@@efgrid]
       [  \v!yes=>\getnoflines\fighei
                  \edef\@@efheight{\the\noflines\lineheight}%
                  \global\gridfiguretrue,
       \v!height=>\getrawnoflines\fighei
                  \scratchdimen\noflines\lineheight
                  \advance\scratchdimen\strutdepth
                  \edef\@@efheight{\the\scratchdimen}%
                  \global\gridfiguretrue,
        \v!depth=>\getrawnoflines\fighei
                  \scratchdimen\noflines\lineheight
                  \advance\scratchdimen-\strutdepth
                  \edef\@@efheight{\the\scratchdimen}%
                  \global\gridfiguretrue,
     \v!halfline=>\getrawnoflines\fighei
                  \scratchdimen\noflines\lineheight
                  \advance\scratchdimen+.5\lineheight
                  \edef\@@efheight{\the\scratchdimen}%
                  \global\gridfiguretrue,
          \v!fit=>\getrawnoflines\fighei
                  \edef\@@efheight{\the\noflines\lineheight}%
                  \global\gridfiguretrue]%
     \ifgridfigure
       \setfactorfiguresize
       \setscalefiguresize
       \setdimensionfiguresize
     \fi
   \fi
   \convertfigureinsertscale\@@epx\figx\figxsca\scax
   \convertfigureinsertscale\@@epy\figy\figysca\scay
   \iftraceexternalfigures
     \message
       {\externalfigurelog
          [\@@effullname:
           t={\@@eftype}\space m={\@@efmethod}\space l=\@@eflabel\space
           w=\number\figwid\space h=\number\fighei\space
           \c!sx=\scax\space\c!sy=\scay\space
           ox=\figx\space oy=\figy]}%
   \fi
   \doif\v!empty\@@exoption
     {\skipexternalfigurestrue
      \let\@@efframe\v!off}% ? ?
   \doifelsenothing\@@efpage % NIEUW ??
     {\let\@@efoptions\empty}
     {\let\@@efoptions\@@efpage}%
   \doif\@@efpreview \v!yes{\addtocommalist\v!preview \@@efoptions}%
   \doif\@@efcontrols\v!yes{\addtocommalist\v!controls\@@efoptions}%
   \doif\@@efrepeat  \v!yes{\addtocommalist\v!repeat  \@@efoptions}%
   \doif\@@eftype\c!mps
     {\ifcase\EPSspecial\else\ifinobject\else
        \@@eftrace{special mps, object forced}%
        \doglobal\increment\forcedMPSobject
        \edef\@@efobjectname{\c!mps::\forcedMPSobject}%
        \let\@@efobject\v!yes
      \fi\fi}%
   \global\let\lastfigureobjectname\@@efobjectname
   \doifelse\@@efobject\v!no
     \donefalse
     {\doifobjectssupportedelse\donetrue\donefalse}%
   % this (for the moment) conveniently maps onto pdf which saves mapping
   \ifdone
     \doifobjectfoundelse{FIG}\@@efobjectname
       \donothing
       {\bgroup % to be cleaned up
        \figwid\@@epw % local ?
        \fighei\@@eph % local ?
        \scratchdimen\@@epx\scratchdimen-\scratchdimen
        \edef\@@epx{\the\scratchdimen}%
        \scratchdimen\@@epy\scratchdimen-\scratchdimen
        \edef\@@epy{\the\scratchdimen}%
       %\scratchdimen\@@epw\edef\@@epw{\the\scratchdimen}%
       %\scratchdimen\@@eph\edef\@@eph{\the\scratchdimen}%
        \setbox0\vbox to \fighei
          {\vfill
           \ifdim\wd\foundexternalfigure=\zeropoint
             \doinsertfile
               {\@@eftype,\@@efmethod}{\@@effullname,\@@eflabel}
               {100}{100}\@@epx\@@epy\@@epw\@@eph\@@efoptions
           \else\ifskipexternalfigures
             \ruledhbox
               {\backgroundline
                  [\@@efsplitcolor]{\fakebox\foundexternalfigure}}%
           \else
             \box\foundexternalfigure
           \fi\fi}%
        \wd0=\figwid
        \setobject{FIG}\@@efobjectname\vbox{\box0}%
        \setxvalue{\@@efobjectname\c!n}{\number\nofinsertpages}%
        \egroup}%
   \fi
   \xdef\figurewidth {\the\figwid}%
   \xdef\figureheight{\the\fighei}%
   \global\setbox\foundexternalfigure\vbox to \fighei
     {\vfill
      \hsize\figwid
      \ifdone
        \scratchdimen\scax\points\divide\scratchdimen \plushundred
        \edef\scax{\withoutpt\the\scratchdimen}%
        \scratchdimen\scay\points\divide\scratchdimen \plushundred
        \edef\scay{\withoutpt\the\scratchdimen}%
       %\scale[\c!sx=\scax,\c!sy=\scay]{\dowithfigure{\getobject{FIG}{\@@efobjectname}}}%
        \doscalenextbox\scax\scay\figwid\fighei{\dowithfigure{\getobject{FIG}{\@@efobjectname}}}%
        \xdef\noffigurepages{\number\getvalue{\@@efobjectname\c!n}}%
      \else\ifdim\wd\foundexternalfigure=\zeropoint
        \dowithfigure
          {\doinsertfile
             {\@@eftype,\@@efmethod}{\@@effullname,\@@eflabel}
             \scax\scay\figx\figy\figwid\fighei\@@efoptions}%
        \xdef\noffigurepages{\number\nofinsertpages}%
      \else
        \scratchdimen\scax\points\divide\scratchdimen \plushundred
        \edef\scax{\withoutpt\the\scratchdimen}%
        \scratchdimen\scay\points\divide\scratchdimen \plushundred
        \edef\scay{\withoutpt\the\scratchdimen}%
       %\scale[\c!sx=\scax,\c!sy=\scay]{\dowithfigure{\box\foundexternalfigure}}%
        \doscalenextbox\scax\scay\figwid\fighei{\dowithfigure{\box\foundexternalfigure}}%
        \xdef\noffigurepages{\number\nofinsertpages}%
      \fi\fi
      \global\let\appliedfigurexscale\scax
      \global\let\appliedfigureyscale\scay}%
\doresetobjects % clean up driver left overs
   \global\wd\foundexternalfigure\figwid
   \finalizeexternalfigure{#2}{\expandedfigurename}}

\def\doscalenextbox#1#2#3#4% for the moment here, faster
  {\bgroup
   \dowithnextbox
     {\dontshowcomposition
      \setbox\nextbox\hbox
        {\smashbox\nextbox
         \dostartscaling#1#2\flushnextbox\dostopscaling}%
      \nextboxwd#3%
      \nextboxht#4%
      \nextboxdp\zeropoint
      \flushnextbox
      \egroup}
   \hbox}

\let\figurelabel   \empty
\let\figurefilename\empty
\let\figurefiletype\empty
\let\figurefilepage\empty
\let\figurefilepath\empty

\def\finalizeexternalfigure#1#2%
  {\globalpushmacro\figurewidth
   \globalpushmacro\figureheight
   \globalpushmacro\figurelabel
   \globalpushmacro\figurefilename
   \globalpushmacro\figurefiletype
   \globalpushmacro\figurefilepage
   \globalpushmacro\figurefileconversion
   \globalpushmacro\figurefileprefix
   \globalpushmacro\figurefilepath
   \globalpushmacro\figurefilecache
   \xdef\figurewidth         {\the\figwid}%
   \xdef\figureheight        {\the\fighei}%
   \xdef\figurelabel         {#1}%
   \xdef\figurefilename      {#2}%
   \xdef\figurefiletype      {\@@eftype}%
   \xdef\figurefilepage      {\@@efpage}%
   \xdef\figurefileconversion{\@@efconversion}%
   \xdef\figurefileprefix    {\@@efprefix}%
   \xdef\figurefilepath      {\@@efcurrentpath}%
   \xdef\figurefilecache     {\@@efcache}%
   \doifmodeelse{\systemmodeprefix\v!first}
     {\let\figurefullname\figurefilename}%
     {\def\figurefullname{\figurefileprefix\figurefilename}}% no \edef
   \global\setbox\foundexternalfigure\vbox
     {\forgetall
      \ifcase\figurestatus
        \resetsystemmode\v!figure
        \let\figurefiletype\empty
      \else
        \setsystemmode  \v!figure % beter resource
      \fi
% begin force convertable image into file
\doifmode{\systemmodeprefix\v!first}{\doifsomething\@@efconversion
  {\let\figurefilepath\@@efconversionpath
   \let\figurefilename\@@efconversionfile
   \let\figurefiletype\@@efconversiontype}}%
% end
\ifconditional\externalfigureflush
      \ifconditional\externalfigurelevel % probably background
        \ifskipexternalfigures
          % nothing
          \fakebox\foundexternalfigure
        \else\ifcase\figurestatus
          % nothing
        \else\ifnum\splitexternalfigure=2\else
          \the\externalfigurepostprocessors
          \box\foundexternalfigure
        \fi\fi\fi
      \else
        \feedbackexternalfigure
        \settrue\externalfigurelevel
        \ifskipexternalfigures
          \externalfigurereplacement\figurelabel\figurefullname{skipped}%
        \else\ifcase\figurestatus
          \externalfigurereplacement\figurelabel\figurefullname{unknown}%
        \else\ifnum\splitexternalfigure=2
          \backgroundline[\@@efsplitcolor]
            {\fakebox\foundexternalfigure}%
        \else
          \the\externalfigurepostprocessors
          \doifelse\@@efreset\v!yes
            {\ht\foundexternalfigure\figureheight
             \dp\foundexternalfigure\zeropoint
             \wd\foundexternalfigure\figurewidth
             \box\foundexternalfigure}
            {\localframed % should also be applied to high res !
               [\??ef]
               [\c!offset=\v!overlay,
                \c!width=\figurewidth,
                \c!height=\figureheight]
               {\vfilll
                \ifnum\splitexternalfigure=1
                  % hm, eigenlijk in dit geval achtergrondkleur
                  \hidesplitcolorfalse % really needed
                  \backgroundline[\@@efsplitcolor]
                    {\box\foundexternalfigure}%
                \else % = 0, no split mode
                  \box\foundexternalfigure
                \fi}}%
        \fi\fi\fi
      \fi
\else
  \feedbackexternalfigure % maybe also; \the\externalfigurepostprocessors
\fi}%
   \globalpopmacro\figurefilecache
   \globalpopmacro\figurefilepath
   \globalpopmacro\figurefileprefix
   \globalpopmacro\figurefileconversion
   \globalpopmacro\figurefilepage
   \globalpopmacro\figurefiletype
   \globalpopmacro\figurefilename
   \globalpopmacro\figurelabel
   \globalpopmacro\figureheight
   \globalpopmacro\figurewidth}

\def\externalfigurereplacement#1#2#3%
  {\setupcolors
     [\c!state=\v!local]%
   \expanded{\localframed
     [\??ef]
     [\c!width=\figurewidth,
      \c!height=\figureheight,
      \c!background=\v!screen,
      \c!backgroundscreen=.8,
     %\c!frame=\ifincolor\v!off\else\v!on\fi]
     %\c!frame=\ifincolor\@@efframe\else\v!on\fi]}%
      \c!frame=\@@efframe]}%
     {\tt\tfxx \nohyphens
        name:  \expanded{\verbatimstring{#1}}\\%
        file:  \expanded{\verbatimstring{#2}}\\%
        state: \expanded{\verbatimstring{#3}}}}

\def\externalfigureplaceholder#1#2#3%
  {\localframed[\??ef]
     [\c!width=#2,\c!height=#3,\c!frame=\v!on]%
     {\tt\tfxx \nohyphens
        name:  \expanded{\verbatimstring{#1}}\\%
        state: \expanded{\verbatimstring{placeholder}}}}

\def\registerexternalfigure % no placement, handy for preprocessing
  {\dotripleempty\doregisterexternalfigure}

\def\doregisterexternalfigure[#1][#2][#3]%
  {\bgroup
   \setfalse\externalfigureflush
   \externalfigure[#1][#2][#3]% or \doexternalfigure
   \egroup}

\newtoks\externalfigureresets
\newtoks\externalfigurepostprocessors

\let\feedbackexternalfigure\relax % \gobblefourarguments

\gdef\appliedfigurexscale{1}
\gdef\appliedfigureyscale{1}

% will go internal

\def\appliedfigurefilename  {\@@effilename}
\def\appliedfigurefilepath  {\@@efcurrentpath}
\def\appliedfigureshortname {\@@efcurrentpath/\@@effilename}
\def\appliedfigurefullname  {\@@efcurrentfile}
\def\appliedfigureidentifier{\@@efobjectname}
\def\appliedfigureoptions   {\@@efoption}
\def\appliedfigurefilesuffix{\figureextension{\@@efcurrenttype}}

%D In \PDF\ one can specify an alternative graphic. This means
%D that for instance a low resolution graphic can be used for
%D viewing and a high res one for printing. Because this
%D feature depends much on the driver, here we only take care
%D of perparations. It is up to the special driver to handle
%D the inclusion. The driver routines can change the content of
%D box \type {\foundexternalfigure} if suitable.
%D
%D One complication is for instance that an alternative may
%D not itself have an alternative, and these kind of situations
%D are best handled by the driver.

\let\lastfigureobjectname\empty

%D The next macro does not work well with figure bases yet.

\def\calculateexternalscreenfigure[#1][#2][#3][#4][#5][#6]%
  {\ifx\@@efdisplay\empty\else
     \doifnot\@@efobject\v!no
       {\doifobjectssupportedelse
          {\doifspecialavailableelse\doregisterfigure
             {\bgroup
              #1[#4][#5][#6]%
              \doregisterfigure{FIG}{\lastfigureobjectname}%
              \let\@@ef@@scherm\@@efdisplay
              \calculateexternalfigure[#1][\@@ef@@scherm][\@@ef@@scherm][#4,\c!display=][#5][#6]%
              \egroup}
             {}}
          {}}%
   \fi}

\let\dowithfigure\relax

\let\naturalfigureheight=\!!zeropoint
\let\naturalfigurewidth =\!!zeropoint
\let\figureheight       =\!!zeropoint
\let\figurewidth        =\!!zeropoint

\def\noffigurepages{\nofinsertpages}

\def\getfiguredimensions
  {\dodoubleempty\dogetfiguredimensions}

\def\dogetfiguredimensions[#1][#2]%
  {{\let\immediate\relax % very dirty but prevents flushing, will change
    \setbox0\hbox{\externalfigure[#1][#2,\c!display=,\c!object=\v!no]}}}

% use the next one when the object must be forgotten (xobj
% nums can migrate to the next object; maybe it should
% always be done; todo ....

\def\getfiguredimensionsonly
  {\dodoubleempty\dogetfiguredimensionsonly}

\def\dogetfiguredimensionsonly[#1][#2]%
  {\dogetfiguredimensions[#1][#2]%
   \doresetobjects}

\presetlocalframed[\??ef]

\newconditional\externalfigurelevel % true=background false=normal
\newconditional\externalfigureflush % true=place false=ignore

\setfalse\externalfigurelevel
\settrue \externalfigureflush

\def\doplaceexternalfigure % used direct and indirect
  {\dosixtupleempty\dodoplaceexternalfigure}

\def\dodoplaceexternalfigure[#1][#2][#3][#4][#5][#6]%
  {\doifsomething{#3}% catches \defineexternalfigure dummies
     {\bgroup
\pushmacro\textunderscore
      \edef\textunderscore{\string_}% brrr, temp hack, still needed?
      \calculateexternalfigure[#1][#2][#3][#4][#5][#6]%
      \calculateexternalscreenfigure[#1][#2][#3][#4][#5][#6]%
\popmacro\textunderscore
      \box\foundexternalfigure
      \egroup}}

% new: more convenient/efficient than
%
%   \use..[a][a][setting] \externalfigure[b][a]
%
% is equivalent to:
%
%   \def..[a][setting]    \externalfigure[b][a]
%
% see x-res modules for usage:
%
% \defineexternalfigure[name][settings]

\def\dodefineexternalfigure[#1][#2]%
  {\setvalue{\??ef\??ef#1}%
     {\doplaceexternalfigure[\dopresetfigure][#1][][#2][]}}

\def\defineexternalfigure
  {\dodoubleargument\dodefineexternalfigure}

\def\getexternalfigure#1%
  {\getvalue{\??ef\??ef#1}}

\def\dopresetfigure[#1][#2][#3]%
  {\getparameters[\??ef][#1,#3]%
   \getparameters[\??ep][#2]}

\def\doprecopfigure[#1][#2][#3]%
  {\def\doplaceexternalfigure[##1][##2][##3][##4][##5]%
     {\getparameters[\??ef][##4,#2,#3]%
      \getparameters[\??ep][##5]}%
   \getvalue{\??ef\??ef#1}}

% \useexternalfigure[alpha][koe]
% \useexternalfigure[beta] [koe]       [breedte=1cm]
% \useexternalfigure[gamma][koe][alpha]
% \useexternalfigure[delta][koe][alpha][breedte=2cm]
%
% volle breedte: \externalfigure[koe]                 \par
% 3cm breed:     \externalfigure[koe]  [breedte=3cm]  \par
% volle breedte: \externalfigure[alpha]               \par
% 1cm breed:     \externalfigure[beta]                \par
% volle breedte: \externalfigure[gamma]               \par
% 2cm breed:     \externalfigure[delta]               \par
% 4cm breed:     \externalfigure[beta] [breedte=4cm]  \par
% 5cm breed:     \externalfigure[gamma][breedte=5cm]  \par

\def\dosetuseexternalfigure[#1][#2][#3][#4]%
  {\doifassignmentelse{#3}
     {\setvalue{\??ef\??ef#1}{\doplaceexternalfigure[\dopresetfigure][#1][#2][#3][#4]}}
     {\doifelsenothing{#3} % catch [1][2][leeg][leeg]
        {\setvalue{\??ef\??ef#1}{\doplaceexternalfigure[\dopresetfigure][#1][#2][#3][#4]}}
        {\setvalue{\??ef\??ef#1}{\doplaceexternalfigure[\doprecopfigure][#1][#2][#3][#4]}}}%
   \doifundefined{#1}{\setvalue{#1}{\getexternalfigure{#1}}}} % upward compatible

\def\douseexternalfigure[#1][#2]%[#3][#4]%
  {\doifelsenothing{#1}
     {\doifsomething{#2}
        {\dosetuseexternalfigure[#2][#2]}}  %[#3][#4]}}
     {\doifelsenothing{#2}
        {\dosetuseexternalfigure[#1][#1]}   %[#3][#4]}
        {\dosetuseexternalfigure[#1][#2]}}} %[#3][#4]}}}

\def\useexternalfigure
  {\doquadrupleempty\douseexternalfigure}

\unexpanded\def\externalfigure
  {\dotripleempty\doexternalfigure}

\def\doexternalfigure[#1][#2][#3]%
  {\bgroup
   \doifelsenothing{#1}
     {\framed[\c!width=8\lineheight,\c!height=6\lineheight]{external\\figure}}
     {\doifundefinedelse{\??ef\??ef#1}
        {\useexternalfigure[\s!dummy][#1][#2][#3]%
         \getexternalfigure{\s!dummy}[#3]}
        {\getexternalfigure{#1}[#2]}}%
   \globallet\currentresourcecomment\empty
   \egroup}

\long\def\resourcecomment#1%
  {\long\gdef\currentresourcecomment{#1}}

\long\def\startresourcecomment#1\stopresourcecomment
  {\long\gdef\currentresourcecomment{#1}}

\let\currentresourcecomment\empty

%D Two alternatives, more settings needed.

\def\showexternalfigurea
  {\bgroup
   \dontcomplain
   \def\presetfigure[##1][##2]%
     {\getvalue{\e!start\v!figure\e!text}[\v!left,\v!none][]
        {}
        {\hbox
           {\externalfigure[##1][\c!frame=\v!on,\c!width=6cm,\c!size=\@@exsize][##2]%
            \tfskip
            \framed[\c!width=\figurewidth,\c!height=\figureheight]{}}}%
        {\tt\tfa\expanded{\asciistr{##1}}}%
        \blank
        \tfx
        \def\docommando####1%
          {\beforesplitstring####1\at=\to\asciia
           \aftersplitstring ####1\at=\to\asciib
           \convertcommand\asciib\to\asciib
           \doifsomething\asciib
             {\hsmash{\hbox to .75em{\asciia\hss}: \asciib}\endgraf}}%
        \processcommalist[##2]\docommando
        \strut
        \endgraf
      \getvalue{\e!stop\v!figure\e!text}}%
   \pushendofline
   \readjobfile\@@exfile\donothing\donothing
   \popendofline
   \egroup}

\def\showexternalfigureb % instelbaar maken
  {\bgroup
   \def\total{5}%
   \globalletempty\allfigures
   \doglobal\newcounter\figurecounter
   \dontcomplain
   \def\docommando##1{##1&}%
   \def\figurecaptions%
     {\crcr
      \noalign{\nobreak\vskip.5em}%
      \@EA\globalprocesscommalist\@EA[\allfigures]\docommando
      \globalletempty\allfigures
      \crcr
      \noalign{\vskip1em\goodbreak}}%
   \def\presetfigure[##1][##2]%
     {\vbox
        {\divide\hsize \total
         \advance\hsize -1em
         \externalfigure
           [##1]
           [\c!frame=\v!on,\c!factor=\v!max,\c!width=\hsize,\c!size=\@@exsize][##2]}%
      \doglobal\addtocommalist{##1}\allfigures
     %\getvalue{\s!figurepreset}%
      \doglobal\increment\figurecounter
      \ifnum\figurecounter=\total
        \doglobal\newcounter\figurecounter
        \def\next{\figurecaptions}%
      \else
        \def\next{&}%
      \fi
      \next}%
   \pushendofline
   \tabskip\zeropoint \!!plus 1fill
   \halign to \hsize
     {&\hss##\hss\cr\readjobfile\@@exfile\donothing\donothing\crcr
      \figurecaptions}
   \popendofline
   \egroup}

\def\showexternalfigurec
  {\bgroup
   \def\presetfigure[##1][##2]{\expanded{\pagefigure[##1][\c!size=\@@exsize]}}% else loop
   \pushendofline
   \readjobfile\@@exfile\donothing\donothing
   \popendofline
   \egroup}

\def\showexternalfigured % to be documented; this one builds a dimension file for
  {\bgroup               % metapost/metafun
   \immediate\openout\scratchwrite=mpfigs.mp
   \def\presetfigure[##1][##2]%
     {\getfiguredimensionsonly[##1]% \pagefigure[##1]%
      \immediate\write\scratchwrite
         {registerfigure("##1",\figurewidth,\figureheight);}}
   \pushendofline
   \readjobfile\@@exfile\donothing\donothing
   \popendofline
   \immediate\closeout\scratchwrite
   \egroup}

\def\startpagefigure
  {\dodoubleempty\dostartpagefigure}

\def\dostartpagefigure[#1][#2]%
  {\bgroup
   \getparameters[\??ex][\c!offset=\v!overlay,#2]%
   \startTEXpage[\c!offset=\@@exoffset]%
     \externalfigure[#1][#2]\ignorespaces} % so we can put some text below the graphic

\def\stoppagefigure
  {\stopTEXpage
   \egroup}

\def\pagefigure
  {\dodoubleempty\dopagefigure}

\def\dopagefigure[#1][#2]%
  {\dostartpagefigure[#1][#2]\stoppagefigure}

% \starttext \pagefigure[two.1] \stoptext

\def\doshowexternalfigures[#1]%
  {\bgroup
   \setupcolors[\c!state=\v!start]% to prevent mps color conversion
   \getparameters[\??ex][\c!alternative=a,\c!offset=\!!zeropoint,\c!size=,#1]%
   \getvalue{\strippedcsname\showexternalfigure\@@exalternative}%
   \egroup}

\def\showexternalfigures
  {\dosingleempty\doshowexternalfigures}

\def\overlayfigure#1%
  {\externalfigure[#1][\c!width=\overlaywidth,\c!height=\overlayheight]}

%D Still undocumented!

\newcount\efreference
\newdimen\efxsteps
\newdimen\efysteps

\def\calculateefsteps
  {\ifnum0\@@exxmax=\zerocount
     \ifnum0\@@exymax=\zerocount
       \def\@@exymax{24}%
     \fi
     \efysteps\figureheight \divide\efysteps \@@exymax
     \efxsteps\efysteps
     \dimen0=\figurewidth
     \advance\dimen0 \efysteps
     \divide \dimen0 \efysteps
     \edef\@@exxmax{\number\dimen0}%
   \else
     \efxsteps\figurewidth  \divide\efxsteps \@@exxmax
     \efysteps\figureheight \divide\efysteps \@@exymax
   \fi}

\def\efcomment#1(#2,#3)#4(#5,#6)%    {kader}(x,y)(h,b)[...]{tekst}
  {\def\complexefdocomment[##1]##2%
     {\position(#2,#3)%
        {\setnostrut
         \framed
           [\c!width=#5\efxsteps,
            \c!height=#6\exysteps,
            \c!offset=\v!none,
            \c!frame=#1,
            ##1]%
           {##2}}}%
   \complexorsimpleempty\efdocomment}

\def\efnocomment(#1,#2)#3(#4,#5)%    (x,y)(h,b)[...]{tekst}
  {\def\complexefdonocomment[##1]##2{}%
   \complexorsimpleempty\efdonocomment}

\def\efdomarker(#1,#2)#3#4%    (h,b){kader}{tekst}
  {\framed
     [\c!width=#1\efxsteps,
      \c!height=#2\efysteps,
      \c!offset=\v!none,
      \c!frame=#3]%
     {#4}}

\def\effigure#1%
  {\position(0,0){\getvalue{#1}}}

\def\efdoarea(#1,#2)#3#4%    (h,b){kader}{tekst}
  {\bgroup
   \setnostrut
   \framed
     [\c!width=#1\efxsteps,
      \c!height=#2\efysteps,
      \c!offset=\!!zeropoint,
      \c!frame=#3]
     {#4}%
   \egroup}

\def\efgoto(#1,#2)#3[#4]%    (h,b)kader[ref]
  {\setbox0=\vbox{\efdoarea(#1,#2)#3{}}%
   \gotobox{\copy0}[#4]}

\def\efmark(#1,#2)#3(#4,#5)#6[#7]%
  {\advance\efreference \plusone
   \position(#1,#2)
     {\hbox{\the\efreference}}%
   \position(#1,#2)
     {\gotosomeinternal\s!vwb{#7}\realfolio
        {\efdomarker(#4,#5)\v!on{\thisissomeinternal\s!vwa{#7}}}}}

\def\eftext#1(#2,#3)#4(#5,#6)#7[#8]%
  {\advance\efreference \plusone
   \hbox
     {\quad
      \thisissomeinternal\s!vwb{#8}%
      \gotosomeinternal  \s!vwa{#8}\realfolio
        {\hbox to 1.5em{\the\efreference\presetgoto\hfill}}%
      \quad#1 (#2,#3) (#5,#6) [#8]\hfill}%
   \endgraf}

\def\efthisis(#1,#2)#3[#4]%
  {\efdoarea(#1,#2){#3}{\pagereference[#4]}}

\newbox\colorbarbox

\def\makecolorbar[#1]%
  {\def\docommando##1%
     {\color[##1]
        {\blackrule
           [\c!width=2em,
            \c!height=1ex,
            \c!depth=\!!zeropoint]}%
      \endgraf}%
   \global\setbox\colorbarbox\vbox
     {\forgetall
      \processcommalist[#1]\docommando}%
   \global\setbox\colorbarbox\vbox
     {\hskip2em\box\colorbarbox}%
   \global\wd\colorbarbox\zeropoint}

\def\placestartfigure[#1][#2][#3]#4\placestopfigure[#5]%
  {\hbox
     {\setbox0\hbox
        {\useexternalfigure[\s!dummy][#2][#3,#5]%
         \externalfigure[\s!dummy]}%
      \calculateefsteps
      \startpositioning
        \def\referring(##1,##2)##3(##4,##5)##6[##7]%
          {\position(##1,##2)
             {\efgoto(##4,##5){\@@exframes}[##7]}}%
        \def\marking(##1,##2)##3(##4,##5)##6[##7]%
          {\position(##1,##2)
             {\efthisis(##4,##5){\@@exframes}[##7]}}%
        \def\remark%
          {\efnocomment}%
        \def\colorbar##1[##2]%
          {}%
        \position(0,0){\box0}%
        \linewidth\onepoint
        \setuppositioning
          [\c!unit=pt,
           \c!xscale=\withoutpt\the\efxsteps,
           \c!yscale=\withoutpt\the\efysteps,
           \c!factor=1]%
        \ignorespaces#4%
        \def\referring(##1,##2)##3(##4,##5)##6[##7]%
          {}%
        \let\marking\referring
        \def\remark%
          {\efcomment\v!no}%
        \def\colorbar##1[##2]%
          {\makecolorbar[##2]}%
        \ignorespaces#4%
      \stoppositioning
      \box\colorbarbox}}

\def\dodostartfigure[#1][#2][#3]#4\stopfigure
  {\doifelse\v!test\@@exoption
     {\teststartfigure[#1][#2][#3]#4\teststopfigure
      \let\@@exframes\v!on}
     {\let\@@exframes\v!off}%
   \setvalue{\??ef\??ef#1}%
     {\dosingleempty{\placestartfigure[#1][#2][#3]#4\placestopfigure}}%
   \doifundefined{#1}{\setvalue{#1}{\getexternalfigure{#1}}}}

% De onderstaande macro mag niet zondermeer worden aangepast
% en is afgestemd op gebruik in de handleiding.

\def\teststartfigure[#1][#2][#3]#4\teststopfigure%
  {\begingroup
     \setbox0\hbox
       {\useexternalfigure[\s!dummy][#2][\c!wfactor=\v!max]%
        \externalfigure[\s!dummy]}%
     \def\referring
       {\efmark}%
     \def\marking
       {\efmark}%
     \def\remark
       {\efcomment\v!yes}%
     \def\colorbar##1[##2]%
       {}%
     \efreference\zerocount
     \setbox0\vbox
       {\hsize240pt
        \startpositioning
          \calculateefsteps
          \position(0,0)
            {\box0}%
          \position(0,0)
            {\basegrid
               [\c!nx=\@@exxmax,
                \c!dx=\withoutpt\the\efxsteps,
                \c!ny=\@@exymax,
                \c!dy=\withoutpt\the\efysteps,
                \c!xstep=1,
                \c!ystep=1,
                \c!scale=1,
                \c!offset=\v!no,
                \c!unit=pt]}%
          \setuppositioning
            [\c!unit=pt,
             \c!xscale=\withoutpt\the\efxsteps,
             \c!yscale=\withoutpt\the\efysteps,
             \c!factor=1]%
          \linewidth\onepoint
          \ignorespaces#4\relax
        \stoppositioning
        \vfill}%
     \efreference\zerocount
     \def\referring%
       {\eftext{$\rightarrow$}}%
     \def\marking%
       {\eftext{$\leftarrow$}}%
     \def\remark%
       {\efnocomment}%
     \def\colorbar##1[##2]%
       {}%
     \setbox2\vbox
       {{\tfa\doifelsenothing{#1}{#2}{#1}}
        \blank
        \tfxx#4
        \vfilll}%
     \ifdim\ht0>\ht2
       \ht2\ht0
     \else
       \ht0\ht2
     \fi
     \hbox
       {\hskip3em
        \vtop{\vskip12pt\box0\vskip6pt}%
        \vtop{\vskip12pt\box2\vskip6pt}}%
   \endgroup}

\def\dodostartfigure[#1][#2][#3]#4\stopfigure
  {\doifelse\v!test\@@exoption
     {\teststartfigure[#1][#2][#3]#4\teststopfigure
      \let\@@exframes\v!on}
     {\let\@@exframes\v!off}%
   \setvalue{\??ef\??ef#1}%
     {\def\next{\placestartfigure[#1][#2][#3]#4\placestopfigure}%
      \dosingleempty\next}%
   \doifundefined{#1}{\setvalue{#1}{\getexternalfigure{#1}}}}

\long\def\dostartfigure#1%
  {\dotripleargument\dodostartfigure#1\stopfigure}

\def\startfigure
  {\grabuntil{\e!stop\v!figure}\dostartfigure}

%D \macros
%D   {clip, setupclipping}
%D
%D Although related to figures, clipping can be applied to
%D arbitrary content. We can use \METAPOST\ to provide a non
%D rectangular clipping path.
%D
%D \starttyping
%D \startMPclip{fun}
%D   clip currentpicture to fullcircle
%D     shifted (.5,.5) xscaled \width yscaled \height ;
%D \stopMPclip
%D \stoptyping
%D
%D We get a rectangular piece of the figure when we say:
%D
%D \starttyping
%D \clip[x=2,y=1]{\externalfigure[photo]}
%D \stoptyping
%D
%D When we want to clip to the oval we defined a few lines ago,
%D we say:
%D
%D \starttyping
%D \clip[nx=1,ny=1,x=1,y=1,mp=fun]{\externalfigure[photo]}
%D \stoptyping
%D
%D The general characteristics of clipping can be set up with
%D
%D \showsetup{setupclipping}

\def\doclip[#1]% nb top->bottom left->right
  {\bgroup
   \getparameters[\??cp][#1]%
   \doifelse\@@cpstate\v!start\dodoclip{\egroup\hbox}}

\def\dodoclip
  {\dowithnextbox
     {\ifdim\@@cpwidth>\zeropoint
        \!!dimena\@@cpwidth
        \!!dimenc\@@cphoffset
      \else
        \!!dimena\nextboxwd
        \divide\!!dimena \@@cpnx
        \!!dimenc\@@cpx\!!dimena
        \advance\!!dimenc -\!!dimena
        \!!dimena\@@cpsx\!!dimena
      \fi
      \relax % sure
      \ifdim\@@cpheight>\zeropoint
        \!!dimenb\@@cpheight
        \!!dimend\nextboxht
        \advance\!!dimend -\@@cpvoffset
        \advance\!!dimend -\!!dimenb
      \else
        \!!dimenb\nextboxht
        \divide\!!dimenb \@@cpny
        \!!dimend-\@@cpy\!!dimenb
        \advance\!!dimend -\@@cpsy\!!dimenb
        \advance\!!dimend \!!dimenb
        \!!dimenb\@@cpsy\!!dimenb
        \advance\!!dimend \nextboxht
      \fi
      \setbox\nextbox\hbox                            % old
        {\advance\!!dimenc -\@@cpleftoffset           % new !
         \advance\!!dimend -\@@cpbottomoffset         % new ! % - added
         \hskip-\!!dimenc\lower\!!dimend\flushnextbox}% old
      \nextboxwd\zeropoint
      \nextboxht\zeropoint
      \nextboxdp\zeropoint
      \setbox\nextbox\hbox
        {\advance\!!dimena \@@cpleftoffset         % new !
         \advance\!!dimena \@@cprightoffset        % new !
         \advance\!!dimenb \@@cpbottomoffset       % new !
         \advance\!!dimenb \@@cptopoffset          % new !
         \dostartclipping\@@cpmp\!!dimena\!!dimenb % old
           \flushnextbox
         \dostopclipping}%
      \setbox\nextbox\hbox                       % new !
        {\!!dimena-\@@cpleftoffset               % new !
         \!!dimenb \@@cpbottomoffset             % new ! % - removed
         \hskip\!!dimena\lower\!!dimenb\flushnextbox}% new !
      \nextboxwd\!!dimena
      \nextboxht\!!dimenb
      \nextboxdp\zeropoint
      \flushnextbox
      \egroup}%
   \hbox}

\def\clip{\dosingleempty\doclip}

\def\setupclipping
  {\dodoubleargument\getparameters[\??cp]}

%D defining sound tracks:
%D
%D \starttyping
%D \useexternalsoundtrack[label][file]
%D \stoptyping
%D
%D associated actions: StartSound StopSound PauseSound ResumeSound
%D
%D Todo: like external figures, also search on path,
%D although, they need to be present ar viewing time, so ...

\def\useexternalsoundtrack
  {\dodoubleargument\douseexternalsoundtrack}

\def\douseexternalsoundtrack[#1][#2]%
  {\setgvalue{\??sd:#1}{#2}}

\def\checksoundtrack#1%
  {\iflocation
     \doifdefined{\??sd:#1}{\doifvaluesomething{\??sd:#1}
       {\doinsertsoundtrack{\getvalue{\??sd:#1}}{#1}\@@sdoption
        % brr, \..empty not really needed and maybe even wrong;
        % also, not here but in driver
        % well, no: sounds need to be reinitialize each time (i.e., be on page), so no
        }}% \letgvalueempty{\??sd:#1}}}%
   \fi}

\setexecutecommandcheck {startsound} \checksoundtrack

\def\setupexternalsoundtracks
  {\dodoubleargument\getparameters[\??sd]}

\setupexternalsoundtracks
  [\c!option=]

%D NEW: used in styledesign manual

% \setbuffer[typeset-b]\endbuffer
% \setbuffer[typeset-a]\endbuffer
%
% todo:
%
% \appendtoks \setbuffer[typeset-b]\endbuffer\to \everystarttext
% \appendtoks \setbuffer[typeset-a]\endbuffer\to \everystarttext

\def\typesetbuffer
  {\dodoubleempty\dotypesetbuffer}

\newcounter\noftypesetbuffers % all loaded at the end

\defineexternalfigure
  [typeset]
  [\c!background=\v!color,
   \c!backgroundcolor=\s!white]

\def\dotypesetbuffer[#1][#2]% beware: this will mix up the mp graphics
  {\bgroup
   \def\TEXbufferfile##1{\bufferprefix##1.tex}%
   \expanded{\setbuffer[typeset]%
     \def\noexpand\bufferprefix{\ifprotectbuffers\jobname-\fi typeset-}}%
     \starttext
     \getbuffer[b,#1,a]%
     \stoptext
   \endbuffer
   \doglobal\increment\noftypesetbuffers
   % batch is needed
   \executesystemcommand{texmfstart texexec --batch --pdf --result=\bufferprefix typeset-\noftypesetbuffers\space \bufferprefix typeset.tex}%
   %\externalfigure[\bufferprefix typeset-\noftypesetbuffers.pdf][\c!object=\v!no,#2]%
   \externalfigure[\bufferprefix typeset-\noftypesetbuffers.pdf][#2]%
   \egroup}

% for me only (manuals and sucn)

\definesystemvariable{tz}

\def\definetypesetting{\dotripleempty\dodefinetypesetting}
\def\typesetfile      {\dotripleempty\dotypesetfile}

\def\dodefinetypesetting[#1][#2][#3]%
  {\doifsomething{#1}{\setvalue{\??tz#1}{\dodotypesetfile{#2}{#3}}}}

\def\dotypesetfile[#1][#2][#3]%
  {\executeifdefined{\??tz#1}\gobbletwoarguments{#2}{#3}}

\def\dodotypesetfile#1#2#3#4% args settings file settings
  {\doifmode{*\v!first}{\executesystemcommand{texmfstart texexec.pl --batch --pdf #1 #3}}%
   \doglobal\beforesplitstring#3\at.\to\typesetfilename
   \externalfigure[\typesetfilename.pdf][#2,#4]}

\appendtoks \setupexternalfigures[\c!option=\v!empty] \to \everyfastmode
\appendtoks \runMPgraphicsfalse                       \to \everyfastmode
\appendtoks \insertMPgraphicsfalse                    \to \everyfastmode

\appendtoks \flushMPgraphics \to \everygoodbye  % \everylastshipout

\setupexternalfigures
  [\c!option=,
   \c!object=\v!yes,
   \c!reset=\v!no,
   \c!maxwidth=\@@efwidth,
   \c!maxheight=\@@efheight,
   \c!bodyfont=\korpsgrootte,
   \c!directory=,
   \c!file=\f!utilityfilename.\f!figureextension,
   \c!radius=.5\korpsgrootte,
   \c!corner=\v!rectangular,
   \c!frame=\v!off,
   \c!background=, % new
   \c!splitcolor=\s!white,
   \c!conversion=,
   \c!prefix=,
   \c!cache=,
%  \c!grid=,
   \c!location={\v!local,\v!global}]

\setupexternalfigures
  [\c!frames=\v!off,
   \c!ymax=24,
   \c!xmax=]

\useexternalfigure
  [buffer] [\jobname] [\c!type=\v!buffer,\c!object=\v!no]

\setupclipping
  [\c!state=\v!start,
   \c!n=1, % was 2
   \c!nx=\@@cpn,\c!x=1,\c!sx=1,
   \c!ny=\@@cpn,\c!y=1,\c!sy=1,
   \c!width=\!!zeropoint,
   \c!height=\!!zeropoint,
   \c!hoffset=\!!zeropoint,
   \c!voffset=\!!zeropoint,
   \c!offset=\zeropoint,
   \c!leftoffset=\@@cpoffset,  % \zeropoint,
   \c!rightoffset=\@@cpoffset, % \zeropoint,
   \c!topoffset=\@@cpoffset,   % \zeropoint,
   \c!bottomoffset=\@@cpoffset,   % \zeropoint,
   \c!mp=]

\protect \endinput
