%D \module
%D   [       file=syst-etx,
%D        version=1999.03.17, % some time ...  
%D          title=\CONTEXT\ System Macros,
%D       subtitle=Efficient \PLAIN\ \TEX\ loading,
%D         author=Hans Hagen,
%D           date=\currentdate,
%D      copyright={PRAGMA / Hans Hagen \& Ton Otten}]
%C
%C This module is part of the \CONTEXT\ macro||package and is
%C therefore copyrighted by \PRAGMA. See mreadme.pdf for 
%C details. 

%D This module prepares \CONTEXT\ for \ETEX. We don't use 
%D the definition files that come with this useful \TEX\ 
%D extension, but implement our own alternatives. 

\unprotect

%D Constants to be used with \type {\grouptype}.

\chardef\@@bottomlevelgroup   =  0
\chardef\@@simplegroup        =  1
\chardef\@@hboxgroup          =  2
\chardef\@@adjustedhboxgroup  =  3
\chardef\@@vboxgroup          =  4
\chardef\@@vtopgroup          =  5
\chardef\@@aligngroup         =  6
\chardef\@@noaligngroup       =  7
\chardef\@@outputgroup        =  8
\chardef\@@mathgroup          =  9
\chardef\@@discretionarygroup = 10
\chardef\@@insertgroup        = 11
\chardef\@@vcentergroup       = 12
\chardef\@@mathchoicegroup    = 13
\chardef\@@semisimplegroup    = 14
\chardef\@@mathshiftgroup     = 15
\chardef\@@mathleftgroup      = 16

\chardef\@@vadjustgroup       = \@@insertgroup

%D Constants to be used with \type {\interactionmode}.

\chardef\@@batchmode     = 0
\chardef\@@nonstopmode   = 1
\chardef\@@scrollmode    = 2
\chardef\@@errorstopmode = 3

%D Constants to be used with \type {\lastnodetype}.

\chardef\@@charnode          =  0
\chardef\@@hlistnode         =  1
\chardef\@@vlistnode         =  2
\chardef\@@rulenode          =  3
\chardef\@@insertnode        =  4
\chardef\@@marknode          =  5
\chardef\@@adjustnode        =  6
\chardef\@@ligaturenode      =  7
\chardef\@@discretionarynode =  8
\chardef\@@whatsitnode       =  9
\chardef\@@mathnode          = 10
\chardef\@@gluenode          = 11
\chardef\@@kernnode          = 12
\chardef\@@penaltynode       = 13
\chardef\@@unsetnode         = 14
\chardef\@@mathsnode         = 15

%D Constants to be used with \type {\iftype}.

\chardef\@@charif     =  1
\chardef\@@catif      =  2
\chardef\@@numif      =  3
\chardef\@@dimif      =  4
\chardef\@@oddif      =  5
\chardef\@@vmodeif    =  6
\chardef\@@hmodeif    =  7
\chardef\@@mmodeif    =  8
\chardef\@@innerif    =  9
\chardef\@@voidif     = 10
\chardef\@@hboxif     = 11
\chardef\@@vboxif     = 12
\chardef\@@xif        = 13
\chardef\@@eofif      = 14
\chardef\@@trueif     = 15
\chardef\@@falseif    = 16
\chardef\@@caseif     = 17
\chardef\@@definedif  = 18
\chardef\@@csnameif   = 19
\chardef\@@fontcharif = 20

%D Just in case we are not using \ETEX, we define some out of
%D range constants.

\beginTEX

\chardef\grouptype       = 255
\chardef\interactionmode = 255
\chardef\nodetype        = 255
\chardef\iftype          = 255

\endTEX

%D Of course we want even bigger log files, so we copied this
%D from the \ETEX\ source files. 

\beginETEX \tracing... 

\def\tracingall
  {\tracingonline    \@ne
   \tracingcommands  \thr@@
   \tracingstats     \tw@
   \tracingpages     \@ne
   \tracingoutput    \@ne
   \tracinglostchars \tw@
   \tracingmacros    \tw@
   \tracingparagraphs\@ne
   \tracingrestores  \@ne
   \showboxbreadth   \maxdimen
   \showboxdepth     \maxdimen
   \tracinggroups    \@ne
   \tracingifs       \@ne
   \tracingscantokens\@ne
   \tracingnesting   \@ne
   \tracingassigns   \tw@
   \errorstopmode}

\def\loggingall
  {\tracingall 
   \tracingonline    \z@}

\def\tracingnone
  {\tracingassigns   \z@
   \tracingnesting   \z@
   \tracingscantokens\z@
   \tracingifs       \z@
   \tracinggroups    \z@
   \showboxdepth     \thr@@
   \showboxbreadth   5
   \tracingrestores  \z@
   \tracingparagraphs\z@
   \tracingmacros    \z@
   \tracinglostchars \@ne
   \tracingoutput    \z@
   \tracingpages     \z@
   \tracingstats     \z@
   \tracingcommands  \z@
   \tracingonline    \z@ }

\endETEX

%D Just to be sure: 

\ifx\eTeX\undefined

  \def\eTeX{$\varepsilon$-\TeX}

\fi

%D In \ETEX\ we have lots of registers, so we redefine a few 
%D low level macros. We reserve some extra space for inserts 
%D and as soon as we near the end of the first register 
%D memory bank (often some 10 less than 255), we switch to the 
%D slower range \type {\@@medallocation}||\type {\@@maxallocation}. 

\beginETEX \new...

%D First we redefine the plain \TEX\ register allocation macros.

\def\newcount   {\myalloc@0\count   \countdef   \@@maxallocation}
\def\newdimen   {\myalloc@1\dimen   \dimendef   \@@maxallocation}
\def\newskip    {\myalloc@2\skip    \skipdef    \@@maxallocation}
\def\newmuskip  {\myalloc@3\muskip  \muskipdef  \@@maxallocation}
\def\newbox     {\myalloc@4\box     \mathchardef\@@maxallocation}
\def\newtoks    {\myalloc@5\toks    \toksdef    \@@maxallocation}
\def\newread    {\myalloc@6\read    \chardef    \@@minallocation}
\def\newwrite   {\myalloc@7\write   \chardef    \@@minallocation}
\def\newmarks   {\myalloc@8\marks   \mathchardef\@@maxallocation}
\def\newlanguage{\myalloc@9\language\chardef    \@@minallocation}

%D Since in \CONTEXT\ we only have one math family left we 
%D redefine \type {\newfam}.

\def\newfam#1{\chardef#1=15 } 

%D Therefore we should reset the related counter. 

\count18=1 

%D We use some constants in the tests.

\mathchardef\@@minallocation =    16
\mathchardef\@@medallocation =   256
\mathchardef\@@maxallocation = 32767

%D I cannot imagine that more than~8 extra insert classes 
%D are needed, so we say:

\chardef\@@insallocation=8 

%D But, for critical editions, we may need many more, so 
%D here we go again:

\chardef\@@insallocation=24

%D My low level allocation macro now comes down to: 

\def\myalloc@#1#2#3#4#5%
  {\global\advance\count1#1by\@ne
   \ifnum\count1#1>\@@medallocation \else 
     \global\advance\insc@unt by -\@@insallocation
     \ifnum\count1#1<\insc@unt \else
       \global\count1#1=\@@medallocation % \wait
     \fi
     \global\advance\insc@unt by +\@@insallocation
   \fi
   \ifnum\count1#1>#4%
     \global\count1#1=#4%
     \errmessage{No room for (\string#2) \string#5}%
   \fi
   \allocationnumber=\count1#1%
   \global#3#5=\allocationnumber
   \wlog{\string#5=\string#2\the\allocationnumber}}

\endETEX

%D These macros can be checked by tests like:
%D
%D \starttypen 
%D \let\wlog\message \dorecurse{1000}{\newcount\dummy}
%D \stoptypen 

%D A few bonus bindings. 

\let\normalprotected  = \protected
\let\normalunexpanded = \unexpanded

\protect \endinput
