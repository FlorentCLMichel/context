%D \module
%D   [       file=core-buf,
%D        version=2000.01.05,
%D          title=\CONTEXT\ Core Macros,
%D       subtitle=Buffers and Blockmoves,
%D         author=Hans Hagen,
%D           date=\currentdate,
%D      copyright={PRAGMA / Hans Hagen \& Ton Otten}]
%C
%C This module is part of the \CONTEXT\ macro||package and is
%C therefore copyrighted by \PRAGMA. See mreadme.pdf for
%C details.

% investigate etex's \readline and \scantokens

\writestatus{loading}{Context Core Macros / Buffers and Blockmoves}

\startmessages  dutch  library: textblocks
  title: tekstblokken
      1: nieuwe versie, tweede run nodig
      2: wegschrijven blokken naar --
      3: inlezen blokken uit --
      4: er is een tweede run nodig
      5: -- niet verborgen
      6: -- verborgen en verwerkt
      7: -- verborgen
      8: -- gehandhaafd
      9: -- niet gehandhaafd
     10: -- geladen en verwerkt
     11: -- geladen en geplaatst
     12: -- overgeslagen
\stopmessages

\startmessages  english  library: textblocks
  title: textblocks
      1: new version, second pass needed
      2: writing blocks to --
      3: reading blocks from --
      4: second pass needed
      5: -- not hidden
      6: -- hidden and processed
      7: -- hidden
      8: -- typeset
      9: -- not typeset
     10: -- loaded and processed
     11: -- loaded and typeset
     12: -- skipped
\stopmessages

\startmessages  german  library: textblocks
  title: textblock
      1: neue Version, zweiter Durchlauf benoetigt
      2: schreibe Bloecke zu --
      3: lese Bloecke von --
      4: zweiter Durchlauf benoetigt
      5: -- nicht verborgen
      6: -- verborgen und verarbeitet
      7: -- verborgen
      8: -- gesetzt
      9: -- nicht gesetzt
     10: -- geladen und verarbeitet
     11: -- geladen und gesetzt
     12: -- ausgelassen
\stopmessages

\startmessages  czech  library: textblocks
  title: textovyblok
      1: nova verze, je treba druhy beh
      2: zapisuji bloky do --
      3: ctu bloky z --
      4: je treba druhy beh
      5: -- neni skryto
      6: -- skryto a zpracovano
      7: -- skryto
      8: -- vysazeno
      9: -- nevysazeno
     10: -- nacteno a zpracovano
     11: -- nacteno a vysazeno
     12: -- preskoceno
\stopmessages

\startmessages  italian  library: textblocks
  title: blocchi di testo
      1: nuova versione, seconda passata necessaria
      2: scrittura dei blocchi su --
      3: lettura dei blocchi da --
      4: seconda passata necessaria
      5: -- non nascosto
      6: -- nascosto ed elaborato
      7: -- nascosto
      8: -- composto
      9: -- non composto
     10: -- caricato ed elaborato
     11: -- caricato e composto
     12: -- saltato
\stopmessages

\startmessages  norwegian  library: textblocks
  title: tekstblokker
      1: ny versjon, andre gjennomkjøring nødvendig
      2: skriver blokker til --
      3: leser blokker fra --
      4: andre gjennomkjøring nødvendig
      5: -- ikke skjult
      6: -- skjult og behandlet
      7: -- skjult
      8: -- tegnsatt
      9: -- ikke tegnsatt
     10: -- lest inn og behandlet
     11: -- lest inn og tegnsatt
     12: -- utelatt
\stopmessages

\startmessages  romanian  library: textblocks
  title: blocuri de text
      1: o noua versiune, este nevoie de inca o trecere
      2: se scriu blocurile in --
      3: se citesc blocurile din --
      4: este nevoie de inca o trecere
      5: -- nu este ascuns
      6: -- ascuns si procesat
      7: -- ascuns
      8: -- cules
      9: -- nu este cules
     10: -- incarcat si procesat
     11: -- incarcat si cules
     12: -- sarit peste
\stopmessages

\startmessages  french  library: textblocks
  title: textblocks
      1: nouvelle version, une seconde passe est nécessaire
      2: ecriture des blocs vers --
      3: lecture des blocs en provenance de --
      4: seconde passe nécessaire
      5: -- non caché
      6: -- caché et traité
      7: -- caché
      8: -- composé
      9: -- non composé
     10: -- chargé et traité
     11: -- chargé et composé
     12: -- sauté
\stopmessages

\unprotect

% more accurate
%
% \@EA\convertcommand\csname\e!start\v!buffer\endcsname\to\beginofblock % else a space
% \@EA\convertcommand\csname\e!stop\v!buffer \endcsname\to\endofblock

\def\resetbuffer
  {\dosingleempty\doresetbuffer}

\def\doresetbuffer[#1]%
  {\unlinkfile{\TEXbufferfile{\iffirstargument#1\else\jobname\fi}}}

% \EveryPar%
%   {\doglobal\newcounter\NOfLines}
%
% \EveryLine%
%   {\doglobal\increment\NOfLines%
%    \hskip-3em%
%    \hbox to 3em{\hss\NOfLines\hskip1em}}

%D For Willy's nested commented buffers, we need the \type
%D {\delcharacter} trick.

%\def\processnextbufferline#1% needs testing ! ! ! !
%  {\relax % checken waarom eerdere macro dit nodig heeft / supp-mps run
%   \convertargument#1 \to\next
%   \doifinstringelse{\delcharacter\texcommentsymbol}{\delcharacter\next}
%     {\let\next\secondoftwoarguments}
%     {\doifincsnameelse\endofblock\next
%        {\ifnum\nestedbufferlevel=\zerocount
%           \let\next\firstoftwoarguments
%         \else
%           \decrement\nestedbufferlevel\relax
%           \let\next\secondoftwoarguments
%         \fi}
%        {\doifincsnameelse\beginofblock\next
%           {\increment\nestedbufferlevel\relax
%            \let\next\secondoftwoarguments}
%           {\let\next\secondoftwoarguments}}}%
%   \next}

\long\def\processnextbufferline#1%
  {\relax % checken waarom eerdere macro dit nodig heeft / supp-mps run
   \convertargument#1 \to\next
   \doifinstringelse{\delcharacter\texcommentsymbol}{\delcharacter\next}
     {\secondoftwoarguments}
     {\doifincsnameelse\endofblock\next
        {\ifnum\nestedbufferlevel=\zerocount
           \expandafter\firstoftwoarguments
         \else
           \decrement\nestedbufferlevel\relax
           \expandafter\secondoftwoarguments
         \fi}
        {\doifincsnameelse\beginofblock\next
           {\increment\nestedbufferlevel\relax
            \secondoftwoarguments}
           {\secondoftwoarguments}}}}

\def\dostartbuffer
  {\bgroup
   \obeylines % nodig, anders gaat 't fout als direct \starttable (bv)
   \doquadrupleempty\dodostartbuffer}

\def\dodostartbuffer[#1][#2][#3][#4]% upward compatible
  {\iffourthargument
     \def\next{\dododostartbuffer[#1][#2][#3][#4]}%
   \else
     \def\next{\dododostartbuffer[][#1][#2][#3]}%
   \fi
   \next}

\chardef\buffernestmode\plusone % 0: not nested, 1: startbuffer nested, 2: all buffers nested

\def\dododostartbuffer[#1][#2][#3][#4]%
  {%\showmessage\m!systems{15}{#2}%
   \doifelsevalue{\??bu#1\c!paragraph}\v!yes
     {\segmentatebuffertrue}
     {\doifnumberelse{\getvalue{\??bu#1\c!paragraph}}\segmentatebuffertrue\segmentatebufferfalse}%
   \doifvalue{\??bu#1\c!local}\v!yes
     {\chardef\buffernestmode\plustwo}% permit nesting
   \doifelsenothing{#4}
     {\letbeundefined{\e!stop\v!buffer}% % \let\stopbuffer=\relax   % \undefined
      \@EA\@EA\@EA\convertargument\@EA\e!start\v!buffer\to\beginofblock % else a space
      \@EA\@EA\@EA\convertargument\@EA\e!stop \v!buffer\to\endofblock
      \ifcase\buffernestmode \else
        \let\processnextblockline\processnextbufferline
      \fi}
     {\letbeundefined{#4}% \letvalue{#4}=\relax     % \undefined
      \@EA\convertargument\csname#3\endcsname\to\beginofblock
      \@EA\convertargument\csname#4\endcsname\to\endofblock
      \ifcase\buffernestmode \or \else
        \let\processnextblockline\processnextbufferline
      \fi}%
   \def\closeblock
     {\ifsegmentatebuffer
        \immediate\write\tmpblocks{\string\stopbufferparagraph}%
      \fi
      \immediate\closeout\tmpblocks
      \egroup
      \getvalue{#4}}%
   \doifelsenothing{#2}
     {\edef\bufferfilename{\TEXbufferfile\jobname}}%
     {\edef\bufferfilename{\TEXbufferfile{#2}}}%
   \message{<\bufferfilename>}%
   \immediate\openout\tmpblocks\bufferfilename
   \ifsegmentatebuffer
     \immediate\write\tmpblocks{\string\startbufferparagraph}%
   \fi
   \newcounter\nestedbufferlevel
   \setupcopyblock
   \let\writeoutblocks\gobbleoneargument
   \copyblockline}

\letvalue{\e!start\v!buffer}\dostartbuffer

% \setbuffer[name]#2\endbuffer : saves to file #1.tmp

\def\setbuffer[#1]#2\endbuffer
  {\edef\bufferfilename{\TEXbufferfile{#1}}%
   \message{<\bufferfilename>}%
   \immediate\openout\tmpblocks\bufferfilename
   \convertargument#2\to\ascii
   \immediate\write\tmpblocks{\ascii}%
   \immediate\closeout\tmpblocks}

\def\dodobuffer#1#2%
  {\beginrestorecatcodes
   \ifundefined{\??bu#2\c!number}%
     \edef\bufferfilename{\TEXbufferfile{#2}}%
   \else
     \edef\bufferfilename{\TEXbufferfile{def-\getvalue{\??bu#2\c!number}}}%
   \fi
   #1\bufferfilename\donothing\donothing
   \endrestorecatcodes}

\def\dobuffer#1[#2]#3%
  {\doifelsenothing{#2}
     {\dodobuffer#3\jobname}
     {\processcommalist[#2]{\dodobuffer#3}}}

\def\processTEXbuffer
  {\dodoubleempty\doprocessTEXbuffer}

\def\doprocessTEXbuffer[#1][#2]%
  {\ifsecondargument
     \dodoprocessTEXbuffer[#1][#2]%
   \else
     \dodoprocessTEXbuffer[][#1]%
   \fi}

\def\dodoprocessTEXbuffer[#1][#2]%
  {\getvalue{\??bu#1\c!before}%
   \dobuffer{16}[#2]\readjobfile
   \getvalue{\??bu#1\c!after}}

\let\getbuffer \processTEXbuffer % handy
\let\getbuffer\processTEXbuffer % will move to mult-com.tex

\def\typebuffer
  {\dodoubleempty\dotypebuffer}

\def\dotypebuffer[#1][#2]%
  {\iffirstargument
     \dobuffer{17}[#1]\typefile
   \else
     \dobuffer{17}[#2]\typefile
   \fi}

\def\setupbuffer
  {\dodoubleempty\dosetupbuffer}

\def\dosetupbuffer[#1][#2]%
  {\ifsecondargument
     \getparameters[\??bu#1][#2]%
   \else
     \getparameters[\??bu][#1]%
   \fi}

\def\dodefinebuffer[#1][#2]%
  {\iffirstargument % else problems
     \doglobal\increment\nofdefinedbuffers
     \letvalue{\??bu#1\c!number}\nofdefinedbuffers
     \letvalue{\??bu#1\c!paragraph}\v!no
     \setevalue{\e!start#1}%
       {\noexpand\dostartbuffer[#1][def-\nofdefinedbuffers][\e!start#1][\e!stop#1]}%
     \setevalue{\e!get#1}%
       {\noexpand\dodoprocessTEXbuffer[#1][def-\nofdefinedbuffers]}%
     \setevalue{\e!type#1}%
       {\noexpand\dodotypebuffer[#1][def-\nofdefinedbuffers]}%
     \getparameters[\??bu#1][#2]%
   \fi}

\def\definebuffer
  {\dodoubleempty\dodefinebuffer}

% yet another undocumented feature, but who cares:

\let\startfilebuffer\startbuffer

\def\usememorybuffers{\let\startbuffer\startmemorybuffer}
\def\usefilebuffers  {\let\startbuffer\startfilebuffer}

\def\startmemorybuffer
  {\dosingleempty\dostartmemorybuffer}

\long\def\dostartmemorybuffer[#1]#2\stopbuffer
  {\setbuffer[\iffirstargument#1\else\jobname\fi]#2\endbuffer}

% \long\def\startcrap#1\stopcrap
%   {\usememorybuffers#1\usefilebuffers} % or {{...}}
%
% \startcrap
% \startbuffer
% some awful code
% \stopbuffer
% \placefigure{crap}{\getbuffer}
% \stopcrap

% TODO: no grouping due to sidefloats

\expandafter \convertargument \gobbleoneargument @ \to \emptybufferline

\newif\ifsegmentatebuffer
\newif\ifemptybufferline

\def\skippedbufferparagraphs{0}

\let\startbufferparagraph\relax
\let\stopbufferparagraph \par   % \relax

\newcount\currentbufferparagraph

\def\getbufferparagraphs
  {\dodoubleempty\dogetbufferparagraphs}

\def\dosetbufferoffset#1%
  {\doifnumberelse{\getvalue{\??bu#1\c!paragraph}}
     {\currentbufferparagraph-\getvalue{\??bu#1\c!paragraph}}
     {\currentbufferparagraph \zerocount}%
   \relax}

\def\dogetbufferparagraphs[#1][#2]%
  {\iffirstargument
     \ifsecondargument
       \dosetbufferoffset{#1}%
       \doifelse{#2}\v!all
         {\def\startbufferparagraph{\normalbufferparagraph{#1}}}
         {\def\startbufferparagraph{\filterbufferparagraph{#1}{#2}}}%
       \def\stopbufferparagraph{\dostopbufferparagraph{#1}}%
       \def\next{\getparagraphedbuffer[#1]}%
     \else
       \dosetbufferoffset\empty
       \def\startbufferparagraph{\filterbufferparagraph{}{#1}}%
       \def\stopbufferparagraph{\dostopbufferparagraph{}}%
       \def\next{\getparagraphedbuffer[]}%
     \fi
   \else
     \dosetbufferoffset\empty
     \def\startbufferparagraph{\normalbufferparagraph{}}%
     \def\stopbufferparagraph{\dostopbufferparagraph{}}%
     \def\next{\getparagraphedbuffer[]}%
   \fi
   \next}

\def\getparagraphedbuffer[#1]%
  {\dobuffer{16}[#1]\readjobfile}

\def\dostopbufferparagraph#1%
  {\getvalue{\??bu#1\c!after}\par}

\def\dostartbufferparagraph#1%
  {\par\getvalue{\??bu#1\c!before}}

\def\normalbufferparagraph
  {\advance\currentbufferparagraph \plusone
   \ifnum\currentbufferparagraph>\zerocount
     \expandafter\dostartbufferparagraph
   \else
     \expandafter\gobbleoneargument
   \fi}

\def\filterbufferparagraph#1#2%
  {\advance\currentbufferparagraph \plusone
   \ifnum\currentbufferparagraph>\zerocount
     \doifinsetelse{\the\currentbufferparagraph}{#2}
       {\let\next\dostartbufferparagraph}
       {\let\next\fakebufferparagraph}%
   \else
     \let\next\gobblebufferparagraph
   \fi
   \next{#1}}

% nicer, but untested
%
% \def\filterbufferparagraph#1#2%
%   {\advance\currentbufferparagraph \plusone
%    \ifcase\currentbufferparagraph
%      \@EA\gobblebufferparagraph
%    \else
%      \doifinsetelse{\the\currentbufferparagraph}{#2}
%        {\@EA\dostartbufferparagraph}
%        {\@EA\fakebufferparagraph}%
%    \fi
%    {#1}}

\long\def\gobblebufferparagraph#1#2\stopbufferparagraph
  {}

\def\fakebufferparagraph#1%
  {\bgroup
   \def\stopbufferparagraph{\dostopbufferparagraph{#1}\egroup\egroup}%
   \setbox\scratchbox\vbox\bgroup\dostartbufferparagraph{#1}}

\def\blockversion {1996.03.10}

\def\@@blockerrormessage%
  {\showmessage\m!textblocks1\empty
   \global\let\@@blockerrormessage\relax}

\def\thisisblockversion#1%
  {\doifnot\blockversion{#1}{\@@blockerrormessage\endinput}}

\def\stopcopyingblocks
  {\ifcopyingblocks
     \immediate\closeout\outblocks
     \copyblockfile
     \global\copyingblocksfalse
   \fi}

\def\dodosetblockcounters[#1]#2%
  {\expanded{\setvalue{\??se\s!old#2}{\@@filterheadpart[#1]}}%
   \doifnot{#2}\lastsection
     {\expanded{\dodosetblockcounters[\@@filtertailpart[#1]]}%
        {\getvalue{\??se#2\c!after}}}} % ????

\def\dosetblockcounters[#1]%
  {\ifblockpermitted
     \expanded{\dodosetblockcounters[\@@filtersecondpart[#1]]}\firstsection
     \expanded{\setsectiontype[\@@filterfirstpart[#1]]}%
     \def\@@sectionvalue##1{\getvalue{\??se\s!old##1}}%
     \let\@@sectionconversion\secondoftwoarguments
   \fi}

\let\blockstatus\empty

\def\setblockcounters
  {\ifx\blockstatus\empty \else
     \@EA\dosetblockcounters\@EA[\blockstatus]%
   \fi}

\def\getblockstatus#1%
  {\dosetfilterlevel{\@@bscriterium}\empty
   \expanded{\doifblklevelelse[#1\sectionseparator\sectionseparator0]}
     {\global\blockpermittedtrue}
     {\global\blockpermittedfalse}%
   \def\blockstatus{#1}}

\def\setupblockparameters
  {\dodoubleargument\dosetupblockparameters}

\def\dosetupblockparameters[#1][#2]%
  {\getparameters[\??tb#1][#2]}

\beginTEX

\def\blockparameter#1#2%
  {\csname\@EA\ifx\csname\??tb#1#2\endcsname\relax\s!empty\else\??tb#1#2\fi\endcsname}

\endTEX

\beginETEX \ifcsname

\def\blockparameter#1#2%
  {\@EA\csname\ifcsname\??tb#1#2\endcsname\??tb#1#2\else\s!empty\fi\endcsname}

\endETEX

\ifx\outblocks\undefined \newwrite\outblocks \fi
\ifx\inpblocks\undefined \newread \inpblocks \fi
\ifx\tmpblocks\undefined \newwrite\tmpblocks \fi
\ifx\blockbox \undefined \newbox  \blockbox  \fi

\newif\ifcopyingblocks
\newif\ifvisible         \visibletrue
\newif\ifblockpermitted
\newif\iftmpblockstarted
\newif\ifoldinbijlagen
\newif\ifdoingblocks

\newcount\blocklevel     \blocklevel=0

%\def\setblocklevel#1%
%  {\global\advance\blocklevel by #11
%   \ifnum\blocklevel>2\relax\doingblockstrue\else\doingblocksfalse\fi}
%
% oeps, got bugged

\def\setblocklevel#1% sign
  {\global\advance\blocklevel #11
   \ifcase\blocklevel\doingblocksfalse\else\doingblockstrue\fi}

\def\opentmpblock
  {\immediate\openout\tmpblocks=\TEXbufferfile{\f!utilityfilename\the\blocklevel}}

\def\closetmpblock
  {\immediate\write\tmpblocks{}%   een lege regel is handig voor \par commando's
   \immediate\closeout\tmpblocks}

\def\writetmpblock#1%
  {\iftmpblockstarted
     \ifsegmentatebuffer
       \ifemptybufferline
         \immediate\write\tmpblocks{\string\stopbufferparagraph }%
         \immediate\write\tmpblocks{\string\startbufferparagraph}%
       \else
         \immediate\write\tmpblocks{#1}%
       \fi
     \else
       \immediate\write\tmpblocks{#1}%
     \fi
   \else
     \doifsomething{#1}
       {\tmpblockstartedtrue
        \immediate\write\tmpblocks{\string#1}}%
   \fi}

\def\startcopyingblocks
  {\global\copyingblocksfalse}

\def\checkcopyingblocks
  {\ifcopyingblocks
   \else
     \immediate\openout\outblocks\f!utilityfilename.\f!blockextension
     \immediate\write\outblocks{\string\thisisblockversion{\blockversion}}%
     \immediate\write\outblocks{\string\thisissectionseparator{\sectionseparator}}%
     \global\copyingblockstrue
   \fi}

\def\stopcopyingblocks
  {\ifcopyingblocks
     \immediate\closeout\outblocks
     \copyblockfile
     \global\copyingblocksfalse
   \fi}

\def\nomoreblocks
  {\stopcopyingblocks}

\def\copyblockfile
  {\ifcopyingblocks
     \begingroup
     \showmessage\m!textblocks2{\jobname.\f!blockextension}%
     \openlocin\inpblocks{\f!utilityfilename.\f!blockextension}%
     \immediate\openout\outblocks\jobname.\f!blockextension
     \setupcopyblock
     \catcode`\^^M=\@@ignore\relax
     \def\copynextline
       {\read\inpblocks to \!!stringa
        \immediate\write\outblocks{\!!stringa}%
        \ifeof\inpblocks\else\expandafter\copynextline\fi}%
     \copynextline
     \immediate\closein\inpblocks
     \immediate\closeout\outblocks
     \immediate\openout\tmpblocks\f!utilityfilename.\f!blockextension
     \immediate\closeout\tmpblocks
     \endgroup
   \fi}

\def\loadallblocks#1%
  {\beginrestorecatcodes
   \catcode`\^^M=\@@endofline\relax
   \readjobfile{#1.\f!blockextension}
     {\showmessage\m!textblocks3{#1.\f!blockextension}}
     {\showmessage\m!textblocks4\empty}%
   \endrestorecatcodes}

\def\setupcopyblock
  {\makeallother
   \obeylines}

\def\writeoutblocks
  {\immediate\write\outblocks}

% readable
%
% \def\processnextblocklineAB#1#2#3%
%   {\convertargument#1 \to\next
%    \doifinstringelse\endofblockA\next
%      {\def\next{#2}}
%      {\doifinstringelse\endofblockB\next
%        {\def\next{#2}}
%        {\def\next{#3}}}%
%    \next}
%
% faster
%
% \def\processnextblocklineAB#1#2#3%
%   {\convertargument#1 \to\next % space is essential
%    \doifinstringelse\endofblockA\next
%      {#2}{\doifinstringelse\endofblockB\next{#2}{#3}}}
%
% even more

\long\def\processnextblocklineAB#1% #2#3%
  {\convertargument#1 \to\next
   \doifinstringelse\endofblockA\next
      \firstoftwoarguments
     {\doifinstringelse\endofblockB\next
        \firstoftwoarguments\secondoftwoarguments}}

\bgroup
\obeylines
\long\gdef\copyblocklineAB#1
  {\processnextblocklineAB{#1}\closeblock{\writeoutblocks{#1}\writetmpblock{#1}\copyblocklineAB}}
\long\gdef\skipblocklineAB#1
  {\processnextblocklineAB{#1}\closeblock\skipblocklineAB}
\egroup

% \def\processnextblockline#1#2#3%
%   {\convertargument#1 \to\next
%    \ifx\next\emptybufferline
%      \ifsegmentatebuffer \emptybufferlinetrue \fi
%      \def\next{#3}%
%    \else
%      \emptybufferlinefalse
%      \doifinstringelse{\endofblock}{\next}
%        {\def\next{#2}}
%        {\def\next{#3}}%
%    \fi
%    \next}
%
% faster

\long\def\processnextblockline#1% #2#3%
  {\convertargument#1 \to\next
   \ifx\next\emptybufferline
     \ifsegmentatebuffer \emptybufferlinetrue \fi
     \expandafter\secondoftwoarguments% #3%
   \else
     \emptybufferlinefalse
     \doifinstringelse\endofblock\next
       {\expandafter\firstoftwoarguments }% #2}
       {\expandafter\secondoftwoarguments}% #3}%
   \fi}

\bgroup
\obeylines
\long\gdef\copyblockline#1
  {\processnextblockline{#1}\closeblock{\writeoutblocks{#1}\writetmpblock{#1}\copyblockline}}
\long\gdef\skipblockline#1
  {\processnextblockline{#1}\closeblock\skipblockline}
\egroup

\def\skipblock#1%
  {\checkcopyingblocks
   \@EA\convertargument\string\thiswasblock{#1}\to\endofblock
   %testen : \expanded{\convertargument\string\thiswasblock{#1}\noexpand\to\noexpand\endofblock}%
   \let\openblock\begingroup
   \let\closeblock\endgroup
   \openblock
   \setupcopyblock
   \skipblockline}

\let\doafterblock \gobbletwoarguments
\let\dobeforeblock\gobbletwoarguments

\def\thisisblock#1%
  {\executeifdefined{\s!thisisblock#1}{\skipblock{#1}}}

\def\thiswasblock#1%
  {\getvalue{\s!thiswasblock#1}}

\def\saveblock#1#2%
  {\checkcopyingblocks
   \obeylines
   \@EA\@EA\@EA\convertargument\@EA\string\csname\e!end#1\endcsname\to\endofblockA
   %testen:  \expanded{\convertargument\string\csname\e!end#1\endcsname\to\endofblockA}%
   \@EA\convertargument\string\endblock[#1]\to\endofblockB % MULTI LINGUAL MAKEN
   \def\openblock
     {\dobeforeblock{#1}{#2}%
      \opentmpblock
      \begingroup
      \makesectionformat
      \immediate\write\outblocks{}%
      \immediate\write\outblocks{\string\thisisblock{#1}{\sectionformat}[#2]}}%
   \def\closeblock
     {\immediate\write\outblocks{}%   handig voor \par commando's
      \immediate\write\outblocks{\string\thiswasblock{#1}}%
      \endgroup
      \closetmpblock
      \doafterblock{#1}{#2}%
      \egroup}%
   \openblock
   \setupcopyblock
   \copyblocklineAB}

\def\copyblock
  {\let\opentmpblock\empty
   \let\closetmpblock\empty
   \let\writetmpblock\gobbleoneargument
   \saveblock}

\def\loadoneblock
  {\edef\blockfilename{\TEXbufferfile{\f!utilityfilename\the\blocklevel}}%
   \setblocklevel+%
   \readjobfile\blockfilename\donothing\donothing
   \setblocklevel-}%

\def\dodefineblock[#1]%
  {\bypassblock[#1]%
   \keepblocks[#1]%
   \setupblock
     [#1]
     [\c!before=\blank,
      \c!after=\blank,
      \c!inner=,
      \c!style=,
      \c!file=\jobname]}

\def\defineblock
  {\dosingleargumentwithset\dodefineblock}

\def\dosetupblock[#1][#2]%
  {\getparameters[\??tb#1][#2]}

\def\setupblock
  {\dodoubleargumentwithset\dosetupblock}

\def\bypassblock[#1]%
  {\setvalue{\s!thisisblock#1}##1[##2]{\skipblock{#1}}}

\def\dohideblock[#1][#2][#3]%
  {\doifassignmentelse{#3}
     {\dodohideblock[#1][#2][][#3]}
     {\dodohideblock[#1][#2][#3][]}}

\def\dodohideblock[#1][#2][#3][#4]%
  {\doifelsenothing{#2}
     {\global\blockpermittedfalse
      \edef\blocktitle{#1}}
     {\doifelsenothing{#3}
        {\global\blockpermittedtrue
         \edef\blocktitle{#1}}
        {\doifcommonelse{#2}{#3}
           {\global\blockpermittedfalse
            \edef\blocktitle{#1:#2}}
           {\global\blockpermittedtrue
            \edef\blocktitle{#1:#3}}}}%
   \ifblockpermitted
     \showwarning\m!textblocks5\blocktitle
     \def\next
       {\def\dobeforeblock####1####2%
          {\begingroup}%
        \def\doafterblock####1####2%
          {\endgroup
           \doexecuteloadedblock{#1}{#4}}%
        \saveblock{#1}{#3#4}}%
   \else
     \doifinsetelse{+}{#3}
       {\showwarning\m!textblocks6\blocktitle
        \def\next
          {\def\dobeforeblock####1####2%
             {\begingroup
              \global\visiblefalse}%
           \def\doafterblock####1####2%
             {{\setbox0\vbox
                 {\catcode`\^^M=\@@endofline\relax
                  \loadoneblock
                  \par}}%
              \endgroup}%
           \saveblock{#1}{#3#4}}}%
       {\showwarning\m!textblocks7\blocktitle
        \def\next
          {\def\dobeforeblock####1####2%
             {\begingroup
              \globaldefs\minusone}%
           \def\doafterblock####1####2%
             {\endgroup}%
           \copyblock{#1}{#3#4}}}%
   \fi
   \next}

\def\dohideblocks[#1][#2]%
  {\def\docommando##1%
     {\setvalue{\e!begin##1}%
        {\bgroup\obeylines\dotripleempty\dohideblock[##1][#2]}}%
   \processcommalist[#1]\docommando}

\def\hideblocks
  {\dodoubleempty\dohideblocks}

\def\doexecuteloadedblock#1#2%
  {\blockpermittedtrue % ?
   \bgroup % before \c!before (think of: \c!before=\startitemize)
   \dosetupblockparameters[#1][#2]% voor 'voor'?
   \getvalue{\??tb#1\c!before}%
   \dostartattributes{\??tb#1}\c!style\c!color\empty
   \visibletrue
   \catcode`\^^M=\@@endofline\relax
   \getvalue{\??tb#1\c!inner}%
   \ignorespaces
   \loadoneblock
   % \par moved
   \dostopattributes
   \getvalue{\??tb#1\c!after}%
   \par
   \egroup}

\def\dokeepblock[#1][#2][#3]%
  {\doifassignmentelse{#3}
     {\dodokeepblock[#1][#2][][#3]}
     {\dodokeepblock[#1][#2][#3][]}}

\def\dodokeepblock[#1][#2][#3][#4]%
  {\doifelsenothing{#2}
     {\global\blockpermittedtrue
      \edef\blocktitle{#1}}
     {\doifcommonelse{#2}{#3}
        {\global\blockpermittedtrue
         \edef\blocktitle{#1:#2}}
        {\doifinsetelse\v!all{#2}
           {\doifelsenothing{#3}
              {\global\blockpermittedtrue
               \edef\blocktitle{#1}}
              {\global\blockpermittedfalse
               \edef\blocktitle{#1:#3}}}
           {\global\blockpermittedfalse
            \doifelsenothing{#3}
              {\edef\blocktitle{#1}}
              {\edef\blocktitle{#1:#3}}}}}%
   \ifblockpermitted
     \showwarning\m!textblocks8\blocktitle
     \def\dobeforeblock##1##2%
       {\begingroup}%
     \def\doafterblock##1##2%
       {\endgroup
        \doexecuteloadedblock{#1}{#4}}%
   \else
     \showwarning\m!textblocks9\blocktitle
   \fi
   \saveblock{#1}{#3#4}}

\def\dokeepblocks[#1][#2]%
  {\def\docommando##1%
     {\setvalue{\e!begin##1}%
        {\bgroup\obeylines\dotripleempty\dokeepblock[##1][#2]}}%
   \processcommalist[#1]\docommando}

\def\keepblocks
  {\dodoubleempty\dokeepblocks}

\newconditional\processblockstatus
\newconditional\dummyblockstatus
\newconditional\blockassignmentstatus

\def\dodouseblock#1#2#3#4%
  {\getblockstatus{#2}%
   \ifblockpermitted
     \setfalse\dummyblockstatus
     \doifassignmentelse{#3}
       {\settrue \blockassignmentstatus}
       {\setfalse\blockassignmentstatus}%
     \doifelsenothing{#4}
       {\edef\blocktitle{#1}}
       {\ifconditional\blockassignmentstatus
          \edef\blocktitle{#1}%
        \else
          \doifnotcommon{#3}{#4}
            {\ifconditional\processblockstatus
               \settrue\dummyblockstatus
             \else
               \global\blockpermittedfalse
             \fi}%
          \edef\blocktitle{#1:#3}%
        \fi}%
   \else
     \edef\blocktitle{#1}%
   \fi
   \ifblockpermitted
     \setblocklevel+%
     \ifconditional\blockassignmentstatus \else
       \doifinset{-}{#3}{\settrue\dummyblockstatus}%
     \fi
     \ifconditional\dummyblockstatus
       \showwarning\m!textblocks{10}\blocktitle
       \setvalue{\s!thiswasblock#1}%
         {\removeunwantedspaces
          \par
          \egroup
          \setblocklevel-}%
       \def\next
         {\setbox0\vbox\bgroup
          \ifconditional\blockassignmentstatus
            \dosetupblockparameters[#1][#3]%
          \fi}%
     \else
       \showwarning\m!textblocks{11}\blocktitle
       \setvalue{\s!thiswasblock#1}%
         {\removeunwantedspaces
          % \par moved
          \dostopattributes
          \getvalue{\??tb#1\c!after}%
          \par
          \egroup
          \setblocklevel-}%
       \def\next
         {\bgroup
          \ifconditional\blockassignmentstatus
            \dosetupblockparameters[#1][#3]%
          \fi
          \getvalue{\??tb#1\c!before}%
          \dostartattributes{\??tb#1}\c!style\c!color\empty
          \visibletrue
          \getvalue{\??tb#1\c!inner}%
          \ignorespaces}%
     \fi
   \else
     \def\next
       {\showwarning\m!textblocks{12}\blocktitle
        \skipblock{#1}}%
   \fi
   \next}

\def\douseblock[#1][#2]%
  {\setvalue{\s!thisisblock#1}##1[##2]{\dodouseblock{#1}{##1}{##2}{#2}}}

\def\dodouseblocks[#1][#2]%
  {\def\docommando##1%
     {\douseblock[##1][#2]}%
   \processcommalist[#1]\docommando
   \dogetcommalistelement1\from#1\to\commalistelement
   \doifdefined{\??tb\commalistelement\c!file}
     {\loadallblocks{\getvalue{\??tb\commalistelement\c!file}}}%
   \endgroup}

\def\douseblocks
  {\begingroup
   \doassign[\??bs][\c!criterium=\v!all]%
   \dodoubleempty\dodouseblocks}

\def\useblocks
  {\setfalse\processblockstatus\douseblocks}

\def\processblocks
  {\settrue \processblockstatus\douseblocks}

\def\doselectblocks[#1][#2][#3]%
  {\begingroup
   \doifelsenothing{#3}
     {\getparameters[\??bs][\c!criterium=\v!all,#2]%
      \dodouseblocks[#1][]}
     {\getparameters[\??bs][\c!criterium=\v!all,#3]%
      \dodouseblocks[#1][#2]}}%

\def\selectblocks
  {\dotripleempty\doselectblocks}

\def\beginblock[#1]%  % we also check \endblock[..]
  {\getvalue{\e!begin#1}}

\def\forceblocks[#1]%
  {\def\docommando##1%
     {\setvalue{\e!begin##1}%
        {\setblocklevel+\bgroup
         \dodoubleempty\doforceblock[##1]}%
      \setvalue{\e!end##1}%
        {\dostopattributes
         \getvalue{\??tb##1\c!after}%
         \egroup\setblocklevel-}}%
   \processcommalist[#1]\docommando}

\def\doforceblock[#1][#2]%
  {\doifassignmentelse{#2}
     {\settrue \blockassignmentstatus}
     {\setfalse\blockassignmentstatus}%
   \ifconditional\blockassignmentstatus
     \dosetupblockparameters[#1][#2]%
   \fi
   \getvalue{\??tb#1\c!before}%
   \dostartattributes{\??tb#1}\c!style\c!color\empty
   \getvalue{\??tb#1\c!inner}%
   \ignorespaces}

\def\bypassblocks[#1]%
  {\def\docommando##1%
     {\setvalue{\e!begin##1}%
        {\setblocklevel+\bgroup
         \obeylines % here, since we look ahead
         \dodoubleempty\dobypassblock[##1]}%}%
      \setvalue{\e!end##1}%
        {}}%
   \processcommalist[#1]\docommando}

\def\dobypassblock[#1][#2]%
  {\def\closeblock
     {\egroup\setblocklevel-}%
   \checkcopyingblocks
   \obeylines
   \@EA\@EA\@EA\convertargument\@EA\string\csname\e!end#1\endcsname\to\endofblockA
   \@EA\convertargument\string\endblock[#1]\to\endofblockB % MULTI LINGUAL MAKEN
   \setupcopyblock
   \skipblocklineAB}

% \setvalue{\e!start\v!hiding}%
%   {\dostartbuffer[buf-\nofpostponedblocks]
%      [\e!start\v!hiding][\e!stop\v!hiding]}

\definebuffer[\v!hiding] \setupbuffer[\v!hiding][\c!local=\v!yes]

\setupbuffer
  [\c!paragraph=\v!no,
   \c!before=,
   \c!after=]

\protect \endinput
