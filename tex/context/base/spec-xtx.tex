%D \module
%D   [       file=spec-xtx,
%D        version=2004.*.*,
%D          title=\CONTEXT\ Special Macros,
%D       subtitle=DVIPDFMx support,
%D         author={Adam Lindsay \& Hans Hagen \& \unknown},
%D           date=\currentdate,
%D      copyright={Adam Lindsay \& Hans Hagen}]

\unprotect

\startspecials[xetex]

%D Rotation looks fine. Scaling and mirroring are also fine.

\definespecial\dostartrotation  #1{\special{x:gsave}\special{x:rotate #1}}
\definespecial\dostartscaling #1#2{\special{x:gsave}\special{x:scale  #1 #2}}
\definespecial\dostartmirroring   {\special{x:gsave}\special{x:scale  -1 1}}

\definespecial\dostoprotation {\special{x:grestore}}
\definespecial\dostopscaling  {\special{x:grestore}}
\definespecial\dostopmirroring{\special{x:grestore}}

\ifx\colorhexcomponent\undefined

    % this will be added to colo-hex.tex

\else

    % because we intercept the zero condition, the .23pt in 1.23pt will disappear in the
    % ifcase zero part branch

    \def\colorhexcomponent#1%
      {\ifdim#1\points<.005\points
         00\else\lchexnumbers{\the\dimexpr(255\dimexpr(#1\points)\relax+.5\points)\relax}%
       \fi}

    % the faster one

    \newdimen\hex@color@a \hex@color@a=.005pt
    \newdimen\hex@color@b \hex@color@b=.5pt
    \chardef \hex@color@c             =255

    \def\colorhexcomponent#1%
      {\ifdim#1\points<\hex@color@a
         00\else\lchexnumbers{\the\dimexpr(#1\points*\hex@color@c+\hex@color@b)\relax}%
       \fi}

\fi

%

\definespecial\dostartrgbcolormode#1#2#3%
  {\bgroup
   \edef\hexcolorstring{\colorhexcomponent{#1}\colorhexcomponent{#2}\colorhexcomponent{#3}}%
   \special{x:textcolor=\hexcolorstring}%
   \special{x:rulecolor=\hexcolorstring}%
   \egroup}

\definespecial\dostopcolormode
  {\special{x:textcolor=000000}%
   \special{x:rulecolor=000000}}

%D Whadda you mean by 'corected;, this hex color model is crazy. Why does
%D \XETEX\ provide high end font support but only hex rgb colors -)

%D once colors are ``corrected'' by Hans, transparency should
%D follow fairly easily. A good side-effect of the above approach
%D is that we need to keep track of the color state (transparency
%D in \XETEX\ is accomplished setting the color to an RGBA quadruple,
%D rather than an RGB triple)

%D File/graphic insertion is fouled up. I don't know why, but
%D the externalfile misses its designated box/frame. It's also not
%D finding files without the help of a \TEXUTIL\ file.

%D Maybe \XETEX\ should provide a way to report the figure dimensions,
%D I suppose it can ask the QuickTime Driver.

%D However it gets resolved, the following is the basic schema
%D of file inclusion for \XETEX. \type {\XeTeXpicfile} supports
%D rotation, as well, but it seems less important for \CONTEXT.
%D \XETEX\ uses QuickTime file import, which means a huge number of
%D file import options.

%D More importantly and interestingly, there is \type
%D {XeTeXpdffile}, which supports a \type {page} option.

\definespecial\doinsertfile#1#2#3#4#5#6#7#8#9%
  {\bgroup
   \dodoinsertfile{xtx}{#1}{#2}{#3}{#4}{#5}{#6}{#7}{#8}{#9}%
   \egroup}

\definefileinsertion{xtx}{jpg}{\handlepdfimage}
\definefileinsertion{xtx}{png}{\handlepdfimage}
\definefileinsertion{xtx}{pdf}{\handlepdfimage}
\definefileinsertion{xtx}{gif}{\handlepdfimage}
\definefileinsertion{xtx}{tif}{\handlepdfimage}

% do we need it this way? either provide width and height or provide scales,
% best provide the dimensions

% \def\handlepdfimage#1#2#3#4#5#6#7#8#9%
%   {\XeTeXpicfile "#1" width #7 height #8 xscaled #3 yscaled #4\relax}

\def\handlepdfimage#1#2#3#4#5#6#7#8#9%
  {\XeTeXpicfile "#1" width #7 height #8\relax}

\stopspecials

\protect \endinput
