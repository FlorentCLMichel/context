%D \module
%D   [       file=strc-ref,
%D        version=2008.10.20,
%D          title=\CONTEXT\ Structure Macros,
%D       subtitle=Cross Referencing,
%D         author=Hans Hagen,
%D           date=\currentdate,
%D      copyright=PRAGMA-ADE / Hans Hagen]
%C
%C This module is part of the \CONTEXT\ macro||package and is
%C therefore copyrighted by \PRAGMA. See mreadme.pdf for
%C details.

\writestatus{loading}{ConTeXt Structure Macros / Cross Referencing}

\registerctxluafile{strc-ref}{1.001}

\unprotect

%D This module is a (partial) rewrite of core-ref.tex for \MKIV. As
%D such it will be a moving target for a while.

%D Later we will do a further cleanup and move much of the code to
%D \LUA\ (i.e.\ better backend integration).

\let\mainreference\gobblefivearguments

% this will go when we got rid of the tuo file

\let\currentfolioreference \!!zerocount % only used in xml-fo
\let\resetreferences       \relax
\let\setreferences         \relax
\let\showcurrentreference  \relax
\let\setexecutecommandcheck\gobbletwoarguments

\def\s!full{full}
\def\s!text{text}
\def\s!page{page}

% todo : unknown/illegal reference no arg
% todo : +n pages check on 'samepage' (contrastcolor)
% todo : multiple text in reference

% Makes more sense to build action data first, especially now
% openaction etc are supported.
%
% \definespecial\doexecuteactionchain w h
% \definespecial\dosetgotolocation
% \definespecial\dosetexecuteJScode
% ...

%D This module deals with referencing. In \CONTEXT\ referencing
%D is one of the core features, although at a first glance
%D probably nobody will notice. This is good, because
%D referencing should be as hidden as possible.
%D
%D In paper documents, referencing comes down to cross
%D referencing, but in their interactive counterparts, is also
%D involves navigation. Many features implemented here are
%D therefore closely related to navigation.
%D
%D Many \CONTEXT\ commands can optionally be fed with a
%D reference. Such a reference, when called upon, returns the
%D number of a figure, table, chapter etc, a piece of text, or
%D a pagenumber.
%D
%D There are three ways of defining a reference:
%D
%D \starttyping
%D \pagereference[here]
%D \textreference[here]{some text}
%D \stoptyping
%D
%D the third alternative combines them in:
%D
%D \starttyping
%D \reference[here]{some text}
%D \stoptyping

\def\textreference {\dosingleargument\dotextreference}
\def\pagereference {\dosingleargument\dopagereference}
\def\reference     {\dosingleargument\doreference    }

%D These are implemented in a low level form as:

\def\dotextreference[#1]{\dosetreference\s!text{#1}}
\def\dopagereference[#1]{\dosetreference\s!page{#1}{}}
\def\doreference    [#1]{\dosetreference\s!full{#1}}

%D Actually there is not much difference between a text and a
%D full reference, but it's the concept that counts. The low
%D level implementation is:

\newcount\crossreferencenumber

\def\dofinishfullreference#1#2%
  {\normalexpanded{\noexpand\ctxlatelua{jobreferences.enhance("#1","#2")}}%
   \referenceinfo>{#1\letterbar#2}}

\let\dofinishpagereference\dofinishfullreference

\def\dofinishtextreference#1#2%
  {\normalexpanded{\noexpand\ctxlatelua{jobreferences.enhance("#1","#2",{})}}%
   \referenceinfo>{#1\letterbar#2}}

\def\dosetreference#1#2#3% kind labels text -> todo: userdata
  {\ifreferencing
     \global\advance\crossreferencenumber\plusone
     \edef\currentreferencekind{#1}%
     \edef\currentreferencelabels{#2}%
     \edef\currentreferenceexpansion{\@@rfexpansion}% {\referenceparameter\c!expansion}
     \ifx\currentreferencelabels\empty \else
       \ifx\currentreferenceexpansion\s!xml
         \xmlstartraw
           \xdef\currentreferencetext{#3}%
         \xmlstopraw
         \globallet\currentreferencecoding\s!xml
       \else
         \ifx\currentreferenceexpansion\v!yes
           \xdef\currentreferencetext{#3}%
         \else
           \xdef\currentreferencetext{\detokenize{#3}}%
         \fi
         \globallet\currentreferencecoding\s!tex
       \fi
       \setnextinternalreference
       \ctxlua {
            jobreferences.set("\currentreferencekind", "\referenceprefix","\currentreferencelabels",
                {
                    references = {
                        internal = \nextinternalreference,
                        block    = "\currentstructureblock",
                        section  = structure.sections.currentid(),
                    },
                    metadata = {
                        kind     = "#1",
                        catcodes = \the\catcodetable,
                        xmlroot  = \ifx\currentreferencecoding\s!xml "\xmldocument" \else nil \fi, % only useful when text
                    },
                    entries = {
                        text = \!!bs\currentreferencetext\!!es
                    }
                })
            jobreferences.setinternalreference("\referenceprefix","\currentreferencelabels",\nextinternalreference)
         }%
     \fi
   \fi}

%D For compatibility we provide:

\def\rawreference    #1#2#3{\dosetreference\s!full{#2}{#3}} % tag, labels, text
\def\rawpagereference  #1#2{\dosetreference\s!page{#2}{}}   % tag, labels
\def\rawtextreference#1#2#3{\dosetreference\s!text{#2}{#3}} % tag, labels, text

\def\defaultreferencepage#1{[[[#1]]]}
\def\defaultreferencetext#1{[[[#1]]]}

%D These macros depend on three other ones,
%D \type {\makesectionformat}, that generated \type
%D {\sectionformat}, \type {\pagenumber}. The not yet used
%D argument \type{#1} is a tag that specifies the type of
%D reference.

%D \macros
%D   {everyreference}
%D
%D For rather tricky purposes, one can assign sanitizing
%D macros to \type{\everyreference} (no longer that relevant).

\newevery \everyreference \relax

%D This is really needed, since for instance Polish has a
%D different alphabet and needs accented entries in registers.

\appendtoks
  \cleanupfeatures
\to \everyreference

%D We did not yet discuss prefixing. Especially in interactive
%D documents, it's not always easy to keep track of duplicate
%D references. The prefix mechanism, which we will describe
%D later on, solves this problem. By (automatically) adding a
%D prefix one keeps references local, but the global ones in
%D view. To enable this feature, we explictly split the prefix
%D from the reference.

\let\referenceprefix\empty

%D For a long time the only way to access an external file was
%D to use the file prefix (\type {somefile::}. However, when
%D you split up a document, redefining the references may be
%D such a pain, that another approach is feasible. By setting
%D the \type {autofile} variable to \type {yes} or \type
%D {page}, you can access the reference directly.
%D
%D \starttabulate[||||]
%D \NC filename::tag \NC page(filename::pnum) \NC tag     \NC\NR
%D \NC   $\star$     \NC                      \NC         \NC\NR
%D \NC   $\star$     \NC $\star$              \NC $\star$ \NC\NR
%D \NC               \NC $\star$              \NC         \NC\NR
%D \stoptabulate

\def\usereferences[#1]%
  {\writestatus\m!systems{references from other files are handled automatically}}

%D As mentioned we will also use the cross reference mechanism
%D for navigational purposes. The main reason for this is that
%D we want to treat both categories alike:
%D
%D \starttyping
%D \goto{go back}[PreviousJump]
%D \goto{colofon}[colofon page]
%D \stoptyping
%D
%D Here \type{PreviousJump} is handled by the viewer, while the
%D \type{colofon page} reference is, apart from hyperlinking, a
%D rather normal reference.
%D
%D We already saw that cross refences are written to and read
%D from a file. The pure navigational ones don't need to be
%D written to file, but both for fast processing and
%D transparant integration, they are saved internally as a sort
%D of reference. We can easily distinguish such system
%D references from real cross reference ones by their tag.
%D
%D We also use the odd/even characteristic to determine the
%D page state.

\let\currentrealreference      \empty
\let\currentpagereference      \empty
\let\currenttextreference      \empty
\let\currentreferenceorder     \empty
\let\currentsubtextreference   \empty
\let\currentsubsubtextreference\empty

%D System references only have one component:

\newif\ifforwardreference
\newif\ifrealreferencepage

\def\docheckrealreferencepage#1% todo
  {\doifnumberelse{#1}
     {\ifnum#1=\realpageno
        \realreferencepagetrue
      \else
        \realreferencepagefalse
      \fi}
     {\realreferencepagefalse}}

%D Text references can contain more than one entry and
%D therefore we check for
%D
%D \starttyping
%D {entry}
%D \stoptyping
%D
%D or
%D
%D \starttyping
%D {{entry}{entry}{entry}}
%D \stoptyping
%D
%D and split accordingly.

% todo:

\def\doifforwardreferenceelse#1#2% todo
  {\iffalse}

%D Cross references appear as numbers (figure~1.1, chapter~2)
%D or pagenumbers (page~2, page 3--2), and are called with
%D \type{\in} and \type{\at}. In interactive documents we also
%D have \type{\goto}, \type{\button} and alike. These are more
%D versatile and look like:
%D
%D \starttyping
%D \goto[reference]
%D \goto[outer reference::]
%D \goto[outer reference::inner reference]
%D \goto[operation(argument)]
%D \goto[operation(action{argument,argument})]
%D \goto[action]
%D \goto[action{argument}]
%D \stoptyping
%D
%D The first one is a normal reference, the second and third
%D are references to a file or \URL. The brace delimited
%D references for instance refer to a \JAVASCRIPT. The last
%D example shows that we can pass arguments to the actions.
%D
%D When we split off the components of such a reference, the
%D results are available in:
%D
%D \starttyping
%D \currentreferencespecial
%D \currentreferenceoperation
%D \currentreferencearguments
%D \currentinnerreference
%D \currentouterreference
%D \currentfullreference
%D \stoptyping

\newif\ifreferencefound

\let\currentfullreference     \empty
\let\currentreferencespecial  \empty
\let\currentreferenceoperation\empty
\let\currentreferencearguments\empty
\let\currentouterreference    \empty
\let\currentinnerreference    \empty

\def\setreferencevariables#1#2#3#4#5%
 {\def\currentreferencespecial  {#1}%
  \def\currentreferenceoperation{#2}%
  \def\currentreferencearguments{#3}%
  \def\currentouterreference    {#4}%
  \def\currentinnerreference    {#5}}

%D Now we've come to the testing step. As we can see below,
%D this macro does bit more than testing: it also resolves
%D the reference. This means that whenever we test for the
%D existance of a reference at an outer level, we have all the
%D relevant properties of that reference avaliable inside the
%D true branche~(\type{#2}).
%D
%D The prefix has to do with localizing references. When a
%D prefix is set, looking for a reference comes to looking for
%D the prefixed one, and when not found, looking for the non
%D prefixed one. Consider for instance the prefix set to
%D \type{sidetrack}.
%D
%D \starttyping
%D \pagereference[important]
%D \pagereference[unimportant]
%D \setupreferencing[prefix=sidetrack]
%D \pagereference[important]
%D \stoptyping
%D
%D results in saving (writing) the references
%D
%D \starttyping
%D ...{}{important}
%D ...{}{unimportant}
%D ...{sidetrack}{important}...
%D \stoptyping
%D
%D Now when we call for \type{unimportant}, we will indeed get
%D the pagenumber associated to this reference. But when we
%D call for \type{important}, while the prefix is still set, we
%D will get the pagenumber bound to the prefixed one.
%D
%D {\em Some day, when processing time and memory are no longer
%D performance factors, we will introduce multi||level
%D prefixes.}
%D
%D Before we start analyzing, I introduce a general
%D definition macro. Consider:
%D
%D \starttyping
%D \goto{do}[JS(My_Script{"test",123}),titlepage]
%D \stoptyping
%D
%D This can also be achieved by:
%D
%D \starttyping
%D \definereference[startup][JS(My_Script{"test",123}),titlepage]
%D \goto{do}[REF(startup)]
%D \stoptyping
%D
%D Now is this is a handy feature or not?
%D
%D \showsetup{definereference}
%D
%D We can trace references by setting the next switch to
%D true.

\def\definereference
  {\dodoubleempty\dodefinereference}

\def\dodefinereference[#1][#2]%
  {\ctxlua{jobreferences.define("\referenceprefix","#1",\!!bs\detokenize{#2}\!!es)}}

\def\resetreference[#1]%
  {\ctxlua{jobreferences.reset("\referenceprefix","#1")}}

\def\setpagereference#1#2% name, specification
  {\ctxlua{jobreferences.define("","#1",\!!bs\v!page(\luaescapestring{#2})\!!es)}}

%D Chained references are defined as:
%D
%D \starttyping
%D \goto{somewhere}[JS(somescript),nextpage,JS(anotherscript)]
%D \stoptyping
%D
%D Actually supporting chains is up to the special driver. Here
%D we only provide the hooks.

\newif   \ifsecondaryreference
\newcount\nofsecondaryreferences

% the counter stuff should move to the (mkiv) backend

\def\doifreferencefoundelse#1%
  {\ctxlua{jobreferences.doifelse("\referenceprefix","#1")}}

\def\doprocessreferenceelse#1#2#3%
  {\doresetgotowhereever
   \nofsecondaryreferences\zerocount
   \def\primaryreferencefoundaction  {\secondaryreferencefalse#2}%
   \def\secondaryreferencefoundaction{\advance\nofsecondaryreferences\plusone\secondaryreferencetrue#2}%
   \def\referenceunknownaction       {#3}%
   \ctxlua{jobreferences.handle("\referenceprefix","#1")}%
   \doresetgotowhereever} % to prevent problems with direct goto's

%D The inner case is simple. Only two cases have to be taken
%D care of:
%D
%D \starttyping
%D \goto{some text}[reference]
%D \goto{some text}[prefix:reference]
%D \stoptyping
%D
%D References to other files however are treated strict or
%D tolerant, depending on their loading and availability:
%D
%D \starttyping
%D \useexternaldocument[somefile][filename][a nice description]
%D
%D \goto{checked   reference}[somefile::reference]
%D \goto{unchecked reference}[somefile::]
%D \goto{unchecked reference}[anotherfile::reference]
%D \stoptyping
%D
%D An unknown reference is reported on the screen, in the log
%D file and, when enabled, in the left margin of the text.

\def\reportreferenceerror#1#2% only once (keep track in lua)
  {\ifinpagebody \else
     \doifconcepttracing{\doifsomething{#2}{\inleft{\infofont\doboundtext{#2}{\dimexpr\leftmarginwidth-2em\relax}{..}->}}}%
   \fi
   \showmessage\m!references{#1}{[\referenceprefix][#2]}}

\def\unknownreference{\reportreferenceerror1}
\def\illegalreference{\reportreferenceerror4}

%D When a reference is not found, we typeset a placeholder
%D (two glyphs are often enough to represent the reference
%D text).

\def\dummyreference{{\tttf ??}}

%D To prevent repetitive messages concerning a reference
%D being defined, we set such an unknown reference to an empty
%D one after the first encounter.

%D Sometimes we want to temporary put a reference out of
%D order. An example can be found in the menu macros.
%D
%D \starttyping
%D \doifreferencepermittedelse{reference}{set}{true}{false}
%D \stoptyping
%D
%D The second argument can be a comma seperated list.

\let\permittedreferences\empty

        \def\doifreferencepermittedelse#1#2#3% ref found notfound
          {\doprocessreferenceelse{#1}
             {\donetrue
              \ifx\permittedreferences\empty \else
                \docheckifreferencepermitted{#1}%
              \fi
              \ifdone#2\else#3\fi}
             {#3\unknownreference{#1}}}

        \def\docheckifreferencepermitted#1%
          {\ifx\currentinnerreference\empty
             \ifx\currentouterreference\empty \else
               \doifinstring{\currentouterreference::}\permittedreferences\donefalse
             \fi
           \else\ifx\currentouterreference\empty
             \doifinstring{\currentinnerreference}\permittedreferences\donefalse
           \else
             \doifinstring{\currentouterreference::\currentinnerreference}\permittedreferences\donefalse
           \fi\fi}

%D Apart from cross references supplied by the user, \CONTEXT\
%D generates cross references itself. Most of them are not
%D saved as a reference, but stored with their source, for
%D instance a list or an index entry. Such automatically
%D generated, for the user invisible, references are called
%D {\em internal references}. The user supplied ones are
%D labeled as {\em external references}.
%D
%D A second important characteristic is that when we want to
%D support different backends (viewers), we need to support
%D named destinations as well as page numbers. I invite readers
%D to take a glance at the special driver modules to understand
%D the fine points of this. As a result we will deal with {\em
%D locations} as well as {\em real page numbers}. We explictly
%D call this pagenumber a real one, because it is independant
%D of the page numbering scheme used in the document.
%D
%D One of the reasons for \CONTEXT\ being the first \TEX\ base
%D macropackage to support sophisticated interactive \PDF\
%D files, lays in the mere fact that real page numbers are
%D available in most two pass data, like references, list data
%D and index entries.
%D
%D We will speak of \type{thisis...} when we are marking a
%D location, and \type{goto...} when we point to such a
%D location. The latter one can be seen as a hyperlink to the
%D former one. In the next macros one we use constructs like:
%D
%D \starttyping
%D \dostart...
%D \dostop...
%D \stoptyping
%D
%D Such macros are used to invoke the relevant specials from
%D the special driver modules (see \type{spec-ini}). The flag
%D \type{\iflocation} signals if we're in interactive mode.

\def\thisisdestination#1% destination
  {\iflocation \ifusepagedestinations \else
     \dostartthisislocation{#1}\dostopthisislocation
   \fi \fi}

\def\thisisrealpage#1% pagenumber
  {\iflocation
     \dostartthisisrealpage{#1}\dostopthisisrealpage
   \fi}

%D The previous tho macros were easy ones, opposite to their
%D counterparts. A common component in these is:
%D
%D \starttyping
%D \dohandlegoto{..}{..}{..}
%D \stoptyping
%D
%D Here data can be whatever needs highlighting, e.g. {\em
%D figure 2.4}, and the start and stop entries handle the
%D specials. The two \DIMENSIONS\ \type{\buttonwidth} and
%D \type{\buttonheight} have to be set when handling the
%D data~(\type{#2}).

\ifx\buttonheight\undefined \newdimen\buttonheight \fi
\ifx\buttonwidth \undefined \newdimen\buttonwidth  \fi

\def\gotodestination#1#2#3#4#5% url file destination page data
  {\iflocation
     \ifusepagedestinations
       \gotorealpage{#1}{#2}{\number#4}{#5}%
     \else
       \dohandlegoto
         {#5}%
         {\the\everyreference\dostartgotolocation\buttonwidth\buttonheight{#1}{#2}{#3}{\number#4}}%
         {\dostopgotolocation}%
     \fi
   \else
     {#5}%
   \fi}

        \def\gotorealpage#1#2#3#4% url file page data
          {\iflocation
             \dohandlegoto
               {#4}%
               {\dostartgotorealpage\buttonwidth\buttonheight{#1}{#2}{\number#3}}%
               {\dostopgotorealpage}%
           \else
             {#4}%
        \fi}

\def\gotoinnerpage#1#2% page data
  {\iflocation
     \dohandlegoto
       {#2}%
       {\dostartgotorealpage\buttonwidth\buttonheight\empty\empty{\number#1}}%
       {\dostopgotorealpage}%
   \else
     {#2}%
   \fi}

\def\gotoouterfilepage#1#2#3% file page data
  {\iflocation
     \dohandlegoto
       {#3}%
       {\dostartgotorealpage\buttonwidth\buttonheight\empty{#1}{\number#2}}%
       {\dostopgotorealpage}%
   \else
     {#3}%
   \fi}

%D \macros
%D   {setreferencefilename}
%D
%D This command can be used in the special drivers to
%D uppercase filenames. This is needed when one wants to
%D produce \CDROM's conforming to ISO9660. We consider is the
%D savest to enable this feature by default. We cannot handle
%D uppercase here, since the suffix is handled in the special
%D driver. Conversion is taken care of by:
%D
%D \starttyping
%D \setreferencefilename somefilename\to\SomeFileName
%D \stoptyping

\chardef\referencefilecase=0

        \def\setreferencefilename#1\to#2%
          {\ifcase\referencefilecase
             \edef#2{#1}%
           \or
             \uppercasestring#1\to#2%
           \or
             \lowercasestring#1\to#2%
           \else
             \edef#2{#1}%
           \fi}

%D Internal references can best be set using the next few
%D macros. Setting such references to unique values is
%D completely up to the macros that call them.
%D
%D \starttyping
%D \thisissomeinternal{tag}{identifier}
%D \gotosomeinternal  {tag}{identifier}{pagenumber}{text}
%D \stoptyping

\def\thisissomeinternal#1#2%  tag reference
  {\doifsomething{#2}{\thisisdestination{#1:#2}}}

\def\gotosomeinternal#1#2% #3#4
  {\gotodestination\empty\empty{#1:#2}}

%D An automatic mechanism is provided too:
%D
%D \starttyping
%D \thisisnextinternal{tag}
%D \gotonextinternal  {tag}{number}{pagenumber}{text}
%D \stoptyping
%D
%D The first macro increments a counter. The value of this
%D counter is available in the macro \type{\nextinternalreference}
%D and should be saved somewhere (for instance in a file) for
%D future reference. The second argument of
%D \type {\gotonextinternal} takes such a saved number. One can
%D turn on tracing these references, in which case the
%D references are a bit more verbose.

\newcount\locationcount

\newif\ifinternalnamedreferences \internalnamedreferencestrue

\def\nextinternalreference
  {\the\locationcount}

\def\setnextinternalreference
  {\global\advance\locationcount\plusone}

\def\thisisnextinternal#1% #1 will be removed when we are done with mkiv
  {\ifinternalnamedreferences
     \thisisdestination{\s!aut:\nextinternalreference}%
   \fi}

\def\insertnextinternal#1%
  {\ifinternalnamedreferences
     \thisisdestination{\s!aut:\number#1}%
   \fi}

\def\gotonextinternal#1#2#3#4% #1 will be removed when we are done with mkiv
  {\ifinternalnamedreferences
     \gotodestination\empty\empty{\s!aut:#2}{#3}{#4}%
   \else
     \gotorealpage\empty\empty{#3}{#4}%
   \fi}

%D We already went through a lot of problems to sort out what
%D kind of reference we're dealing with. Sorting out the user
%D supplied cross references (show/goto this or that) as well
%D as user supplied system references (invoke this or that) is
%D already taken care of in the test routine, but we still have
%D to direct the request to the right (first) routine.

\def\gotolocation#1#2{\doprocessreferenceelse{#1}{#2}{\unknownreference{#1}}} % obsolete

%D An inner reference refers to some place in the document
%D itself.

        \def\gotoinnerlocation#1% #2%
          {\gotodestination\empty\empty{\referenceprefix\currentinnerreference}\currentrealreference} % {#2}

\def\gotoinner#1#2#3% prefix inner page data
  {\gotodestination\empty\empty{#1#2}{#3}} % {#4}

%D The outer location refers to another document, specified as
%D file or \URL.

        \def\gotoouterlocation#1#2% % page checken!
          {\bgroup
           \let\referenceprefix\empty
           \setouterlocation\currentouterreference
           \ifx\currentinnerreference\empty
             \gotorealpage\otherURL\otherfile1{#2}%
           \else
             \gotodestination\otherURL\otherfile\currentinnerreference\currentrealreference{#2}%
           \fi
           \egroup}

\def\gotoouterfile#1#2% file location page data #3 #4
  {\doifelsenothing{#2}{\gotorealpage\empty{#1}}{\gotodestination\empty{#1}{#2}}}

\def\gotoouterfilepage#1% file page data
  {\gotorealpage\empty{#1}\empty}

\def\gotoouterfilelocation% file location page data
  {\gotodestination\empty}

\def\gotoouterurl#1#2% url args data #2
  {\gotodestination{#1}\empty{#2}1}

%D Special locations are those that are accessed by saying
%D things like:
%D
%D \starttyping
%D \goto{calculate total}[JS(summarize{10,23,56}]
%D \stoptyping
%D
%D After several intermediate steps this finally arrives at
%D the next macro and expands into (simplified):
%D
%D \starttyping
%D \gotoJSlocation{total{summarize{10,23,56}}}{calculate total}
%D \stoptyping
%D
%D The first argument is the full reference, the second one
%D is the text, in some kind of manipulated form. In practice
%D we split references, so we get:
%D
%D \starttyping
%D \gotoJSlocation{summarize{10,23,56}}{calculate}
%D \gotoJSlocation{summarize{10,23,56}}{total}
%D \stoptyping
%D
%D where \type{calculate} and \type{total} are colored, boxed
%D or whatever \type{\goto} is told to do.
%D
%D The macro \type{\gotoJSlocation} can use \type
%D {\currentreferenceoperation} (in our example
%D \type{summarize}) and \type{\currentreference} (here
%D being \type {10,23,56}) to perform its task.

        \def\gotospeciallocation
          {\executeifdefined{goto\currentreferencespecial location}\gobbleoneargument}

%D Such special macros can be defined by:

        \def\definespeciallocation#1%
          {\setvalue{goto#1location}}

%D The associated test is to be defined by:

\def\definespecialtest#1%
  {\setvalue{\s!do:\v!test:#1}}

%D This \type{\def} alike macro is to be used as:
%D
%D \starttyping
%D \definespeciallocation{JS}#1#2{... #1 ... #2 ...}
%D \stoptyping
%D
%D In module \type {java-ini} one can see that \type
%D {\gotoJSlocation} looks much like the previous goto
%D definitions.

%D In this module we define three system references: one for
%D handling navigational, viewer specific, commands, another
%D for jumping to special pages, like the first or last one,
%D and a third reference for linking tree like lists, like
%D tables of contents. The latter two adapt themselves to the
%D current state.
%D
%D An example of an action is:
%D
%D \starttyping
%D \goto{some action}[PreviousJump]
%D \stoptyping
%D
%D as well as:
%D
%D \starttyping
%D \goto{some text}[\v!action(PreviousJump]
%D \stoptyping

% compatibility hack

\def\setglobalsystemreference#1#2#3{\definereference[#2][\v!action(#3)]}

% action actions

\def\gotoactionspecial#1#2#3#4% special operation arguments data
  {\begingroup
   \iflocation
     \dohandlegoto
       {#4}%
       {\dostartexecutecommand\buttonwidth\buttonheight{#2}{#3}}%
       {\dostopexecutecommand}%
   \else
     #4%
   \fi
   \endgroup}

\def\gotopagespecial#1#2#3#4% page(n) page(+n) page(-n) page(file::1)
  {\begingroup
   \iflocation
     \doifnonzeropositiveelse{#2}
       {\doifinstringelse+{#2}
          {\edef\currenttargetpage{\the\numexpr\realpageno#2}}
          {\doifinstringelse-{#2}
             {\edef\currenttargetpage{\the\numexpr\realpageno#2}}
             {\edef\currenttargetpage{#2}}}}%
       {\edef\currenttargetpage{1}}%
     \docheckrealreferencepage\currenttargetpage % new
     \gotorealpage\empty\empty\currenttargetpage{#4}%
   \else
     #4%
   \fi
   \endgroup}

%D It is possible to disable the writing of references to the
%D utility file by setting:

\newif\ifreferencing \referencingtrue

%D One can also activate an automatic prefix mechanism. By
%D setting the \type{\prefix} variable to \type{+}, the prefix
%D is incremented, when set to \type{-} or empty, the prefix is
%D reset. Other values become the prefix.

\newcount\prefixcounter

%D These settings are accomplished by:
%D
%D \showsetup{setupreferencing}
%D
%D In interactive documents verbose references don't always
%D make sense (what is a page number in an unnumbered
%D document). By setting the \type{interaction} variable, one
%D can influences the way interactive references are set.

\chardef\autocrossfilereferences=0

\def\setupreferencing
  {\dosingleargument\dosetupreferencing}

\def\dosetupreferencing[#1]%
  {\getparameters
     [\??rf]
     [\c!prefix=\s!unknown,#1]%
   \processaction
     [\@@rfstate]
     [  \v!stop=>\referencingfalse,
       \v!start=>\referencingtrue]%
   \processaction
     [\@@rfinteraction]
     [   \v!all=>\let\dowantedreference\docompletereference,
       \v!label=>\let\dowantedreference\dolabelonlyreference,
        \v!text=>\let\dowantedreference\dotextonlyreference,
      \v!symbol=>\let\dowantedreference\dosymbolreference]%
   \chardef\autocrossfilereferences\zerocount
   \processaction
     [\@@rfautofile]
     [  \v!yes=>\chardef\autocrossfilereferences\plusone,
       \v!page=>\chardef\autocrossfilereferences\plustwo]%
   \chardef\referencefilecase\zerocount
   \processaction[\@@rfconvertfile]
     [  \v!yes=>\chardef\referencefilecase\plusone,
        \v!big=>\chardef\referencefilecase\plusone,
      \v!small=>\chardef\referencefilecase\plustwo]%
   \setupreferenceprefix[\@@rfprefix]%
   \doifelse\@@rfglobal\v!yes
     {\settrue \autoglobalfilereferences}
     {\setfalse\autoglobalfilereferences}}

\def\incrementreferenceprefix{+}
\def\decrementreferenceprefix{-}

\def\setupreferenceprefix[#1]%
  {\edef\@@rfprefix{#1}%
   \ifx\@@rfprefix\empty
     \let\referenceprefix\empty
   \else\ifx\@@rfprefix\incrementreferenceprefix
     \advance\prefixcounter \plusone % should be global
     \edef\referenceprefix{\the\prefixcounter:}%
     \let\@@rfprefix\s!unknown
   \else\ifx\@@rfprefix\decrementreferenceprefix
     \let\referenceprefix\empty
     \let\@@rfprefix\s!unknown
   \else\ifx\@@rfprefix\s!unknown
     % forget about it
   \else
     \edef\referenceprefix{\@@rfprefix:}%
   \fi\fi\fi\fi}

%D \macros
%D   {handlereferenceactions,
%D    collectreferenceactions}
%D
%D Sometimes we need to pass the actions connected to
%D references to variables instead of rectangular areas on
%D which one can click. The next macro collects the actions
%D and passes them to a handle. This is a rather dreadfull
%D hack!
%D
%D \starttyping
%D \handlereferenceactions{references}\handle
%D \stoptyping
%D
%D So, \type {\handle} does the final job, which in for
%D instance the \PDF\ drivers comes down to doing something
%D with \type {\lastPDFaction}.

\newif\ifcollectreferenceactions

\def\handlereferenceactions#1#2%
  {\doifsomething{#1}
     {\bgroup
      \collectreferenceactionstrue
      \doprocessreferenceelse{#1}{#2}{\unknownreference{#1}}%
      \egroup}}

%D The most straightforward way of retrieving references is
%D using \type{\ref}. Consider the reference:
%D
%D \startbuffer
%D \reference[my ref]{{Look}{Here}{I am}}
%D \stopbuffer
%D
%D \typebuffer
%D
%D \getbuffer
%D
%D We can ask for upto five reference components:
%D
%D \startbuffer
%D user page reference:  \ref[p][my ref]
%D text reference:       \ref[t][my ref]
%D real page reference:  \ref[r][my ref]
%D sub text reference:   \ref[s][my ref]
%D extra text reference: \ref[e][my ref]
%D \stopbuffer
%D
%D \typebuffer
%D
%D And get back:
%D
%D \startlines
%D \getbuffer
%D \stoplines

\def\ref{\dodoubleargument\doref}

\def\reftypep{\currentpagereference}
\def\reftypet{\currenttextreference}
\def\reftyper{\currentrealreference}
\def\reftypes{\currentsubtextreference}
\def\reftypee{\currentsubsubtextreference}

\def\doref[#1][#2]%
  {\ifsecondargument
%      \doifreferencefoundelse{#2}
%        {\executeifdefined{reftype#1}\reftypep}
%        {\unknownreference{#2}\dummyreference}%
   \else
     \dummyreference
   \fi}

%D We can typeset a reference using \type{\in}, \type{\at} and
%D \type{\about} and goto specific locations using
%D \type{\goto}. The last one does not make that much sense in
%D a paper document. To complicate things, \PLAIN\ \TEX\ also
%D implements an \type {\in} but fortunately that one only
%D makes sense in math mode.
%D
%D Typesetting the reference is a bit more complicated than one
%D would at first sight expect. This is due to the fact that we
%D distinguish three (five) alternative calls:
%D
%D \placefigure
%D   [here][three calls]
%D   {Three alternatives reference calls.}
%D   {\startcombination[1*3]
%D      {\framed{\type{ \in }}} {a}
%D      {\framed{\type{ \at }}} {b}
%D      {\framed{\type{\goto}}} {c}
%D    \stopcombination}
%D
%D \startbuffer
%D \in figure[fig:three calls]
%D \in{figure}[fig:three calls]
%D \in figure a[fig:three calls]
%D \in{figure}{a}[fig:three calls]
%D figure~\in[fig:three calls]
%D \stopbuffer
%D
%D \typebuffer
%D
%D This turns up as:
%D
%D \startlines
%D \getbuffer
%D \stoplines
%D
%D The dual \type{{}} results in a split reference. In a
%D document meant for paper, one is tempted to use the last
%D (most straightforward) alternative. When a document is also
%D meant voor electronic distribution, the former alternatives
%D have preference, because everything between the \type{\in}
%D and~\type{[} becomes active (and when asked for, typeset
%D in a different color and typeface).

\definecommand in    {\dospecialin}
\definecommand at    {\dospecialat}
\definecommand about {\dospecialabout}
\definecommand from  {\dospecialfrom}
\definecommand over  {\dospecialabout} % needed here, else math problems

\def\currentreferencenumber{\ctxlua{jobreferences.filter("number")}}
\def\currentreferencepage  {\ctxlua{jobreferences.filter("page")}}
\def\currentreferencetitle {\ctxlua{jobreferences.filter("title")}}

\unexpanded\def\dospecialin{\doinatreference\currentreferencenumber}
\unexpanded\def\dospecialat{\doinatreference\currentreferencepage}

\def\doinatreference#1%
  {\doifnextoptionalelse{\dodoinatreference{#1}{}}{\dodoinatreference{#1}}}

\def\dodoinatreference#1%
  {\def\dododoinatreference{\dodododoinatreference{#1}}%
   \futurelet\next\dododoinatreference}

\unexpanded\def\dospecialabout[#1]%
  {\dontleavehmode
   \bgroup
   \@@rfleft
   \doprocessreferenceelse{#1}
     {\let\crlf\space
      \let\\\space
      \let\dogotofixed\dogotospace
      \dogotospace{\limitatetext\currentreferencetitle\@@rfwidth\unknown}[#1]}
     {\unknownreference{#1}\dummyreference}%
   \@@rfright
   \referenceinfo{<}{#1}%
   \egroup}

%D We arrived at the last step. Before we do the typesetting,
%D we forget all previous (paragraph bound) settings and make
%D sure that we remain in horizontal mode. Next we choose
%D among the several representations.

%D The previously discussed setup macro lets us specify the
%D representation of references. A symbol reference does not
%D show the specific data, like the number of a figure, but
%D shows one of: \hbox {$^\goforwardcharacter$
%D $^\gobackwardcharacter$ $^\gonowherecharacter$}, depending
%D on the direction to go.

        \def\dosymbolreference#1#2[#3]% todo
          {\bgroup
           \setupsymbolset[\@@iasymbolset]%
           \removelastskip
           \ifx\currentreferencespecial\empty
             \ifx\currentouterreference\empty
               \ifnum0\currentrealreference=\zerocount
                 \ifhmode\strut\high{\symbol[\v!nowhere]}\fi
               \else\ifnum0\currentrealreference>\realpageno
                 \dodosymbolreference{#2}{\high{\symbol[\v!next]}}%
               \else\ifnum0\currentrealreference<\realpageno
                 \dodosymbolreference{#2}{\high{\symbol[\v!previous]}}%
               \else
                 \ifhmode\strut\high{\symbol[\v!nowhere]}\fi
               \fi\fi\fi
             \else
               \gotoouterlocation{#3}{\showlocation{\high{\symbol[\v!somewhere]}}}%
             \fi
           \else
             \gotospeciallocation{#3}{\showlocation{\high{\symbol[\v!somewhere]}}}%
           \fi
        \egroup}

        \def\dodosymbolreference#1#2% todo
          {#1\hbox{\gotorealpage\empty\empty\currentrealreference{\dolocationattributes\??ia\c!style\c!color{#2}}}}

%D The other alternatives just conform their names: only the
%D label, only the text, or the label and the text.

\def\dounknownreference#1#2[#3]%
  {\unknownreference{#3}\dotextprefix{#2}\dummyreference}%

\def\docompletereference#1#2[#3]%
  {\iflocationsplit
     \doifsomespaceelse{#2}\dogotospace\dogotofixed{\dotextprefix{#2}#1}[#3]%
   \else
     \dogotofixed{\dotextprefix{#2}#1}[#3]%
   \fi}

\def\dolabelonlyreference#1#2[#3]%
  {\doifsomespaceelse{#2}
     {\doifsomething{#2}{\dogotospace{#2}[#3]}}
     {\dogotofixed{\dotextprefix{#2}}[#3]}}

\def\dotextonlyreference#1#2[#3]%
  {\dotextprefix{#2}\dogotofixed{#1}[#3]}

\let\dowantedreference\docompletereference

%D \macros
%D   {definereferenceformat}
%D
%D The next few macros were made for for David Arnold and Taco
%D Hoekwater. They can be used for predefining reference
%D texts, and thereby stimulate efficiency.
%D
%D [more documentation will be added]
%D
%D \starttyping
%D \definereferenceformat[informula]  [left=(,right=),text=formula]
%D \definereferenceformat[informulas] [left=(,right=),text=formulas]
%D \definereferenceformat[andformula] [left=(,right=),text=and]
%D \definereferenceformat[andformulas][left=(,right=),text=and]
%D
%D \informula [b] and \informula [for:c]
%D the \informula {formulas}[b] \informula {and} [for:c]
%D the \informulas {formulas}[b] \informula {and} [for:c]
%D the \informulas [b] \informula {en} [for:c]
%D the \informulas [b] \andformula [for:c]
%D \stoptyping
%D
%D Instead of a text, one can specify a label, which should
%D be defined with \type {\setuplabeltext}.

% todo: inherit

\def\definereferenceformat
  {\dodoubleargument\dodefinereferenceformat}

\def\dodefinereferenceformat[#1][#2]%
  {\iffirstargument
     \getparameters[\??rf#1]
       [\c!left=,  % of the number
        \c!right=, % of the number
        \c!text=,  % before the number
        \c!label=, % can be {left}{right}
        \c!command=\in,
        #2]%
     \unexpanded\setvalue{#1}%
       {\dontleavehmode\doexecutereferenceformat{#1}}%
   \fi}

\def\noexecutelabelreferenceformat#1%
  {\doifvaluesomething{\??rf#1\c!text}
     {\gdef\textofreference{\csname\??rf#1\c!text\endcsname}}%
   \csname\??rf#1\c!command\endcsname}

\def\doexecutelabelreferenceformat#1%
  {\csname\??rf#1\c!command\endcsname
     {\leftlabeltext {\csname\??rf#1\c!label\endcsname}}%
     {\rightlabeltext{\csname\??rf#1\c!label\endcsname}}}

\def\doexecutereferenceformat#1%
  {\gdef\leftofreference {\csname\??rf#1\c!left \endcsname}%
   \gdef\rightofreference{\csname\??rf#1\c!right\endcsname}%
   \global\let\textofreference\empty % otherwise ~ added
   \doifelsevaluenothing{\??rf#1\c!label}
     \noexecutelabelreferenceformat\doexecutelabelreferenceformat{#1}}

\let\leftofreference \relax
\let\rightofreference\relax
\let\textofreference \relax

% fails on metafun  {\leftofreference#1\ignorespaces#3\removeunwantedspaces\rightofreference}{#2}[#4]%

\def\dodododoinatreference#1#2#3[#4]% no \removeunwantedspaces (fails on metafun)
  {\ifx\next\bgroup
     \dododododoinatreference{\leftofreference#1\ignorespaces#3\rightofreference}{#2}[#4]%
   \else
     \dododododoinatreference{\leftofreference#1\rightofreference}{#2#3}[#4]%
   \fi}

\let\dosymbolreference\dowantedreference

\def\dododododoinatreference#1#2[#3]%
  {\dontleavehmode   % replaces \leaveoutervmode
   \begingroup
   \forgetall
   \postponenotes
   \doifreferencefoundelse{#3}
     {\doifelsenothing{#1}\dosymbolreference\dowantedreference{#1}{#2}[#3]}%
     {\dounknownreference{#1}{#2}[#3]}%
   \referenceinfo<{#3}%
   \endgroup}


%D In interactive documents going to a specific location is not
%D bound to cross references. The \type{\goto} commands can be
%D used to let users access another part of the document. In
%D this respect, interactive tables of contents and registers
%D can be considered goto's. Because in fact a \type{\goto} is
%D just a reference without reference specific data, the
%D previous macros are implemented using the goto
%D functionality.
%D
%D \showsetup{goto}
%D
%D One important chaacteristic is that the first argument of
%D \type{\goto} (and therefore \type{\at} and \type{\in} is
%D split at spaces. This means that, although hyphenation is
%D prevented, long references can cross line endings.

\newif\ifsharesimilarreferences \sharesimilarreferencestrue
\newcount\similarreference % 0=noppes 1=create/refer 2,3,..=refer

\unexpanded\def\goto#1#2%
  {\dogoto{#1}#2}

\def\dogoto#1[#2]%
  {\dontleavehmode
   \bgroup
   \postponenotes
   % todo: handle empty #1
   \doifelsenothing{#1}
     {\dosymbolreference{}{}[#2]}
     {\dogotospace{#1}[#2]}%
   \egroup
   \referenceinfo{<}{#2}}

% inefficient, we need to save the shared one (just reuse last command in lua)

\def\dogotospace#1[#2]%
  {\iflocationsplit
     \ifsecondaryreference
       \setbox\scratchbox\hbox % will change anyway
     \fi % due to space insertion
       {\let\dogotospace\dogotofixed
        \iflocation
          \def\processisolatedword##1%
            {\ifisolatedwords\ifsharesimilarreferences
               \global\advance\similarreference \plusone
             \fi\fi
             \hbox\bgroup
               \doprocessreferenceelse{#2}{##1\presetgoto}{\unknownreference{#2}##1\relax}%
             \egroup}%
          \doattributes\??ia\c!style\c!color{\processisolatedwords{#1}\processisolatedword}%
        \else
          #1\relax % \relax prevents #1's next macros from gobbling \fi
        \fi}%
   \else
     \iflocation
       \hbox{\doattributes\??ia\c!style\c!color{\doprocessreferenceelse{#2}{#1\presetgoto}{\unknownreference{#2}#1\relax}}}%
     \else
       #1\relax % \relax prevents #1's next macros from gobbling \fi
     \fi
   \fi
   \global\similarreference\zerocount}

\def\dogotofixed#1[#2]%
  {{\iflocation
      \hbox{\doattributes\??ia\c!style\c!color{\doprocessreferenceelse{#2}{#1\presetgoto}{\unknownreference{#2}#1\relax}}}%
    \else
      #1%
    \fi}}

%D In case the auto split feature is not needed or even not
%D even wanted, \type{\gotobox} can be used.

\unexpanded\def\gotobox#1[#2]%
  {\dontleavehmode
   \bgroup
   \locationstrutfalse
   \doprocessreferenceelse{#2}
     {\dogotofixed{#1}[#2]}
     {\hbox{\unknownreference{#2}#1}}%
   \referenceinfo{<}{#2}%
   \egroup}

%D An reference to another document can be specified as a file
%D or as an \URL. Both are handled by the same mechanism and
%D can be issued by saying something like:
%D
%D \starttyping
%D \goto[dictionary::the letter a]
%D \stoptyping
%D
%D One can imagine that many references to such a dictionary
%D are made, so in most cases such a document reference in an
%D indirect one.
%D
%D \showsetup{useexternaldocument}
%D
%D For example:
%D
%D \starttyping
%D \useexternaldocument
%D   [dictionary][engldict]
%D   [The Famous English Dictionary]
%D \stoptyping
%D
%D The next macro implements these relations, and also take
%D care of loading the document specific references.
%D
%D The \URL\ alternative takes four arguments:
%D
%D \showsetup{useURL}
%D
%D like:
%D
%D \starttyping
%D \useURL
%D   [dictionary][http://www.publisher.com/public][engldict]
%D   [The Famous English Dictionary]
%D \stoptyping
%D
%D Several specifications are possible:
%D
%D \starttyping
%D \useURL [id] [url] [file] [description]
%D \useURL [id] [url] [file]
%D \useURL [id] [url]
%D \stoptyping
%D
%D This time we don't load the references when no file is
%D specified. This is logical when one keeps in mind that a
%D valid \URL\ can also be a mail address.

\def\usefile{\dotripleargument\dousefile}
\def\useurl {\doquadrupleempty\douseurl}

\let\useURL             \useurl
\let\useexternaldocument\usefile

\def\douseurl[#1][#2][#3][#4]%
  {\ctxlua{jobreferences.urls.define("#1",\!!bs\detokenize{#2}\!!es,\!!bs\detokenize{#3}\!!es,\!!bs\detokenize{#4}\!!es)}}

\def\dousefile[#1][#2][#3]%
  {\ctxlua{jobreferences.files.define("#1",\!!bs\detokenize{#2}\!!es,\!!bs\detokenize{#3}\!!es)}}

% \doifsomething\@@urstyle{\let\@@iastyle\@@urstyle\let\@@urstyle\empty}%
% \doifsomething\@@urcolor{\let\@@iacolor\@@urcolor\let\@@urcolor\empty}%

%D \macros
%D   {url,setupurl}
%D
%D We also have: \type{\url} for directly calling the
%D description. So we can say:
%D
%D \starttyping
%D \useURL [one] [http://www.test.nl]
%D \useURL [two] [http://www.test.nl] [] [Some Site]
%D
%D \url[one] or \from[two] or \goto{Whatever Site}[URL(two)]
%D \stoptyping
%D
%D An \URL\ can be set up with
%D
%D \showsetup{setupurl}

\def\setupurl
  {\dodoubleargument\getparameters[\??ur]}

\unexpanded\def\url[#1]%
  {\dontleavehmode
   \begingroup
   \dosetfontattribute\??ur\c!style
   \dosetcolorattribute\??ur\c!color
   \ctxlua{jobreferences.urls.get("#1","\@@uralternative","\@@urspace")}%
   \dostopattributes
   \endgroup}

%D This macro is hooked into a support macro, and thereby
%D \URL's break ok, according to the setting of a switch,
%D
%D \startbuffer
%D \useURL
%D   [test]
%D   [sentence_sentence%sentence#sentence~sentence/sentence//sentence:sentence.sentence]
%D \stopbuffer
%D
%D \typebuffer
%D
%D Such an \URL\ is, depending on the settings, hyphenated as:
%D
%D \getbuffer
%D
%D \startlinecorrection
%D \hbox to \hsize
%D   {\hss\en
%D    \setupreferencing[urlalternative=both]%
%D    \vbox{\hsize.25cm\hbox{\bf both}\prewordbreak\url[test]}%
%D    \hss
%D    \setupreferencing[urlalternative=before]%
%D    \vbox{\hsize.25cm\hbox{\bf before}\prewordbreak\url[test]}%
%D    \hss
%D    \setupreferencing[urlalternative=after]%
%D    \vbox{\hsize.25cm\hbox{\bf after}\prewordbreak\url[test]}%
%D    \hss}
%D \stoplinecorrection
%D
%D By setting \type{urlspace=yes} one can get slightly better
%D spacing when using very long \URL's.
%D
%D When defining the external source of information, one can
%D also specify a suitable name (the last argument). This name
%D can be called upon with:
%D
%D \showsetup{from}

\def\dospecialfrom
  {\dosingleempty\dodospecialfrom}

\def\dodospecialfrom[#1]%
  {\dontleavehmode\ctxlua{jobreferences.from("#1","","")}}

%D We also support:
%D
%D \starttyping
%D \goto{some text}[file(identifier{location}]
%D \stoptyping
%D
%D which is completely equivalent with
%D
%D \starttyping
%D \goto{some text}[identifier::location]
%D \stoptyping

\def\gotofilespecial#1#2#3#4% special operation arguments data
  {\begingroup\iflocation\gotoouterfile{#2}{#3}{#4}\else#4\fi\endgroup}

\def\gotourlspecial#1#2#3#4% special operation arguments data
  {\begingroup\iflocation\gotoouterurl{#2}{#3}{#4}\else#4\fi\endgroup}

%D A special case of references are those to programs. These,
%D very system dependant references are implemented by abusing
%D some of the previous macros.
%D
%D \showsetup{setupprograms}
%D \showsetup{defineprogram}
%D \showsetup{program} % changed functionality !
%D
%D The latter gives access to the description of the program,
%D being the last argument to the definition command.

% also lua, like urls and files

\def\setupprograms
  {\dodoubleargument\getparameters[\??pr]}

\def\defineprogram
  {\dotripleargument\dodefineprogram}

\def\dodefineprogram[#1][#2][#3]%
  {\ctxlua{jobreferences.programs.define("#1","#2","#3")}}

\def\program[#1]% incompatible, more consistent, hardy used anyway
  {\dontleavehmode
   \begingroup
   \dosetfontattribute\??pr\c!style
   \dosetcolorattribute\??pr\c!color
   \ctxlua{jobreferences.programs.get("#1","\@@pralternative","\@@prspace")}%
   \endgroup}

% needs an update: program(abc{arg})

\def\gotoprogramspecial#1#2#3#4% special operation arguments data
  {\begingroup
   \iflocation
     \dohandlegoto
       {#4}%
       {\dostartrunprogram\buttonwidth\buttonheight{\@@prdirectory#2}{#3}}%
       {\dostoprunprogram}%
   \else
     #4%
   \fi
   \endgroup}

%D As we can see, we directly use the special reference
%D mechanism, which means that
%D
%D \starttyping
%D \goto{some text}[program(name{args})]
%D \stoptyping
%D
%D is valid.

%D The next macro provides access to the actual pagenumbers.
%D When documenting and sanitizing the original reference
%D macros, I decided to keep the present meaning as well as to
%D make this meaning available as a special reference method.
%D So now one can use:
%D
%D \starttyping
%D \gotopage{some text}[location]
%D \gotopage{some text}[number]
%D \gotopage{some text}[file::number]
%D \stoptyping
%D
%D as well as:
%D
%D \starttyping
%D \goto{some text}[page(location)]
%D \goto{some text}[page(number)]
%D \goto{some text}[file::page(number)]
%D \stoptyping
%D
%D Here location is a keyword like \type{nextpage}.
%D
%D \showsetup{gotopage}

\def\definepage
  {\dodoubleargument\dodefinepage}

\def\dodefinepage[#1][#2]%
  {\definereference[#1][page(#1)]}

\def\gotopage#1[#2]%
  {\goto{#1}[\v!page(#2)]}

%D The previous definitions are somewhat obsolete so we don't
%D use it here.

%D A still very rudimentary|/|experimental forward|/|backward
%D reference mechanism is provided by the macro \type{\atpage}:
%D
%D \starttyping
%D ... \somewhere{backward text}{forward text}[someref] ...
%D ... \atpage[someref] ...
%D \stoptyping
%D
%D In future versions there will be more sophisticated

%D support, also suitable for references to floating bodies.

\def\analysedreference#1%
  {\ctxlua{jobreferences.analysis("\referenceprefix","#1")}}

\unexpanded\def\somewhere#1#2#3[#4]% #3 gobbles space around #2 % todo
  {\dontleavehmode
   \ifcase\analysedreference{#4}\relax
     \unknownreference{#4}#1/#2%
   \or
     \doifelsenothing{#2}{\dosymbolreference{}{}[#4]}{\dogotospace{#2}[#4]}%
   \or % forward
     \doifelsenothing{#1}{\dosymbolreference{}{}[#4]}{\dogotospace{#1}[#4]}%
   \or % backward
     \doifelsenothing{#2}{\dosymbolreference{}{}[#4]}{\dogotospace{#2}[#4]}%
   \fi
   \referenceinfo{<}{#4}}

\unexpanded\def\atpage[#1]% todo
  {\dontleavehmode
% \docheckrealreferencepage{}%
%    \doifreferencefoundelse{#1}
%      {\ifrealreferencepage
%         \ifforwardreference
%           \dogotofixed{\labeltext\v!hencefore}[#1]%
%         \else
%           \dogotofixed{\labeltext\v!hereafter}[#1]%
%         \fi
%       \else
%         \dogotofixed{\labeltexts\v!atpage\currentpagereference}[#1]%
%       \fi}
%      {\unknownreference{#1}%
%       \labeltexts\v!page\dummyreference}%
   \referenceinfo{<}{#1}}

%D We can cross link documents by using:
%D
%D \showsetup{coupledocument}
%D
%D like:
%D
%D \starttyping
%D \coupledocument[print][somefile][chapter,section]
%D \stoptyping
%D
%D After which when applicable, we have available the
%D references:
%D
%D \starttyping
%D \goto{print version}[print::chapter]
%D \stoptyping
%D
%D and alike. The title placement definition macros have a
%D key \type{file}, which is interpreted as the file to jump
%D to, that is, when one clicks on the title.

\newif\ifautocrossdocument

\def\coupledocument
  {\doquadrupleempty\docoupledocument}

\def\docoupledocument[#1][#2][#3][#4]% [name] [file] [sections] [description]
  {\ifthirdargument
     % this will be done differently (when it's needed)
   \fi}

%D Buttons are just what their names says: things that can be
%D clicked (pushed) on. They are similar to \type{\goto},
%D except that the text argument is not interpreted.
%D Furthermore one can apply anything to them that can be done
%D with \type{\framed}.
%D
%D \startbuffer
%D \button[width=3cm,height=1.5cm]{Exit}[ExitViewer]
%D \stopbuffer
%D
%D \typebuffer
%D
%D gives
%D
%D \getbuffer
%D
%D This command is formally specified as:
%D
%D \showsetup{button}
%D
%D The characteristics can be set with:
%D
%D \showsetup{setupbuttons}

\def\setupbuttons
  {\dodoubleargument\getparameters[\??bt]}

\definecomplexorsimpleempty\button

\def\complexbutton
  {\docomplexbutton\??bt}

\presetlocalframed[\??bt]

\long\def\docomplexbutton#1[#2]#3#4% get rid of possible space before [#4]
  {\dodocomplexbutton#1[#2]{#3}#4} % #4 == [

\def\buttonframed{\dodoubleempty\localframed[\??bt]} % goodie

\long\def\dodocomplexbutton#1[#2]#3[#4]% #3 can contain [] -> {#3} later
  {\begingroup
   \doifvalue{#1\c!state}\v!stop\locationfalse
   \iflocation
     \resetgoto
     \ConvertConstantAfter\doifelse{#3}\v!none\hphantom\hbox
       {\doifelsenothing{#4}
          {\setlocationboxnop#1[#2]{#3}[#4]}
          {\doifreferencefoundelse{#4} % INEFFICIENT
             {\setlocationboxyes#1[#2]{#3}[#4]}
             {\unknownreference{#4}%
              \setlocationboxnop#1[#2]{#3}[#4]}}}%
   \fi
   \endgroup}

%D Interaction buttons, in fact a row of tiny buttons, are
%D typically only used for navigational purposed. The next
%D macro builds such a row based on a specification list.
%D
%D \startbuffer
%D \interactionbuttons
%D   [width=\hsize][page,PreviousJump,ExitViewer]
%D \stopbuffer
%D
%D \typebuffer
%D
%D gives
%D
%D \getbuffer
%D
%D Apart from individual entries, one can use \type{page} and
%D \type {subpage} as shortcuts to their four associated buttons.
%D The symbols are derived from the symbols linked to the
%D entries.

% does not work well with for instance SomeRef{whatever}

\def\interactionbuttons
  {\dodoubleempty\dointeractionbuttons}

\def\dointeractionbuttons[#1][#2]% er is een verdeel macro \horizontalfractions
  {\iflocation
     % BUG: fails when frame=off; best is to rewrite this macro
     \bgroup
     \doif\@@ibstate\v!stop\locationfalse
     \iflocation
       \ifsecondargument
         \setupinteractionbar[#1]%
         \checkinteractionbar{1.5em}\v!broad\!!zeropoint % brrrrr
         \setbox2\hbox{\localframed[\??ib][\c!background=]{\symbol[\@@iasymbolset][\v!previouspage]}}%
         \!!heighta\ht2 % needed because we default to nothing
         \setupinteractionbar[\c!strut=\v!no]%
         \setinteractionparameter\c!width\!!zeropoint
         \!!counta\zerocount % new, was 1
         \processallactionsinset
           [#2]
           [   \v!page=>\advance\!!counta 4,
            \v!subpage=>\advance\!!counta 4,
            \s!unknown=>\advance\!!counta 1]%
         \ifdim\@@ibwidth=\zeropoint
           \!!widtha2em
           \advance\!!widtha \@@ibdistance  % new
           \!!widthb\!!counta\!!widtha
           \advance\!!widthb -\@@ibdistance % new
         \else
           \!!widtha\@@ibwidth
           \!!widthb\@@ibdistance           % new
           \multiply\!!widthb \!!counta     % new
           \advance\!!widthb -\@@ibdistance % new
           \advance\!!widtha -\!!widthb     % new
           \divide\!!widtha \!!counta
           \!!widthb\@@ibwidth
         \fi
         \def\goto##1% clash ?
           {\setnostrut
            \edef\localreference{##1}%
            \normalexpanded{\noexpand\dodocomplexbutton\??ib[\c!height=\the\!!heighta,\c!width=\the\!!widtha]}%
              {\dontleavehmode\symbol[\@@iasymbolset][\localreference]}%
              [\localreference]%
            \hss}%
         \hbox to \!!widthb
           {\processallactionsinset
              [#2]
              [      \v!page=>\goto\v!firstpage
                              \goto\v!nextpage
                              \goto\v!previouspage
                              \goto\v!lastpage,
                  \v!subpage=>\goto\v!firstsubpage
                              \goto\v!nextsubpage
                              \goto\v!previoussubpage
                              \goto\v!lastsubpage,
                  \s!unknown=>\goto\commalistelement]%
            \unskip}%
       \else
         \interactionbuttons[][#1]%
       \fi
     \fi
     \egroup
   \fi}

%D \macros
%D   {overlaybutton}
%D
%D For converience we provide:
%D
%D \starttyping
%D \overlaybutton[reference]
%D \stoptyping
%D
%D This command can be used to define overlays an/or can be
%D used in the whatevertext areas, like:
%D
%D \starttyping
%D \defineoverlay[PrevPage][\overlaybutton{PrevPage}]
%D \setupbackgrounds[page][background=PrevPage]
%D \setuptexttexts[\overlaybutton{NextPage}]
%D \stoptyping
%D
%D For practical reasons, this macro accepts square brackets
%D as well as braces.

\definecomplexorsimple\overlaybutton

\def\simpleoverlaybutton#1%
  {\complexoverlaybutton[#1]}

\def\complexoverlaybutton[#1]%
  {\iflocation
     \doprocessreferenceelse{#1}
       {\overlayfakebox  {#1}}
       {\unknownreference{#1}}%
   \fi}

\def\overlayfakebox#1%
  {\hbox
     {\setbox\scratchbox\null
      \wd\scratchbox\overlaywidth
      \ht\scratchbox\overlayheight
      \locationstrutfalse
      \box\scratchbox}}

%D \macros
%D   {dotextprefix}
%D
%D In previous macros we used \type {\dotextprefix} to
%D generate a space between a label and a number.
%D
%D \starttyping
%D \dotextprefix{text}
%D \stoptyping
%D
%D Only when \type {text} is not empty, a space is inserted.

\def\dotextprefix#1%
  {\begingroup
   \global\labeltextdonefalse  % this is an ugly dependancy,
   \setbox\scratchbox\hbox{#1}% to be solved some day
   \ifdim\wd\scratchbox>\zeropoint
     \unhbox\scratchbox
     \iflabeltextdone\else\@@rfseparator\fi
   \else
     \unhbox\scratchbox
   \fi
   \endgroup}

%D In the next settings we see some variables that were not
%D used here and that concern the way the pagenumbers refered
%D to are typeset.

\setupreferencing
  [\c!state=\v!start,
   \c!autofile=\v!no,
   \v!part\c!number=\v!yes,
   \v!chapter\c!number=\v!no,
   \c!interaction=\v!all,
   \c!convertfile=\v!no,
  %\c!strut=\v!no, % some day an option
   \c!prefix=,
   \c!width=.75\makeupwidth,
   \c!left=\quotation\bgroup,
   \c!right=\egroup,
   \c!global=\v!no,
   \c!expansion=\v!no,
   \c!separator=\nonbreakablespace]

\setupurl
  [\c!alternative=\v!both,
   \c!space=\v!no,
   \c!style=\v!type,
   \c!color=]

\setupprograms
  [\c!directory=,
   \c!alternative=\v!both,
   \c!space=\v!no,
   \c!style=\v!type,
   \c!color=]

\definereference [\v!CloseDocument    ] [action(close)]
\definereference [\v!ExitViewer       ] [action(exit)]
\definereference [\v!FirstPage        ] [action(first)]
\definereference [\v!LastPage         ] [action(last)]
\definereference [\v!NextJump         ] [action(forward)]
\definereference [\v!NextPage         ] [action(next)]
\definereference [\v!PauseMovie       ] [action(pausemovie)]
\definereference [\v!PauseSound       ] [action(pausesound)]
\definereference [\v!PauseRendering   ] [action(pauserendering)]
\definereference [\v!PreviousJump     ] [action(backward)]
\definereference [\v!PreviousPage     ] [action(previous)]
\definereference [\v!PrintDocument    ] [action(print)]
\definereference [\v!SaveForm         ] [action(exportform)]
\definereference [\v!LoadForm         ] [action(importform)]
\definereference [\v!ResetForm        ] [action(resetform)]
\definereference [\v!ResumeMovie      ] [action(resumemovie)]
\definereference [\v!ResumeSound      ] [action(resumesound)]
\definereference [\v!ResumeRendering  ] [action(resumerendering)]
\definereference [\v!SaveDocument     ] [action(save)]
\definereference [\v!SaveNamedDocument] [action(savenamed)]
\definereference [\v!OpenNamedDocument] [action(opennamed)]
\definereference [\v!SearchDocument   ] [action(search)]
\definereference [\v!SearchAgain      ] [action(searchagain)]
\definereference [\v!StartMovie       ] [action(startmovie)]
\definereference [\v!StartSound       ] [action(startsound)]
\definereference [\v!StartRendering   ] [action(startrendering)]
\definereference [\v!StopMovie        ] [action(stopmovie)]
\definereference [\v!StopSound        ] [action(stopsound)]
\definereference [\v!StopRendering    ] [action(stoprendering)]
\definereference [\v!SubmitForm       ] [action(submitform)]
\definereference [\v!ToggleViewer     ] [action(toggle)]
\definereference [\v!ViewerHelp       ] [action(help)]
\definereference [\v!HideField        ] [action(hide)]
\definereference [\v!ShowField        ] [action(show)]
\definereference [\v!GotoPage         ] [action(gotopage)]
\definereference [\v!GotoPage         ] [action(gotopage)]
\definereference [\v!Query            ] [action(query)]
\definereference [\v!QueryAgain       ] [action(queryagain)]
\definereference [\v!FitWidth         ] [action(fitwidth)]
\definereference [\v!FitHeight        ] [action(fitheight)]
\definereference [\v!ShowThumbs       ] [action(thumbnails)]
\definereference [\v!ShowBookmarks    ] [action(bookmarks)]

\definereference [\v!firstpage]       [page(\firstpage)]
\definereference [\v!previouspage]    [page(\prevpage)]
\definereference [\v!nextpage]        [page(\nextpage)]
\definereference [\v!lastpage]        [page(\lastpage)]
\definereference [\v!firstsubpage]    [page(\firstsubpage)]
\definereference [\v!previoussubpage] [page(\prevsubpage)]
\definereference [\v!nextsubpage]     [page(\nextsubpage)]
\definereference [\v!lastsubpage]     [page(\lastsubpage)]
\definereference [\v!first]           [page(\firstpage)]
\definereference [\v!previous]        [page(\prevpage)]
\definereference [\v!next]            [page(\nextpage)]
\definereference [\v!last]            [page(\lastpage)]
\definereference [\v!first\v!sub]     [page(\firstsubpage)]
\definereference [\v!previous\v!sub]  [page(\prevsubpage)]
\definereference [\v!next\v!sub]      [page(\nextsubpage)]
\definereference [\v!last\v!sub]      [page(\lastsubpage)]

%D We cannot set up buttons (not yet, this one calls a menu macro):

\protect \endinput
