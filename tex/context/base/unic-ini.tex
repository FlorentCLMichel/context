%D \module
%D   [       file=unic-ini,
%D        version=2002.12.03,
%D          title=\CONTEXT\ \UNICODE\ Support,
%D       subtitle=\UNICODE\ \& UTF-8 support,
%D         author=Hans Hagen,
%D           date=\currentdate,
%D      copyright={PRAGMA / Hans Hagen \& Ton Otten}]
%C
%C This module is part of the \CONTEXT\ macro||package and is
%C therefore copyrighted by \PRAGMA. See mreadme.pdf for
%C details.

\writestatus{loading}{Context UNICODE Macros (ini)}

%D Sorry, we only support his in \ETEX.

\beginTEX
  \def\useunicodevector[#1]{}
  \expandafter \endinput
\endTEX

\unprotect

%D This module deals with unicode, and in particular with
%D \UTF-8 conversion. The prelude to this module was \type
%D {xtag-utf}, which is now replaced by a one||liner. The
%D macros below deal with conversions. Thanks to Taco for
%D providing the following conversion rules.
%D
%D \starttabulate[|c|c|c|c|c|]
%D \NC  $b_1$  \NC  $b_2$  \NC  $b_3$  \NC  $b_4$  \NC     unicode     \NC \NR
%D \NC192---223\NC128---191\NC         \NC         \NC   0x80---0x7FF  \NC \NR
%D \NC224---239\NC128---191\NC128---191\NC         \NC  0x800---0xFFFF \NC \NR
%D \NC240---247\NC128---191\NC128---191\NC128---191\NC0x10000---0x1FFFF\NC \NR
%D \stoptabulate
%D
%D In \UTF-8 the characters in the range 128---191 are illegal
%D as first characters. The characters 254 and 255 are
%D completely illegal and should not appear at all (they are
%D related to UTF16).
%D
%D The unicode number for an \UTF-8 sequence can be calculated
%D as follows:
%D
%D \starttabulate[|mc|m|mc|m|mc|m|mc|m|]
%D \NC       b_1     \NC \NC             \NC \NC           \NC \NC         \NC if     b_1<=127\NC \NR
%D \NC    64(b_1-192)\NC+\NC    (b_2-128)\NC \NC           \NC \NC         \NC if 192<=b1<=223\NC \NR
%D \NC  4096(b_1-224)\NC+\NC  64(b_2-128)\NC+\NC  (b_3-128)\NC \NC         \NC if 224<=b1<=239\NC \NR
%D \NC262144(b_1-240)\NC+\NC4096(b_2-128)\NC+\NC64(b_3-128)\NC+\NC(b_4-128)\NC if 240<=b1<=247\NC \NR
%D \stoptabulate
%D
%D A lot of information abou tunicode can be found on the
%D web (search for Markus Kuhn and unicode and you'll
%D probably end up at the right place).

%D In \ETEX\ vocabulary such a conversion looks as follows.
%D We need the \type {`} in order to turn a character into a
%D number.
%D
%D \starttyping
%D \def\utftwounicode#1#2%
%D   {\number\numexpr    (64*(\numexpr (#1-192))+%
%D                           (\numexpr(`#2-128)))}
%D
%D \def\utfthreeunicode#1#2#3%
%D   {\number\numexpr  (4096*(\numexpr (#1-224))+
%D                        64*(\numexpr(`#2-128))+%
%D                           (\numexpr(`#3-128)))}
%D
%D \def\utffourunicode#1#2#3#4%
%D   {\number\numexpr(262144*(\numexpr (#1-240))+
%D                      4096*(\numexpr(`#1-128))+
%D                        64*(\numexpr(`#2-128))+%
%D                           (\numexpr(`#3-128)))}
%D \stoptyping
%D
%D When we map the unicode number on one of the 256 char wide
%D unicode tables, we need to do a bit of div and mod. Watch
%D out: an \ETEX\ \type {/} is not the same as \TEX's \type
%D {\divide}. The former rounds, while the later truncates, so
%D we need to trucate ourselves. In case you wonder why we
%D use \type {\numexpr}: this is not only more convenient, but
%D also makes it possible to avoid scratch counters, so that we
%D get fast and fully expandable conversions.
%D
%D \starttyping
%D \def\utfdiv#1{\number\numexpr((#1-128)/256)}
%D \def\utfmod#1{\number\numexpr((#1)-(256*(\utfdiv{#1})))}
%D \stoptyping
%D
%D So far for the readable alternatives. When using \type
%D {\numexpr} you should be aware of rather unexpected look
%D ahead effects. The next implementation uses registers,
%D which saves tokens and is faster. In this case we gain
%D some 10\% time.

\chardef       \utf@a=64
\mathchardef   \utf@b=4096
\newcount\utf@c\utf@c=262144

\chardef       \utf@d=192
\chardef       \utf@e=224
\chardef       \utf@f=240

\chardef       \utf@g=128
\mathchardef   \utf@h=256
\chardef       \utf@i=127

%D The definitions now become:
%D
%D \starttyping
%D \def\utftwounicode#1#2%
%D   {\number\numexpr(\utf@a*(\numexpr (#1-\utf@d))+%
%D                           (\numexpr(`#2-\utf@g)))}
%D
%D \def\utfthreeunicode#1#2#3%
%D   {\number\numexpr(\utf@b*(\numexpr (#1-\utf@e))+
%D                    \utf@a*(\numexpr(`#2-\utf@g))+%
%D                           (\numexpr(`#3-\utf@g)))}
%D
%D \def\utffourunicode#1#2#3#4%
%D   {\number\numexpr(\utf@c*(\numexpr (#1-\utf@f))+
%D                    \utf@b*(\numexpr(`#2-\utf@g))+
%D                    \utf@a*(\numexpr(`#3-\utf@g))+%
%D                           (\numexpr(`#4-\utf@g)))}
%D \stoptyping
%D
%D And:
%D
%D \starttyping
%D \def\utfdiv#1{\number\numexpr((#1-\utf@g)/\utf@h)}
%D \def\utfmod#1{\number\numexpr((#1)-(\utf@h*(\utfdiv{#1})))}
%D \stoptyping
%D
%D Depending on the usage, you can rely on parenthesis only:
%D
%D \starttyping
%D \def\utftwounicode#1#2%
%D   {\numexpr(\utf@a*(#1-\utf@d)+%
%D                    `#2-\utf@g)}
%D
%D \def\utfthreeunicode#1#2#3%
%D   {\numexpr(\utf@b*(#1-\utf@e)+%
%D            \utf@a*(`#2-\utf@g)+%
%D                    `#3-\utf@g)}
%D
%D \def\utffourunicode#1#2#3#4%
%D   {\numexpr(\utf@c*(#1-\utf@f)+%
%D            \utf@b*(`#2-\utf@g)+%
%D            \utf@a*(`#3-\utf@g)+%
%D                    `#4-\utf@g)}
%D \stoptyping

\def\utfdiv#1{\the\numexpr           (#1-\utf@g)/\utf@h \relax}
\def\utfmod#1{\the\numexpr#1-\utf@h*((#1-\utf@g)/\utf@h)\relax}

%D The next one also handles the zero case well:

\def\utfdiv#1{\the\numexpr\ifcase\numexpr#1\relax0\else           (#1-\utf@g)/\utf@h \fi\relax}
\def\utfmod#1{\the\numexpr\ifcase\numexpr#1\relax0\else#1-\utf@h*((#1-\utf@g)/\utf@h)\fi\relax}

% or
%
% \def\utfdiv#1{\ifcase\numexpr#1\relax0\else\the\numexpr(#1-\utf@g)/\utf@h\relax\fi}
% \def\utfmod#1{\ifcase\numexpr#1\relax0\else\the\numexpr#1-\utf@h*((#1-\utf@g)/\utf@h)\relax\fi}

%D When tracing we also need:

\def\utfvid#1{\the\numexpr(#1-\medcard)/\maxcard\relax}

%D Using the three conversion macros, we can now implement
%D a few handlers. They all call the general \type
%D {\unicodechar} conversion macro.
%D
%D \starttyping
%D \def\utftwouniglph#1#2%
%D   {\unicodechar{\utftwounicode  {#1}{#2}}}
%D
%D \def\utfthreeuniglph#1#2#3%
%D   {\unicodechar{\utfthreeunicode{#1}{#2}{#3}}}
%D
%D \def\utffouruniglph#1#2#3#4%
%D   {\unicodechar{\utffourunicode {#1}{#2}{#3}{#4}}}
%D \stoptyping
%D
%D Because the unicode number is used a few times per
%D conversion, we can expand it once (\type {\the} and \type
%D {\number} make sure of this). This saves us another 10\%.
%D
%D \starttyping
%D \def\utftwouniglph#1#2%
%D   {\@EA\unicodechar\@EA{\the\utftwounicode{#1}{#2}}}
%D
%D \def\utfthreeuniglph#1#2#3%
%D   {\@EA\unicodechar\@EA{\the\utfthreeunicode{#1}{#2}{#3}}}
%D
%D \def\utffouruniglph#1#2#3#4%
%D   {\@EA\unicodechar\@EA{\the\utffourunicode{#1}{#2}{#3}{#4}}}
%D \stoptyping
%D
%D We can rewrite these macros to faster alternatives: the
%D less arguments we pass, the faster the conversion will be,
%D but at the price of readability. So we have:
%D
%D \starttyping
%D \def\utftwouniglph#1#2%
%D   {\@EA\unicodechar\@EA{\the\numexpr(\utf@a*(#1-\utf@d)+%
%D      `#2-\utf@g)}}
%D
%D \def\utfthreeuniglph#1#2#3%
%D   {\@EA\unicodechar\@EA{\the\numexpr(\utf@b*(#1-\utf@e)+%
%D      \utf@a*(`#2-\utf@g)+`#3-\utf@g)}}
%D
%D \def\utffouruniglph#1#2#3#4%
%D   {\@EA\unicodechar\@EA{\the\numexpr(\utf@c*(#1-\utf@f)+%
%D       \utf@b*(`#2-\utf@g)+\utf@a*(`#3-\utf@g)+`#4-\utf@g)}}
%D \stoptyping
%D
%D Less parsing, and therefore faster:

\def\utftwouniglph#1#2%
  {\@EA\unicodechar\@EA{\the\numexpr\utf@a*(#1-\utf@d)+%
     `#2-\utf@g\relax}}

\def\utfthreeuniglph#1#2#3%
  {\@EA\unicodechar\@EA{\the\numexpr\utf@b*(#1-\utf@e)+%
     \utf@a*(`#2-\utf@g)+`#3-\utf@g\relax}}

\def\utffouruniglph#1#2#3#4%
  {\@EA\unicodechar\@EA{\the\numexpr\utf@c*(#1-\utf@f)+%
      \utf@b*(`#2-\utf@g)+\utf@a*(`#3-\utf@g)+`#4-\utf@g\relax}}

%D Now we come to the unicode handler itself. We will use a few
%D constants, which saves us (at least at the time of writing
%D and testing these macros) another 10\%.

\def\@@univector  {univ}
\def\@@unknownchar{unknownchar}

%D Now comes the nice part: turning codes into glyphs. The
%D actual conversion does not take place here, but is done by
%D macros in \type{unic-nnn} files. There we map a range onto
%D named glyphs, so that they fit well into the rest of
%D \CONTEXT.

%D \macros
%D   {utfunicodetracer}
%D
%D By default, the converter produces a character representation,
%D but for tracing purposes, you can set a trace option.

\chardef\utfunicodetracer=0

%D \def\TraceUnic#1%
%D   {\chardef\utfunicodetracer#1\relax\enableregime[utf]Ű}
%D
%D \starttabulate[|c|c|c|c|c|c|]
%D \NC option   \NC number\NC mapping\NC glyph\NC string\NC example    \NC \NR
%D \NC 0        \NC       \NC        \NC \star\NC       \NC \TraceUnic0\NC \NR
%D \NC 1        \NC \star \NC        \NC      \NC       \NC \TraceUnic1\NC \NR
%D \NC 2        \NC       \NC \star  \NC      \NC       \NC \TraceUnic2\NC \NR
%D \NC 3        \NC \star \NC \star  \NC      \NC       \NC \TraceUnic3\NC \NR
%D \NC 4        \NC \star \NC        \NC \star\NC       \NC \TraceUnic4\NC \NR
%D \NC 5        \NC       \NC \star  \NC \star\NC       \NC \TraceUnic5\NC \NR
%D \NC 6        \NC \star \NC \star  \NC \star\NC       \NC \TraceUnic6\NC \NR
%D \NC 7        \NC       \NC        \NC      \NC \star \NC \TraceUnic7\NC \NR
%D \NC 8        \NC \star \NC        \NC      \NC       \NC \TraceUnic8\NC \NR
%D \NC otherwise\NC       \NC        \NC \star\NC       \NC \TraceUnic9\NC \NR
%D \stoptabulate

%D \macros
%D   {unicodechar}
%D
%D Next we implement the character handler:

\def\unicodechar
  {\ifcase\utfunicodetracer
     \expandafter\utfunihash      \or
     \expandafter\utfunichar      \or
     \expandafter\utfunisplit     \or
     \expandafter\utfuniboth      \or
     \expandafter\utfunihashchar  \or
     \expandafter\utfunihashsplit \or
     \expandafter\utfunihashboth  \or
     \expandafter\utfuniglyphname \or
     \expandafter\utfunientity    \else
     \expandafter\utfunihash
  \fi}

%D \startbuffer
%D \enableregime[utf] \dostepwiserecurse{0}{8}{1}
%D   {\recurselevel:
%D    \chardef\utfunicodetracer=\recurselevel aap‒noot coördinatie – één
%D    \crlf}
%D \stopbuffer
%D
%D \typebuffer \start \getbuffer \stop

%D \macros
%D   {unicodehexnumber}
%D
%D A few auxiliary macros, producing the range||char pair:

\def\unicodepair#1%
  {\utfdiv{#1}:\utfmod{#1}}

\def\unicodenumber#1{\number#1}

\def\unicodehexnumber#1%
  {\ifnum#1>\maxcard
     \expanded{\uchexnumbers{\utfvid{#1}}}%
     \expanded{\uchexnumbers{\utfdiv{\utfdiv{#1}}}}%
   \else
     00%
     \expanded{\uchexnumbers{\utfdiv{#1}}}%
   \fi
   \expanded{\uchexnumbers{\utfmod{#1}}}}

%D The following macros visualize the unicode character. The
%D \type {\relax} in front of the \type {-} prevents lookahead
%D problems; somehow \type {\numexpr} cannot look beyond this
%D sign, and expects a number.

\def\utfunichar   #1{{\tttf U\low{\tx\unicodenumber{#1}}}}
\def\utfunisplit  #1{{\tttf U\low{\tx\unicodepair{#1}}}}
\def\utfuniboth   #1{{\tttf U\low{\tx\unicodenumber{#1}->\unicodepair{#1}}}}
\def\utfunientity #1{{\tttf\&\#x\unicodehexnumber{#1};}}

%D The character itself is accessed and typeset by:
%D
%D \starttyping
%D \def\utfunihash#1%
%D   {\executeifdefined{\@@univector\utfdiv{#1}}%
%D      \gobbleoneargument{\utfmod{#1}}}
%D \stoptyping
%D
%D Again, we can provide a faster alternative, because inside
%D the conditional executer, the argument is expanded twice,
%D and therefore the calculation done once more than needed.
%D So, we make sure that the argument is expansion on
%D forehand. Just to remind you: \type {#1} is the \UNICODE\
%D number.
%D
%D \starttyping
%D \def\utfunihash#1%
%D   {\@EA\executeifdefined\@EA{\@EA\@@univector\number\utfdiv{#1}}%
%D      {\unknownchar\gobbleoneargument}{\utfmod{#1}}}
%D \stoptyping
%D
%D In order to save calculation time, I decided to change
%D this definition into:

%D \starttyping
%D \def\utfunihash#1%
%D   {\@EA\doutfunihash\@EA{\number\utfdiv{#1}}{#1}}
%D
%D \def\doutfunihash#1#2%
%D   {\ifcsname\@@univector\number#1\endcsname
%D      \csname\csname\@@univector#1\endcsname{\utfmod{#2}}\endcsname
%D    \else
%D      \unknownchar
%D    \fi}
%D \stoptyping
%D
%D Or leaner and meaner:
%D
%D \starttyping
%D \def\doutfunihash#1#2%
%D   {\csname
%D      \ifcsname\@@univector\number#1\endcsname
%D        \csname\@@univector#1\endcsname{\utfmod{#2}}%
%D      \else
%D        \@@unknownchar
%D      \fi
%D    \endcsname}
%D \stoptyping
%D
%D And finaly it became:

\def\doutfunihash#1#2%
  {\ifcsname\@@univector\number#1\endcsname
     \csname\@@univector#1\endcsname{\utfmod{#2}}%
   \else
     \@@unknownchar
   \fi}

\def\utfunihashglyph#1%
  {\csname\@EA\doutfunihash\@EA{\number\utfdiv{#1}}{#1}\endcsname}

\def\utfunihashcommand
  {\@EAEAEA\string\utfunihashglyph}

%D For practical purposes, we handle the normal \ASCII\
%D characters here:

\def\utfunihashglyph#1%
  {\csname
     \ifnum#1<\utf@i
       \strippedcsname\unicodeasciicharacter\endcsname{#1}%
     \else
       \@EA\doutfunihash\@EA{\number\utfdiv{#1}}{#1}\endcsname
     \fi}

%D With:

\let\unicodeasciicharacter\rawcharacter

%D Now we can say:

\let\utfunihash\utfunihashglyph

%D We also need:

\def\utfuniglyphname#1%
  {{\tttf
    \ifnum#1<\utf@i
      \unicodeasciicharacter{#1}%
    \else
      \expandafter\string\csname\doutfunihash{\number\utfdiv{#1}}{#1}\endcsname
    \fi}}

%D The combined presentation is implemented by:

\def\utfunihashchar #1%
  {\utfunihash{#1}\low{\infofont\unicodenumber{#1}}}

\def\utfunihashsplit#1%
  {\utfunihash{#1}\low{\infofont\unicodepair{#1}}}

\def\utfunihashboth #1%
  {\utfunihash{#1}\low{\infofont\unicodenumber{#1}->\unicodepair{#1}}}

%D Unknown characters get a placeholder.

\unexpanded\def\unknownchar % {} prevents problems with arguments
  {{\hbox{\vrule\!!width.5em\!!height1ex\!!depth\zeropoint}}}

%D So far for the conversion macros. The optimizations we
%D did, brought down the runtime some 50\%, which, given that
%D the majority of characters will be normal \ASCII\
%D characters, the penalty of conversion is not that large.

%D \macros
%D   {useunicodevector}
%D
%D Since we end up with many encodings, it starts making
%D sense to postpone loading, so let's start doing this
%D with \UNICODE.

\def\doifunicodevector#1%
  {\doifdefined{\@@univector#1}}

\def\useunicodevector[#1]%
  {\processcommalist[#1]\douseunicodevector}

\def\douseunicodevector#1%
  {\ifundefined{\@@univector#1}%
     \readsysfile{\f!unicprefix\threedigits{#1}}
       {\writestatus{unicode}{loading vector #1}}
       {\writestatus{unicode}{unknown vector #1}}%
   \fi}

%D \macros
%D   {startunicodevector}
%D
%D A vector roughly looks as follows. By putting the text
%D inside the name constructor, we prevent problems with
%D partial expansion in macros and special cases.
%D
%D \starttyping
%D \startunicodevector 0
%D   \ifcase\numexpr(#1-159)\or
%D     \@@unknownchar\or % NO-BREAK SPACE
%D     exclamdown\or
%D     textcent\or
%D     ....\else
%D     \@@unknowncharacter
%D   \fi
%D \stopunicodevector
%D \stoptyping
%D
%D In vector \type {unix-000} you will find another
%D optimizations. By using as less tokens as possible, we limit
%D the time skipping branches in the test, and save upto 20\%
%D runtime.

\def\startunicodevector #1 #2\stopunicodevector
  {\setgvalue{\@@univector#1}##1{#2}}

%D We define (as a practical example) the utf signal FEFF:

\ifx\zwnbsp\undefined
  \let\zwnbsp\relax % zerowidthnonbreakablespace
\fi

\startunicodevector 254
  \expandafter\strippedcsname\ifnum#1<255 \unknownchar\else\zwnbsp\fi
\stopunicodevector

%D Here we provide another auxiliary macro:
%D
%D \startbuffer
%D \unicodeinfoline{196}{Ä}{LATIN CAPITAL LETTER A WITH DIAERESIS}
%D \unicodeinfoline{197}{Å}{LATIN CAPITAL LETTER A WITH RING ABOVE}
%D \unicodeinfoline{198}{Æ}{LATIN CAPITAL LETTER AE}
%D \unicodeinfoline{199}{Ç}{LATIN CAPITAL LETTER C WITH CEDILLA}
%D \unicodeinfoline{200}{È}{LATIN CAPITAL LETTER E WITH GRAVE}
%D \unicodeinfoline{201}{É}{LATIN CAPITAL LETTER E WITH ACUTE}
%D \stopbuffer
%D
%D \typebuffer
%D
%D \start \enableregime[utf]\getbuffer \stop

\def\unicodeinfoline#1#2#3%
  {\ifnum#1>\utf@g % 128
     \noindent \hbox
       {\hbox to 4em{\tttf\unicodehexnumber{#1}\hss}\quad
        \hbox to 1em{#2\hss}\quad
        \hbox to 9em{\tttf\unicodenumber{#1}->\unicodepair{#1}\hss}\quad
        \hbox to 9em{\tttf\let\utfunihash\utfunihashcommand#2\hss}\quad % tricky
        \lowercase  {\tttf#3}}\par
   \fi}

%D The next code permits utf code in hyperlinks:

\def\cleanunicodechar#1{.#1.}

\appendtoks \let\unicodechar\cleanunicodechar \to \everycleanupfeatures

%D We will now hook this mechanism in the existing font
%D handler. More documentation will follow. Probably, some
%D features in \type {font-uni.tex} will be generalized
%D and moved here.

\def\unidiv{0} \def\unimod{0}

\chardef\utfunihashmode=0 % 0=hash glyph 1=font glyph

\def\utfunifontglyph#1%
  {\xdef\unidiv{\number\utfdiv{#1}}%
   \xdef\unimod{\number\utfmod{#1}}%
   \ifnum#1<\utf@i
     \char\unimod % \unicodeascii\unimod
   \else\ifcsname\@@univector\unidiv\endcsname
     \csname\doutfunihash{\unidiv}{#1}\endcsname
   \else % so, these can be different fonts !
     \unicodeglyph\unidiv\unimod % no \uchar (yet)
   \fi\fi}

\chardef\utfunicommandmode=0 % 1 = hex

\def\unicodecommandchar#1#2%
  {\string\char
   \ifcase\utfunicommandmode
     #1:#2\else\lchexnumbers#1:\lchexnumbers#2%
   \fi}

\def\utfunifontcommand#1%
  {\xdef\unidiv{\number\utfdiv{#1}}%
   \xdef\unimod{\number\utfmod{#1}}%
   \ifnum#1<\utf@i
     \unicodecommandchar\unidiv\unimod
   \else\ifcsname\@@univector\unidiv\endcsname
     \@EA\string\csname\doutfunihash{\unidiv}{#1}\endcsname
   \else
     \unicodecommandchar\unidiv\unimod
   \fi\fi}

\def\utfunihash
  {\ifcase\utfunihashmode
     \@EA\utfunihashglyph
   \else
     \@EA\utfunifontglyph
   \fi}

\def\utfunihushcommand
  {\@EAEAEA\string\utfunihashglyph}

\def\utfunihashcommand
  {\ifcase\utfunihashmode
   % \@EA\@EAEAEA\@EA\string\@EA\utfunihashglyph
     \@EA\utfunihushcommand
   \else
     \@EA\utfunifontcommand
   \fi}

%D Goodies:

\fetchruntimecommand \showunicodevector {\f!unicprefix\s!run}
\fetchruntimecommand \showunicodetable  {\f!unicprefix\s!run}

%D Well, let's at least preload a few familiar ones. Here we
%D also load the \UTF\ regime.

\useunicodevector[0,1,2,3,4,5,30,32,33,34,39] \useregime[utf]

% 31, text mem usage first

\protect  \endinput
