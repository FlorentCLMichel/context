%D \module
%D   [       file=spec-def,
%D        version=1996.01.25,
%D          title=\CONTEXT\ Special Macros,
%D       subtitle=Definitions,
%D         author=Hans Hagen,
%D           date=\currentdate,
%D      copyright={PRAGMA / Hans Hagen \& Ton Otten}]
%C
%C This module is part of the \CONTEXT\ macro||package and is
%C therefore copyrighted by \PRAGMA. See mreadme.pdf for
%C details.

%D This module used to be integrated in \type {spec-ini},
%D but testing optimization is more convenient this way.

\writestatus{loading}{Context Special Macros / Definitions}

%D The following libraries are defined. Two postscript
%D drivers are supported, as well as two mechanisms for
%D interactive texts.
%D
%D \startlinecorrection
%D \starttable[|l|l|l|l|l|]
%D \HL
%D \NC \bf file             \NC
%D     \bf name             \NC
%D     \bf calls            \NC
%D     \bf support          \NC
%D     \bf program / driver \NC\SR
%D \HL
%D \NC \tttf spec-tex       \NC
%D     \tttf tex            \NC
%D                          \NC
%D     Generic \TEX\ (\DVI) \NC
%D     (default)            \NC\FR
%D \NC \tttf spec-tpd       \NC
%D     \tttf \PDF           \NC
%D                          \NC
%D     Han The Thanh        \NC
%D     (pdftex)             \NC\MR
%D \NC \tttf spec-ps        \NC
%D     \tttf postscript     \NC
%D                          \NC
%D     Adobe PostScript     \NC
%D     (default)            \NC\MR
%D \NC \tttf spec-tr        \NC
%D     \tttf rokicki        \NC
%D     \tttf postscript     \NC
%D     Thomas Rokicki       \NC
%D     (dvips)              \NC\MR
%D \NC \tttf spec-yy        \NC
%D     \tttf yandy          \NC
%D     \tttf postscript     \NC
%D     YandY                \NC
%D     (dvipsone, dviwindo) \NC\MR
%D \NC \tttf spec-pdf       \NC
%D     \tttf pdf            \NC
%D                          \NC
%D     Adobe PDF            \NC
%D     (Acrobat)            \NC\MR
%D \NC \tttf spec-win       \NC
%D     \tttf dviwindo       \NC
%D     YandY                \NC
%D     (dviwindo)           \NC\MR
%D \NC \tttf spec-htm       \NC
%D     \tttf html           \NC
%D                          \NC
%D     HTML V 2.0           \NC
%D     (dvips)              \NC\LR
%D \HL
%D \stoptable
%D \stoplinecorrection

%D \macros
%D   {dostartgraymode,dostopgraymode,
%D    dostartrgbcolormode,dostartcmykcolormode,dostartgraycolormode,dostopcolormode}
%D
%D Switching to and from color can be done in two ways:
%D
%D \startitemize[packed,n]
%D \item  insert driver specific commands
%D \item  pass instructions to the output device
%D \stopitemize
%D
%D The first approach is more general and lays the
%D responsibility at the driver side. Probably due to the fact
%D that \TEX\ does not directly support color, we have been
%D confronted for the last few years with changing special
%D definitions. The need for support depends on how a macro
%D package handles colored text that crosses the page boundary.
%D Again, there are two approaches.
%D
%D \startitemize[packed,n]
%D \item  let \TEX\ do the job
%D \item  let the driver handle things
%D \stopitemize
%D
%D The first approach is as driver independant as possible and
%D can easily be accomplished by using \TEX's mark mechanism.
%D In \CONTEXT\ we follow this approach. More and more, drivers
%D are starting to support color, including stacking them.
%D
%D Colors as well as grayscales can be represented in scales
%D from~0 to~1. When drivers use values in the range 0..255,
%D this value has to be adapted in the translation process.
%D Technically it's possible to get a grayscale from combining
%D colors. In the \cap{RGB} color system, a color with Red,
%D Green and Blue components of 0.80 show the same gray as a
%D Gray Scale specified 0.80. The \cap{CMYK} color system
%D supports a Black component apart from Cyan, Magenta and
%D Yellow.
%D
%D Depending on the target format, color support differs from
%D gray support. PostScript for example offers different
%D operators for setting gray and color. This is because
%D printing something using three colors is someting else than
%D printing with just black.
%D
%D In \CONTEXT\ we have implemented a color subsystem that
%D supports the use of well defined colors that, when printed
%D in black and white, still can be distinguished. This
%D approach enables us to serve both printed and electronic
%D versions, using colored text and illustrations. More on the
%D fundamentals of this topic can be found in the \cap{MAPS} of
%D the Dutch User Group, 14 (95.1).
%D
%D To satisfy all those needs, we define four specials which
%D supply enough information for drivers to act upon. We
%D could have used more general commands with the keywords
%D 'rgb' and 'gray', but because these specials are used often,
%D we prefer the more direct and shorter alternative.
%D
%D We start with the installation of color and grayscale
%D specials. The values are in the range 0..1 (e.g. 0.25).
%D
%D \starttyping
%D \dostartgraymode      {gray} ... \dostopgraymode
%D \dostartrgbcolormode  {red} {green} {blue} ... \dostopcolormode
%D \dostartcmykcolormode {cyan} {magenta} {yellow} {black} ... \dostopcolormode
%D \dostartgraycolormode {gray} ... \dostopcolormode
%D \stoptyping
%D
%D Because we can expect conflicts between drivers, we
%D implement them as category \type{or}. In previous versions
%D of \DVIPSONE\ the use of their color||specials did not
%D interfere with the PostScript ones, but recent versions do.

\installspecial [\dostartgraymode]      [or] [1]
\installspecial [\dostopgraymode]       [or] [0]

\installspecial [\dostartrgbcolormode]  [or] [3]
\installspecial [\dostartcmykcolormode] [or] [4]
\installspecial [\dostartgraycolormode] [or] [1]
\installspecial [\dostopcolormode]      [or] [0]

%D New and experimental:

\installspecial [\dostartspotcolormode]    [or] [2]

\installspecial [\doregisterrgbspotcolor]  [or] [7]
\installspecial [\doregistercmykspotcolor] [or] [8]
\installspecial [\doregistergrayspotcolor] [or] [5]

\installspecial [\doregisterrgbindexcolor]  [or] [7]
\installspecial [\doregistercmykindexcolor] [or] [8]
\installspecial [\doregistergrayindexcolor] [or] [5]

\installspecial [\doregisterspotcolorname] [or] [2]

\installspecial [\dostartnonecolormode]    [or] [0]
\installspecial [\doregisternonecolor]     [or] [0]

%D For some drivers, the stop special is of no use and can
%D simply call the start one with zero arguments.

%D \macros
%D   {dogetfiguresize,doinsertfile,doinsertMPfile}
%D
%D Probably the most problematic special is the following
%D one. Because we want to be able to support different
%D schemes, we pass a lot of data to it.
%D
%D The support of inserting files (like illustrations) comes in
%D many flavors. Some drivers use scales, some take dimensions.
%D Some need offsets and others act on stored characteristics.
%D They need one thing in common: a filename. Although separate
%D specials for different formats sometimes are more clear, we
%D decided to combine them all in one:
%D
%D \starttyping
%D \doinsertfile {type,method} {file,label}
%D               {xscale} {yscale} {x} {y} {w} {h}
%D               {options}
%D \stoptyping
%D
%D The scale is given percents, the other values are base
%D points.
%D
%D Per 2006 we have simplified this model. The amount of drivers
%D is not growing and all support scaling. We pass variables via
%D Driver specific macros.
%D
%D \starttyping
%D \doinsertfile
%D \stoptyping
%D
%D The special is implemented as \type{or}. Because \DVIPSONE\
%D understands them all, a chain of alternatives would generate
%D multiple occurrences of the same illustration.
%D
%D When option 1 is passed, the viewers is asked to present a
%D preview, like the first frame of a movie.

\installspecial [\dogetfiguresize][or] [0] % now a special
\installspecial [\doinsertfile]   [or] [0] % [9]
\installspecial [\doinsertMPfile] [or] [1] % speed up

%D No start||stop construction is needed here, because there in
%D no further interference of \TEX. All dimensions are output
%D in points and scales as a number, where 100 equal 100\%. (We
%D used to pass scaled points).

%D \macros
%D   {doinsertsoundtrack}
%D
%D Sounds are (for the moment) just files with
%D associated options.
%D
%D \starttyping
%D \doinsertsoundtrack {file} {label} {options}
%D \stoptyping

\installspecial [\doinsertsoundtrack] [or] [3]

%D \macros
%D   {dogetnofinsertpages}
%D
%D Some file formats support more than one page, like \PDF,
%D and for special applications, one may want to have access
%D to the total number of pages.
%D
%D \starttyping
%D \dogetnofinsertpages{filename}
%D \stoptyping
%D
%D The number is also available after the insert is placed,
%D since inclusion may take place immediate when an insert is
%D called upon.

\def\nofinsertpages{1} % one of the few 'talk backs'

\installspecial [\dogetnofinsertpages] [or] [1]

%D \macros
%D   {dostartrotation,
%D    dostoprotation}
%D
%D We support rotation with the special:
%D
%D \starttyping
%D \dostartrotation {angle} ... \dostoprotation
%D \stoptyping
%D
%D For the moment these specials are installed as
%D category \type{or}.

\installspecial [\dostartrotation] [or] [1]
\installspecial [\dostoprotation]  [or] [0]

%D \macros
%D   {dostartscaling,
%D    dostopscaling}
%D
%D Scaling is also supported:
%D
%D \starttyping
%D \dostartscaling {x} {y} ... \dostopscaling
%D \stoptyping
%D
%D Like the previous one, these specials are of category
%D \type{or}.

\installspecial [\dostartscaling] [or] [2]
\installspecial [\dostopscaling]  [or] [0]

%D \macros
%D   {dostartmirroring,
%D    dostopmirroring}
%D
%D And indeed, mirroring is there too:
%D
%D \starttyping
%D \dostartmirroring {x} {y} ... \dostopmirroring
%D \stoptyping
%D
%D Again these specials are installed as category \type{or}.

\installspecial [\dostartmirroring] [or] [0]
\installspecial [\dostopmirroring]  [or] [0]

%D \macros
%D   {dostartnegative,
%D    dostopnegative}
%D
%D When producing output for an image setter, negating the
%D page comes into view. Here are the tools:

\installspecial [\dostartnegative] [or] [0]
\installspecial [\dostopnegative]  [or] [0]

%D \macros
%D   {dostartoverprint,
%D    dostopoverprint}
%D
%D Rudimentary overprint (knockout) control is implemented
%D using:

\installspecial [\dostartoverprint] [or] [0]
\installspecial [\dostopoverprint]  [or] [0]

%D The following two specials are used in for instance \type
%D {\vadjust}'d margin material inside colored paragraphs.

\installspecial [\dostartgraphicgroup] [or] [0]
\installspecial [\dostopgraphicgroup ] [or] [0]

%D \macros
%D   {doselectfirstpaperbin,
%D    doselectsecondpaperbin}
%D
%D Here are some very printer||specific ones. No further
%D comment.

\installspecial [\doselectfirstpaperbin]  [or]  [0]
\installspecial [\doselectsecondpaperbin] [or]  [0]

%D \macros
%D   {doovalbox}
%D
%D When we look at the implementation, this is a complicated
%D one. There are seven arguments.
%D
%D \starttyping
%D \doovalbox {w} {h} {d} {linewidth} {radius} {stroke} {fill} {variant}
%D \stoptyping
%D
%D This command has to return a \type{\vbox} which can be used
%D to lay over another one (with text). The radius is in
%D degrees, the stroke and fill are~\type{1} (true) of~\type{0}
%D (false).

\installspecial [\doovalbox] [or] [8]

%D \macros
%D   {dostartclipping,dostopclipping}
%D
%D Clipping is implemented in such a way that an arbitrary
%D can be fed.
%D
%D \starttyping
%D \dostartclipping {pathname}{width} {height}
%D \dostopclipping
%D \stoptyping
%D
%D

\installspecial [\dostartclipping] [or] [3]
\installspecial [\dostopclipping]  [or] [0]

%D \macros
%D   {dosetupidentity}
%D
%D We can declare some characteristics of the document with
%D
%D \starttyping
%D \dosetupidentity {title} {subject} {author} {creator} {date} {keys}
%D \stoptyping
%D
%D All data is in string format.

\installspecial [\dosetupidentity] [and] [6]

%D \macros
%D   {dosetuppaper}
%D
%D This special can be used to tell the driver what page size
%D to use. The special takes three arguments.
%D
%D \starttyping
%D \dosetuppaper {type} {width} {height}
%D \stoptyping
%D
%D The type is one of the common identifiers, like A4, A5 or
%D B2.

\installspecial [\dosetuppaper] [and] [3]

%D \macros
%D   {dosetupprinter}
%D
%D Some drivers enable the user to specify the paper type
%D used and/or page dimensions to be taken into account.
%D
%D \starttyping
%D \dosetupprinter {type} {hoffset} {voffset} {width} {height}
%D \stoptyping
%D
%D The first argument is one of \type{letter}, \type{legal},
%D \type{A4}, \type{A5} etc. The dimensions are in
%D basepoints.

\installspecial [\dosetupprinter] [and] [5]

%D \macros
%D   {dosetupopenaction, dosetupclosaction,
%D    dosetupopenpageaction, dosetupclospageaction,
%D    dosetupinteraction,
%D    dosetupscreen,
%D    dosetupviewmode}
%D
%D Here come some obscure interactive commands. Probably the
%D specs will change with the development of the macros that
%D use them.
%D
%D The first ones can be used to set up the interaction.
%D
%D \starttyping
%D \dosetupinteraction
%D \stoptyping
%D
%D Normally this command does nothing but giving a message
%D that some scheme is supported.
%D
%D \starttyping
%D \dosetupstartaction
%D \dosetupstopaction
%D \stoptyping
%D
%D These two setup the actions to be executed when the document
%D is opened and closed.
%D
%D The next commands sets up the page and screen. They are
%D kind of related.
%D
%D \starttyping
%D \dosetuppage   {hoffset} {voffset} {width} {height} {options}
%D \dosetupscreen {hoffset} {voffset} {width} {height} {options}
%D \stoptyping
%D
%D The first four arguments are in points. Option~1 results in a
%D full screen launch.
%D
%D \starttyping
%D \dosetuppageview {keyword}
%D \stoptyping
%D
%D For the moment we only support \type{fit}.

\installspecial [\dosetupinteraction]     [and] [0]

\installspecial [\dosetupopenaction]      [and] [0]
\installspecial [\dosetupscreen]          [and] [5]
\installspecial [\dosetuppageview]        [and] [1]

\installspecial [\dosetupcloseaction]              [and] [0]
\installspecial [\dosetupopenpageaction]           [and] [0]
\installspecial [\dosetupclosepageaction]          [and] [0]
\installspecial [\dosetuprenderingopenpageaction]  [and] [0]
\installspecial [\dosetuprenderingclosepageaction] [and] [0]

% new, yet undocumented, replaces \dosetupscreen

\installspecial [\dosetupcropbox]  [and] [4]
\installspecial [\dosetuptrimbox]  [and] [4]
\installspecial [\dosetupartbox]   [and] [4]
\installspecial [\dosetupbleedbox] [and] [4]

%D \macros
%D   {dostarthide,
%D    dostophide}
%D
%D Not every part of the screen is suitable for paper. Menus
%D for instance have no meaning on an non||interactive medium.
%D These elements are hidden by means of:
%D
%D \starttyping
%D \dostarthide
%D \dostophide
%D \stoptyping

\installspecial [\dostarthide] [or] [0]
\installspecial [\dostophide]  [or] [0]

%D \macros
%D   {dostartgotolocation, dostopgotolocation,
%D    dostartgotorealpage, dostopgotorealpage}
%D
%D When we want to support hypertext buttons, again we have
%D to deal with two concepts.
%D
%D \startitemize[packed,n]
%D \item let \TEX\ highlight the text
%D \item let the driver show us where to click
%D \stopitemize
%D
%D The first approach is the most secure one. It gives us
%D complete control over the visual appearance of hyper
%D buttons. The second alternative lets the driver guess what
%D part of the text needs highlighting. As long as we deal with
%D not too complicated textual buttons, this is no problem.
%D It's even a bit more efficient when we take long mid
%D paragraph active regions into account. When we let \TEX\
%D handle active sentences {\em for instance marked like this
%D one}, we have to take care of line- and pagebreaks ourselve.
%D However, it's no trivial matter to let a driver find out
%D where things begin and end. Because most hyperlinks can be
%D found in tables of contents and registers, the saving in
%D terms of bytes can be neglected and the first approach is a
%D clear winner.
%D
%D The most convenient way of cross||referencing is using named
%D destinations. A more simple scheme is using page numbers as
%D destinations. Because the latter alternative can often be
%D implemented more efficient, and because we cannot be sure
%D what scheme a driver supports, we always have to supply a
%D pagenumber, even when we use named destinations.
%D
%D To enable a driver to find out what to make active, we have
%D to provide begin and endpoints, so like with color, we use
%D pairs of specials. The first scheme can be satisfied with
%D proper dimensions of the areas to be made active.
%D
%D The interactive real work is done by the following four
%D specials. The reason for providing the first one with both
%D a label and a number, is a result of the quite poor
%D implementation of \type{pdfmarks} in version 1.0 of
%D Acrobat. Because only pagenumbers were supported as
%D destination, we had to provide both labels (\DVIWINDO) and
%D pagenumbers (\PDF). Some drivers use start stop pairs.
%D
%D \starttyping
%D \dostartgotolocation {w} {h} {url} {file} {label} {page}
%D \dostartgotorealpage {w} {h} {url} {file}         {page}
%D \stoptyping
%D
%D Their counterparts are:
%D
%D \starttyping
%D \dostopgotolocation
%D \dostopgotorealpage
%D \stoptyping
%D
%D The internal alternative is used for system||generated
%D links, the external one for user||generated links. The
%D Uniform Resource Locator can be used to let the reader
%D surf the net.

\installspecial [\dostartgotolocation] [and] [6]
\installspecial [\dostopgotolocation]  [and] [0]
\installspecial [\dostartgotorealpage] [and] [5]
\installspecial [\dostopgotorealpage]  [and] [0]

%D One may wonder why jumps to page and location are not
%D combined. By splitting them, we enable macro||packages to
%D force the prefered alternative, while on the other hand
%D drivers can pick up the alternative desired most.

%D \macros
%D  {dostartgotoJS, doflushJSpreamble}
%D
%D Rather special is the option to include and execute
%D JavaScript code. This is a typical \PDF\ option.
%D
%D \starttyping
%D \dostartgotoJS {w} {h} {script}
%D \stoptyping
%D
%D This not so standard \TEX\ feature should be used with
%D care. Preamble scripts are flushed by
%D
%D \doflushJSpreamble {script}

\installspecial [\dostartgotoJS]    [and] [3]
\installspecial [\dostopgotoJS]     [and] [0]
\installspecial [\doflushJSpreamble][and] [1]

%D \macros
%D   {dostartthisislocation, dostopthisislocation,
%D    dostartthisisrealpage, dostopthisisrealpage}
%D
%D Before we can goto some location or page, we have to tell
%D the system where it can be found. Because some drivers
%D follow the \SGML\ approach of begin||end tags, we have to
%D support pairs. A possible extension to this scheme is
%D supplying coordinates for viewing the text.
%D
%D The opposite commands of \type{\dogotosomething} have only
%D one argument:
%D
%D \starttyping
%D \dostartthisislocation {label}
%D \dostartthisisrealpage {page}
%D \stoptyping
%D
%D These commands are accompanied by:
%D
%D \starttyping
%D \dostopthisislocation
%D \dostopthisisrealpage
%D \stoptyping
%D
%D As with all interactive commands's they are installed as
%D \type{and} category specials.

\installspecial [\dostartthisislocation] [and] [1]
\installspecial [\dostopthisislocation]  [and] [0]
\installspecial [\dostartthisisrealpage] [and] [1]
\installspecial [\dostopthisisrealpage]  [and] [0]

%D In \CONTEXT\ we don't use the \type{\stopsomething}
%D macros because we let \TEX\ take care of typographic
%D issues.

%D \macros
%D   {doresetgotowhereever}
%D
%D These and others need:

\installspecial [\doresetgotowhereever] [and] [0]

%D \macros
%D   {dostartexecutecommand, dostopexecutecommand}
%D
%D The actual behavior of the next pair of commands depends
%D much on the viewing engine. Therefore one cannot depend
%D too much on their support.
%D
%D \starttyping
%D \dostartexecutecommand {w} {h} {command} {options}
%D \stoptyping
%D
%D At least the next commands are supported (more examples
%D can be found in \type {spec-fdf.tex}:
%D
%D \startlinecorrection\setupalign[middle]\leavevmode
%D \starttable[|l|l|]
%D \HL
%D \NC \bf command  \NC \bf action                  \NC\SR
%D \HL
%D \NC first        \NC go to the first page        \NC\FR
%D \NC previous     \NC go to the previous page     \NC\MR
%D \NC next         \NC go to the next page         \NC\MR
%D \NC last         \NC go to the last page         \NC\MR
%D \NC backward     \NC go back to the  link list   \NC\MR
%D \NC forward      \NC go forward in the link list \NC\MR
%D \NC print        \NC enter print mode            \NC\MR
%D \NC exit         \NC exit viewer                 \NC\MR
%D \NC close        \NC close document              \NC\MR
%D \NC enter        \NC enter viewer                \NC\MR
%D \NC help         \NC show help on the viewer     \NC\LR
%D \HL
%D \stoptable
%D \stoplinecorrection
%D
%D Options are to be passed as a comma separated list of
%D assignments.

\installspecial [\dostartexecutecommand] [and] [4]
\installspecial [\dostopexecutecommand]  [and] [0]

%D \macros
%D   {dostartobject,
%D    dostopobject,
%D    doresetobjects,
%D    doinsertobject}
%D
%D Reuse of object can reduce the output filesize
%D considerably. Reusable objects are implemented with:
%D
%D \starttyping
%D \dostartobject{class}{name}{width}{height}{depth}
%D some typeset material
%D \dostopobject
%D \stoptyping
%D
%D \starttyping
%D \doinsertobject{class}{name}
%D \stoptyping
%D
%D The savings can be huge in interactive texts. The next macro needs
%D to be called after a graphic is inserted (in order to clean up
%D global references).
%D
%D \starttyping
%D \doresetobjects
%D \stoptyping

\installspecial [\dostartobject]  [or] [5]
\installspecial [\dostopobject]   [or] [0]
\installspecial [\doinsertobject] [or] [2]
\installspecial [\doresetobjects] [or] [0]

%D \macros
%D   {doregisterfigure, doregisterfigurecolor}
%D
%D Images can be objects as well and it's up to the driver to
%D handle this. Alternative images are also up to the driver,
%D and the next macro tells the driver that the previous image
%D is somehow followed by another and that both have to be
%D handled together. This is a rather fuzzy model, but for the
%D moment it suits its purpose: low res screen versions combined
%D with high res printable ones.

\installspecial [\doregisterfigure]     [or] [2]
\installspecial [\doregisterfigurecolor][or] [1]

% %D \macros
% %D   {dogetobjectreference}
% %D
% %D For very special purposes, one can ask for the internal
% %D reference to the object. Beware!
%
% \installspecial [\dogetobjectreference] [or] [3]
%
% %D The first argument is the name, the second a macro that
% %D gets the associated value.

%D \macros
%D   {dostartrunprogram, dostoprunprogram,
%D    dostartgotoprofile, dostopgotoprofile,
%D    dobeginofprofile,
%D    doendofprofile}
%D
%D These specials are still experimental. They are not yet
%D supported by the programs the way they should be.
%D
%D {\em --- still undocumented ---}

\installspecial [\dostartrunprogram]  [and] [4]
\installspecial [\dostoprunprogram]   [and] [0]
\installspecial [\dostartgotoprofile] [and] [3]
\installspecial [\dostopgotoprofile]  [and] [0]
\installspecial [\dobeginofprofile]   [and] [4]
\installspecial [\doendofprofile]     [and] [0]

%D \macros
%D   {doinsertbookmark}
%D
%D Bookmarks, that is viewer generated tables of contents, are
%D a strange phenomena, mainly because \TEX\ can provide
%D whatever kind of table in much better quality.

\installspecial [\doinsertbookmark] [and] [5]

%D This special is called as:
%D
%D \starttyping
%D \doinstallbookmark {level} {nofsubentries} {text} {page} {open}
%D \stoptyping
%D
%D This definition is very \PDF\ oriented, so for more
%D information we kindly refer to the \PDF\ manuals.

%D \macros
%D   {dosetpagetransition}
%D
%D In presentations, fancy page transitions can, at least for a
%D short moment, let the audience focus at the screen. Like the
%D previous one, this special is very \PDF.
%D
%D \starttyping
%D \dosetpagetransition{dissolve}{0}
%D \stoptyping
%D
%D Transitions have symbolic names, like dissolve, box, split,
%D blinds, wipe and glitter. The second argument determines
%D the wait time (unless zero).

\installspecial [\dosetpagetransition] [or] [2]

%D \macros
%D   {dopresettextfield,dopresetlinefield,
%D    dopresetchoicefield,dopresetpopupfield,dopresetcombofield,
%D    dopresetbuttonfield,dopresetcheckfield,
%D    dopresetradiofield,dopresetradiorecord}
%D
%D The special drivers are programmed independant from their
%D calling macros are thereby use the standard \TEX\ way of
%D passing parameters. Unfortunately fields often have more
%D than nine characteristics, so we pack some arguments in one.
%D
%D \starttyping
%D \dopresettextfield / \dopresetlinefield
%D   {name} {width} {height} {default} {length}
%D   {style,color} {options} {alignment} {actions}
%D
%D \dopresetchoicefield / \dopresetpopupfield / \dopresetcombofield
%D   {name} {width} {height} {default}
%D   {style,color} {options} {values} {actions}
%D
%D \dopresetpushfield
%D   {name} {width} {height} {default}
%D   {options} {values} {actions}
%D
%D \dopresetcheckfield
%D   {name} {width} {height} {default}
%D   {options} {values} {actions}
%D
%D \dopresetradiofield
%D   {name} {width} {height} {default}
%D   {options} {parent} {values} {actions}
%D
%D \dopresetradiorecord
%D   {name} {top} {options} {kids} {actions}
%D \stoptyping

\installspecial [\dopresetlinefield]   [or] [9]
\installspecial [\dopresettextfield]   [or] [9]
\installspecial [\dopresetchoicefield] [or] [8]
\installspecial [\dopresetpopupfield]  [or] [8]
\installspecial [\dopresetcombofield]  [or] [8]
\installspecial [\dopresetpushfield]   [or] [7]
\installspecial [\dopresetcheckfield]  [or] [7]
\installspecial [\dopresetradiofield]  [or] [8]
\installspecial [\dopresetradiorecord] [or] [5]

%D \macros
%D   {dodefinefieldset,dogetfieldset,doiffieldset}
%D
%D Field sets, used in resetting and submitting, are handled
%D by:

\installspecial [\dodefinefieldset] [or] [2]
\installspecial [\dogetfieldset]    [or] [1]
\installspecial [\doiffieldset]     [or] [2]

%D \macros
%D   {dosetfieldstatus}
%D
%D For practical reasons we set some field characteristics
%D using:
%D
%D \starttyping
%D \dosetfieldstatus {mode} {parent} {kids} {root}
%D \stoptyping

\installspecial [\dosetfieldstatus] [or] [4]

%D with:

\def\fieldlonermode {0} % no \chardef here
\def\fieldparentmode{1} % no \chardef here
\def\fieldchildmode {2} % no \chardef here
\def\fieldcopymode  {3} % no \chardef here

%D \macros
%D   {doregistercalculationset}
%D
%D We can define a calculation order list with:
%D
%D \starttyping
%D \doregistercalculationset {set identifier}
%D \stoptyping

\installspecial [\doregistercalculationset] [or] [1]

%D \macros
%D   {doinsertcomment, doflushcomments}
%D
%D Not so much out of need, but to be complete, we also
%D implement text annotations, so called  comment:
%D
%D \starttyping
%D \doinsertcomment
%D   {title} {width} {height} {color} {open} {symbol} {collect} {data}
%D \stoptyping
%D
%D When enables, comments can be collected and flushed:
%D
%D \starttyping
%D \doflushcomments
%D \stoptyping

\installspecial[\doinsertcomment] [and] [8]
\installspecial[\doflushcomments] [and] [0]

%D \macros
%D   {dosetposition, dosetpositionwhd, dosetpositionplus,
%D    dosetpositionpapersize}
%D
%D Not natural to \TEX, but available in \PDFTEX, and by
%D means of postprocessed \DVI, we can save and call upon
%D positions.
%D
%D \starttyping
%D \dosetposition          {identifier}
%D \dosetpositionwhd       {identifier} {width} {height} {depth}
%D \dosetpositionplus      {identifier} {width} {height} {depth} {list}
%D \dosetpositionpapersize {width} {height}
%D \stoptyping
%D
%D This is one of the few specials where when using \PDFTEX\
%D the driver directly deals with the utility file.

%D Support is moved to spec-mis.tex. The whole idea originates
%D and was implemented first in \PDFTEX. Later \DVIPDFMX\
%D supported the \CONTEXT\ specific specials and nowadays
%D \PDFTEX\ supports this feature in \DVI\ mode and \XETEX\
%D implements it as well. This means that only \ALEPH\ is
%D still dependent on specials, but that variant of \TEX\ has
%D become obsolete anyway.

% \installspecial [\dosetposition]          [or] [1]
% \installspecial [\dosetpositionwhd]       [or] [4]
% \installspecial [\dosetpositionplus]      [or] [5]

% \installspecial [\dosetpositionpapersize] [or] [2]

%D \macros
%D   {dostarttransparency,dostoptransparency}
%D
%D \starttyping
%D \dostarttransparency{fraction}{type}
%D \dostoptransparency
%D \stoptyping
%D
%D Although in \CONTEXT\ transparency is closely integrated
%D in the color drivers, in the end it is an independent
%D feature.

\installspecial [\dostarttransparency] [or] [2]
\installspecial [\dostoptransparency]  [or] [0]

%D \macros
%D   {doattachfile}
%D
%D \starttyping
%D \doattachfile{title}{width}{height}{depth}{color}{symbol}{filename}{source}
%D \stoptyping

\installspecial [\doattachfile] [or] [8]

%D Experimental (properties):

\installspecial[\dostartviewerlayer]      [or] [1]
\installspecial[\dostopviewerlayer]       [or] [0]
\installspecial[\dodefineviewerlayer]     [or] [5]
%installspecial[\doflushviewertextlayers] [or] [0]
%installspecial[\doflushviewerpagelayers] [or] [0]
\installspecial[\domakeviewerlayerlist]   [or] [1]

\installspecial[\doinsertrenderingwindow] [or] [4]
\installspecial[\doinsertrendering]       [or] [4]
\installspecial[\doinsertrenderingobject] [or] [4]
\installspecial[\doinsertrenderingobject] [or] [4]

\installspecial[\dostartfonteffect]       [or] [3]
\installspecial[\dostopfonteffect]        [or] [0]

%D From now on, mapfile loading is also a special; we assume the
%D more or less standard dvips syntax.

\installspecial[\doresetmapfilelist] [and] [0]
\installspecial[\doloadmapfile]      [and] [2] % + - = | filename
\installspecial[\doloadmapline]      [and] [2] % + - = | fileline

%D We define a couple of backends:

\defineoutput [dvipsone] [dvi,ps,yy]
\defineoutput [dviwindo] [dvi,ps,yy,win]
\defineoutput [dvips]    [dvi,ps,tr]
\defineoutput [dviview]  [dvi,ps,tr,dv]
\defineoutput [pdftex]   [tpd]
\defineoutput [pdf]      [tpd]
\defineoutput [dvipdfm]  [dpm]
\defineoutput [dpm]      [dpm]
\defineoutput [dvipdfmx] [dpx]
\defineoutput [dpx]      [dpx]
\defineoutput [acrobat]  [pdf,ps,tr]

\defineoutput [xetex]    [dpx]
\defineoutput [xtx]      [dpx]
\defineoutput [xdv]      [dvi,xtx]
\defineoutput [xdv2pdf]  [dvi,xtx]

%D Please let me know if we need more. From now on we default
%D to:

\setupoutput [dvips]

%D We don't enable \ACROBAT, because pure \POSTSCRIPT\ is not
%D that strong on objects and \PDFTEX\ does a better job.
%D Some reasonable alternatives are:
%D
%D \starttyping
%D \setupoutput [dvipsone,acrobat]
%D \setupoutput [dviwindo,acrobat]
%D \stoptyping
%D
%D Although, better is:
%D
%D \starttyping
%D \setupoutput [pdftex]
%D \stoptyping

\loadmarkfile{spec-def}

\protect \endinput
