%D \module
%D   [       file=enco-ini,
%D        version=2000.12.27, % 1998.12.03,
%D          title=\CONTEXT\ Encoding Macros,
%D       subtitle=Initialization,
%D         author=Hans Hagen,
%D           date=\currentdate,
%D      copyright={PRAGMA / Hans Hagen \& Ton Otten}]
%C
%C This module is part of the \CONTEXT\ macro||package and is
%C therefore copyrighted by \PRAGMA. See mreadme.pdf for 
%C details. 

%D This module is a reimplementation of the module that handled
%D composed characters and non \ASCII\ characters. The changed
%D are not that fundamental, and mainly concerns moving
%D definitions of specific glyphs and accents to other files as
%D well as moving plain handling of accents to this module
%D instead of overloading plain \TEX\ commands. 

% \everyuppercase
% \dotlessi
% single/double quotes 
% hyphens 
% characterencoding  => encoding 
% charactermapping => mapping 

%D Most of this module used to be part of the font and language
%D modules. While implementing Czech support, I decided to
%D isolate this code.

%D While dealing with input (the text source) and output (the
%D glyphs), encoding comes into view. To summarize a few:
%D
%D \startopsomming
%D \som  Bytes in the input file are mapped to an internal
%D       representation. An~\type {a} often stays an~\type {a},
%D       but~\type {\"e} can become either one code or become
%D       two codes (ending in overlapping glyphs).
%D \som  Characters can be made active and mapped onto another
%D       character.
%D \som  When changing case, characters are mapped onto
%D       themselves, their case||counterpart or a reasonable
%D       alternative, like~\"e onto~e.
%D \som  Single character representations in a \DVI\ file can 
%D       be mapped onto one or more characters, either of not
%D       in more than one font file (virtual fonts).
%D \som  In the final format, fonts collections can be 
%D       partially embedded, thereby losing the one||to||one 
%D       relation between several instances of one font.
%D \som  For special purposes, individual characters should be
%D       mapped onto a dedicated encoding vector, for instance
%D       \PDF\ document encoding.
%D \stopopsomming
%D
%D These and other kind of mappings are to be dealt with, and
%D the exact way of dealing often depends on the language to be
%D typeset.

\writestatus{loading}{Context Encoding Macros (ini)}

\unprotect 

\startmessages  dutch  library: encodings
  title: encoding
      1: codering --
      2: codering -- wordt geladen
      3: onbekende codering --
\stopmessages

\startmessages  english  library: encodings
  title: encoding
      1: coding --
      2: coding -- is loaded
      3: unknown coding --
\stopmessages

\startmessages  german  library: encodings
  title: Kodierung
      1: Kodierung --
      2: Kodierung -- ist geladen
      3: Unbekannte Kodierung --
\stopmessages

\startmessages  czech  library: encodings
  title: kodovani
      1: kodovani --
      2: je nacteno kodovani --
      3: nezname kodovani --
\stopmessages

\startmessages  italian  library: encodings
  title: codifica
      1: codifica --
      2: codifica -- caricata
      3: codifica sconosciuta --
\stopmessages

\startmessages  norwegian  library: encodings
  title: koding
      1: koding --
      2: koding -- er lest inn
      3: ukjent koding --
\stopmessages

\startmessages  romanian  library: encodings
  title: codificari
      1: codificarea --
      2: codificarea -- este �ncarcata
      3: codificarea -- este necunoscuta
\stopmessages

%D First we define a few local or not yet initialized constants. 

\def\@map@{@m@ap@} % mapping prefix 
\def\@reg@{@r@eg@} % regime prefix 
\def\@fha@{@f@ha@} % font prefix 

\ifx\currentlanguage\undefined \let\currentlanguage\s!en \fi

%D \macros
%D   {protectregime}
%D
%D The next boolean is used later on to prevent unwanted 
%D catcode changes. Use it with care. 

\newif\ifprotectregime \protectregimetrue

\def\setregimecode#1#2%
  {\ifprotectregime\ifnum\catcode#1=\active\else
     \catcode#1=#2\relax
   \fi\else     
     \catcode#1=#2\relax
   \fi}

%D \macros 
%D   {startregime, enableregime}
%D
%D Sometimes it makes sense to activate the characters in the 
%D upper half of the character table. Such a bundle of 
%D characters can be packages in a regime. Later we will see
%D encodings (that links characters slots to glyphs) and 
%D mappings (that take care of hyphenation and case changes). 
%D
%D When character~231 is of category code letter, it 
%D directly maps to glyph~231 (unless of course some virtual 
%D font is used). By making character~231 active, we can map
%D it onto for instance the glyph in slot 233. This mapping 
%D can in itself be indirect, in the sense that it is for 
%D instance handled by an accent command. 
%D
%D Regimes are implemented roughly the same as mappings, but
%D enabled under different circumstances. In the future, the 
%D low level implementation may change.  

\def\startregime[#1]%
  {\localpushmacro\characterregime
   \edef\characterregime{@#1@}%
   \doifundefined{\@reg@\characterregime}
     {\expanded{\newtoks\csname\@reg@\characterregime\endcsname}}}

\def\stopregime%
  {\localpopmacro\characterregime}

%\long\def\startregime[#1]#2\stopregime{}

\def\setregimetoks%
  {\@EA\let\@EA\regimetoks\csname\@reg@\characterregime\endcsname}

\let\enabledregime\empty 

\def\enableregime[#1]%
  {\edef\characterregime{@#1@}%
   \ifx\enabledregime\characterregime \else
     \doifdefined{\@reg@\characterregime}
       {\the\csname\@reg@\characterregime\endcsname}%
     \let\enabledregime\characterregime
   \fi}

%D \macros 
%D   {defineactivedecimal, defineactivedecimals, defineactivetoken} 
%D
%D The following commands are rather ugly ones. It makes a
%D character active and assigns it a value. When expanded, 
%D the decimal number of the character is passed as first 
%D argument. 
%D
%D \starttypen 
%D \def\decimalcharacter#1{\message{#1 is now active}}
%D
%D \defineactivedecimal 122 {\decimalcharacter}
%D
%D \defineactivedecimals 128 to 255 as {\decimalcharacter}
%D \stoptypen 
%D
%D This command is typically used in coding definitions, 
%D like the \UNICODE\ one. 

\def\dodefineactivedecimal#1#2% \unexpanded ? pdfdoc encoding 
  {\catcode#1=\active % maybe \protectregimetrue
   \scratchcounter=\the\uccode`~
   \uccode`~=#1\relax
   \uppercase{\unexpanded\edef~{\noexpand#2{\number#1}}}%
   \uccode`~=\scratchcounter}

\long\def\defineactivedecimal#1 #2 %
  {\setregimetoks
   \appendtoks\dodefineactivedecimal{#1}{#2}\to\regimetoks}

\long\def\defineactivedecimals#1 to #2 as #3 %
  {\setregimetoks
   \dostepwiserecurse{#1}{#2}{1}
     {\@EA\appendtoks\@EA\dodefineactivedecimal\@EA{\recurselevel}{#3}\to\regimetoks}}

\long\def\defineactivetoken #1 #2% watch the {} 
  {\setregimetoks
   \appendtoks\defineactivecharacter#1 {#2{}}\to\regimetoks}

%D ....
 
\edef\nocharacterregime{@\s!default @}

\def\definetoken #1 % #1 = rawtoken or number 
  {\doifnumberelse{\string#1}
     {\expanded{\dodefinetoken{\rawcharacter{#1}}}}
     {\expanded{\dodefinetoken{\string#1}}}}

\def\dodefinetoken#1#2% 
  {\defineactivecharacter#1 {\dohandletoken{#1}} %
   \setvalue{\characterregime#1}{#2}}

\beginTEX

\def\dohandletoken#1%
  {\csname\expandafter\ifx\csname\characterregime#1\endcsname\relax
     \nocharacterregime\else\characterregime\fi#1\endcsname}

\endTEX

\beginETEX \ifcsname

\def\dohandletoken#1%
  {\csname\ifcsname\characterregime#1\endcsname
     \characterregime\else\nocharacterregime\fi#1\endcsname}

\endETEX

%D .... 

\def\doautosetregime#1#2%
  {\ifnum#2>127
     \def\!!stringa{#2 }%
     \@EA\@EA\@EA\defineactivetoken\@EA\!!stringa\@EA{\csname#1\endcsname}%
   \fi}

%D \macros 
%D   {useencoding}
%D
%D Encodings things are defined in separate files and are 
%D loaded only once, using: 
%D
%D \showsetup{\y!useencoding}

\def\douseencoding#1%
  {\doifundefined{\c!file\f!encodingprefix#1}%
     {\setvalue{\c!file\f!encodingprefix#1}{}%
      \makeshortfilename[\f!encodingprefix#1]%
      \startreadingfile
      \readsysfile{\shortfilename}
        {\showmessage{\m!encodings}{2}{#1}}
        {\showmessage{\m!encodings}{3}{#1}}%
      \stopreadingfile}}

\def\useencoding[#1]%
  {\processcommalist[#1]\douseencoding}

%D \macros
%D   {startmapping,enablemapping}
%D
%D In order to process patterns, convert from lower to
%D uppercase and vise versa and some more, we provide a
%D mechanism to define mappings. The first real application
%D of this command was:
%D
%D \starttypen
%D \startmapping [something]
%D   \definecasemap 165 181 165   
%D   \definecasemap 171 187 171   
%D   ...
%D   \defineuppercasecom \i  {I} 
%D   \defineuppercasecom \l  \L 
%D   \definelowercasecom \AE \ae
%D   ...
%D \stopmapping
%D \stoptypen
%D
%D So, character 165 becomes 181 in uppercase and 165 in 
%D lowercase. A mapping is activated with \type {\enablemapping}.

\def\startmapping[#1]%
  {\pushmacro\charactermapping
   \edef\charactermapping{@#1@}%
   \doifundefined{\@map@\charactermapping}
     {\expanded{\newtoks\csname\@map@\charactermapping\endcsname}}}

\def\stopmapping%
  {\popmacro\charactermapping}

\def\setmappingtoks%
  {\@EA\let\@EA\mappingtoks\csname\@map@\charactermapping\endcsname}

\def\definecasemap #1 #2 #3 % code lower upper
  {\setmappingtoks
   \doifelse{#2}{to}
     {\appendtoks\presetcaserange{#1}{#3}\to\mappingtoks}
     {\appendtoks\setcasemap #1 #2 #3 \to\mappingtoks}%
   \ignorespaces}

%D Watch the \type {\definecasemap 127 to 255} option! 
%D Dedicated to Taco there is also: 

\def\definecasemaps #1 to #2 lc #3 uc #4 % from to lc+ uc+ 
  {\dostepwiserecurse{#1}{#2}{1}
     {\!!counta=\recurselevel\advance\!!counta by #3\relax
      \!!countb=\recurselevel\advance\!!countb by #4\relax
      \expanded{\definecasemap
        \recurselevel\space\the\!!counta\space\the\!!countb\space}}%
   \ignorespaces}

%D This can be used like: 
%D
%D \starttypen 
%D \definecasemaps 128 to 156 lc  32 uc 0 
%D \definecasemaps 160 to 188 lc -32 uc 0 
%D \definecasemaps 192 to 255 lc  32 uc 0 
%D \stoptypen 
%D
%D and saves a lot of typing (copying). 

\def\presetcaserange#1#2%
  {\dostepwiserecurse{#1}{#2}{1}
     {\setregimecode\recurselevel\@@letter
      \lccode \recurselevel=\recurselevel
      \uccode \recurselevel=\recurselevel}}

\def\setcasemap #1 #2 #3 %
  {\setregimecode{#1}\@@letter
   \lccode #1=#2 
   \uccode #1=#3 }

\def\definespacemap #1 #2 % code sfcode
  {\setmappingtoks
   \appendtoks\setspacemap #1 #2 \to\mappingtoks
   \ignorespaces}

\def\setspacemap #1 #2 %
  {\setregimecode{#1}\@@other
   \lccode #1=0 
   \uccode #1=0 
   \sfcode #1=#2 } 

\def\defineuppercasecom#1#2% 
  {\setmappingtoks
   \appendtoks\setuppercasecom#1{#2}\to\mappingtoks
   \ignorespaces}

\def\definelowercasecom#1#2% 
  {\setmappingtoks
   \appendtoks\setlowercasecom#1{#2}\to\mappingtoks
   \ignorespaces}

\let\setuppercasecom\gobbletwoarguments
\let\setlowercasecom\gobbletwoarguments

\def\setcasecom#1#2{\def#1{#2}}

\let\enabledmapping\empty % indirect, needed to handle default too

\def\enablemapping[#1]%
  {\edef\charactermapping{@#1@}%
   \ifx\enabledmapping\charactermapping \else
     \doifdefined{\@map@\charactermapping}
       {\the\csname\@map@\charactermapping\endcsname}%
     \enablelanguagespecifics[\currentlanguage]% new 
     \edef\enabledmapping{\charactermapping\currentlanguage}%
   \fi}

%D This macro wil be implemented in \type {lang-ini.tex}. 

\ifx\enablelanguagespecifics\undefined
  \def\enablelanguagespecifics[#1]{}
\fi

%D Further on we have to take some precautions when dealing
%D with special characters like~\type{~}, \type{_}
%D and~\type{^}, so let us define ourselve some handy macros
%D first.

\def\protectfontcharacters%
  {\edef\unprotectfontcharacters%
     {\catcode`\noexpand ~=\the\catcode`~\relax
      \catcode`\noexpand _=\the\catcode`_\relax
      \catcode`\noexpand ^=\the\catcode`^\relax}%
   \catcode`~=\@@letter
   \catcode`_=\@@letter
   \catcode`^=\@@letter\relax}

%D The completeness of the Computer Modern Roman typefaces
%D makes clear how incomplete other faces are. To honour 7~bit
%D \ASCII, these fonts were designed using only the first 127
%D values of the 256 ones that can be presented by one byte.
%D Nowadays 8~bit character codings are more common, mainly
%D because they permit us to predefine some composed
%D characters, which are needed in most european languages.
%D
%D Supporting more than the standard \TEX\ encoding vector
%D |<|which in itself is far from standard and differs per
%D font|>| puts a burden on the fonts mechanism. The \CONTEXT\
%D mechanism is far from complete, but can handle several
%D schemes at once. The main problem lays in the accented
%D characters and ligatures like~ff, although handling
%D ligatures is not the responsibility of this module.
%D
%D By default, we use \PLAIN\ \TEX's approach of placing
%D accents. All other schemes sooner or later give problems
%D when we distribute \DVI||files are distributed across
%D machines and platforms. Nevertheless, we have to take care
%D of different encoding vectors, which tell us where to find
%D the characters we need. This means that all kind of
%D character placement macro's like \type{\"} and \type{\ae}
%D have to be implemented  and adapted in a way that suits
%D these vectors.
%D
%D The main difference between different vector is the way
%D accents are ordered and/or the availability of prebuilt
%D accented characters. Accented characters can for instance be
%D called for by sequences like \type{\"e}. Here the \type{\"}
%D is defined as:
%D
%D \starttypen
%D \def\"#1{{\accent"7F #1}}
%D \stoptypen
%D
%D This macro places the accent \accent"7F {} on top of an~e
%D gives \"e. Some fonts however can have prebuild accents and
%D use a more direct approach like
%D
%D \starttypen
%D \def\"#1{\if#1e\char 235\else ... \fi}
%D \stoptypen
%D
%D The latter approach is not used in \CONTEXT, because we
%D store relevant combinations of accents and characters in
%D individual macros.

%D We define character substitutes and commands with definition
%D commands like:
%D
%D \starttypen
%D \startcoding[texnansi]
%D
%D \defineaccent " a           228
%D \defineaccent ^ e           234
%D \defineaccent ' {\dotlessi} 237
%D
%D \definecharacter ae 230
%D \definecharacter oe 156
%D
%D \definecommand b \texnansiencodedb
%D \definecommand c \texnansiencodedc
%D
%D \stopcoding
%D \stoptypen
%D
%D The last argument of \type{\defineaccent} and
%D \type{\definecharacter} tells \TEX\ the position of the
%D accented character in the encoding vector. In order to
%D complish this, we tag each implementation with the character
%D coding identifier. We therefore need two auxiliary variables
%D \type{\characterencoding} and \type{\nocharacterencoding}. These
%D contain the current and default encoding vectors and both
%D default to the \PLAIN\ one.

\edef\characterencoding   {@\s!default @}
\edef\nocharacterencoding {@\s!default @}
\edef\charactermapping    {@\s!default @}
\edef\characterregime     {@\s!default @}

%D \macros
%D   {startcoding, reducetocoding}
%D
%D Before we can redefine accents and special characters, we
%D have to tell \CONTEXT\ what encoding is in force. The next
%D command is responsible for doing this and also takes care of
%D the definition of the recoding commands. We use the \type 
%D {\start}||\type {\stop}||commands for definitions and the 
%D \type {\reduceto}||command for local switching to 
%D simplified commands. 

%\def\donthandleaccent#1#2% \empty makes #2={} save % no \unexpanded 
%  {\ifundefined{\characterencoding#1\string#2\empty}% 
%     #2%
%   \else
%     \getvalue{\characterencoding#1\string#2\empty}%
%   \fi}

\def\donthandlecommand#1% % no \unexpanded, otherwise pdfdoc will fail 
  {\ifundefined{\characterencoding#1}%
     #1%
   \else
     \getvalue{\characterencoding#1}%
   \fi}

\def\enablecoding%
  {\dodoubleempty\doenableencoding}

\def\doenableencoding[#1][#2]% main fallback 
  {\iffirstargument\edef\characterencoding{@#1@}\fi
   \edef\nocharacterencoding{@\ifsecondargument#2\else\s!default\fi @}}

\def\startcoding%
  {\dodoubleempty\dostartcoding}

\def\dostartcoding[#1][#2]% encoding regime 
  {\doifelsenothing{#1}
     {\let\stopcoding\relax}
     {%\protectfontcharacters % problematic in language loading
      \showmessage{\m!encodings}{1}{#1}%
      \pushmacro\dohandleaccent
      \pushmacro\dohandlecommand
      \pushmacro\definesortkey
      \pushmacro\currentcharacterregime
      \pushmacro\dosetautoregime
      \let\dohandleaccent\donthandleaccent
      \let\dohandlecommand\donthandlecommand
      \let\definesortkey\savesortkey
      \doifelsenothing{#2}%
        {% message 
         \let\doautosetregime\gobbletwoarguments}
        {\def\currentcharacterregime{@#2@}}%
      \enablecoding[#1]%
      \def\stopcoding%
        {\popmacro\dosetautoregime
         \popmacro\currentcharacterregime
         \popmacro\definesortkey
         \popmacro\dohandlecommand
         \popmacro\dohandleaccent
         \enablecoding[\s!default]%
         }}}% \unprotectfontcharacters}}} % ?? 

% probably obsolete 

\def\reducetocoding[#1]% use grouped! 
  {\doifsomething{#1}
     {\let\dohandleaccent\donthandleaccent
      \let\dohandlecommand\donthandlecommand
      \enablecoding[#1]%
      \enablelanguagespecifics[\currentlanguage]}}

\def\startencoding {\startcoding}
\def\stopencoding  {\stopcoding}
\def\enableencoding{\enablecoding}

%D The use of these macros are not limited to font 
%D definition files, but may also be used when loading 
%D patterns. 

%D \macros 
%D   {definesortkey,flushsortkeys,flushsortkey}
%D
%D Yet another definition concerns sorting of indexes and 
%D lists.  
%D
%D \starttypen
%D \definesortkey {\'e} {e} {a} {\'e}  
%D \stoptypen
%D
%D The first argument denotes the string to be treated. The 
%D second argument is the raw replacement, while the third 
%D argument determines the sort order given the replacement. 
%D The last argument is used as entry in the index (a, b, etc).
%D
%D The keys can be flished using \type {\flushsortkeys} 
%D which in turn results in a sequence of calls to \type 
%D {\flushsortkey}, a macro taking 4~arguments.  
%D
%D This mechanism is currently being tested and subjected to
%D changes!  

\def\savesortkey#1#2#3#4%
  {\let\flushsortkey\relax % important  
   \edef\!!stringa{sort:\characterencoding}%
   \ifundefined\!!stringa 
     \let\!!stringb\empty
   \else
     \@EA\def\@EA\!!stringb\@EA{\csname\!!stringa\endcsname}%
   \fi
   \convertargument#1\to\asciiA \convertargument#2\to\asciiB
   \convertargument#3\to\asciiC \convertargument#4\to\asciiD
   \setevalue{\!!stringa}{\!!stringb\flushsortkey{\asciiA}{\asciiB}{\asciiC}{\asciiD}}}

\def\definesortkey#1#2#3#4%
  {}

\def\flushsortkeys%
  {\enablelanguagespecifics[\currentlanguage]%
   \getvalue{sort:\characterencoding}}

\let\flushsortkey\relax

%D \macros
%D   {defineaccent, definecharacter, definecommand}
%D
%D The actual definition of accents, special characters and
%D commands is done with the next three commands.

\def\defineaccent%
  {\protectfontcharacters
   \dodefineaccent}

\def\dodefineaccent#1 #2 %
  {\unprotectfontcharacters
   \dododefineaccent#1 #2 }

% obsolete 
%
% \def\dododefineaccent#1 #2 #3 %
%   {\redefineaccent #1 % just to be sure
%    \doifnumberelse{\string#3}
%      {\setvalue{\characterencoding#1\string#2}{\char#3 }} % space added
%      {\setvalue{\characterencoding#1\string#2}{#3}}%
%    \unprotectfontcharacters}

\def\dododefineaccent#1 #2 #3 %
  {\setvalue{#1}{\dohandleaccent{#1}}%
   \doifnumberelse{\string#3}
     {\setvalue{\characterencoding#1\string#2}{\char#3 }} % space added
     {\setvalue{\characterencoding#1\string#2}{#3}}%
   \unprotectfontcharacters}

\beginTEX

\def\dohandleaccent#1#2%
  {\@EA\ifx\csname\characterencoding#1\string#2\empty\endcsname\relax 
     \@EA\ifx\csname\nocharacterencoding#1\string#2\empty\endcsname\relax
       \@EA\ifx\csname\characterencoding#1\endcsname\relax
%         \@EA\ifx\csname\nocharacterencoding#1\endcsname\relax
%           \donormaltextaccent{#1}{#2}%
%         \else
           \csname\nocharacterencoding#1\endcsname#2%
%         \fi
       \else
         \csname\characterencoding#1\endcsname#2%
       \fi
     \else
       \csname\nocharacterencoding#1\string#2\empty\endcsname
     \fi
   \else
     \csname\characterencoding#1\string#2\endcsname
   \fi}

\endTEX

\beginETEX \ifcsname

\def\dohandleaccent#1#2%
  {\ifcsname\characterencoding#1\string#2\empty\endcsname 
     \csname\characterencoding#1\string#2\endcsname
   \else\ifcsname\nocharacterencoding#1\string#2\empty\endcsname
     \csname\nocharacterencoding#1\string#2\empty\endcsname
   \else\ifcsname\characterencoding#1\endcsname
     \csname\characterencoding#1\endcsname#2%
   \else%\ifcsname\nocharacterencoding#1\endcsname
     \csname\nocharacterencoding#1\endcsname#2%
%   \else
%     \donormaltextaccent{#1}{#2}%
   \fi\fi\fi}%\fi}

\endETEX

%\def\presetcharacter#1%
%  {\setvalue{#1}{\dohandlecharacter{#1}}}

\def\definecharacter#1 #2 %
  {\setvalue{#1}{\dohandlecharacter{#1}}%
%  {\@EA\presetcharacter\@EA{\@EA\strippedcsname\csname#1\endcsname}%
   \doifnumberelse{\string#2}
     {\setvalue{\characterencoding\string#1}{\char#2 }% watch the space
      \doautosetregime{#1}{#2}}
     {\setvalue{\characterencoding\string#1}{#2}}}

\beginTEX

\def\dohandlecharacter#1%
  {\csname\expandafter\ifx\csname\characterencoding#1\endcsname\relax
     \nocharacterencoding\else\characterencoding\fi#1\endcsname}

\endTEX

\beginETEX \ifcsname

\def\dohandlecharacter#1%
  {\csname\ifcsname\characterencoding#1\endcsname
     \characterencoding\else\nocharacterencoding\fi#1\endcsname}

\endETEX

%D Instead of numbers, a command may be entered.

\def\definecommand#1 #2 %
  {\setvalue{\string#1}{\dohandlecommand{#1}}%
  %\redefinecommand #1 % just to be sure
   \setvalue{\characterencoding\string#1}{#2}}
 
%D Here we see that redefining accents is characters is more
%D or less the same as redefining commands. We also could have
%D said:
%D
%D \starttypen
%D \def\defineaccent#1 #2 {\definecommand#1\string#2 \char}
%D \def\definecharacter#1 {\definecommand#1 \char}
%D \stoptypen

% obsolete
% 
% %D \macros
% %D   {redefineaccent}
% %D
% %D Telling \CONTEXT\ how to treat accents and special
% %D characters is a two stage process. First we signal the
% %D system which commands are to be adapted, after which we can
% %D redefine their behavior when needed. We showed this in the
% %D previous paragraphs. These redefinitions are grouped at the
% %D end of this file, but we show some examples here.
% %D
% %D Accents or accent generating commands are redefined by:
% %D
% %D \starttypen
% %D \redefineaccent  '  % grave
% %D \redefineaccent  "  % diaeresis
% %D \redefineaccent  ^  % circumflex
% %D \redefineaccent  v  % caron
% %D \stoptypen
% %D
% %D The original \PLAIN\ \TEX\ meaning of each accent generating
% %D command is saved first. Next these commands are redefined to
% %D do an indirect call to a macro that acts according to the
% %D encoding vector in use.
%
% \def\redefineaccent%
%   {\protectfontcharacters
%    \doredefineaccent}
% 
% \def\doredefineaccent#1 %
%   {\def\!!stringa{\nocharacterencoding\string#1}%
%    \doifundefined{\!!stringa}
%      {\@EA\letvalue\@EA\!!stringa\@EA=\csname\string#1\endcsname}%
%       % no \unexpanded, else pdfdoc fails 
%       \setvalue{\string#1}{\dohandleaccent#1}%
%    \unprotectfontcharacters}
%
% \def\doredefineaccent#1 %
%   {\setvalue{#1}{\dohandleaccent{#1}}%
%    \unprotectfontcharacters}

%D \macros
%D   {defineaccentcommand}
%D
%D When needed, one can overload the default positions of the
%D accents. The \PLAIN\ \TEX\ defaults are: 
%D
%D \starttypen
%D \defineaccentcommand `  18
%D \defineaccentcommand '  19
%D \defineaccentcommand v  20
%D \defineaccentcommand u  21
%D \defineaccentcommand =  22
%D \defineaccentcommand ^  94
%D \defineaccentcommand .  95
%D \defineaccentcommand H 125  % "7D 
%D \defineaccentcommand ~ 126  % "7E 
%D \defineaccentcommand " 127  % "7F 
%D \stoptypen

\def\defineaccentcommand%
  {\protectfontcharacters
   \dodefineaccentcommand}

\def\dodefineaccentcommand#1 #2 % \string toegevoegd 
  {\doifnumberelse{\string#2}
     {\setvalue{\characterencoding\string#1}##1{{\accent#2 ##1}}}
     {\setvalue{\characterencoding\string#1}##1{{#2##1}}}%
   \unprotectfontcharacters}

%D We don't have to define them for the default \PLAIN\ case. 
%D Commands may be used instead of character codes. 

%D \macros
%D   {normalaccent,normalchar}
%D
%D Accents are either placed by \TEX's \type {\accent} 
%D primitive, or part of the glyph. By default the former 
%D method is used, unless overruled in the encoding 
%D definitions.

\let\normalchar  =\char
\let\normalaccent=\accent

% \beginETEX \ifcsname
%
% \unexpanded\def\dohandleaccent#1#2%
%   {\def\glyph{#2}%
%    \ifx\glyph\empty
%       \dohandleaccent#1\relax
%    \else\ifx\glyph\space
%      \dohandleaccent#1\relax
%    \else\ifcsname\characterencoding#1\string#2\empty\endcsname
%      \csname\characterencoding#1\string#2\endcsname
%    \else\ifcsname\characterencoding#1\endcsname
%      \csname\characterencoding#1\endcsname#2%
%    \else
%      \csname\nocharacterencoding#1\endcsname#2%
%    \fi\fi\fi\fi
%    \relax} % prevents further reading 
%
% \endETEX
%
% \beginTEX 
%
% \unexpanded\def\dohandleaccent#1#2%
%   {\def\glyph{#2}%
%    \ifx\glyph\empty
%      \dohandleaccent#1\relax
%    \else\ifx\glyph\space
%      \dohandleaccent#1\relax
%    \else\expandafter\ifx\csname\characterencoding#1\string#2\empty\endcsname\relax
%      \expandafter\ifx\csname\characterencoding#1\endcsname\relax
%        \csname\nocharacterencoding#1\endcsname#2%
%      \else
%        \csname\characterencoding#1\endcsname#2%
%       \fi
%    \else
%      \csname\characterencoding#1\string#2\endcsname
%    \fi\fi\fi
%    \relax} % prevents further reading 
% 
% \endTEX
%
% %D The trick with \type{\\} is needed to prevent spaces from
% %D being gobbled after the accented character, should we have
% %D used \type{\next}, we should have ended up with gobbled 
% %D spaces. The \type {\empty} after \type {#2} takes care of
% %D empty arguments, so that we can savely say~\type{\"{}} 
% %D and alike. 

%D \macros
%D   {redefinecommand}
%D
%D Redefinition of encoding dependant commands like \type{\b}
%D and \type{\c} can be triggered by:
%D
%D \starttypen
%D \redefinecommand  b  % something math
%D \redefinecommand  c  % something math
%D \stoptypen
%D
%D Handling of characters is easier than handling accents
%D because here we don't have to take care of arguments. We
%D just call for the right glyph in the right place.
%D
%D The \type{\next} construction permits handling of commands
%D that take arguments. This means that we can use this
%D command to redefine accent handling commands too.

\def\redefinecommand#1 %
 {%{\def\!!stringa{\nocharacterencoding#1}%
  % \doifundefined{\!!stringa}
  %   {\doifundefined{#1}{\letvalue{#1}\relax}%
  %    \@EA\letvalue\@EA\!!stringa\csname#1\endcsname}%
   % no \unexpanded, else pdfdoc fails 
   \setvalue{\string#1}{\dohandlecommand{#1}}}%

\unexpanded\def\dohandlecommand#1%
  {\doifdefinedelse{\characterencoding#1}
     {\def\next{\getvalue{\characterencoding#1}}}
     {\def\next{\getvalue{\nocharacterencoding#1}}}%
   \next}

% %D \macros
% %D   {redefinecharacter}
% %D
% %D Special characters, which differ from accented characters
% %D in that they are to be presented as they are, are redefined
% %D by
% %D
% %D \starttypen
% %D \redefinecharacter  ae  % ae
% %D \redefinecharacter  cc  % ccedilla
% %D \stoptypen
% %D
% %D To keep things simple, we just copy this command:
%
% \let\redefinecharacter=\redefinecommand

%D \macros 
%D   {currentencoding, currentregime, currentmapping}
%D
%D When we show 'm, we don't want to see the protection 
%D measures. 

\def\currentencoding{\@EA\docurrentencoding\characterencoding}
\def\currentregime  {\@EA\docurrentencoding\characterregime  }
\def\currentmapping {\@EA\docurrentencoding\charactermapping }

\def\docurrentencoding @#1@{#1}

%D \macros 
%D   {showaccents, showcharacters}
%D
%D Encoding is a tricky business. Therefore we provide a 
%D a few macros that show most of the characters involved. The
%D next two tables show the result of \type {\showaccents}.
%D
%D \plaatstabel
%D   {The special glyphs in default encoding.} 
%D   {\showaccents}
%D
%D \plaatstabel
%D   {The special glyphs in texnansi encoding.} 
%D   {\switchtobodyfont[lbr]\showaccents}

\fetchruntimecommand \showaccents    {\f!encodingprefix run}
\fetchruntimecommand \showcharacters {\f!encodingprefix run}

%D {\em The next section is experimental and implements font 
%D specific features, like hanging punctuation.}

\let\fonthandling\empty

\def\startfonthandling[#1]%
  {\def\fonthandling{#1}%
   \doifundefined{\@fha@\fonthandling}
     {\expanded{\newtoks\csname\@fha@\fonthandling\endcsname}}}

\def\stopfonthandling%
  {\let\fonthandling\empty}

\def\setfonttoks%
  {\@EA\let\@EA\fonttoks\csname\@fha@\fonthandling\endcsname}

\def\definefonthandling%
  {\dotripleempty\dodefinefonthandling}

\def\dodefinefonthandling[#1][#2][#3]%
  {\setvalue{\@fha@\@fha@#1}{#2}%
   \getparameters[\@fha@\@fha@#1][\c!links=1,\c!rechts=1,#3]}

\def\setupfonthandling%
  {\dodoubleempty\dosetupfonthandling}

\def\dosetupfonthandling[#1][#2]%
  {\getparameters[\@fha@\@fha@#1][#2]}  

\def\doenablehandling#1%
  {\doifdefined{\@fha@#1}{\@EA\the\csname\@fha@#1\endcsname}}

\def\enablehandling[#1]%
  {\doifdefinedelse{\@fha@\@fha@#1}%
     {\setprotrudingfactor
        {\getvalue{\@fha@\@fha@#1\c!links}}
        {\getvalue{\@fha@\@fha@#1\c!rechts}}%
      \edef\fonthandling{\getvalue{\@fha@\@fha@#1}}%
      \@EA\rawprocesscommalist\@EA[\fonthandling]\doenablehandling}
     {\edef\fonthandling{#1}% new
      \doenablehandling{#1}}}

\ifx\undefined\pdfprotrudechars % we don't use pdftex 

  \def\defineprotrudefactor#1 #2 #3 {}
  \def\setprotrudingfactor      #1#2{}
  \def\enableprotruding             {}
  \def\disableprotruding            {}

\else

  \newdimen\lproddimen 
  \newdimen\rproddimen 
  \let\prodfont\font 

  \def\enableprotruding {\pdfprotrudechars=2 }
  \def\disableprotruding{\pdfprotrudechars=0 }

  \appendtoks \disableprotruding \to \everyforgetall % Here or not here? 

  \def\setprotrudingfactor#1#2%
    {\lproddimen=#1pt\multiply\lproddimen\!!thousand\divide\lproddimen \!!maxcard\relax
     \rproddimen=#2pt\multiply\rproddimen\!!thousand\divide\rproddimen \!!maxcard\relax}

  \setprotrudingfactor{1}{1} 

  \def\dodefineprotrudefactor#1 #2 #3 %
    {\scratchdimen=#2\lproddimen\lpcode\prodfont#1=\number\scratchdimen 
     \scratchdimen=#3\rproddimen\rpcode\prodfont#1=\number\scratchdimen}
 
  \def\dodefinefonthandling[#1][#2][#3]%
    {\setvalue{\@fha@\@fha@#1}{#2}%
     \getparameters[\@fha@\@fha@#1][\c!links=1,\c!rechts=1,#3]}
  
  \def\defineprotrudefactor#1 #2 #3 %
    {\setfonttoks
     \doifnumberelse{\string#1}
       {\appendtoks\dodefineprotrudefactor #1 #2 #3 \to\fonttoks} % 
       {\appendtoks\dodefineprotrudefactor`#1 #2 #3 \to\fonttoks}}% 

  \def\defineprotrudefactor#1 #2 #3 %
    {\setfonttoks
     \doifnumberelse{\string#1}
       {\appendtoks\dodefineprotrudefactor #1 #2 #3 \to\fonttoks} 
       {\doifcharactercodeelse{#1}
          {\@EA\appendtoks\@EA\dodefineprotrudefactor\charactercode #2 #3 \to\fonttoks} 
          {\appendtoks\dodefineprotrudefactor`#1 #2 #3 \to\fonttoks}}} 

\fi

%D {\em Another experiment: named glyphs. Incomplete yet.}

\def\definecharactercode#1 #2 %
  {\setvalue{\characterencoding-#1}{#2}}

\def\doifcharactercodeelse#1#2#3%
  {\doifdefinedelse{\characterencoding-\string#1}
     {\edef\charactercode{\getvalue{\characterencoding-#1}\space}#2}
     {#3}}

%D {\em So far for the experiment.}

%D \macros
%D   {everyuppercase, EveryUppercase,
%D    everyuppercase, EveryUppercase}
%D
%D When we want to uppercase strings of characters, we have to
%D take care of those characters that have a special meaning or
%D are only accessible by means of macros. The next hack was
%D introduced when Tobias Burnus started translating head and
%D label texts into spanish and italian. The first application
%D of this token register therefore can be found in the module
%D that deals with these texts.

\newevery \everyuppercase \EveryUppercase
\newevery \everylowercase \EveryLowercase

%D This magic trick maps takes care of mapping from lower to 
%D upper case and reverse. 

\appendtoks\let\setuppercasecom\setcasecom\to\everyuppercase
\appendtoks\let\setlowercasecom\setcasecom\to\everylowercase

\newtoks\everyULmap 

\appendtoks\let\remapcase\remapuppercase\the\everyULmap\to\everyuppercase
\appendtoks\let\remapcase\remaplowercase\the\everyULmap\to\everylowercase

\let\remapcase\gobbletwoarguments
\def\remapuppercase#1#2{\let#2#1}
\def\remaplowercase#1#2{\let#1#2}

\def\defineLCcharacter #1 #2 %
  {\appendtoks\let\to\everylowercase
   \@EA\appendtoks\csname#1\endcsname\to\everylowercase
   \@EA\appendtoks\csname#2\endcsname\to\everylowercase}

\def\defineUCcharacter #1 #2 %
  {\appendtoks\let\to\everyuppercase
   \@EA\appendtoks\csname#1\endcsname\to\everyuppercase
   \@EA\appendtoks\csname#2\endcsname\to\everyuppercase}

\def\defineULcharacter #1 #2 %
  {\appendtoks\remapcase\to\everyULmap
   \@EA\appendtoks\csname#1\endcsname\to\everyULmap
   \@EA\appendtoks\csname#2\endcsname\to\everyULmap}

%D \macros
%D   {everysanitize, EverySanitize}
%D
%D Whenever we are sanitizing strings, like we sometimes do
%D when we deal with specials, the next token register can be
%D called.

\newevery \everysanitize \EverySanitize

%D \macros
%D   {obeylccodes}
%D
%D One way of manipulating characters is changing the their
%D \type{\lccode} and applying \type{\lowcase}. An example of
%D this can be found in \type{spec-mis}.

\def\obeylccodes%
  {\scratchcounter=32
   \loop
     \ifnum\scratchcounter<127
       \lccode\scratchcounter=\scratchcounter
       \advance\scratchcounter by 1
   \repeat
   \ifeightbitcharacters
     \scratchcounter=128
     \loop
       \ifnum\scratchcounter<255
         \lccode\scratchcounter=`.
         \advance\scratchcounter by 1
     \repeat
   \fi}

% %D \macros 
% %D   {cc,CC}
% %D
% %D Hm, not in plain at all, those \cc's and \CC's. 
% 
% \def\CC{\c{C}}  
% \def\cc{\c{c}}  
% 
% %D \macros
% %D   {dotlessi,dotlessj}
% %D
% %D We also save both dotless~\dotlessi\ and~\dotlessj. This
% %D way we still have them were we expect them, even when
% %D macros of font providers redefine them.
% 
% \let\dotlessi=\i
% \let\dotlessj=\j

%D \macros
%D   {defineuclass,defineudigit,udigit}
%D
%D The next few macros are experimental and needed for unicoded
%D chinese characters. 

\def\defineuclass #1 #2 #3 {\setvalue{uc#2#3}{#1}}
\def\defineudigit #1 #2 #3 {\setvalue{\characterencoding uc#1}{\uchar{#2}{#3}}}

%D It may look strange, but for the moment, we want the encoding 
%D to be part of the digit specification. This may change! 

\unexpanded\def\udigit#1#2{\getvalue{@#1@uc\number#2}}

%D \macros 
%D   {uchar, octuchar, hexuchar} 

\ifx\uchar\undefined \def\uchar#1#2{(\number#1,\number#2)} \fi 

\def\octuchar#1#2{\uchar{`#1}{`#2}}
\def\hexuchar#1#2{\uchar{"#1}{"#2}}

% %D Just to be sure, we save the original values of \type {\ss}. 
%
% \ifx\undefined\SS \let\SS=\ss \fi
% \ifx\undefined\sz \let\sz=\ss \fi
%
% %D Here come the definitions.
%
% \redefineaccent    '   % grave
% \redefineaccent    `   % acute
% \redefineaccent    "   % diaeresis
% \redefineaccent    ^   % circumflex
% \redefineaccent    ~   % tilde
% \redefineaccent    v   % caron
% \redefineaccent    u   % breve
% \redefineaccent    .   % dotaccent
% \redefineaccent    H   % hungarumlaut
% \redefineaccent    t   % ........
% \redefineaccent    r   % ........
% \redefineaccent    =   
% \redefineaccent    b   
% \redefineaccent    c   
% \redefineaccent    d   
% \redefineaccent    k   

% obsolete (moved)  
% 
% \redefinecharacter ae  % ae
% \redefinecharacter AE  % AE
% \redefinecharacter oe  % oe
% \redefinecharacter OE  % OE
% \redefinecharacter o   % ostroke
% \redefinecharacter O   % Ostroke
% \redefinecharacter sz  % germandbls
% \redefinecharacter SS  % germandbls
% \redefinecharacter aa  % aring
% \redefinecharacter AA  % Aring

% \redefinecharacter th
% \redefinecharacter TH
% \redefinecharacter ng
% \redefinecharacter NG
% \redefinecharacter ij
% \redefinecharacter IJ
% 
% \redefinecharacter i  \redefinecharacter dotlessi 
% \redefinecharacter j  \redefinecharacter dotlessj 
% 
% \redefinecharacter l 
% \redefinecharacter L 

% replaced
%
% \defineaccent " i {\"\i}  \defineaccent " j {\"\j}
% \defineaccent ^ i {\^\i}  \defineaccent ^ j {\^\j}
% \defineaccent ` i {\`\i}  \defineaccent ` j {\`\j}
% \defineaccent ' i {\'\i}  \defineaccent ' j {\'\j}
% \defineaccent ~ i {\~\i}  \defineaccent ~ j {\~\j}

% \redefinecharacter leftguillemot 
% \redefinecharacter rightguillemot  
% \redefinecharacter leftsubguillemot 
% \redefinecharacter rightsubguillemot  

% obsolete 
% 
% %D Some more:
% 
% \startmapping[\s!default] 
% 
% \defineuppercasecom \i  {I}
% \defineuppercasecom \j  {J}
% \defineuppercasecom \sz {SS}
% \defineuppercasecom \SS {SS}
% \defineuppercasecom \l  \L 
% \defineuppercasecom \ae \AE
% \defineuppercasecom \aa \AA
% \defineuppercasecom \o  \O 
% \defineuppercasecom \oe \OE
% \definelowercasecom \L  \l 
% \definelowercasecom \AE \ae
% \definelowercasecom \AA \aa
% \definelowercasecom \O  \o 
% \definelowercasecom \OE \oe
% 
% \stopmapping

%D Basics and fallbacks.

\newif\ifignoreaccent 

\let\textaccent      \accent

\let\normalaccent    \accent
\let\normaltextaccent\textaccent
\let\normalmathaccent\mathaccent
\let\normalchar      \char

\def\buildtextaccent%
  {\ifignoreaccent
     \expandafter\nobuildtextaccent
   \else
     \expandafter\dobuildtextaccent
   \fi}

\unexpanded\def\dobuildtextaccent#1#2%
  {{\let\char\normalaccent#1\let\char\normalchar#2}}

\unexpanded\def\nobuildtextaccent#1#2%
  {#2}

\def\buildmathaccent#1%
  {\mathaccent#1 }

% will be overloaded later 

%\def\definetextaccent#1 #2%
%  {\setvalue{\string#1}{#2}% will be overloaded 
%   \setvalue{normaltextaccent\string#1}{#2}}
%
%\def\donormaltextaccent#1%
%  {\getvalue{normaltextaccent\string#1}}
%
%\definetextaccent ` {\buildtextaccent\textgrave}
%\definetextaccent ' {\buildtextaccent\textacute}
%\definetextaccent v {\buildtextaccent\textcaron}
%\definetextaccent u {\buildtextaccent\textbreve}
%\definetextaccent = {\buildtextaccent\textmacron}
%\definetextaccent ^ {\buildtextaccent\textcircumflex}
%\definetextaccent . {\buildtextaccent\textdotaccent}
%\definetextaccent H {\buildtextaccent\texthungarumlaut}
%\definetextaccent ~ {\buildtextaccent\texttilde}
%\definetextaccent " {\buildtextaccent\textdiaeresis}

\definecommand ` {\buildtextaccent\textgrave}
\definecommand ' {\buildtextaccent\textacute}
\definecommand v {\buildtextaccent\textcaron}
\definecommand u {\buildtextaccent\textbreve}
\definecommand = {\buildtextaccent\textmacron}
\definecommand ^ {\buildtextaccent\textcircumflex}
\definecommand . {\buildtextaccent\textdotaccent}
\definecommand H {\buildtextaccent\texthungarumlaut}
\definecommand ~ {\buildtextaccent\texttilde}
\definecommand " {\buildtextaccent\textdiaeresis}

% some fake ones, name will change into build 

\def\bottomaccent#1#2#3#4#5% down right slantcorrection accent char 
  {\leavevmode
   \vtop
     {\forgetall
      \baselineskip\!!zeropoint
      \lineskip#1%
      \everycr\emptytoks
      \tabskip\!!zeropoint
      \lineskiplimit\!!zeropoint
      \setbox0=\hbox{#4}%
      \halign
        {##\crcr#5\crcr
         \hidewidth
         \hskip#2\wd0
         \hskip-#3\fontdimen1\font % in plain 1ex * dimenless value
         \vbox to .2ex{\box0\vss}\hidewidth
         \crcr}}} 

\def\buildtextmacron   {\bottomaccent{.25ex}{0}{15}{\textmacron}}
\def\buildtextbottomdot{\bottomaccent{.25ex}{0}{5}{\textbottomdot}}
\def\buildtextcedilla  {\bottomaccent{0ex}{0}{5}{\textcedilla}}
\def\buildtextogonek   {\bottomaccent{-.1ex}{.5}{0}{\textogonek}}

%\definetextaccent c {\buildtextcedilla}
%\definetextaccent b {\buildtextmacron} 
%\definetextaccent d {\buildtextbottomdot}
%\definetextaccent k {\buildtextogonek}

\definecommand c {\buildtextcedilla}
\definecommand b {\buildtextmacron} 
\definecommand d {\buildtextbottomdot}
\definecommand k {\buildtextogonek}

% math stuff, will change 

\def\definemathaccent#1 #2% 
  {\setvalue{\string#1}{#2}%
   \setvalue{normalmathaccent\string#1}{#2}}

\def\donormalmathaccent#1%
  {\getvalue{normalmathaccent\string#1}}

\definemathaccent acute     {\buildmathaccent\mathacute}
\definemathaccent grave     {\buildmathaccent\mathgrave}
\definemathaccent ddot      {\buildmathaccent\mathddot}
\definemathaccent tilde     {\buildmathaccent\mathtilde}
\definemathaccent bar       {\buildmathaccent\mathbar}
\definemathaccent breve     {\buildmathaccent\mathbreve}
\definemathaccent check     {\buildmathaccent\mathcheck}
\definemathaccent hat       {\buildmathaccent\mathhat}
\definemathaccent vec       {\buildmathaccent\mathvec}
\definemathaccent dot       {\buildmathaccent\mathdot}
\definemathaccent widetilde {\buildmathaccent\mathwidetilde}
\definemathaccent widehat   {\buildmathaccent\mathwidehat}

%D Some precautions:

\ifx\usepdffontresource\undefined
  \def\usepdffontresource #1 {} % this will be defined elsewhere 
\fi

%D Some day \unknown\ 

% \def\useencodingvector #1 % file tag 
%   {\pushmacro\definecharacter
%    \pushmacro\startencoding
%    \pushmacro\stopencoding
%    \def\definecharacter ##1 ##2 % 
%      {\doifnumberelse{##2}
%         {\ifnum##2>127
%            \def\!!stringa{##2 }%
%            \@EA\@EA\@EA\defineactivetoken\@EA\!!stringa\@EA{\csname##1\endcsname}%
%          \fi}
%         {}}%
%    \def\startencoding[##1]{}
%    \def\stopencoding{\endinput}  
%    \readfile{xxxx-#1}{}{}% 
%    \popmacro\stopencoding
%    \popmacro\startencoding
%    \popmacro\definecharacter}
% 
% \startregime[ec]
%   \useencodingvector ec
% \stopregime

%D We preload several encodings: 

%\input enco-def \input enco-acc \input enco-raw
%\input enco-com \input enco-cas \input enco-mis

\useencoding[def,acc,raw,com,cas,mis]

\useencoding[ans,il2,ec,pdf,uc,pol,x5]

\protect \endinput 
