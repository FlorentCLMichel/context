%D \module
%D   [       file=spec-dpx,
%D        version=2002.11.28,
%D          title=\CONTEXT\ Special Macros,
%D       subtitle=DVIPDFMx support,
%D         author=Jin-Hwan Cho,
%D           date=\currentdate,
%D      copyright={Jin-Hwan Cho \& Hans Hagen}]
%C
%C DVIPDFMx is an eXtended version of the DVIPDFM, a DVI to PDF
%C translator, still under construction by Jin-Hwan Cho and
%C Shunsaku Hirata.
%C
%C It supports multi-byte character encodings and large character
%C sets for East Asian languages by CID-keyed font technology,
%C and many features including PDF encryption, PDF bookmarks and
%C annotations with Chinese, Japanese, Korean characters, etc.
%C
%C DVIPDFMx (and information) can be downloaded from:
%C
%C   http://project.ktug.or.kr/dvipdfmx/

\unprotect

%D This driver is build on top of the general \PDF\ macros,
%D as defined in \type{spec-fdf}, so we inherit that one.

\startspecials[dpx][reset,fdf]

%D 1. Modified codes from \type{spec-dpm}

%D \macros
%D   {jobsuffix}
%D
%D Because no intermediate output format is used, we set the
%D jobsuffix to \PDF.

\def\jobsuffix{pdf}

%D DVIPDFMx supports the special command \type{pdf: literal ...}
%D since the version \type{dvipdfmx-20021028}.
%D
%D 1. \type{pdf: literal #1} puts the given PDF commands \#1
%D    after changing the text matrix with \type{"1 0 0 1 x y cm"}
%D    to set the current DVI position $(x,y)$ to the origin.
%D
%D 2. \type{pdf: literal reverse #1} puts the given PDF commands \#1
%D    after changing the text matrix with \type{"1 0 0 1 -x -y cm"}.
%D
%D 3. \type{pdf: literal direct #1} puts directly the given PDF
%D    commands \#1 without changing the text matrix.
%D
%D Literal \PDF\ inclusion is implemented as:

\def\PDFcode#1{\special{pdf: literal direct #1}}

%D \type{\dosetuppaper} in \type{spec-dpm} did not work properly
%D because DVIPDFM did not permit changing the page size.
%D
%D However, DVIPDFMx permits different page size in each page
%D since the version \type{dvipdfmx-20021028}.
%D
%D \macros
%D   {dosetuppaper}
%D
%D A simple one.

\definespecial\dosetuppaper#1#2#3%
  {\bgroup
   \scratchdimen#2\edef\width {\the\scratchdimen\space}%
   \scratchdimen#3\edef\height{\the\scratchdimen\space}%
   \special{pdf: pagesize width \width height \height}%
   \egroup}

%D \macros
%D   {doinsertfile}
%D
%D Only \METAPOST, \JPG\ \PNG\ and \PDF\ inclusion are supported.

\definespecial\doinsertfile#1#2#3#4#5#6#7#8#9%
  {\dodoinsertfile{dpx}{#1}{#2}{#3}{#4}{#5}{#6}{#7}{#8}{#9}}

%D Even though DVIPDFM supports \METAPOST directly, the funtionality
%D is not good. It conflicts much with \CONTEXT.
%D
%D So, \METAPOST will be treated as the same way as PDFTeX using
%D MPtoPDF in DVIPDFMx since the version \type{dvipdfmx-20021028}.

\definefileinsertion{dpx}{mps}#1#2#3#4#5#6#7#8#9%
  {\hbox
     {%\convertMPcolors{#1}% plugged in supp-mpe
      \scratchdimen#3pt \PointsToReal{.01\scratchdimen}\xscale
      \scratchdimen#4pt \PointsToReal{.01\scratchdimen}\yscale
      \special{pdf: literal direct q}\special{pdf: literal}%
      \convertMPtoPDF{#1}\xscale\yscale
      \special{pdf: literal direct Q}
      \global\let\PDFimagereference\empty}}

%D DVIPDFM (and DVIPDFMx too) supports the image files with
%D the extension, \PDF, \JPG, \PNG, and \EPS.

\def\handlepdfimage#1#2#3#4#5#6#7#8#9%
  {\bgroup
   \scratchdimen#7\edef\width {\the\scratchdimen\space}%
   \scratchdimen#8\edef\height{\the\scratchdimen\space}%
   \special{pdf: image width \width height \height (#1)}%
   \egroup}

\definefileinsertion{dpx}{pdf}{\handlepdfimage}
\definefileinsertion{dpx}{jpg}{\handlepdfimage}
\definefileinsertion{dpx}{png}{\handlepdfimage}
\definefileinsertion{dpx}{eps}{\handlepdfimage} % unstable

\definefileinsertion{dpx}{mov}{\doPDFinsertmov}
\definefileinsertion{dpx}{avi}{\doPDFinsertmov}

%D \macros
%D  {doinsertsoundtrack}
%D
%D Sounds are supported too.

\definespecial\doinsertsoundtrack{\doPDFinsertsoundtrack}

%D \type{\doPDFovalbox} in \type{spec-fdf} was modifed because
%D the definition of \type{\PDFcode} was changed in this module.
%D
%D \macros
%D  {doPDFovalbox}
%D
%D For drawing ovals we use quite raw \PDF\ code. The next
%D implementation does not differ that much from the one
%D implemented in the \POSTSCRIPT\ driver.

\def\doPDFovalbox#1#2#3#4#5#6#7% todo: \scratchdimen/\scatchbox
  {\bgroup
   \dimen0=#4\divide\dimen0 2
   \doPDFovalcalc{0pt}{+\dimen0}\xmin
   \doPDFovalcalc{#1}{-\dimen0}\xmax
   \doPDFovalcalc{#2}{-\dimen0}\ymax
   \doPDFovalcalc{-#3}{+\dimen0}\ymin
   \advance\dimen0 by #5%
   \doPDFovalcalc{0pt}{+\dimen0}\xxmin
   \doPDFovalcalc{#1}{-\dimen0}\xxmax
   \doPDFovalcalc{#2}{-\dimen0}\yymax
   \doPDFovalcalc{-#3}{+\dimen0}\yymin
   \doPDFovalcalc{#4}{\zeropoint}\stroke
   \doPDFovalcalc{#5}{\zeropoint}\radius
   \edef\dostroke{#6}%
   \edef\dofill{#7}%
   \setbox0\hbox
     {\ifnum\dostroke\dofill>0
        \ifPDFstrokecolor\else\ifnum\dostroke=1
          \writestatus\m!colors{pdf stroke color will fail}\wait
        \fi\fi
        \special{pdf: content
          \stroke\space w
          \xxmin\space \ymin\space  m
          \xxmax\space \ymin\space  l
          \xmax \space \ymin\space  \xmax\space  \yymin\space y
          \xmax \space \yymax\space l
          \xmax \space \ymax\space  \xxmax\space \ymax\space  y
          \xxmin\space \ymax\space  l
          \xmin \space \ymax\space  \xmin\space  \yymax\space y
          \xmin \space \yymin\space l
          \xmin \space \ymin\space  \xxmin\space \ymin\space  y
          \ifnum\dostroke=1 S \fi
          \ifnum\dofill=1 f \fi}%
      \fi}%
   \wd0=#1\ht0=#2\dp0=#3\box0
   \egroup}

%D \macros
%D  {doovalbox}

\definespecial\doovalbox{\doPDFovalbox}

%D \macros
%D   {dostartgraymode,dostopgraymode,
%D    dostartrgbcolormode,dostartcmykcolormode,dostartgraycolormode,
%D    dostopcolormode,
%D    dostartrotation,dostoprotation,
%D    dostartscaling,dostopscaling,
%D    dostartmirroring,dostopmirroring,
%D    dostartnegative,dostopnegative}
%D
%D Unfortunately the direct \PDF\ inclusion is not suited
%D for the next macros, which means that we cannot use the
%D \type {\doPDF..} alternatives. Since \CONTEXT\ maintains
%D its own colorstack, we use the \DVIPS\ alternatives.

\definespecial\dostartgraymode           #1{\special{color gray #1}}
\definespecial\dostopgraymode              {\special{color gray 0}}
\definespecial\dostartrgbcolormode   #1#2#3{\special{color rgb  #1 #2 #3}}
\definespecial\dostartcmykcolormode#1#2#3#4{\special{color cmyk #1 #2 #3 #4}}
\definespecial\dostartgraycolormode      #1{\special{color gray #1}}
\definespecial\dostopcolormode             {\special{color gray 0}}
\definespecial\dostartrotation           #1{\special{pdf: bt rotate #1}}
\definespecial\dostoprotation              {\special{pdf: et}}
\definespecial\dostartscaling          #1#2{\special{pdf: bt xscale #1 yscale #2}}
\definespecial\dostopscaling               {\special{pdf: et}}
\definespecial\dostartmirroring            {\special{pdf: bt xscale -1}} % ?
\definespecial\dostopmirroring             {\special{pdf: et}}

%D Negation is not (yet) supported:

% we need resource access
%
% \definespecial\dostartnegative      {}
% \definespecial\dostopnegative       {}
% \definespecial\dostarttransparency  {}
% \definespecial\dostoptransparency   {}

\definespecial\dostarttransparency{\doPDFstarttransparency}
\definespecial\dostoptransparency {\doPDFstoptransparency}

\PDFtransparencysupportedtrue

\def\@@PDT{@PDT@}

\ifx\PDFcurrenttransparency\undefined
  \newcount\PDFcurrenttransparency \PDFcurrenttransparency=0 % -1
\fi

\def\assignPDFtransparency#1#2%
  {\def\PDFtransparencyidentifier{/Tr#1}%
   %\def\PDFtransparencyreference{#2 0 R}}
   \def\PDFtransparencyreference{@TR:#2}}

\def\presetPDFtransparency#1#2%
  {\initializePDFtransparency
   \executeifdefined{\@@PDT#1:#2}{\dopresetPDFtransparency{#1}{#2}}}

\def\dopresetPDFtransparency#1#2%
  {\global\advance\PDFcurrenttransparency \plusone
   %\immediate\pdfobj{\PDFtransparancydictionary{#1}{#2}{}}%
%   \special{pdf: object @TR:\the\PDFcurrenttransparency\space \PDFtransparancydictionary{#1}{#2}{}}%
   \doPDFreserveDPXobject{TR:\the\PDFcurrenttransparency}{<< >>}%
   \special{pdf: \doPDFcheckedDPXobject{TR:\the\PDFcurrenttransparency}\PDFtransparancydictionary{#1}{#2}{}}%
   \edef\PDFtransparencyidentifier{/Tr\the\PDFcurrenttransparency}%
   %\edef\PDFtransparencyreference {\the\pdflastobj\space 0 R}%
   \edef\PDFtransparencyreference {@TR:\the\PDFcurrenttransparency}%
   \setxvalue{\@@PDT#1:#2}%
     {\noexpand\assignPDFtransparency{\the\PDFcurrenttransparency}{\the\PDFcurrenttransparency}}%
   \appendtoPDFdocumentextgstates
     {\PDFtransparencyidentifier\space
      \PDFtransparencyreference\space}}

\def\initializePDFtransparency
  %{\immediate\pdfobj{\PDFtransparancydictionary{1}{1}{/AIS false}}%
%  {\special{pdf: object @TR:0 \PDFtransparancydictionary{1}{1}{/AIS false}}%
  {\doPDFreserveDPXobject{TR:0}{<< >>}%
   \special{pdf: \doPDFcheckedDPXobject{TR:0}\PDFtransparancydictionary{1}{1}{/AIS false}}%
   \xdef\PDFtransparencyresetidentifier{/Tr0}%
   %\xdef\PDFtransparencyresetreference{\the\pdflastobj\space 0 R}%
   \xdef\PDFtransparencyresetreference{@TR:0}%
   \setxvalue{\@@PDT0:0}%
     %{\noexpand\assignPDFtransparency{0}{\the\pdflastobj}}%
     {\noexpand\assignPDFtransparency{0}{0}}%
   \appendtoPDFdocumentextgstates
     {\PDFtransparencyresetidentifier\space
      \PDFtransparencyresetreference\space}%
   \global\let\initializePDFtransparency\relax}

%D \macros
%D   {dosetupinteraction,
%D    dosetupopenaction,dosetupcloseaction}

\definespecial\dosetupinteraction
  {\showmessage\m!interactions{21}{DVIPDFMx}}

\definespecial\dosetupopenaction {\doPDFsetupopenaction}
\definespecial\dosetupcloseaction{\doPDFsetupcloseaction}

%D \macros
%D   {doresetgotowhereever,
%D    dostartthisisrealpage,dostartthisislocation,
%D    dostartgotorealpage,dostartgotolocation,dostartgotoJS}

\definespecial\doresetgotowhereever {\doPDFresetgotowhereever}
\definespecial\dostartthisislocation{\doPDFstartthisislocation}

\definespecial\dostartgotolocation{\doPDFstartgotolocation}
\definespecial\dostartgotorealpage{\doPDFstartgotorealpage}
\definespecial\dostartgotoJS      {\doPDFstartgotoJS}

%D \macros
%D   {doflushJSpreamble}

\definespecial\doflushJSpreamble#1%
  {\bgroup
   \let\compositeJScode\empty
   \def\docommando##1%
     {\edef\sanitizedJScode{\getJSpreamble{##1}}%
      \@EA\doPSsanitizeJScode\sanitizedJScode\to\sanitizedJScode
      \special{pdf: object @JS:#1 <</S /JavaScript /JS (\sanitizedJScode)>>}%
      \edef\compositeJScode{\compositeJScode\space (##1) @JS:#1}}%
   \processcommalist[#1]\docommando
   \special{pdf: object @JS:JS <</Names [\compositeJScode]>>}%
   \special{pdf: put @names <</JavaScript @JS:JS>>}%
   \egroup}

%D \macros
%D   {dostarthide,dostophide}

\definespecial\dostarthide{}
\definespecial\dostophide {}

%D \macros
%D   {dosetupscreen}

\definespecial\dosetupscreen  {\doPDFsetupscreen  \printpapierhoogte}

\definespecial\dosetupartbox  {\doPDFsetupartbox  \printpapierhoogte}
\definespecial\dosetupcropbox {\doPDFsetupcropbox \printpapierhoogte}
\definespecial\dosetupbleedbox{\doPDFsetupbleedbox\printpapierhoogte}
\definespecial\dosetuptrimbox {\doPDFsetuptrimbox \printpapierhoogte}

%D \macros
%D   {dostartexecutecommand}

\definespecial\dostartexecutecommand{\doPDFstartexecutecommand}

%D \macros
%D   {dosetupidentity}

\definespecial\dosetupidentity{\doPDFsetupidentity}

%D \macros
%D   {dostartrunprogam}

\definespecial\dostartrunprogram{\doPDFstartrunprogram}

%D \macros
%D   {dostartgotoprofile, dostopgotoprofile,
%D    dobeginofprofile, doendofprofile}

\definespecial\dostartgotoprofile{\doPDFstartgotoprofile}

\definespecial\dobeginofprofile#1#2#3#4%
  {\bgroup
   \setPDFdestination{#1}%
   \scratchdimen#2\edef\width {\the\scratchdimen\space}%
   \scratchdimen#3\edef\height{\the\scratchdimen\space}%
   \doifsomething{\PDFdestination}
     {\special
        {pdf: thread @ART::\PDFdestination\space
              width \width height \height
              <</Title (\PDFdestination)>>}}%
   \egroup}

\definespecial\doendofprofile
  {}

%D \macros
%D  {doinsertbookmark}

\definespecial\doinsertbookmark{\doPDFinsertbookmark}

%D \macros
%D  {dostartobject,dostopobject,doinsertobject}
%D
%D Modified to support the color shading feature since version
%D \type{dvipdfmx-20021128}.

% wd nextbox > #3 ivm offset

\definespecial\dostartobject#1#2#3#4#5%
  {\bgroup
   \dowithnextbox
     {\dosetobjectreference{#1}{#2}{@#1::#2}%
      \scratchdimen#3\edef\width {\the\scratchdimen\space}%
      \scratchdimen#4\edef\height{\the\scratchdimen\space}%
      \setbox\nextbox\vbox
        {\special{pdf: bxobj @#1::#2 width \width height \height}%
         % we need to compensate for the box offset (ugly, sigh)
         \scratchdimen\nextboxht
         \advance\scratchdimen\nextboxdp
         \advance\scratchdimen-#4\relax
         \nextboxdp\zeropoint
         \nextboxht\zeropoint
         \hbox to #3{\hss\lower.5\scratchdimen\box\nextbox\hss}%
         \ifx\currentPDFresources\empty
           \special{pdf: exobj}%
         \else
           \special{pdf: exobj <<\currentPDFresources\the\pdfpageresources>>}%
           \global\let\currentPDFresources\empty
         \fi}%
      \smashbox\nextbox
      \flushatshipout{\box\nextbox}%
      \egroup}%
    \hbox\bgroup}

\definespecial\dostopobject
  {\egroup}

\definespecial\doinsertobject#1#2%
  {\hbox
     {\doPDFgetobjectreference{#1}{#2}\PDFobjectreference
      \ifx\PDFobjectreference\empty \else
        \special{pdf: usexobj @#1::#2}%
      \fi}}

%D \macros
%D   {dosetpagetransition}

\definespecial\dosetpagetransition{\doPDFsetpagetransition}

%D \macros
%D   {doinsertcomment, doflushcomments}

\definespecial\doinsertcomment{\doPDFinsertcomment}
\definespecial\doflushcomments{\doPDFflushcomments}

%D \macros
%D   {dopresetlinefield,dopresettextfield,
%D    dopresetchoicefield,dopresetpopupfield,dopresetcombofield,
%D    dopresetpushfield,dopresetcheckfield,
%D    dopresetradiofield,dopresetradiorecord}

\definespecial\dopresetlinefield  {\doFDFpresetlinefield}
\definespecial\dopresettextfield  {\doFDFpresettextfield}
\definespecial\dopresetchoicefield{\doFDFpresetchoicefield}
\definespecial\dopresetpopupfield {\doFDFpresetpopupfield}
\definespecial\dopresetcombofield {\doFDFpresetcombofield}
\definespecial\dopresetpushfield  {\doFDFpresetpushfield}
\definespecial\dopresetcheckfield {\doFDFpresetcheckfield}
\definespecial\dopresetradiofield {\doFDFpresetradiofield}
\definespecial\dopresetradiorecord{\doFDFpresetradiorecord}

%D \macros
%D   {dodefinefieldset,dogetfieldset,doiffieldset}

\definespecial\dodefinefieldset{\doFDFdefinefieldset}
\definespecial\dogetfieldset   {\doFDFgetfieldset}
\definespecial\doiffieldset    {\doFDFiffieldset}

%D \macros
%D   {doregistercalculationset}

\definespecial\doregistercalculationset{\doFDFregistercalculationset}

%D \type{\doPDFdestination} in \type{spec-dpm} had a bug.
%D
%D \macros
%D   {doPDFdestination}

\def\doPDFdestination name #1%
  {\special{pdf: dest (#1) [@thispage\PDFpageviewwrd]}}

%D \macros
%D   {doPDFaction,doPDFannotation,ifsharePDFactions}
%D
%D Sharing is not yet supported.

\newif\ifsharePDFactions \sharePDFactionsfalse

\def\dodoPDFaction#1#2#3#4%
  {\ifcollectreferenceactions
     \xdef\lastPDFaction{#4}%
   \else
     \bgroup
    % this is yet untested
    %\ifsharePDFactions
    %  \ifcase\similarreference\relax
    %    \xdef\lastPDFaction{<<#4>>}%
    %  \or
    %    \global\advance\nofPDFsimilar by 1
    %    \special{pdf: object @PDF::sim:\the\nofPDFsimilar\space<<#4>>}%
    %    \xdef\lastPDFaction{@PDF::sim:\the\nofPDFsimilar}%
    %  \else
    %    % leave \lastPDFaction untouched
    %  \fi
    %\else
       \xdef\lastPDFaction{<<#4>>}%
    %\fi
     \scratchdimen#2\edef\width {\the\scratchdimen\space}%
     \scratchdimen#3\edef\height{\the\scratchdimen\space}%
     \special{pdf: ann #1 width \width height \height
       <</Subtype /Link /Border [0 0 0]
         \ifhighlighthyperlinks \else /H /N \fi
         /A \lastPDFaction\space>>}%
     \egroup
   \fi}

\def\doPDFaction width #1 height #2 action #3%
  {\dodoPDFaction\empty{#1}{#2}{#3}}

%D \type{\doPDFannotation} in \type{spec-dpm} had a bug.

\def\doPDFannotation width #1 height #2 data #3%
  {\bgroup
   \scratchdimen#1\edef\width {\the\scratchdimen\space}%
   \scratchdimen#2\edef\height{\the\scratchdimen\space}%
   \special{pdf: ann width \width height \height <<#3>>}%
   \egroup}

%D \macros
%D   {doPDFannotationobject,doPDFactionobject}

\def\doPDFannotationobject class #1 name #2 width #3 height #4 data #5%
  {\bgroup
   \scratchdimen#3\edef\width {\the\scratchdimen\space}%
   \scratchdimen#4\edef\height{\the\scratchdimen\space}%
   \special{pdf: ann @#1::#2 width \width height \height <<#5>>}%
   \dosetobjectreference{#1}{#2}{@#1::#2}%
   \egroup}

\def\doPDFactionobject class #1 name #2 width #3 height #4 action #5%
  {\dodoPDFaction{@#1::#2}{#3}{#4}{#5}%
   \dosetobjectreference{#1}{#2}{@#1::#2}}

%D \macros
%D   {doPDFaddtocatalog,doPDFaddtoinfo,
%D    doPDFpageattribute,doPDFpagesattribute}

% we could move much more to spec-fdf

% \ifx\pdfcatalog      \undefined \newtoks\pdfcatalog       \fi
% \ifx\pdfinfo         \undefined \newtoks\pdfinfo          \fi
% \ifx\pdfpageattr     \undefined \newtoks\pdfpageattr      \fi
% \ifx\pdfpageresources\undefined \newtoks\pdfpageresources \fi
% \ifx\pdfpagesattr    \undefined \newtoks\pdfpagesattr     \fi

% \def\doPDFaddtocatalog  #1{\expanded{\global\pdfcatalog      {#1\the\pdfcatalog      }}}
% \def\doPDFaddtoinfo     #1{\expanded{\global\pdfinfo         {#1\the\pdfinfo         }}}
% \def\doPDFpageattribute #1{\expanded{\global\pdfpageattr     {#1\the\pdfpageattr     }}}
% \def\doPDFpageresource  #1{\expanded{\global\pdfpageresources{#1\the\pdfpageresources}}}
% \def\doPDFpagesattribute#1{\expanded{\global\pdfpagesattr    {#1\the\pdfpagesattr    }}}

% \def\doPDFresetpageattributes{\global\pdfpageattr\emptytoks}
% \def\doPDFresetpageresources {\global\pdfpageresources\emptytoks}

% \appendtoks
%   \special{pdf: put @catalog <<#1>>}%
%   \special{pdf: docinfo      <<#1>>}%
%   \special{pdf: put @pages   <<#1>>}%
% \to \everylastshipout

% \appendtoks
%   \special{pdf: put @thispage <<#1>>}%
% \to\everyshipout

\def\doPDFaddtocatalog#1%
  {\special{pdf: put @catalog <<#1>>}}

\def\doPDFaddtoinfo#1% Is this auto appended? Not checked!
  {\special{pdf: docinfo <<#1>>}} % put @docinfo <<#1>>}}

\def\doPDFpageattribute#1%
  {\special{pdf: put @thispage <<#1>>}}

\def\doPDFpagesattribute#1%
  {\special{pdf: put @pages <<#1>>}}

\def\doPDFpageresource#1%
  {\special{pdf: put @resources <<#1>>}}

\let\doPDFresetpageresources =\relax
\let\doPDFresetpageattributes=\relax

%D \type{\doPDFbookmark} in \type{spec-dpm} had a bug.
%D The openbookmark option \#5 is not supported yet.
%D
%D \macros
%D   {doPDFbookmark}

\def\doPDFbookmark level #1 n #2 text #3 page #4 open #5%
  {\ifcase#1\else
     \special{pdf: outline #1 %\ifcase\the#5-\fi#1
                <</Title (#3) /A <</S /GoTo /D (page:#4)>>>>}%
   \fi}

%D \macros
%D   {doPDFdictionaryobject,doPDFarrayobject}

% Dvipdfmx can't handle
%
% \special{pdf:put @foo << /Bar @bar >>}
% \special{pdf:put @bar << /Foo @foo >>}
%
% Objects must be defined before they are used.
%
% \special{pdf:obj @foo << >>}
% \special{pdf:obj @bar << >>}
% \special{pdf:put @foo << /Bar @bar >>}
% \special{pdf:put @bar << /Foo @foo >>}
%
% However, this only works for dictionary and array.

\def\doPDFreserveDPXobject#1#2%
  {\ifundefined{r:pdx:d:#1}%
     % we need a \flushatshipoutprep (prepended, normally appended)
     \flushatshipout{\special{pdf: object @#1 #2}}%
     \global\letvalue{r:pdx:d:#1}\empty
   \fi}

\def\doPDFreserveDPXobjectfirst#1#2%
  {\ifundefined{r:pdx:d:#1}%
     \doglobal\prependtoks\special{pdf: object @#1 #2}\to\everyfirstshipout
     \global\letvalue{r:pdx:d:#1}\empty
   \fi}

\def\doPDFcheckedDPXobject#1{\ifundefined{r:pdx:d:#1}object\else put\fi\space @#1\space}

% todo when etex is fixed, \everyPDFpresets, leeg voor pdftex, nodig voor dvipdfmx

\doPDFreserveDPXobjectfirst{FDF::docuextgstates}{<< >>}

\def\doPDFdictionaryobject class #1 name #2 data #3%
  {\flushatshipout
     {\special{pdf: \doPDFcheckedDPXobject{#1::#2}<<#3>>}%
      \dosetobjectreference{#1}{#2}{@#1::#2}}}

\def\doPDFarrayobject class #1 name #2 data #3%
  {\flushatshipout
     {\special{pdf: \doPDFcheckedDPXobject{#1::#2}[#3]}%
      \dosetobjectreference{#1}{#2}{@#1::#2}}}

%D \macros
%D   {defaultobjectreference,doPDFgetobjectreference}

\def\defaultobjectreference#1#2{@#1::#2}

%D \type{\doPDFgetobjectreference} in \type{spec-dpm} had a bug.

\def\doPDFgetobjectreference#1#2#3%
  {\dogetobjectreference{#1}{#2}#3%
   \ifx#3\empty\else\edef#3{#3}\fi}

% \def\doPDFgetobjectpage         #1#2#3{..}
% \def\doPDFgetobjectpagereference#1#2#3{..}

\def\doPDFgetpagereference#1#2%
  {\edef#2{@page#1}}

%D Done.

% %D 2. Modified codes from \type{spec-fdf}
%
% \definespecial\dostartgraphicgroup{\special{pdf: literal direct q}}
% \definespecial\dostopgraphicgroup {\special{pdf: literal direct Q}}

%D 3. Modified codes from \type{spec-tpd}

%D \macros
%D   {dostartclipping,dostopclipping}
%D
%D Clipping in \PDFTEX\ is rather trivial. We can even hook
%D in \METAPOST\ without problems.

\definespecial\dostartclipping#1#2#3%
  {\PointsToBigPoints{#2}\width
   \PointsToBigPoints{#3}\height
   \grabMPclippath{#1}{1}\width\height
     {0 0 m \width\space 0 l \width \height l 0 \height l}%
   \special{pdf: literal direct q}%
   \special{pdf: literal 0 w \MPclippath\space W n}%
   \special{pdf: literal reverse}}

\definespecial\dostopclipping
  {\special{pdf: literal direct Q n}}

%D 4. Modified codes from \type{supp-mpe}

\def\stopMPshading
  {\global\advance\currentPDFshade \plusone
   \setxvalue{obj:Sh:\currentMPspecial}%
     {/Sh\the\currentPDFshade\space @obj:Sh:\currentMPspecial\space}%
   \setxvalue{mps:Sh:\currentMPspecial}%
     {\the\currentPDFshade}}

\defineMPspecial{30}
  {\startMPshading{14}% type 2
   \setMPshadingcolors{4}{5}{6}{9}{10}{11}%
   \special{pdf: object @ftn:Sh:\currentMPspecial\space
     <</FunctionType 2
       /Domain [\gMPs1 \gMPs2]
       /C0 [\MPshadeA]
       /C1 [\MPshadeB]
       /N \gMPs3>>}%
   \special{pdf: object @obj:Sh:\currentMPspecial\space
     <</ShadingType 2
       /ColorSpace /\MPshadeC\space
       /Function @ftn:Sh:\currentMPspecial\space
       /Coords [\gMPs7 \gMPs8 \gMPs{12} \gMPs{13}]
       /Extend [true true]>>}%
   \stopMPshading}

\defineMPspecial{31}
  {\startMPshading{16}% type 3
   \setMPshadingcolors{4}{5}{6}{10}{11}{12}%
   \special{pdf: object @ftn:Sh:\currentMPspecial\space
     <</FunctionType 2
       /Domain [\gMPs1 \gMPs2]
       /C0 [\MPshadeA]
       /C1 [\MPshadeB]
       /N \gMPs3>>}%
   \special{pdf: object @obj:Sh:\currentMPspecial\space
     <</ShadingType 3
       /ColorSpace /\MPshadeC\space
       /Function @ftn:Sh:\currentMPspecial\space
       /Coords [\gMPs7 \gMPs8 \gMPs9 \gMPs{13} \gMPs{14} \gMPs{15}]
       /Extend [true true]>>}%
   \stopMPshading}

%D 5. Modified codes from \type{supp-pdf}

\def\handleMPfshow
  {\special{pdf: literal direct q}%
   \special{pdf: literal reverse}%
   \dohandleMPfshow
   \special{pdf: literal direct Q}}

\newcounter\MPPDFcounter

\def\setMPPDFobject#1#2% resources boxnumber / see other object macros / unchecked
  {\doglobal\increment\MPPDFcounter
   \bgroup
   \setbox\nextbox\vbox
     {\scratchdimen\wd#2\edef\width {\the\scratchdimen\space}%
      \scratchdimen\ht#2\edef\height{\the\scratchdimen\space}%
      \special{pdf: bxobj @MPPDF:\MPPDFcounter\space width \width height \height}%
      \box#2%
      \special{pdf: exobj <<\currentPDFresources>>}}%
    \smashbox\nextbox
    \flushatshipout{\box\nextbox}%
    \egroup
    \edef\getMPPDFobject{\special{pdf: usexobj @MPPDF:\MPPDFcounter}}}

\let\getMPPDFobject\relax

\definespecial\doinsertMPfile#1%
  {\doiffileelse{./#1}{\includeMPasPDF{./#1}}{\message{[MP #1]}}}

%D Experimental (untested):

\definespecial\dostartfonteffect{\doPDFstartfonteffect}
\definespecial\dostopfonteffect {\doPDFstopfonteffect}

\stopspecials

\protect \endinput