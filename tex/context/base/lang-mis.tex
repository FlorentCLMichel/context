%D \module
%D   [       file=lang-mis,
%D        version=1997.03.20, % used to be supp-lan.tex
%D          title=\CONTEXT\ Language Macros,
%D       subtitle=Language Options,
%D         author=Hans Hagen,
%D           date=\currentdate,
%D      copyright={PRAGMA / Hans Hagen \& Ton Otten}]
%C
%C This module is part of the \CONTEXT\ macro||package and is
%C therefore copyrighted by \PRAGMA. See mreadme.pdf for
%C details.

\writestatus{loading}{Context Language Macros / Compounds}

%D \gdef\starttest
%D   {\blank
%D    \noindent
%D    \halign\bgroup\tt##\hskip2em&##\hskip2em&##\cr}
%D
%D \gdef\stoptest
%D   {\egroup
%D    \blank}
%D
%D \gdef\test#1%
%D   {\convertargument#1\to\ascii\ascii&\hyphenatedword{#1}&#1\cr}

\unprotect

%D One of \TEX's strong points in building paragraphs is the way
%D hyphenations are handled. Although for real good hyphenation
%D of non||english languages some extensions to the program are
%D needed, fairly good results can be reached with the standard
%D mechanisms and an additional macro, at least in Dutch.

%D \CONTEXT\ originates in the wish to typeset educational
%D materials, especially in a technical environment. In
%D production oriented environments, a lot of compound words
%D are used. Because the Dutch language poses no limits on
%D combining words, we often favor putting dashes between those
%D words, because it facilitates reading, at least for those
%D who are not that accustomed to it.
%D
%D In \TEX\ compound words, separated by a hyphen, are not
%D hyphenated at all. In spite of the multiple pass paragraph
%D typesetting this can lead to parts of words sticking into
%D the margin. The solution lays in saying \type
%D {spoelwater||terugwinunit} instead of \type
%D {spoelwater-terugwinunit}. By using a one character command
%D like \type {|}, delimited by the same character \type {|},
%D we get ourselves both a decent visualization (in \TEXEDIT\
%D and colored verbatim we color these commands yellow) and an
%D efficient way of combining words.
%D
%D The sequence \type{||} simply leads to two words connected by
%D a hyphen. Because we want to distinguish such a hyphen from
%D the one inserted when \TEX\ hyphenates a word, we use a bit
%D longer one.
%D
%D \hyphenation {spoel-wa-ter te-rug-win-unit}
%D
%D \starttest
%D \test {spoelwater||terugwinunit}
%D \stoptest
%D
%D As we already said, the \type{|} is a command. This commands
%D accepts an optional argument before it's delimiter, which is
%D also a \type{|}.
%D
%D \hyphenation {po-ly-meer che-mie}
%D
%D \starttest
%D \test {polymeer|*|chemie}
%D \stoptest
%D
%D Arguments like \type{*} are not interpreted and inserted
%D directly, in contrary to arguments like:
%D
%D \starttest
%D \test {polymeer|~|chemie}
%D \test {|(|polymeer|)|chemie}
%D \test {polymeer|(|chemie|)| }
%D \stoptest
%D
%D Although such situations seldom occur |<|we typeset thousands
%D of pages before we encountered one that forced us to enhance
%D this mechanism|>| we also have to take care of comma's.
%D
%D \hyphenation {uit-stel-len}
%D
%D \starttest
%D \test {op||, in|| en uitstellen}
%D \stoptest
%D
%D The next special case (concerning quotes) was brought to my
%D attention by Piet Tutelaers, one of the driving forces
%D behind rebuilding hyphenation patterns for the dutch
%D language.\footnote{In 1996 the spelling of the dutch
%D language has been slightly reformed which made this topic
%D actual again.} We'll also take care of this case.
%D
%D \starttest
%D \test {AOW|'|er}
%D \test {cd|'|tje}
%D \test {ex|-|PTT|'|er}
%D \test {rock|-|'n|-|roller}
%D \stoptest
%D
%D Tobias Burnus pointed out that I should also support
%D something like
%D
%D \starttest
%D \test {well|_|known}
%D \stoptest
%D
%D to stress the compoundness of hyphenated words.
%D
%D Of course we also have to take care of the special case:
%D
%D \starttest
%D \test {text||color and ||font}
%D \stoptest

%D \macros
%D   {installdiscretionaries}
%D
%D The mechanism described here is one of the older inner parts
%D of \CONTEXT. The most recent extensions concerns some
%D special cases as well as the possibility to install other
%D characters as delimiters. The prefered way of specifying
%D compound words is using \type{||}, which is installed by:
%D
%D \starttyping
%D \installdiscretionaries || -
%D \stoptyping
%D
%D Some alternative definitions are:
%D
%D \startbuffer
%D \installdiscretionaries ** -
%D \installdiscretionaries ++ -
%D \installdiscretionaries // -
%D \installdiscretionaries ~~ -
%D \stopbuffer
%D
%D \typebuffer
%D
%D after which we can say:
%D
%D \bgroup
%D \getbuffer
%D \starttest
%D \test {test**test**test}
%D \test {test++test++test}
%D \test {test//test//test}
%D \test {test~~test~~test}
%D \stoptest
%D \egroup

%D \macros
%D   {compoundhyphen,
%D    beginofsubsentence,endofsubsentence}
%D
%D Now let's go to the macros. First we define some variables.
%D In the main \CONTEXT\ modules these can be tuned by a setup
%D command. Watch the (maybe) better looking compound hyphen.

\ifx\compoundhyphen           \undefined \def\compoundhyphen{\hbox{-\kern-.25ex-}} \fi

\ifx\beginofsubsentence       \undefined \def\beginofsubsentence{\hbox{---}} \fi
\ifx\endofsubsentence         \undefined \def\endofsubsentence  {\hbox{---}} \fi

%D The last two variables are needed for subsentences
%D |<|like this one|>| which we did not yet mention.
%D
%D We want to enable breaking but at the same time don't want
%D compound characters like |-| or || to be separated from the
%D words. \TEX\ hackers will recognise the next two macro's:

\ifx\prewordbreak             \undefined \def\prewordbreak {\penalty\plustenthousand\hskip\zeropoint\relax} \fi
%ifx\postwordbreak            \undefined \def\postwordbreak{\penalty\zerocount      \prewordbreak         } \fi
\ifx\postwordbreak            \undefined \def\postwordbreak{\penalty\zerocount      \hskip\zeropoint\relax} \fi

\ifx\hspaceamount             \undefined \def\hspaceamount#1#2{\kern.16667em} \fi % language specific

%D \macros
%D   {beginofsubsentencespacing,endofsubsentencespacing}
%D
%D In the previous macros we provided two hooks which can be
%D used to support nested sub||sentences. In \CONTEXT\ these
%D hooks are used to insert a small space when needed.

\ifx\beginofsubsentencespacing\undefined \let\beginofsubsentencespacing\relax \fi
\ifx\endofsubsentencespacing  \undefined \let\endofsubsentencespacing  \relax \fi

%D The following piece of code is a torture test compound
%D hndling. The \type {\relax} before the \type {\ifmmode} is
%D needed because of the alignment scanner (in \ETEX\ this
%D problem is not present because there a protected macro is
%D not expanded. Thanks to Tobias Burnus for providing this
%D example.
%D
%D \startformula
%D   \left|f(x_n)-{1\over2}\right| =
%D      {\cases{|{1\over2}-x_n| &for $0\le x_n < {1\over2}$\cr
%D              |x_n-{1\over2}| &for ${1\over2}<x_n\le1$   \cr}}
%D \stopformula

\def\@tmd@action@{@tmd@a@}
\def\@tmd@text@  {@tmd@t@}
\def\@tmd@math@  {@tmd@m@}
\def\@tmd@both@  {@tmd@b@}

\def\installdiscretionary#1#2%
  {\setevalue{\@tmd@math@\detokenize{#1}}{\detokenize{#1}}%
   \setvalue {\@tmd@text@\detokenize{#1}}{#2}%
   \setvalue {\@tmd@both@\detokenize{#1}}{\discretionarycommand#1}%
   \scratchcounter\expandafter`\detokenize{#1}%
   \@EA\uedcatcodecommand\@EA\ctxcatcodes\@EA\scratchcounter\csname\@tmd@both@\detokenize{#1}\endcsname}

\def\handlemathmodediscretionary#1{\executeifdefined{\@tmd@math@\detokenize{#1}}\donothing}
\def\handletextmodediscretionary#1{\executeifdefined{\@tmd@text@\detokenize{#1}}\donothing}

\def\installdiscretionaries#1#2{\writestatus\m!systems{use \string \installdiscretionary}} % obsolete

\chardef\discretionarymode\plusone

\def\ignorediscretionaries
  {\chardef\discretionarymode\zerocount}

\def\discretionarycommand
  {% if direct if, we need \relax for lookahead in math mode
   \csname
     \ifcase\discretionarymode
       \strippedcsname\dononemodediscretionary
     \else\ifmmode
       \strippedcsname\domathmodediscretionary
     \else
       \strippedcsname\dotextmodediscretionary
     \fi\fi
   \endcsname}

\def\dononemodediscretionary#1%
  {\detokenize{#1}}

%D The macro \type{\checkbeforediscretionary} takes care of
%D loners like \type{||word}, while it counterpart
%D \type{\checkafterdiscretionary} is responsible for handling
%D the comma.

\newsignal\compoundbreakpoint

\newconditional\punctafterdiscretionary
\newconditional\spaceafterdiscretionary

\def\checkbeforediscretionary
  {\ifvmode\dontleavehmode\fi
   \ifhmode
     \begingroup
     \setbox\scratchbox\lastbox
     \ifzeropt\wd\scratchbox
       \let\postwordbreak\prewordbreak
     \fi
     \box\scratchbox\relax
     \endgroup
  \fi}

\def\checkafterdiscretionary
  {\setfalse\punctafterdiscretionary
   \setfalse\spaceafterdiscretionary
   \ifx\blankspace\nextnext \settrue \spaceafterdiscretionary \else
   \ifx\space     \nextnext \settrue \spaceafterdiscretionary \else
   \ifx          .\nextnext \settrue \punctafterdiscretionary \else
   \ifx          ,\nextnext \settrue \punctafterdiscretionary \else
   \ifx          :\nextnext \settrue \punctafterdiscretionary \else
   \ifx          ;\nextnext \settrue \punctafterdiscretionary \fi\fi\fi\fi\fi\fi}

\let\domathmodediscretionary\handlemathmodediscretionary

\def\dotextmodediscretionary#1%
  {\bgroup
   \let\nextnextnext\egroup
   \def\next##1#1%
     {\def\next{\activedododotextmodediscretionary#1{##1}}%
      \futurelet\nextnext\next}%
   \next}

\def\activedododotextmodediscretionary#1#2%
  {\edef\discretionarytoken{\detokenize{#2}}%
   \def\textmodediscretionary{\handletextmodediscretionary{#1}}%
   \checkafterdiscretionary
   \ifx\discretionarytoken\empty
     \ifx#1\nextnext % takes care of ||| and +++ and ......
       \ifcsname\@tmd@action@\string#1\endcsname
         \csname\@tmd@action@\string#1\endcsname
       \else\ifconditional\spaceafterdiscretionary
         \prewordbreak\hbox{\string#1}\relax
       \else\ifconditional\punctafterdiscretionary
         \prewordbreak\hbox{\string#1}\relax
       \else
         \prewordbreak\hbox{\string#1}\prewordbreak
       \fi\fi\fi
       \def\nextnextnext{\afterassignment\egroup\let\next=}%
     \else
       \checkbeforediscretionary
       % the next line has been changed (20050203)
       % \prewordbreak\hbox{\textmodediscretionary\nextnext}\allowbreak\postwordbreak
       % but an hbox blocks a possible \discretionary
       \ifcsname\@tmd@action@\endcsname
         \csname\@tmd@action@\endcsname
       \else\ifconditional\spaceafterdiscretionary
         \prewordbreak\textmodediscretionary\relax
       \else\ifconditional\punctafterdiscretionary
         \prewordbreak\textmodediscretionary\relax
       \else
         \prewordbreak\textmodediscretionary\prewordbreak
       \fi\fi\fi
     %  \prewordbreak\textmodediscretionary\nextnext\allowbreak\postwordbreak
     \fi
   \else\ifcsname\@tmd@action@\discretionarytoken\endcsname
     \csname\@tmd@action@\discretionarytoken\endcsname
   \else
     \checkbeforediscretionary
     \ifconditional\spaceafterdiscretionary
       \prewordbreak\hbox{#2}\relax
     \else\ifconditional\punctafterdiscretionary
       \prewordbreak\hbox{#2}\relax
     \else
       \prewordbreak\discretionary{\hbox{#2}}{}{\hbox{#2}}\allowbreak\postwordbreak
     \fi\fi
   \fi\fi
   \nextnextnext} % lookahead in commands

%D \macros
%D   {directdiscretionary}
%D
%D In those situations where the nature of characters is
%D less predictable, we can use the more direct approach:

\def\directdiscretionary
  {\csname
     \ifcase\discretionarymode
       \strippedcsname\dononemodediscretionary
     \else
       \strippedcsname\dodirectdiscretionary
     \fi
   \endcsname}

\def\indirectdiscretionary
  {\csname
     \ifcase\discretionarymode
       \strippedcsname\dononemodediscretionary
     \else
       \strippedcsname\doindirectdiscretionary
     \fi
   \endcsname}

\unexpanded\def\dodirectdiscretionary#1%
  {\edef\discretionarytoken{\detokenize{#1}}%
   \let\textmodediscretionary\compoundhyphen
   \executeifdefined{\@tmd@action@\discretionarytoken}{\indirectdiscretionary{#1}}}

\unexpanded\def\doindirectdiscretionary#1%
  {\prewordbreak\discretionary{\hbox{#1}}{}{\hbox{#1}}\allowbreak\postwordbreak}

\def\definetextmodediscretionary #1
  {\setvalue{\@tmd@action@\detokenize{#1}}}

% \start \hsize 1mm
% test |||test test|||, test\blank
% test test|-|, test|-| and test|-|test\blank
% test test|_|, test|_| and test|_|test\blank
% test cd|'|tje\blank
% test |(|test test|)|, test\blank
% test test test|x|, test\blank
% test|~|test
% test|^|test
% \stop

\def\hyphenliketextmodediscretionary#1#2#3#4%
  {\ifconditional\spaceafterdiscretionary
     \prewordbreak\hbox{#1}\relax
   \else\ifconditional\punctafterdiscretionary
     \prewordbreak\hbox{#1}\relax
   \else
     \prewordbreak\discretionary{#2}{#3}{#4}\postwordbreak % was prewordbreak
   \fi\fi}

\definetextmodediscretionary {}
  {\hyphenliketextmodediscretionary\compoundhyphen\compoundhyphen\empty\compoundhyphen}

\definetextmodediscretionary -
  {\hyphenliketextmodediscretionary\hyphen\hyphen\empty\hyphen}

\definetextmodediscretionary ~
  {\prewordbreak\discretionary{-}{}{\thinspace}\postwordbreak}

\definetextmodediscretionary _
  {\hyphenliketextmodediscretionary\compoundhyphen\compoundhyphen\compoundhyphen\compoundhyphen}

\definetextmodediscretionary (
  {\ifdim\lastskip>\zeropoint
     (\prewordbreak
   \else
     \prewordbreak\discretionary{}{(-}{(}\prewordbreak
   \fi}

\definetextmodediscretionary )
  {\hyphenliketextmodediscretionary{)}{-)}{}{)}}

\definetextmodediscretionary '
  {\prewordbreak\discretionary{-}{}{'}\postwordbreak}

\definetextmodediscretionary ^
  {\prewordbreak\discretionary{\hbox{$|$}}{}{\hbox{$|$}}%
   \allowbreak\postwordbreak} % bugged

\definetextmodediscretionary <
  {\beginofsubsentence\prewordbreak\beginofsubsentencespacing}

\definetextmodediscretionary >
  {\endofsubsentencespacing\prewordbreak\endofsubsentence}

\definetextmodediscretionary =
  {\prewordbreak\midsentence\prewordbreak} % {\prewordbreak\compoundhyphen}

% french

\definetextmodediscretionary : {\removeunwantedspaces\prewordbreak\kern\hspaceamount\empty{:}:}
\definetextmodediscretionary ; {\removeunwantedspaces\prewordbreak\kern\hspaceamount\empty{;};}
\definetextmodediscretionary ? {\removeunwantedspaces\prewordbreak\kern\hspaceamount\empty{?}?}
\definetextmodediscretionary ! {\removeunwantedspaces\prewordbreak\kern\hspaceamount\empty{!}!}

\definetextmodediscretionary *
  {\prewordbreak\discretionary{-}{}{\kern.05em}\prewordbreak}

% spanish

\definetextmodediscretionary ?? {\prewordbreak\questiondown}
\definetextmodediscretionary !! {\prewordbreak\exclamdown}

% \ifx\normalcompound\undefined \let\normalcompound=| \fi

%D \installdiscretionary  | +
%D \installdiscretionary + =

\def\defaultdiscretionaryhyphen{\compoundhyphen}

\installdiscretionary | \defaultdiscretionaryhyphen % installs in ctx and prt will fall back on it

%D \macros
%D   {fakecompoundhyphen}
%D
%D In headers and footers as well as in active pieces of text
%D we need a dirty hack. Try to imagine what is needed to
%D savely break the next text across a line and at the same
%D time make the words interactive.
%D
%D \starttyping
%D \goto{Some||Long||Word}
%D \stoptyping

\def\fakecompoundhyphen
  {\def\|{\mathortext\vert\dofakecompoundhyphen}}

\def\dofakecompoundhyphen
  {\def##1|%
     {\doifelsenothing{##1}\compoundhyphen{##1}%
      \kern\compoundbreakpoint\allowbreak}}

%D \macros
%D   {midworddiscretionary}
%D
%D If needed, one can add a discretionary hyphen using \type
%D {\midworddiscretionary}. This macro does the same as
%D \PLAIN\ \TEX's \type {\-}, but, like the ones implemented
%D earlier, this one also looks ahead for spaces and grouping
%D tokens.

\def\midworddiscretionary
  {\futurelet\next\domidworddiscretionary}

\def\domidworddiscretionary
  {\ifx\next\blankspace\else
   \ifx\next\bgroup    \else
   \ifx\next\egroup    \else
     \discretionary{-}{}{}%
   \fi\fi\fi}

%D \macros
%D   {installcompoundcharacter}
%D
%D When Tobias Burnus started translating the dutch manual of
%D \PPCHTEX\ into german, he suggested to let \CONTEXT\ support
%D the \type{german.sty} method of handling compound
%D characters, especially the umlaut. This package is meant for
%D use with \PLAIN\ \TEX\ as well as \LATEX.
%D
%D I decided to implement compound character support as
%D versatile as possible. As a result one can define his own
%D compound character support, like:
%D
%D \starttyping
%D \installcompoundcharacter "a {\"a}
%D \installcompoundcharacter "e {\"e}
%D \installcompoundcharacter "i {\"i}
%D \installcompoundcharacter "u {\"u}
%D \installcompoundcharacter "o {\"o}
%D \installcompoundcharacter "s {\SS}
%D \stoptyping
%D
%D or even
%D
%D \starttyping
%D \installcompoundcharacter "ck {\discretionary {k-}{k}{ck}}
%D \installcompoundcharacter "ff {\discretionary{ff-}{f}{ff}}
%D \stoptyping
%D
%D The support is not limited to alphabetic characters, so the
%D next definition is also valid.
%D
%D \starttyping
%D \installcompoundcharacter ". {.\doifnextcharelse{\spacetoken}{}{\kern.125em}}
%D \stoptyping
%D
%D The implementation looks familiar and uses the same tricks as
%D mentioned earlier in this module. We take care of two
%D arguments, which complicates things a bit.

\def\@nc@{@nc@} % normal character
\def\@cc@{@cc@} % compound character
\def\@cs@{@cs@} % compound characters
\def\@cx@{@cx@} % compound definition

%D When we started working on MK IV code, we needed a different
%D approach for defining the active character itself. In MK II as
%D well as in MK IV we now use the catcode vectors.

\chardef\compoundcharactermode\plusone

\def\installcompoundcharacter #1#2#3 #4% {#4} no grouping
  {\ifcase\compoundcharactermode
      % ignore mode
   \else
     \chardef\thecompoundcharacter`#1%
     \@EA\chardef\csname\@nc@\string#1\endcsname\thecompoundcharacter
     \def\!!stringa{#3}%
     \@EA\def\csname\ifx\!!stringa\empty\@cc@\else\@cs@\fi\detokenize{#1#2#3}\endcsname{#4}%
     \setevalue{\@cx@\detokenize{#1}}{\noexpand\handlecompoundcharacter{\detokenize{#1}}}% beter nr's
%      \@EA\letcatcodecommand\@EA\prtcatcodes\@EA\thecompoundcharacter\csname\@cx@\detokenize{#1}\endcsname
%      \@EA\letcatcodecommand\@EA\texcatcodes\@EA\thecompoundcharacter\csname\@cx@\detokenize{#1}\endcsname
     \@EA\letcatcodecommand\@EA\ctxcatcodes\@EA\thecompoundcharacter\csname\@cx@\detokenize{#1}\endcsname
   \fi}

%D In order to serve the language specific well, we will introduce
%D a namespace:

% \ifx\currentlanguage\undefined
  \let\compoundcharacterclass\empty
% \else
%   \def\compoundcharacterclass{\currentlanguage}
% \fi

\def\@cc@{@cc@\compoundcharacterclass} % compound character
\def\@cs@{@cs@\compoundcharacterclass} % compound characters

%D We can also ignore definitions (needed in for instance \XML). Beware,
%D this macro is supposed to be used grouped!

\def\ignorecompoundcharacter
  {\chardef\compoundcharactermode\zerocount}

\let\restorecompoundcharacter   \gobbleoneargument % obsolete
\let\enableactivediscretionaries\relax             % obsolete

%D In handling the compound characters we have to take care of
%D \type{\bgroup} and \type{\egroup} tokens, so we end up with
%D a multi||step interpretation macro. We look ahead for a
%D \type{\bgroup}, \type{\egroup} or \type{\blankspace}. Being
%D no user of this mechanism, the credits for testing them goes
%D to Tobias Burnus, the first german user of \CONTEXT.
%D
%D We define these macros as \type{\long} because we can
%D expect \type{\par} tokens. We need to look into the future
%D with \type{\futurelet} to prevent spaces from
%D disappearing.

\def\handlecompoundcharacter#1%
  {\def\xhandlecompoundcharacter{\dohandlecompoundcharacter{#1}}%
   \futurelet\next\xhandlecompoundcharacter}

\def\dohandlecompoundcharacter
  {\ifx\next\bgroup
    %\@EA\dodohandlecompoundcharacter % handle "{ee} -> \"ee
    %\@EA\gobbleoneargument           % forget "{ee} -> ee
     \@EA\handlecompoundcharacterone  % ignore "{ee} -> "ee
   \else\ifx\next\egroup
     \@EAEAEA\donohandlecompoundcharacter
   \else\ifx\next\blankspace
     \@EA\@EAEAEA\@EA\donohandlecompoundcharacter
   \else
     \@EA\@EAEAEA\@EA\dodohandlecompoundcharacter
   \fi\fi\fi}

\def\donohandlecompoundcharacter#1{\csname\@nc@\string#1\endcsname}

\def\dododohandlecompoundcharacter
  {\ifx\next\bgroup
     \@EA\handlecompoundcharacterone
   \else\ifx\next\egroup
     \@EAEAEA\handlecompoundcharacterone
   \else\ifx\next\blankspace
     \@EA\@EAEAEA\@EA\handlecompoundcharacterone
   \else
     \@EA\@EAEAEA\@EA\handlecompoundcharactertwo
   \fi\fi\fi}

\def\dodohandlecompoundcharacter#1#2% preserve space
  {\def\xdodohandlecompoundcharacter{\dododohandlecompoundcharacter#1#2}%
   \futurelet\next\xdodohandlecompoundcharacter}

%D Besides taken care of the grouping and space tokens, we have
%D to deal with three situations. First we look if the next
%D character equals the first one, if so, then we just insert
%D the original. Next we look if indeed a compound character is
%D defined. We either execute the compound character or just
%D insert the first. So we have
%D
%D \starttyping
%D <key><key>  <key><known>  <key><unknown>
%D \stoptyping
%D
%D In later modules we will see how these commands are used.

\long\def\handlecompoundcharacterone#1#2%
  {\if\string#1\string#2% was: \ifx#1#2%
     \def\next{\csname\@nc@\string#1\endcsname}%
   \else\ifcsname\@cc@\string#1\string#2\endcsname
     \def\next{\csname\@cc@\string#1\string#2\endcsname}%
   \else
     \def\next{\csname\@nc@\string#1\endcsname#2}%
   \fi\fi
   \next}

\long\def\handlecompoundcharactertwo#1#2#3%
  {\if\string#1\string#2%
     \def\next{\csname\@nc@\string#1\endcsname#3}%
   \else\ifcsname\@cs@\string#1\string#2\string#3\endcsname
     \def\next{\csname\@cs@\string#1\string#2\string#3\endcsname}%
   \else\ifcsname\@cc@\string#1\string#2\endcsname
     \def\next{\csname\@cc@\string#1\string#2\endcsname#3}%
   \else
     \def\next{\csname\@nc@\string#1\endcsname#2#3}%
   \fi\fi\fi
   \next}

%D For very obscure applications (see for an application \type
%D {lang-sla.tex}) we provide:

\def\simplifiedcompoundcharacter#1#2%
  {\ifcsname\@cc@\string#1\string#2\endcsname
     \@EA\@EA\@EA\firstofoneargument\csname\@cc@\string#1\string#2\endcsname
   \else
     #2%
   \fi}

%D \macros
%D   {disablediscretionaries,disablecompoundcharacter}
%D
%D Occasionally we need to disable this mechanism. For the
%D moment we assume that \type {|} is used.

\let\disablediscretionaries   \ignorediscretionaries
\let\disablecompoundcharacters\ignorecompoundcharacter

%D \macros
%D   {normalcompound}
%D
%D Handy in for instance XML. (Kind of obsolete)

\ifx\normalcompound\undefined \let\normalcompound=| \fi

%D ! ! This will be reimplemented !!

%D \macros
%D   {hyphenatedurl}
%D
%D For those who want to put full \URL's in a text, we offer
%D
%D \startbuffer
%D \hyphenatedurl{http://optimist.optimist/optimist/optimist.optimist#optimist}
%D \stopbuffer
%D
%D \typebuffer
%D
%D which breaks at the appropriate places. Watch the \type{#}
%D hack.
%D
%D When passed as argument, like in \type {\goto}, one needs
%D to substitute a \type {\\} for each \type{#}.
%D
%D \startbuffer
%D \hyphenatedurl{http://this.is.a.rather/strange/reference#indeed}
%D \stopbuffer
%D
%D \typebuffer

\ifx\\\undefined \let\\\crlf \fi

\chardef\urlsplitmode=1

% 0 => don't split
% 1 => . : na, rest voor
% 2 => alles na
% 3 => alles voor

% \bgroup \catcode`\~=\active \catcode`\/=\active

% Why not convert to ascii first? I will redo this one!

\bgroup \catcode`\~=\active \catcode`\/=\active

\unexpanded\gdef\hyphenatedurl#1% {}{} handles accents
  {\bgroup
   \ifnum\hyphenpenalty<10000 \else
     \def\discretionary##1##2##3{##1\allowbreak##2}%
   \fi
   \obeyhyphens
   \def\splitbefore##1%
     {\setbox\scratchbox=\hbox{##1{}{}}%
      \ifcase\urlsplitmode
        \box\scratchbox
      \or
        \postwordbreak\box\scratchbox\prewordbreak
      \or
        \prewordbreak\discretionary{\box\scratchbox}{}{\box\scratchbox}\prewordbreak
      \else
        \postwordbreak\box\scratchbox\prewordbreak
      \fi}%
   \def\splitafter##1%
     {\ifcase\urlsplitmode
        ##1{}{}%
      \or
        \prewordbreak\discretionary{##1{}{}}{}{##1{}{}}\prewordbreak
      \or
        \prewordbreak\discretionary{##1{}{}}{}{##1{}{}}\prewordbreak
      \else
        \prewordbreak\discretionary{}{##1{}{}}{##1{}{}}\prewordbreak
      \fi}%
   \def\splitanyway##1%
     {\prewordbreak##1\prewordbreak}%
   \def\flushurl%
     {\savedurl\let\savedurl\empty}%
   \def\\%
     {\spliturl\#}%
   \let\~=\lettertilde\let~=\~%
   \let\/=\letterslash\let/=\/%
   \let\savedurl\empty
   \scratchcounter\zerocount % used for hyphenmethod
   \handletokens#1\with\scanurl\savedurl
   \egroup}

\egroup

%D This would be better, but it spoils \type {\~} and so:
%D
%D \starttyping
%D \convertargument#1\to\ascii
%D \expandafter\handletokens\ascii\with\scanurl
%D \stoptyping

\chardef\urlhyphenmethod=0

\def\scanurl#1%
  {\advance\scratchcounter\plusone
   \ifx#1\blankspace
     \flushurl\splitanyway\normalspace
   \else\ifx#1\ %
     \flushurl\splitanyway\normalspace
   \else\ifx#1\space
     \flushurl\splitanyway\normalspace
   \else\ifx#1\~%
     \flushurl\splitbefore\~%
   \else\ifx#1\#%
     \flushurl\splitbefore\#%
   \else\ifx#1\&%
     \flushurl\splitbefore\&%
   \else\ifx#1\%%
     \flushurl\splitbefore\%%
   \else\ifx#1\_%
     \flushurl\splitbefore\_%
   \else\if\noexpand#1\relax
     #1%
   \else\ifnum\catcode`#1=8
     \flushurl\splitbefore\_%
   \else\ifnum\catcode`#1=6
     \flushurl\splitbefore\#%
   \else\ifnum\catcode`#1=4
     \flushurl\splitbefore\&%
   \else\expandafter\if\string#1\lettertilde
     \flushurl\splitbefore\~%
   \else\expandafter\if\string#1\letterpercent
     \flushurl\splitbefore\%%
   \else\expandafter\if\string#1\letterunderscore
     \flushurl\splitbefore\_%
   \else\expandafter\if\string#1\letterquestionmark
     \flushurl\splitafter\letterquestionmark
   \else\expandafter\if\string#1\letterat
     \flushurl\splitafter\letterat
   \else\expandafter\if\string#1\letterslash
     \edef\savedurl{\savedurl\letterslash}%
   \else\expandafter\if\string#1+%
     \flushurl\splitafter+%
   \else\expandafter\if\string#1:%
     \flushurl\splitafter:%
   \else\expandafter\if\string#1.%
     \flushurl\splitafter.%
   \else\expandafter\if\string#1(%
     \flushurl\splitbefore(%
   \else\expandafter\if\string#1)%
     \flushurl\splitafter)%
   \else
     \ifx\savedurl\empty\else
       \splitbefore\savedurl
       \let\savedurl\empty
     \fi
     \ifcase\urlhyphenmethod
       \string#1%
     \else
       \ifnum\scratchcounter>\plusthree % so, \http: will not break
         \edef\savedurl{\string#1}%
       \else
         \string#1%
       \fi
     \fi
   \fi\fi\fi\fi\fi\fi\fi\fi\fi\fi\fi\fi\fi\fi\fi\fi\fi\fi\fi\fi\fi\fi\fi}

% \setupinteraction[state=start]
% \def\gotoURL#1{\useURL[foo][#1]\goto{\url[foo]}[url(foo)]}
% \starttext
% \endgraf \chardef\urlhyphenmethod=0
% \hsize1pt\gotoURL{http://www.physik.fu-berlin.de/SomeVeryVeryVeryLongDirectory/And/AQuiteLongFileName.html}
% \endgraf \chardef\urlhyphenmethod=1
% \hsize1pt\gotoURL{http://www.physik.fu-berlin.de/SomeVeryVeryVeryLongDirectory/And/AQuiteLongFileName.html}
% \stoptext

% \useencoding[ffr]
% \mainlanguage[fr]
% \starttext
% \hyphenatedurl{http://somewhere.to/go}
% \stoptext

%D When Joop Susan asked (on the \CONTEXT\ mailing list) how
%D to handle url's passed as argument, the following solutions
%D came to my mind:
%D
%D \starttyping
%D \def\whateverurl#1%
%D   {{\def~{\string~}\useURL[dummy][#1]\goto{\url[dummy]}[URL(dummy)]}}
%D
%D \def\whateverurl#1%
%D   {{\let~\lettertilde\useURL[dummy][#1]\goto{\url[dummy]}[URL(dummy)]}}
%D
%D \def\whateverurl#1%
%D   {\convertargument#1\to\ascii
%D    \expanded{\useURL[dummy][\ascii]}\goto{\url[dummy]}[URL(dummy)]}
%D \stoptyping

%D \macros
%D   {hyphenatedfile}
%D
%D For the moment we treat filenames in a similar way,
%D
%D \starttyping
%D \hyphenatedfile{here/there/filename.suffix}
%D \stoptyping

\let\hyphenatedfile\hyphenatedurl

% to be finished
%
% \def\hyphenatedstring#1%
%   {\bgroup
%    \nohyphens
%    \def\next##1{##1\doif{##1}{-}{\allowbreak}}%
%    \handletokens#1\with\next
%    \egroup}
%
% {\hsize1cm\hyphenatedstring{ABXXXXXXXXXXC-12345-12345}}

\protect \endinput
