%D \module
%D   [       file=luat-ini,
%D        version=2005.08.11,
%D          title=\CONTEXT\ Lua Macros,
%D       subtitle=Initialization,
%D         author=Hans Hagen,
%D           date=\currentdate,
%D      copyright=PRAGMA]
%C
%C This module is part of the \CONTEXT\ macro||package and is
%C therefore copyrighted by \PRAGMA. See mreadme.pdf for
%C details.

\writestatus{loading}{ConTeXt Lua Macros / Initialization}

\unprotect

%D Loading lua code can be done using \type {startup.lua}. The following
%D method uses the \TEX\ input file locator of kpse. At least we need to
%D use that way of loading when we haven't yet define our own code, which
%D we keep outside the format. We will keep code outside \TEX\ files as
%D much as possible.

\ifx\setnaturalcatcodes\undefined \let\setnaturalcatcodes\relax \fi
\ifx\obeylualines      \undefined \let\obeylualines      \relax \fi
\ifx\obeyluatokens     \undefined \let\obeyluatokens     \relax \fi

%D A few more goodies:

\long\def\dostartlua
  {\begingroup
   \obeylualines
   \dodostartlua}

\long\def\dodostartlua#1\stoplua
  {\normalexpanded{\endgroup\noexpand\directlua\zerocount{#1}}}

\long\def\dostartluacode
  {\begingroup
   \obeylualines
   \obeyluatokens
   \dodostartluacode}

\long\def\dodostartluacode#1\stopluacode
  {\normalexpanded{\endgroup\noexpand\directlua\zerocount{#1}}}

\def\startlua    {\dostartlua    } % tex catcodes
\def\startluacode{\dostartluacode} % lua catcodes

%D Some delayed definitions:

\ifx\obeylines        \undefined \let\obeylines        \relax \fi
\ifx\obeyedline       \undefined \let\obeyedline       \relax \fi
\ifx\obeyspaces       \undefined \let\obeyspaces       \relax \fi
\ifx\obeyedspace      \undefined \let\obeyedspace      \relax \fi
\ifx\outputnewlinechar\undefined \let\outputnewlinechar\relax \fi

%D A previous version used a bit less code and no catcode table,
%D simply becaus ethey were not around at the time of writing.
%
% we keep it around for archival purposes
%
% \def\obeylualines
%   {\obeylines  \let\obeyedline \outputnewlinechar
%    \obeyspaces \let\obeyedspace\space}
%
% \def\obeyluatokens % todo: make this a proper catcode table, use let's
%   {\catcode`\%=12 \catcode`\#=12
%    \catcode`\_=12 \catcode`\^=12
%    \catcode`\&=12 \catcode`\|=12
%    \catcode`\{=12 \catcode`\}=12
%    \catcode`\~=12 \catcode`\$=12
%    \def\\{\string\\}\def\|{\string\|}\def\-{\string\-}%
%    \def\({\string\(}\def\){\string\)}\def\{{\string\{}\def\}{\string\}}%
%    \def\'{\string\'}\def\"{\string\"}%
%    \def\n{\string\n}\def\r{\string\r}\def\f{\string\f}\def\t{\string\t}%
%    \def\a{\string\a}\def\b{\string\b}\def\v{\string\v}\def\s{\string\s}%
%    \def\1{\string\1}\def\2{\string\2}\def\3{\string\3}\def\4{\string\4}\def\5{\string\5}%
%    \def\6{\string\6}\def\7{\string\7}\def\8{\string\8}\def\9{\string\9}\def\0{\string\0}}

\let\obeylualines\relax

\newtoks\everyluacode

\edef\lualetterbackslash{\string\\}
\edef\lualetterbar      {\string\|} \edef\lualetterdash     {\string\-}
\edef\lualetterlparent  {\string\(} \edef\lualetterrparent  {\string\)}
\edef\lualetterlbrace   {\string\{} \edef\lualetterrbrace   {\string\}}
\edef\lualettersquote   {\string\'} \edef\lualetterdquote   {\string\"}
\edef\lualettern        {\string\n} \edef\lualetterr        {\string\r}
\edef\lualetterf        {\string\f} \edef\lualettert        {\string\t}
\edef\lualettera        {\string\a} \edef\lualetterb        {\string\b}
\edef\lualetterv        {\string\v} \edef\lualetters        {\string\s}
\edef\lualetterone      {\string\1} \edef\lualettertwo      {\string\2}
\edef\lualetterthree    {\string\3} \edef\lualetterfour     {\string\4}
\edef\lualetterfive     {\string\5} \edef\lualettersix      {\string\6}
\edef\lualetterseven    {\string\7} \edef\lualettereight    {\string\8}
\edef\lualetternine     {\string\9} \edef\lualetterzero     {\string\0}

\appendtoks
   \let\\\lualetterbackslash
   \let\|\lualetterbar       \let\-\lualetterdash
   \let\(\lualetterlparent   \let\)\lualetterrparent
   \let\{\lualetterlbrace    \let\}\lualetterrbrace
   \let\'\lualettersquote    \let\"\lualetterdquote
   \let\n\lualettern         \let\r\lualetterr
   \let\f\lualetterf         \let\t\lualettert
   \let\a\lualettera         \let\b\lualetterb
   \let\v\lualetterv         \let\s\lualetters
   \let\1\lualetterone       \let\2\lualettertwo
   \let\3\lualetterthree     \let\4\lualetterfour
   \let\5\lualetterfive      \let\6\lualettersix
   \let\7\lualetterseven     \let\8\lualettereight
   \let\9\lualetternine      \let\0\lualetterzero
\to \everyluacode

\def\obeyluatokens
  {\setcatcodetable \luacatcodes
   \the\everyluacode}

%D \macros
%D   {definenamedlua}
%D
%D We provide an interface for defining instances:

\def\s!lua{lua} \def\v!code{code} \def\!!name{name} \def\s!data{data}

%D Beware: because \type {\expanded} is een convert command, the error
%D message will show \type{<inserted text>} as part of the message.

\long\def\dostartnamedluacode#1%
  {\begingroup
   \obeylualines
   \obeyluatokens
   \csname dodostartnamed#1\v!code\endcsname}

\ifdefined\closelua

    \def\definenamedlua[#1]#2[#3]% no optional arg handling here yet
      {\expanded{\long\def\csname dodostartnamed#1\v!code\endcsname####1\csname\e!stop#1\v!code\endcsname}%
         {\normalexpanded{\endgroup\noexpand\directlua\!!name{#3}\zerocount{protect("#1\s!data")##1}}}%
       \long\expandafter\def\csname\e!start#1\v!code\endcsname   {\dostartnamedluacode{#1}}%
       \long\expandafter\def\csname        #1\v!code\endcsname##1{\directlua\!!name{#3}\zerocount{protect("#1\s!data")##1}}}

\else

    \def\definenamedlua[#1]#2[#3]% no optional arg handling here yet
      {\scratchcounter\ctxlua{lua.registername("#1","#3")}%
       \expanded{\long\edef\csname dodostartnamed#1\v!code\endcsname####1\csname\e!stop#1\v!code\endcsname}%
         {\endgroup\noexpand\directlua\the\scratchcounter{protect("#1\s!data")##1}}%
       \long\expandafter\def \csname\e!start#1\v!code\endcsname   {\dostartnamedluacode{#1}}%
       \long\expandafter\edef\csname        #1\v!code\endcsname##1{\noexpand\directlua\the\scratchcounter{protect("#1\s!data")##1}}}

\fi

%D We predefine a few.

\definenamedlua[user]    [private user instance]
\definenamedlua[third]   [third party module instance]
\definenamedlua[module]  [module instance]
\definenamedlua[isolated][isolated instance]

%D In practice this works out as follows:
%D
%D \startbuffer
%D \startluacode
%D tex.print("LUA")
%D \stopluacode
%D
%D \startusercode
%D global.tex.print("USER 1")
%D tex.print("USER 2")
%D if characters then
%D     tex.print("ACCESS")
%D else
%D     tex.print("NO ACCESS")
%D end
%D \stopusercode
%D \stopbuffer
%D
%D \typebuffer \getbuffer

%D We need a way to pass strings safely to \LUA\ without the
%D need for tricky escaping. Compare:
%D
%D \starttyping
%D \ctxlua {something("anything tricky can go here")}
%D \ctxlua {something([\luastringsep[anything tricky can go here]\luastringsep])}
%D \stoptyping

\def\luastringsep{===} % this permits \typefile{self} otherwise nested b/e sep problems

\edef\!!bs{[\luastringsep[}
\edef\!!es{]\luastringsep]}

%D We have a the following available as primitive so there is no need
%D for it:
%D
%D \starttyping
%D \long\edef\luaescapestring#1{\!!bs#1\!!es}
%D \stoptyping

\def\setdocumentfilename       #1#2{\ctxlua{document.setfilename(#1,"#2")}}
\def\setdocumentargument       #1#2{\ctxlua{document.setargument("#1","#2")}}
\def\setdefaultdocumentargument#1#2{\ctxlua{document.getargument("#1","#2")}}
\def\getdocumentfilename         #1{\ctxlua{document.getfilename(#1)}}
\def\getdocumentargument         #1{\ctxlua{document.getargument(#1)}}
\def\doifdocumentargumentelse    #1{\doifsomethingelse{\getdocumentargument{#1}}}
\def\doifdocumentargument        #1{\doifsomething    {\getdocumentargument{#1}}}
\def\doifnotdocumentargument     #1{\doifnothing      {\getdocumentargument{#1}}}

\let\doifelsedocumentargument\doifdocumentargumentelse

%D A handy helper:

\def\luaexpanded#1{\luaescapestring\expandafter{\normalexpanded{#1}}}

\protect \endinput
