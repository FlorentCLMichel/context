%D \module
%D   [       file=supp-mrk,
%D        version=1995.10.10,
%D          title=\CONTEXT\ Support Macros,
%D       subtitle=Marks,
%D         author=Jim Fox / Hans Hagen,
%D           date=\currentdate,
%D      copyright={PRAGMA / Hans Hagen \& Ton Otten}]
%C
%C This module is part of the \CONTEXT\ macro||package and is
%C therefore copyrighted by \PRAGMA. Non||commercial use is 
%C granted. 

%D There are 256 \COUNTERS, \DIMENSIONS, \SKIPS, \MUSKIPS\ and
%D \BOXES, 16~in- and output buffers, but there is only one
%D \MARK. In TugBoat~8 (1987, no~1) Jim Fox presents a set of
%D macros that can be used to mimmick multiple marks. We
%D gladly adopt them here.

\writestatus{loading}{Context Support Macros / Marks}

\unprotect

%D This implementation is more or less compatible with the
%D other register macros in \PLAIN\ \TEX. A mark is defined by:
%D
%D \starttypen
%D \newmark\name
%D \stoptypen
%D
%D and can be called upon with:
%D
%D \starttypen
%D \topname
%D \botname
%D \firstname
%D \stoptypen
%D
%D The only drawback of his approach is that the marks must be
%D preloaded in the output routine. This is accomplished by
%D means of:
%D
%D \starttypen
%D \getmarks\name
%D \stoptypen
%D
%D The macros presented here are in most aspects copies of
%D those presented by Jim Fox. We've taken the freedom to
%D change a few things for more or less obvious reasons:
%D
%D \startopsomming
%D \som  Because the original macros look quite complicated,
%D       which is mainly due to extensive use of
%D       \type{\expandafter}'s and \type{\csname}'s, we changed
%D       those in favor of \type{\getvalue}.
%D \som  To be more in line with the rest of \CONTEXT, we've
%D       changed some of the names of macros.
%D \som  Because we are already short on \COUNTERS\ we use
%D       macros when possible.
%D \som  We maintain a list of defined marks and use one
%D       call for getting them all at once.
%D \som  We have extended the mechanism to splitmarks (not 
%D       perfected yet).
%D \som  We've introduced optional expansion of the contents
%D       of marks.
%D \stopopsomming
%D
%D Whatever changes we've made, the credits still go to Jim,
%D whatever goes wrong is due to me. The method is described 
%D in the TugBoat mentioned before, so we won't go into 
%D details. All marks belonging to a group are packed in a 
%D list. In this list they are preceded by a macro that can 
%D be defined at will and a number concerning the position at
%D which it was defined.
%D
%D \starttypen
%D \def\somelist{... \domark5{this} ... \domark31{that} ...}
%D \stoptypen
%D
%D The original \type{\mark} keeps track of the number and
%D \type{\topmark} and \type{\botmark} are used to extract the
%D actual marks from the list. The counting is done by
%D
%D \starttypen
%D \currentmarker
%D \stoptypen
%D
%D In \CONTEXT\ we use the mark mechanism to keep track of
%D colors. In a complicated documents with many colors per 
%D page, \type{\currentmarker} can therefore get pretty high.
%D (Well, this is not completely true, because we don't 
%D always have to use marks.) 

\newcount\currentmarker

%D The original implementation used a few more \COUNTERS. Two
%D have been substituted by macros, one has been replaced by
%D our scratch counter.
%D 
%D \starttypen
%D % \newcount\topmarker
%D % \newcount\botmarker
%D % \newcount\foundmarker
%D \stoptypen
%D 
%D We've also introduced some constants, one for the lists and
%D three for composing the mark commands.

\def\@@marklist@@  {marklist}
\def\@@marktop@@   {top}
\def\@@markbot@@   {bot}
\def\@@markfirst@@ {first}

%D The next one is new too. All defined marks are packed in a
%D comma seperated list. This could of course have been a token
%D list but \CONTEXT\ has some preference for comma lists.

\def\markers {}

%D \macros
%D   {expandmarks}
%D   {}
%D
%D There are two booleans. The first one handles the first
%D marks, the second concerns expansion. This second one is 
%D new.

\newif\ifnofirstmarker
\newif\ifexpandmarks     \expandmarkstrue

%D We use an indirect call to the mack mechanism. 

\let\normalmark           = \mark
\let\normaltopmark        = \topmark
\let\normalbotmark        = \botmark
\let\normalfirstmark      = \firstmark
\let\normalsplitbotmark   = \splitbotmark
\let\normalsplitfirstmark = \splitfirstmark

%D The next macro replaces the multiple step expansion and
%D command name constructors of Jim. This alternative leads to
%D a more readable source (we hope).

\def\makemarknames#1%
  {\bgroup
   \escapechar=-1
   \xdef\markname{\string#1}%
   \xdef\marklist{\@@marklist@@\string#1}%
   \egroup}

%D \macros
%D   {newmark}
%D   {}
%D
%D A mark is defined by \type{\newmark}. At the same time,
%D the name of the mark is added to a commalist. The 
%D three initializations were not in the original design, but 
%D make calls from outside the output routine a bit more 
%D robust. 

\def\newmark#1%
  {\bgroup
   \makemarknames{#1}%
   \doglobal\addtocommalist{\markname}\markers%
   \long\setgvalue{\@@marktop@@\markname}{}%
   \long\setgvalue{\@@markfirst@@\markname}{}%
   \long\setgvalue{\@@markbot@@\markname}{}%
   \setgvalue{\marklist}{\domark0{}}%
   \long\gdef#1{\addmarker#1}%
   \egroup}

%D Setting a new mark and adding a mark to the designated
%D list is done by \type{\addmarker}. This is an internal
%D command, the user set a marks bij calling it's name:
%D
%D \starttypen
%D \mymark{some text}
%D \stoptypen
%D
%D Where \type{\mymark} is previously defined by
%D \type{\newmark}.

\long\def\addmarker#1#2%
  {\bgroup
   \makemarknames{#1}%
   \global\advance\currentmarker by 1\relax
   \normalmark{\the\currentmarker}%
   \@EA\!!toksa\@EA=\@EA\@EA\@EA{\csname\marklist\endcsname}%
   \ifexpandmarks
     \setxvalue{\marklist}%
       {\the\!!toksa
        \noexpand\domark
        \the\currentmarker{#2}}%
   \else
     \!!toksb=\@EA{#2}%
     \setxvalue{\marklist}%
       {\the\!!toksa
        \noexpand\domark
        \the\currentmarker{\the\!!toksb}}%
   \fi
   \egroup}

%D \macros
%D   {getmarks,getallmarks,
%D    getsplitmarks,getallsplitmarks}
%D   {}
%D
%D In fact, marks make only sense in the output routine. Marks
%D are derived from their list by means of \type{\getmarks}. 
%D Only one call per mark is permitted in the output routine.
%D Therefore, it's far more easy to get them all at once, by 
%D means of \type{\getallmarks}, which is not part of the 
%D original design. 
%D
%D This grabbing is done by processing the list using the
%D embedded \type{\domark} macros. When a relevant mark is
%D found, this macro is reassigned and from then on serves
%D in building the new list.

\def\getmarks#1%
  {\bgroup
   \makemarknames{#1}%
   \edef\topmarker{0\normaltopmark}%
   \edef\botmarker{0\normalbotmark}%
   \!!toksb={}%
   \nofirstmarkertrue
   \let\@fi=\fi     \let\fi=\relax
   \let\@or=\or     \let\or=\relax
   \let\@else=\else \let\else=\relax
   \let\domark=\doscanmarks
   \getvalue{\marklist}\lastmark
   %\message{markstatus : [\the\!!toksa\the\!!toksb\the\!!toksc]}%
   \long\setxvalue{\marklist}{\the\!!toksa\the\!!toksb\the\!!toksc}%
   \egroup}

\def\getallmarks%
  {\processcommacommand[\markers]\getmarks}

\def\getsplitmarks#1%
  {\bgroup
   \makemarknames{#1}%
   \@EA\let\@EA\savedmarklist\@EA=\csname\marklist\endcsname
   \edef\topmarker{0\normalsplitfirstmark}%
   \edef\botmarker{0\normalsplitbotmark}%
   \!!toksb={}%
   \nofirstmarkertrue
   \let\@fi=\fi     \let\fi=\relax
   \let\@or=\or     \let\or=\relax
   \let\@else=\else \let\else=\relax
   \let\domark=\doscanmarks
   \getvalue{\marklist}\lastmark
   \@EA\global\@EA\let\csname\marklist\endcsname=\savedmarklist
   \egroup}

\def\getallsplitmarks%
  {\processcommacommand[\markers]\getsplitmarks}

\long\def\dodoscanmarks#1%
  {\ifnum\scratchcounter>\topmarker\relax
   \@else
     \long\setgvalue{\@@marktop@@\markname}{#1}%
   \@fi
   \ifnum\scratchcounter>\botmarker\relax
     \let\domark=\dorecovermarks
     \!!toksb=\@EA{\@EA\domark\the\scratchcounter{#1}}%
   \@else
     \ifnofirstmarker
       \long\setgvalue{\@@markfirst@@\markname}{#1}%
       \ifnum\scratchcounter>\topmarker\relax
         \nofirstmarkerfalse
       \@fi
     \@fi
     \long\setgvalue{\@@markbot@@\markname}{#1}%
     \!!toksa=\@EA{\@EA\domark\the\scratchcounter{#1}}%
   \@fi}

\def\doscanmarks%
  {\afterassignment\dodoscanmarks\scratchcounter=}

\long\def\dorecovermarks#1\lastmark%
  {\!!toksc={\domark#1}}

\def\lastmark%
  {\!!toksc={}}

%D No watch what happens next. Because we used an indirect 
%D call to the mark mechanism we can redefine the original 
%D \type{\mark} command. 

\newmark\mark

%D One final advice. Use marks with care. When used in globally
%D assigned boxes, the list can grow quite big, and processing
%D can slow down considerably.

\protect

\endinput
