%D \module
%D   [       file=supp-mrk,
%D        version=1995.10.10,
%D          title=\CONTEXT\ Support Macros,
%D       subtitle=Marks,
%D         author=Jim Fox / Hans Hagen,
%D           date=\currentdate,
%D      copyright={PRAGMA / Hans Hagen \& Ton Otten}]
%C
%C This module is part of the \CONTEXT\ macro||package and is
%C therefore copyrighted by \PRAGMA. See licen-en.pdf for 
%C details. 

%D Remark: due to the lack of \type {\clearmark}, the \ETEX\ 
%D dedicated mechanism is not yet operational. 

%D There are 256 \COUNTERS, \DIMENSIONS, \SKIPS, \MUSKIPS\ and
%D \BOXES, 16~in- and output buffers, but there is only one
%D \MARK. In TugBoat~8 (1987, no~1) Jim Fox presents a set of
%D macros that can be used to mimmick multiple marks. We
%D gladly adopt them here.
%D
%D This module was changed on behalf of \ETEX. The main
%D extension is that \type{\get..} and alike is used instead of
%D direct calls. The \TEX\ based multiple marks needs to store
%D the mark data but \ETEX\ uses a different approach.

\writestatus{loading}{Context Support Macros / Marks}

\unprotect

%% \beginTEX

%D This implementation is more or less compatible with the
%D other \type {\new} macros in \PLAIN\ \TEX. A mark is
%D defined by:
%D
%D \starttypen
%D \newmark\name
%D \stoptypen
%D
%D and can be called upon with:
%D
%D \starttypen
%D \gettopmark  \name  % or \topname
%D \getbotmark  \name  % or \botname
%D \getfirstmark\name  % or \firstname
%D \stoptypen
%D
%D The only drawback of his approach is that the marks must be
%D preloaded in the output routine. This is accomplished by
%D means of:
%D
%D \starttypen
%D \getmarks\name
%D \stoptypen
%D
%D The macros presented here are in most aspects copies of
%D those presented by Jim Fox. We've taken the freedom to
%D change a few things for more or less obvious reasons:
%D
%D \startopsomming
%D \som  Because the original macros look quite complicated,
%D       which is mainly due to extensive use of
%D       \type{\expandafter}'s and \type{\csname}'s, we changed
%D       those in favor of \type{\getvalue}.
%D \som  To be more in line with the rest of \CONTEXT, we've
%D       changed some of the names of macros.
%D \som  Because we are already short on \COUNTERS\ we use
%D       macros when possible.
%D \som  We maintain a list of defined marks and use one
%D       call for getting them all at once.
%D \som  We have extended the mechanism to splitmarks (not
%D       perfected yet).
%D \som  We've introduced optional expansion of the contents
%D       of marks.
%D \stopopsomming
%D
%D Whatever changes we've made, the credits still go to Jim,
%D whatever goes wrong is due to me. The method is described
%D in the TugBoat mentioned before, so we won't go into
%D details. All marks belonging to a group are packed in a
%D list. In this list they are preceded by a macro that can
%D be defined at will and a number concerning the position at
%D which it was defined.
%D
%D \starttypen
%D \def\somelist{... \domark5{this} ... \domark31{that} ...}
%D \stoptypen
%D
%D The original \type{\mark} keeps track of the number and
%D \type{\topmark} and \type{\botmark} are used to extract the
%D actual marks from the list. The counting is done by
%D
%D \starttypen
%D \currentmarker
%D \stoptypen
%D
%D In \CONTEXT\ we use the mark mechanism to keep track of
%D colors. In a complicated documents with many colors per
%D page, \type{\currentmarker} can therefore get pretty high.
%D (Well, this is not completely true, because we don't
%D always have to use marks.)

\newcount\currentmarker

%D The original implementation used a few more \COUNTERS. Two
%D have been substituted by macros, one has been replaced by
%D our scratch counter.
%D
%D \starttypen
%D % \newcount\topmarker
%D % \newcount\botmarker
%D % \newcount\foundmarker
%D \stoptypen
%D
%D We've also introduced some constants, one for the lists and
%D three for composing the mark commands.

\def\@@marklist@@  {marklist}
\def\@@marktop@@   {top}
\def\@@markbot@@   {bot}
\def\@@markfirst@@ {first}

%D The next one is new too. All defined marks are packed in a
%D comma seperated list. This could of course have been a token
%D list but \CONTEXT\ has some preference for comma lists.

\let\allmarks=\empty

%D \macros
%D   {expandmarks}
%D
%D There are two booleans. The first one handles the first
%D marks, the second concerns expansion. This second one is
%D new.

\newif\ifnofirstmarker
\newif\ifexpandmarks     \expandmarkstrue

%D We use an indirect call to the mark mechanism.

\let\normalmark           = \mark
\let\normaltopmark        = \topmark
\let\normalbotmark        = \botmark
\let\normalfirstmark      = \firstmark
\let\normalsplitbotmark   = \splitbotmark
\let\normalsplitfirstmark = \splitfirstmark

%D The next macro replaces the multiple step expansion and
%D command name constructors of Jim. This alternative leads to
%D a more readable source (we hope).

\def\makemarknames#1%
  {\bgroup
   \escapechar=-1
   \xdef\markname{\string#1}%
   \xdef\marklist{\@@marklist@@\string#1}%
   \egroup}

%D \macros
%D   {newmark,resetmark}
%D
%D A mark is defined by \type{\newmark}. At the same time,
%D the name of the mark is added to a commalist. The
%D three initializations were not in the original design, but
%D make calls from outside the output routine a bit more
%D robust.

\def\definenewmark#1#2%
  {\bgroup
   \makemarknames{#1}%
   #2%
   \global\letvalue{\@@marktop@@  \markname}\empty
   \global\letvalue{\@@markfirst@@\markname}\empty
   \global\letvalue{\@@markbot@@  \markname}\empty
   \setgvalue{\marklist}{\domark0{}}% beware of halfway definitions
   \long\gdef#1{\addmarker#1}%
   \egroup}

\def\newmark#1%
  {\definenewmark#1{\doglobal\addtocommalist\markname\allmarks}}

\let\setmark\empty
\let\resetmark\newmark

%D Some more natural interfacing macros:

\def\gettopmark        #1{\getvalue{\@@marktop@@  \strippedcsname#1}}
\def\getbottommark     #1{\getvalue{\@@markbot@@  \strippedcsname#1}}
\def\getfirstmark      #1{\getvalue{\@@markfirst@@\strippedcsname#1}}
\def\getsplitbottommark#1{\getvalue{\@@markbot@@  \strippedcsname#1}}
\def\getsplitfirstmark #1{\getvalue{\@@markfirst@@\strippedcsname#1}}

\let\getbotmark     \getbottommark
\let\getsplitbotmark\getsplitbottommark
\let\getsplittopmark\getsplitfirstmark

%D Don't ask me, but sometimes we need more control over
%D updating the marks, thereby we have:

\def\newpersistentmark#1% for an example see core-grd.tex
  {\definenewmark#1\relax}

%D \macros
%D   {setmark}
%D
%D Setting a new mark and adding a mark to the designated
%D list is done by \type{\addmarker}. This is an internal
%D command, the user set a marks bij calling it's name:
%D
%D \starttypen
%D \setmark\mymark{some text} % or \mymark{some text}
%D \stoptypen
%D
%D Where \type{\mymark} is previously defined by
%D \type{\newmark}.

\long\def\addmarker#1#2%
  {\bgroup
   \makemarknames{#1}%
   \global\advance\currentmarker by 1
   \normalmark{\the\currentmarker}%
   \@EA\!!toksa\@EA=\@EA\@EA\@EA{\csname\marklist\endcsname}%
   \ifexpandmarks
     \setxvalue{\marklist}%
       {\the\!!toksa
        \noexpand\domark
        \the\currentmarker{#2}}%
   \else
     \!!toksb=\@EA{#2}% one level
     \setxvalue{\marklist}%
       {\the\!!toksa
        \noexpand\domark
        \the\currentmarker{\the\!!toksb}}%
   \fi
   \egroup}

%D \macros
%D   {getmarks,getallmarks,
%D    getsplitmarks,getallsplitmarks}
%D
%D In fact, marks make only sense in the output routine. Marks
%D are derived from their list by means of \type{\getmarks}.
%D Only one call per mark is permitted in the output routine.
%D Therefore, it's far more easy to get them all at once, by
%D means of \type{\getallmarks}, which is not part of the
%D original design.
%D
%D This grabbing is done by processing the list using the
%D embedded \type{\domark} macros. When a relevant mark is
%D found, this macro is reassigned and from then on serves
%D in building the new list.

\def\getmarks#1%
  {\bgroup
   \makemarknames{#1}%
   \edef\topmarker{0\normaltopmark}%
   \edef\botmarker{0\normalbotmark}%
   \!!toksb={}%
   \nofirstmarkertrue
   \let\@fi=\fi     \let\fi=\relax
   \let\@or=\or     \let\or=\relax
   \let\@else=\else \let\else=\relax
   \let\domark=\doscanmarks
   \getvalue{\marklist}\lastmark
   %\message{markstatus : [\the\!!toksa\the\!!toksb\the\!!toksc]}%
   \long\setxvalue{\marklist}{\the\!!toksa\the\!!toksb\the\!!toksc}%
   \egroup}

\def\getallmarks%
  {\processcommacommand[\allmarks]\getmarks}

\def\getsplitmarks#1%
  {\bgroup
   \makemarknames{#1}%
%   \@EA\let\@EA\savedmarklist\@EA=\csname\marklist\endcsname
   \edef\topmarker{0\normalsplitfirstmark}%
   \edef\botmarker{0\normalsplitbotmark}%
   \!!toksb={}%
   \nofirstmarkertrue
   \let\@fi=\fi     \let\fi=\relax
   \let\@or=\or     \let\or=\relax
   \let\@else=\else \let\else=\relax
   \let\domark=\doscanmarks
   \getvalue{\marklist}\lastmark
%   \@EA\global\@EA\let\csname\marklist\endcsname=\savedmarklist
   \egroup}

\def\getallsplitmarks%
  {\processcommacommand[\allmarks]\getsplitmarks}

\long\def\dodoscanmarks#1%
  {\ifnum\scratchcounter>\topmarker\relax
   \@else
     \long\setgvalue{\@@marktop@@\markname}{#1}%
   \@fi
   \ifnum\scratchcounter>\botmarker\relax
     \let\domark=\dorecovermarks
     \!!toksb=\@EA{\@EA\domark\the\scratchcounter{#1}}%
   \@else
     \ifnofirstmarker
       \long\setgvalue{\@@markfirst@@\markname}{#1}%
       \ifnum\scratchcounter>\topmarker\relax
         \nofirstmarkerfalse
       \@fi
     \@fi
     \long\setgvalue{\@@markbot@@\markname}{#1}%
     \!!toksa=\@EA{\@EA\domark\the\scratchcounter{#1}}%
   \@fi}

\def\doscanmarks%
  {\afterassignment\dodoscanmarks\scratchcounter=}

\long\def\dorecovermarks#1\lastmark%
  {\!!toksc={\domark#1}}

\def\lastmark%
  {\!!toksc={}}

%D \macros
%D   {noninterferingmarks}
%D
%D Marks can interfere badly with for instance postprocessing
%D paragraphs, for instance when we want to grab the last box
%D using \type {\lastbox}, when at the same time using colors.

\def\noninterferingmarks%
  {\let\savednormalmark\normalmark
   \let\noninterferingmarks\relax
   \def\normalmark##1%
     {\ifhmode\prewordbreak\hbox\fi{\savednormalmark{##1}}}}

%D This macro is for instance used in the inline headings
%D postprocessing, as needed when we want to make those
%D clickable.

%% \endTEX

\protect \endinput 

%D Right from the beginning, \CONTEXT\ supported more than one
%D mark, using an extended version of Jim Fox multiple mark
%D mechanism. In \ETEX\ we can however directly access more
%D marks than we will ever need.

\beginETEX \marks \topmarks \botmarks \firstmarks

\let\normalmark           = \mark
\let\normaltopmark        = \topmark
\let\normalbotmark        = \botmark
\let\normalfirstmark      = \firstmark
\let\normalsplitbotmark   = \splitbotmark
\let\normalsplitfirstmark = \splitfirstmark

\let\normalmarks\marks

%D The 100\% compatible solution is:
%D
%D \starttypen
%D \def\@@marktop@@        {top}
%D \def\@@markbot@@        {bot}
%D \def\@@markfirst@@      {first}
%D \def\@@marksplitbot@@   {splitbot}
%D \def\@@marksplitfirst@@ {splitfirst}
%D 
%D \def\newmark#1% temporary 5 \cs's, will be rewritten ; global needed
%D   {\newmarks#1%
%D    \setxvalue{\@@marktop@@       \strippedcsname#1}{\noexpand\topmarks  \the#1 }%
%D    \setxvalue{\@@markbot@@       \strippedcsname#1}{\noexpand\botmarks  \the#1 }%
%D    \setxvalue{\@@markfirst@@     \strippedcsname#1}{\noexpand\firstmarks\the#1 }%
%D    \setxvalue{\@@marksplitbot@@  \strippedcsname#1}{\noexpand\splitbotmarks  \the#1 }%
%D    \setxvalue{\@@marksplitfirst@@\strippedcsname#1}{\noexpand\splitfirstmarks\the#1 }%
%D    \xdef#1{\noexpand\donormalmarks{\the#1}}}
%D
%D \def\gettopmark        #1{\getvalue{\@@marktop@@       \strippedcsname#1}}
%D \def\getbottommark     #1{\getvalue{\@@markbot@@       \strippedcsname#1}}
%D \def\getfirstmark      #1{\getvalue{\@@markfirst@@     \strippedcsname#1}}
%D \def\getsplitbottommark#1{\getvalue{\@@marksplitbot@@  \strippedcsname#1}}
%D \def\getsplitfirstmark #1{\getvalue{\@@marksplitfirst@@\strippedcsname#1}}
%D
%D \def\getsplitmarks#1%
%D   {\setvalue{\@@markbot@@\strippedcsname#1}%
%D      {\getvalue{\@@marksplitbot@@\strippedcsname#1}}%
%D    \setvalue{\@@marktop@@\strippedcsname#1}%
%D      {\getvalue{\@@marksplitfirst@@\strippedcsname#1}}%
%D    \setvalue{\@@markfirst@@\strippedcsname#1}%
%D      {\getvalue{\@@marksplitfirst@@\strippedcsname#1}}}
%D
%D \def\noninterferingmarks%
%D   {\let\saveddonormalmarks\donormalmarks
%D    \let\noninterferingmarks\relax
%D    \long\def\donormalmarks##1##2%
%D      {\ifhmode\prewordbreak\hbox\fi{\saveddonormalmarks{##1}{##2}}}}
%D 
%D \long\def\donormalmarks#1#2%
%D   {\bgroup
%D    \scratchcounter=#1\relax
%D    \ifexpandmarks
%D      \expanded{\normalmarks\scratchcounter{#2}}%
%D    \else
%D      \normalmarks\scratchcounter{#2}%
%D    \fi
%D    \egroup}
%D
%D \let\setmark\empty
%D \def\resetmark#1{#1{}}
%D \stoptypen
%D
%D However, I prefer the less \type{\cs} hungry ones. Beware, 
%D these ones (and the next) do no longer support direct marks. 

\newif\ifexpandmarks   \expandmarkstrue

\let\newmark           \newmarks
\let\gettopmark        \topmarks
\let\getbottommark     \botmarks
\let\getfirstmark      \firstmarks
\let\getsplitbottommark\splitbotmarks
\let\getsplitfirstmark \splitfirstmarks

\let\getbotmark        \botmarks
\let\getsplitbotmark   \splitbotmarks
\let\getsplittopmark   \splitfirstmarks

\long\def\setmark#1#2%
  {\ifexpandmarks
     \expanded{\normalmarks#1{#2}}%
   \else
     \normalmarks#1{#2}%
   \fi}

%D Resetting marks in not compatible with the old method. 
%D Here a node is inserted, which can interfere badly. In 
%D fact, a real \type {\clearmarks\name} is needed. 

\def\resetmark#1% interferes ! test this one on the big manual footers 
  {\marks#1{}}

%D What a hack! 

\newcount\resettedmarks

\def\clearmarks#1% a rather memory hungry alternative 
  {\edef\rubish{\topmarks#1\botmarks#1\firstmarks#1}%
   \ifx\rubish\empty\else
    %\global\advance\resettedmarks by 1 \message{(m:\the\resettedmarks)}%
     \newmarks#1%
   \fi}

\def\resetmark% 
  {\clearmarks}

\def\noninterferingmarks%
  {\let\savedsetmark\setmark
   \let\noninterferingmarks\relax
   \long\def\setmark##1##2%
     {\ifhmode\prewordbreak\hbox\fi{\savedsetmark{##1}{##2}}}}

\let\getmarks        \gobbleoneargument
\let\getallmarks     \relax
\let\getsplitmarks   \gobbleoneargument
\let\getallsplitmarks\relax

\let\newpersistentmark \newmark          % checken
\newif\ifnofirstmarker                   % checken

\endETEX

%D For those who want to know the story behind resetting 
%D marks, here are some examples of interference 
%D 
%D \startbuffer
%D \setbox0=\vbox{test}
%D \unvbox0\setbox0=\lastbox
%D \ruledhbox{\unhbox0}
%D \stopbuffer
%D
%D \typebuffer\blanko\haalbuffer\blanko
%D
%D \startbuffer
%D \setbox0=\vbox{test\normalmark{}}
%D \unvbox0\setbox0=\lastbox
%D \ruledhbox{\unhbox0}
%D \stopbuffer
%D
%D \typebuffer\blanko\haalbuffer\blanko
%D
%D \startbuffer
%D \setbox0=\vbox{test\hbox{\normalmark{}}}
%D \unvbox0\setbox0=\lastbox
%D \ruledhbox{\unhbox0}
%D \stopbuffer
%D
%D \typebuffer\blanko\haalbuffer\blanko
%D
%D \startbuffer
%D \setbox0=\vbox{test\vbox{\normalmark{}}}
%D \unvbox0\setbox0=\lastbox
%D \ruledhbox{\unhbox0}
%D \stopbuffer
%D
%D \typebuffer\blanko\haalbuffer\blanko

%D Because we used an indirect call to the mark mechanism 
%D we can redefine the original \type{\mark} command.

\beginTEX

\newmark\mark

\endTEX

%D One final advice. Use marks with care. When used in globally
%D assigned boxes, the list can grow quite big, and processing
%D can slow down considerably. This drawback is removed in 
%D \ETEX\ mode.  

\protect

\endinput
