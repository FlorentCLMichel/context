%D \module
%D   [       file=spec-dvi,
%D        version=1996.01.25,
%D          title=\CONTEXT\ Special Macros,
%D       subtitle=Generic \TEX\ Solutions,
%D         author=Hans Hagen,
%D           date=\currentdate,
%D      copyright={PRAGMA / Hans Hagen \& Ton Otten}]
%C
%C This module is part of the \CONTEXT\ macro||package and is
%C therefore copyrighted by \PRAGMA. See mreadme.pdf for 
%C details. 

\unprotect

%D \macros
%D   {dostartobject, 
%D    dostopobject,
%D    doinsertobject}
%D
%D Reuse of object is not supported by the \DVI\ format. We 
%D therefore just duplicate them using boxes. 

\startspecials[tex]

\newbox\DVIobjects \newcounter\DVIobjectcounter

\definespecial\dostartobject#1#2#3#4#5% 
  {\setbox\nextbox=\vbox\bgroup
     \def\dodostopobject%
       {\egroup
        \doglobal\increment\DVIobjectcounter
        \global\setbox\DVIobjects=\vbox
          {\offinterlineskip
           \forgetall
           \unvbox\DVIobjects
           \setbox\nextbox=\hbox{\box\nextbox}
           \wd\nextbox=\!!zeropoint
           \dp\nextbox=\!!zeropoint
           \ht\nextbox=\!!onepoint
           \allowbreak
           \box\nextbox}%
        \dosetobjectreference{#1}{#2}{\DVIobjectcounter}}}

\definespecial\dostopobject%
  {\dodostopobject}

\definespecial\doinsertobject#1#2%
  {\bgroup
   \dogetobjectreference{#1}{#2}\DVIobjectreference
   \splittopskip\!!zeropoint
   \setbox0=\copy\DVIobjects
   \dimen0=\DVIobjectreference pt
   \advance\dimen0 by -\!!onepoint
   \setbox2=\vsplit0 to \dimen0
   \ifdim\ht0>\!!onepoint
     \setbox0=\vsplit0 to \!!onepoint
   \fi
   \unvbox0\setbox0=\lastbox\unhbox0
   \egroup}

%D \macros
%D   {dosetposition,
%D    dosetpositionwhd,
%D    dosetpositionplus}
%D
%D The next specials only identify a position. It is up to 
%D a \DVI\ postprocessing utility to merge the right commands
%D into the utility file. Since in \CONTEXT, we only deal 
%D with relative positions, the reference point is not so 
%D important.  
%D
%D The postprocessor should translate the specials into 
%D commands and append these to \type {jobname.tuo} using the 
%D format: 
%D 
%D \starttypen
%D \pospxy    {identifier}{page}{x}{y} 
%D \pospxywhd {identifier}{page}{x}{y}{w}{h}{d}
%D \pospxyplus{identifier}{page}{x}{y}{w}{h}{d}{list}
%D \stoptypen
%D 
%D The postprocessor should, of course, provide the \type 
%D {page}, \type {x}, and \type {y} values. 

\definespecial\dosetposition#1%
  {\special{pos:pxy "#1"}}

\definespecial\dosetpositionwhd#1#2#3#4%
  {\special{pos:pxywhd "#1" #2 #3 #4}}

\definespecial\dosetpositionplus#1#2#3#4#5%
  {\special{pos:pxyplus "#1" #2 #3 #4 #5}}

%D The next special tells the position postprocessor what 
%D page dimensions were used.  

\let\flushDVIpositionpapersize\relax

\definespecial\dosetpositionpapersize#1#2%
  {\xdef\flushDVIpositionpapersize%
     {\special{pos:papersize #1 #2}%
      \global\noexpand\let\noexpand\flushDVIpositionpapersize\relax}}

\prependtoksonce \flushDVIpositionpapersize \to \everyshipout

\stopspecials

\protect

\endinput
