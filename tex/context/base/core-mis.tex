%D \module
%D   [       file=core-01a,
%D        version=1998.01.29,
%D          title=\CONTEXT\ Core Macros,
%D       subtitle=Miscelaneous,
%D         author=Hans Hagen,
%D           date=\currentdate,
%D      copyright={PRAGMA / Hans Hagen \& Ton Otten}]
%C
%C This module is part of the \CONTEXT\ macro||package and is
%C therefore copyrighted by \PRAGMA. See mreadme.pdf for 
%C details. 

\writestatus{loading}{Context Core Macros / Misc Commands}

\startmessages  dutch  library: systems
  title: systeem
      3: probeer LaTeX eens
\stopmessages

\startmessages  english  library: systems
  title: system
      3: try LaTeX
\stopmessages

\startmessages  german  library: systems
  title: system
      3: Versuche LaTeX
\stopmessages

\startmessages  czech  library: systems
  title: system
      3: zkuste LaTeX
\stopmessages

\startmessages  italian  library: systems
  title: sistema
      3: provare LaTeX
\stopmessages

\startmessages  norwegian  library: systems
  title: system
      3: fors�ker LaTeX
\stopmessages

\startmessages  romanian  library: systems
  title: sistem
      3: incercati LaTeX
\stopmessages

\unprotect

%D \macros 
%D   {simplifiedcommands, simplifycommands}
%D
%D I first needed this simplification in bookmarks. Users can
%D add their own if needed. 

\ifx\simplifiedcommands\undefined \newtoks\simplifiedcommands \fi

\def\simplifycommands{\the\simplifiedcommands}

%D A possibly growing list: 

\appendtoks        \def\executesynonym#1#2#3#4{#3}\to\simplifiedcommands
\appendtoks                              \def\ { }\to\simplifiedcommands
\appendtoks\def\type#1{\string\\\strippedcsname#1}\to\simplifiedcommands
\appendtoks                          \def\TeX{TeX}\to\simplifiedcommands
\appendtoks                  \def\ConTeXt{ConTeXt}\to\simplifiedcommands
\appendtoks                \def\MetaPost{MetaPost}\to\simplifiedcommands
\appendtoks                \def\MetaPost{MetaFont}\to\simplifiedcommands
\appendtoks                 \def\MetaPost{MetaFun}\to\simplifiedcommands
\appendtoks                              \def||{-}\to\simplifiedcommands

%D You would not expect the next macro in \CONTEXT, 
%D wouldn't you? It's there to warn \LATEX\ users that 
%D something is wrong. 

\def\documentstyle% 
  {\showmessage{\m!systems}{3}{}\stoptekst}

\let\documentclass=\documentstyle

% THIS WAS MAIN-002.TEX 

%\def\checkinterlineskip%
%  {\ifvmode
%     \ifdim\lastskip>\!!zeropoint\relax
%       \nointerlineskip
%     \else\ifdim\lastkern>\!!zeropoint\relax
%       \nointerlineskip
%     \fi\fi
%   \fi}

\def\horitems#1#2% #1=breedte #2=commandos
  {\dimen0=#1\relax
   \divide\dimen0 by \nofitems
   \!!counta=0\relax
   \def\docommando##1%
     {\advance\!!counta by 1\relax
      \processaction
        [\@@isuitlijnen]
        [  \v!links=>\hbox to \dimen0{\strut##1\hss},
          \v!rechts=>\hbox to \dimen0{\hss\strut##1},
          \v!midden=>\hbox to \dimen0{\hss\strut##1\hss},
           \v!marge=>\ifnum\!!counta=1\hss\else\hfill\fi
                     \strut##1%
                     \ifnum\!!counta=\nofitems\hss\else\hfill\fi,
         \s!default=>\hbox to \dimen0{\hss\strut##1\hss}, % midden
         \s!unknown=>\hbox to \dimen0{\strut##1\hss}]}%   % links
   \hbox to #1{\hss#2\hss}}

\def\veritems#1#2% #1=breedte #2=commandos
  {\dimen0=#1\relax
   \def\docommando##1%
     {\ifdim\dimen0<\!!zeropoint\relax   % the - was a signal
        \hbox to -\dimen0{\hss\strut##1}%
      \else\ifdim\dimen0>\!!zeropoint\relax
        \hbox to \dimen0{\strut##1\hss}%
      \else
        \hbox{\strut##1}%
      \fi\fi}%
   \vbox{#2}}

\def\dostelitemsin[#1]%
  {\getparameters[\??is][#1]%
   \doif{\@@isbreedte}{\v!onbekend}
     {\def\@@isbreedte{\hsize}}%
   \doifconversiondefinedelse{\@@issymbool}
     {\def\doitembullet##1{\convertnumber{\@@issymbool}{##1}}}
     {\doifsymboldefinedelse{\@@issymbool} 
        {\def\doitembullet##1{\symbol[\@@issymbool]}}{}}}

\def\makeitemsandbullets#1%
  {\doifelse{\@@isn}{\v!onbekend}
     {\getcommalistsize[#1]%
      \edef\nofitems{\commalistsize}}
     {\edef\nofitems{\@@isn}}%
   \setbox0=\hbox
     {\doitems
        {\@@isbreedte}
        {\processcommalist[#1]\docommando}}%
   \setbox2=\hbox
     {\doitems
        {\@@isbulletbreedte}
        {\dorecurse{\nofitems}
           {\docommando{\strut\doitembullet{\herhaler}}}}}}

\def\dostartitems#1#2#3%
  {\let\doitems=#2
   \def\@@isbulletbreedte{#3}%
   \makeitemsandbullets{#1}%
   \@@isvoor}

\def\dostopitems%
  {\@@isna
   \egroup}

\setvalue{doitems\v!boven}#1%
  {\dostartitems{#1}\horitems\@@isbreedte
   \noindent\vbox
     {\forgetall
      \doifsomething{\@@issymbool}
        {\doifnot{\@@issymbool}{\v!geen}
           {\box2
            \@@istussen
            \nointerlineskip}}%
      \box0}%
   \dostopitems}

\setvalue{doitems\v!onder}#1%
  {\dostartitems{#1}\horitems\@@isbreedte
   \noindent\vbox
     {\forgetall
      \box0
      \doifsomething{\@@issymbool}
        {\@@istussen
         \nointerlineskip
         \box2}}%
   \dostopitems}

\setvalue{doitems\v!inmarge}#1%
  {\dostartitems{#1}{\veritems}{-1.5em}%  - is a signal
   \noindent\hbox{\llap{\box2\hskip\linkermargeafstand}\box0}%
   \dostopitems}

\setvalue{doitems\v!links}#1%
  {\advance\hsize by -1.5em\relax
   \dostartitems{#1}{\veritems}{1.5em}%
   \noindent\hbox{\box2\box0}%
   \dostopitems}

\setvalue{doitems\v!rechts}#1%
  {\dostartitems{#1}{\veritems}{0em}%
   \noindent\hbox{\box0\hskip-\wd2\box2}%
   \dostopitems}

\def\stelitemsin%
  {\dosingleargument\dostelitemsin}

\def\complexitems[#1]%
  {\bgroup
   \stelitemsin[#1]%
   \parindent=\!!zeropoint
   \setlocalhsize
   \hsize=\localhsize
   \mindermeldingen
   \doifundefined{doitems\@@isplaats}%
     {\let\@@isplaats\v!links}%
   \getvalue{doitems\@@isplaats}}

\definecomplexorsimpleempty\items

\stelitemsin
  [\c!plaats=\v!links,
   \c!symbool=5,
   \c!breedte=\hsize,
   \c!uitlijnen=\v!midden,
   \c!n=\v!onbekend,
   \c!voor=\blanko,
   \c!tussen={\blanko[\v!middel]},
   \c!na=\blanko]

% Te zijner tijd [plaats=boven,onder,midden] implementeren,
% in dat geval moet eerst de maximale hoogte worden bepaald.
%
% Overigens kan een en ander mooier met \halign.

\def\dodefinieeralineas[#1][#2]%
  {\setvalue{\s!do\s!next#1}%
     {\def\\{\getvalue{#1}}}%
   \setvalue{#1}%
     {\getvalue{\s!do\s!next#1}%
      \dostartalineas{#1}}%
   \setvalue{\e!volgende#1}%
     {\getvalue{#1}}%
   \setvalue{\e!start#1}%
     {\bgroup
      \setvalue{\s!do\s!next#1}{}%
      \setvalue{\e!stop#1}%
        {\getvalue{#1}%
         \egroup}%
      \getvalue{#1}}%
   \getparameters[\??al#1]%
     [\c!n=3,
      \c!voor=\blanko,
      \c!na=\blanko,
      \c!afstand=1em,
      \c!hoogte=\v!passend,
      \c!lijn=\v!uit,
      \c!commando=,
      \c!uitlijnen=,
      \c!tolerantie=\v!soepel,
      \c!letter=,
      \c!kleur=,
      \c!boven=,
      \c!boven=\vss,
      \c!onder=\vfill,
      #2]%
   \setvalue{\e!stel#1\e!in}%
     {\stelalineasin[#1]}%
   \dorecurse
      {\getvalue{\??al#1\c!n}}
      {\stelalineasin[#1][\recurselevel]
         [\c!breedte=,
          \c!onder=\getvalue{\??al#1\c!onder},
          \c!boven=\getvalue{\??al#1\c!boven},
          \c!hoogte=\getvalue{\??al#1\c!hoogte},
          \c!letter=\getvalue{\??al#1\c!letter},
          \c!kleur=\getvalue{\??al#1\c!kleur},
          \c!lijn=\getvalue{\??al#1\c!lijn},
          \c!uitlijnen=\getvalue{\??al#1\c!uitlijnen},
          \c!tolerantie=\getvalue{\??al#1\c!tolerantie},
          \c!afstand=\getvalue{\??al#1\c!afstand}]}%
   \stelalineasin[#1][1][\c!afstand=0em]}

% nog monster
%
%\assignwidth
%  {\!!widtha}
%  {\getvalue{\??dd#1\c!breedte}}
%  {\doifelsevaluenothing{\??dd#1\c!monster}
%     {\hskip
%     {\doattributes
%        {\??al#1}\c!letter\c!kleur
%        {\getvalue{\??dd#1\c!monster}}}}
%  {0pt}

\def\definieeralineas%
  {\dodoubleargument\dodefinieeralineas}

\def\dostelalineasin[#1][#2][#3]%
  {\doifelse{#2}{\v!elk}
     {\dorecurse
        {\getvalue{\??al#1\c!n}}
        {\getparameters[\??al#1\herhaler][#3]}}
     {\ConvertToConstant\doifelse{#3}{}
        {\getparameters[\??al#1][#2]}
        {\def\docommando##1%
           {\getparameters[\??al#1##1][#3]}%
         \processcommalist[#2]\docommando}}}

\def\stelalineasin%
  {\dotripleempty\dostelalineasin}

\newcount\alteller
\newcount\alnsize
\newdimen\alhsize

\def\doalinealijn#1#2%
  {\doifelsevalue{\??al#2\the\alteller\c!lijn}{\v!aan}
     {\expandafter\dimen2=#1\relax
      \hskip.5\dimen2
      \hskip-\linewidth
      \vrule\!!width\linewidth
      \hskip.5\dimen2}
     {\hskip#1}}

\def\dostartalinea#1%
  {\doifelsevaluenothing{\??al#1\the\alteller\c!breedte}
     {\!!widtha=\alhsize\relax
      \divide\!!widtha by \alnsize}
     {\!!widtha=\getvalue{\??al#1\the\alteller\c!breedte}\relax}%
   \dostartattributes
     {\??al#1\the\alteller}\c!letter\c!kleur
     {}%
   \doifelsevalue{\??al#1\the\alteller\c!hoogte}{\v!passend}
     {\setbox0=\vtop}
     {\setbox0=\vtop to \getvalue{\??al#1\the\alteller\c!hoogte}}%
   \bgroup
   \blanko[\v!blokkeer]%
   \forgetall
   \getvalue{\??al#1\the\alteller\c!boven}%
   \getvalue{\??al#1\c!binnen}%
   \hsize=\!!widtha  % setting \wd afterwards removed
   \getvalue{\??al#1\the\alteller\c!binnen}%
   \edef\!!stringa{\getvalue{\??al#1\the\alteller\c!uitlijnen}}%  nodig?
   \expandafter\steluitlijnenin\expandafter[\!!stringa]%
   \edef\!!stringa{\getvalue{\??al#1\the\alteller\c!tolerantie}}% nodig?
   \expandafter\steltolerantiein\expandafter[\!!stringa]%
   \ignorespaces
   \endgraf
   \ignorespaces
   %
   % Nadeel van de onderstaande constructie is dat \everypar
   % binnen een groep kan staan en zo steeds \begstruts
   % worden geplaatst. Mooi is anders dus moet het anders!
   %
   % Hier is \Everypar niet nodig.
   %
   \everypar{\begstrut\everypar{}}%
   %
   \ignorespaces\geenspatie % dubbel: \ignorespaces
   \getvalue{\??al#1\the\alteller\c!commando}}

\def\dostopalinea#1%
  {\ifvmode
     \removelastskip
   \else
     \unskip\endstrut\endgraf
   \fi
   \getvalue{\??al#1\the\alteller\c!onder}%
   \egroup
   \ifdim\wd0=\!!zeropoint % no data
     \wd0=\!!widtha
   \fi
   \box0
   \dostopattributes
   \ifnum\alteller<\getvalue{\??al#1\c!n}\relax
     \def\next{\doalinea{#1}}%
   \else
     \def\next{\dostopalineas{#1}}%
   \fi
   \next}

\def\doalinea#1%
  {\global\advance\alteller by 1\relax
   \doifelsevaluenothing{\??al#1\the\alteller\c!afstand}
     {\doifnot{\the\alteller}{1}
        {\hskip\getvalue{\??al#1\c!afstand}}}
     {\doifelse{\the\alteller}{1}%
        {\hskip\getvalue{\??al#1\the\alteller\c!afstand}}
        {\doalinealijn{\getvalue{\??al#1\the\alteller\c!afstand}}{#1}}}%
   \setvalue{#1}{\dostopalinea{#1}}%
   \dostartalinea{#1}}

\def\dostartalineas#1%
  {\global\alteller=0\relax
   \parindent=\!!zeropoint
   \setlocalhsize
   \alhsize=\localhsize
   \alnsize=\getvalue{\??al#1\c!n}\relax
   \dorecurse
     {\getvalue{\??al#1\c!n}}
     {\doifelsevaluenothing{\??al#1\recurselevel\c!afstand}
        {\doifnot{\recurselevel}{1}
           {\global\advance\alhsize by -\getvalue{\??al#1\c!afstand}\relax}}
        {\global\advance\alhsize by -\getvalue{\??al#1\recurselevel\c!afstand}\relax}%
      \doifvaluesomething{\??al#1\recurselevel\c!breedte}
        {\global\advance\alnsize by -1\relax
         \global\advance\alhsize by -\getvalue{\??al#1\recurselevel\c!breedte}\relax}}%
   %\witruimte                 % gaat fout bij \framed
   \getvalue{\??al#1\c!voor}%
   \leavevmode                 % gaat wel goed bij \framed
   \vbox\bgroup\hbox\bgroup
     \doalinea{#1}}

\def\dostopalineas#1%
  {\egroup
   \egroup
   \par
   \getvalue{\??al#1\c!na}}%

\def\dosteltabin[#1]%
  {\getparameters[\??ta]
     [\c!kopletter=\v!normaal,
      \c!kopkleur=,
      \c!letter=\v!normaal,
      \c!kleur=,
      \c!breedte=\v!ruim,
      \c!monster={\hskip4em},
      \c!voor=,
      \c!na=,
      #1]%
   \doordefinieren
     [tab]
     [\c!kopletter=\@@takopletter,
      \c!kopkleur=\@@takleur,
      \c!monster=\@@tamonster,
      \c!breedte=\@@tabreedte,
      \c!voor=\@@tavoor,
      \c!na=\@@tana]}

\def\steltabin%
  {\dosingleargument\dosteltabin}

\steltabin
  [\c!plaats=\v!links]

% The following macro's are derived from PPCHTEX and
% therefore take some LaTeX font-switching into account.

\newif\ifloweredsubscripts

% Due to some upward incompatibality of LaTeX to LaTeX2.09
% and/or LaTeX2e we had to force \@@chemieletter. Otherwise
% some weird \nullfont error comes up.

\doifundefined{@@chemieletter}{\def\@@chemieletter{\rm}}

\def\beginlatexmathmodehack%
  {\ifmmode
     \let\endlatexmathmodehack=\relax
   \else
     \def\endlatexmathmodehack{$}$\@@chemieletter
   \fi}

\def\setsubscripts%
  {\beginlatexmathmodehack
   \def\dosetsubscript##1##2##3%
     {\dimen0=##3\fontdimen5##2%
      \setxvalue{@@\string##1\string##2}{\the##1##2\relax}%
      ##1##2=\dimen0\relax}%
   \def\dodosetsubscript##1##2%
     {\dosetsubscript{##1}{\textfont2}{##2}%
      \dosetsubscript{##1}{\scriptfont2}{##2}%
      \dosetsubscript{##1}{\scriptscriptfont2}{##2}}%
   %\dodosetsubscript{\fontdimen14}{?}%
   \dodosetsubscript{\fontdimen16}{.7}%
   \dodosetsubscript{\fontdimen17}{.7}%
   \global\loweredsubscriptstrue
   \endlatexmathmodehack}

\def\resetsubscripts%
  {\ifloweredsubscripts
     \beginlatexmathmodehack
     \def\doresetsubscript##1##2%
       {\dimen0=\getvalue{@@\string##1\string##2}\relax
        ##1##2=\dimen0}%
     \def\dodoresetsubscript##1%
       {\doresetsubscript{##1}{\textfont2}%
        \doresetsubscript{##1}{\scriptfont2}%
        \doresetsubscript{##1}{\scriptscriptfont2}}%
     %\dodoresetsubscript{\fontdimen14}%
     \dodoresetsubscript{\fontdimen16}%
     \dodoresetsubscript{\fontdimen17}%
     \global\loweredsubscriptsfalse
     \endlatexmathmodehack
   \fi}

\let\beginlatexmathmodehack = \relax
\let\endlatexmathmodehack   = \relax

\def\chem#1#2#3%
  {\bgroup
   \setsubscripts
   \mathematics{\hbox{#1}_{#2}^{#3}}%
   \resetsubscripts
   \egroup}

\def\celsius#1%
  {#1\mathematics{^\circ}C}

\def\graden%
  {\mathematics{^\circ}}

\def\inch%
  {\hbox{\rm\char125\relax}}

\def\breuk#1#2%
  {\mathematics{#1\over#2}}

%\def\bedrag#1%
%  {\mathematics{f~}\hbox{#1}}

\def\bedragprefix{\mathematics{f\normalfixedspace}}
\def\bedragsuffix{}

\def\bedrag#1%
  {\strut\hbox\bgroup
   \let\normalfixedspace~%
  % \def~{\futurelet\next\dofixedspace}%
  % \def\dofixedspace%
  %   {\hskip.5em\relax
  %    \ifx\next,%
  %      \hphantom{,}\let\next\gobbleoneargument
  %    \else\ifx\next.%
  %      \hphantom{.}\let\next\gobbleoneargument
  %    \else
  %      \let\next\relax
  %    \fi\fi
  %    \next}%
  % \bedragprefix#1\bedragsuffix
   \bedragprefix\digits{#1}\bedragsuffix
   \egroup}

% \definieeralineas[test][n=3]
%
% \stelalineasin[test][3][breedte=4cm,uitlijnen=links]
%
% \startopelkaar
% \test hans \\ ton \\ \bedrag{1.000,--} \\
% \test hans \\ ton \\ \bedrag{~.~~1,--} \\
% \test hans \\ ton \\ \bedrag{~.~~1,~~} \\
% \test hans \\ ton \\ \bedrag{~.100,--} \\
% \test hans \\ ton \\ \subtot{1.000,--} \\
% \test hans \\ ton \\ \bedrag{1.000,--} \\
% \test hans \\ ton \\ \bedrag{1.000,--} \\
% \test hans \\ ton \\ \totaal{1.000,--} \\
% \test hans \\ ton \\ \bedrag{nihil,--} \\
% \test hans \\ ton \\ \totaal{nihil,--} \\
% \test hans \\ ton \\ \subtot{nihil,--} \\
% \stopopelkaar

\def\doorsnede%
  {\hbox{\rlap/$\circ$} }

\def\complexpunten[#1]%
  {\dimen0=.5em\relax
   \multiply\dimen0 by #1\relax
   \hbox to \dimen0
     {\leaders\hbox to .5em{\hss.\hss}\hss}}

\def\simplepunten%
  {\complexpunten[5]}

\definecomplexorsimple\punten

\def\ongeveer%
  {\mathematics{\pm}}

\def\permille%
  {\mathematics{^{\scriptscriptstyle0}\kern-.25em/\kern-.2em_{\scriptscriptstyle00}}}

\def\percent%
  {\mathematics{^{\scriptscriptstyle0}\kern-.25em/\kern-.2em_{\scriptscriptstyle0}}}

\let\promille=\permille
\let\procent =\percent

\def\permine%
  {\dontleavehmode
   \bgroup
   \setbox0=\hbox
     {\mathematics{+}}%
   \hbox to \wd0
     {\hss
      \mathematics{^{\scriptscriptstyle-}\kern-.4em/\kern-.3em_{\scriptscriptstyle-}}%
      \hss}%
   \egroup}

% for compatibility

\def\unknown%
  {\dontleavehmode\punten[3]}

% currency

\def\dollar%
  {\bgroup
   \ifnum\fam=\itfam
     \sl
   \else\ifnum\fam=\bifam
     \bs
   \fi\fi
   \$%
   \egroup}

\def\sterling%
  {\bgroup
   \ifnum\fam=\bffam
     \bi
   \else\ifnum\fam=\bifam
     \bi
   \else\ifnum\fam=\bsfam
     \bi
   \else
     \it
   \fi\fi\fi
   \$%
   \egroup}

\def\florijn%
  {\bgroup
   \ifnum\fam=\bffam
     \bi
   \else\ifnum\fam=\bifam
     \bi
   \else\ifnum\fam=\bsfam
     \bi
   \else
     \it
   \fi\fi\fi
   f%
   \egroup}

% \newsignal\quotationsignal
% \def\quotationskip{.125em}
% 
% \def\stelciterenin%
%   {\dodoubleargument\getparameters[\??ci]}
% 
% \def\stelcitatenin%
%   {\stelciterenin}
% 
% \def\dostartcitaat[#1]%
%   {\bgroup
%    \@@civoor
%    \doifelsenothing{#1}
%      {\let\dostopcitaat=\relax}
%      {\startsmaller[#1]
%       \let\dostopcitaat=\stopsmaller}%
%    \dostartattributes\??ci\c!letter\c!kleur{}%
%    \setbox0=\hbox{\getvalue{\??la\currentlanguage\c!leftquotation}}%
%    \hskip-\wd0
%    \box0\relax
%    \ignorespaces}
% 
% \def\stopcitaat%
%   {\unskip\hsmash{\getvalue{\??la\currentlanguage\c!rightquotation}}%
%    \dostopattributes
%    \dostopcitaat
%    \@@cina
%    \egroup}
% 
% \def\startcitaat%
%   {\dosingleempty\dostartcitaat}
% 
% \def\dohandlequotation#1%
%   {\ifdim\lastskip=\quotationsignal
%      \unskip\hskip\quotationskip
%    \fi
% \ifhmode % else funny pagebeaks 
%    \penalty\!!tenthousand\hskip\!!zeropoint      % == \prewordbreak
% \fi
%    \strut % new, needed below 
%    \getvalue{\??la\currentlanguage#1}%
%    \penalty\!!tenthousand\hskip\quotationsignal} % +- \prewordbreak
% 
% \unexpanded\def\citaat%
%   {\groupedcommand
%      {\dohandlequotation\c!leftquotation}
%      {\dohandlequotation\c!rightquotation}}
% 
% \unexpanded\def\citeer%
%   {\doifelse{\@@ciletter}{\v!normaal}
%      {\let\next=\doquotedcite}
%      {\let\next=\doattributedcite}%
%    \next}
% 
% \def\doquotedcite%
%   {\groupedcommand
%      {\dohandlequotation\c!leftquote}
%      {\dohandlequotation\c!rightquote}}
% 
% \def\doattributedcite%
%   {\groupedcommand
%      {\dostartattributes\??ci\c!letter\c!kleur}
%      {\dostopattributes}}
% 
% % The previous one fails in \placefloat[left]{}{}, so instead
% % we use the next alternative, where the first one is handled 
% % outside group. Watch the strut. 
% 
% \unexpanded\def\citaat%
%   {\dohandlequotation\c!leftquotation
%    \groupedcommand{}{\dohandlequotation\c!rightquotation}}
% 
% \def\doquotedcite%
%   {\dohandlequotation\c!leftquote
%    \groupedcommand{}{\dohandlequotation\c!rightquote}}
% 
% \stelciterenin
%   [\c!letter=\v!normaal,
%    \c!kleur=,
%    \c!voor=\startsmaller,
%    \c!na=\stopsmaller]

\newsignal\quotationsignal \def\quotationskip{.125em}

\def\stelciterenin%
  {\dodoubleargument\getparameters[\??ci]}

\def\stelcitatenin%
  {\stelciterenin}

\def\startcitaat%
  {\bgroup
   \dosingleempty\dostartcitaat}

\def\dostartcitaat[#1]%
  {\@@civoor
   \doifelsenothing{#1}
     {\let\dostopcitaat=\relax}
     {\startsmaller[#1]
      \let\dostopcitaat=\stopsmaller}%
   \dostartattributes\??ci\c!letter\c!kleur{}%
   \setbox0=\hbox{\getvalue{\??la\currentlanguage\c!leftquotation}}%
   \hskip-\wd0
   \box0\relax
   \ignorespaces}

\def\stopcitaat%
  {\unskip\hsmash{\getvalue{\??la\currentlanguage\c!rightquotation}}%
   \dostopattributes
   \dostopcitaat
   \@@cina
   \egroup}

\def\dohandlequotation#1#2%
  {\ifdim\lastskip=\quotationsignal
     \unskip\hskip\quotationskip 
   \else
     #2%
   \fi
   \ifhmode % else funny pagebeaks 
     \penalty\!!tenthousand\hskip\!!zeropoint      % == \prewordbreak
   \fi
   \strut % new, needed below 
   \getvalue{\??la\currentlanguage#1}%
   \penalty\!!tenthousand\hskip\quotationsignal} % +- \prewordbreak

\unexpanded\def\citaat%
  {\groupedcommand
     {\dohandlequotation\c!leftquotation\relax}
     {\dohandlequotation\c!rightquotation\unskip}}

\unexpanded\def\citeer%
  {\doifelse{\@@ciletter}{\v!normaal}
     {\let\next=\doquotedcite}
     {\let\next=\doattributedcite}%
   \next}

\def\doquotedcite%
  {\groupedcommand
     {\dohandlequotation\c!leftquote\relax}
     {\dohandlequotation\c!rightquote\unskip}}

\def\doattributedcite%
  {\groupedcommand
     {\dostartattributes\??ci\c!letter\c!kleur}
     {\dostopattributes}}

%D The previous one fails in \placefloat[left]{}{}, so instead
%D we use the next alternative, where the first one is handled 
%D outside group. Watch the strut. 

\unexpanded\def\citaat%
  {\dohandlequotation\c!leftquotation\relax
   \groupedcommand{}{\dohandlequotation\c!rightquotation\unskip}}

\def\doquotedcite%
  {\dohandlequotation\c!leftquote\relax
   \groupedcommand{}{\dohandlequotation\c!rightquote\unskip}}

\stelciterenin
  [\c!letter=\v!normaal,
   \c!kleur=,
   \c!voor=\startsmaller,
   \c!na=\stopsmaller]

%D The next features was so desperately needed by Giuseppe
%D Bilotta that he made a module for it. Since this is a
%D typical example of core functionality, I decided to extend
%D the low level quotation macros in such a way that a speech
%D feature could be build on top of it. The speech opening and
%D closing symbols are defined per language. Italian is an
%D example of a language that has them set. 

\newcounter\speechlevel \newconditional\insidespeech

\def\startspeech 
  {\doglobal\increment\speechlevel\relax
   \dohandlequotation\c!leftspeech\relax 
   \doglobal\settrue\insidespeech
   \ignorespaces}

\def\stopspeech
  {\dohandlequotation\c!rightspeech\unskip
   \doglobal\decrement\speechlevel\relax
   \ifcase\speechlevel\relax \doglobal\setfalse\insidespeech \fi}

\def\dohandlespeech%
  {\relax \ifcase\speechlevel 
   \or\ifconditional\insidespeech
     \dohandlequotation\c!middlespeech\relax 
   \else
     \doglobal\settrue\insidespeech
   \fi\fi}

\unexpanded\def\speech% 
  {\doglobal\increment\speechlevel\relax
   \dohandlequotation\c!leftspeech\relax
   \groupedcommand
     {\ignorespaces}
     {\dohandlequotation\c!rightspeech\unskip
      \doglobal\decrement\speechlevel\relax}}

\appendtoks \dohandlespeech \to \everypar

% Tijden horen hier niet thuis en zullen in een aparte
% module worden ondergebracht. voorlopig handhaven we ze nog
% even. Een implementatie met \doordefinieren zou beter voldoen
% omdat een en ander dan instelbaar wordt. Het is trouwens
% zowieso beter het commando \tijd te reserveren voor de
% systeemtijd.

\def\tijd#1%
  {\setbox0=\hbox{00.00}%
   \hbox to \wd0{\hfill#1}}

\def\tijdspan#1#2%
  {\hbox{\tijd{#1}~---~\tijd{#2}}}

\def\activiteit#1#2%
  {\activity{\tijdspan{#1}{#2}}}

\def\activiteit#1#2%
  {\sym{\tijdspan{#1}{#2}}}


\def\dotoevoegen#1%
  {\def\next{#1}%
   \dorecurse{#1}{\inlinker{\next~+}\def\next{}\crlf}}

\def\complextoevoegen[#1]%
  {\blanko
   \processaction
     [#1]
     [   \v!klein=>\dotoevoegen{3},
        \v!middel=>\dotoevoegen{6},
         \v!groot=>\dotoevoegen{9},
     \s!default=>\dotoevoegen{6},
     \s!unknown=>\dotoevoegen{#1}]
   \blanko}

\definecomplexorsimpleempty\toevoegen


\def\dorooster[#1]%
  {\begingroup
   \getparameters[\??rt]
     [\c!x=0,\c!y=0,
      \c!nx=10,\c!ny=10,
      \c!dx=.5,\c!dy=.5,
      \c!xstap=0,\c!ystap=0,
      \c!eenheid=\s!cm,
      \c!schaal=1,
      \c!factor=1,
      \c!offset=\v!ja,
      \c!plaats=\v!links,
      #1]%
   \startpositioning
     \dimen0=\@@rtdx\@@rteenheid\relax
     \dimen0=\@@rtschaal\dimen0\relax
     \dimen0=\@@rtfactor\dimen0\relax
     \multiply\dimen0 by \@@rtnx\relax
     \dimen2=\@@rtdy\@@rteenheid\relax
     \dimen2=\@@rtschaal\dimen2\relax
     \dimen2=\@@rtfactor\dimen2\relax
     \multiply\dimen2 by \@@rtny\relax
     \def\horline
       {\vbox
          {\hrule
             \!!width \dimen0
             \!!height \linewidth
             \!!depth \!!zeropoint}}%
     \def\verline%
       {\vrule
          \!!width \linewidth
          \!!height \dimen2
          \!!depth \!!zeropoint}%
     \doglobal\newcounter\@@roosterc
     \doglobal\newcounter\@@roosterd
     \doglobal\newcounter\@@roostere
     \def\setlegend##1##2##3%
       {\gdef\@@roosterc{0}%
        \dimen0=2em\relax
        \dimen2=##2\@@rteenheid\relax
        \dimen2=\@@rtschaal\dimen2\relax
        \dimen2=\@@rtfactor\dimen2\relax
        \divide\dimen0 by \dimen2\relax
        \xdef\@@roostere{\number\dimen0}%
        \ifnum\@@roostere>50
          \gdef\@@roostere{100}%
        \else\ifnum\@@roostere>10
          \gdef\@@roostere{50}%
        \else\ifnum\@@roostere>5
          \gdef\@@roostere{10}%
        \else\ifnum\@@roostere>1
          \gdef\@@roostere{5}%
        \else
          \gdef\@@roostere{1}%
        \fi\fi\fi\fi
        \gdef\@@roosterd{0}%
        \def\legend%
          {\ifnum\@@roosterd=0\relax
             \vbox
               {\increment(\@@roosterc,##1)%
                \hbox to 2em{\hss\@@roosterc\hss}}%
             \global\let\@@roosterd=\@@roostere
           \fi
             \doglobal\decrement\@@roosterd
             \doglobal\increment(\@@roosterc,##1)}}%
     \def\draw##1##2##3##4##5##6##7##8##9%
       {\setuppositioning
          [\c!status=##8,
           \c!xstap=\v!absoluut,
           \c!ystap=\v!absoluut,
           \c!eenheid=\@@rteenheid,
           \c!schaal=\@@rtschaal,
           \c!factor=\@@rtfactor,
           \c!offset=\@@rtoffset,
           \c!xoffset=##6,
           \c!yoffset=##7]%
        \doifelse{##9}{\v!midden}
          {\scratchdimen=##3pt\scratchdimen=.5\scratchdimen
           \edef\@@psxx{\withoutpt\the\scratchdimen}%
           \scratchdimen=##4pt\scratchdimen=.5\scratchdimen
           \edef\@@psyy{\withoutpt\the\scratchdimen}%
           \scratchcounter=##2\advance\scratchcounter by -1
           \edef\@@pszz{\the\scratchcounter}}
          {\edef\@@psxx{0}\edef\@@psyy{0}\edef\@@pszz{##2}}%
        \position(\@@psxx,\@@psyy){##1}%
        \setuppositioning
          [\c!status=##8,
           \c!xstap=\v!relatief,
           \c!ystap=\v!relatief,
           \c!schaal=\@@rtschaal,
           \c!factor=\@@rtfactor,
           \c!offset=\@@rtoffset,
           \c!eenheid=\@@rteenheid]%
        \dorecurse{\@@pszz}{\position(##3,##4){##5}}}%
     \draw
       \verline\@@rtnx\@@rtdx0\verline\!!zeropoint\!!zeropoint\v!start\empty
     \draw
       \horline\@@rtny0\@@rtdy\horline\!!zeropoint\!!zeropoint\v!start\empty
     \tfx
     \doifnot{\@@rtxstap}{0}
       {\setlegend\@@rtxstap\@@rtdx\@@rtx
        \draw\legend\@@rtnx\@@rtdx0\legend{-1em}{-1.5em}\v!overlay\@@rtplaats}%
     \doifnot{\@@rtystap}{0}
       {\setlegend\@@rtystap\@@rtdy\@@rty
        \draw\legend\@@rtny0\@@rtdy\legend{-2em}{-.75ex}\v!overlay\@@rtplaats}%
  \stoppositioning
  \endgroup}

\def\rooster%
  {\dosingleempty\dorooster}

% Dit wordt:
%
%   \doorverwijzen[naam][instellingen] enz.
%
% waarbij <naam> bijvoorbeeld publicatie is. Dit levert:
%
%   \start<naam>
%   \stop<naam>
%
%   \beginvan<naam>
%   \eindvan<naam>
%
%   \publicatie
%
%   \volledigelijstmetpublicaties
%
% eigenlijk kan ook door... zo worden uitgebreid!

\doornummeren
  [@publicatie]
  [\c!plaats=\v!links,
   \c!breedte=\@@pbbreedte,\c!hang=,\c!monster=,
   \c!voor=\@@pbvoor,\c!na=\@@pbna,\c!tussen=,
   \c!kopletter=\@@pbkopletter,\c!letter=,
   \c!kopkleur=\@@pbkopkleur,\c!kleur=,
   \c!wijze=\@@pbwijze,\c!blokwijze=\@@pbblokwijze,
   \c!tekst=,\c!links=\@@pblinks,\c!rechts=\@@pbrechts]

\def\dostelpublicatiesin[#1]%
  {\getparameters[\??pb][#1]}

\def\stelpublicatiesin%
  {\dosingleargument\dostelpublicatiesin}

\def\apa@publicatie%
  {\doifsomething{\@@pb@naam}{\@@pb@naam,\spatie}%
   \doifsomething{\@@pb@titel}{{\sl\@@pb@titel}.\spatie}%
   \doifsomething{\@@pb@jaar}{(\@@pb@jaar).\spatie}%
   \doifsomething{\@@pb@plaats}{\@@pb@plaats\doifelsenothing{\@@pb@uitgever}{.}{:\spatie}}%
   \doifsomething{\@@pb@uitgever}{\@@pb@uitgever.}}

\def\normaal@publicatie%
  {\@@pb@naam, \@@pb@titel, \@@pb@jaar, \@@pb@pagina, \@@pb@plaats, \@@pb@uitgever.}

\def\complexstartpublicatie[#1]#2\stoppublicatie%
  {\bgroup%
   \def\dosetpublicatie%
     {\processcommalist
        [naam,titel,jaar,plaats,pagina,uitgever]
        \setpublicatie
      \ignorespaces}%
   \def\setpublicatie##1%
      {\setvalue{\??pb @##1}{}%
       \setvalue{##1}####1{\setvalue{\??pb @##1}{####1}\ignorespaces}}%
   \def\getpublicatie%
     {\doifsomething{\@@pbvariant}{\getvalue{\@@pbvariant @publicatie}}}%
   \doifelse{\@@pbnummeren}{\v!ja}%
      {\@publicatie[#1]\dosetpublicatie#2\getpublicatie\par}%
      {\@@pbvoor
       \dosetpublicatie\ignorespaces#2\getpublicatie
       \@@pbna}%
   \egroup}

\definecomplexorsimpleempty\startpublicatie

\def\publicatie#1[#2]%
  {\@@pblinks\in{#1}[#2]\@@pbrechts}

\stelpublicatiesin
  [\c!nummeren=\v!ja,
   \c!variant=\c!apa,
   \c!breedte=2em,
   \c!hang=,
   \c!monster=,
   \c!voor=,
   \c!na=,
   \c!tussen=,
   \c!kopletter=,
   \c!kopkleur=,
   \c!letter=,
   \c!kleur=,
   \c!blokwijze=\v!per\v!tekst,
   \c!wijze=\v!per\v!tekst,
   \c!tekst=,
   \c!links={[},
   \c!rechts={]}]

\def\kenmerkdatum%
  {\currentdate[\v!kenmerk]}

\def\dokenmerk[#1]%
   {\noheaderandfooterlines
    \bgroup  
    \getparameters
      [\??km]
      [\c!bet=\unknown,\c!dat=\unknown,\c!ken=\unknown,
       \c!van=,        \c!aan=,        \c!ref=,        #1]%
    % moet anders, hoort niet in 01b
    \assigntranslation[nl=referentie,en=reference,du=Referenz,   sp=referencia]\to\@@@kmref
    \assigntranslation[nl=van,       en=from,     du=Von,        sp=de]\to\@@@kmvan
    \assigntranslation[nl=aan,       en=to,       du=An,         sp=a]\to\@@@kmaan
    \assigntranslation[nl=betreft,   en=concerns, du=Betreff,    sp=]\to\@@@kmbet
    \assigntranslation[nl=datum,     en=date,     du=Datum,      sp=fecha]\to\@@@kmdat
    \assigntranslation[nl=kenmerk,   en=mark,     du=Kennzeichen,sp=]\to\@@@kmken
    %
    \definetabulate[\s!dummy][|l|p|]    
    \startdummy
      \NC\@@@kmbet\EQ\@@kmbet\NC\NR
      \NC\@@@kmdat\EQ\@@kmdat\NC\NR
      \NC\@@@kmken\EQ\expanded{\kap{\@@kmken}}\NC\NR
      \doifsomething{\@@kmvan\@@kmaan}{\NC\NC\NC\NR}%
      \doifsomething{\@@kmvan}{\NC\@@@kmvan\EQ\@@kmvan\NC\NR}%
      \doifsomething{\@@kmaan}{\NC\@@@kmaan\EQ\@@kmaan\NC\NR}%
      \doifsomething{\@@kmref}{\NC\NC\NC\NR\NC\@@@kmref\EQ\@@kmref\NC\NR}%
    \stopdummy
    \egroup}

\def\kenmerk%
  {\dosingleargument\dokenmerk}

% NIEUW NIEUW NIEUW NIEUW NIEUW NIEUW NIEUW NIEUW NIEUW NIEUW NIEUW
% NIEUW NIEUW NIEUW NIEUW NIEUW NIEUW NIEUW NIEUW NIEUW NIEUW NIEUW

\def\??ri{@@ri}

\def\stelrijenin%
  {\dodoubleargument\getparameters[\??ri]}

\def\complexstartrijen[#1]%
  {\bgroup
   \stelrijenin[#1]%
   \let\do@@rionder=\relax
   \def\rij%
     {\do@@rionder
      \egroup
      \dimen0=\vsize
      \divide\dimen0 by \@@rin
      \advance\dimen0 by -\lineskip
      \vbox to \dimen0
        \bgroup
        \@@riboven
        \let\do@@rionder=\@@rionder
        \ignorespaces}%
   \bgroup
   \rij}

\definecomplexorsimpleempty\startrijen

\def\stoprijen%
  {\do@@rionder
   \egroup
   \egroup}

\stelrijenin
  [\c!n=2,
   \c!boven=,
   \c!onder=\vfill]

% THIS WAS MAIN-003.TEX 

\startmessages  dutch  library: systems
     41: externe file -- in groep -- bestaat niet
\stopmessages

\startmessages  english  library: systems
     41: external file -- in group -- does not exist
\stopmessages

\startmessages  german  library: systems
     41: Externe Datei -- in Gruppe -- existiert nicht
\stopmessages

\startmessages  czech  library: systems
     41: externi soubor -- ve skupine -- neexistuje
\stopmessages

\startmessages  italian  library: systems
     41: il file esterno -- del gruppo -- non esiste
\stopmessages

\startmessages  norwegian  library: systems
     41: ekstern fil -- i gruppe -- eksisterer ikke
\stopmessages

\startmessages  romanian  library: systems
     41: fisierul extern -- din grupul -- nu exista
\stopmessages

\definetabulate
  [\e!legenda]
  [|emj1|i1|mR|]

\setuptabulate
  [\e!legenda]
  [\c!eenheid=.75em,\c!binnen=\setquicktabulate\leg,EQ={=}]

\definetabulate
  [\e!legenda][\v!twee]
  [|emj1|emk1|i1|mR|]

\definetabulate
  [\e!gegeven]
  [|R|ecmj1|i1mR|]

\setuptabulate
  [\e!gegeven]
  [\c!eenheid=.75em,\c!binnen=\setquicktabulate\geg,EQ={=}]

\unexpanded\def\xbox%
  {\bgroup\aftergroup\egroup\hbox\bgroup\tx\let\next=}

\unexpanded\def\xxbox%
  {\bgroup\aftergroup\egroup\hbox\bgroup\txx\let\next=}

% \def\mrm#1%
%   {$\rm#1$}

%D \macros 
%D   {definepairedbox, setuppairedbox, placepairedbox}
%D
%D Paired boxes, formally called legends, but from now on a
%D legend is just an instance, are primarily meant for
%D typesetting some text alongside an illustration. Although
%D there is quite some variation possible, the functionality is
%D kept simple, if only because in most cases such pairs are
%D typeset sober. 
%D 
%D The location specification accepts a pair, where the first
%D keyword specifies the arrangement, and the second one the
%D alignment. The first key of the location pair is one of 
%D \type {left}, \type {right}, \type {top} or \type {bottom}, 
%D while the second key can also be \type {middle}. 
%D
%D The first box is just collected in an horizontal box, but 
%D the second one is a vertical box that gets passed the 
%D bodyfont and alignment settings. 

%  \startbuffer[test]
%  \test left   \test left,top    \test left,bottom  \test left,middle
%  \test right  \test right,top   \test right,bottom \test right,middle
%  \test top    \test top,left    \test top,right    \test top,middle
%  \test bottom \test bottom,left \test bottom,right \test bottom,middle
%  \stopbuffer
%  
%  \def\showtest#1% 
%    {\pagina
%     \typebuffer[demo]
%     \def\test##1
%       {\startlinecorrection[blank]
%        \getbuffer[demo]%
%        \ruledhbox\placelegend
%          [bodyfont=6pt,location={##1}]
%          {\framed[width=.25\textwidth]{\tttf##1}}
%          {#1}
%        \stoplinecorrection}
%     \getbuffer[test]}
%  
%  \startbuffer[demo]
%  \setuplegend
%    [width=\hsize,maxwidth=\makeupwidth,
%     height=\vsize,maxheight=\makeupheight]
%  \stopbuffer
%  
%  \showtest{These examples demonstrate the default settings.}
%  
%  \startbuffer[demo]
%  \setuplegend
%    [width=\textwidth,
%     maxwidth=\textwidth]
%  \stopbuffer
%  
%  \showtest{\input tufte }
%  
%  \startbuffer[demo]
%  \setuplegend
%    [width=.65\textwidth]
%  \stopbuffer
%  
%  \showtest{\input knuth }
%  
%  \startbuffer[demo]
%  \setuplegend
%    [height=2cm]
%  \stopbuffer
%  
%  \showtest{These examples demonstrate some other settings.}
%  
%  \startbuffer[demo]
%  \setuplegend
%    [width=.65\textwidth,
%     height=2cm]
%  \stopbuffer
%  
%  \showtest{These examples demonstrate some other settings.}
%  
%  \startbuffer[demo]
%  \setuplegend
%    [n=2,align=right,width=.5\textwidth]
%  \stopbuffer
%  
%  \showtest{\input zapf }

%D \macros 
%D   {setuplegend, placelegend}
%D
%D It makes sense to typeset a legend to a figure in \TEX\ 
%D and not in a drawing package. The macro \type {\placelegend}
%D combines a figure (or something else) and its legend. This
%D command is just a paired box.
%D
%D The legend is placed according to \type {location}, being 
%D \type {bottom} or \type {right}. The macro macro is used as
%D follows. 
%D
%D \starttypen 
%D \placefigure
%D   {whow}
%D   {\placelegend
%D      {\externalfigure[cow]}
%D      {\starttabulation
%D       \NC 1 \NC head \NC \NR
%D       \NC 2 \NC legs \NC \NR
%D       \NC 3 \NC tail \NC \NR
%D       \stoptabulation}}
%D 
%D \placefigure
%D   {whow}
%D   {\placelegend
%D      {\externalfigure[cow]}
%D      {\starttabulation[|l|l|l|l|]
%D       \NC 1 \NC head \NC 3 \NC tail \NC \NR
%D       \NC 2 \NC legs \NC   \NC      \NC \NR
%D       \stoptabulation}}
%D 
%D \placefigure
%D   {whow}
%D   {\placelegend[n=2]
%D      {\externalfigure[cow]}
%D      {\starttabulation
%D       \NC 1 \NC head \NC \NR
%D       \NC 2 \NC legs \NC \NR
%D       \NC 3 \NC tail \NC \NR
%D       \stoptabulation}}
%D 
%D \placefigure
%D   {whow}
%D   {\placelegend[n=2]
%D      {\externalfigure[cow]}
%D      {head \par legs \par tail}}
%D 
%D \placefigure
%D   {whow}
%D   {\placelegend[n=2]
%D      {\externalfigure[cow]}
%D      {\startitemize[packed]
%D       \item head \item legs \item  tail \item belly \item horns 
%D       \stopitemize}}
%D 
%D \placefigure
%D   {whow}
%D   {\placelegend[n=2,width=.8\hsize]
%D      {\externalfigure[cow]}
%D      {\startitemize[packed]
%D       \item head \item legs \item  tail \item belly \item horns 
%D       \stopitemize}}
%D \stoptypen 

% \def\setuplegend%
%   {\dodoubleargument\getparameters[\??ld]}
% 
% \setuplegend
%   [\c!n=1,
%    \c!afstand=1em,
%    \c!tussen={\blanko[\v!middel]},
%    \c!breedte=\hsize,
%    \c!hoogte=\vsize,
%    \c!korps=,
%    \c!plaats=\v!onder]
% 
% \def\placelegend%
%   {\bgroup
%    \dosingleempty\doplacelegend}
% 
% \def\doplacelegend[#1]% watch the hsize/vsize tricks
%   {\setuplegend[#1]%  % and don't change them 
%    \dowithnextbox
%      {\switchtobodyfont[\@@ldkorps]% split under same regime  
%       \scratchdimen=\wd\nextbox
%       \doifelse{\@@ldplaats}{\v!rechts}
%         {\vsize=\ht\nextbox
%          \vsize=\@@ldhoogte
%          \hsize=\zetbreedte
%          \advance\hsize by -\scratchdimen
%          \advance\hsize by -\@@ldafstand
%          \plaatsnaastelkaar{\box\nextbox}\bgroup}
%         {\hsize\scratchdimen
%          \plaatsonderelkaar{\box\nextbox}\bgroup}%
%        \hsize\@@ldbreedte
%        \doif{\@@ldplaats}{\v!rechts}{\hskip\@@ldafstand}%
%        \ifnum\@@ldn>1
%          \setrigidcolumnhsize\hsize\@@ldafstand\@@ldn
%        \fi
%        \dowithnextbox
%          {\doifelse{\@@ldplaats}{\v!rechts}
%             {\vbox to \vsize
%                {\ifnum\@@ldn>1
%                   \rigidcolumnbalance\nextbox
%                 \else
%                   \box\nextbox
%                 \fi
%                 \vfill}}
%             {\vbox
%                {\@@ldtussen
%                 \ifnum\@@ldn>1
%                   \rigidcolumnbalance\nextbox
%                 \else
%                   \box\nextbox
%                 \fi}}%
%           \egroup\egroup}%
%          \vbox
%            \bgroup
%            \forgetall
%            \tolerantTABLEbreaktrue % hm.
%            \blanko[\v!blokkeer]%
%            \everypar{\begstrut}%
%            \let\next=}
%    \hbox}

\newbox\firstpairedbox 
\newbox\secondpairedbox 

\def\definepairedbox%
  {\dodoubleempty\dodefinepairedbox}

\def\dodefinepairedbox[#1][#2]%
  {\getparameters
     [\??ld#1]
     [\c!n=1,
      \c!afstand=\bodyfontsize,
      \c!tussen={\blanko[\v!middel]},
      \c!breedte=\hsize,
      \c!hoogte=\vsize,
      \c!maxbreedte=\zetbreedte,
      \c!maxhoogte=\zethoogte,
      \c!korps=,
      \c!uitlijnen=,
      \c!plaats=\v!onder,
      #2]%
   \setvalue{\e!stel#1\e!in}{\setuppairedbox[#1]}%
   \setvalue{\e!plaats#1}{\placepairedbox[#1]}}

\def\setuppairedbox%
  {\dodoubleempty\dosetuppairedbox}

\def\dosetuppairedbox[#1]%
  {\getparameters[\??ld#1]}

\def\placepairedbox% 
  {\bgroup\dodoubleempty\doplacepairedbox}

\def\doplacepairedbox[#1][#2]% watch the hsize/vsize tricks
  {\setuppairedbox[#1][#2]%     % and don't change them 
   \copyparameters 
     [\??ld][\??ld#1]
     [\c!n,\c!afstand,\c!tussen,
      \c!breedte,\c!hoogte,\c!maxbreedte,\c!maxhoogte,
      \c!korps,\c!uitlijnen,\c!plaats]%
   \beforefirstpairedbox
   \dowithnextbox
     {\betweenbothpairedboxes
      \dowithnextbox
        {\afterbothpairedboxes
         \egroup}
      \vbox\bgroup
        \insidesecondpairedbox
        \let\next=}
   \hbox}

\def\beforefirstpairedbox%
  {\chardef\pairedlocationa=1 % left 
   \chardef\pairedlocationb=4 % middle
   \getfromcommacommand[\@@ldplaats][1]%
   \processaction 
     [\commalistelement]
     [ \v!links=>\chardef\pairedlocationa=0,
      \v!rechts=>\chardef\pairedlocationa=1,
       \v!boven=>\chardef\pairedlocationa=2,
       \v!onder=>\chardef\pairedlocationa=3]%
   \getfromcommacommand[\@@ldplaats][2]%
   \processaction 
     [\commalistelement]
     [ \v!links=>\chardef\pairedlocationb=0,
      \v!rechts=>\chardef\pairedlocationb=1,
        \v!hoog=>\chardef\pairedlocationb=2,
       \v!boven=>\chardef\pairedlocationb=2,
        \v!laag=>\chardef\pairedlocationb=3,
       \v!onder=>\chardef\pairedlocationb=3,
      \v!midden=>\chardef\pairedlocationb=4]}

\def\betweenbothpairedboxes%
  {\switchtobodyfont[\@@ldkorps]% split under same regime  
   \setbox\firstpairedbox=\box\nextbox
   \ifnum\pairedlocationa<2
     \hsize\wd\firstpairedbox % trick 
     \hsize=\@@ldbreedte
     \scratchdimen=\wd\firstpairedbox
     \advance\scratchdimen by \@@ldafstand
     \bgroup\advance\scratchdimen by \hsize
     \ifdim\scratchdimen>\@@ldmaxbreedte\relax
       \egroup
       \hsize=\@@ldmaxbreedte
       \advance\hsize by -\scratchdimen
     \else
       \egroup
     \fi
   \else
     \hsize\wd\firstpairedbox 
     \hsize\@@ldbreedte % can be \hsize
     \ifdim\hsize>\@@ldmaxbreedte\relax \hsize=\@@ldmaxbreedte \fi % can be \hsize
   \fi
   \ifnum\@@ldn>1
     \setrigidcolumnhsize\hsize\@@ldafstand\@@ldn
   \fi}

\def\afterbothpairedboxes%
  {\setbox\secondpairedbox=\vbox
     {\ifnum\@@ldn>1 \rigidcolumnbalance\nextbox \else \box\nextbox \fi}%
   \ifnum\pairedlocationa<2\hbox\else\vbox\fi\bgroup % hide vsize 
   \forgetall
   \ifnum\pairedlocationa<2 
     \scratchdimen=\maxoftwoboxdimens\ht\firstpairedbox\secondpairedbox
     \vsize=\scratchdimen 
     \ifdim\scratchdimen<\@@ldhoogte\relax % can be \vsize
       \scratchdimen=\@@ldhoogte 
     \fi
     \ifdim\scratchdimen>\@@ldmaxhoogte\relax
       \scratchdimen=\@@ldmaxhoogte 
     \fi
     \valignpairedbox\firstpairedbox \scratchdimen
     \valignpairedbox\secondpairedbox\scratchdimen
   \else
     \scratchdimen=\maxoftwoboxdimens\wd\firstpairedbox\secondpairedbox
     \halignpairedbox\firstpairedbox \scratchdimen
     \halignpairedbox\secondpairedbox\scratchdimen
     \scratchdimen=\ht\secondpairedbox
     \vsize=\scratchdimen
     \ifdim\ht\secondpairedbox<\@@ldhoogte\relax % can be \vsize
       \scratchdimen=\@@ldhoogte\relax % \relax needed 
     \fi
     \ifdim\scratchdimen>\@@ldmaxhoogte\relax % todo: totale hoogte
       \scratchdimen=\@@ldmaxhoogte\relax % \relax needed 
     \fi
     \ifdim\scratchdimen>\ht\secondpairedbox
       \setbox\secondpairedbox\vbox to \scratchdimen
         {\ifnum\pairedlocationa=3 \vss\fi
          \box\secondpairedbox
          \ifnum\pairedlocationa=2 \vss\fi}%
     \fi
   \fi
   \ifcase\pairedlocationa
     \box\secondpairedbox\hskip\@@ldafstand\box\firstpairedbox \or
     \box\firstpairedbox \hskip\@@ldafstand\box\secondpairedbox\or
     \box\secondpairedbox\par  \@@ldtussen \box\firstpairedbox \or
     \box\firstpairedbox \par  \@@ldtussen \box\secondpairedbox\else
   \fi
   \egroup}

\def\insidesecondpairedbox%
  {\forgetall
   \steluitlijnenin[\@@lduitlijnen]%
   \tolerantTABLEbreaktrue % hm.
   \blanko[\v!blokkeer]%
   \everypar{\begstrut}}

\def\maxoftwoboxdimens#1#2#3%
  {#1\ifdim#1#2>#1#3 #2\else#3\fi}

\def\valignpairedbox#1#2%
  {\setbox#1=\vbox to #2
     {\ifcase\pairedlocationb\or\or\or\vss\or\vss\fi
      \box#1\relax
      \ifcase\pairedlocationb\or\or\vss\or\or\vss\fi}}

\def\halignpairedbox#1#2%
  {\setbox#1=\hbox to #2
     {\ifcase\pairedlocationb\or\hss\or\or\or\hss\fi
      \box#1\relax
      \ifcase\pairedlocationb\hss\or\or\or\or\hss\fi}}

\definepairedbox[\e!legenda]

\newcount\horcombinatie  % counter
\newcount\totcombinatie

\def\stelcombinatiesin%
  {\dodoubleargument\getparameters[\??co]}

\long\def\dodostartcombinatie[#1*#2*#3]%
  {\stelfractiesin
     [\c!n=\v!passend,
      \c!afstand=\@@coafstand]%
   \global\horcombinatie=#1\relax
   \global\totcombinatie=#2\relax
   \xdef\maxhorcombinatie{\the\horcombinatie}%
   \multiply\totcombinatie by \horcombinatie
   \tabskip=\!!zeropoint
   \doifelse{\@@cobreedte}{\v!passend}
     {\halign}
     {\halign to \@@cobreedte}%
   \bgroup&\hfil##\hfil&\tabskip\!!zeropoint \!!plus 1fill##\cr
   \docombinatie}

% \def\docombinatie%
%   {\dowithnextbox
%      {\setbox0=\box\nextbox
%       \dowithnextbox
%         {\setbox2=\box\nextbox
%          \dodocombinatie}
%      \hbox}
%   \hbox}

\def\docombinatie% we want to add struts but still ignore an empty box
  {\dowithnextbox%
     {\setbox0=\box\nextbox
      \dowithnextbox
        {\setbox2=\box\nextbox
         \dodocombinatie}
      \vtop\bgroup
        \def\next%
          {\futurelet\nexttoken\nextnext}
        \def\nextnext%
          {\ifx\nexttoken\egroup \else % the next box is empty  
             \hsize\wd0
             \steluitlijnenin[\@@couitlijnen]
             \bgroup
             \aftergroup\endstrut
             \aftergroup\egroup
             \begstrut
           \fi}
        \afterassignment\next\let\nexttoken=}
  \hbox}

\def\dodocombinatie%
  {\vbox
     {\forgetall % \stelwitruimtein[\v!geen]%
      \vbox
        {\copy0}%
      \ifdim\ht2>\!!zeropoint\relax % beter dan \wd2, nu \strut mogelijk
        \@@cotussen
       %\vtop
       %  {\nointerlineskip  % recently added
       %   \hsize\wd0
       %   \steluitlijnenin[\@@couitlijnen]%  % \raggedcenter
       %   \begstrut\unhbox2\endstrut}%
        \box2
      \fi}%
   \ifnum\totcombinatie>1
     \global\advance\totcombinatie by -1
     \global\advance\horcombinatie by -1
     \ifnum\horcombinatie=0
       \def\next%
         {\cr\noalign
            {\forgetall %\stelwitruimtein[\v!geen]%
             \nointerlineskip
             \@@cona
             \@@covoor
             \vss
             \nointerlineskip}%
          \global\horcombinatie=\maxhorcombinatie\relax
          \docombinatie}%
     \else
       \def\next%
         {&&&\hskip\@@coafstand&\docombinatie}%
     \fi
   \else
     \def\next%
       {\cr\egroup}%
   \fi
   \next}

\def\complexdostartcombinatie[#1]%
  {\dodostartcombinatie[#1*1*]}

\def\simpledostartcombinatie%
  {\complexdostartcombinatie[2]}

\def\startcombinatie%
  {\bgroup
   \forgetall
   \doifelse{\@@cohoogte}{\v!passend}
     {\vbox}
     {\vbox to \@@cohoogte}%
   \bgroup
   \complexorsimple\dostartcombinatie}

\def\stopcombinatie%
  {\egroup
   \egroup}

\stelcombinatiesin
  [\c!breedte=\v!passend,
   \c!hoogte=\v!passend,
   \c!afstand=1em,
   \c!voor=\blanko,
   \c!tussen={\blanko[\v!middel]},
   \c!na=,
   \c!uitlijnen=\v!midden]

\def\plaatsondernaastelkaar#1#2%
  {\bgroup
   \def\doplaatsondernaastelkaar%
     {#2\cr\omit\bgroup#2%
      \aftergroup#2%
      \aftergroup\cr
      \aftergroup\egroup
      \aftergroup\egroup
      \let\next=}%
   #1\bgroup##\cr
   \omit\bgroup#2%
   \aftergroup\doplaatsondernaastelkaar
   \let\next=}

\def\plaatsonderelkaar%
  {\plaatsondernaastelkaar\halign\hss}

\def\plaatsnaastelkaar%
  {\plaatsondernaastelkaar\valign\vss}

\def\dogebruikexternefiles[#1][#2]%
  {\getparameters
     [\??fi#1]
     [\c!file=,
      \c!korps=,
      \c!optie=,
      #2]}

\def\gebruikexternefiles%
  {\dodoubleargument\dogebruikexternefiles}

\def\dostelexternefilesin[#1][#2]%
  {\doifundefinedelse{\??fi#1\c!file}
     {\gebruikexternefiles[#1][#2]}
     {\getparameters[\??fi#1][#2]}}

\def\stelexternefilesin%
  {\dodoubleargument\dostelexternefilesin}

\def\verwerkexternefile#1#2#3%
  {\bgroup
   \getparameters[\??fi#1][\c!file=,#3]%
   \doinputonce{\getvalue{\??fi#1\c!file}}%
   \ExpandFirstAfter\switchtobodyfont[\getvalue{\??fi#1\c!korps}]%
   \readsysfile{#2}  % beter: loc of fix gebied
     {}
     {\showmessage{\m!systems}{41}{#2,#1}}%
   \egroup}

\def\dogebruikexternefile[#1][#2][#3][#4]%
  {\stelexternefilesin[#1][]%
   \doinputonce{\getvalue{\??fi#1\c!file}}%
   \doifelsenothing{#2}
     {\setvalue{#3}{\verwerkexternefile{#1}{#3}{#4}}}
     {\setvalue{#2}{\verwerkexternefile{#1}{#3}{#4}}}}

\def\gebruikexternefile%
  {\doquadrupleargument\dogebruikexternefile}

\gebruikexternefiles
  [pictex]
  [\c!korps=\v!klein,
   \c!file=pictex]

\gebruikexternefiles
  [table]
  [\c!file=table]

\presetlocalframed[\??ro]

\def\stelroterenin%
  {\dodoubleargument\getparameters[\??ro]}

% \ht, \vfillvoor, \vfillna, \wd, \hfillvoor, \hfillna

\def\dodostoproteer#1#2#3#4#5#6%
  {\dontshowcomposition
   \vbox to #1\nextbox
     {#2\relax
      \hbox to #4\nextbox
        {#5\relax % \number removes leading spaces too 
         \edef\@@rorotatie{\number\@@rorotatie}%
         \doifelsenothing{\@@rorotatie}
           {\dostartrotation{90}}
           {\dostartrotation{\@@rorotatie}}% 
         \wd\nextbox=\!!zeropoint
         \ht\nextbox=\!!zeropoint
         \box\nextbox
         \dostoprotation
         #6}
      #3}%
   \egroup}


\def\dostoproteer%
  {\!!counta=\@@rorotatie
   \divide\!!counta by 90
   \ifcase\!!counta
     \dodostoproteer\ht\relax\vfill\wd\relax\hfill
   \or
     \dodostoproteer\wd\vfill\relax\ht\relax\hfill
   \or
     \dodostoproteer\ht\vfill\relax\wd\hfill\relax
   \or
     \dodostoproteer\wd\relax\vfill\ht\hfill\relax
   \or
     \dodostoproteer\ht\relax\vfill\wd\relax\hfill
   \else
     \def\@@rotatie{90}%
     \dodostoproteer\ht\relax\vfill\wd\relax\hfill
   \fi}

\def\dorotatebox#1% {angle} \hbox/\vbox/\vtop
  {\bgroup
   \hbox\bgroup % compatibility hack
   \dowithnextbox
     {\edef\@@rorotatie{#1}%
      \setbox\nextbox=\vbox{\box\nextbox}%
      \dostoproteer
      \egroup}}

\def\complexroteer[#1]%
  {\dowithnextbox
     {\getparameters[\??ro][#1]%
      \dostoproteer}%
   \vbox\localframed[\??ro][#1]}

\def\roteer%
  {\bgroup     % \roteer kan argument zijn
   \complexorsimpleempty\roteer}

\stelroterenin
  [\c!rotatie=90,
   \c!breedte=\v!passend,
   \c!hoogte=\v!passend,
   \c!offset=\v!overlay,
   \c!kader=\v!uit]

% schaal

\def\doscalelikeafigure%
  {\doifsomething{\@@xyfactor\@@xyschaal\@@xyhfactor\@@xybreedte\@@xyhoogte}
     {\let \@@efschaal \@@xyschaal
      \let \@@effactor \@@xyfactor
      \let \@@efbfactor\@@xybfactor 
      \let \@@efhfactor\@@xyhfactor
      \let \@@efbreedte\@@xybreedte
      \let \@@efhoogte \@@xyhoogte
      \let \@@epx      \!!zeropoint
      \let \@@epy      \!!zeropoint   
      \edef\@@epw     {\the\wd\nextbox}%
      \edef\@@eph     {\the\ht\nextbox}%
      \setfactorfiguresize
      \setscalefiguresize
      \setdimensionfiguresize
      \convertfigureinsertscale\@@epx\figx\figxsca\scax
      \convertfigureinsertscale\@@epy\figy\figysca\scay
      \scratchdimen=\scax pt \divide\scratchdimen by 100 
      \edef\@@xysx{\withoutpt\the\scratchdimen}%
      \scratchdimen=\scay pt \divide\scratchdimen by 100 
      \edef\@@xysy{\withoutpt\the\scratchdimen}}}   

\def\doschaal[#1]%
  {\bgroup
   \forgetall
   \getparameters
     [\??xy]
     [\c!schaal=,\c!breedte=,\c!hoogte=,
      \c!factor=,\c!hfactor=,\c!bfactor=,
      \c!sx=1,\c!sy=1,#1]%
   \dowithnextbox
     {\dontshowcomposition
      \doscalelikeafigure
      \dimen0=\@@xysy\ht\nextbox
      \dimen2=\@@xysy\dp\nextbox
      \dimen4=\@@xysx\wd\nextbox
      \dimen6=\dimen0\advance\dimen6 by \dimen2 
      \setbox\nextbox=\vbox to \dimen6
        {\ht\nextbox=\!!zeropoint
         \dp\nextbox=\!!zeropoint
         \vfill % erbij
         \dostartscaling\@@xysx\@@xysy\box\nextbox\dostopscaling}%
      \ht\nextbox=\dimen0
      \dp\nextbox=\dimen2
      \wd\nextbox=\dimen4
      \box\nextbox
      \egroup}
   \hbox}

\def\schaal%
  {\dosingleempty\doschaal}

% mirror

\def\domirrorbox% \hbox/\vbox/\vtop
  {\bgroup
   \dowithnextbox
     {\dontshowcomposition
      \dimen0=\wd\nextbox
      \setbox\nextbox=\vbox
        {\dostartmirroring\hskip-\wd\nextbox\box\nextbox\dostopmirroring}%
      \wd\nextbox=\dimen0
      \box\nextbox
      \egroup}}

\def\spiegel%
  {\domirrorbox\hbox}

%\setbox0=\hbox{gans}
%
%\ruledhbox{\copy0 \schaal[sx=2,sy=2]{\copy0}}
%
%\spiegel{\ruledhbox{\copy0 \schaal{\box0}}}

% verdelen \hsize in fracties, wordt nog wat algemener, 
% beetje vaag nu 
%
% \fractie[n/m,elementen,afstand]
%
% \fractie[2/5,3,1em]
% \fractie[2/5,3,1em]
% \fractie[1/5,3,1em]
%
% \stelfractiesin[afstand=,aantal=]  (passend,passend)

\def\??fr{@@fr}

\def\stelfractiesin%
  {\dodoubleargument\getparameters[\??fr]}

\def\dodofractie[#1/#2,#3,#4,#5]%
  {\doifelsenothing{#3}
     {\doifelse{\@@frn}{\v!passend}
        {\!!counta=#2\relax}
        {\!!counta=\@@frn\relax}}
     {\!!counta=#3\relax}%
   \doifelsenothing{#4}
     {\doifelse{\@@frafstand}{\v!passend}
        {\!!widtha=\!!zeropoint}
        {\!!widtha=\@@frafstand}}
     {\!!widtha=#4}%
   \advance\!!counta by -1\relax
   \multiply\!!widtha by \!!counta
   \advance\hsize by -\!!widtha
   \divide\hsize by #2\relax
   \hsize=#1\hsize}

\def\dofractie[#1]%
  {\dodofractie[#1,,,,,,]}

\def\fractie%
  {\dosingleargument\dofractie}

\stelfractiesin
  [\c!afstand=\tfskipsize,
   \c!n=\v!passend]

\protect \endinput
