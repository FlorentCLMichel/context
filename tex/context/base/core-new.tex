%D \module
%D   [       file=core-nav,
%D        version=1995.1.1,
%D          title=\CONTEXT\ Core Macros,
%D       subtitle=New ones,
%D         author=Hans Hagen,
%D           date=\currentdate,
%D      copyright={PRAGMA / Hans Hagen \& Ton Otten}]
%C
%C This module is part of the \CONTEXT\ macro||package and is
%C therefore copyrighted by \PRAGMA. See mreadme.pdf for 
%C details. 

\writestatus{loading}{Context Core Macros / New Ones}

\unprotect

%D To be placed elsewhere.

% Some kind of new feature, for the moment a private one.
%
% \enablemode[screen,paper,bound]
%
% \doifmodeelse {paper}        {this} {that}
% \doifmode     {paper,screen} {this}
% \doifnotmode  {paper,bound}  {that} 
%
% \startmode [list]
% \stopmode
%
% \startnotmode [list]
% \stopnotmode
%
% system modes have a * as prefix 
%
% to be implemented: mode naar texutil/scherm + message  
%                  : geen #2 en nesting (\startregels)

\def\systemmodeprefix{*}

\let\currentmode=\empty

% \def\enablemode[#1]%
%   {\expandafter\addtocommalist\expandafter{#1}\currentmode}
%
% \def\disablemode[#1]%
%   {\expandafter\removefromcommalist\expandafter{#1}\currentmode}
%
% this one fails in removing system modes 

% \def\enablemode[#1]%
%   {\expanded{\addtocommalist{#1}\noexpand\currentmode}}

%D Sometimes, we want to prevent a mode for being set. Think
%D of situations where a style enables a mode, but an outer
%D level style does not want that. Preventing can be
%D considered a permanent disabling on forehand. 

\let\preventedmodes\empty

\def\preventmode[#1]%
  {\expanded{\addtocommalist{#1}\noexpand\preventedmodes}}

\def\enablemode[#1]%
  {\expanded
     {\doifnotinset{#1}{\preventedmodes}
        {\noexpand\addtocommalist{#1}\noexpand\currentmode}}}

\def\disablemode[#1]%
  {\expanded{\removefromcommalist{#1}\noexpand\currentmode}}

\def\doifmodeelse{\unprotect\dodoifmodeelse}
\def\doifmode    {\unprotect\dodoifmode}
\def\doifnotmode {\unprotect\dodoifnotmode}
\def\startmode   {\unprotect\dostartmode}
\def\startnotmode{\unprotect\dostartnotmode}

\long\def\dodoifmodeelse#1#2#3%
  {\protect\ExpandBothAfter\doifcommonelse{#1}{\currentmode}{#2}{#3}}

\long\def\dodoifmode#1#2%
  {\protect\ExpandBothAfter\doifcommonelse{#1}{\currentmode}{#2}{}}

\long\def\dodoifnotmode#1#2%
  {\protect\ExpandBothAfter\doifcommonelse{#1}{\currentmode}{}{#2}}

\long\def\dostartmode[#1]#2\stopmode%   
  {\dodoifmode{#1}{#2}}

\long\def\dostartnotmode[#1]#2\stopnotmode%
  {\dodoifnotmode{#1}{#2}}

% new 
%
% \startnointerference
% all kind of code 
% \stopnointerference

\newbox\nointerferencebox

\def\startnointerference%
  {\setbox\nointerferencebox\vbox
   \bgroup}

\def\stopnointerference%
  {\egroup
   \setbox\nointerferencebox\box\voidb@x}

% new 

\def\doifallmodeselse{\unprotect\dodoifallmodeselse}
\def\doifallmodes    {\unprotect\dodoifallmodes}
\def\doifnotallmodes {\unprotect\dodoifnotallmodes}
\def\startallmodes   {\unprotect\dostartallmodes}
\def\startnotallmodes{\unprotect\dostartnotallmodes}

\long\def\dodoifallmodeselse#1#2#3%
  {\protect\ExpandBothAfter\doifallcommonelse{#1}{\currentmode}{#2}{#3}}

\long\def\dodoifallmodes#1#2%
  {\protect\ExpandBothAfter\doifallcommonelse{#1}{\currentmode}{#2}{}}

\long\def\dodoifnotallmodes#1#2%
  {\protect\ExpandBothAfter\doifallcommonelse{#1}{\currentmode}{}{#2}}

\long\def\dostartallmodes[#1]#2\stopallmodes
  {\dodoifallmodes{#1}{#2}}

\long\def\dostartnotallmodes[#1]#2\stopnotallmodes
  {\dodoifnotallmodes{#1}{#2}}

% will go to ... 

\def\alignedbox%
  {\dodoubleempty\doalignedbox[]}

\def\doalignedbox[#1][#2]%
  {\bgroup
   \let\iftraceboxplacement\iftracelayers
   \dowithnextbox
     {\let\next\middlebox
      \processaction
        [#2]
        [ t=>\let\next\topbox       , b=>\let\next\bottombox     ,
          l=>\let\next\leftbox      , r=>\let\next\rightbox      ,
         bl=>\let\next\bottomleftbox,br=>\let\next\bottomrightbox,
         tl=>\let\next\topleftbox   ,tr=>\let\next\toprightbox   ,
         lt=>\let\next\lefttopbox   ,lb=>\let\next\leftbottombox ,
         rt=>\let\next\righttopbox  ,rb=>\let\next\rightbottombox]%
      \next{\box\nextbox}%
      \egroup}#1}

\def\offsetbox%
  {\dodoubleempty\dooffsetbox[]}

\def\dooffsetbox[#1][#2]%
  {\bgroup
   \dowithnextbox
     {\getparameters[\??ox]
        [\c!x=\!!zeropoint,
         \c!y=\!!zeropoint,
         \c!breedte=\wd\nextbox,
         \c!hoogte=\wd\nextbox,
         \c!diepte=\dp\nextbox,
         \c!plaats=,
         #2]%
      \freezedimenmacro\@@oxbreedte
      \freezedimenmacro\@@oxhoogte
      \freezedimenmacro\@@oxdiepte
      \setbox\nextbox\hbox
        {\hskip\@@oxx\lower\@@oxy\hbox
           {\doifelsenothing{\@@oxplaats}
              {\box\nextbox}
              {\alignedbox[\@@oxplaats]\hbox{\box\nextbox}}}}%
      \wd\nextbox\@@oxbreedte
      \ht\nextbox\@@oxhoogte 
      \dp\nextbox\@@oxdiepte 
      \box\nextbox
      \egroup}#1}

\def\offset {\dodoubleempty\dooffsetbox [\hbox]} % yes or no
\def\aligned{\dosingleempty\doalignedbox[\hbox]} % yes or no

%\ruledhbox{\offsetbox[x=-1cm,y=-1cm,location=c]
%  {\framed[width=4cm,height=4cm]{x}}}

% actually this is pretty old, but temporary moved here 

\installdiscretionaries || \@@kpteken

\newsignal\subsentencesignal
\newcounter\subsentencelevel
\def\subsentenceskip{.25em\relax}

\def\stelkoppeltekenin%
  {\dodoubleargument\getparameters[\??kp]}

\def\beginofsubsentence%
  {\ifdim\lastkern=\subsentencesignal
     \unskip
     \kern\subsentenceskip
   \fi
   \doglobal\increment\subsentencelevel
   \ifnum\subsentencelevel=1
     \leaveoutervmode
   \fi
   \ifodd\subsentencelevel
     \getvalue{\??la\currentlanguage\c!leftsentence}%
   \else
     \getvalue{\??la\currentlanguage\c!leftsubsentence}%
   \fi
   \ignorespaces}

\def\beginofsubsentencespacing%
  {\kern\subsentencesignal\ignorespaces}

\def\endofsubsentence%
  {\ifodd\subsentencelevel
     \getvalue{\??la\currentlanguage\c!rightsentence}%
   \else
     \getvalue{\??la\currentlanguage\c!rightsubsentence}%
   \fi
   \doglobal\decrement\subsentencelevel
   \unskip
   \kern\subsentencesignal}

\def\endofsubsentencespacing%
  {%\ifdim\lastkern=\subsentencesignal \else
   %  \unskip
   %\fi
   \ifdim\lastkern=\subsentencesignal
     \unskip
     \hskip\subsentenceskip
     \ignorespaces
   \else
     \unskip
   \fi}

% test |<|test |<|test|>| test|>| test \par
% test|<|test|<|test|>|test|>|test     \par
% test |<||<|test|>||>| test           \par

\enableactivediscretionaries

%D new and beta 

% \def\defineshortcut%
%   {\dodoubleargument\dodefineshortcut}
% 
% \bgroup
% 
%   \catcode`\<=\@@active 
% 
%   \gdef\dodefineshortcut[#1][#2]%
%     {\ifsecondargument
%        \catcode`\<=\@@active 
%        \def<{\ifmmode\expandafter\normalless\else\expandafter\doshortcut\fi}%
%        \getparameters[\??te#1][\c!commandos=,\c!commando=,\c!letter=,\c!kleur=,#2]%
%      \else
%        \defineshortcut[][#1]%   
%      \fi}
% 
% \egroup
% 
% \def\doshortcut% 
%   {\bgroup
%    \catcode`\>=\@@other
%    \dodoshortcut}
% 
% \def\dodoshortcut#1>% 
%   {\def\shortcut{#1}%
%    \dododoshortcut#1:\end}
% 
% \def\dododoshortcut#1:#2\end 
%   {\doifelsenothing{#2}
%      {\doifundefinedelse{\??te\c!commandos}
%         {\shortcut} 
%         {\@EA\dodododoshortcut\@EA\??te\@EA:\shortcut:\end}}
%      {\doifundefinedelse{\??te#1\c!commandos}
%         {\shortcut} 
%         {\dodododoshortcut\??te#1:#2\end}}%
%    \egroup}
% 
% \def\dodododoshortcut#1:#2:\end
%   {\getvalue{#1\c!commandos}%
%    \doattributes{#1}\c!letter\c!kleur{\getvalue{#1\c!commando}{#2}}}

\def\defineshortcut%
  {\dotripleargument\dodefineshortcut}

\def\dodefineshortcut[#1][#2][#3]%
  {\ifthirdargument
     \ConvertConstantAfter\doifelse{#1}{}
       {\dododefineshortcut[<>][#2][#3]}
       {\dododefineshortcut[#1][#2][#3]}%
   \else\ifsecondargument
     \dododefineshortcut[<>][#1][#2]%   
   \else
     \dododefineshortcut[<>][][#1]%   
   \fi\fi}

\def\dododefineshortcut[#1#2][#3][#4]% #1 is the trigger, #2 the delimiter/tag
  {\doifundefined{\??te\??te\string#2}{\letvalue{\??te\??te\string#2}=#1}%
   \defineactivecharacter #1 {\@EA\doshortcut\string#2} %
   \getparameters
     [\??te\string#2#3]
     [\c!commandos=,\c!commando=,\c!letter=,\c!kleur=,#4]}

\def\doshortcut#1% 
  {\ifmmode
     \getvalue{\??te\??te#1}%
   \else
     \bgroup
     \catcode`#1=\@@other
     \def\dodoshortcut##1#1% 
       {\def\shorttag{\??te#1}%
        \def\shortcut{##1}%
        \dododoshortcut##1:\end}%
     \@EA\dodoshortcut
   \fi}
 
\def\dododoshortcut#1:#2\end 
  {\doifelsenothing{#2}
     {\doifundefinedelse{\shorttag\c!commandos}
        {\shortcut} 
        {\@EA\dodododoshortcut\@EA\shorttag\@EA:\shortcut:\end}}
     {\doifundefinedelse{\shorttag#1\c!commandos}
        {\shortcut} 
        {\dodododoshortcut\shorttag#1:#2\end}}%
   \egroup}

\def\dodododoshortcut#1:#2:\end
  {\getvalue{#1\c!commandos}%
   \doattributes{#1}\c!letter\c!kleur{\getvalue{#1\c!commando}{#2}}}

%D \defineshortcut     [\c!letter=\v!type]
%D \defineshortcut [b] [\c!letter=\v!vet]
%D \defineshortcut [e] [\c!letter=\em] 
%D \defineshortcut [t] [\c!letter=\v!type]
%D \defineshortcut [c] [\c!letter=\v!kap]
%D \defineshortcut [k] [\c!letter=\v!kap]
%D \defineshortcut [u] [\c!letter=\v!type,\c!commando=\hyphenatedurl]
%D 
%D \startregels
%D test <ziezo> test 
%D test test <t:ziezo> 
%D test test <b:ziezo> 
%D test test <w:ziezo> 
%D zus<>zo zus<:>zo zus<::>zo 
%D test test <t:ziezo> dat (ziezo)          
%D test test <t::ziezo> dat (:ziezo)        
%D test test <t:ziezo:> dat (ziezo:)        
%D test test <t:zi:ezo:> dat (zi:ezo:)      
%D well, <u:http://www.pragma-ade.nl> looks fuzzy 
%D $10<20$
%D \stopregels
%D 
%D \defineshortcut [<>] [i] [\c!letter=\it]
%D \defineshortcut [()] [b] [\c!letter=\bf]
%D \defineshortcut [++] [s] [\c!letter=\sl]
%D \defineshortcut [//] [u] [\c!letter=\underbars]
%D \defineshortcut [--] [a] [\c!letter=\overstrike]
%D 
%D \startregels
%D it seems <i:to work> well 
%D it seems (b:to work) well 
%D it seems +s:to work+ well 
%D it seems /u:to work/ well 
%D it seems -a:to work- well 
%D \stopregels

\protect 

\endinput
