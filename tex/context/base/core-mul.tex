%D \module
%D   [       file=core-mul,
%D        version=1998.03.15,
%D          title=\CONTEXT\ Core Macros,
%D       subtitle=Multi Column Output,
%D         author=Hans Hagen,
%D           date=\currentdate,
%D      copyright={PRAGMA / Hans Hagen \& Ton Otten}]
%C
%C This module is part of the \CONTEXT\ macro||package and is
%C therefore copyrighted by \PRAGMA. See mreadme.pdf for
%C details.

\writestatus{loading}{Context Core Macros / Multi Column Output}

\unprotect

% check \count<insert> multiplications  

% some day try this in balancing routine 
%
% \ifdim\pagetotal>\pagegoal
%   \eject
% \else
%   \goodbreak
% \fi
 
%D The following macro's implement a multi||column output
%D routine. The original implementation was based on Donald
%D Knuth's implementation, which was adapted by Craig Platt to
%D support balancing of the last page. I gradually adapted
%D Platt's version to our needs but under certain
%D circumstances things still went wrong. I considered all
%D calls to Platt's \type{\balancingerror} as undesirable.

\startmessages  dutch  library: columns
  title: kolommen
      1: maximaal -- kolommen
      2: gebruik eventueel \string\filbreak
      3: probleempje, probeer [balanceren=nee]
      4: plaatsblok boven nog niet mogelijk
      5: plaatsblok onder nog niet mogelijk
      6: -- plaatsblok(en) opgeschort
      7: balanceren afgebroken na 100 stappen
      8: gebalanceerd in -- stap(pen)
      9: uitlijnen controleren!
     10: (minder dan) 1 regel over
     11: plaatsblok te breed voor kolom
     12: plaatsblok verplaatst naar volgende kolom / --
     13: breed figuur geplaatst boven kolommen
\stopmessages

\startmessages  english  library: columns
  title: columns
      1: only -- columns possible
      2: use \string\filbreak\space as alternative
      3: problems, disable balancing
      4: top float not yet supported
      5: bottom float not yet supported
      6: -- float(s) postponed
      7: balancing aborted after 100 steps
      8: balanced in -- step(s)
      9: check raggedness
     10: (less than) 1 line left
     11: float to wide for column
     12: float moved to next column / --
     13: wide float moved to top of columns
\stopmessages

\startmessages  german  library: columns
  title: Spalten
      1: nur -- Spalten moeglich
      2: benutzte \string\filbreak\space als Alternative
      3: Problem, verwende [ausgleich=nein]
      4: Gleitobjekt oben ncoh nicht unterstuetzt
      5: Gleitobjekt unten ncoh nicht unterstuetzt
      6: -- Gleitobjekt(e) verschoben
      7: ausgleich nach 100 Schritten abgebrocheb
      8: ausgeglichen nach  -- Schritt(en)
      9: Ausrichtung ueberpruefen
     10: (weniger als) 1 Zeile uebrig
     11: Gleitobjekt zu breit fuer Spalte
     12: Gleitobjekt in naechste Zeile verschoben / --
     13: breites Gleitobjekt an den Anfang der Spalten verschoben
\stopmessages

\startmessages  czech  library: columns
  title: sloupce
      1: je mozno pouze -- sloupcu
      2: pouzijte \string\filbreak\space jako alternativu
      3: problem, vypina se vyvazovani
      4: horni plovouci objekt jeste neni podporovan
      5: spodni plovouci objekt jeste neni podporovan
      6: -- plovouci objekt(y) odlozeny
      7: vyvazovani ukonceno po 100 krocich
      8: vyvazeno v -- krocich
      9: kontrola nerovnost
     10: zbyl (mene nez) 1 radek
     11: plovouci objekt je pro sloupec prilis siroky
     12: plovouci objekt je presunut do nasledujiciho sloupce / -- 
     13: siroky plovouci objekt je presunut nad sloupce
\stopmessages

\startmessages  italian  library: columns
  title: colonne
      1: solo -- colonne possibili
      2: in alternativa, usare \string\filbreak
      3: problemi, disabilitare il bilanciamento
      4: float in cima non ancora supportato
      5: float in fondo non ancora supportato
      6: -- float(s) posticipate
      7: bilanciamento annullato dopo 100 passi
      8: bilanciamento in -- passo/i
      9: controllare seghettamento
     10: (meno di) una riga rimasta
     11: oggetto mobile troppo ampio per la colonna
     12: oggetto mobile spostata alla colonna successiva / --
     13: oggetto mobile ampio spostato sopra le colonne
\stopmessages

\startmessages  norwegian  library: columns
  title: kolonner
      1: maksimalt -- kolonner
      2: bruk \string\filbreak\space som et alternativ
      3: problemer, sl�r av balansering
      4: flytblokker �verst er ikke st�ttet enda
      5: flytblokker nedert er ikke st�ttet enda
      6: -- flytblokk forskj�vet
      7: balansering avbrutt etter 100 iterasjoner
      8: balansert etter -- iterasjoner
      9: kontroller tekstlayout!
     10: (mindre enn) 1 linje igjen
     11: flytblokk for bredt for kolonna
     12: flytblokk forskj�vet til neste kolonne / --
     13: bred flytblokk forksj�vet til toppen av kolonnene
\stopmessages

%D This completely new implementation can handle enough
%D situations for everyday documents, but is still far from
%D perfect. While at the moment the routine doesn't support
%D all kind of floats, it does support:
%D
%D \startopsomming[opelkaar]
%D \som  an unlimitted number of columns
%D \som  ragged or not ragged bottoms
%D \som  optional balancing without \type{\balancingerrors}
%D \som  different \type{\baselineskips}, \type{\spacing},
%D       \type{\topskip} and \type{\maxdepth}
%D \som  left- and right indentation, e.g. within lists
%D \som  moving columns floats to the next column or page
%D \som  handling of floats that are to wide for a columns
%D \stopopsomming
%D
%D One could wonder why single and multi||columns modes are
%D still separated. One reason for this is that \TeX\ is not
%D suited well for handling multi||columns. As a result, the
%D single columns routines are more robust. Handling one
%D column as a special case of multi||columns is posible but at
%D the cost of worse float handling, worse page breaking,
%D worse etc. Complicated multi||column page handling should
%D be done in \kap{DTP}||systems anyway.
%D
%D There are three commands provided for entering and leaving
%D multi||column mode and for going to the next column:
%D
%D \interface \type{\beginmulticolumns} \\ \\
%D \interface \type{\endmulticolumns}   \\ \\
%D \interface \type{\ejectcolumn}       \\ \\
%D
%D This routines are sort of stand||alone. They communicate
%D with the rest of \CONTEXT\ by means of some interface
%D macro's, which we only mention.
%D
%D \interface \type{\nofcolumns} \\
%D   the number of columns \\
%D \interface \type{\minbalancetoplines} \\
%D   the minimum number op balanced top lines \\ 
%D \interface \type{\betweencolumns} \\
%D   the stuff between columns \\
%D \interface \type{\finaloutput{action}{box}} \\
%D   some kind of \type{\pagebody} and \type{\shipout} \\
%D
%D \interface \type{\ifbalancecolumns} \\
%D   balancing the colums or not \\
%D \interface \type{\ifstretchcolumns} \\
%D   ragging the bottom or not \\
%D
%D \interface \type{\ifheightencolumns} \\
%D     fix the heigh tor not \\
%D \interface \type{\fixedcolumnheight} \\
%D     the optional fixed height \\
%D
%D \interface \type{\ifinheritcolumns} \\
%D     handle ragging or not \\
%D \interface \type{\ifr@ggedbottom} \\
%D     use ragged bottoms \\
%D \interface \type{\ifb@selinebottom} \\
%D     put the bottom line on the baseline \\
%D \interface \type{\ifnormalbottom} \\
%D     put the bottom line at the baseline \\
%D
%D \interface \type{\ifreversecolumns} \\
%D     reverse the order in wich columns are flushed \\
%D
%D \interface \type{\usercolumnwidth} \\
%D     the calculated width of a column \\
%D \interface \type{\columntextwidth} \\
%D     the maximum width of a column \\
%D \interface \type{\columntextheight} \\
%D     the minimum width of a column \\
%D
%D \interface \type{\spacingfactor} \\
%D     the spacing factor \\
%D \interface \type{\bodyfontsize} \\
%D     the (local) bodyfontsize \\
%D \interface \type{\openlineheight} \\
%D     the lineheight (including \type{\spacing}) \\
%D
%D \interface \type{\Everybodyfont} \\
%D     communication channel to font switching routines \\
%D
%D \interface \type{\global\settopskip} \\
%D   set \type{\topskip} \\
%D \interface \type{\setcolumnwarnings} \\
%D   set \type{\badness} and \type{\fuzz} \\
%D \interface \type{\setcolumninserts} \\
%D   set \type{\insert}'s \\
%D \interface \type{\setvsize} \\
%D   set \type{\vsize} and \type{\pagegoal} \\
%D \interface \type{\sethsize} \\
%D   set \type{\hsize} \\
%D
%D \interface \type{\flushcolumnfloats} \\
%D   push saved column floats (next page) \\
%D \interface \type{\flushcolumnfloat} \\
%D   push saved column floats (next column) \\
%D \interface \type{\setcolumnfloats} \\
%D   initialize column floats \\
%D
%D \interface \type{\finishcolumnbox} \\
%D   do something special (a hook) \\
%D \interface \type{\postprocesscolumnpagebox} \\
%D   do something with each columnbox (also a hook) \\
%D \interface \type{\postprocesscolumnbox} \\
%D   do something with each columnbox (also a hook) \\
%D \interface \type{\postprocesscolumnline} \\
%D   do something with each columnline (also a hook) \\
%D \interface \type{\currentcolumn} \\
%D   the current column \\
%D
%D These interface macro's are called upon or initialized
%D by the multi||column macro's.

%D A lot of footnote stuff added!

\def\finalcolumntextwidth   {\zetbreedte}
\def\finalcolumntextheight  {\teksthoogte}
\def\columntextwidth        {\zetbreedte}
\def\columntextheight       {\teksthoogte}
\def\usercolumnwidth        {\tekstbreedte}
\def\columntextoffset       {\!!zeropoint}

\def\fixedcolumnheight      {\teksthoogte}
\def\betweencolumns         {\hskip\bodyfontsize}

\def\setcolumnwarnings      {\dontcomplaincolumnboxes}
\def\setcolumninserts       {\dontpermitcolumninserts}

\let\setcolumnfloats        \relax % in CONTEXT used for floats
\let\flushcolumnfloats      \relax % in CONTEXT used for floats
\let\flushcolumnfloat       \relax % in CONTEXT used for floats
\let\finishcolumnbox        \relax % in CONTEXT used for backgrounds

% %D In fact, the column height and width are set by means of
% %D two macro's. One can change their meaning if needed:
%
% \def\setcolumntextheight%
%   {\def\columntextheight{\teksthoogte}}
%
% \def\setcolumntextwidth%
%   {\def\columntextwidth{\zetbreedte}}

%D Both macros are redefined in \CONTEXT\ when backgrounds
%D are applied to columns. The final values are used when
%D flushing the columns.

\newcount\nofcolumns       \nofcolumns=2

\def\maxnofcolumns        {16}
\def\allocatednofcolumns  {0}
\def\minbalancetoplines   {1}

\newif\ifbalancecolumns    \balancecolumnsfalse
\newif\ifstretchcolumns    \stretchcolumnsfalse
\newif\ifinheritcolumns    \inheritcolumnsfalse
\newif\ifheightencolumns   \heightencolumnsfalse

\newbox\partialpage
\newbox\restofpage

\newbox\savedfloatlist

\newdimen\intercolumnwidth
\newdimen\localcolumnwidth
\newdimen\partialpageheight
\newdimen\savedpagetotal

\newtoks\singlecolumnout

%D The next dimension reports the final column height 

\newdimen\finalcolumnheights 
\newcount\finalcolumnlines

%D It's more convenient to use \type {\columnwidth} instead 
%D of messing around with boxes each time.

\newdimen\columnwidth
\newdimen\gutterwidth

\def\determinecolumnwidth% 
  {\bgroup
   \setbox\scratchbox=\hbox
     {\setcolumnhsize
      \global\columnwidth=\usercolumnwidth
      \global\gutterwidth=\intercolumnwidth}%
   \egroup}

%D During initialization the temporary boxes are allocated.
%D This enables us to use as much columns as we want, without
%D exhausting the pool of boxes too fast. We could have packed
%D them in one box, but we've got enough boxes.
%D
%D Two sets of boxes are declared, the txtboxes are used for
%D the text, the topboxes are for moved column floats.

\def\@@txtcol{@@txtcol}
\def\@@topcol{@@topcol}

\def\initializemulticolumns#1%
  {\ifnum#1>\maxnofcolumns\relax
     \showmessage{\m!columns}{1}{\maxnofcolumns}%
     \nofcolumns=\maxnofcolumns
   \else
     \nofcolumns=#1\relax
   \fi
   \ifnum\nofcolumns>\allocatednofcolumns\relax
     \dorecurse
       {#1}
       {\ifnum\recurselevel>\allocatednofcolumns\relax
          \newbox\next
          \global\letvalue{\@@txtcol\recurselevel}=\next
          \newbox\next
          \global\letvalue{\@@topcol\recurselevel}=\next
        \fi}%
     \xdef\allocatednofcolumns{\the\nofcolumns}%
   \fi
   \edef\firstcolumnbox{\getvalue{\@@txtcol1}}%
   \edef\firsttopcolumnbox{\getvalue{\@@topcol1}}%
   \edef\lastcolumnbox{\getvalue{\@@txtcol\the\nofcolumns}}%
   \edef\lasttopcolumnbox{\getvalue{\@@topcol\the\nofcolumns}}}

%D Without going in details we present two macro's which
%D handle the columns. The action which is transfered by the
%D the first and only parameter can do something with
%D \type{\currentcolumnbox}. In case of the mid columns,
%D \type{\firstcolumnbox} and \type{\lastcolumnbox} are handled
%D outside these macro's.

%\def\dohandlemidcolumns#1%
%  {\dorecurse
%     {\nofcolumns}
%     {\ifnum\recurselevel>1
%        \ifnum\recurselevel<\nofcolumns\relax
%           \edef\currentcolumnbox{\getvalue{\@@txtcol\recurselevel}}%
%           \edef\currenttopcolumnbox{\getvalue{\@@topcol\recurselevel}}%
%           \let\currentcolumn=\recurselevel
%           #1\relax
%        \fi
%      \fi}}
%
%\def\dohandleallcolumns#1%
%  {\dorecurse
%     {\nofcolumns}
%     {\edef\currentcolumnbox{\getvalue{\@@txtcol\recurselevel}}%
%      \edef\currenttopcolumnbox{\getvalue{\@@topcol\recurselevel}}%
%      \let\currentcolumn=\recurselevel
%      #1\relax}}

\def\dohandlecolumn#1%
  {\edef\currentcolumnbox   {\getvalue{\@@txtcol\recurselevel}}%
   \edef\currenttopcolumnbox{\getvalue{\@@topcol\recurselevel}}%
   \let\currentcolumn=\recurselevel
   #1\relax}

\def\dohandleallcolumns#1%
  {\dorecurse{\nofcolumns}{\dohandlecolumn{#1}}}

\def\dohandlerevcolumns#1%
  {\dostepwiserecurse{\nofcolumns}{1}{-1}{\dohandlecolumn{#1}}}

\def\dohandlemidcolumns#1%
  {\dohandleallcolumns
     {\ifnum\recurselevel>1\relax\ifnum\recurselevel<\nofcolumns\relax
        \dohandlecolumn{#1}%
      \fi\fi}}

%D Going to a new columns is done by means of a
%D \type{\ejectcolumn}. The following definition does not
%D always work.

\def\ejectcolumn%
  {\goodbreak
   \showmessage{\m!columns}{2}{}}

%D The next macro should never be called so let's deal with it.
%D There were several solutions to these kind of errors. First
%D we check for a good breakpoint before firing up the
%D multi||column routine (\type{\break} or \type{\allowbreak}).
%D We do the same at the end of the routine
%D (\type{\allowbreak}). These allowances are definitely
%D needed!
%D
%D Some on first sight redundant calls to for instance
%D \type{\setvsize} in the flushing, splitting and balancing
%D macro's can definitely not be omitted! Some are just there
%D to handle situations that only few times arise. One of
%D those can be that the output routine is invoked before
%D everything is taken care of. This happens when we
%D flush (part of) the current page with an \type{\unvbox}
%D with a \type{\pagetotal}~$\approx$ \type{\pagegoal}. One
%D simply cannot balance columns that are just balanced.
%D
%D I hope one never sees the following message. Because it
%D took me a lot of time to develop the multi||columns
%D routines, every (although seldom) warning gives me the
%D creeps!

\def\balancingerror%
  {\showmessage{\m!columns}{3}{}%
   \finaloutput\unvbox\normalpagebox}

%D Here we present the two \type{\dont...} macro's, which are
%D of course \CONTEXT||specific ones.

\def\dontcomplaincolumnboxes%
  {\mindermeldingen}

\def\dontpermitcolumninserts%
  {\def\dotopfloat%
     {\showmessage{\m!columns}{4}{}%
      \doexecfloat}%
   \def\dobotfloat%
     {\showmessage{\m!columns}{5}{}%
      \doexecfloat}}

\def\getinsertionheights\to#1\\% \relax'm
  {#1=\!!zeropoint
   \def\doaddinsertionheight##1%
     {\ifvoid##1\else
        \advance#1 by 1\skip##1     
        \advance#1 by \ht##1        
       \fi}%
   \doaddinsertionheight\topins
   \doaddinsertionheight\botins
   \ifcleverfootnotes
     \doaddinsertionheight\savedfootins
   \else
     \doaddinsertionheight\footins
   \fi}

%D The local column width is available in the dimension
%D register \type{\localcolumnwidth}, which is calculated as:

\def\setcolumnhsize% beware, this one is available for use in macros 
  {\setbox0=\hbox
     {\parindent\!!zeropoint\betweencolumns}%
   \intercolumnwidth=\wd0
   \localcolumnwidth=\columntextwidth
   \advance\localcolumnwidth by -\leftskip
   \advance\localcolumnwidth by -\rightskip
   \advance\localcolumnwidth by -\nofcolumns\intercolumnwidth
   \advance\localcolumnwidth by \intercolumnwidth
   \divide\localcolumnwidth  by \nofcolumns
   \dimen0=\columntextoffset
   \multiply\dimen0 by 2
   \advance\localcolumnwidth by -\dimen0
   \usercolumnwidth=\localcolumnwidth
   \hsize=\localcolumnwidth} % we don't do it \global

%D One should be aware that when font related dimensions are
%D used in typesetting the in||between material, these
%D dimensions are influenced by bodyfont switches inside
%D multi||column mode.

\def\setcolumnvsize%
  {%\global\vsize=\columntextheight
   \global\vsize=-\columntextoffset
   \global\multiply\vsize by 2
   \global\advance\vsize by \columntextheight
   \ifdim\partialpageheight>\!!zeropoint
     \global\advance\vsize by -\partialpageheight
   \fi
   \getinsertionheights\to\dimen0\\% 
   \global\advance\vsize by -\dimen0
   \ifgridsnapping % evt altijd, nog testen
     \getnoflines\vsize
     \vsize=\noflines\openlineheight
     \advance\vsize by .5\openlineheight % collect enough data
   \fi
   \global\vsize=\nofcolumns\vsize
   \global\pagegoal=\vsize} % let's do it only here

%D It really starts here. After some checks and initializations
%D we change the output routine to continous multi||column
%D mode. This mode handles columns that fill the current and
%D next full pages. The method used is (more or less)
%D multiplying \type{\vsize} and dividing \type{\hsize} by
%D \type{\nofcolumns}. More on this can be found in the
%D \TeX book. We save the top of the current page in box
%D \type{\partialpage}.
%D
%D We manipulate \type{\topskip} a bit, just to be shure that
%D is has no flexibility. This has te be done every time a
%D font switch takles place, because \type{\topskip} can depend
%D on this.
%D
%D Watch the trick with the \type{\vbox}. This way we get the
%D right interlining and white space.

\def\beginmulticolumns%
  {\par
   \flushfootnotes
   \xdef\precolumndepth{\the\prevdepth}%
   \begingroup
   \dontshowcomposition
  %\setcolumntextwidth\relax
  %\setcolumntextheight\relax
   \widowpenalty=0 % is gewoon beter
   \clubpenalty=0  % zeker bij grids
   \ifsomefloatwaiting
     \showmessage{\m!columns}{6}{\the\savednoffloats}%
     \global\setbox\savedfloatlist=\box\floatlist
     \xdef\restoresavedfloats%
       {\global\savednoffloats=\the\savednoffloats
        \global\setbox\floatlist=\box\savedfloatlist
        \global\noexpand\somefloatwaitingtrue}%
     \global\savednoffloats=0
     \global\somefloatwaitingfalse
   \else
     \global\let\restoresavedfloats=\relax
   \fi
   \dimen0=\pagetotal
   \advance\dimen0 by \parskip
   \advance\dimen0 by \openlineheight
   \ifdim\dimen0<\pagegoal
     \allowbreak
   \else
     \break % Sometimes fails
   \fi
   \appendtoks\topskip=1\topskip\to\everybodyfont
   \the\everybodyfont
   \initializemulticolumns\nofcolumns
   \setcolumninserts
   \hangafter=0
   \hangindent=\!!zeropoint
   \everypar{}%
   \ifdim\pagetotal=\!!zeropoint \else
     \vbox{\forgetall\strut}%
     \vskip-\openlineheight
   \fi
   \global\savedpagetotal=\pagetotal
   \global\singlecolumnout=\output
   \global\output={\global\setbox\partialpage=\vbox{\unvbox\normalpagebox}}%
   \eject % no \holdinginserts=1, can make footnote disappear ! 
   \global\partialpageheight=\ht\partialpage
   \global\output={\continuousmulticolumnsout}%
   \setcolumnfloats
   \dohandleallcolumns
     {\global\setbox\currenttopcolumnbox=\box\voidb@x}%
   \checkbegincolumnfootnotes
   \let\sethsize=\setcolumnhsize
   \let\setvsize=\setcolumnvsize
   \sethsize
   \setvsize
   \showcomposition}

%D When we leave the multi||column mode, we have to process the
%D not yet shipped out part of the columns. When we don't
%D balance, we simply force a continuous output, but a balanced
%D output is more tricky.

%D First we try to fill up the page and when all or something
%D is left we try to balance things. This is another useful
%D adaption of the ancesters of these macro's. It takes some
%D reasoning to find out what happens and maybe I'm making
%D some mistake, but it works.
%D
%D Voiding box \type{\partialpage} is sometimes necessary,
%D e.g. when there is no text given between \type{\begin..}
%D and \type{\end..}. The \type{\par} is needed!

\def\endmulticolumns%
  {%\par
   \vskip\lineheight\vskip-\lineheight % take footnotes into account
   \dontshowcomposition
   \doflushcolumnfloat % added recently
  %\doflushcolumnfloats % no, since it results in wrong top floats
   \flushfootnotes      % before start of columns 
   \par
   \ifbalancecolumns
     \global\output={\continuousmulticolumnsout}%
     \goodbreak
     \global\output={\balancedmulticolumnsout}%
   \else
     \goodbreak
   \fi
   \eject                 % the prevdepth is important, try e.g. toclist in
   \prevdepth\!!zeropoint % columns before some noncolumned text text
   \global\output=\singlecolumnout
   \ifvoid\partialpage\else
     \unvbox\partialpage
   \fi
   \global\partialpageheight=\!!zeropoint
   \endgroup % here
   \nofcolumns=1
   \setvsize % the outer one!
   \checkendcolumnfootnotes
   \dosomebreak\allowbreak
   \restoresavedfloats}

%D NEW: still to be documented!

\newinsert\savedfootins

\def\checkbegincolumnfootnotes%
  {\ifcleverfootnotes
     \doflushfootnotes
     \ifdim\ht\footins>\!!zeropoint % hm, actually unknown 
       \global\setbox\savedfootins=\box\footins
     \else
       \global\setbox\savedfootins=\box\voidb@x
     \fi
   \else
     \global\setbox\savedfootins=\box\voidb@x
   \fi
   \global\skip\savedfootins=\skip\footins
   \global\count\savedfootins=\count\footins
   \setupfootnotes}

\def\checkendcolumnfootnotes%
  {\ifcleverfootnotes\ifvoid\savedfootins\else
     \global\setbox\footins=\box\savedfootins
   \fi\fi
   \global\skip\footins=\skip\savedfootins
   \global\count\footins=\count\savedfootins}

%D Because some initializations happen three times, we
%D defined a macro for them. The \type{\everypar{}} is
%D needed because we don't want anything to interfere.

\def\setmulticolumnsout%
  {\everypar{}%
   \setcolumnwarnings
   \settopskip
   \setmaxdepth
   \topskip=1\topskip
   \splittopskip=\topskip
   \splitmaxdepth=\maxdepth
   \boxmaxdepth=\maxdepth % dangerous
   \emergencystretch=\!!zeropoint\relax} % sometimes needed !

%D Flushing the page comes to pasting the columns together and
%D appending the result to box \type{\partialpage}, if not
%D void. I've seen a lot of implementations in which some skip
%D was put between normal text and multi||column text. When we
%D don't want this, the baselines can be messed up. I hope the
%D seemingly complicated calculation of a correction
%D \type{\kern} is adequate to overcome this. Although not
%D watertight, spacing is taken into account and even multiple
%D mode changes on one page go well. But cross your fingers and
%D don't blame me.
%D
%D One of the complications of flushing out the boxes is that
%D \type{\partialpage} needs to be \type{\unvbox}'ed, otherwise
%D there is too less flexibility in the page when using
%D \type{\r@ggedbottom}. It took a lot of time before these
%D kind of problems were overcome. Using \type{\unvbox} at the
%D wrong moment can generate \type{\balancingerror}'s.
%D
%D One can use the macros \type {\maxcolumnheight} and \type
%D {\maxcolumndepth} when generating material between columns
%D as well as postprocessing column lines.

\let\maxcolumnheight=\!!zeropoint
\let\maxcolumndepth =\!!zeropoint

\newbox\columnpagebox

\def\setmaxcolumndimensions%
  {\let\maxcolumnheight=\!!zeropoint
   \let\maxcolumndepth =\!!zeropoint
   \dohandleallcolumns
     {\ifdim\ht\currentcolumnbox>\maxcolumnheight
        \edef\maxcolumnheight{\the\ht\currentcolumnbox}%
      \fi
      \ifdim\dp\currentcolumnbox>\maxcolumndepth
        \edef\maxcolumndepth{\the\dp\currentcolumnbox}%
      \fi}}

% \def\flushcolumnedpage%
%   {\bgroup
%    \forgetall
%    \setmulticolumnsout
%    \showcomposition
%    \setmaxcolumndimensions
%    \postprocesscolumns
%    \dohandleallcolumns % \hbox i.v.m. \showcomposition
%      {\global\setbox\currentcolumnbox=\hbox to \localcolumnwidth
%         {\box\currentcolumnbox
%          \global\wd\currentcolumnbox=\localcolumnwidth
%          \ifheightencolumns
%            \global\ht\currentcolumnbox=\fixedcolumnheight
%          \fi}}%
%    \setmaxcolumndimensions
%    \overlaycolumnfootnotes
%    \setbox0=\vbox
%      {\hbox to \finalcolumntextwidth
%         {\ifreversecolumns
%            \@EA\dohandlerevcolumns
%          \else
%            \@EA\dohandleallcolumns
%          \fi
%            {\finishcolumnbox{\hbox
%               {\ifx\finishcolumnbox\relax\else\strut\fi
%                \box\currentcolumnbox}}%
%             \hfil}%
%          \unskip}}%
%    \scratchdimen=\!!zeropoint
%    \dohandleallcolumns
%      {\ifdim-\ht\currenttopcolumnbox<\scratchdimen
%         \scratchdimen=-\ht\currenttopcolumnbox
%       \fi
%       \global\setbox\currenttopcolumnbox=\box\voidb@x}%
%    \advance\scratchdimen by \ht0
%    \setbox2=\hbox to \columntextwidth
%      {\vrule\!!width\!!zeropoint\!!height\scratchdimen\!!depth\dp0
%       \dostepwiserecurse{2}{\nofcolumns}{1}{\hfil\betweencolumns}\hfil}%
%    \setbox0=\hbox
%      {\box0\hskip-\columntextwidth\color[black]{\box2}}%
%    \ifvoid\partialpage \else
%      \ifgridsnapping % do you believe this junk?
%        \scratchdimen=\savedpagetotal
%        \advance\scratchdimen by -\ht\partialpage
%        \advance\scratchdimen by -\dp\partialpage
%        \advance\scratchdimen by -\topskip
%        \box\partialpage
%        \kern\scratchdimen
%      \else
%        \unvbox\partialpage
%      \fi
%    \fi
%    \global\partialpageheight=\!!zeropoint
%    \setvsize
%    \dosomebreak\nobreak
%    \ifgridsnapping \else
%      \dimen0=\topskip
%      \advance\dimen0 by -\openstrutheight
%      \nointerlineskip
%      \vskip-\dimen0
%    \fi
%    \prevdepth\openstrutdepth
%    \nointerlineskip
%    \dp0=\!!zeropoint
%    \global\finalcolumnheights=\ht0
%    \getnoflines\finalcolumnheights
%    \global\finalcolumnlines=\noflines
%    \box0
%    \egroup}

\def\flushcolumnedpage%
  {\bgroup
   \forgetall
   \setmulticolumnsout
   \showcomposition
   \setmaxcolumndimensions
   \postprocesscolumns
   \dohandleallcolumns % \hbox i.v.m. \showcomposition
     {\global\setbox\currentcolumnbox=\hbox to \localcolumnwidth
        {\box\currentcolumnbox
         \global\wd\currentcolumnbox=\localcolumnwidth
         \ifheightencolumns
           \global\ht\currentcolumnbox=\fixedcolumnheight
         \fi}}%
   \setmaxcolumndimensions
   \overlaycolumnfootnotes
   \setbox\columnpagebox=\vbox
     {\hbox to \finalcolumntextwidth
        {\ifreversecolumns
           \@EA\dohandlerevcolumns
         \else
           \@EA\dohandleallcolumns
         \fi
           {\finishcolumnbox{\hbox
              {\ifx\finishcolumnbox\relax\else\strut\fi
               \box\currentcolumnbox}}%
            \hfil}%
         \unskip}}%
   \scratchdimen=\!!zeropoint
   \dohandleallcolumns
     {\ifdim-\ht\currenttopcolumnbox<\scratchdimen
        \scratchdimen=-\ht\currenttopcolumnbox
      \fi
      \global\setbox\currenttopcolumnbox=\box\voidb@x}%
   \advance\scratchdimen by \ht\columnpagebox
   \setbox\scratchbox=\hbox to \columntextwidth
     {\vrule
        \!!width\!!zeropoint
        \!!height\scratchdimen
        \!!depth\dp\columnpagebox
      \dostepwiserecurse{2}{\nofcolumns}{1}{\hfil\betweencolumns}\hfil}%
   \setbox\columnpagebox=\hbox
     {\box\columnpagebox
      \hskip-\columntextwidth
      \color[black]{\box\scratchbox}}%
   \postprocesscolumnpagebox % new, acts upon \box\columnpagebox
   \ifvoid\partialpage \else
     \ifgridsnapping % do you believe this junk?
       \scratchdimen=\savedpagetotal
       \advance\scratchdimen by -\ht\partialpage
       \advance\scratchdimen by -\dp\partialpage
       \advance\scratchdimen by -\topskip
       \box\partialpage
       \kern\scratchdimen
     \else
       \unvbox\partialpage
     \fi
   \fi
   \global\partialpageheight=\!!zeropoint
   \setvsize
   \dosomebreak\nobreak
   \ifgridsnapping \else
     \scratchdimen=\topskip
     \advance\scratchdimen by -\openstrutheight
     \nointerlineskip
     \vskip-\scratchdimen
   \fi
   \prevdepth\openstrutdepth
   \nointerlineskip
   \dp\columnpagebox=\!!zeropoint
   \global\finalcolumnheights=\ht\columnpagebox
   \getnoflines\finalcolumnheights
   \global\finalcolumnlines=\noflines
   \box\columnpagebox
   \egroup}

%D In case one didn't notice, finaly \type{\finishcolumnbox} is
%D applied to all boxes. One can use these hooks for special
%D purposes.
%D
%D Once upon a time I wanted to manipulate the individual lines
%D in a column. This feature is demonstrated in the two examples
%D below.
%D
%D \startbuffer
%D \def\postprocesscolumnline#1% or \postprocesscolumnbox
%D   {\ruledhbox{\box#1}\hss}
%D
%D \startkolommen[n=4]
%D \dorecurse{25}{line: \recurselevel\par}
%D \stopkolommen
%D \stopbuffer
%D
%D \typebuffer
%D
%D Here we show the natural width of the lines:
%D
%D {\haalbuffer}
%D
%D The next example does a bit more advanced manipulation:
%D
%D \startbuffer
%D \def\postprocesscolumnline#1%
%D   {\ifodd\currentcolumn
%D      \hfill\unhbox#1\relax
%D    \else
%D      \relax\unhbox#1\hfill
%D    \fi}
%D
%D \startkolommen[n=4]
%D \dorecurse{25}{line \recurselevel\par}
%D \stopkolommen
%D \stopbuffer
%D
%D \typebuffer
%D
%D Here we also see an application of \type{\currentcolumn}:
%D
%D {\haalbuffer}
%D
%D This feature is implemented using the reshape macros
%D presented in \type{supp-box}.

\def\postprocesscolumns%
  {\ifx\postprocesscolumnline\undefined \else
     \dohandleallcolumns
       {\global\setbox\currentcolumnbox=\vtop
          {\beginofshapebox
           \unvbox\currentcolumnbox
           \unskip\unskip
           \endofshapebox
           \reshapebox
             {\dimen0=\ht\shapebox
              \dimen2=\dp\shapebox
              \setbox\shapebox=\hbox to \hsize
                {\postprocesscolumnline\shapebox}%
              \ht\shapebox=\dimen0
              \dp\shapebox=\dimen2
              \box\shapebox}%
           \flushshapebox
           \everypar{}\parskip\!!zeropoint % = \forgetall
           \strut\endgraf
           \vskip-\lineheight
           \vfil}}%
   \fi
   \ifx\postprocesscolumnbox\undefined \else
     \dohandleallcolumns
       {\global\setbox\currentcolumnbox=\hbox
          {\postprocesscolumnbox\currentcolumnbox}}
   \fi}

%D We default to doing nothing!

\let\postprocesscolumnline   =\undefined
\let\postprocesscolumnbox    =\undefined
\let\postprocesscolumnpagebox=\relax

%D \macros
%D   {reversecolumnstrue} 
%D
%D We can force the macro that takes care of combining 
%D the columns, to flush them in the revere order. Of 
%D course, by default we don't reverse. 

\newif\ifreversecolumns 
 
%D Here comes the simple splitting routine. It's a bit
%D longer than expected because of ragging bottoms or not.
%D This part can be a bit shorter but I suppose that I will
%D forget what happens. The splitting takes some already
%D present material (think of floats) into account!
%D
%D First we present some auxiliary routines. Any material,
%D like for instance floats, that is already present in the
%D boxes is preserved.

% \def\splitcolumn#1from \box#2to \dimen#3 top \box#4%
%   {\bgroup
%    \ifdim\ht#4>\!!zeropoint
%      \dimen0=\dimen#3\relax
%      \dimen2=\dimen#3\relax
%      \advance\dimen0 by -\ht#4
%      \setbox0=\vsplit#2 to \dimen0
%      \global\setbox#1=\vbox to \dimen2{\unvcopy#4\unvbox0}%
%    \else
%      \global\setbox#1=\vsplit#2 to \dimen#3
%    \fi
%    \egroup}

% \def\splitcolumn#1from \box#2to \dimen#3 top \box#4%
%   {\bgroup
%    \ifdim\ht#4>\!!zeropoint
%      \dimen0=\dimen#3\relax
%      \dimen2=\dimen2
%      \advance\dimen0 by -\ht#4%
%      \columnfootnotecorrection{#1}{\dimen0}%
%      \setbox0=\vsplit#2 to \dimen0
%      \global\setbox#1=\vbox to \dimen2
%        {\ifgridsnapping
%           \dimen0=-\openstrutheight\advance\dimen0 by \topskip
%           \vskip\dimen0\copy#4\vskip-\dimen0
%         \else
%           \unvcopy#4%
%         \fi
%         \unvbox0
%         \fakecolumnfootnotes{#1}}%
%    \else
%      \ifcleverfootnotes
%        \columnfootnotecorrection{#1}{\dimen#3}%
%        \setbox0=\vsplit#2 to \dimen#3%
%        \global\setbox#1=\vbox to \dimen#3%
%          {\unvbox0
%           \fakecolumnfootnotes{#1}}%
%     \else
%        \global\setbox#1=\vsplit#2 to \dimen#3%
%      \fi
%    \fi
%    \egroup}

\def\splitcolumn#1from \box#2to \dimen#3 top \box#4%
  {\bgroup
   \ifdim\ht#4>\!!zeropoint
     \dimen0=\dimen#3\relax
     \dimen2=\dimen0
     \advance\dimen0 by -\ht#4%
     \columnfootnotecorrection{#1}{\dimen0}%
     \setbox0=\vsplit#2 to \dimen0
     \global\setbox#1=\vbox to \dimen2
       {\ifgridsnapping
          \dimen0=-\openstrutheight\advance\dimen0 by \topskip
          \vskip\dimen0\copy#4\vskip-\dimen0
        \else
          \unvcopy#4%
        \fi
        \fuzzysnappedbox\unvbox0
        \fakecolumnfootnotes{#1}}%
   \else
     \ifcleverfootnotes
       \columnfootnotecorrection{#1}{\dimen#3}%
       \setbox0=\vsplit#2 to \dimen#3%
       \global\setbox#1=\vbox to \dimen#3%
         {\fuzzysnappedbox\unvbox0
          \fakecolumnfootnotes{#1}}%
     \else
       \global\setbox#1=\vsplit#2 to \dimen#3%
       \global\setbox#1=\vbox
         {\fuzzysnappedbox\unvbox{#1}}% % or \box ?
     \fi
   \fi
   \egroup}

\def\splitcurrentcolumn from \box#1to \dimen#2%
  {\splitcolumn\currentcolumnbox from \box#1 to \dimen#2 top \box\currenttopcolumnbox}

\def\splitfirstcolumn from \box#1to \dimen#2%
  {\splitcolumn\firstcolumnbox from \box#1 to \dimen#2 top \box\firsttopcolumnbox}

\def\splitlastcolumn from \box#1to \dimen#2%
  {\global\setbox\lastcolumnbox=\vbox
     {\unvcopy\lasttopcolumnbox
      \fuzzysnappedbox\unvbox{#1}%
      \fakecolumnfootnotes\lastcolumnbox}}

%D NEW: still to be documented.

\def\fakecolumnfootnotes#1%
  {\relax
   \ifcleverfootnotes
     \ifnum#1=\lastcolumnbox
       \ifdim\ht\footins>\!!zeropoint
         \vskip1\skip\footins % remove stretch and shrink 
         \kern\ht\footins % a \vskip would be is discarded! 
       \fi
     \fi
   \fi}

\def\columnfootnotecorrection#1#2%
  {\relax
   \ifcleverfootnotes
     \ifnum#1=\lastcolumnbox\relax
       \ifdim\ht\footins>\!!zeropoint
         \advance#2 by -\ht\footins
         \advance#2 by -\skip\footins
       \fi
     \fi
   \fi}

\def\overlaycolumnfootnotes%   VERVANGEN !!!
  {\relax
   \ifcleverfootnotes\ifdim\ht\footins>\!!zeropoint
     \bgroup
     \scratchdimen=\ht\firstcolumnbox
     \advance\scratchdimen by -\openstrutdepth % \dp\strutbox
     \getnoflines\scratchdimen
     \advance\noflines by -2
     \scratchdimen=\noflines\lineheight
     \advance\scratchdimen by \topskip
     \setbox0=\hbox
       {\lower\scratchdimen\vbox{\placefootnoteinserts}}%
     \ht0=\openstrutheight % \ht\strutbox
     \dp0=\openstrutdepth  % \dp\strutbox
     \scratchdimen=\ht\lastcolumnbox
     \global\setbox\lastcolumnbox=\vbox to \scratchdimen
       {\box\lastcolumnbox
        \vskip-\scratchdimen
        \color[black]{\box0}}%
     \egroup
   \fi\fi}

%D Here comes the routine that splits the long box in columns.
%D The macro \type{\flushcolumnfloats} can be used to flush
%D either floats that were present before the multi||column
%D mode was entered, or floats that migrate to next columns.
%D Flushing floats is a delicate process.

\def\continuousmulticolumnsout% 
  {\bgroup
   \forgetall
   \setmulticolumnsout
   \dontshowcomposition
   \dimen0=\columntextheight
   \getinsertionheights\to\dimen2\\% toegevoegd ivm voetnoten
   \advance\dimen2 by \partialpageheight
   \dimen0=\columntextheight
   \advance\dimen0 by -\partialpageheight
   \getinsertionheights\to\dimen2\\%
   \advance\dimen0 by -\dimen2
   \ifgridsnapping % evt altijd, nog testen
     \getnoflines{\dimen0}
     \dimen0=\noflines\openlineheight
   \fi
   \dohandleallcolumns
     {\splitcurrentcolumn from \box\normalpagebox to \dimen0}
   \setbox\restofpage=\vbox{\unvbox\normalpagebox}%
   \ifinheritcolumns
     \ifr@ggedbottom % vreemd 
%\ifbottomnotes % can better be a state 
       \dohandleallcolumns
         {\global\setbox\currentcolumnbox=\vbox to \ht\firstcolumnbox
            {\dimen0=\dp\currentcolumnbox
             \unvbox\currentcolumnbox
             \vskip-\dimen0
             \vskip\openstrutdepth % \dp\strutbox
             \prevdepth\openstrutdepth % \dp\strutbox
             \vfill}}%
%\else
%  \dimen0=\ht\firstcolumnbox
%\fi
\ifbottomnotes \else
  \dimen0=\ht\firstcolumnbox
\fi
     \fi
     \ifn@rmalbottom
       \advance\dimen0 by \maxdepth
       \dohandleallcolumns
         {\global\setbox\currentcolumnbox=\vbox to \dimen0
            {\unvbox\currentcolumnbox}}%
     \fi
     \ifb@selinebottom
       % the columns are on top of the baseline
     \fi
   \else
     \dohandleallcolumns
       {\global\setbox\currentcolumnbox=\vbox to \dimen0
          {\ifstretchcolumns
             \unvbox\currentcolumnbox
           \else
             \unvbox\currentcolumnbox % wel of niet \unvbox ?
             \vfill
           \fi}}%
     \dohandleallcolumns
       {\global\ht\currentcolumnbox=\dimen0}%
   \fi
   \setbox\partialpage=\vbox{\flushcolumnedpage}%
   \finaloutput\box\partialpage
   \sethsize
   \setvsize
   \flushcolumnfloats
   \unvbox\restofpage
   % \penalty\outputpenalty % gaat gruwelijk mis in opsommingen
   \egroup}

%D And this is the balancing stuff. Again, part of the routine
%D is dedicated to handling ragged bottoms, but here we also
%D see some handling concerning the stretching of columns.
%D We set \type{\widowpenalty} at~0, which enables us to
%D balance columns with few lines. The use of \type{\box2} and
%D \type{\box4} garantees a more robust check when skips are
%D used.

\def\balancedmulticolumnsout% 
  {\bgroup
   \setmulticolumnsout
   \dontshowcomposition
   \widowpenalty=0
   \setbox0=\vbox{\unvbox\normalpagebox}%
\ifdim\ht0>\openlineheight % at least one line 
  \ifnum\minbalancetoplines<2 % balance anyway 
    \donetrue 
  \else % check criterium to available lines  
    \getnoflines{\ht0}%
    \divide\noflines by \nofcolumns \relax
    \ifnum\noflines<\minbalancetoplines \relax
      \dimen0=\ht0 
      \advance\dimen0 by \ht\firsttopcolumnbox
      \advance\dimen0 by \openlineheight \relax % let's play safe
      \ifdim\dimen0>\columntextheight % column exceeding text height 
        \donetrue
      \else % it seems to fit 
        \donefalse 
      \fi
    \else % balance indeed
      \donetrue
    \fi
  \fi
\else % balancing does not make sense 
  \donefalse
\fi
\ifdone % start balancing 
  %\ifdim\ht0>\openlineheight
     \dimen0=\ht0
     \advance\dimen0 by \topskip
     \advance\dimen0 by -\baselineskip
     \dohandleallcolumns
       {\advance\dimen0 by \ht\currenttopcolumnbox}%
     \divide\dimen0 by \nofcolumns
     \vbadness=\!!tenthousand\relax
     \count255=0
     \bgroup
     \ifgridsnapping
       \dimen2=\lineheight
     \else
       \dimen2=\!!onepoint % RUBISH
       \dimen2=\spacingfactor\dimen2
     \fi
     \loop
       \advance\count255 by 1
       \global\setbox\restofpage=\copy0\relax
       \splitfirstcolumn from \box\restofpage to \dimen0
       \dohandlemidcolumns
         {\splitcurrentcolumn from \box\restofpage to \dimen0}%
       \splitlastcolumn from \box\restofpage to \dimen0
       \setbox2=\vbox{\unvcopy\firstcolumnbox}%
       \dimen4=\!!zeropoint
       \dohandleallcolumns
         {\setbox4=\vbox 
            {\unvcopy\currentcolumnbox
            %rather new, test this on pdftex-z.tex
             \unpenalty\unskip\unpenalty\unskip}% maybe better in main splitter
            %\writestatus{balance}{\the\currentcolumnbox: \the\ht4}%
             \dimen6=\ht4
             \ifdim\dimen6>\dimen4 \dimen4=\dimen6 \fi}%
\advance\dimen4 by -.0005pt % get rid of accurracy problem, pretty new  
       \ifnum\count255>100\relax
         \donefalse
       \else\ifdim\dimen4>\ht2
         \donetrue
       \else
         \donefalse
       \fi\fi
       \ifdone
         \advance\dimen0 by \dimen2\relax
     \repeat
     \dohandleallcolumns
       {\global\setbox\currentcolumnbox=\vbox{\unvcopy\currentcolumnbox}}% NIEUW
     \ifnum\count255>100\relax
       \showmessage{\m!columns}{7}{}%
     \else
       \showmessage{\m!columns}{8}{\the\count255\space}%
     \fi
     \egroup
     \ifinheritcolumns
       \dimen0=\ht\firstcolumnbox
       \dimen2=\ht\firstcolumnbox
       \advance\dimen2 by -\openlineheight
       \dohandleallcolumns
         {\dimen4=\ht\currentcolumnbox
          \dimen6=10\openlineheight
          \global\setbox\currentcolumnbox=\vbox to \dimen0
            {\unvbox\currentcolumnbox
             \ifdim\dimen4>\dimen6
               \ifdim\dimen4<\dimen0
                 \ifdim\dimen4>\dimen2
                   \vskip\!!zeropoint  % !!
                 \else
                   \vskip\openlineheight
                   \vfill
                 \fi
               \else
                 \vskip\!!zeropoint
               \fi
             \else
               \vskip\openlineheight
               \vfill
             \fi}}%
     \else
       \bgroup
       \ifstretchcolumns
         \dimen0=\ht\firstcolumnbox
         \dimen2=\bottomtolerance\ht\firstcolumnbox
         \setbox0=\vbox{\unvcopy\lastcolumnbox}%
         \advance\dimen0 by -\ht0\relax
         \advance\dimen0 by -\dp0\relax
         \ifdim\dimen0>\openlineheight\relax
           \ifdim\dimen0>\dimen2\relax
             % \stretchcolumnsfalse % beter goed slecht dan slecht goed
             \showmessage{\m!columns}{9}{}%
           \fi
         \fi
       \fi
       \dohandleallcolumns
         {\global\setbox\currentcolumnbox=\vbox to \ht\firstcolumnbox
            {\ifstretchcolumns
               \unvbox\currentcolumnbox
             \else
               \box\currentcolumnbox
               \vfill
             \fi}}%
       \egroup
     \fi
   \else
     \showmessage{\m!columns}{10}{}%
     \global\setbox\firstcolumnbox=\vbox{\unvbox0}%
   \fi
   \global\output={\balancingerror}%
   \b@selinebottomtrue % forces depth in separation rule
   \flushcolumnedpage
   \allowbreak
   \egroup}

%D The multicolumn mechanism is incorporated in a \CONTEXT\
%D interface, which acts like:
%D
%D \starttypen
%D \startcolumns[n=4,balance=no,stretch=no,line=on]
%D   some text
%D \stopcolumns
%D \stoptypen
%D
%D The setup is optional. The default behaviour of columns
%D can be set up with:
%D
%D \starttypen
%D \setupcolumns
%D   [n=2,
%D    balance=yes,
%D    stretch=text,
%D    line=off]
%D \stoptypen
%D
%D In this case, stretching is according to the way it's
%D done outside columns (\type{\inheritcolumnstrue}). Also
%D we can setup the \type{tolerance} within a column, the
%D \type{distance} between columns and the fixed
%D \type{height} of a column.

%D Multi||column output: the float routines
%D
%D Here come the routines that handle the placement of column
%D floats. Floats that are to big migrate to the next
%D column. Floats that are too wide, migrate to the top of the
%D next page, where they span as much columns as needed.
%D Floats that are left over from outside the multi||column
%D mode are flushed first. In macro \type{\finaloutput} the
%D topfloats that are left from previous text should be set.
%D
%D When there are some floats in the queue, we inhibit the
%D flushing of floats on top of columns. The number of
%D waiting floats is preswent in \type{\savednoftopfloats} and
%D is saved. As long as there are floats waiting, the topfloats
%D are places as if we are outside multi||column mode. This is
%D neccessary for e.g. multicolumn lists.
%D
%D When all those floats are flushed, we switch to the local
%D flushing routine.

\def\setcolumnfloats%
  {\xdef\globalsavednoffloats{\the\savednoffloats}%
   \ifnum\globalsavednoffloats>0
     \setglobalcolumnfloats
   \else
     \setlocalcolumnfloats
   \fi}

\def\setglobalcolumnfloats%
  {\everypar={}%
   \let\flushcolumnfloat=\relax
   \let\doroomfloat=\relax
   \let\flushcolumnfloats=\noflushcolumnfloats}

\def\setlocalcolumnfloats%
  {\everypar=
     {\flushfootnotes\flushcolumnfloat\flushmargincontents\checkindentation}%
   \let\flushcolumnfloat=\doflushcolumnfloat
   \let\doroomfloat=\docolumnroomfloat
   \let\flushcolumnfloats=\doflushcolumnfloats
\let\doflushfloats\doflushcolumnfloats % new
   \let\dosetbothinserts=\relax
   \let\dotopinsertions=\relax}

\def\noflushcolumnfloats%
  {\bgroup
   \xdef\localsavednoffloats{\the\savednoffloats}%
   \global\savednoffloats=\globalsavednoffloats
   \dotopinsertions
   \xdef\globalsavenoffloats{\the\savednoffloats}%
   \ifnum\globalsavednoffloats=0
     \setlocalcolumnfloats
   \fi
   \global\savednoffloats=\localsavednoffloats
   \egroup}

%D We need to calculate the amount of free space in a columns.
%D When there is not enough room, we migrate the float to the
%D next column. These macro's are alternatives (and
%D look||alikes) of \type{\doroomfloat}. When a float is to
%D wide, for one column, it is moved to the top of the next
%D page. Of course such moved floats have to be taken into
%D account when we calculate the available space. It's a pitty
%D that such things are no integral part of \TEX.

% \def\getcolumnstatus\column#1\total#2\goal#3\\%
%   {\ifdim\pagegoal<\maxdimen
%      \dimen0=\pagegoal
%      \divide\dimen0 by \nofcolumns
%      \dimen2=\!!zeropoint
%      \count255=0\relax
%      \dimen8=\columntextheight
%      \advance\dimen8 by -\partialpageheight
%      %\advance\dimen8 by -\maxdepth % recently deleted
%      \def\dogetcolumnstatus%
%        {\advance\count255 by 1\relax
%         \advance\dimen2 by \ht\currenttopcolumnbox
%         \advance\dimen2 by \dp\currenttopcolumnbox
%         \dimen4=\dimen2\relax
%         \advance\dimen4 by \pagetotal
%         \dimen6=\count255\dimen8
%         \ifdim\dimen4>\dimen6
%         \else
%           \let\dogetcolumnstatus=\relax
%         \fi}%
%      \dohandleallcolumns{\dogetcolumnstatus}%
%      \ifdim\dimen4=\dimen6
%        \dimen4=\!!zeropoint
%        \advance\count255 by 1
%      \fi
%      #1=\count255
%      #2=\dimen4
%      #3=\dimen6
%    \else
%      #1=1
%      #2=\!!zeropoint
%      #3=\teksthoogte
%      \advance#3 by -\partialpageheight
%    \fi}

\def\getcolumnstatus\column#1\total#2\goal#3\\%
  {\ifdim\pagegoal<\maxdimen
     \dimen0=\pagetotal
   \else
     \dimen0=\!!zeropoint
   \fi
   \dimen2=\!!zeropoint
   \count255=0
   \dimen8=\columntextheight
   \advance\dimen8 by -\partialpageheight
   \def\dogetcolumnstatus%
     {\advance\count255 by 1
      \advance\dimen2 by \ht\currenttopcolumnbox
      \advance\dimen2 by \dp\currenttopcolumnbox
      \dimen4=\dimen2
      \advance\dimen4 by \dimen0
      \dimen6=\count255\dimen8
      \ifdim\dimen4>\dimen6
      \else
        \let\dogetcolumnstatus=\relax
      \fi}%
   \dohandleallcolumns{\dogetcolumnstatus}%
   \ifnum\count255=0 \count255=1 \fi
   #1=\count255
   #2=\dimen4
   #3=\dimen6 }

\def\getinsertionheight%
  {\ifdim\pagegoal<\maxdimen
     \bgroup
     \dimen0=\columntextheight
     \advance\dimen0 by -\pagegoal
     \xdef\insertionheight{\the\dimen0}%
     \egroup
   \else
     \global\let\insertionheight=\!!zeropoint
   \fi}

\def\docolumnroomfloat%
  {\ifpostponecolumnfloats 
     \global\roomforfloatfalse
   \else\ifnofloatpermitted
     \global\roomforfloatfalse
   \else
     \bgroup
     \getcolumnstatus\column\count255\total\dimen0\goal\dimen2\\%
     \advance\dimen0 by 2\openlineheight % nog nodig ?
    %\ifnum\count255=\nofcolumns
    %  \getinsertionheight
    % %\message{\insertionheight}\wait
    %  \advance\dimen0 by \insertionheight
    %\fi
     \setbox\scratchbox=\vbox % tricky met objecten ?
       {\blanko[\@@bkvoorwit]
        \snaptogrid\vbox{\copy\floatbox}}%
     \advance\dimen0 by \ht\scratchbox
     \advance\dimen0 by .5\lineheight % needed because goal a bit higher
    %\message{column: \the\count255; total: \the\dimen0; goal: \the\dimen2}\wait
     \ifdim\dimen0>\dimen2
       \global\roomforfloatfalse
     \else
       \global\roomforfloattrue
     \fi
     \ifdim\wd\floatbox>\hsize
       \showmessage{\m!columns}{11}{}%
       \global\roomforfloatfalse
     \fi
     \egroup
   \fi\fi}

%D Flushing one float is done as soon as possible, i.e.
%D \type{\everypar}. This means that (at the moment)
%D sidefloats are not supported (overulled)!

\newif\ifflushingcolumnfloats \flushingcolumnfloatstrue

\def\doflushcolumnfloat%
  {\ifpostponecolumnfloats\else\ifflushingcolumnfloats\ifprocessingverbatim\else\ifsomefloatwaiting
     \bgroup
     \forgetall
     \let\doflushcolumnfloat=\relax
     \getcolumnstatus\column\count255\total\dimen0\goal\dimen2\\%
     \ifdim\dimen0>\!!zeropoint
       \dogetfloat
       \ifdim\wd\floatbox>\hsize
         \doresavefloat
       \else
        %\setbox2=\vbox
        %  {\blanko[\@@bkvoorwit]
        %   \snaptogrid\vbox{\copy\floatbox}%
        %   \blanko[\@@bknawit]
         \setbox2=\vbox
           {\blanko[\@@bkvoorwit]
            \snaptogrid\vbox{\copy\floatbox}}%
         \advance\dimen0 by \ht2 
         \ifdim\dimen0>\dimen2
           \ifnum\count255<\nofcolumns
             \advance\count255 by 1
             \edef\currenttopcolumnbox{\getvalue{\@@topcol\the\count255}}%
             \ifdim\ht\currenttopcolumnbox=\!!zeropoint
               \global\setbox\currenttopcolumnbox=\vbox
                 {\snaptogrid\vbox{\copy\floatbox}
                  \witruimte % nodig ?
                  \blanko[\@@bknawit]}%
               \dimen4=\ht\currenttopcolumnbox
               \advance\dimen4 by \dp\currenttopcolumnbox
               \global\advance\vsize by -\dimen4
               \advance\dimen4 by -\pagegoal
               \global\pagegoal=-\dimen4
               \showmessage{\m!columns}{12}{a}%
             \else
               \showmessage{\m!columns}{12}{b}%
               \doresavefloat
             \fi
           \else
             \showmessage{\m!columns}{12}{c}%
             \doresavefloat
           \fi
         \else
           \ifhmode{\setbox0=\lastbox}\fi% waar is die er in geslopen
           \par
           \ifdim\prevdepth<\!!zeropoint\relax % anders bovenaan kolom witruimte
           \else
             \nobreak
             \blanko[\@@bkvoorwit]
             \nobreak
           \fi
           \flushfloatbox
           \blanko[\@@bknawit]
         \fi
       \fi
     \fi
     \egroup
   \fi\fi\fi\fi}

%D This one looks complicated. Upto \type{\nofcolumns} floats
%D are placed, taking the width of a float into account. This
%D routine can be improved on different ways:
%D
%D \startopsomming[intro,opelkaar]
%D \som taking into account some imaginary baseline, just to
%D      get the captions in line
%D \som multipass flushing until as many floats are displaced
%D      as possible
%D \stopopsomming
%D
%D When handling lots of (small) floats spacing can get worse
%D because of lining out the columns.

\def\doflushcolumnfloats%
  {\ifpostponecolumnfloats\else
     \bgroup
     \forgetall
     \ifsomefloatwaiting
       \dimen8=\!!zeropoint
       \dimen4=\!!zeropoint
       \count0=0            % count0 can be used local
       \count2=\nofcolumns  % count2 can be used local
       \dohandleallcolumns
         {\ifnum\count0>0\relax % the wide one's reserved space
            \global\setbox\currenttopcolumnbox=\vbox
              {\snaptogrid\vbox
                 {\copy\currenttopcolumnbox
                  \hbox{\vphantom{\copy\floatbox}}}
                  \witruimte % nodig ?
                  \blanko[\@@bknawit]}%
          \else
            \dogetfloat
\ifdim\wd\floatbox>\finalcolumntextwidth % better somewhere else too 
  \global\setbox\floatbox=\hbox to \finalcolumntextwidth{\hss\box\floatbox\hss}%
\fi % otherwise the graphic may disappear
            \ifdim\wd\floatbox>\hsize
              \dimen0=\wd\floatbox
              \advance\dimen0 by \intercolumnwidth
              \dimen2=\hsize
              \advance\dimen2 by \intercolumnwidth
              \advance\dimen0 by .5pt % hm, why 1  
              \advance\dimen2 by .5pt % hm, why 2
              \divide\dimen0 by \dimen2
              \count0=\dimen0
              \advance\count0 by 1
              \ifnum\count0>\count2
                \doresavefloat
                \count0=0
              \else
                \dimen0=\count0\hsize
                \advance\dimen0 by \count0\intercolumnwidth
                \advance\dimen0 by -\intercolumnwidth
                \global\setbox\floatbox=\hbox to \dimen0
                 %{\hss\hbox{\copy\floatbox}\hss}%
                  {\processaction[\@@bkplaats] % how easy to forget 
                     [  \v!links=>\copy\floatbox\hss,
                       \v!rechts=>\hss\copy\floatbox,
                      \s!default=>\hss\copy\floatbox\hss,
                      \s!unknown=>\hss\copy\floatbox\hss]}%
              \fi
              \showmessage{\m!columns}{13}{}%
            \else
            %  \showmessage{\m!columns}{13}{}%
            \fi
            \ifdim\ht\floatbox>\!!zeropoint\relax
              \global\setbox\currenttopcolumnbox=\vbox
                {\snaptogrid\vbox
                   {\copy\currenttopcolumnbox
                    \copy\floatbox}
                 \witruimte % nodig ?
                 \blanko[\@@bknawit]}%
            \fi
            \dimen6=\ht\currenttopcolumnbox
            \advance\dimen6 by \dp\currenttopcolumnbox
          \fi
          \ifdim\dimen4<\ht\currenttopcolumnbox
            \dimen4=\ht\currenttopcolumnbox
          \fi
          \advance\dimen8 by \dimen6
          \advance\count2 by -1
          \advance\count0 by -1 }%
       \setvsize
       \global\advance\vsize by -\dimen8
       \global\pagegoal=\vsize
     \else
       %\doflushfloats % does not snap!
     \fi
     \egroup
   \fi}

%D This were the multi||column routines. They can and need to
%D be improved but at the moment their behaviour is acceptable.
%D
%D One inprovement can be to normalize the height of floats
%D to $n\times$\type{\lineheight} with a macro like:
%D
%D \starttypen
%D \normalizevbox{...}
%D \stoptypen

\protect \endinput

% border case, should fit on one page 
%
% \startkolommen 
% 
% 1 \input tufte  \par \plaatsfiguur{}{\omlijnd[breedte=\hsize,hoogte=3cm]{1}}
% 2 \input tufte  \par \plaatsfiguur{}{\omlijnd[breedte=\hsize,hoogte=3cm]{2}}
% 3 \input tufte  \par \plaatsfiguur{}{\omlijnd[breedte=\hsize,hoogte=3cm]{3}}
% 
% \stopkolommen
