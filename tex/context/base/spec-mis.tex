%D \module
%D   [       file=spec-mis,
%D        version=1997.04.01,
%D          title=\CONTEXT\ Special Macros,
%D       subtitle=Miscellaneous Macros,
%D         author=Hans Hagen,
%D           date=\currentdate,
%D      copyright={PRAGMA / Hans Hagen \& Ton Otten}]
%C
%C This module is part of the \CONTEXT\ macro||package and is
%C therefore copyrighted by \PRAGMA. Non||commercial use is 
%C granted. 

%D Quite some modules in this group are dedicated to supporting
%D \PDF\ directly by means of \PDFTEX or indirectly by using
%D Acrobat Distiller. This module implements some common
%D features. 

\writestatus{loading}{Context Special Macros / Miscellaneous Macros}

\unprotect

%D \macros 
%D   {setPDFdestination}
%D 
%D \PDF\ destinations should obey the specifications laid down 
%D in the \PDF\ reference manual. The next macro strips illegal 
%D characters from the destination name. 

\def\setPDFdestination#1%
  {\bgroup
   \obeylccodes
   \lccode`\/=`-\lccode`\#=`-\lccode`\<=`-\lccode`\>=`-%
   \lccode`\[=`-\lccode`\]=`-\lccode`\(=`-\lccode`\)=`-%
   \stripcharacter{ }\from#1\to\PDFdestination
   \@EA\lowercase\@EA{\@EA\xdef\@EA\PDFdestination\@EA{\PDFdestination}}%
   \egroup}

%D \macros
%D   {URLhash}
%D
%D A rather trivial macro: 

\expandafter\def\expandafter\URLhash\expandafter{\string#}

%D \macros 
%D   {ifusepagedestinations}
%D 
%D In \PDF\ version 1.0 only page references were supported,
%D while in \DVIWINDO\ 1.N only named references were accepted.
%D Therefore \CONTEXT\ supports both methods of referencing. In
%D \PDF\ version 1.1 named destinations arrived. Lack of
%D continuous support of version 1.1 viewers for \MSDOS\
%D therefore sometimes forces us to prefer page references. As
%D a bonus, they are faster too and have no limitations. How
%D fortunate we were having both mechanisms available when the
%D version 3.0 (\PDF\ version 1.2) viewers proved to be too 
%D bugged to support named destinations. 

\newif\ifusepagedestinations

%D \macros
%D   {sanitizePDFstring}
%D
%D This macro at least tries to convert a arbitrary string 
%D into a sequence of characters valid for \PDF\ bookmarks and 
%D alike. It's a slow one, that uses \type{\lccode}'s to 
%D change the glyph as well as converts sensisitve ones into a 
%D \PDF\ command sequence, so \type{(} becomes \type{\(}.  In
%D fact we translate the string to lowercase inactive and non 
%D special characters, limit their number and finaly convert 
%D some of the characters to save ones. 

\chardef\maxPDFstringsize=60

\def\sanitizePDFstring#1\to#2%
  {\bgroup
   \obeylccodes
   \lccode`( =1  \lccode`) =1
   \lccode`< =1  \lccode`> =1
   \lccode`[ =1  \lccode`] =1
   \lccode`\\=1  \lccode`/ =1
   \lowercase{\convertargument#1\to\ascii}%
   % by integrating the split in the loop below
   % \splitofftokens\maxPDFstringsize\from\ascii\to\ascii 
   % we diminish the processing time considerably
   \scratchcounter=\maxPDFstringsize
   \def\docommando##1%
     {\ifnum\scratchcounter>0
        \advance\scratchcounter by -1
        \ifcase\lccode`##1\relax
          \xdef#2{#2.}% let's show that something is missing
        \or
          \xdef#2{#2\expandafter\string\csname##1\endcsname}%
        \else
          \xdef#2{#2##1}%
        \fi
      \fi}%
   %\global\let#2=\empty 
   % or to permit #2 to be \ascii too:
   \@EA\global\@EA\let\@EA#2\@EA\empty
   \@EA\handletokens\ascii\with\docommando
   \egroup}

%D \macros 
%D   {dodoinsertfile}
%D 
%D File insertion depend on the driver or \TEX\ variant used.
%D All driver modules use the same scheme for file insertion,
%D and therefore have the next macro in common: 

\def\dododoinsertfile[#1][#2,#3][#4]%
  {\def\fileinsertionclass{do#1insert}%
   \doifdefinedelse{\fileinsertionclass#3}
     {\def\next{\getvalue{\fileinsertionclass#3}}}
     {\doifdefinedelse{\fileinsertionclass#2}
        {\def\next{\getvalue{\fileinsertionclass#2}}}
        {\def\next{\gobbleeightarguments}}}%
   \next{#4}}

\def\dodoinsertfile#1#2#3%
  {\dododoinsertfile[#1][#2][#3]}

%D This macro is called with 10 arguments, where the first 
%D one specifies the driver, like \type{yy} or \type{tr}. The
%D second argument is a \type{{type,method}} pair and the 
%D third the filename. 

\protect

\endinput
