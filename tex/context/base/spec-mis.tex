%D \module
%D   [       file=spec-mis,
%D        version=1997.04.01,
%D          title=\CONTEXT\ Special Macros,
%D       subtitle=Miscellaneous Macros,
%D         author=Hans Hagen,
%D           date=\currentdate,
%D      copyright={PRAGMA / Hans Hagen \& Ton Otten}]
%C
%C This module is part of the \CONTEXT\ macro||package and is
%C therefore copyrighted by \PRAGMA. See mreadme.pdf for 
%C details. 

%D Quite some modules in this group are dedicated to supporting
%D \PDF\ directly by means of \PDFTEX or indirectly by using
%D Acrobat Distiller. This module implements some common
%D features. 

\writestatus{loading}{Context Special Macros / Miscellaneous Macros}

\unprotect

%D \macros
%D   {URLhash}
%D
%D A rather trivial macro: 

\expandafter\def\expandafter\URLhash\expandafter{\string#}

%D \macros 
%D   {ifusepagedestinations}
%D 
%D In \PDF\ version 1.0 only page references were supported,
%D while in \DVIWINDO\ 1.N only named references were accepted.
%D Therefore \CONTEXT\ supports both methods of referencing. In
%D \PDF\ version 1.1 named destinations arrived. Lack of
%D continuous support of version 1.1 viewers for \MSDOS\
%D therefore sometimes forces us to prefer page references. As
%D a bonus, they are faster too and have no limitations. How
%D fortunate we were having both mechanisms available when the
%D version 3.0 (\PDF\ version 1.2) viewers proved to be too 
%D bugged to support named destinations. 

\newif\ifusepagedestinations

%D \macros 
%D   {ifhighlighthyperlinks} 
%D 
%D The next switch can be used to make user hyperlinks are 
%D not highlighted when clicked on. 

\newif\ifhighlighthyperlinks 

%D \macros 
%D   {ifgotonewwindow}
%D 
%D To make the {\em goto previous jump} feature more 
%D convenient when using more than one file, it makes sense 
%D to force the viewer to open a new window for each file 
%D opened. 

\newif\ifgotonewwindow 

%D \macros 
%D   {ifPDFstrokecolor}
%D
%D We can reduce the filesize a bit by setting the next switch 
%D to false. The amount of reduction depends on the use of 
%D color, but don't expect more than a few percent. Zip 
%D compression is already rather efficient in itself. 

\newif\ifPDFstrokecolor \PDFstrokecolortrue

%D \macros 
%D   {dodoinsertfile,dofileinsertion,
%D    definefileinsertion,doiffileinsertionsupported}
%D 
%D File insertion depend on the driver or \TEX\ variant used.
%D All driver modules use the same scheme for file insertion,
%D and therefore have the next macro in common: 

\def\dododoinsertfile[#1][#2,#3][#4,#5]%
  {\def\fileinsertionclass{do#1insert}%
   \doifdefinedelse{\fileinsertionclass#3}
     {\def\next{\getvalue{\fileinsertionclass#3}}}
     {\doifdefinedelse{\fileinsertionclass#2}
        {\def\next{\getvalue{\fileinsertionclass#2}}}
        {\def\next{\gobbleninearguments}}}%
   \next{#4}{#5}}

\def\dodoinsertfile#1#2#3%
  {\dododoinsertfile[#1][#2][#3]}

\def\definefileinsertion#1#2%
  {\addtocommalist{#2}\supportedfileinsertions
   \setvalue{do#1insert#2}}

\def\dofileinsertion#1#2%
  {\getvalue{do#1insert#2}}

\newevery \everyresetspecials \relax

\appendtoks \let\supportedfileinsertions\empty \to \everyresetspecials

\let\supportedfileinsertions\empty

\def\doiffileinsertionsupportedelse#1%
  {\ExpandBothAfter\doifinstringelse{#1}{\c!tex,\c!tmp,\supportedfileinsertions}}

%D This macro is called with 10 arguments, where the first 
%D one specifies the driver, like \type{yy} or \type{tr}. The
%D second argument is a \type{{type,method}} pair and the 
%D third the filename. 

%D Just in case this one is undefined (which can happen if 
%D this module is used outside \CONTEXT): 

\ifx\setreferencefilename\undefined

  \def\setreferencefilename#1\to#2{\edef#2{#1}}
 
\fi

%D When submitting forms, we need to communicate the format.

\chardef\submitoutputformat=0 % 0=unknown 1=HTML 2=FDF

\def\setsubmitoutputformat#1%
  {\doifinsetelse{#1}{FDF,fdf}
     {\chardef\submitoutputformat=2} 
     {\chardef\submitoutputformat=1}} 

\protect

\endinput
