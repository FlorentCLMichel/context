%D \module
%D   [       file=core-sec,
%D        version=1997.03.31,
%D          title=\CONTEXT\ Core Macros,
%D       subtitle=Sectioning,
%D         author=Hans Hagen,
%D           date=\currentdate,
%D      copyright={PRAGMA / Hans Hagen \& Ton Otten}]
%C
%C This module is part of the \CONTEXT\ macro||package and is
%C therefore copyrighted by \PRAGMA. See mreadme.pdf for
%C details.

% start-stop per section en dan combineren met sectieblok; in dat geval
% eenvoudiger per-* acties

% nummeren per sectieblok implementeren

% this module needs a clean up, currently some manipulations
% take place multiple times; also, some clever recursive level
% thing makes more sense

% in manual (zie prikkels) : tussen=\blanko is enige hook om
% met kop-in-hoofd een spatiering af te dwingen

\writestatus{loading}{Context Core Macros / Sectioning}

\startmessages  dutch  library: structures
  title: structuur
      1: begin van sectieblok --
      2: eind van sectieblok --
\stopmessages

\startmessages  english  library: structures
  title: structure
      1: begin of sectionblock --
      2: end of sectionblock --
\stopmessages

\startmessages  german  library: structures
  title: struktur
      1: Begin des Abschnittsblock --
      2: Ende des Abschnittsblock --
\stopmessages

\startmessages  czech  library: structures
  title: struktury
      1: zacatek oddilu (sekce) --
      2: konec oddilu (sekce) --
\stopmessages

\startmessages  italian  library: structures
  title: struttura
      1: inizio del blocco (sezione) --
      2: fine del blocco (sezione) --
\stopmessages

\startmessages  norwegian  library: structures
  title: struktur
      1: starten av blokk -- (seksjon)
      2: slutten av blokk -- (seksjon)
\stopmessages

\startmessages  romanian  library: structures
  title: structuri
      1: inceput de bloc sectiune --
      2: sfarsit de bloc sectiune --
\stopmessages

\unprotect

% from now on, internaly numbers are separated by a period
% and postprocessed on demand

\def\numberseparator {.} % reasonable default
\def\sectionseparator{:} % was : but is now -

\def\@@filterfirstpart      [#1::#2]{#1}
\def\@@filtersecondpart     [#1::#2]{#2}

\def\@@filterblockpart      [#1::#2::#3]{#1}
\def\@@filternumberpart     [#1::#2::#3]{#2}
\def\@@filterpagepart       [#1::#2::#3]{#3}
\def\@@filterblocknumberpart[#1::#2::#3]{#1::#2}

\def\@@filterheadpart[#1]{\@EA\@@dofilterheadpart\@EA[#1:0]}
\def\@@filtertailpart[#1]{\@EA\@@dofiltertailpart\@EA[#1:0]}

\def\@@dofilterheadpart[#1:#2]{#1}
\def\@@dofiltertailpart[#1:#2]{#2}

\def\@@filterlevelpart[#1::#2::#3]{\@@dofilterlevelpart[#2:0:0:0:0]}

\def\@@dofilterlevelpart[#1:0:0:0:#2]{#1}

\def\gobbleuntilrelax#1\relax{}

\def\separatednumber  #1{\doseparatednumber  #1.\empty\relax}
\def\removefirstprefix#1{\doremovefirstprefix#1.\empty\relax}
\def\removeallprefixes#1{\doremoveallprefixes#1.\empty\relax}

\def\doseparatednumber#1.#2%
  {#1%
   \ifx#2\empty
     \@EA\gobbleuntilrelax
   \else \numberseparator
     \@EA\doseparatednumber
   \fi#2}

\def\doremoveallprefixes#1.#2%
  {\ifx#2\empty
     #1\@EA\gobbleuntilrelax
   \else
     \@EA\doremoveallprefixes
   \fi#2}

\def\doremovefirstprefix#1.#2%
  {\ifx#2\empty
     #1\@EA\gobbleuntilrelax
   \else
     \@EA\noremovefirstprefix
   \fi#2}

\def\noremovefirstprefix#1.\empty\relax
  {#1}

% we need to expand in order to get something separatable

\def\dohandleheadnumber#1%
  {\expanded{\separatednumber{#1}}}

\def\dodochecknummer#1#2#3% will become ugly after speed up
  {\bgroup
   \doifinstringelse{.0}{.#2}
     {\doifnot{#3}\v!by
        {%\debuggerinfo\m!systems{number #1 #3 becomes \getnumbervariable{#1\c!way}}%
         \setevalue{\@@thenumber{#1}\c!way}{#3}% geen \xdef, gaat mis met \subpage
         \dochecknummer{#1}}} % tricky and ugly
     {\doifnotvalue{\@@thenumber{#1}\s!check}{#2}
        {\setcounter{\@@thenumber{#1}}{0\getvalue{\@@thenumber{#1}\c!start}}%
         \setxvalue{\@@thenumber{#1}\c!way\c!local}%
           {\getvalue{\@@thenumber{#1}\c!way}}%
         \setxvalue{\@@thenumber{#1}\s!check}%
           {#2}}}%
   \egroup}

\def\dochecknummer#1%
  {\edef\currentsection{\csname\??by\csname\@@thenumber{#1}\c!way\endcsname\endcsname}%
   \ifx\currentsection\empty\else
     \dodochecknummer
       {#1}%
       {\csname\currentsection\c!number\endcsname}%
       {\v!by\previoussection\currentsection}%
   \fi}

\def\checknummer#1%
  {\bgroup
  %\ifcase\blocklevel\else
   \ifdoingblocks
     \doifnotvalue{\@@thenumber{#1}\c!blockway}\v!no\setblockcounters
   \fi
   \dochecknummer{#1}%
   \egroup}

\def\rawsectionnumber#1%
  {\countervalue{\??se#1}}

\def\domakeprecedingsectionnumber[#1]% will become ugly after speed up
  {\bgroup % added
   \globallet\precedingsectionnumber\empty
   \ifsectienummer
     \doifvalue{\??sb\@@sectionblock\c!number}\v!yes % added
       {\doifelsevalue{\@@thenumber{#1}\c!sectionnumber}\v!yes
          \donetrue\donefalse
        \doifvalue{\@@thenumber{#1}\c!sectionnumber}\v!number
          {\donetrue\let\@@sectionconversion\gobbleoneargument}%
        \ifdone
          \edef\currentsection
            {\getvalue{\??by\getvalue{\@@thenumber{#1}\c!way\c!local}}}%
          \doifnot\currentsection\zerosection
            {\doifnot{\@@sectionvalue\currentsection}{0}
               {\xdef\precedingsectionnumber%
                  {\getvalue{\currentsection\c!number}.}}}%
        \fi}%
   \fi
   \egroup}

\def\makeprecedingsectionnumber[#1]%
  {\bgroup
  %\ifnum\blocklevel>0
  %\ifcase\blocklevel\else
   \ifdoingblocks
     \doifnotvalue{\@@thenumber{#1}\c!blockway}\v!no\setblockcounters
   \fi
   \domakeprecedingsectionnumber[#1]%
   \egroup}

% \def\makesectionnumber[#1]%
%   {\makeprecedingsectionnumber[#1]%
%    \xdef\composedsectionnumber%
%      {\precedingsectionnumber\convertednumber[#1]}}%
%
% hack needed for chinese and oldstyle in normal tex, will change

\def\makesectionnumber[#1]%
  {\bgroup
   \forceunexpanded % i don't like this hack
   \makeprecedingsectionnumber[#1]%
   \xdef\composedsectionnumber% was \xdef maar dat gaat fout met font switches
     {\precedingsectionnumber\convertednumber[#1]}%
   \egroup}

\def\preparethenumber#1#2#3% {\??id#1} \number \result
  {\doifelsevaluenothing{#1\c!separator}
     {\let\numberseparator\empty
      \let#3#2}
     {% was \unexpanded \edef, but we need it unexpanded !
      \edef\numberseparator{\getvalue{#1\c!separator}}%
      \edef#3{\@EA\separatednumber\@EA{#2}}}}  % hm, etex

\def\sectionnumberonly[#1]%
  {\makesectionnumber[#1]%
   \composedsectionnumber}

\def\currentnumber[#1]% kan tekst hier weg ?
  {\dotextprefix{\getvalue{\@@thenumber{#1}\c!text}}\sectionnumberonly[#1]}

\def\nextnumber[#1][#2][#3]%
  {\incrementnumber[#1]%
   \currentnumber[#1]%
   \rawreference{#2}{#3}{\composedsectionnumber}}

% sectioning

\newcount\nofsections

\let\zerosection \v!text
\let\firstsection\empty
\let\lastsection \empty
\let\@@sectie    \empty
\let\@@koppeling \empty

\makecounter{\??se\v!text}

\letvalueempty{\??se\v!text\c!before}
\letvalueempty{\??se\v!text\c!after  }

\setvalue     {\v!text\c!number}{0}
\letvalueempty{\v!text\s!format}

\letvalueempty{\??sk\v!text}
\letvalueempty{\??sk        }

\letvalue{\??by               }\v!text
\letvalue{\??by\v!text       }\v!text
\letvalue{\??by\v!all       }\v!text
\letvalue{\??by\v!by         }\v!text
\letvalue{\??by\v!by\v!text }\v!text
\letvalue{\??by\v!by\v!all }\v!text
\letvalue{\??by\v!by\v!page}\v!text % see footnotes

%%%%%%%%% old

\def\dosetupsection[#1][#2]%
  {\getparameters[\??se#1][#2]%
   \doifelsevalue{\??se#1\c!previousnumber}\v!yes
     {\setvalue{#1\c!number}{\@@longsectionnumber{#1}}}
     {\setvalue{#1\c!number}{\@@shortsectionnumber{#1}}}}

\def\setupsection
  {\dodoubleargument\dosetupsection}

%%%%%%%%% new, multilingual

\def\dosetupsection[#1][#2][#3]%
  {\ifthirdargument
     \getparameters[\??se#1#2][#3]%
   \else
     \getparameters[\??se#1][#2]%
   \fi
   \doifelsevalue{\??se#1\c!previousnumber}\v!yes
     {\setvalue{#1\c!number}{\@@longsectionnumber {#1}}}
     {\setvalue{#1\c!number}{\@@shortsectionnumber{#1}}}}

\def\setupsection%
  {\dotripleempty\dosetupsection}

%%%%%%%%%

\def\docouplemarking[#1][#2]%
  {\doifdefinedelse{\??ko#2\c!section}
     {\docouplemarking[#1][\getvalue{\??ko#2\c!section}]}
     {\def\donexttrackcommando##1%
        {\edef\coupledmarkings{\getvalue{\??se##1\c!marking}}%
         \doifelse{##1}{#2}
           {\addtocommalist{#1}\coupledmarkings}
           {\removefromcommalist{#1}\coupledmarkings}%
         \setevalue{\??se##1\c!marking}{\coupledmarkings}%
         \donexttracklevel{##1}}%
      \donexttracklevel{\zerosection}}} % \firstsection

\def\couplemarking
  {\dodoubleargument\docouplemarking}

\def\decouplemarking[#1]%
  {\couplemarking[#1][]}

% \def\definesection[#1]%
%   {\doifundefined{\??se#1}
%      {\doifelsenothing\firstsection
%         {\def\firstsection{#1}%
%          \setevalue{\??se#1\c!before}{\v!text}%
%          \setevalue{\??se\v!text\c!after}{#1}}
%         {\setevalue{\??se\commalistelement\c!after}{#1}%
%          \setevalue{\??se#1\c!before}{\lastsection}%
%          \setevalue{\??se\lastsection\c!after}{#1}}%
%       \advance\nofsections \plusone
%       \setevalue{\??se#1\c!level}{\the\nofsections}%
%       \letvalue{\??se#1\c!after}\empty
%       \setvalue{\e!next#1}{\@@nextsectionnumber{#1}}%
%       \setvalue{#1\c!number}{\@@longsectionnumber{#1}}%
%       \setvalue{#1\s!format}{\@@longformatnumber{#1}}%
%       \setevalue{\??by#1}{#1}%
%       \setevalue{\??by\v!by#1}{#1}%
%       \makecounter{\??se#1}%
%       \edef\lastsection{#1}%
%       \setvalue{\??sk#1}{#1}%
%       \letvalue{\??se#1\c!marking}\empty
%       \setupsection[#1][\c!previousnumber=\v!yes]}}%

\def\definesection[#1]%
  {\doifundefined{\??se#1}
     {\doifelsenothing\firstsection
        {\def\firstsection{#1}%
         \setevalue{\??se#1\c!before}{\v!text}%
         \setevalue{\??se\v!text\c!after}{#1}}
        {\setevalue{\??se\commalistelement\c!after}{#1}%
         \setevalue{\??se#1\c!before}{\lastsection}%
         \setevalue{\??se\lastsection\c!after}{#1}}%
      \advance\nofsections \plusone
      \setevalue{\??se#1\c!level}{\the\nofsections}%
      \letvalue{\??se#1\c!after}\empty
      \setvalue{\e!next#1}{\@@nextsectionnumber{#1}}%
      \setvalue{#1\c!number}{\@@longsectionnumber{#1}}%
      \setvalue{#1\s!format}{\@@longformatnumber{#1}}%
      \setevalue{\??by#1}{#1}%
      \setevalue{\??by\v!by#1}{#1}%
      \makecounter{\??se#1}%
      \makecounter{\??se\v!last#1}% GB
      \edef\lastsection{#1}%
      \setvalue{\??sk#1}{#1}%
      \letvalue{\??se#1\c!marking}\empty
      \setupsection[#1][\c!previousnumber=\v!yes]}}%

\def\previoussection#1{\csname\??se#1\c!before\endcsname}
\def\nextsection    #1{\csname\??se#1\c!after  \endcsname}

\let\preservedsection\v!unknown % \def\preservedsection{\firstsection}

\def\checkpreservevalueafter#1% GB
  {\ifnum\getvalue{\??se#1\c!level}<\nofsections
     \edef\preservedsection{\getvalue{\??se#1\c!after}}%
     \ifconditional\@@resetsubheadnumbers
       \setcounter{\??se\v!last\preservedsection}\zerocount % {0}%
     \else
       \setcounter{\??se\v!last\preservedsection}{\countervalue{\??se\preservedsection}}%
     \fi
   \fi}

% \def\@@setsectionnumber#1#2%
%   {\letgvalueempty{\??se#1\s!start}% signal i.p.v. boolean
%    \setcounter{\??se#1}{#2}%
%    \ifconditional\@@resetsubheadnumbers\resetsectioncounters{#1}\fi
%    \checkpagecounter}

\def\@@setsectionnumber#1#2%
  {\letgvalueempty{\??se#1\s!start}% signal i.p.v. boolean
   \setcounter{\??se#1}{#2}%
   \checkpreservevalueafter{#1}% GB
   \resetsectioncounters{#1}%
   \checkpagecounter}

% \def\@@nextsectionnumber#1%
%   {\letgvalueempty{\??se#1\s!start}% signal i.p.v. boolean
%    \pluscounter{\??se#1}%
%    \ifconditional\@@resetsubheadnumbers\resetsectioncounters{#1}\fi
%    \checkpagecounter}

\def\@@nextsectionnumber#1% patched by GB
  {\letgvalueempty{\??se#1\s!start}% signal i.p.v. boolean
   \ifnum\countervalue{\??se\v!last#1}>\zerocount
     \setcounter{\??se#1}{\countervalue{\??se\v!last#1}}%
     \setcounter{\??se\v!last#1}\zerocount % {0}%
   \fi
   \pluscounter{\??se#1}%
   \checkpreservevalueafter{#1}%
   \resetsectioncounters{#1}%
   \checkpagecounter}

\def\@@sectionvalue#1%       % nog niet overal doorgevoerd
  {\countervalue{\??se#1}}   % zoeken op \??se

% \def\@@sectionconversion#1%
%   {\getvalue{\??cv\getvalue{\??se#1\@@sectionblock\c!conversion}}}

% suited for chinese too:

% \def\@@sectionconversion#1#2% a doublure with \@@shortsectionnumber
%   {\ifnum#2=0 0\else % else troubles with \uchar
%      \@EA\ifx\csname\??se#1\@@sectionblock\c!conversion\endcsname\relax
%        \@EA\ifx\csname\??se#1\c!conversion\endcsname\relax
%          #2%
%        \else
%          \getvalue{\??cv\getvalue{\??se#1\c!conversion}}{#2}%
%        \fi
%      \else
%        \getvalue{\??cv\getvalue{\??se#1\@@sectionblock\c!conversion}}{#2}%
%      \fi
%    \fi}

\def\@@sectionconversion#1#2% a doublure with \@@shortsectionnumber
  {\ifnum#2=0 0\else % else troubles with \uchar
     \@EA\ifx\csname\??se#1\@@sectionblock\c!conversion\endcsname\relax
       \@EA\ifx\csname\??se#1\c!conversion\endcsname\relax
         #2%
       \else
         \convertnumber{\getvalue{\??se#1\c!conversion}}{#2}%
       \fi
     \else
       \convertnumber{\getvalue{\??se#1\@@sectionblock\c!conversion}}{#2}%
     \fi
   \fi}

\def\@@sectionlevel#1%
  {\ifundefined{\??se#1\c!level}0\else\getvalue{\??se#1\c!level}\fi}

% Omdat een markering kan worden herdefinieerd moeten we
% eerst testen of er wel een keten||afhankelijkheid is.

\def\resetsectionmarks#1% can invoke a break
  {\ifundefined{\??se#1}%
     \fastresetmarker[\mainmarking{#1}]% % redundant \mainmarking
   \else
     \let\donexttrackcommando\doresetsectionmarks
     \donexttracklevel{#1}%
   \fi}

\def\doresetsectionmarks#1%
  {\ifundefined{\??se#1\c!marking}\else % skip zero level
     \fastresetmarkerlist[\csname\??se#1\c!marking\endcsname]%
   \fi
   \donexttracklevel{#1}}

% I'm not sure if the next one is better:
%
% \def\doresetsectionmarks#1%
%   {\ifundefined{\??se#1\c!markering}% skip zero level
%      \donexttracklevel{#1}%
%    \else
%      \fastresetmarkerlist[\csname\??se#1\c!markering\endcsname]%
%    \fi}
%
% and indeed, it isn't, actually, it does not work at all, so let's drop it.

% packaged:
%
% \def\resetsectioncounters#1%
%   {\def\donexttrackcommando##1%
%      {\resetcounter{\??se##1}%
%       \donexttracklevel{##1}}%
%    \donexttracklevel{#1}}
%
% nicer
%
% \def\doresetsectioncounters#1%
%   {\resetcounter{\??se#1}%
%    \donexttracklevel{#1}}
%
% obey eigennummer

\def\doresetsectioncounters#1%
  {\resetcounter{\??se#1}%
   \letgvalue{\??se#1\c!ownnumber}\relax
   \donexttracklevel{#1}}

\def\resetsectioncounters % #1
  {\let\donexttrackcommando\doresetsectioncounters
   \donexttracklevel} % #1

% bij checken kan geen prefix worden bekeken, anders vallen
% er titels buiten de inhoudsopgave

% evt ook level gaan opslaan tbv snelle selectie

% \def\makesectionformat
%   {\edef\sectionformat
%      {\@@sectiontype\sectionseparator
%       \csname\lastsection\s!format\endcsname}}

\unprotected \def\makesectionformat % we don't want eigennummers here
  {\pushmacro\@@shortsectionnumber
   \let\@@shortsectionnumber\@@sectionvalue
   \edef\sectionformat
     {\@@sectiontype\sectionseparator
      \csname\lastsection\s!format\endcsname}%
   \popmacro\@@shortsectionnumber}

\def\dobacktracklevel#1%
  {\doifnot{\previoussection{#1}}\zerosection
     {\dobacktrackcommando{\previoussection{#1}}}}

\def\donexttracklevel#1%
  {\doifnot{#1}\lastsection
     {\donexttrackcommando{\nextsection{#1}}}}

\chardef\alltoclevels\zerocount

\let\currentlevel\empty

\def\dosetcurrentlevel#1%
  {\global\chardef\alltoclevels\zerocount
   \xdef\currentlevel{\getvalue{\lastsection\s!format}}}

\def\dosetpreviouslevel#1%
  {\global\chardef\alltoclevels\plusone
   \globallet\currentlevel\empty
   \def\dobacktrackcommando##1%
     {\ifnum\countervalue{\??se##1}>\zerocount
        \global\chardef\alltoclevels\zerocount
        \xdef\currentlevel{\getvalue{\previoussection{##1}\s!format}}%
      \else
        \dobacktracklevel{##1}%
      \fi}%
   \dobacktrackcommando\lastsection}

\def\dosettextlevel#1%
  {\global\chardef\alltoclevels\plusone
   \globallet\currentlevel\empty}

\def\dosetotherlevel#1%
  {\doifdefinedelse{\??ko#1\c!section}              % beter alteratief: ook
     {\edef\@@sectie{\getvalue{\??ko#1\c!section}}} % hoofdstuk\c!format
     {\edef\@@sectie{#1}}%
   \doifdefinedelse{\??se\@@sectie}
     {\global\chardef\alltoclevels\zerocount
      \xdef\currentlevel{\getvalue{\@@sectie\s!format}}}
     {\global\chardef\alltoclevels\plusone
      \globallet\currentlevel\empty
      \def\dobacktrackcommando##1%
        {\@EA\ifx\csname\??se##1\c!start\endcsname\relax
           \dobacktracklevel{##1}%
         \else
           \ifnum\countervalue{\??se##1}>\zerocount
             \global\chardef\alltoclevels\zerocount
             \xdef\currentlevel{\getvalue{##1\s!format}}%
           \else
             \dobacktracklevel{##1}%
           \fi
         \fi}%
      \dobacktrackcommando\lastsection}}

\def\dosetfilterlevel#1#2% beware: this one is \let
  {\bgroup
   \edef\askedlevel{#1}%
   \edef\askedfilter{#2}%
   \ifx\askedlevel\v!current
     \dosetcurrentlevel\askedlevel
   \else\ifx\askedlevel\v!previous
     \dosetpreviouslevel\askedlevel
   \else\ifx\askedlevel\v!all
     \global\chardef\alltoclevels\plusone
   \else\ifx\askedlevel\v!text
     \global\chardef\alltoclevels\plusone
   \else
     \edef\byaskedlevel{\csname\??by\askedlevel\endcsname}%
     \ifx\byaskedlevel\v!text
       \dosettextlevel\askedlevel
     \else
       \dosetotherlevel\askedlevel
     \fi
   \fi\fi\fi\fi
   % experiment
   \ifx\askedfilter\empty \else
     \xdef\currentlevel{\currentlevel\sectionseparator\askedfilter}%
   \fi
   \egroup}

% \def\dontsetfilterlevel#1#2%
%   {\let\currentlevel\somesavedlevel
%    \chardef\alltoclevels\zerocount}

\def\dontsetfilterlevel#1#2%
  {\let\currentlevel\somesavedlevel
   \let\@@sectiontype\@@tocsectiontype
   \chardef\alltoclevels\zerocount}

\def\honorlocalfilterlevel  % local lists will be real local
  {\let\dosetfilterlevel\dontsetfilterlevel}

% cleaner
%
% \def\doifnextlevelelse[#1::#2]#3#4%
%   {\ifcase\alltoclevels
%      \doifelse{\@@sectiontype}{#1}
%        {\doifinstringelse{=\currentlevel:}{=:#2:}
%           {\doifinstringelse{=\currentlevel:0}{=:#2:}{#4}{#3}}
%           {#4}}
%        {#4}%
%    \else
%      #3%
%    \fi}
%
% \def\doifprevlevelelse[#1::#2]#3#4%
%   {\ifcase\alltoclevels
%      \doifelse{\@@sectiontype}{#1}
%        {\doifinstringelse{=\currentlevel:}{=:#2:}{#3}{#4}}
%        {#4}%
%    \else
%      #3%
%    \fi}
%
% faster
%
% \def\doifnextlevelelse[#1::#2]%
%   {\ifcase\alltoclevels
%      \doifelse{\@@sectiontype}{#1}
%        {\doifinstringelse{=\currentlevel:}{=:#2:}
%           {\doifinstringelse{=\currentlevel:0}{=:#2:}\donefalse\donetrue}
%           \donefalse}
%        \donefalse
%    \else
%      \donetrue
%    \fi
%    \ifdone
%      \expandafter\firstoftwoarguments
%    \else
%      \expandafter\secondoftwoarguments
%    \fi}
%
% \def\doifprevlevelelse[#1::#2]%
%   {\ifcase\alltoclevels
%      \doifelse{\@@sectiontype}{#1}
%        {\doifinstringelse{=\currentlevel:}{=:#2:}\donetrue\donefalse}
%        \donefalse
%    \else
%      \donetrue
%    \fi
%    \ifdone
%      \expandafter\firstoftwoarguments
%    \else
%      \expandafter\secondoftwoarguments
%    \fi}
%
% meaner
%
% \setuplist
%   [chapter]
%   [after={\startcolumns\placelist[section]\stopcolumns}]

\def\somesavedlevel{0}

% \def\dosavesomelevel[#1:0:0:0:#2]%
%   {\def\somesavedlevel{:#1}}

% \def\doifnextlevelelse[#1::#2]%
%   {\dosavesomelevel[#2:0:0:0:0]%
%    \ifcase\alltoclevels
%      \doifelse{\@@sectiontype}{#1}
%        {\doifinstringelse{=\currentlevel:}{=:#2:}
%           {\doifinstringelse{=\currentlevel:0}{=:#2:}\donefalse\donetrue}
%           \donefalse}
%        \donefalse
%    \else
%      \donetrue
%    \fi
%    \ifdone
%      \expandafter\firstoftwoarguments
%    \else
%      \expandafter\secondoftwoarguments
%    \fi}
%
% \def\doifprevlevelelse[#1::#2]%
%   {\dosavesomelevel[#2:0:0:0:0]%
%    \ifcase\alltoclevels
%      \doifelse{\@@sectiontype}{#1}
%        {\doifinstringelse{=\currentlevel:}{=:#2:}\donetrue\donefalse}
%        \donefalse
%    \else
%      \donetrue
%    \fi
%    \ifdone
%      \expandafter\firstoftwoarguments
%    \else
%      \expandafter\secondoftwoarguments
%    \fi}
%
% again faster:

% \def\doifnextlevelelse[#1::#2]% beware: this one is \let
%  {\dosavesomelevel[#2:0:0:0:0]%
%   \ifcase\alltoclevels
%     \ifnum\@@sectiontype=#1
%       \def\levelstring{=:#2:}%
%       \doifincsnameelse{=\currentlevel:}\levelstring
%         {\doifincsnameelse{=\currentlevel:0}\levelstring\donefalse\donetrue}
%         \donefalse
%     \else
%       \donefalse
%     \fi
%   \else
%     \donetrue
%   \fi
%   \ifdone
%     \expandafter\firstoftwoarguments
%   \else
%     \expandafter\secondoftwoarguments
%   \fi}
%
%\def\doifprevlevelelse[#1::#2]% beware: this one is \let
%  {\dosavesomelevel[#2:0:0:0:0]%
%   \ifcase\alltoclevels
%     \ifnum\@@sectiontype=#1
%       \doifinstringelse{=\currentlevel:}{=:#2:}\donetrue\donefalse
%     \else
%       \donefalse
%     \fi
%   \else
%     \donetrue
%   \fi
%   \ifdone
%     \expandafter\firstoftwoarguments
%   \else
%     \expandafter\secondoftwoarguments
%   \fi}
%
% \let\doiftoclevelelse\doifnextlevelelse
% \let\doifreglevelelse\doifprevlevelelse
% \let\doifblklevelelse\doifprevlevelelse
%
% we want to be able to overload them globally

% This will be reimplemented some day soon
%
% {nn}{xx}{yy}
%
% -> \scan{..}{..}{0} met 0 als sentinel

% still not perfect
%
% \def\doifnextlevelelse[#1]% !! this one is \let / uti seperator --
%   {\edef\somesavedlevel{\sectionseparator\@@filterlevelpart[#1]}%
%    \ifcase\alltoclevels
%      \ifnum\@@sectiontype=\@@filterblockpart[#1]\relax
%        \edef\levelstring{=\sectionseparator\@@filternumberpart[#1]\sectionseparator}%
%        \doifincsnameelse{=\currentlevel\sectionseparator}\levelstring
%          {\doifincsnameelse{=\currentlevel\sectionseparator0}\levelstring
%             \donefalse
%             \donetrue}
%          \donefalse
%      \else
%        \donefalse
%      \fi
%    \else
%      \donetrue
%    \fi
%    \ifdone
%      \expandafter\firstoftwoarguments
%    \else
%      \expandafter\secondoftwoarguments
%    \fi}
%
% \def\doifprevlevelelse[#1]% !! this one is \let / uti seperator --
%   {\edef\somesavedlevel{\sectionseparator\@@filterlevelpart[#1]}%
%    \ifcase\alltoclevels
%      \ifnum\@@sectiontype=\@@filterblockpart[#1]\relax
%        \doifinstringelse
%          {=\currentlevel\sectionseparator}
%          {=\sectionseparator\@@filternumberpart[#1]\sectionseparator}
%          \donetrue\donefalse
%      \else
%        \donefalse
%      \fi
%    \else
%      \donetrue
%    \fi
%    \ifdone
%      \expandafter\firstoftwoarguments
%    \else
%      \expandafter\secondoftwoarguments
%    \fi}

\def\doifnextlevelelse[#1]% !! this one is \let / uti seperator --
  {\edef\somesavedlevel{\sectionseparator\@@filterlevelpart[#1]}%
   \edef\@@tocsectiontype{\@@filterblockpart[#1]}% needed for nested tocs
   \ifcase\alltoclevels
     \ifnum\@@sectiontype=\@@tocsectiontype\relax
       \edef\levelstring{=\sectionseparator\@@filternumberpart[#1]\sectionseparator}%
       \doifincsnameelse{=\currentlevel\sectionseparator}\levelstring
         {\doifincsnameelse{=\currentlevel\sectionseparator0}\levelstring
            \donefalse
            \donetrue}
         \donefalse
     \else
       \donefalse
     \fi
   \else
     \donetrue
   \fi
   \ifdone
     \expandafter\firstoftwoarguments
   \else
     \expandafter\secondoftwoarguments
   \fi}

\def\doifprevlevelelse[#1]% !! this one is \let / uti seperator --
  {\edef\somesavedlevel{\sectionseparator\@@filterlevelpart[#1]}%
   \edef\@@tocsectiontype{\@@filterblockpart[#1]}% needed for nested tocs
   \ifcase\alltoclevels
     \ifnum\@@sectiontype=\@@tocsectiontype\relax
       \doifinstringelse
         {=\currentlevel\sectionseparator}
         {=\sectionseparator\@@filternumberpart[#1]\sectionseparator}
         \donetrue\donefalse
     \else
       \donefalse
     \fi
   \else
     \donetrue
   \fi
   \ifdone
     \expandafter\firstoftwoarguments
   \else
     \expandafter\secondoftwoarguments
   \fi}

% we need to cover the special case of nested lists in section blocks
%
% \starttext
%
% \def\ChapterEntry#1#2#3%
%   {chapter : \hbox to \hsize{\strut\bf#2\hss#3}\endgraf\placelist[section]}
%
% \startfrontmatter % optional
%   \placelist[chapter][alternative=command,command=\ChapterEntry,criterium=text] \page
% \stopfrontmatter  % optional
%
% \startbodymatter % optional
%   \chapter{first}  \section{one}   test \section{two}  test \page
%   \chapter{second} \section{alpha} test \section{beta} test \page
% \stopbodymatter  % optional
%
% \stoptext

\def\doiftoclevelelse{\doifnextlevelelse}
\def\doifreglevelelse{\doifprevlevelelse}
\def\doifblklevelelse{\doifprevlevelelse}

\def\@@longformatnumber#1%
  {\csname\previoussection{#1}\s!format\endcsname
   \sectionseparator
   \@@shortsectionnumber{#1}}

% \def\@@longsectionnumber#1%
%   {\ifnum\countervalue{\??se\previoussection{#1}}>\zerocount
%      \csname\previoussection{#1}\c!nummer\endcsname.%
%    \fi
%    \@@shortsectionnumber{#1}}

\def\@@longsectionnumber#1%
  {\ifreversesectionnumbers
     \@@shortsectionnumber{#1}%
     \ifnum\countervalue{\??se\previoussection{#1}}>\zerocount
       .\csname\previoussection{#1}\c!number\endcsname
     \fi
   \else
     \ifnum\countervalue{\??se\previoussection{#1}}>\zerocount
       \csname\previoussection{#1}\c!number\endcsname.%
     \fi
     \@@shortsectionnumber{#1}%
   \fi}

% suited for chinese too:
%
% \def\@@shortsectionnumber#1%
%   {\@EA\ifx\csname\??se#1\@@sectionblock\c!conversie\endcsname\relax
%      \@@sectionvalue{#1}%
%    \else
%      \@@sectionconversion{#1}{\@@sectionvalue{#1}}%
%    \fi}
%
% obey eigennummer
%
% \def\@@shortsectionnumber#1%
%  {\@EA\ifx\csname\??se#1\c!eigennummer\endcsname\relax
%     \@EA\ifx\csname\??se#1\@@sectionblock\c!conversie\endcsname\relax
%       \@EA\ifx\csname\??se#1\c!conversie\endcsname\relax
%         \@@sectionvalue{#1}%
%       \else
%         \@@sectionconversion{#1}{\@@sectionvalue{#1}}%
%       \fi
%     \else
%       \@@sectionconversion{#1}{\@@sectionvalue{#1}}%
%     \fi
%   \else
%     \csname\??se#1\c!eigennummer\endcsname
%   \fi}

\def\@@shortsectionnumber#1%
  {\@EA\ifx\csname\??se#1\c!ownnumber\endcsname\relax
     \@EA\ifx\csname\??se#1\@@sectionblock\c!conversion\endcsname\relax
       \@EA\ifx\csname\??se#1\c!conversion\endcsname\relax
         \@@sectionvalue{#1}%
       \else
         \@@sectionconversion{#1}{\@@sectionvalue{#1}}%
       \fi
     \else
       \@@sectionconversion{#1}{\@@sectionvalue{#1}}%
     \fi
   \else
     \csname\??se#1\c!ownnumber\endcsname
   \fi}

\def\dosetlocalsectionblock#1#2#3% new \edef's
  {\edef\@@sectiontype  {#1}%
   \edef\@@sectionblock {#2}%
   \edef\@@sectionblocks{#3}}

% beware, the \resetsectionmarks generates some nodes that
% will result in an additional last page, which needs to be
% captured at the end

% \def\doaroundsectionblock#1%
%   {\doifvaluesomething{\??sb#1\c!page}
%      {\ExpandFirstAfter\page[\getvalue{\??sb#1\c!page}]}%
%    \resetsectioncounters\zerosection % was firstsection
%    \resetsectionmarks\zerosection}

% \def\dostartsectionblock#1#2%
%   {\begingroup
%    \doaroundsectionblock{#1}%              % going to a new page or so
%    \getvalue{\??sb#1}%                     % set name of section block
%    \getsectionblockenvironment{#1}%        % special settings, grouped
%   %\expandafter\csname#2true\endcsname     % obsolete
%    \setsystemmode{#1}%                     % can be used in conditionals
%    \getvalue{\??sb\@@sectionblock\c!before}% this one is not to be moved!
%    \showmessage\m!structures1\@@sectionblocks}

% \def\dostopsectionblock
%   {\showmessage\m!structures2\@@sectionblocks
%    \getvalue{\??sb\@@sectionblock\c!after}% don't move
%    \doaroundsectionblock\@@sectionblock
%    \endgroup}

\def\doaroundsectionblock
  {\doifvaluesomething{\??sb\@@sectionblock\c!page}
     {\page[\getvalue{\??sb\@@sectionblock\c!page}]}%
   \resetsectioncounters\zerosection % was firstsection
   \resetsectionmarks\zerosection}

\def\dostartsectionblock#1#2%
  {\begingroup
   \getvalue{\??sb#1}%
   \doaroundsectionblock
%    \doifvaluesomething{\??sb\@@sectionblock\c!page}{\page[\getvalue{\??sb\@@sectionblock\c!page}]}%
%    \resetsectioncounters\zerosection % was firstsection
%    \resetsectionmarks\zerosection
   \getsectionblockenvironment\@@sectionblock
   \setsystemmode\@@sectionblock
   \getvalue{\??sb\@@sectionblock\c!before}%
   \showmessage\m!structures1\@@sectionblocks}

\def\dostopsectionblock
  {\showmessage\m!structures2\@@sectionblocks
   \getvalue{\??sb\@@sectionblock\c!after}% don't move
   \doaroundsectionblock
%    \doifvaluesomething{\??sb\@@sectionblock\c!page}{\page[\getvalue{\??sb\@@sectionblock\c!page}]}%
%    \resetsectioncounters\zerosection % was firstsection
%    \resetsectionmarks\zerosection
   \endgroup}

\def\dosetupsectionblock[#1]% [#2]
  {\getparameters[\??sb#1]}

\def\setupsectionblock
  {\dodoubleargument\dosetupsectionblock}

\long\def\setsectionblockenvironment#1#2%
  {\long\setvalue{\??sb\s!do#1}{\do{#2}}}

\def\getsectionblockenvironment#1%
  {\let\do\firstofoneargument\getvalue{\??sb\s!do#1}}

\setvalue{\e!start\v!sectionblockenvironment}%
  {\dosingleargument\dostartsectionblockenvironment}

\def\dostartsectionblockenvironment[#1]% evt \pushendofline \popendofline
  {\long\def\do##1##2{\setsectionblockenvironment{#1}{##1##2}}%
   \grabuntil{\e!stop\v!sectionblockenvironment}{\getvalue{\??sb\s!do#1}}}

%D \starttyping
%D \startsectionblockenvironment[frontpart]
%D   \setuppagenumbering[conversion=romannumerals]
%D \stopsectionblockenvironment
%D
%D \startsectionblockenvironment[bodypart]
%D   \setuppagenumber[number=1]
%D \stopsectionblockenvironment
%D
%D \startsectionblockenvironment[frontpart]
%D   \setuppagenumbering[conversion=character]
%D \stopsectionblockenvironment
%D
%D \starttext
%D   \startfrontmatter \chapter{test} \stopfrontmatter
%D   \startbodymatter  \chapter{test} \stopbodymatter
%D   \startappendices  \chapter{test} \stopappendices
%D \stoptext
%D \stoptyping

% We used to use the first char as id, but a counter is
% better, because in english we get a name clash.

\newcounter\currentsectionblock

\def\currentsection{\@@sectionblock}

\def\dodefinesectionblock[#1][#2][#3]%
  {\getparameters
     [\??sb#1]
     [\c!number=\v!yes,
      \c!page=\v!right, % anders worden marks te vroeg gereset !
      %\c!before=,
      %\c!after=,
      #3]%
   \expandafter\newif\csname if#2\endcsname % better a mode
   \doglobal\increment\currentsectionblock
   \setsectionblockenvironment{#1}{}%
   \setevalue{\??sb#1}%
     {\noexpand\dosetlocalsectionblock{\currentsectionblock}{#1}{#2}}%
   \setvalue{\e!start#2}%
     {\dostartsectionblock{#1}{#2}}%
   \setvalue{\e!stop#2}%
     {\dostopsectionblock}}

\def\definesectionblock
  {\dotripleargument\dodefinesectionblock}

\def\sectionblocklabel#1#2%
  {\@EA\ifx\csname\??ko#1\@@sectionblock\c!label\endcsname\relax
     \labeltexts{#1}{#2}%
   \else
     \labeltexts{\getvalue{\??ko#1\@@sectionblock\c!label}}{#2}%
   \fi}

\dosetlocalsectionblock{2}{\v!bodypart}{\v!bodymatter} % hm, dirty

\def\setsectiontype[#1]%
  {\getvalue{\??sb#1}}

\def\writesection#1#2#3% #3 -> \asciititle
  {\bgroup
   \edef\!!stringa{#1}%
   \@EA\writestatus\@EA
     {\!!stringa}
     {\ifsectienummer#2\else(#2)\fi\normalspace\asciititle}%
   \egroup}

\def\@@kolevel{1}        \def\headlevel{\@@kolevel}

\def\dohandlepagebreakAA#1%
  {\ifnum\lastpenalty>0
     \global\paginageblokkeerdtrue
   \fi}

\def\dohandlepagebreakAB#1% will be replaced by a more clever (signaling) mechanism (in beta)
  {\flushsidefloats
   \getvalue{\??ko#1\c!before}%
   %\whitespace vervangen door \noindent elders
   \relax
   \ifpaginageblokkeerd
     \global\paginageblokkeerdfalse
   \else
     \!!countb\getvalue{\??se\@@sectie\c!level}\relax
     \ifnum\!!countb>\@@kolevel\relax
       \!!counta20000
       \multiply\!!countb 500
       \advance\!!counta \!!countb
       \dosomebreak{\penalty\!!counta}%
     \else
       \dosomebreak\allowbreak
     \fi
   \fi
   \xdef\@@kolevel{\getvalue{\??se\@@sectie\c!level}}}

\def\dohandlepagebreakBB#1#2#3%
  {\doifinset{\getvalue{\??tk#2\c!state}}{\v!normal,\v!start}
     {\doifvaluesomething{\??ko#1#3}
        {\setuplayouttext[#2][\c!state=\getvalue{\??ko#1#3}]}}}

\def\dohandlepagebreakB#1%
  {\doifvaluesomething{\??ko#1\c!page}
     {\def\resetcurrentsectionmarks% toegevoegd, zie \page
        {\resetsectionmarks{\previoussection\@@sectie}}%
      \page[\getvalue{\??ko#1\c!page}]%
      \dohandlepagebreakBB{#1}\v!header\c!header
      \dohandlepagebreakBB{#1}\v!text  \c!text
      \dohandlepagebreakBB{#1}\v!footer\c!footer}}

\def\dohandlepagebreakX#1% zie doordefinieren / boven
  {\bgroup
   \!!countb\@@kolevel
   \advance\!!countb #1
   \multiply\!!countb 500
   \!!counta20000
   \advance\!!counta \!!countb
   \dosomebreak{\penalty\!!counta}%
   \egroup}

\def\handlepagebreak#1%
  {\dohandlepagebreakAA{#1}%
   \ifnum\countervalue{\??se\previoussection\@@sectie}>\zerocount\relax
     \ifnum\countervalue{\??se\@@sectie}>\zerocount
       \dohandlepagebreakB{#1}%
     \else
       \doifnotvalue{\??ko#1\c!continue}\v!yes{\dohandlepagebreakB{#1}}%
     \fi
   \else
     \dohandlepagebreakB{#1}%
   \fi
   \dohandlepagebreakAB{#1}}

\def\handlepagebreakC#1%
  {\xdef\@@kolevel{\getvalue{\??se\@@sectie\c!level}}%
   \nobreak}

\def\localheadheight    {\strutht}
\def\localheaddepth     {\strutdp}
\def\localheadlineheight{\lineheight}

\def\dolocalheadsetup#1%  koppeling met standaard kopcommando / engels
  {\forgetall         %  traag dus ...
   \doifvaluesomething{\??ko#1\c!align} % wordt al expanded in spa
     {\expanded{\setupalign[\getvalue{\??ko#1\c!align}]}}%
   \doifvaluesomething{\??ko#1\c!tolerance} % wordt al expanded in spa
     {\expanded{\setuptolerance[\getvalue{\??ko#1\c!tolerance}]}}%
   \def\\{\crlf\strut\ignorespaces}}

\def\localkopsetup{\localheadsetup} % kan tzt weg

% todo: make them conditionals:

\newif\ifincrementnumber
\newif\ifreversesectionnumbers  % todo: key/val
\newif\ifsectienummer           \sectienummertrue
\newif\ifdisplaysectionhead     \displaysectionheadtrue
\newif\ifplacehead
\newif\ifemptyhead
\newif\ifwritetolist
\newif\ifheadnumber
\newif\ifcontinuoushead
\newif\ifheadnumbercontent      % niet meer wijzigen / wordt mode
\newif\ifheadprefix
\newif\ifsomeheadconversion

% new

\newconditional\@@resetsubheadnumbers

\def\setsectieenkoppeling#1%
  {\edef\@@koppeling{\getvalue{\??ko#1\c!coupling}}%
   \edef\@@sectie{\getvalue{\??ko#1\c!section}}%
   \doifnothing\@@koppeling
     {\edef\@@koppeling{#1}}%
   \doifnothing\@@sectie
     {\edef\@@sectie{\getvalue{\??ko\@@koppeling\c!section}}}}

% \handlepagebreak komt het eerst omdat eventueel
% subpaginanummers moeten worden afgehandeld. Vervolgens
% worden de nummers opgehoogd en referenties geset, dan
% volgt de kop en tot slot de worden de marks en de prefix
% geset.

% \hoofdstuk {tekst}
% \hoofdstuk tekst
% \hoofdstuk <niets>

\let\finalsectionnumber\empty

\def\dofinalsectionnumber
  {\ifundefined{\@@sectie\c!number}\else
     \ifsomeheadconversion
       \@@shortsectionnumber\@@sectie
     \else
       \getvalue{\@@sectie\c!number}%
     \fi
   \fi}

\def\findsectionnumber#1#2#3% class file title / uti seperator --
  {\begingroup
   \setsectieenkoppeling{#1}%
   \xdef\foundsectionnumber{1}%
   \def\dolistelement##1##2##3##4##5##6%
     {\doif{##1}{#1}
        {\ConvertConstantAfter\doif{##4}{#3}
           {\global\utilitydonetrue
            \scratchcounter=0\getvalue{\??se\@@sectie\c!level}%
            %
            %\advance\scratchcounter 2
            %\@EA\def\@EA\do\@EA####\@EA1\sectionseparator####2]%
            %  {\advance\scratchcounter -1
            %   \ifcase\scratchcounter
            %     \xdef\foundsectionnumber{####1}%
            %   \else
            %     \do####2]%
            %   \fi}%
            %\do##5]}}}%
            %
            \def\do####1\relax % :/- clean
              {\advance\scratchcounter \minusone
               \ifcase\scratchcounter
                 \xdef\foundsectionnumber{\@@filterheadpart[####1]}%
               \else
                 \@EAEAEA\do\@@filtertailpart[####1]\relax
               \fi}%
            \@EA\do\@@filternumberpart[##5]\relax}}}%
   \setbox0\vbox
     {\doutilities{#1}{#2}{#1}\relax\relax}%
   \endgroup
   \doifnumberelse\foundsectionnumber
     {\doif\foundsectionnumber\!!zerocount
        {\globallet\foundsectionnumber\!!plusone}}
     {\globallet\foundsectionnumber\!!plusone}% an appendix or so
   \setupheadnumber[#1][\foundsectionnumber]%
   \setupheadnumber[#1][-1]}

% \def\setsomeheadconversion#1#2%
%    {\someheadconversionfalse
%     \doifelsevalue{\??ko#1\c!eigennummer}\v!ja
%       {\def\someheadconversion{#2}}
%       {\bepaalkopnummer[#1]%
%        \@EA\ifx\csname\??se\@@sectie\@@sectionblock\c!kopconversie\endcsname\relax
%          \@EA\ifx\csname\??se\@@sectie\c!kopconversie\endcsname\relax
%            \def\someheadconversion{#2}%
%          \else
%            \@EA\ifx\csname\??se\@@sectie\c!kopconversie\endcsname\empty
%              \def\someheadconversion{#2}%
%            \else
%              \someheadconversiontrue
%              \def\someheadconversion%
%                {\fullsectionnumber{#1}{\getvalue{\??se\@@sectie\c!kopconversie}}{#2}}%
%            \fi
%          \fi
%        \else
%          \@EA\ifx\csname\??se\@@sectie\@@sectionblock\c!kopconversie\endcsname\empty
%            \def\someheadconversion{#2}%
%          \else
%            \someheadconversiontrue
%            \def\someheadconversion%
%              {\fullsectionnumber{#1}{\getvalue{\??se\@@sectie\@@sectionblock\c!kopconversie}}{#2}}%
%          \fi
%        \fi}}
%
% deal with eigennummer

\def\setsomeheadconversion#1#2%
   {\someheadconversionfalse
    \doifelsevalue{\??ko#1\c!ownnumber}\v!yes
      {\setgvalue{\??se\@@sectie\c!ownnumber}{#2}%
       \def\someheadconversion{#2}}
      {\letgvalue{\??se\@@sectie\c!ownnumber}\relax
       \determineheadnumber[#1]%
       \@EA\ifx\csname\??se\@@sectie\@@sectionblock\c!headconversion\endcsname\relax
         \@EA\ifx\csname\??se\@@sectie\c!headconversion\endcsname\relax
           \def\someheadconversion{#2}%
         \else
           \@EA\ifx\csname\??se\@@sectie\c!headconversion\endcsname\empty
             \def\someheadconversion{#2}%
           \else
             \someheadconversiontrue
             \def\someheadconversion%
               {\fullsectionnumber{#1}{\getvalue{\??se\@@sectie\c!headconversion}}{#2}}%
           \fi
         \fi
       \else
         \@EA\ifx\csname\??se\@@sectie\@@sectionblock\c!headconversion\endcsname\empty
           \def\someheadconversion{#2}%
         \else
           \someheadconversiontrue
           \def\someheadconversion%
             {\fullsectionnumber{#1}{\getvalue{\??se\@@sectie\@@sectionblock\c!headconversion}}{#2}}%
         \fi
       \fi}}

            \def   \writtenfullsectionnumber      {\string\fullsectionnumber}
            \def   \ignoredfullsectionnumber#1#2#3{#3}
            \let    \storedfullsectionnumber       \relax
%           \def\expandablefullsectionnumber#1#2#3{#3}
%\unexpanded\def   \naturalfullsectionnumber#1#2#3{\sectionblocklabel{#1}{\getvalue{\??cv#2}{#3}}}
%\unexpanded\def   \limitedfullsectionnumber#1#2#3{\getvalue{\??cv#2}{#3}}

% under test:
%
%           \def\expandablefullsectionnumber#1#2#3{\getvalue{\??cv#2}{#3}}

\def\expandablefullsectionnumber#1#2#3%
  {\convertnumber{#2}{#3}}

\unexpanded\def\naturalfullsectionnumber#1#2#3%
  {\sectionblocklabel{#1}{\convertnumber{#2}{#3}}}

\unexpanded\def\limitedfullsectionnumber#1#2#3%
  {\convertnumber{#2}{#3}}

\def\setfullsectionnumber#1%
  {\doifelsevalue{#1\c!headconversion}\v!yes
     {\doifelsevalue{#1\c!headlabel}\v!yes
        {\let\fullsectionnumber\naturalfullsectionnumber}
        {\let\fullsectionnumber\limitedfullsectionnumber}}
     {\let\fullsectionnumber\ignoredfullsectionnumber}}

\let\fullsectionnumber\limitedfullsectionnumber

% \dodododoconstructhead IS NON GROUPED, SO WE NEED TO RESTORE !!!!
%
% dit kan dus beter \everyaroundhead zijn

\let\currentheadnumber\empty
\let\currentheadtext  \empty

\def\dodoconstructhead#1[#2]#3% [ref] {title}
  {\doifelsevalue{\??ko#1\c!ownnumber}\v!yes
     {\doquadruplegroupempty\dododoconstructhead{#1}{#2}{#3}}
     {\fourthargumentfalse  \dododoconstructhead{#1}{#2}{#3}{}}}

\def\dododoconstructhead#1#2#3#4% [ref] {own} {title}
  {\iffourthargument
     \def\next{\dodododoconstructhead{#1}[#2]{#3}{#4}}%
   \else
     \def\next{\dodododoconstructhead{#1}[#2]{\finalsectionnumber}{#3}}%
   \fi
   \next}

% pas met \ExpandFirstAfter op bij twee||taligheid

\ifx\dohandleheadnumber\undefined
  \let\dohandleheadnumber\firstofoneargument
\fi

\unexpanded\def\\{\space}

\def\emptyheadcorrection % experimental, should work
  {\ifemptyhead            % well with na=\blank
     \vskip-\lineheight
     \dosomebreak\nobreak
     \kern\zeropoint
     \prevdepth\strutdepth
   \fi}

\let\localkopprefix\empty

% \def\headparameter#1% to do: everywhere in core-sec
%   {\csname\??ko\currenthead#1\endcsname}

\def\dodododoconstructhead#1[#2]#3#4% [ref] {number} {title}
  {\def\currenthead{#1}% dus #1 overal vervangen
   \let\finalsectionnumber\dofinalsectionnumber % overloaded ungrouped -)
   \unexpanded\def\\{\space}%
   \def\numberseparator{\getvalue{\??ko\currenthead\c!separator}}%
   \flushingcolumnfloatsfalse % {number} can be \finalsectionnumber
   \someheadconversionfalse
   \let\fullsectionnumber\limitedfullsectionnumber
   \setsectieenkoppeling{#1}%
   \doifelsevaluenothing{\??ko#1\c!prefix}
     \headprefixfalse\headprefixtrue
   \ifheadprefix
     \doifelsevalue{\??ko#1\c!prefix}{+}
       {\doifelsenothing{#2}
          {\def\localkopprefix{+}}
          {\def\localkopprefix{#2}}} % eigenlijk alleen eerste
       {\edef\localkoprefix{\getvalue{\??ko#1\c!prefix}}}%
   \else
     \let\localkoprefix\empty
   \fi
   \placeheadtrue
   \processaction
     [\getvalue{\??ko#1\c!placehead}]
     [  \v!yes=>\emptyheadfalse,
      \v!empty=>\emptyheadtrue,
         \v!no=>\emptyheadtrue\placeheadfalse]%
   \doifelsevalue{\??ko#1\c!resetnumber}\v!no
     {\setfalse\@@resetsubheadnumbers}%
     {\settrue \@@resetsubheadnumbers}%
   \writetolistfalse
   \processaction
     [\getvalue{\??ko#1\c!incrementnumber}]
     [   \v!yes=>\incrementnumbertrue,
          \v!no=>\incrementnumberfalse,
        \v!list=>\incrementnumberfalse
                  % beware, since no numbers are used, no nested lists are
                  % possible here
                  \writetolisttrue,
      \s!unknown=>{\ifx\currentproduct\empty
                     \findsectionnumber{#1}\commalistelement{#4}%
                   \fi
                   \incrementnumbertrue}]%
   \edef\numberheaddistance   {\getvalue{\??ko#1\c!distance}}%
   \edef\numberheadalternative{\getvalue{\??ko#1\c!alternative}}%
   \doifelsevalue{\??ko:\numberheadalternative}\v!horizontal
     \displaysectionheadfalse
     \displaysectionheadtrue
   \ifsectienummer
     \doifelsevalue{\??sb\@@sectionblock\c!number}\v!yes
       {\doifelsevalue{\??ko#1\c!number}\v!yes
          \headnumbertrue
          \headnumberfalse}
       {\headnumberfalse}%
   \else
     \headnumberfalse
   \fi
   \convertexpanded{\getvalue{\??ko#1\c!expansion}}{#4}\asciititle
   %
   \gdef\currentheadtext{#4}% scheelt args
   \globallet\currentheadnumber\empty
   %
   \ifincrementnumber
     \ifplacehead
       \checknexthead\handlepagebreak{#1}%
       \setsectieenkoppeling{#1}% can be changed when [voor=\somehead{..}...]
       \ifheadprefix
        %\setupreferencing[\c!prefix=-]%
         \setupreferenceprefix[-]%
       \fi
       \getvalue{\e!next\@@sectie}%
       \ifheadnumber
         \setsomeheadconversion{#1}{#3}%
         \let\fullsectionnumber\expandablefullsectionnumber
         \xdef\currentheadnumber{\someheadconversion}%
         \getvalue{\??ko#1\c!inbetween}%
         \ifsomeheadconversion
           \let\fullsectionnumber\naturalfullsectionnumber
           \doplaceheadnumbertext
             {#1}
             {\setsectionlistreference{\@@sectie}{#1}%
              \pagetype[\@@koppeling]%
              \let\fullsectionnumber\writtenfullsectionnumber
              \rawreference\s!sec{#2}{{\someheadconversion}{\asciititle}}%
              \resetsectionmarks\@@sectie
              \setlistparameter\@@koppeling\c!expansion{\getvalue{\??ko#1\c!expansion}}%
              \let\fullsectionnumber\writtenfullsectionnumber
              \dowritetolist\@@koppeling\someheadconversion{#4}\v!head}%
             {\dohandleheadnumber\someheadconversion}% handle is new
             {#4}
             {\marking[#1]{#4}%
              \let\fullsectionnumber\storedfullsectionnumber
              \expanded{\marking[#1\v!number]{\someheadconversion}}}%
           \let\fullsectionnumber\ignoredfullsectionnumber
           \writesection{#1}{\someheadconversion}{#4}%
         \else
           \doplaceheadnumbertext
             {#1}
             {\setsectionlistreference{\@@sectie}{#1}%
              \pagetype[\@@koppeling]%
              \rawreference\s!sec{#2}{{#3}{\asciititle}}%
              \resetsectionmarks\@@sectie
              \setlistparameter\@@koppeling\c!expansion{\getvalue{\??ko#1\c!expansion}}%
              \dowritetolist\@@koppeling{#3}{#4}\v!head}
             {\sectionblocklabel{#1}{\dohandleheadnumber{#3}}}% handle is new
             {#4}
             {\marking[#1]{#4}%
              \doifelsevalue{\??ko#1\c!ownnumber}\v!yes % rommelig omdat
                {\edef\finalsectionnumber{#3}} % #3 al is toegekend
                {\determineheadnumber[#1]}% migreert naar 3e argument
              \expanded{\marking[#1\v!number]{\finalsectionnumber}}}%
           \writesection{#1}{#3}{#4}%
         \fi
       \else
         \getvalue{\??ko#1\c!inbetween}%
         \doplaceheadtext
           {#1}
           {\setsectionlistreference{\@@sectie}{#1}%
            \pagetype[\@@koppeling]%
            \rawreference\s!sec{#2}{{#3}{\asciititle}}%
            \resetsectionmarks\@@sectie
            \setlistparameter\@@koppeling\c!expansion{\getvalue{\??ko#1\c!expansion}}%
\doifelsevalue{\??ko#1\c!ownnumber}\v!yes % brrr, new per 18/1/2005, sometimes we need
  {\dowritetolist\@@koppeling{#3}{#4}\v!head} % entries in the list (special purpose) but
  {\dowritetolist\@@koppeling  {}{#4}\v!head}% not in the header, ok we could pop in a command
}%            \dowritetolist\@@koppeling{}{#4}\v!head}
           {#4}
           {\marking[#1]{#4}%
            \doifelsevalue{\??ko#1\c!ownnumber}\v!yes % brrr
              {\edef\finalsectionnumber{#3}}
              {\determineheadnumber[#1]}%
% todo : geen markering (leeg maken)
            \expanded{\marking[#1\v!number]{\finalsectionnumber}}}%
         \writesection{#1}{-}{#4}%
       \fi
       \ifheadprefix
         \setupreferenceprefix[\localkopprefix]%
       \fi
       \ifdisplaysectionhead
         \dosomebreak\nobreak
         \emptyheadcorrection
         \getvalue{\??ko#1\c!after}%
       \fi
     \else
       % Whatever future tex's will do with nodes,
       % we assume a node here, because other \c!after=\blank
       % will fail! See 'prikkels'
       %
       % so, maybe we need an explicit \kern
       %
       % do nothing / should be vbox to 0pt
       %
       \checknexthead\dohandlepagebreakB{#1}% toegevoegd ivm subpaginanr / tug sheets
       \setsectieenkoppeling{#1}% can be changed when [voor=\somehead{..}...]
       \ifheadprefix
         \setupreferenceprefix[-]%
       \fi
       \getvalue{\e!next\@@sectie}%
       \ifheadnumber
         \setsomeheadconversion{#1}{#3}%
         \let\fullsectionnumber\expandablefullsectionnumber
         \xdef\currentheadnumber{\someheadconversion}%
       \fi
       \getvalue{\??ko#1\c!inbetween}% documenteren, is enige hook
       \bgroup
       \setsectionlistreference{\@@sectie}{#1}%
       \resetsectionmarks\@@sectie
       \marking[#1]{#4}%
       \doifelsevalue{\??ko#1\c!ownnumber}\v!yes
         {\edef\finalsectionnumber{#3}}
         {\determineheadnumber[#1]}%
       \expanded{\marking[#1\v!number]{\finalsectionnumber}}%
       \pagetype[\@@koppeling]%
%       \bgroup
       \setlistparameter\@@koppeling\c!expansion{\getvalue{\??ko#1\c!expansion}}%
       \ifheadnumber
         \rawreference\s!sec{#2}{{#3}{\asciititle}}%
         \dowritetolist\@@koppeling{#3}{#4}\v!head
         \writesection{#1}{#3}{#4}%
       \else % hm, also no own number
         \rawreference\s!sec{#2}{{#3}{\asciititle}}%
         \dowritetolist\@@koppeling{}{#4}\v!head
         \writesection{#1}{-}{#4}%
       \fi
       \egroup
       \ifheadprefix
         \setupreferenceprefix[\localkopprefix]%
       \fi
     \fi
   \else
     % todo : ref prefix
     \ifplacehead
       \checknexthead\handlepagebreak{#1}%
       \setsectieenkoppeling{#1}% can be changed when [voor=\somehead{..}...]
       \getvalue{\??ko#1\c!inbetween}%
       \doplaceheadtext
         {#1}
         {\forcesectiontolist{#1}{#4}%
          \rawreference\s!sec{#2}{{#3}{\asciititle}}} % #3 ?
         {#4}
        %{}% new:
         {\marking[#1]{#4}%
          \marking[#1\v!number]{}}%
       \writesection{#1}{-}{#4}%
       \ifdisplaysectionhead
         \dosomebreak\nobreak
         \emptyheadcorrection
         \getvalue{\??ko#1\c!after}%
       \fi
     \else
       % do nothing / should be vbox to 0pt
       \checknexthead\handlepagebreak{#1}%
       \setsectieenkoppeling{#1}% can be changed when [voor=\somehead{..}...]
       \getvalue{\??ko#1\c!inbetween}%
       \forcesectiontolist{#1}{#4}%
       \rawreference\s!sec{#2}{{#3}{\asciititle}}% #3 ?
       \marking[#1]{#4}%
       \marking[#1\v!number]{}%
       \writesection{#1}{-}{#4}%
     \fi
   \fi
   \flushingcolumnfloatstrue
   \someheadconversionfalse
   \let\fullsectionnumber\limitedfullsectionnumber
   \ifdisplaysectionhead\else\expandafter\GotoPar\fi}

\def\forcesectiontolist#1#2%
  {\ifwritetolist
     % we need to make sure that there is a number set (non
     % zero) else the list mechanism cannot determine the
     % level
     \bgroup
     \setupheadnumber[#1][+1]% traag, wordt \getvalue{\c!next...}
     \setlistparameter\@@koppeling\c!expansion{\getvalue{\??ko#1\c!expansion}}%
     \dowritetolist\@@koppeling{}{#2}\v!head
     \setupheadnumber[#1][-1]% traag, wordt \getvalue{\c!previous...}
     \egroup
   \fi}

\let\previoussectionformat\empty
\let\currentsectionformat \empty

\let\updatelistreferences \relax
\let\updatedlistreferences\empty

\def\setsectionlistreference#1#2%
  {\ifnum\countervalue{\??se\previoussection{#1}}>0\relax
     \xdef\previoussectionformat{\@@longformatnumber{\previoussection{#1}}}%
   \else
     \globallet\previoussectionformat\empty
   \fi
   \xdef\currentsectionformat{\@@longformatnumber{#1}}}

\def\startlistreferences#1%
  {\thisissomeinternal{\s!lst}{#1\currentsectionformat}%
   \setxvalue{\s!lst:#1}{\realfolio}% to be sure
   \setxvalue{\s!lst:#1\currentsectionformat}{\realfolio}%
   \setxvalue{\e!previouslocal#1}{\s!lst:#1\previoussectionformat}%
   \setxvalue{\e!currentlocal#1}{\s!lst:#1\currentsectionformat}%
   \doifelse{\currentsectionformat}{}
     {\setglobalcrossreference
        {\e!previous#1}{}{\realfolio}{}}
%
     {\setglobalsystemreference\rt!list
        {\e!previous#1}{\getvalue{\e!previouslocal#1}}}%
%
%         {\definereference[\e!previous#1][\getvalue{\e!previouslocal#1}]%
%
   \def\stoplistreferences{\dostoplistreferences}}

\def\dostoplistreferences#1%
  {\ifutilitydone
     \addtocommalist{#1}\updatedlistreferences              % nog global (\doglobal)
     \globallet\updatedlistreferences\updatedlistreferences % een noodverbandje
     \gdef\updatelistreferences%
       {\def\docommando####1%
%
          {\setglobalsystemreference\rt!list
             {\e!previous####1}{\getvalue{\e!currentlocal####1}}}%
%
%         {\definereference[\e!previous####1][\getvalue{\e!currentlocal####1}]%
%
        \processcommacommand[\updatedlistreferences]\docommando
        \globallet\updatelistreferences\relax
        \globallet\updatedlistreferences\empty}%
   \fi}

\def\stoplistreferences%
  {\gobbleoneargument}

% \prevdepth\strutdp % is belangrijk, vergelijk naast elkaar:
%
% \subject{test} \input tufte
% \subject{test} \strut \input tufte
% \subject{test} \placelist[...]

% todo: kap

% to be documented: \placeheadtext \placeheadnumber

\unexpanded\def\placeheadtext
  {\doquintupleempty\doplaceheadtextornumber
     [\c!textstyle][\c!textcolor][\empty]}

\unexpanded\def\placeheadnumber
  {\doquintupleempty\doplaceheadtextornumber
     [\c!numberstyle][\c!numbercolor][\v!number]}

\def\doplaceheadtextornumber[#1][#2][#3][#4][#5]%
  {\bgroup
   \edef\@@sectie{\??ko\iffifthargument#5\else#4\fi}%
   \dostartattributes\@@sectie\c!style\c!color\empty
     \dontconvertfont
     \dostartattributes\@@sectie{#1}{#2}\empty
       \setupinterlinespace
       \begstrut\getmarking[\mainmarking{#4#3}]\endstrut
       \endgraf
     \dostopattributes
   \dostopattributes
   \egroup}

\chardef\headtimingmode=0

% \chardef\headtimingmode=1 % 0 also works ok now too
%
% Martin Kolarik's problem:
%
% \setuphead[section][command=\doTitle]
%
% \def\doTitle#1#2%
%   {\ruledvbox{\forgetall \hsize=4cm
%      \ruledhbox{\ruledvtop{#1}\ruledvtop{#2}}}}
%
% \section{test test test test test test test test test test
% test test test test test test test}

\newevery \everyheadstart \relax

\def\placeheadmargintexts#1%
  {\the\everyheadstart
   \doifvalue{\??ko#1\c!margintext}\v!yes\placemargincontent}

\def\doplaceheadtext#1#2#3#4%
  {\beginheadplacement{#1}%
   \ifemptyhead % = needed
     \setbox0=\ifvertical\vbox\else\hbox\fi to \zeropoint
       {\headnumbercontentfalse
        \resetsystemmode\v!sectionnumber
        #2}%
     \makestrutofbox0
   \else % = needed
     \setbox0=\ifvertical\vbox\else\hbox\fi % \vhbox
       {\headnumbercontentfalse
        \resetsystemmode\v!sectionnumber
        % less interfering
        \ifcase\headtimingmode\or#2\fi
        % outerside font determines distance
        \dosetfontattribute{\??ko#1}\c!style
        % but we don't want color to influence user commands
        % todo: get the if-else out of it
        \getvalue{\??ko#1\c!command}
          {} % no number
          {\dostartattributes{\??ko#1}\c!style\c!color\empty
             \dostartattributes{\??ko#1}\c!textstyle\c!textcolor\empty
               \dontconvertfont
               \ifdisplaysectionhead
                 \setupinterlinespace
               \else
                 \setupspacing
               \fi
               % \ifcase\headtimingmode#2\fi % can introduce cr
               \getvalue{\??ko#1\c!commandbefore}%
               \placeheadmargintexts{#1}% binnen #3?
               \ifdisplaysectionhead
                 \getvalue{\??ko#1\c!textcommand}% struts can be nilled with \setnostrut
                   {\setstrut\begstrut\ifcase\headtimingmode\hbox{#2}\fi#3\endstrut}% \hbox prevents break
                 \xdef\localheadheight    {\the\strutht}%
                 \xdef\localheaddepth     {\the\strutdp}%
                 \xdef\localheadlineheight{\the\lineheight}%
                 % == \globallet\localheaddepth\strutdepth
               \else
                 \ifcase\headtimingmode#2\fi
                 \getvalue{\??ko#1\c!textcommand}{#3}%
               \fi
               \getvalue{\??ko#1\c!commandafter}%
               \ifdisplaysectionhead\endgraf\fi
             \dostopattributes
           \dostopattributes}}%
   \fi
   \endheadplacement{#1}{#4}}

\def\doplaceheadnumbertext#1#2#3#4#5% maybe move modes outside box
  {\beginheadplacement{#1}%
   \ifemptyhead % = needed
     \setbox0=\ifvertical\vbox\else\hbox\fi to \zeropoint
       {\doiftextelse{#3}
          {\setsystemmode  \v!sectionnumber\headnumbercontenttrue }
          {\resetsystemmode\v!sectionnumber\headnumbercontentfalse}%
        #2}%
     \makestrutofbox0
   \else % = needed
     \setbox0=\ifvertical\vbox\else\hbox\fi % \vhbox
       {\doiftextelse{#3}
          {\setsystemmode  \v!sectionnumber\headnumbercontenttrue }
          {\resetsystemmode\v!sectionnumber\headnumbercontentfalse}%
        % less interfering
        \ifcase\headtimingmode\or#2\fi
        % outerside font determines distance
        \dosetfontattribute{\??ko#1}\c!style
        % but we don't want color to influence user commands
        \getvalue{\??ko#1\c!command}%
          {\dostartattributes{\??ko#1}\c!style\c!color\empty
             \dostartattributes{\??ko#1}\c!numberstyle\c!numbercolor\empty
               % \getvalue{\??ko#1\c!commandbefore}% strange, why here? moved 21/11/2005
               \placeheadmargintexts{#1}% binnen #3?
               \ifdisplaysectionhead
                 % can be nilled with \setnostrut
                 \getvalue{\??ko#1\c!numbercommand}{\setstrut\begstrut#3\endstrut}%
               \else
                 \getvalue{\??ko#1\c!numbercommand}{#3}%
               \fi
             \dostopattributes
           \dostopattributes}
          {\dostartattributes{\??ko#1}\c!style\c!color\empty
             \dostartattributes{\??ko#1}\c!textstyle\c!textcolor\empty
               \dontconvertfont
               \ifdisplaysectionhead
                 \setupinterlinespace
               \else
                 \setupspacing
               \fi
               % \ifcase\headtimingmode#2\fi % can introduce cr
               \getvalue{\??ko#1\c!commandbefore}% makes more sense here
               \placeheadmargintexts{#1}% binnen #3?
               \ifdisplaysectionhead
                 \getvalue{\??ko#1\c!textcommand}% struts can be nilled with \setnostrut
                   {\setstrut\begstrut\ifcase\headtimingmode\hbox{#2}\fi#4\endstrut}% \hbox prevents break
                 \xdef\localheadheight    {\the\strutht}%
                 \xdef\localheaddepth     {\the\strutdp}%
                 \xdef\localheadlineheight{\the\lineheight}%
                 % == \globallet\localheaddepth\strutdepth
               \else
                 \ifcase\headtimingmode#2\fi % inside textcommand ?
                 \getvalue{\??ko#1\c!textcommand}{#4}%
               \fi
               \getvalue{\??ko#1\c!commandafter}%
               \ifdisplaysectionhead\endgraf\fi
            \dostopattributes
          \dostopattributes}}%
   \fi
   \endheadplacement{#1}{#5}}

\newsignal\headsignal
\let\headlastlinewidth\!!zeropoint

\def\beginheadplacement#1%
  {\bgroup
   \setsystemmode{#1}% to be documented
   \ifgridsnapping\iftracegridsnapping\showstruts\fi\fi
   \xdef\localheadheight    {\the\strutht}%
   \xdef\localheaddepth     {\the\strutdp}%
   \xdef\localheadlineheight{\the\lineheight}%
   % == \globallet\localheaddepth\strutdp
   \everypar\emptytoks % needed indeed
   \noindent           % ipv \whitespace elders, na \forgetall !
   \bgroup
   \doifelsevalue{\??ko#1\c!aligntitle}\v!yes % new
     {\skip0 1\leftskip
      \skip2 1\rightskip
      \xdef\localheadskip{\the\skip0}%
      \forgetall
      \leftskip\skip0
      \rightskip\skip2
      \setlocalhsize\hsize\localhsize
      \forgetbothskips}
     {\globallet\localheadskip\!!zeropoint
      \forgetall}%
   \dontcomplain
   \postponefootnotes
   \iflocation\ifdisplaysectionhead\else\noninterferingmarks\fi\fi
   \resetinteractionparameter\c!style
   \resetinteractionparameter\c!color
   \resetinteractionparameter\c!contrastcolor
   \strictouterreferencestrue % tzt instelling
   \def\localheadsetup{\dolocalheadsetup{#1}}%
   \startsynchronization}

% \def\endheadplacement#1#2%
%   {\doifelsevalue{\??rf#1\c!state}\v!start
%      {\doifvaluenothing{\??ko#1\c!file}{\autocrossdocumentfalse}}
%      {\autocrossdocumentfalse}%
%    % no message needed here, should be a proper switch
%    % \let\unknownreference\gobbleoneargument
%    \ifdisplaysectionhead
%      \let\headlastlinewidth\!!zeropoint
%      \snaptogrid[\getvalue{\??ko#1\c!grid}]\hbox
%        {\hskip\localheadskip
%         \hskip\getvalue{\??ko#1\c!margin}\relax
%         \iflocation
%           \ifautocrossdocument
%             \doifreferencefoundelse{\getvalue{\??ko#1\c!file}::#1}
%               {\edef\currentinnerreference{\s!aut:\currenttextreference}% stored in
%                \gotoouterlocation{}{\box0}}                             % text slot
%               {\hbox{\box0}}%
%           \else
%             \hbox{\box0}%
%           \fi
%         \else
%           \hbox{\box0}%
%         \fi}%
%      \doflushnotes % new, not really needed
%      \endgraf
%      \nointerlineskip
%      \dosomebreak\nobreak
%      #2%
%    \else
%      \strut
%      \doflushnotes % new, here since we're in par mode
%      \iflocation
%        \ifautocrossdocument
%          \hhboxindent=\ifcontinuoushead\headlastlinewidth\else\zeropoint\fi
%          \unhhbox0\with{\gotobox{\box\hhbox}[\getvalue{\??ko#1\c!file}::#1]}%
%          \advance\lasthhboxwidth by \numberheaddistance
%          \xdef\headlastlinewidth{\the\lasthhboxwidth}%
%        \else
%          \unhbox0
%          \globallet\headlastlinewidth\!!zeropoint
%        \fi
%      \else
%        \unhbox0
%        \globallet\headlastlinewidth\!!zeropoint
%      \fi
%      #2%
%      \dimen0=\numberheaddistance
%      \hskip\dimen0 \!!plus \dimen0 \!!minus .25\dimen0
%      \hskip\headsignal\ignorespaces
%    \fi
%    \ifdisplaysectionhead \ifvmode
%      \ifgridsnapping % important, font related depth, see comment
%        \prevdepth\strutdp
%      \else
%        \prevdepth\localheaddepth
%      \fi
%    \fi \fi
%    \stopsynchronization
%    \egroup
%    \egroup
%    \ifdisplaysectionhead
%      \dochecknextindentation{\??ko#1}%
%    \else
%      \nonoindentation % recently added, was a bug
%    \fi}

% \setuphead[chapter]         [style=\bfd,after=,hang=line] % fit broad 2
% \setuphead[section]         [style=\bfc,after=,hang=line]
% \setuphead[subsection]      [style=\bfb,after=,hang=line]
% \setuphead[subsubsection]   [style=\bfa,after=,hang=line]
% \setuphead[subsubsubsection][style=\bf ,after=,hang=line]
%
% \chapter         {Test} \input tufte \page
% \section         {Test} \input tufte \page
% \subsection      {Test} \input tufte \page
% \subsubsection   {Test} \input tufte \page
% \subsubsubsection{Test} \input tufte \page
%
% \chapter         {Test\\Test} \input tufte \page
% \section         {Test\\Test} \input tufte \page
% \subsection      {Test\\Test} \input tufte \page
% \subsubsection   {Test\\Test} \input tufte \page
% \subsubsubsection{Test\\Test} \input tufte \page

\def\hangheadplacement
  {\scratchdimen\localheadlineheight
   \bgroup
   \openlineheight\scratchdimen
   \scratchdimen\ht0
   \advance\scratchdimen\dp0
   \getnoflines\scratchdimen
   \advance\noflines\minusone
   \expanded{\egroup\noflines\the\noflines}% brrr
   \setbox0\hbox{\lower\noflines\scratchdimen\box0}%
   \scratchdimen\ht0
   \advance\scratchdimen\dp0
   \advance\scratchdimen-\localheadheight
   \advance\scratchdimen+\strutdp
   \ht0 \strutht
   \dp0 \strutdp
   \edef\localheaddepth{\the\strutdp}}

\def\endheadplacement#1#2%
  {\doifelsevalue{\??rf#1\c!state}\v!start
     {\doifvaluenothing{\??ko#1\c!file}{\autocrossdocumentfalse}}
     {\autocrossdocumentfalse}%
   % no message needed here, should be a proper switch
\noflines\zerocount
   \ifdisplaysectionhead
% new
\processaction
  [\getvalue{\??ko#1\c!hang}]
  [    \v!line=>\hangheadplacement\noflines\zerocount,
      \v!broad=>\hangheadplacement\getnoflines\scratchdimen,
        \v!fit=>\hangheadplacement\getrawnoflines\scratchdimen,
       \v!none=>\noflines\zerocount,
    \v!default=>\noflines\zerocount,
    \v!unknown=>\hangheadplacement\noflines0\commalistelement\advance\noflines\minusone]%
% so far
     \let\headlastlinewidth\!!zeropoint
     \snaptogrid[\getvalue{\??ko#1\c!grid}]\hbox
       {\hskip\localheadskip
        \hskip\getvalue{\??ko#1\c!margin}\relax
        \iflocation
          \ifautocrossdocument
            \doifreferencefoundelse{\getvalue{\??ko#1\c!file}::#1}
              {\edef\currentinnerreference{\s!aut:\currenttextreference}% stored in
               \gotoouterlocation{}{\box0}}                             % text slot
              {\hbox{\box0}}%
          \else
            \hbox{\box0}%
          \fi
        \else
          \hbox{\box0}%
        \fi}%
     \doflushnotes % new, not really needed
     \endgraf
\ifnum\noflines>\zerocount
  \dorecurse\noflines{\nointerlineskip\dosomebreak\nobreak\strut\endgraf}%
\fi
     \nointerlineskip
     \dosomebreak\nobreak
     #2%
   \else
     \strut
     \doflushnotes % new, here since we're in par mode
     \iflocation
       \ifautocrossdocument
         \hhboxindent=\ifcontinuoushead\headlastlinewidth\else\zeropoint\fi
         \unhhbox0\with{\gotobox{\box\hhbox}[\getvalue{\??ko#1\c!file}::#1]}%
         \advance\lasthhboxwidth by \numberheaddistance
         \xdef\headlastlinewidth{\the\lasthhboxwidth}%
       \else
         \unhbox0
         \globallet\headlastlinewidth\!!zeropoint
       \fi
     \else
       \unhbox0
       \globallet\headlastlinewidth\!!zeropoint
     \fi
     #2%
     \dimen0=\numberheaddistance
     \hskip\dimen0 \!!plus \dimen0 \!!minus .25\dimen0
     \hskip\headsignal\ignorespaces
   \fi
   \ifdisplaysectionhead \ifvmode
     \ifgridsnapping % important, font related depth, see comment
       \prevdepth\strutdp
     \else
       \prevdepth\localheaddepth
     \fi
   \fi \fi
   \stopsynchronization
   \egroup
   \egroup
   \ifdisplaysectionhead
     \dochecknextindentation{\??ko#1}%
   \else
     \nonoindentation % recently added, was a bug
   \fi}

\def\checknexthead#1#2% nog optioneel
  {\ifhmode
     \scratchcounter=\lastpenalty\unpenalty % no beauty in this
     \ifdim\lastskip=\headsignal
       \handlepagebreakC{#1}%
       \global\continuousheadtrue
     \else
       \penalty\scratchcounter
       \global\continuousheadfalse
       #1{#2}%
     \fi
   \else
     \global\continuousheadfalse
     #1{#2}%
   \fi}

\def\dosetupheadnumber[#1][#2#3]% todo: = (don't reset)
  {\bgroup
   \setsectieenkoppeling{#1}%
   \doifinstringelse{#2}{+-}
     {\doifelsenothing{#3}
        {\@@nextsectionnumber\@@sectie}
        {\!!counta=#2#3\relax
         \advance\!!counta \@@sectionvalue\@@sectie
         \@@setsectionnumber\@@sectie\!!counta}}
     {\@@setsectionnumber\@@sectie{#2#3}}%
   \egroup}

\def\setupheadnumber
  {\dodoubleargument\dosetupheadnumber}

\def\currentheadnumber{0}

\def\determineheadnumber[#1]%
  {\bgroup
   \setsectieenkoppeling{#1}%
   \xdef\currentheadnumber{\@@sectionvalue{\@@sectie}}%
   \egroup}

\def\complexheadnumber[#1]%
  {\bgroup
   \edef\currentheadnumber{#1}%
   \doifinsetelse{-}{#1} % br undocumented
     {\removefromcommalist{-}\currentheadnumber % br
      \setsectieenkoppeling\currentheadnumber
      \setupsection[\@@sectie][\c!previousnumber=\v!no]}%
     {\setsectieenkoppeling\currentheadnumber}%
   \xdef\currentheadnumber{\@@sectionvalue{\@@sectie}}%
   \doifnot{\currentheadnumber}{0}{\finalsectionnumber}%
   \egroup}

\def\simpleheadnumber
  {\currentheadnumber}

\definecomplexorsimple\headnumber

\def\alinea
  {\par}

% nice testcase
%
% \setupheads[aligntitle=yes]
%
% \startnarrower
%   \subject{\dorecurse{100}{x }}
%   \section{\dorecurse{100}{x }}
%   \input tufte \par
%   \setupheads[alternative=inmargin]
%   \subject{\dorecurse{100}{x }}
%   \section{\dorecurse{100}{x }}
%   \input tufte \par
% \stopnarrower

\let\numberheadalternative\v!normal

\def\defineheadplacement
  {\dodoubleargument\dodefineheadplacement}

\def\dodefineheadplacement[#1][#2]% #3#4
  {\setvalue{\??ko:#1}{#2}%
   \setvalue{\??ko::#1}}

\def\@@placehead
  {\executeifdefined
     {\??ko::\numberheadalternative}
     {\getvalue{\??ko::\v!normal}}}

\defineheadplacement[\v!paragraph][\v!vertical]#1#2%
  {\vbox
     {\localheadsetup
      \begstrut\ifheadnumbercontent#1\hskip\numberheaddistance\fi#2}}

\defineheadplacement[\v!normal][\v!vertical]#1#2%
  {\ifheadnumbercontent
     \setbox0\hbox{{#1}\hskip\numberheaddistance}%
     \vbox
       {\localheadsetup
        \hangindent 1\wd0
        \hangafter 1
        \noindent
        \unhbox0    %  don't use \strut's here!
        #2}%
   \else
     \vbox
       {\localheadsetup\noindent#2}%
   \fi}

\def\placeheadmargin#1#2%
  {\vbox
     {\localheadsetup
      \begstrut % use one \strut here!
      \ifheadnumbercontent
        \llap{\hbox to 5em{\hfill{#1}%
          \hskip\localheadskip\hskip\leftmargindistance}}%
      \fi
      {#2}}}

\defineheadplacement[\v!inmargin][\v!vertical]#1#2{\placeheadmargin{#1}{#2}}
\defineheadplacement[\v!margin]  [\v!vertical]#1#2{\placeheadmargin{#1}{#2}}

\defineheadplacement[\v!middle][\v!vertical]#1#2%
  {\vbox
     {\localheadsetup
      \veryraggedcenter
      \let\\\endgraf
      \let\crlf\endgraf
      \ifheadnumbercontent\strut#1\par\fi\begstrut#2}}

\defineheadplacement[\v!text][\v!horizontal]#1#2%
  {\bgroup
   \localheadsetup % no stretch in distance
   \ifheadnumbercontent{#1}\kern\numberheaddistance\fi{\begstrut#2}%
   \egroup}

\def\placeheadlohi#1#2#3%
  {\ifheadnumbercontent
     \setbox0\hbox{#2}
     \setbox2#1{\localheadsetup\advance\hsize-\wd0\relax#3}%
     \hbox{\box0\hskip\numberheaddistance\box2}%
   \else
     #1{\localheadsetup\noindent#3}%
   \fi}

% onder/boven lijnt het nummer op de onderste/bovenste regel
% uit van een meerregelige kop

\defineheadplacement[\v!bottom][\v!vertical]#1#2{\placeheadlohi\vbox{#1}{#2}}
\defineheadplacement[\v!top]   [\v!vertical]#1#2{\placeheadlohi\vtop{#1}{#2}}

% default   == instellingen
% koppeling == koppen, breaks, marks, enz.
% sectie    == nummering

\let\@@kolist=\empty

\def\dodefinehead[#1][#2]%   % don't preset prefix to much
  {\presetlabeltext[#1=]%
   \getparameters
     [\??ko#1]
     [\c!numberstyle=\getvalue{\??ko#1\c!style},
      \c!textstyle=\getvalue{\??ko#1\c!style},
      \c!numbercolor=\getvalue{\??ko#1\c!color},
      \c!textcolor=\getvalue{\??ko#1\c!color}]%
   \doifassignmentelse{#2}
     {\getparameters
        [\??ko#1]
        [\c!section=\getvalue{\??ko\getvalue{\??ko#1\c!coupling}\c!section},
         \c!default=,
         \c!coupling=,
         \c!prefix=,
         \c!before=,
         \c!after=,
         \c!distance=\!!zeropoint,
         \c!page=,
         \c!header=,
         \c!text=,
         \c!footer=,
         \c!style=,
         \c!numbercommand=,
         \c!textcommand=,
         \c!ownnumber=\v!no,
         \c!number=\v!yes,
         \c!color=,
         \c!continue=\v!yes,
         \c!placehead=\v!yes,
         \c!resetnumber=\v!yes,
         \c!incrementnumber=\v!yes,
         \c!alternative=\@@koalternative,
         \c!command=\@@placehead,
         \c!separator=\@@koseparator,
         \c!align=\@@koalign,
         \c!aligntitle=\@@koaligntitle,
         \c!tolerance=\@@kotolerance,
         \c!indentnext=\@@koindentnext,
         \c!hang=\@@kohang,
         \c!file=,
         \c!expansion=,
         \c!grid=,
         \c!margintext=,
         \c!margin=\@@komargin,
         #2]%
      \ConvertToConstant\doifnot{#1}{\getvalue{\??ko#1\c!default}}
        {\doifsomething{\getvalue{\??ko#1\c!default}}
           {\copyparameters
              [\??ko#1][\??ko\getvalue{\??ko#1\c!default}]
              [\c!before,\c!after,\c!command,\c!file,\c!page,\c!continue,
               \c!header,\c!text,\c!footer,\c!separator,\c!resetnumber,
               \c!number,\c!ownnumber,\c!placehead,\c!incrementnumber,
               \c!style,\c!color,\c!distance,\c!alternative,\c!indentnext,
               % new per 20/03/3002 (o-pbu-l) / was too confusing
               % \c!numberstyle,\c!textstyle,\c!expansion,
               % again too confusing
               \c!align,\c!aligntitle,\c!tolerance,\c!grid,\c!hang,
               \c!numbercommand,\c!textcommand,\c!margintext,\c!margin]}}%
      \getparameters[\??ko#1][#2]%
      \doifsomething{\getvalue{\??ko#1\c!section}}
        {\doifelsemarking{#1}% \doifundefined{\??mk#1}
           {}% marking #1 already defined
           {\definemarking[#1]%
            \couplemarking[#1][\getvalue{\??ko#1\c!section}]%
            \definemarking[#1\v!number]%
            \couplemarking[#1\v!number][\getvalue{\??ko#1\c!section}]}}%
      \doifundefined{\??li#1}{\definelist[#1]}}
     {\ConvertToConstant\doifelse{#1}{#2}
        {\doifundefined{\??li#1}{\definelist[#1]}}
        {\copyparameters
           [\??ko#1][\??ko#2]
           [\c!level,\c!section,\c!coupling,\c!prefix,
            \c!before,\c!after,\c!command,\c!file,\c!page,\c!continue,
            \c!separator,
            \c!header,\c!text,\c!footer,\c!resetnumber,
            \c!number,\c!ownnumber,\c!placehead,\c!incrementnumber,
            \c!style,\c!color,\c!distance,\c!alternative,\c!indentnext,
            % new per 20/03/3002 (o-pbu-l) / was too confusing
            % \c!numberstyle,\c!textstyle,\c!expansion,
            % again too confusing
            \c!align,\c!aligntitle,\c!tolerance,\c!grid,\c!hang,
            \c!numbercommand,\c!textcommand,\c!margintext,\c!margin]%
\getparameters[\??ko#1][\c!expansion=]% iig een value, rather fuzzy
         \definemarking[#1][#2]%
         \definemarking[#1\v!number][#2\v!number]%
         \doifundefined{\??li#1}{\definelist[#1][#2]}}}%
   \addtocommalist{#1}\@@kolist
   \setevalue{\??sk#1}{\getvalue{\??ko#1\c!coupling}}%
   \setevalue{\??by#1}{\getvalue{\??ko#1\c!section}}%
   \setevalue{\??by\v!by#1}{\getvalue{\??ko#1\c!section}}%
   \setvalue{#1}{\dodoubleempty\doconstructhead[#1]}}

\def\definehead
  {\dodoubleemptywithset\dodefinehead}

\def\doconstructhead[#1][#2]%
  {\dowithpargument{\dodoconstructhead{#1}[#2]}}

\def\dosetuphead[#1][#2]%
  {\getparameters[\??ko#1][#2]%
   % The next check prevents hard to trace problems. I once
   % set \c!command to nothing and (quite natural) got the
   % wrong references etc. The whole bunch should be boxed!
   \expandafter\convertcommand\csname\??ko#1\c!command\endcsname\to\ascii
   \doifnothing\ascii{\setvalue{\??ko#1\c!command}{\@@placehead}}}

\def\setuphead
  {\dodoubleargumentwithset\dosetuphead}

\def\dosetupheads[#1]%
  {\getparameters[\??ko][#1]%
   \doifelse{\@@kosectionnumber}\v!yes\sectienummertrue\sectienummerfalse}

\def\setupheads
  {\dosingleargument\dosetupheads}

\def\systemsuppliedchapter {\getvalue{\v!chapter}}
\def\systemsuppliedtitle   {\getvalue{\v!title}}

% a left over

\def\complexbijlage[#1]#2%
  {\page[\v!right]
   \setuppagenumbering[\c!state=\v!stop]
   \systemsuppliedchapter[#1]{#2}
   \page[\v!right]
   \setuppagenumbering[\c!state=\v!start]
   \setuppagenumbering[\c!number=1]}

\setvalue{\v!appendix}%
  {\complexorsimpleempty\bijlage}

\setupheads
  [\c!alternative=\v!normal,
   \c!sectionnumber=\v!yes,
   \c!separator=.,
   \c!limittext=\v!yes,
   \c!align=,
   \c!aligntitle=,
   \c!tolerance=,
   \c!indentnext=\v!no,
   \c!margin=\zeropoint,
   \c!hang=\v!none,
   \c!command=]

\definesectionblock [\v!frontpart] [\v!frontmatter] [\c!number=\v!no]
\definesectionblock [\v!bodypart]  [\v!bodymatter]  [\c!number=\v!yes]
\definesectionblock [\v!appendix]  [\v!appendices]  [\c!number=\v!yes]
\definesectionblock [\v!backpart]  [\v!backmatter]  [\c!number=\v!no]

\definesection[\s!section-1]   % part
\definesection[\s!section-2]   % chapter
\definesection[\s!section-3]   % section
\definesection[\s!section-4]   % subsection
\definesection[\s!section-5]   % subsubsection
\definesection[\s!section-6]   % subsubsubsection
\definesection[\s!section-7]   % subsubsubsubsection

% \c!eigennummer ook hier?

\definehead
  [\v!part]
  [\c!section=\s!section-1]

\definehead
  [\v!chapter]
  [\c!section=\s!section-2]

\definehead
  [\v!section]
  [\c!section=\s!section-3]

\definehead
  [\v!subsection]
  [\c!section=\s!section-4,
   \c!default=\v!section]

\definehead
  [\v!subsubsection]
  [\c!section=\s!section-5,
   \c!default=\v!subsection]

\definehead
  [\v!subsubsubsection]
  [\c!section=\s!section-6,
   \c!default=\v!subsubsection]

\definehead
  [\v!subsubsubsubsection]
  [\c!section=\s!section-7,
   \c!default=\v!subsubsubsection]

\definehead
  [\v!title]
  [\c!coupling=\v!chapter,
   \c!default=\v!chapter,
   \c!incrementnumber=\v!no]

\definehead
  [\v!subject]
  [\c!coupling=\v!section,
   \c!default=\v!section,
   \c!incrementnumber=\v!no]

\definehead
  [\v!subsubject]
  [\c!coupling=\v!subsection,
   \c!default=\v!subsection,
   \c!incrementnumber=\v!no]

\definehead
  [\v!subsubsubject]
  [\c!coupling=\v!subsubsection,
   \c!default=\v!subsubsection,
   \c!incrementnumber=\v!no]

\definehead
  [\v!subsubsubsubject]
  [\c!coupling=\v!subsubsubsection,
   \c!default=\v!subsubsubsection,
   \c!incrementnumber=\v!no]

\definehead
  [\v!subsubsubsubsubject]
  [\c!coupling=\v!subsubsubsubsection,
   \c!default=\v!subsubsubsubsection,
   \c!incrementnumber=\v!no]

\setupsection
  [\s!section-2]
  [\v!appendix\c!conversion=\v!Character,
   \c!previousnumber=\v!no]

\setuphead
  [\v!part]
  [\c!placehead=\v!no]

\setuphead
  [\v!chapter]
  [\v!appendix\c!label=\v!appendix,
   \v!bodypart\c!label=\v!chapter]       %   bijlageconversie=\Character

\setuphead
  [\v!section]
  [\v!appendix\c!label=\v!section,
   \v!bodypart\c!label=\v!section]       %   bijlageconversie=\Character

\setuphead
  [\v!subsection]
  [\v!appendix\c!label=\v!subsection,
   \v!bodypart\c!label=\v!subsection]    %   bijlageconversie=\Character

\setuphead
  [\v!subsubsection]
  [\v!appendix\c!label=\v!subsubsection,
   \v!bodypart\c!label=\v!subsubsection] %   bijlageconversie=\Character

\setuphead
  [\v!part,\v!chapter]
  [%\c!align=,
   %\c!indentnext=\v!no,
   \c!continue=\v!no,
   \c!page=\v!right,
   \c!header=,
   \c!style=\tfc,
   \c!distance=.75em,
   \c!before={\blank[2*\v!big]},
   \c!after={\blank[2*\v!big]}]

\setuphead
  [\v!section]
  [%\c!align=,
   %\c!indentnext=\v!no,
   \c!style=\tfa,
   \c!distance=.75em,
   \c!before={\blank[2*\v!big]},
   \c!after=\blank]

\setuphead                 % nieuw
  [\v!subsection]
  [\c!page=]

\definecombinedlist
  [\v!content]
  [\v!part,
   \v!chapter,
   \v!section,
   \v!subsection,
   \v!subsubsection,
   \v!subsubsubsection,
   \v!subsubsubsubsection]
  [\c!level=\v!subsubsubsubsection,
   \c!criterium=\v!local]

\setuplist
  [\v!part]
  [\c!before={\blank\page[\v!preference]},
   \c!after=\blank,
   \c!label=\v!yes,
   \c!separator=:,
   \c!distance=1em]

\setuplist
  [\v!chapter]
  [\c!before={\blank\page[\v!preference]},
   \c!after=]

\setuplist [\v!part]                [\c!width=0em]
\setuplist [\v!chapter]             [\c!width=2em]
\setuplist [\v!section]             [\c!width=3em]
\setuplist [\v!subsection]          [\c!width=4em]
\setuplist [\v!subsubsection]       [\c!width=5em]
\setuplist [\v!subsubsubsection]    [\c!width=6em]
\setuplist [\v!subsubsubsubsection] [\c!width=7em]

% hm

\setuppagenumbering % na instellen hoofdteksten !
  [\c!alternative=\v!singlesided,
   \c!location={\v!header,\v!middle},
   \c!conversion=\v!numbers,
   \c!width=, % in geval van \v!marginedge
   \c!left=,
   \c!right=,
   \c!way=\v!by\v!part,
   \c!text=,
   \v!chapter\v!number=\v!no, % v
   \v!part\v!number=\v!yes,       % v
   \c!numberseparator=--,
   \c!textseparator=\tfskip,
   \c!state=\v!start,
   \c!command=,
   \c!strut=\v!yes, % nieuw
   \c!style=, % \v!normal, % empty, otherwise conflict
   \c!color=]

\protect \endinput
