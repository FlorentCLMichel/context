%D \module
%D   [       file=core-sec,
%D        version=1997.03.31,
%D          title=\CONTEXT\ Core Macros,
%D       subtitle=Sectioning,
%D         author=Hans Hagen,
%D           date=\currentdate,
%D      copyright={PRAGMA / Hans Hagen \& Ton Otten}]
%C
%C This module is part of the \CONTEXT\ macro||package and is
%C therefore copyrighted by \PRAGMA. See mreadme.pdf for
%C details.

\writestatus{loading}{Context Core Macros / Sectioning}

\startmessages  dutch  library: structures
  title: structuur
      1: begin van sectieblok --
      2: eind van sectieblok --
\stopmessages

\startmessages  english  library: structures
  title: structure
      1: begin of sectionblock --
      2: end of sectionblock --
\stopmessages

\startmessages  german  library: structures
  title: struktur
      1: Begin des Abschnittsblock --
      2: Ende des Abschnittsblock --
\stopmessages

\startmessages  czech  library: structures
  title: struktury
      1: zacatek oddilu (sekce) --
      2: konec oddilu (sekce) --
\stopmessages

\startmessages  italian  library: structures
  title: struttura
      1: inizio del blocco (sezione) --
      2: fine del blocco (sezione) --
\stopmessages

\startmessages  norwegian  library: structures
  title: struktur
      1: starten av blokk -- (seksjon)
      2: slutten av blokk -- (seksjon)
\stopmessages

\startmessages  romanian  library: structures
  title: structuri
      1: inceput de bloc sectiune --
      2: sfarsit de bloc sectiune --
\stopmessages

\unprotect

\def\dodochecknummer#1#2#3%
  {\bgroup
   \doifinstringelse{.0}{.#2}   % waarom \instring en \@koscheider
     {\doifnot{#3}{\v!per}
        {%\debuggerinfo{\m!systems}{number #1 #3 becomes \getnumbervariable{#1\c!wijze}}%
         \setevalue{\s!number#1\c!wijze}{#3}% geen \xdef, gaat mis met \subpage
         \dochecknummer{#1}}} % tricky and ugly
     {\doifnotvalue{\s!number#1\s!check}{#2}
        {\setcounter{\s!number#1}{0\getvalue{\s!number#1\c!start}}%
         \setxvalue{\s!number#1\c!wijze\c!lokaal}%
           {\getvalue{\s!number#1\c!wijze}}%
         \setxvalue{\s!number#1\s!check}%
           {#2}}}%
   \egroup}

\def\dochecknummer#1%
  {\edef\currentsection{\getvalue{\??by\getvalue{\s!number#1\c!wijze}}}%
   \doifsomething{\currentsection}
     {\dodochecknummer
        {#1}
        {\getvalue{\currentsection\c!nummer}}
        {\v!per\previoussection{\currentsection}}}}

\def\checknummer#1%
  {\bgroup
   \ifnum\blocklevel>0
     \doifelsevalue{\s!number#1\c!blokwijze}{\v!nee}
       {\dochecknummer{#1}}
       {\setblockcounters    % dit kan sneller omdat de waarden
        \dochecknummer{#1}}% % en het type bekend zijn
   \else
     \dochecknummer{#1}%
   \fi
   \egroup}

\def\domaakvoorafgaandenummer[#1]%
  {\bgroup % added
   \global\let\voorafgaandenummer\empty
   \ifsectienummer
     \doifvalue{\??sb\@@sectieblok\c!nummer}{\v!ja} % added
       {\doifelsevalue{\s!number#1\c!sectienummer}{\v!ja}
          {\donetrue}{\donefalse}%
        \doifvalue{\s!number#1\c!sectienummer}{\v!nummer}
          {\donetrue\let\@@sectionconversion\gobbleoneargument}%
        \ifdone
          \edef\currentsection%
            {\getvalue{\??by\getvalue{\s!number#1\c!wijze\c!lokaal}}}%
          \doifnot{\currentsection}{\zerosection}
            {\doifnot{\@@sectionvalue{\currentsection}}{0}
               {\xdef\voorafgaandenummer%
                  {\getvalue{\currentsection\c!nummer}.}}}%
        \fi}%
   \fi
   \egroup}

\def\maakvoorafgaandenummer[#1]%
  {\bgroup
   \ifnum\blocklevel>0
     \doifelsevalue{\s!number#1\c!blokwijze}{\v!nee}
       {\domaakvoorafgaandenummer[#1]}%
       {\setblockcounters               % dit kan sneller omdat de waarden
        \domaakvoorafgaandenummer[#1]}% % en het type bekend zijn
   \else
     \domaakvoorafgaandenummer[#1]%
   \fi
   \egroup}

% \def\maakhetnummer[#1]%
%   {\maakvoorafgaandenummer[#1]%
%    \xdef\hetnummer%
%      {\voorafgaandenummer\nummer[#1]}}%
% 
% hack needed for chinese and oldstyle in normal tex, will change 

\def\maakhetnummer[#1]%
  {\bgroup
   \forceunexpanded % i don't like this hack
   \maakvoorafgaandenummer[#1]%
   \xdef\hetnummer% was \xdef maar dat gaat fout met font switches
     {\voorafgaandenummer\nummer[#1]}%
   \egroup}

\def\lossenummer[#1]%
  {\maakhetnummer[#1]%
   \hetnummer}

\def\huidigenummer[#1]%
  {%\getvalue{\getvalue{\s!number#1\c!zetwijze}}%
   \getvalue{\getvalue{\s!number#1\c!plaats}}%
     {\dotextprefix{\getvalue{\s!number#1\c!tekst}}\lossenummer[#1]}}

\def\volgendenummer[#1][#2][#3]%
  {\verhoognummer[#1]%
   \huidigenummer[#1]%
   \rawreference{#2}{#3}{\hetnummer}}

% sectioning

\newcount\nofsections

\def\zerosection{\v!tekst}
\def\firstsection{}
\def\lastsection{}
\let\@@sectie\empty
\let\@@koppeling\empty

\makecounter{\??se\v!tekst}

\setevalue{\??se\v!tekst\c!voor}{}
\setevalue{\??se\v!tekst\c!na  }{}

\setevalue{\v!tekst\c!nummer}{0}
\setevalue{\v!tekst\s!format}{}

\setevalue{\??sk\v!tekst}{}
\setevalue{\??sk        }{}

\setvalue{\??by               }{\v!tekst}
\setvalue{\??by\v!tekst       }{\v!tekst}
\setvalue{\??by\v!alles       }{\v!tekst}
\setvalue{\??by\v!per         }{\v!tekst}
\setvalue{\??by\v!per\v!tekst }{\v!tekst}
\setvalue{\??by\v!per\v!alles }{\v!tekst}
\setvalue{\??by\v!per\v!pagina}{\v!tekst} % see footnotes

%%%%%%%%% old

\def\dostelsectiein[#1][#2]%
  {\getparameters[\??se#1][#2]%
   \doifelsevalue{\??se#1\c!vorigenummer}{\v!ja}
     {\setvalue{#1\c!nummer}{\@@longsectionnumber{#1}}}
     {\setvalue{#1\c!nummer}{\@@shortsectionnumber{#1}}}}

\def\stelsectiein%
  {\dodoubleargument\dostelsectiein}

%%%%%%%%% new, multilingual

\def\dostelsectiein[#1][#2][#3]%
  {\ifthirdargument
     \getparameters[\??se#1#2][#3]%
   \else
     \getparameters[\??se#1][#2]%
   \fi
   \doifelsevalue{\??se#1\c!vorigenummer}{\v!ja}
     {\setvalue{#1\c!nummer}{\@@longsectionnumber{#1}}}
     {\setvalue{#1\c!nummer}{\@@shortsectionnumber{#1}}}}

\def\stelsectiein%
  {\dotripleempty\dostelsectiein}

%%%%%%%%%

\def\dokoppelmarkering[#1][#2]%
  {\doifdefinedelse{\??ko#2\c!sectie}
     {\dokoppelmarkering[#1][\getvalue{\??ko#2\c!sectie}]}
     {\def\donexttrackcommando##1%
        {\edef\gekoppeldemarkeringen{\getvalue{\??se##1\c!markering}}%
         \doifelse{##1}{#2}
           {\addtocommalist{#1}\gekoppeldemarkeringen}
           {\removefromcommalist{#1}\gekoppeldemarkeringen}%
         \setevalue{\??se##1\c!markering}{\gekoppeldemarkeringen}%
         \donexttracklevel{##1}}%
      \donexttracklevel{\zerosection}}} % \firstsection

\def\koppelmarkering%
  {\dodoubleargument\dokoppelmarkering}

\def\ontkoppelmarkering[#1]%
  {\koppelmarkering[#1][]}

\def\definieersectie[#1]%
  {\doifundefined{\??se#1}
     {\doifelsenothing{\firstsection}
        {\def\firstsection{#1}%
         \setevalue{\??se#1\c!voor}{\v!tekst}%
         \setevalue{\??se\v!tekst\c!na}{#1}}
        {\setevalue{\??se\commalistelement\c!na}{#1}%
         \setevalue{\??se#1\c!voor}{\lastsection}%
         \setevalue{\??se\lastsection\c!na}{#1}}%
      \advance\nofsections by 1
      \setevalue{\??se#1\c!niveau}%
        {\the\nofsections}%
      \setevalue{\??se#1\c!na}%
        {}%
      \setvalue{\e!volgende#1}%
        {\@@nextsectionnumber{#1}}%
      \setvalue{#1\c!nummer}%
        {\@@longsectionnumber{#1}}%
      \setvalue{#1\s!format}%
        {\@@longformatnumber{#1}}%
      \setevalue{\??by#1}{#1}%
      \setevalue{\??by\v!per#1}{#1}%
      \makecounter{\??se#1}%
      \def\lastsection{#1}%
      \setvalue{\??sk#1}%
        {#1}%
      \setvalue{\??se#1\c!markering}%
        {}%
      \stelsectiein[#1]
        [\c!vorigenummer=\v!ja]}}%

\def\previoussection#1%
  {\getvalue{\??se#1\c!voor}}

\def\nextsection#1%
  {\getvalue{\??se#1\c!na}}

\def\@@setsectionnumber#1#2%
  {\setgvalue{\??se#1\s!start}{}%   % signal i.p.v. boolean
   \setcounter{\??se#1}{#2}%
   \resetsectioncounters[#1]%
   \checkpagecounter}

\def\@@nextsectionnumber#1%
  {\setgvalue{\??se#1\s!start}{}%   % signal i.p.v. boolean
   \pluscounter{\??se#1}%
   \resetsectioncounters[#1]%
   \checkpagecounter}

\def\@@sectionvalue#1%       % nog niet overal doorgevoerd
  {\countervalue{\??se#1}}   % zoeken op \??se

% \def\@@sectionconversion#1%
%   {\getvalue{\??cv\getvalue{\??se#1\@@sectieblok\c!conversie}}}

% suited for chinese too:

\def\@@sectionconversion#1#2% a doublure with \@@shortsectionnumber
  {\ifnum#2=0 0\else % else troubles with \uchar
     \@EA\ifx\csname\??se#1\@@sectieblok\c!conversie\endcsname\relax
       \@EA\ifx\csname\??se#1\c!conversie\endcsname\relax
         #2%
       \else
         \getvalue{\??cv\getvalue{\??se#1\c!conversie}}{#2}%
       \fi
     \else
       \getvalue{\??cv\getvalue{\??se#1\@@sectieblok\c!conversie}}{#2}%
     \fi
   \fi}

\def\@@sectionlevel#1%
  {\ifundefined{\??se#1\c!niveau}0\else\getvalue{\??se#1\c!niveau}\fi}

% Omdat een markering kan worden herdefinieerd moeten we
% eerst testen of er wel een keten||afhankelijkheid is.

\def\resetsectionmarks[#1]%
  {\doifdefinedelse{\??se#1}
     {\def\donexttrackcommando##1%
        {\doifdefined{\??se##1\c!markering} % skip zero level
           {\def\docommando####1%
              {\ExpandFirstAfter\resetmarkering[####1]}%
            \processcommacommand[\getvalue{\??se##1\c!markering}]\docommando}%
         \donexttracklevel{##1}}%
      \donexttracklevel{#1}}%
     {\ExpandFirstAfter\resetmarkering[\hoofdmarkering{#1}]}}

\def\resetsectioncounters[#1]%
  {\def\donexttrackcommando##1%
     {\resetcounter{\??se##1}%
      \donexttracklevel{##1}}%
   \donexttracklevel{#1}}

% bij checken kan geen prefix worden bekeken, anders vallen
% er titels buiten de inhoudsopgave

% evt ook level gaan opslaan tbv snelle selectie

\def\makesectionformat%
  {\@EA\edef\@EA\sectionformat\@EA%
     {\@@sectiontype:\getvalue{\lastsection\s!format}}}

\def\dobacktracklevel#1%
  {\doifnot{\previoussection{#1}}{\zerosection}
     {\dobacktrackcommando{\previoussection{#1}}}}

\def\donexttracklevel#1%
  {\doifnot{#1}{\lastsection}
     {\donexttrackcommando{\nextsection{#1}}}}

\newif\ifalllevels

\def\dosetlevel#1% opvoeren met \ifcsname
  {\bgroup
   \doifelse{#1}{\v!vorige}
     {\global\alllevelstrue
      \global\let\currentlevel\empty
      \def\dobacktrackcommando##1%
        {\ifnum\countervalue{\??se##1}>0
           \global\alllevelsfalse
           \xdef\currentlevel{\getvalue{\previoussection{##1}\s!format}}%
         \else
           \dobacktracklevel{##1}%
         \fi}%
      \dobacktrackcommando\lastsection}
     {\doifelse{\getvalue{\??by#1}}{\v!tekst}
        {\global\alllevelstrue
         \global\let\currentlevel\empty}
        {\doifdefinedelse{\??ko#1\c!sectie}            % beter alteratief: ook
           {\edef\@@sectie{\getvalue{\??ko#1\c!sectie}}} % hoofdstuk\c!format
           {\edef\@@sectie{#1}}%
         \doifdefinedelse{\??se\@@sectie}
           {\global\alllevelsfalse
            \xdef\currentlevel{\getvalue{\@@sectie\s!format}}}
           {\global\alllevelstrue
            \global\let\currentlevel\empty
            \def\dobacktrackcommando##1%
              {\@EA\ifx\csname\??se##1\c!start\endcsname\relax
                 \dobacktracklevel{##1}%
               \else
                 \ifnum\countervalue{\??se##1}>0
                   \global\alllevelsfalse
                   \xdef\currentlevel{\getvalue{##1\s!format}}%
                 \else
                   \dobacktracklevel{##1}%
                 \fi
               \fi}%
            \dobacktrackcommando\lastsection}}}%
   \egroup}

\let\currentlevel\empty

\def\doifnextlevelelse[#1::#2]#3#4%
  {\ifalllevels
     #3%
   \else
     \doifelse{\@@sectiontype}{#1} % \@EA kunnen denk ik weg
       {\@EA\doifinstringelse\@EA{\@EA=\currentlevel:}{=:#2:}
          {\@EA\doifinstringelse\@EA{\@EA=\currentlevel:0}{=:#2:}{#4}{#3}}
          {#4}}
       {#4}%
   \fi}

\def\doifprevlevelelse[#1::#2]#3#4%
  {\ifalllevels
     #3%
   \else
     \doifelse{\@@sectiontype}{#1}
      {\@EA\doifinstringelse\@EA{\@EA=\currentlevel:}{=:#2:}
         {#3}
         {#4}}
       {#4}%
   \fi}

\def\dosettoclevel{\dosetlevel}
\def\dosetreglevel{\dosetlevel}
\def\dosetblklevel{\dosetlevel}

\def\doiftoclevelelse{\doifnextlevelelse}
\def\doifreglevelelse{\doifprevlevelelse}
\def\doifblklevelelse{\doifprevlevelelse}

\def\@@longformatnumber#1%
  {\getvalue{\previoussection{#1}\s!format}:\@@shortsectionnumber{#1}}

\def\@@longsectionnumber#1%
  {\ifnum\countervalue{\??se\previoussection{#1}}>0
     \getvalue{\previoussection{#1}\c!nummer}\@@koscheider
   \fi
   \@@shortsectionnumber{#1}}

\def\@@shortsectionnumber#1%
  {\@EA\ifx\csname\??se#1\@@sectieblok\c!conversie\endcsname\relax
     \@@sectionvalue{#1}%
   \else
     \@@sectionconversion{#1}{\@@sectionvalue{#1}}%
   \fi}

% suited for chinese too:

\def\@@shortsectionnumber#1%
  {\@EA\ifx\csname\??se#1\@@sectieblok\c!conversie\endcsname\relax
     \@EA\ifx\csname\??se#1\c!conversie\endcsname\relax
       \@@sectionvalue{#1}%
     \else
       \@@sectionconversion{#1}{\@@sectionvalue{#1}}%
     \fi
   \else
     \@@sectionconversion{#1}{\@@sectionvalue{#1}}%
   \fi}

\def\dosetlocalsectieblok#1#2#3%
  {\def\@@sectiontype{#1}%
   \def\@@sectieblok{#2}%
   \def\@@sectieblokken{#3}}

\def\doaroundsectieblok#1%
  {\doifvaluesomething{\??sb#1\c!pagina}
     {\ExpandFirstAfter\pagina[\getvalue{\??sb#1\c!pagina}]}%
   \resetsectioncounters[\zerosection]% was firstsection
   \resetsectionmarks[\zerosection]}

\def\dostartsectieblok#1#2%
  {\begingroup
   \doaroundsectieblok{#1}%            % going to a new page or so
   \getvalue{\??sb#1}%                 % set name of section block
   \getsectieblokomgeving{#1}%         % special settings, grouped
  %\expandafter\csname#2true\endcsname % obsolete
   \enablemode[\systemmodeprefix#1]%   % can be used in conditionals
   \getvalue{\??sb\@@sectieblok\c!voor}% this one is not to be moved!
   \showmessage{\m!structures}{1}{\@@sectieblokken}}

\def\dostopsectieblok%
  {\showmessage{\m!structures}{2}{\@@sectieblokken}%
   \getvalue{\??sb\@@sectieblok\c!na}% don't move
   \doaroundsectieblok{\@@sectieblok}%
   \endgroup}

\def\dostelsectieblokin[#1][#2]%
  {\getparameters[\??sb#1][#2]}

\def\stelsectieblokin%
  {\dodoubleargument\dostelsectieblokin}

\long\def\setsectieblokomgeving#1#2%
  {\long\setvalue{\??sb\s!do#1}{\do{#2}}}

\def\getsectieblokomgeving#1%
  {\let\do\firstofoneargument\getvalue{\??sb\s!do#1}}

\setvalue{\e!start\e!sectieblokomgeving}%
  {\dosingleargument\dostartsectieblokomgeving}

\def\dostartsectieblokomgeving[#1]% evt \pushendofline \popendofline
  {\long\def\do##1##2{\setsectieblokomgeving{#1}{##1##2}}%
   \grabuntil{\e!stop\e!sectieblokomgeving}{\getvalue{\??sb\s!do#1}}}

%D \starttypen
%D \startsectionblockenvironment[frontpart]
%D   \setuppagenumbering[conversion=romannumerals]
%D \stopsectionblockenvironment
%D
%D \startsectionblockenvironment[bodypart]
%D   \setuppagenumber[number=1]
%D \stopsectionblockenvironment
%D
%D \startsectionblockenvironment[frontpart]
%D   \setuppagenumbering[conversion=character]
%D \stopsectionblockenvironment
%D
%D \starttext
%D   \startfrontmatter \chapter{test} \stopfrontmatter
%D   \startbodymatter  \chapter{test} \stopbodymatter
%D   \startappendices  \chapter{test} \stopappendices
%D \stoptext
%D \stoptypen

% We used to use the first char as id, but a counter is
% better, because in english we get a name clash.

\newcounter\currentsectionblock

\def\dodefinieersectieblok[#1][#2][#3]%
  {\getparameters
     [\??sb#1]
     [\c!nummer=\v!ja,
      \c!pagina=\v!rechts, % anders worden marks te vroeg gereset !
      %\c!voor=,
      %\c!na=,
      #3]%
   \expandafter\newif\csname if#2\endcsname
   \doglobal\increment\currentsectionblock
   \setsectieblokomgeving{#1}{}%
   \setevalue{\??sb#1}%
     {\noexpand\dosetlocalsectieblok{\currentsectionblock}{#1}{#2}}%
   \setvalue{\e!start#2}%
     {\dostartsectieblok{#1}{#2}}%
   \setvalue{\e!stop#2}%
     {\dostopsectieblok}}

\def\definieersectieblok%
  {\dotripleargument\dodefinieersectieblok}

\def\sectiebloklabel#1#2%
  {\@EA\ifx\csname\??ko#1\@@sectieblok\c!label\endcsname\relax
     \labeltexts{#1}{#2}%
   \else
     \labeltexts{\getvalue{\??ko#1\@@sectieblok\c!label}}{#2}%
   \fi}

\dosetlocalsectieblok{2}{\v!hoofdtekst}{\v!hoofdteksten} % hm, dirty

\def\setsectiontype[#1]%
  {\getvalue{\??sb#1}}

\def\writesection#1#2#3% #3 -> \asciititle
  {\bgroup
   \edef\!!stringa{#1}%
   \@EA\writestatus\@EA
     {\!!stringa}
     {\ifsectienummer#2\else(#2)\fi\normalspace\asciititle}%
   \egroup}

\def\@@koniveau{1}        \def\kopniveau{\@@koniveau}

\def\dohandelpaginaafAA#1%
  {\ifnum\lastpenalty>0
     \global\paginageblokkeerdtrue
   \fi}

\def\dohandelpaginaafAB#1%
  {\flushsidefloats
   \getvalue{\??ko#1\c!voor}%
   %\witruimte vervangen door \noindent elders
   \relax
   \ifpaginageblokkeerd
     \global\paginageblokkeerdfalse
   \else
     \!!countb=\getvalue{\??se\@@sectie\c!niveau}\relax
     \ifnum\!!countb>\@@koniveau\relax
       \!!counta=20000
       \multiply\!!countb by 500
       \advance\!!counta by \!!countb
       \dosomebreak{\penalty\!!counta}%
     \else
       \dosomebreak{\allowbreak}%
     \fi
   \fi
   \xdef\@@koniveau{\getvalue{\??se\@@sectie\c!niveau}}}

\def\dohandelpaginaafB#1%
  {\doifinset{\getvalue{\??ko#1\c!pagina}}{\v!ja,\v!rechts,\v!links}
     {\def\resetcurrentsectionmarks% toegevoegd, zie \pagina
        {\resetsectionmarks[\previoussection{\@@sectie}]}%
      \pagina[\getvalue{\??ko#1\c!pagina}]%
      \doifinset{\getvalue{\??tk\v!hoofd\v!tekst\c!status}}{\v!normaal,\v!start}
        {\doifvaluesomething{\??ko#1\c!hoofd}
           {\stelhoofdin[\c!status=\getvalue{\??ko#1\c!hoofd}]}}%
      \doifinset{\getvalue{\??tk\v!voet\v!tekst\c!status}}{\v!normaal,\v!start}
        {\doifvaluesomething{\??ko#1\c!voet} % new
           {\stelvoetin[\c!status=\getvalue{\??ko#1\c!voet}]}}}}

\def\dohandelpaginaafX#1% zie doordefinieren / boven
  {\bgroup
   \!!countb=\@@koniveau
   \advance\!!countb by #1
   \multiply\!!countb by 500
   \!!counta=20000
   \advance\!!counta by \!!countb
   \dosomebreak{\penalty\!!counta}%
   \egroup}

\def\handelpaginaaf#1%
  {\dohandelpaginaafAA{#1}%
   \ifnum\countervalue{\??se\previoussection{\@@sectie}}>0
     \ifnum\countervalue{\??se\@@sectie}>0
       \dohandelpaginaafB{#1}%
     \else
       \doifnotvalue{\??ko#1\c!doorgaan}{\v!ja}
         {\dohandelpaginaafB{#1}}%
     \fi
   \else
     \dohandelpaginaafB{#1}%
   \fi
   \dohandelpaginaafAB{#1}}

\def\handelpaginaafC#1%
  {\xdef\@@koniveau{\getvalue{\??se\@@sectie\c!niveau}}%
   \nobreak}

%\def\dolocalkopsetup#1%  koppeling met standaard kopcommando / engels
%  {\forgetall
%   \doifvaluesomething{\??ko#1\c!uitlijnen}
%     {\ExpandFirstAfter\steluitlijnenin[\getvalue{\??ko#1\c!uitlijnen}]}%
%   \doifvaluesomething{\??ko#1\c!tolerantie}
%     {\ExpandFirstAfter\steltolerantiein[\getvalue{\??ko#1\c!tolerantie}]}%
%   \def\\{\crlf\strut\ignorespaces}}

\def\dolocalkopsetup#1%  koppeling met standaard kopcommando / engels
  {\forgetall
   \doifvaluesomething{\??ko#1\c!uitlijnen}
     {\expanded{\steluitlijnenin[\getvalue{\??ko#1\c!uitlijnen}]}}%
   \doifvaluesomething{\??ko#1\c!tolerantie}
     {\expanded{\steltolerantiein[\getvalue{\??ko#1\c!tolerantie}]}}%
   \def\\{\crlf\strut\ignorespaces}}

\newif\ifplaatskop
\newif\ifverhoognummer
\newif\ifkopnummer

\def\setsectieenkoppeling#1%
  {\edef\@@koppeling{\getvalue{\??ko#1\c!koppeling}}%
   \edef\@@sectie{\getvalue{\??ko#1\c!sectie}}%
   \doifnothing{\@@koppeling}
     {\edef\@@koppeling{#1}}%
   \doifnothing{\@@sectie}
     {\edef\@@sectie{\getvalue{\??ko\@@koppeling\c!sectie}}}}

\newif\ifkopprefix

% \handelpaginaaf komt het eerst omdat eventueel
% subpaginanummers moeten worden afgehandeld. Vervolgens
% worden de nummers opgehoogd en referenties geset, dan
% volgt de kop en tot slot de worden de marks en de prefix
% geset.

% \hoofdstuk {tekst}
% \hoofdstuk tekst
% \hoofdstuk <niets>

\def\dodosomekop#1[#2]#3%
  {\doifelsevalue{\??ko#1\c!eigennummer}{\v!ja}
     {\def\next{\doquadruplegroupempty\dododosomekop{#1}{#2}{#3}}}
     {\def\next{\fourthargumentfalse\dododosomekop{#1}{#2}{#3}{}}}%
   \next}

\def\finalsectionnumber%
  {\ifundefined{\@@sectie\c!nummer}\else
     \ifsomeheadconversion
       \@@shortsectionnumber{\@@sectie}%
     \else
       \getvalue{\@@sectie\c!nummer}%
     \fi
   \fi}

\def\dododosomekop#1#2#3#4%
  {\iffourthargument
     \def\next%
       {\dodododosomekop{#1}[#2]{#1}{#3}{#4}}%
   \else
     \def\next%
       {\dodododosomekop{#1}[#2]{#1}{\finalsectionnumber}{#3}}%
   \fi
   \next}

\def\findsectionnumber#1#2#3% class file title
  {\begingroup
   \setsectieenkoppeling{#1}%
   \xdef\foundsectionnumber{1}%
   \def\dolijstelement##1##2##3##4##5##6%
     {\doif{##1}{#1}
        {\ConvertConstantAfter\doif{##4}{#3}
           {\global\utilitydonetrue
            \scratchcounter=0\getvalue{\??se\@@sectie\c!niveau}%
            \advance\scratchcounter by 2
            \def\do####1:####2]%
              {\advance\scratchcounter by -1
               \ifcase\scratchcounter
                 \xdef\foundsectionnumber{####1}%
               \else
                 \do####2]%
               \fi}%
            \do##5]}}}%
   \setbox0=\vbox
     {\doutilities{#1}{#2}{#1}{}{}}%
   \endgroup
   \doifnumberelse{\foundsectionnumber}
     {\doif{\foundsectionnumber}{0}{\xdef\foundsectionnumber{1}}}
     {\xdef\foundsectionnumber{1}}% an appendix or so
   \stelkopnummerin[#1][\foundsectionnumber]%
   \stelkopnummerin[#1][-1]}

\newif\ifsomeheadconversion

\def\setsomeheadconversion#1#2%
   {\someheadconversionfalse
    \doifelsevalue{\??ko#1\c!eigennummer}{\v!ja}
      {\def\someheadconversion{#2}}
      {\bepaalkopnummer[#1]%
       \@EA\ifx\csname\??se\@@sectie\@@sectieblok\c!kopconversie\endcsname\relax
         \@EA\ifx\csname\??se\@@sectie\c!kopconversie\endcsname\relax
           \def\someheadconversion{#2}%
         \else
           \@EA\ifx\csname\??se\@@sectie\c!kopconversie\endcsname\empty
             \def\someheadconversion{#2}%
           \else
             \someheadconversiontrue
             \def\someheadconversion%
               {\fullsectionnumber{#1}{\getvalue{\??se\@@sectie\c!kopconversie}}{#2}}%
           \fi
         \fi
       \else
         \@EA\ifx\csname\??se\@@sectie\@@sectieblok\c!kopconversie\endcsname\empty
           \def\someheadconversion{#2}%
         \else
           \someheadconversiontrue
           \def\someheadconversion%
             {\fullsectionnumber{#1}{\getvalue{\??se\@@sectie\@@sectieblok\c!kopconversie}}{#2}}%
         \fi
       \fi}}

           \def\writtenfullsectionnumber      {\string\fullsectionnumber}
           \def\ignoredfullsectionnumber#1#2#3{#3}
           \let \storedfullsectionnumber       \relax
\unexpanded\def\naturalfullsectionnumber#1#2#3{\sectiebloklabel{#1}{\getvalue{\??cv#2}{#3}}}
\unexpanded\def\limitedfullsectionnumber#1#2#3{\getvalue{\??cv#2}{#3}}

\def\setfullsectionnumber#1%
  {\doifelsevalue{#1\c!kopconversie}{\v!ja}
     {\doifelsevalue{#1\c!koplabel}{\v!ja}
        {\let\fullsectionnumber\naturalfullsectionnumber}
        {\let\fullsectionnumber\limitedfullsectionnumber}}
     {\let\fullsectionnumber\ignoredfullsectionnumber}}

%\let\fullsectionnumber\naturalfullsectionnumber
\let\fullsectionnumber\limitedfullsectionnumber

% \dodododosomekop IS NON GROUPED, SO WE NEED TO RESTORE !!!!
%
% dit kan dus beter \everyaroundhead zijn

\def\dodododosomekop#1[#2]#3#4#5% % pas met \ExpandFirstAfter op bij twee||taligheid
  {\flushingcolumnfloatsfalse     % #3=#1=redundant
   \someheadconversionfalse
%   \let\fullsectionnumber\naturalfullsectionnumber
\let\fullsectionnumber\limitedfullsectionnumber
   \setsectieenkoppeling{#1}%
   \doifelsevalue{\??ko#1\c!prefix}{}
     {\kopprefixfalse}
     {\kopprefixtrue}%
   \ifkopprefix
     \doifelsevalue{\??ko#1\c!prefix}{+}
       {\doifelsenothing{#2}
          {\def\localkopprefix{+}}
          {\def\localkopprefix{#2}}} % eigenlijk alleen eerste
       {\edef\localkoprefix{\getvalue{\??ko#1\c!prefix}}}%
   \fi
   \doifelsevalue{\??ko#1\c!plaatskop}{\v!ja}
     {\plaatskoptrue}
     {\plaatskopfalse}%
   \processaction
     [\getvalue{\??ko#1\c!verhoognummer}]
     [     \v!ja=>\verhoognummertrue,
          \v!nee=>\verhoognummerfalse,
      \s!unknown=>{\ifx\currentproduct\empty
                     \findsectionnumber{#1}\commalistelement{#5}%
                   \fi
                   \verhoognummertrue}]%
   \edef\numberheaddistance   {\getvalue{\??ko#1\c!afstand}}%
   \edef\numberheadalternative{\getvalue{\??ko#1\c!variant}}%
   \dostelkopvariantin[\numberheadalternative]%
   \ifsectienummer
     \doifelsevalue{\??sb\@@sectieblok\c!nummer}{\v!ja}
       {\doifelsevalue{\??ko#1\c!nummer}{\v!ja}
          {\kopnummertrue}{\kopnummerfalse}}
       {\kopnummerfalse}%
   \else
     \kopnummerfalse
   \fi
   \convertexpanded{\??ko#1}{#5}\asciititle
   \ifverhoognummer
     \ifplaatskop
       \checknexthead\handelpaginaaf{#1}%
       \setsectieenkoppeling{#1}% can be changed when [voor=\somehead{..}...]
       \ifkopprefix
         \setupreferencing[\c!prefix=-]%
       \fi
       \getvalue{\e!volgende\@@sectie}%
       \getvalue{\??ko#1\c!tussen}%
       \ifkopnummer
\setsomeheadconversion{#1}{#4}%
\ifsomeheadconversion
         \let\fullsectionnumber\naturalfullsectionnumber
         \doplaatskopnummertekst
           {#1}
           {\setsectionlistreference{\@@sectie}{#1}%
            \ExpandFirstAfter\soortpagina[\@@koppeling]%
%            \let\fullsectionnumber\storedfullsectionnumber
            \let\fullsectionnumber\writtenfullsectionnumber
            \rawreference{\s!sec}{#2}{{\someheadconversion}{\asciititle}}%
            \resetsectionmarks[\@@sectie]%
            \stellijstin[\@@koppeling][\c!expansie=\getvalue{\??ko#1\c!expansie}]%
            \let\fullsectionnumber\writtenfullsectionnumber
            \doschrijfnaarlijst{\@@koppeling}{\someheadconversion}{#5}{\v!kop}}%
           {\someheadconversion}
           {#5}
           {\marking[#1]{#5}%
            \let\fullsectionnumber\storedfullsectionnumber
            \expanded{\marking[#1\v!nummer]{\someheadconversion}}}%
         \let\fullsectionnumber\ignoredfullsectionnumber
         \writesection{#1}{\someheadconversion}{#5}%
\else
         \doplaatskopnummertekst
           {#1}
           {\setsectionlistreference{\@@sectie}{#1}%
            \ExpandFirstAfter\soortpagina[\@@koppeling]%
            \rawreference{\s!sec}{#2}{{#4}{\asciititle}}%
            \resetsectionmarks[\@@sectie]%
            \stellijstin[\@@koppeling][\c!expansie=\getvalue{\??ko#1\c!expansie}]%
            \doschrijfnaarlijst{\@@koppeling}{#4}{#5}{\v!kop}}
           {\sectiebloklabel{#3}{#4}}
           {#5}
           {\marking[#1]{#5}%
            \doifelsevalue{\??ko#1\c!eigennummer}{\v!ja} % rommelig omdat
              {\edef\finalsectionnumber{#4}}             % #4 al is toegekend
              {\bepaalkopnummer[#1]}% migreert naar 3e argument
            \expanded{\marking[#1\v!nummer]{\finalsectionnumber}}}%
         \writesection{#1}{#4}{#5}%
 \fi
       \else
         \doplaatskoptekst
           {#1}
           {\setsectionlistreference{\@@sectie}{#1}%
            \ExpandFirstAfter\soortpagina[\@@koppeling]%
            \rawreference{\s!sec}{#2}{{#4}{\asciititle}}%
            \resetsectionmarks[\@@sectie]%
            \stellijstin[\@@koppeling][\c!expansie=\getvalue{\??ko#1\c!expansie}]%
            \doschrijfnaarlijst{\@@koppeling}{}{#5}{\v!kop}}
           {#5}
           {\marking[#1]{#5}%
            \doifelsevalue{\??ko#1\c!eigennummer}{\v!ja}
              {\edef\finalsectionnumber{#4}}
              {\bepaalkopnummer[#1]}%
            \expanded{\marking[#1\v!nummer]{\finalsectionnumber}}}%
         \writesection{#1}{-}{#5}%
       \fi
       \ifkopprefix
         \ExpandFirstAfter\setupreferencing[\c!prefix=\localkopprefix]%
       \fi
       \dosomebreak\nobreak
       \ifdisplaysectionhead\getvalue{\??ko#1\c!na}\fi
     \else
       \checknexthead\dohandelpaginaafB{#1}% toegevoegd ivm subpaginanr / tug sheets
       \setsectieenkoppeling{#1}% can be changed when [voor=\somehead{..}...]
       \ifkopprefix
         \setupreferencing[\c!prefix=-]%
       \fi
       \getvalue{\e!volgende\@@sectie}%
       \getvalue{\??ko#1\c!tussen}%
       \setsectionlistreference{\@@sectie}{#1}%
       \resetsectionmarks[\@@sectie]%
       \marking[#1]{#5}%
       \doifelsevalue{\??ko#1\c!eigennummer}{\v!ja}
         {\edef\finalsectionnumber{#4}}
         {\bepaalkopnummer[#1]}%
       \expanded{\marking[#1\v!nummer]{\finalsectionnumber}}%
       \ExpandFirstAfter\soortpagina[\@@koppeling]%
       \bgroup
       \stellijstin[\@@koppeling][\c!expansie=\getvalue{\??ko#1\c!expansie}]%
       \ifkopnummer
         \rawreference{\s!sec}{#2}{{#4}{\asciititle}}%
         \doschrijfnaarlijst{\@@koppeling}{#4}{#5}{\v!kop}%
         \writesection{#1}{#4}{#5}%
       \else
         \rawreference{\s!sec}{#2}{{#4}{\asciititle}}%
         \doschrijfnaarlijst{\@@koppeling}{}{#5}{\v!kop}%
         \writesection{#1}{-}{#5}%
       \fi
       \egroup
       \ifkopprefix
         \ExpandFirstAfter\setupreferencing[\c!prefix=\localkopprefix]%
       \fi
     \fi
   \else
     \ifplaatskop
       \checknexthead\handelpaginaaf{#1}%
       \setsectieenkoppeling{#1}% can be changed when [voor=\somehead{..}...]
       \getvalue{\??ko#1\c!tussen}%
       \doplaatskoptekst
         {#1}
         {\rawreference{\s!sec}{#2}{{#4}{\asciititle}}}
         {#5}
         {}%
       \writesection{#1}{-}{#5}%
       \dosomebreak\nobreak
       \ifdisplaysectionhead\getvalue{\??ko#1\c!na}\fi
     \else
       % do nothing
     \fi
   \fi
   \flushingcolumnfloatstrue
   \someheadconversionfalse
%   \let\fullsectionnumber\naturalfullsectionnumber
\let\fullsectionnumber\limitedfullsectionnumber
   \ifdisplaysectionhead\else\expandafter\GotoPar\fi}

% \prevdepth\dp\strutbox is belangrijk, vergelijk naast elkaar:
%
% \onderwerp{test} \input tufte
% \onderwerp{test} \strut \input tufte
% \onderwerp{test} \plaatslijst[...]

\newif\ifheadnumbercontent

\def\doplaatskoptekst#1#2#3#4%
  {\beginheadplacement{#1}%
   \setbox0=\ifvertical\vbox\else\hbox\fi % \vhbox
     {\headnumbercontentfalse
      \getvalue{\??ko#1\c!commando}
        {} % no number
        {\doattributes
           {\??ko#1}\c!letter\c!kleur
           {\doattributes
              {\??ko#1}\c!tekstletter\c!tekstkleur
              {\dontconvertfont
               \ifdisplaysectionhead
                 \stelinterliniein
               \else
                 \stelspatieringin
               \fi
               #2%
               \getvalue{\??ko#1\c!voorcommando}%
               \ifdisplaysectionhead
                 \getvalue{\??ko#1\c!tekstcommando}%
                   {\setstrut\begstrut#3\endstrut}
                 \xdef\localheaddepth{\the\dp\strutbox}%
               \else
                 \getvalue{\??ko#1\c!tekstcommando}{#3}%
               \fi
               \getvalue{\??ko#1\c!nacommando}%
               \ifdisplaysectionhead\endgraf\fi}}}}%
   \endheadplacement{#1}{#4}}

\def\doplaatskopnummertekst#1#2#3#4#5%
  {\beginheadplacement{#1}%
   \setbox0=\ifvertical\vbox\else\hbox\fi % \vhbox
     {\doiftextelse{#3}
        {\headnumbercontenttrue}{\headnumbercontentfalse}%
      \getvalue{\??ko#1\c!commando}%
        {\doattributes{\??ko#1}\c!letter\c!kleur
           {\doattributes{\??ko#1}\c!nummerletter\c!nummerkleur
              {\getvalue{\??ko#1\c!voorcommando}%
               \ifdisplaysectionhead
                 \getvalue{\??ko#1\c!nummercommando}%
                   {\setstrut\begstrut#3\endstrut}%
               \else
                 \getvalue{\??ko#1\c!nummercommando}{#3}%
               \fi}}}
        {\doattributes{\??ko#1}\c!letter\c!kleur
           {\doattributes{\??ko#1}\c!tekstletter\c!tekstkleur
              {\dontconvertfont
               \ifdisplaysectionhead
                 \stelinterliniein
               \else
                 \stelspatieringin
               \fi
               #2%
               \ifdisplaysectionhead
                 \getvalue{\??ko#1\c!tekstcommando}%
                   {\setstrut\begstrut#4\endstrut}%
                 \xdef\localheaddepth{\the\dp\strutbox}%
               \else
                 \getvalue{\??ko#1\c!tekstcommando}{#4}%
               \fi
               \getvalue{\??ko#1\c!nacommando}%
               \ifdisplaysectionhead\endgraf\fi}}}}%
   \endheadplacement{#1}{#5}}

\newsignal\headsignal
\let\headlastlinewidth\!!zeropoint
\newif\ifcontinuoushead

\def\beginheadplacement#1%
  {\bgroup
   \gdef\localheaddepth{\dp\strutbox}%
   \everypar{}% needed indeed
   \noindent  % ipv \witruimte elders, na \forgetall !
   \bgroup
   \forgetall % now we may forget everything
  %\showcomposition
   \mindermeldingen
   \postponefootnotes
   \iflocation\ifdisplaysectionhead\else\noninterferingmarks\fi\fi
   \setupinteraction
     [\c!letter=,
      \c!kleur=,
      \c!contrastkleur=]%
   \strictouterreferencestrue % tzt instelling
   \def\localkopsetup%
     {\dolocalkopsetup{#1}}%
   \startsynchronisatie}

\def\endheadplacement#1#2%
  {\doifelsevalue{\??rf#1\c!status}{\v!start}
     {\doifvaluenothing{\??ko#1\c!file}{\autocrossdocumentfalse}}
     {\autocrossdocumentfalse}%
   % no message needed here, should be a proper switch
   \let\unknownreference\relax
   %
   \ifdisplaysectionhead
     \let\headlastlinewidth\!!zeropoint
     \snaptogrid\hbox
       {\iflocation
          \ifautocrossdocument
            \doifreferencefoundelse{\getvalue{\??ko#1\c!file}::#1}
              {\edef\currentinnerreference{\s!aut:\currenttextreference}% stored in
               \gotoouterlocation{}{\box0}}                             % text slot
              {\hbox{\box0}}%
          \else
            \hbox{\box0}%
          \fi
        \else
          \hbox{\box0}%
        \fi}%
     \doflushfootnotes % new, not really needed
     \endgraf
     \nointerlineskip
     \dosomebreak\nobreak
     #2%
   \else
     \strut
     \doflushfootnotes % new, here since we're in par mode
     \iflocation
       \ifautocrossdocument
         \hhboxindent=\ifcontinuoushead\headlastlinewidth\else\!!zeropoint\fi
         \unhhbox0\with{\naarbox{\box\hhbox}[\getvalue{\??ko#1\c!file}::#1]}%
         \advance\lasthhboxwidth by \numberheaddistance
         \xdef\headlastlinewidth{\the\lasthhboxwidth}%
       \else
         \unhbox0
         \global\let\headlastlinewidth\!!zeropoint
       \fi
     \else
       \unhbox0
       \global\let\headlastlinewidth\!!zeropoint
     \fi
     #2%
     \dimen0=\numberheaddistance
     \hskip\dimen0 \!!plus \dimen0 \!!minus .25\dimen0
     \hskip\headsignal\ignorespaces
   \fi
   \ifdisplaysectionhead
     \ifgridsnapping % important, font related depth, see comment
       \prevdepth\dp\strutbox
     \else
       \prevdepth\localheaddepth
     \fi
   \fi
   \stopsynchronisatie
   \egroup
   \egroup
   \ifdisplaysectionhead
     \doifvalue{\??ko#1\c!springvolgendein}{\v!nee}{\noindentation}%
   \else
     \nonoindentation % recently added, was a bug
   \fi}

\def\checknexthead#1#2% nog optioneel
  {\ifhmode
     \scratchcounter=\lastpenalty\unpenalty % no beauty in this
     \ifdim\lastskip=\headsignal
       \handelpaginaafC{#1}%
       \global\continuousheadtrue
     \else
       \penalty\scratchcounter
       \global\continuousheadfalse
       #1{#2}%
     \fi
   \else
     \global\continuousheadfalse
     #1{#2}%
   \fi}

\def\dostelkopnummerin[#1][#2#3]%
  {\bgroup
   \setsectieenkoppeling{#1}%
   \doifinstringelse{#2}{+-}
     {\doifelse{#3}{}
        {\@@nextsectionnumber{\@@sectie}}
        {\!!counta=#2#3\relax
         \advance\!!counta by \@@sectionvalue{\@@sectie}%
         \@@setsectionnumber{\@@sectie}{\!!counta}}}
     {\@@setsectionnumber{\@@sectie}{#2#3}}%
   \egroup}

\def\stelkopnummerin%
  {\dodoubleargument\dostelkopnummerin}

% \def\dokopnummer[#1]%
%   {\bgroup
%    \setsectieenkoppeling{#1}%
%    \doifnot{\finalsectionnumber}{0} % kan effienter
%      {\finalsectionnumber}%
%    \egroup}
%
% beter :

\def\huidigekopnummer{0}

\def\bepaalkopnummer[#1]%
  {\bgroup
   \setsectieenkoppeling{#1}%
   \xdef\huidigekopnummer{\@@sectionvalue{\@@sectie}}%
   \egroup}

%\def\complexkopnummer[#1]%
%  {\bgroup
%   \setsectieenkoppeling{#1}%
%   \xdef\huidigekopnummer{\@@sectionvalue{\@@sectie}}%
%   \doifnot{\huidigekopnummer}{0}
%     {\finalsectionnumber}%
%   \egroup}

\def\complexkopnummer[#1]%
  {\bgroup
   \edef\huidigekopnummer{#1}%
   \doifinsetelse{-}{#1}
     {\removefromcommalist{-}\huidigekopnummer
      \setsectieenkoppeling\huidigekopnummer
      \stelsectiein[\@@sectie][\c!vorigenummer=\v!nee]}%
     {\setsectieenkoppeling\huidigekopnummer}%
   \xdef\huidigekopnummer{\@@sectionvalue{\@@sectie}}%
   \doifnot{\huidigekopnummer}{0}{\finalsectionnumber}%
   \egroup}

\def\simplekopnummer%
  {\huidigekopnummer}

\definecomplexorsimple\kopnummer

\def\alinea%
  {\par}

\def\plaatskopalinea#1#2%
  {\vbox
     {\localkopsetup
      \begstrut\ifheadnumbercontent#1\hskip\numberheaddistance\fi#2}}

\def\plaatskopnormaal#1#2%
  {\ifheadnumbercontent
     \setbox0=\hbox{{#1}\hskip\numberheaddistance}%
     \vbox
       {\localkopsetup
        \hangindent 1\wd0
        \hangafter 1
        \noindent
        \unhbox0    %  don't use \strut's here!
        #2}%
   \else
     \vbox
       {\localkopsetup\noindent#2}%
   \fi}

\def\plaatskopinmarge#1#2%
  {\vbox
     {\localkopsetup
      \begstrut            % but use one \strut here!
      \ifheadnumbercontent      
        \llap{\hbox to 5em{\hfill{#1}\hskip\linkermargeafstand}}%
      \fi
      {#2}}}

\def\plaatskopmidden#1#2%
  {\vbox
     {\localkopsetup
      \veryraggedcenter
      \let\\\endgraf
      \let\crlf\endgraf
      \ifheadnumbercontent\strut#1\par\fi\begstrut#2}}

\def\plaatskopintekst#1#2%
  {\bgroup
   \localkopsetup % no stretch in distance
   \ifheadnumbercontent{#1}\kern\numberheaddistance\fi{\begstrut#2}%
   \egroup}

% default   == instellingen
% koppeling == koppen, breaks, marks, enz.
% sectie    == nummering

\let\@@kolijst=\empty

\def\dodefinieerkop[#1][#2]%   % don't preset prefix to much
  {\presetlabeltext[#1=]%
   \getparameters
     [\??ko#1]
     [\c!nummerletter=\getvalue{\??ko#1\c!letter},
      \c!tekstletter=\getvalue{\??ko#1\c!letter},
      \c!nummerkleur=\getvalue{\??ko#1\c!kleur},
      \c!tekstkleur=\getvalue{\??ko#1\c!kleur}]%
   \ConvertToConstant\doifinstringelse{=}{#2}
     {\getparameters
        [\??ko#1]
        [\c!sectie=\getvalue{\??ko\getvalue{\??ko#1\c!koppeling}\c!sectie},
         \c!default=,
         \c!koppeling=,
         \c!prefix=,
         \c!voor=,
         \c!na=,
         \c!afstand=,
         \c!pagina=,
         \c!hoofd=,
         \c!voet=,
         \c!letter=,
         \c!nummercommando=,
         \c!tekstcommando=,
         \c!eigennummer=\v!nee,
         \c!nummer=\v!ja,
         \c!kleur=,
         \c!springvolgendein=\v!nee,
         \c!doorgaan=\v!ja,
         \c!plaatskop=\v!ja,
         \c!verhoognummer=\v!ja,
         \c!variant=\@@kovariant,
         \c!commando=\@@plaatskop,
         \c!uitlijnen=,
         \c!tolerantie=,
         \c!file=,
         \c!expansie=,
         #2]%
      \ConvertToConstant\doifnot{#1}{\getvalue{\??ko#1\c!default}}
        {\doifsomething{\getvalue{\??ko#1\c!default}}
           {\copyparameters
              [\??ko#1][\??ko\getvalue{\??ko#1\c!default}]
              [\c!voor,\c!na,\c!commando,\c!file,\c!pagina,\c!doorgaan,\c!hoofd,\c!voet,
               \c!nummer,\c!eigennummer,\c!plaatskop,\c!verhoognummer,
               \c!letter,\c!kleur,\c!afstand,\c!variant,\c!springvolgendein,
              %\c!nummerletter,\c!tekstletter,
              %\c!expansie, % njet
               \c!uitlijnen,\c!tolerantie,\c!nummercommando,\c!tekstcommando]}}%
      \getparameters[\??ko#1][#2]%
      \doifsomething{\getvalue{\??ko#1\c!sectie}}
        {\doifundefined{\??mk#1}
           {\definieermarkering[#1]%
            \koppelmarkering[#1][\getvalue{\??ko#1\c!sectie}]%
            \definieermarkering[#1\v!nummer]%
            \koppelmarkering[#1\v!nummer][\getvalue{\??ko#1\c!sectie}]}}%
%            \koppelmarkering[#1\v!nummer][\getvalue{\??ko#1\c!sectie}\v!nummer]}}%
      \doifundefined{\??li#1}{\definieerlijst[#1]}}
     {\ConvertToConstant\doifelse{#1}{#2}
        {\doifundefined{\??li#1}{\definieerlijst[#1]}}
        {\copyparameters
           [\??ko#1][\??ko#2]
           [\c!niveau,\c!sectie,\c!koppeling,\c!prefix,
            \c!voor,\c!na,\c!commando,\c!file,\c!pagina,\c!doorgaan,\c!hoofd,\c!voet,
            \c!nummer,\c!eigennummer,\c!plaatskop,\c!verhoognummer,
            \c!letter,\c!kleur,\c!afstand,\c!variant,\c!springvolgendein,
           %\c!nummerletter,\c!tekstletter,
           %\c!expansie, % njet
            \c!uitlijnen,\c!tolerantie,\c!nummercommando,\c!tekstcommando]%
         \definieermarkering[#1][#2]%
         \definieermarkering[#1\v!nummer][#2\v!nummer]%
         \doifundefined{\??li#1}{\definieerlijst[#1][#2]}}}%
   \addtocommalist{#1}\@@kolijst
   \setevalue{\??sk#1}%
     {\getvalue{\??ko#1\c!koppeling}}%
   \setevalue{\??by#1}%
     {\getvalue{\??ko#1\c!sectie}}%
   \setevalue{\??by\v!per#1}%
     {\getvalue{\??ko#1\c!sectie}}%
   \setvalue{#1}%
     {\dodoubleempty\dosomekop[#1]}}

\def\definieerkop%
  {\dodoubleemptywithset\dodefinieerkop}

\def\dosomekop[#1][#2]%
  {\dowithpargument{\dodosomekop{#1}[#2]}}

\def\dostelkopin[#1][#2]%
  {\getparameters[\??ko#1][#2]%
   % The next check prevents hard to trace problems. I once
   % set \c!commando to nothing and (quite natural) got the
   % wrong references etc. The whole bunch should be boxed!
   \expandafter\convertcommand\csname\??ko#1\c!commando\endcsname\to\ascii
   \doifnothing{\ascii}
     {\setvalue{\??ko#1\c!commando}{\@@plaatskop}}}

\def\stelkopin%
  {\dodoubleargumentwithset\dostelkopin}

\newif\ifsectienummer       \sectienummertrue
\newif\ifdisplaysectionhead \displaysectionheadtrue

\def\@@plaatskop{\plaatskopnormaal}

\def\dostelkopvariantin[#1]%
  {\displaysectionheadtrue
   \processaction
     [#1]
     [ \v!normaal=>\def\@@plaatskop{\plaatskopnormaal},
        \v!midden=>\def\@@plaatskop{\plaatskopmidden},
         \v!marge=>\def\@@plaatskop{\plaatskopinmarge},
       \v!inmarge=>\def\@@plaatskop{\plaatskopinmarge},
         \v!tekst=>\def\@@plaatskop{\plaatskopintekst}\displaysectionheadfalse,
        \v!alinea=>\def\@@plaatskop{\plaatskopalinea},
       \s!unknown=>\def\@@plaatskop{\plaatskopnormaal}]}

\def\dostelkoppenin[#1]%
  {\getparameters[\??ko][#1]%
   \doifelse{\@@kosectienummer}{\v!ja}
     {\sectienummertrue}
     {\sectienummerfalse}%
   \dostelkopvariantin[\@@kovariant]}

\def\stelkoppenin%
  {\dosingleargument\dostelkoppenin}

\def\systemsuppliedchapter {\getvalue{\v!hoofdstuk}}
\def\systemsuppliedtitle   {\getvalue{\v!titel}}

% a left over

\def\complexbijlage[#1]#2%
  {\pagina[\v!rechts]
   \stelnummeringin[\c!status=\v!stop]
   \systemsuppliedchapter[#1]{#2}
   \pagina[\v!rechts]
   \stelnummeringin[\c!status=\v!start]
   \stelpaginanummerin[\c!nummer=1]}

\setvalue{\v!bijlage}%
  {\complexorsimpleempty\bijlage}

\stelkoppenin
  [\c!variant=\v!normaal,
   \c!sectienummer=\v!ja,
   \c!scheider=.,
   \c!commando=]

\definieersectieblok [\v!hoofdtekst] [\v!hoofdteksten] [\c!nummer=\v!ja]
\definieersectieblok [\v!bijlage]    [\v!bijlagen]     [\c!nummer=\v!ja]
\definieersectieblok [\v!inleiding]  [\v!inleidingen]  [\c!nummer=\v!nee]
\definieersectieblok [\v!uitleiding] [\v!uitleidingen] [\c!nummer=\v!nee]

\definieersectie[\v!sectionlevel-1]   % deel
\definieersectie[\v!sectionlevel-2]   % hoofdstuk
\definieersectie[\v!sectionlevel-3]   % paragraaf
\definieersectie[\v!sectionlevel-4]   % subparagraaf
\definieersectie[\v!sectionlevel-5]   % subsubparagraaf
\definieersectie[\v!sectionlevel-6]   % subsubsubparagraaf
\definieersectie[\v!sectionlevel-7]   % subsubsubsubparagraaf

% \c!eigennummer ook hier?

\definieerkop
  [\v!deel]
  [\c!sectie=\v!sectionlevel-1]

\definieerkop
  [\v!hoofdstuk]
  [\c!sectie=\v!sectionlevel-2]

\definieerkop
  [\v!paragraaf]
  [\c!sectie=\v!sectionlevel-3]

\definieerkop
  [\v!sub\v!paragraaf]
  [\c!sectie=\v!sectionlevel-4,
   \c!default=\v!paragraaf]

\definieerkop
  [\v!sub\v!sub\v!paragraaf]
  [\c!sectie=\v!sectionlevel-5,
  %\c!default=\v!paragraaf]
   \c!default=\v!sub\v!paragraaf]             % nieuw

\definieerkop
  [\v!sub\v!sub\v!sub\v!paragraaf]
  [\c!sectie=\v!sectionlevel-6,
  %\c!default=\v!paragraaf]
   \c!default=\v!sub\v!sub\v!paragraaf]       % nieuw

\definieerkop
  [\v!sub\v!sub\v!sub\v!sub\v!paragraaf]
  [\c!sectie=\v!sectionlevel-7,
  %\c!default=\v!paragraaf]
   \c!default=\v!sub\v!sub\v!sub\v!paragraaf] % nieuw

\definieerkop
  [\v!titel]
  [\c!koppeling=\v!hoofdstuk,
   \c!default=\v!hoofdstuk,
   \c!verhoognummer=\v!nee]

\definieerkop
  [\v!onderwerp]
  [\c!koppeling=\v!paragraaf,
   \c!default=\v!paragraaf,
   \c!verhoognummer=\v!nee]

\definieerkop
  [\v!sub\v!onderwerp]
  [\c!koppeling=\v!sub\v!paragraaf,
   \c!default=\v!sub\v!paragraaf,
   \c!verhoognummer=\v!nee]

\definieerkop
  [\v!sub\v!sub\v!onderwerp]
  [\c!koppeling=\v!sub\v!sub\v!paragraaf,
   \c!default=\v!sub\v!sub\v!paragraaf,
   \c!verhoognummer=\v!nee]

\definieerkop
  [\v!sub\v!sub\v!sub\v!onderwerp]
  [\c!koppeling=\v!sub\v!sub\v!sub\v!paragraaf,
   \c!default=\v!sub\v!sub\v!sub\v!paragraaf,
   \c!verhoognummer=\v!nee]

\definieerkop
  [\v!sub\v!sub\v!sub\v!sub\v!onderwerp]
  [\c!koppeling=\v!sub\v!sub\v!sub\v!sub\v!paragraaf,
   \c!default=\v!sub\v!sub\v!sub\v!sub\v!paragraaf,
   \c!verhoognummer=\v!nee]

\stelsectiein
  [\v!sectionlevel-2]
  [\v!bijlage\c!conversie=\v!Letter,
   \c!vorigenummer=\v!nee]

\stelkopin
  [\v!deel]
  [\c!plaatskop=\v!nee]

\stelkopin
  [\v!hoofdstuk]
  [\v!bijlage\c!label=\v!bijlage,
   \v!hoofdtekst\c!label=\v!hoofdstuk]             %   bijlageconversie=\Character

\stelkopin
  [\v!paragraaf]
  [\v!bijlage\c!label=\v!paragraaf,
   \v!hoofdtekst\c!label=\v!paragraaf]             %   bijlageconversie=\Character

\stelkopin
  [\v!sub\v!paragraaf]
  [\v!bijlage\c!label=\v!sub\v!paragraaf,
   \v!hoofdtekst\c!label=\v!sub\v!paragraaf]       %   bijlageconversie=\Character

\stelkopin
  [\v!sub\v!sub\v!paragraaf]
  [\v!bijlage\c!label=\v!sub\v!sub\v!paragraaf,
   \v!hoofdtekst\c!label=\v!sub\v!sub\v!paragraaf] %   bijlageconversie=\Character

\stelkopin
  [\v!deel,\v!hoofdstuk]
  [\c!uitlijnen=,
   \c!doorgaan=\v!nee,
   \c!springvolgendein=\v!nee,
   \c!pagina=\v!rechts,
   \c!hoofd=,
   \c!letter=\tfc,
   \c!afstand=.75em,
   \c!voor={\blanko[2*\v!groot]},
   \c!na={\blanko[2*\v!groot]}]

\stelkopin
  [\v!paragraaf]
  [\c!uitlijnen=,
   \c!letter=\tfa,
   \c!afstand=.75em,
   \c!springvolgendein=\v!nee,
   \c!voor={\blanko[2*\v!groot]},
   \c!na=\blanko]

\stelkopin                 % nieuw
  [\v!sub\v!paragraaf]
  [\c!pagina=]

\definieersamengesteldelijst
  [\v!inhoud]
  [\v!deel,
   \v!hoofdstuk,
   \v!paragraaf,
   \v!sub\v!paragraaf,
   \v!sub\v!sub\v!paragraaf,
   \v!sub\v!sub\v!sub\v!paragraaf,
   \v!sub\v!sub\v!sub\v!sub\v!paragraaf]
  [\c!niveau=\v!sub\v!sub\v!sub\v!sub\v!paragraaf,
   \c!criterium=\v!lokaal]

\stellijstin
  [\v!deel]
  [\c!breedte=0em,
   \c!voor={\blanko\pagina[\v!voorkeur]},
   \c!na=\blanko,
   \c!label=\v!ja,
   \c!scheider=:,
   \c!afstand=1em]

\stellijstin
  [\v!hoofdstuk]
  [\c!breedte=2em,
   \c!voor={\blanko\pagina[\v!voorkeur]},
   \c!na=]

\stellijstin
  [\v!paragraaf]
  [\c!breedte=3em]

\stellijstin
  [\v!sub\v!paragraaf]
  [\c!breedte=4em]

\stellijstin
  [\v!sub\v!sub\v!paragraaf]
  [\c!breedte=5em]

\stellijstin
  [\v!sub\v!sub\v!sub\v!paragraaf]
  [\c!breedte=6em]

\stellijstin
  [\v!sub\v!sub\v!sub\v!sub\v!paragraaf]
  [\c!breedte=7em]

% hm

\stelnummeringin % na instellen hoofdteksten !
  [\c!variant=\v!enkelzijdig,
   \c!plaats={\v!hoofd,\v!midden},
   \c!conversie=\v!cijfers,
   \c!links=,
   \c!rechts=,
   \c!wijze=\v!per\v!deel,
   \c!tekst=,
   \v!hoofdstuk\v!nummer=\v!nee, % v
   \v!deel\v!nummer=\v!ja,       % v
   \c!nummerscheider=--,
   \c!tekstscheider=\tfskip,
   \c!status=\v!start,
   \c!commando=,
   \c!letter=, % \v!normaal, % empty, otherwise conflict 
   \c!kleur=]

\protect \endinput
