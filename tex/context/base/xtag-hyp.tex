%D \module
%D   [       file=xtag-hyp,
%D        version=2003.11.24,
%D          title=\CONTEXT\ XML Support,
%D       subtitle=hyphenation support,
%D         author=Hans Hagen,
%D           date=\currentdate,
%D      copyright={PRAGMA / Hans Hagen \& Ton Otten}]
%C
%C This module is part of the \CONTEXT\ macro||package and is
%C therefore copyrighted by \PRAGMA. See mreadme.pdf for
%C details.

\writestatus{loading}{Context XML Macros (hyphenation)}

%D This filter is kind of obsolete, since \UTF\ is not
%D limited to \XML.  So, here we only enable \UTF\ support.

\defineXMLenvironment [hyphenations] [language=\currentlanguage,regime=utf,encoding=\defaultencoding]
  {\startnointerference
   \defineXMLargument [hyphenation] \hyphenation
   \language[\XMLop{language}]%
   \enableregime[\XMLop{regime}]%
   \enableencoding[\XMLop{encoding}]}
  {\stopnointerference}

\endinput

% \mainlanguage[nl] \setupbodyfont[pos] \useXMLfilter[utf,hyp]
%
% \starttext
%
% \hyphenatedword{pati\ediaeresis nten}
% \hyphenatedword{pati\ediaeresis ntenorganisatie}
% \hyphenatedword{pati\ediaeresis ntenplatform}
%
% \startXMLdata
%   <hyphenations language='nl' regime='utf'>
%     <hyphenation>pa-tiën-ten</hyphenation>
%     <hyphenation>pa-tiën-ten-or-ga-ni-sa-tie</hyphenation>
%     <hyphenation>pa-tiën-ten-plat-form</hyphenation>
%   </hyphenations>
% \stopXMLdata
%
% \hyphenatedword{pati\ediaeresis nten}
% \hyphenatedword{pati\ediaeresis ntenorganisatie}
% \hyphenatedword{pati\ediaeresis ntenplatform}
%
% \stoptext