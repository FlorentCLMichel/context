% interface=en

%M \usemodule [contml] \autoXMLnamespace [context]
%M \definefilesynonym [context] [x-contml.xsd]

%D \module
%D   [       file=x-contml,
%D        version=mid 2001,
%D          title=\CONTEXT\ XML Support,
%D       subtitle=Basic \CONTEXT\ commands,
%D         author=Hans Hagen,
%D           date=\currentdate,
%D      copyright={PRAGMA / Hans Hagen \& Ton Otten}]
%C
%C This module is part of the \CONTEXT\ macro||package and is
%C therefore copyrighted by \PRAGMA. See mreadme.pdf for
%C details.

% This module provides some basic \XML\ elements. These definitions
% are highly experimental and serve as a playground for interface
% development.

\unprotect

%D \elements {include}
%D
%D \startbuffer
%D You can include another file in the current document with the
%D <element>include</element> element. When including the file, &context;
%D will look at the suffix, in order to decide how to include the file.
%D
%D <verbatim>
%D <line><include name="mine"/></line>
%D <line><include name="mine" type="txt"/></line>
%D <line><include name="mine" type="tex"/></line>
%D </verbatim>
%D \stopbuffer
%D
%D \showelements [context] [include]

\defineXMLsingular [context:include] [name=,type=xml]
  {\doifsomethingXMLop{name}
     {\processaction
         [\XMLop{type}]
         [xml=>\readfile{\XMLop{name}}\donothing\donothing,
          tex=>{{\disableXML\readfile{\XMLop{name}}\donothing\donothing}},
          txt=>{{\disableXML\typefile{\XMLop{name}}}}]}}

% or, nicer:
%
% \defineXMLsingular [context:include] [name=unknown,type=txt]
%   {\XMLval{include:type}{\XMLop{type}}{}}
%
% \mapXMLvalue {include:type} {xml}             {\readfile{\XMLop{name}}\donothing\donothing}
% \mapXMLvalue {include:type} {tex} {{\disableXML\readfile{\XMLop{name}}\donothing\donothing}}
% \mapXMLvalue {include:type} {txt} {{\disableXML\typefile{\XMLop{name}}}}

%D \elements {compound}
%D
%D \startbuffer
%D Instead of using hard coded compound tokens, you should use the
%D <element>compound</element> element, as in high<compound token="/" />low.
%D The overhead in keying is rewarded with proper symbols and hyphenation.
%D \stopbuffer
%D
%D \showelements [context] [compound]

\ifx\normalcompound\undefined \let\normalcompound=| \fi

% \defineXMLsingular [context:compound] [token=]
%   {\expanded{\normalcompound\XMLop{token}}|}

% \defineXMLsingular [context:compound] [token=]
%   {\ifmmode
%      \doifXMLop{token}{\XMLop{token}}{\compoundhyphen}%
%    \else
%      \expanded{\normalcompound\XMLop{token}}|%
%    \fi}

\defineXMLsingular [context:compound] [token=]
  {\mathortext % disc comm looks ahead, so \relax
     {\doifXMLop{token}{\XMLop{token}}\compoundhyphen}%
     {\expanded{\directdiscretionary{\XMLop{token}}}\relax}}

\defineXMLenvironmentsave [context:cp]
  {}
  {\expanded{\directdiscretionary{\XMLflush{cp}}}\relax}

%D \elements {p}
%D
%D \startbuffer
%D <p>Although for &tex; it is often enough to mark the end of a paragraph,
%D in &xml; we want to add a bit more structure. <p/> This permits a more
%D robust implementation of begin<compound/>of<compound/>par actions</p>
%D \stopbuffer
%D
%D \showelements [context] [p]

\defineXMLenvironment [context:p] {} \endgraf
\defineXMLsingular    [context:p]    \endgraf

%D \elements {pageref,textref,lineref}
%D
%D \startbuffer
%D You can ask for a page (<pageref label="lastpage">the last pagenumber
%D is</pageref> aka page <pageref label="lastpage"/>), text or line reference
%D with the following three elements. The label may be any valid &context;
%D reference label.
%D \stopbuffer
%D
%D \showelements [context] [references]

\defineXMLpickup [context:pageref] [label=] {\at}     {[\XMLop{label}]}
\defineXMLpickup [context:textref] [label=] {\in}     {[\XMLop{label}]}
\defineXMLpickup [context:lineref] [label=] {\inline} {[\XMLop{label}]}

%D \elements{text}
%D
%D \startbuffer
%D If you have a self contained &xml; file, you need to signal &context; the
%D begin and end of the document. The following elements can be used for
%D that purpose:
%D
%D <verbatim>
%D <line><text></line>
%D <line>  ...</line>
%D <line></text></line>
%D </verbatim>
%D \stopbuffer
%D
%D \showelements [context] [text]

\defineXMLenvironment [context:text] \starttext \stoptext

%D \elements {em}
%D
%D \startbuffer
%D Authors often want some control over the way a text is typeset, which is
%D why we provide the <element>em</element> element. We may only hope that
%D the author is <em>consistent</em> in his decisions on what to emphasize.
%D \stopbuffer
%D
%D \showelements [context] [em]

\defineXMLgrouped [context:em] \em

%D \elements {b}
%D
%D \startbuffer
%D Bold is not always <b>beautiful</b> but if you really want it, you can
%D get it by using this element.
%D \stopbuffer
%D
%D \showelements [context] [b]

\defineXMLgrouped [context:b] \bf

%D \elements {verbatim,typing,line,verb,type}
%D
%D \startbuffer
%D Although the following method can be used to typeset a piece of code
%D verbatim
%D
%D <![CDATA[
%D Dit \is nogal verbatim !
%D Dit is {nogal} verbatim !
%D Dit is <nogal> verbatim !
%D ]]>
%D
%D we prefer the more structured:
%D
%D <verbatim>
%D <line>Dit \is nogal verbatim !</line>
%D <line>Dit is {nogal} verbatim !</line>
%D <line>Dit is <nogal> verbatim !</line>
%D </verbatim>
%D \stopbuffer
%D
%D The element to tag in<compound/>line verbatim is <type><verb></type>.
%D
%D \showelements [context] [verbatim]

\defineXMLenvironment [context:verbatim]
  {\startpacked
   \defineXMLargument[context:line]{\endgraf\type}}
  {\stoppacked}

\defineXMLenvironment [context:typing]
  {\startpacked\defineXMLargument[context:line]{\endgraf\type}}
  {\stoppacked}

\defineXMLargument [context:verb] \type
\defineXMLargument [context:type] \type

%D \elements {itemize,item}
%D
%D \startbuffer
%D Itemized lists are quite common in documents, al least in the ones that
%D we produce. For the moment we only provide a few options, later we will
%D hook it into the &context; attribute handler.
%D
%D <itemize type="a">
%D   <item label="bla"> test </item>
%D   <item> test </item>
%D </itemize>
%D
%D <itemize packed="yes">
%D   <item label="more bla"> test </item>
%D   <item> test <em>what?</em></item>
%D </itemize>
%D \stopbuffer
%D
%D \showelements [context] [itemize]

\defineXMLenvironment [context:itemize] [type=,packed=]
  {\let\XMLoptions\empty
   \doifsomethingXMLop{type}{\addtocommalist{\XMLop{type}}\XMLoptions}%
   \doifXMLop{packed}{yes}{\addtocommalist{packed}\XMLoptions}%
   \expanded{\startitemize[\XMLoptions]}}
  {\stopitemize}

\defineXMLenvironment [context:item] [label=]
  {\expanded{\item[\XMLop{label}]}}
  {\endgraf}

%D \elements {externalfigure}
%D
%D \startbuffer
%D The previous examples already demonstrated how we can include a graphic:
%D
%D <verbatim>
%D <line><externalfigure file="cow" width="5cm" /></line>
%D </verbatim>
%D \stopbuffer
%D
%D \showelements [context] [externalfigure]

\defineXMLsingular [context:externalfigure] [\??ef] [base=,label=,file=]
  {\bgroup % \getXMLta \expandXMLta \expandXMLtp{file}%
   \expandXMLta
   \getXMLta % expand entities first
   \doifelsenothing{\XMLtp{label}}
     {\expanded{\externalfigure[\XMLtp{file}][\XMLta]}}
     {\doifsomething{\XMLtp{base}}{\usefigurebase[\XMLtp{base}]}%
      \expanded{\externalfigure[\XMLtp{label}][\XMLta]}}
   \egroup}

%D \elements {fixed}
%D \setupexternalfigures[directory={../sample}]
%D \startbuffer
%D Something fixed will end up at the place where it defined in the input
%D stream. The main idea behind this element is that it gives you control
%D over the placement.
%D
%D <itemize>
%D   <item>
%D     <fixed align="high">
%D       <content>
%D         <externalfigure file="cow" frame="on" height="1cm" />
%D       </content>
%D     </fixed>
%D   </item>
%D </itemize>
%D \stopbuffer
%D
%D \showelements [context] [fixed]

\defineXMLenvironment [context:fixed] [type=figure,location=,label=]
  {\bgroup
   \defineXMLsave[context:caption]
   \defineXMLsave[context:content]}
  {\expanded{\startfixed[\XMLop{location}]}%
     \doifXMLdataelse{context:caption}
       {\startcombination[1*1]
          {\XMLflush{context:content}} {\XMLflush{context:caption}}
        \stopcombination}
       {\XMLflush{context:content}}%
   \stopfixed
   \egroup}

%D \elements {float}
%D \setupexternalfigures[directory={../sample}]
%D \startbuffer
%D A floating body will be placed at the first location available, unless
%D a location is specified. As with the <element>fixed</element> element,
%D you can provide a caption.
%D
%D <float type="figure">
%D   <content>
%D     <externalfigure file="cow" frame="on" height="3cm" />
%D   </content>
%D   <caption>This is a cow!</caption>
%D </float>
%D \stopbuffer
%D
%D \showelements [context] [float]

\defineXMLenvironment [context:float] [type=figure,location=here,label=]
  {\bgroup
   \defineXMLsave[context:caption]
   \defineXMLsave[context:content]}
  {\expanded
     {\placefloat
        [\XMLop{type}] [\XMLop{location}] [\XMLop{label}]
        {\XMLflush{context:caption}} {\XMLflush{context:content}}}
   \egroup}

%D \elements {quotation,quote}
%D
%D \startbuffer
%D There is a (not so) subtle difference between a display
%D <quotation>quotation</quotation> and an <quote>in<compound/>line</quote>
%D one.
%D \stopbuffer
%D
%D \showelements [context] [table]

\defineXMLgrouped [context:quote]     \quote
\defineXMLgrouped [context:quotation] \quotation

%D \elements {table,tr,td}
%D
%D \startbuffer
%D There are (currently) three table mechanisms in &context;. One of them
%D resembles the well known &html; tables.
%D
%D <?context-command \startlinecorrection[blank] ?>
%D <table>
%D <tr> <td>one</td> <td>a</td> <td>first </td> </tr>
%D <tr> <td>two</td> <td>b</td> <td>second</td> </tr>
%D </table>
%D <?context-command \stoplinecorrection ?>
%D
%D As you can see here, we use a similar syntax but stick to the &context;
%D attributes (which provide quite advanced control over the layout).
%D
%D <?context-command \startlinecorrection[blank] ?>
%D <table frame="off" background="color" color="white">
%D <tr backgroundcolor="red">   <td>xx</td> <td>xx</td> </tr>
%D <tr backgroundcolor="green"> <td>xx</td> <td>xx</td> </tr>
%D </table>
%D <?context-command \stoplinecorrection ?>
%D \stopbuffer
%D
%D \showelements [context] [table]

\defineXMLenvironment [context:table] [\@@tbl\@@tbl]
  {\bgroup
   \defineXMLnested [context:tr] [\@@tbl] {\expanded{\bTR[\theXMLarguments{\@@tbl}}]} \eTR
   \defineXMLnested [context:td] [\@@tbl] {\expanded{\bTD[\theXMLarguments{\@@tbl}}]} \eTD
   \expanded{\bTABLE[\theXMLarguments{\@@tbl\@@tbl}]}}
  {\eTABLE
   \egroup}

%D \elements {tabulate,tspec,thead,tbody,ttail,trule,tr,td}
%D
%D \startbuffer
%D The second mechanism that we support is tabulation. The advantage of this
%D mechanism is that it it well tuned for tables that have much text in the
%D cells and cross page boundaires.
%D
%D <tabulate>
%D   <tspec>
%D     <tcell align="left"/> <tcell align="middle"/> <tcell align="right"/>
%D   </tspec>
%D   <thead>
%D     <trule/>
%D     <tr> <td> bagger </td> <td> bagger </td> <td> bagger </td> </tr>
%D     <trule/>
%D   </thead>
%D   <ttail>
%D     <trule/>
%D   </ttail>
%D   <tbody>
%D     <tr> <td> bagger </td> <td> bagger </td> <td> bagger </td> </tr>
%D     <tr> <td> bagg   </td> <td> ger    </td> <td> gr     </td> </tr>
%D     <tr> <td> bag    </td> <td> er     </td> <td> gger   </td> </tr>
%D   </tbody>
%D </tabulate>
%D \stopbuffer
%D
%D \showelements [context] [tabulate]

\newtoks\XMLtabtoks

\defineXMLgrouped [context:tabulate] {\XMLtabtoks{|l|p|}}

\defineXMLpickup [context:tbody]
  {\expanded{\definetabulate[dummy][\the\XMLtabtoks]}
   \startdummy\XMLflush{context:thead}}
  {\XMLflush{context:ttail}\stopdummy}

\defineXMLsave [context:thead]
\defineXMLsave [context:ttail]

\defineXMLenvironment[context:tspec]
  {\XMLtabtoks\emptytoks}
  {\appendtoks|\to\XMLtabtoks}

\defineXMLsingular [context:trule] % verrrry ugly
  {\crcr\noalign{\kern-\lineheight}\HL}

\defineXMLsingular [context:tcell] [align=]
  {\appendtoks|\to\XMLtabtoks
   \expanded{\processallactionsinset
     [\XMLop{align}]}
     [  paragraph=>\appendtoks p\to\XMLtabtoks,
             left=>\appendtoks l\to\XMLtabtoks,
            right=>\appendtoks r\to\XMLtabtoks,
           center=>\appendtoks c\to\XMLtabtoks,
           middle=>\appendtoks c\to\XMLtabtoks]}

\defineXMLenvironment [context:tr] {\ignorespaces} {\NC\NR}
\defineXMLenvironment [context:td] {\NC}           {\ignorespaces}

%D \elements {hide}
%D
%D \startbuffer
%D This is the way to [<hide>this is gone</hide>] something for the
%D typesetting engine. Normally this element is only used for testing
%D purposes.
%D \stopbuffer
%D
%D \showelements [context] [tabulate]

\defineXMLignore[context:hide]

%D \elements {unknown}
%D
%D \startbuffer
%D We can go on and on and <unknown/> with defining elements that map onto
%D &context; commands, but why not just use &tex; input syntax then?
%D \stopbuffer
%D
%D \showelements [context] [unknown]

\defineXMLsingular [context:unknown] \unknown

%D A (for the moment) private one.

\defineXMLargument [context:element] \type

%D The following common schema definitions apply:
%D
%D {\setupcolors[state=stop]\showXSDcomponent[context][definitions]}

\defineXMLargument [context:chapter] [label=] {\chapter[\XMLop{label}]}
\defineXMLargument [context:section] [label=] {\section[\XMLop{label}]}
\defineXMLargument [context:subsection] [label=] {\subsection[\XMLop{label}]}
\defineXMLargument [context:subsubsection] [label=] {\subsubsection[\XMLop{label}]}
\defineXMLargument [context:subsubsubsection] [label=] {\subsubsubsection[\XMLop{label}]}

\defineXMLargument [context:title] [label=] {\title[\XMLop{label}]}
\defineXMLargument [context:subject] [label=] {\subject[\XMLop{label}]}
\defineXMLargument [context:subsubject] [label=] {\subsubject[\XMLop{label}]}
\defineXMLargument [context:subsubsubject] [label=] {\subsubsubject[\XMLop{label}]}
\defineXMLargument [context:subsubsubsubject] [label=] {\subsubsubsubject[\XMLop{label}]}

\defineXMLenvironment [context:frontmatter] \startfrontmatter \stopfrontmatter
\defineXMLenvironment [context:bodymatter]  \startbodymatter  \stopbodymatter
\defineXMLenvironment [context:backmatter]  \startbackmatter  \stopbackmatter
\defineXMLenvironment [context:appendices]  \startappendices  \stopappendices

\defineXMLargument [context:index] [key=]
  {\doifelsenothingXMLop{key}{\index}{\expanded{\index[\XMLop{key}]}}}

% \enableXMLfiledata

% Needed for example (stickers and so):

\defineXMLenvironment [context:makeup]
  \startstandardmakeup \stopstandardmakeup

\protect \endinput

% TO DO

\defineXMLenvironment [combination] [columns=2,rows=1]
  {\scratchtoks\emptytoks
   \expanded{\appendtoks \noexpand \startcombination
     [\XMLop{columns}*\XMLop{rows}]}\to \scratchtoks}
  {\appendtoks \stopcombination \to \scratchtoks
   \the\scratchtoks}

\defineXMLprocess[combinationentry]

\defineXMLpickup [combinationitem]
  {\appendtoks\bgroup}{\egroup\to\scratchtoks}

\defineXMLpickup [combinationcaption]
  {\appendtoks\bgroup}{\egroup\to\scratchtoks}

\endinput
