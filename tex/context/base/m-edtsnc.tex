%D \module
%D   [       file=m-editsnc,
%D        version=2003.12.23,
%D          title=\CONTEXT\ Modules,
%D       subtitle=Editor Synchronization
%D         author={Hans Hagen \& Ton Otten},
%D           date=\currentdate,
%D      copyright={PRAGMA / Hans Hagen \& Ton Otten}]
%C
%C This module is part of the \CONTEXT\ macro||package and is
%C therefore copyrighted by \PRAGMA. See mreadme.edit for
%C details.

% \enablemode[demo]

% This file is a variation of the file m-pdfsync. This time we do it
% in a more abstract way and prepare ourselves for support in dvi
% output.

% \enableeditsync
% \disableeditsync
% \editsync
% \editsynctext      {string}
% \editsyncnode      {char}
% \editsynctracemode [chardef: 0-4]

\writestatus{editsync}{loading and enabling editor synchronization support}

\unprotect % not really needed

\ifx\pdfoutput\undefined
  \donefalse
\else\ifcase\pdfoutput
  \donefalse
\else\ifx\pdfsavepos\undefined
  \donefalse
\else
  \donetrue
\fi\fi\fi

\chardef  \editsyncstate     \zerocount
\chardef  \editsynctracemode \zerocount
\newcount \editsynccounter
\newwrite \editsyncwrite

\def\enableeditsync {\chardef\editsyncstate\plusone}
\def\disableeditsync{\chardef\editsyncstate\zerocount}

% The following definitions can go into the special drivers, but
% for the moment we do it this way.

\ifdone

  \def\dostarteditsync
    {\immediate\openout\editsyncwrite\jobname.pdfsync
     \immediate\write\editsyncwrite{\jobname}%
     \immediate\write\editsyncwrite{version 0}}

  \def\dostopeditsync
    {\closeout\editsyncwrite}

  \def\doeditsyncregisterpage#1{\immediate\write\editsyncwrite{s\space\number#1}}
  \def\doeditsyncopenfile    #1{\immediate\write\editsyncwrite{(\space#1}}
  \def\doeditsyncclosefile   #1{\immediate\write\editsyncwrite{)}}

  \def\doregistereditsyncnode#1#2#3% tag counter linenumber
    {\pdfsavepos
     \immediate\write\editsyncwrite{l\space\number#2\space\number#3}%
     \expanded{\write\editsyncwrite{p\ifnum#1=\plusone*\fi\space\number#2\space\noexpand\the\pdflastxpos\space\noexpand\the\pdflastypos}}}

\else

  \let\dostarteditsync       \relax
  \let\dostopeditsync        \relax

  \let\doeditsyncregisterpage\gobbleoneargument
  \let\doeditsyncopenfile    \gobbleoneargument
  \let\doeditsyncclosefile   \gobbleoneargument
  \let\doregistereditsyncnode\gobblethreearguments

  % These specials could be filtered by a dvitopdf backend and
  % written to a jobname.pdfsync file. In that case, the x and
  % y coordinates have to be to be resolved by driver.

  % \def\doeditsyncregisterpage#1{\special{pdfsync: s\space\number#1}}
  % \def\doeditsyncopenfile    #1{\special{pdfsync: ( #1}}
  % \def\doeditsyncclosefile   #1{\special{pdfsync: )}}

  % \def\doregistereditsyncnode#1#2#3%
  %   {\special{pdfsync: l the\editsynccounter\space\number#3}%
  %    \special{pdfsync: p\ifnum#1=\plusone*\fi\space\number#2}}

\fi

\let\editsynctracer\gobbleoneargument

\editsynccounter\minusone % counting starts at zero and we increment beforehand

\def\registereditsyncnode#1% we assume #1 is one token
  {\ifcase\editsyncstate
     \expandafter\gobbleoneargument
   \else
     \global\advance\editsynccounter\plusone
     \doregistereditsyncnode#1\editsynccounter\inputlineno
     \expandafter\editsynctracer
   \fi}

\def\editsync    {\registereditsyncnode\plusone *}
\def\editsyncnode{\registereditsyncnode\zerocount}
\def\editsynctext{\registereditsyncnode\plusone  }

\appendtoks \dostarteditsync                  \to \everystarttext
%appendtoks \dostopeditsync                   \to \everystoptext
\appendtoks \doeditsyncregisterpage\realfolio \to \everyshipout
\appendtoks \doeditsyncopenfile\readfilename  \to \everybeforereadfile
\appendtoks \doeditsyncclosefile\readfilename \to \everyafterreadfile

\appendtoks \enableeditsync         \to \everystarttext
\appendtoks \disableeditsync        \to \everypagebody

\def\editsynctracer#1%
  {\ifcase\editsynctracemode\else
     \begingroup
     \forgetall
     \disableeditsync
     \ifcase\editsynctracemode\or % could be done more efficient, box around ifcase
       \setbox\scratchbox\hbox to \zeropoint{\hss\infofont#1\hss}%   1
     \else
       \setbox\scratchbox\hbox to \zeropoint{\hss\traceboxplacementtrue\boxcursor\hss}% 2/3/4
     \fi
     \smashbox\scratchbox\box\scratchbox
     \ifcase\editsynctracemode\or\or\or
       \setbox\scratchbox\hbox to \zeropoint{\hss\raise1.25ex\hbox{\infofont#1}\hss}%  3
       \smashbox\scratchbox\box\scratchbox
     \or
       \setbox\scratchbox\hbox to \zeropoint{\hss\lower1.25ex\hbox{\infofont#1}\hss}%  4
       \smashbox\scratchbox\box\scratchbox
     \fi
     \endgroup
   \fi}

% \def\editsynctracer#1% more efficient but unreadable
%   {\ifcase\editsynctracemode\else
%      \ifcase\editsyncstate\else
%        \begingroup
%        \forgetall
%        \disableeditsync
%        \setbox\scratchbox\hbox to \zeropoint
%           {\hss
%            \ifcase\editsynctracemode\or
%              \infofont#1%   1
%            \else
%              \traceboxplacementtrue\boxcursor% 2/3/4
%            \fi
%            \hss}%
%        \smashbox\scratchbox\box\scratchbox
%        \setbox\scratchbox\hbox to \zeropoint
%          {\hss
%           \ifcase\editsynctracemode\or\or\or
%             \raise1.25ex\hbox{\infofont#1}%  3
%           \or
%             \lower1.25ex\hbox{\infofont#1}%  4
%           \fi
%           \hss}%
%        \smashbox\scratchbox\box\scratchbox
%        \endgroup
%      \fi
%    \fi}

% beware, adding nodes this way will interfere with the typesetting

\appendtoks \editsyncnode p\to \everypar
\appendtoks \editsyncnode m\to \everymath
\appendtoks \editsyncnode h\to \everyhbox

% For compatibility with 'pdfsync4context' we provide:

\let\pdfsyncstart\enableeditsync
\let\pdfsyncstop \disablepdfsync
\let\pdfsync     \editsync

\protect

\doifnotmode{demo}{\endinput}

\chardef\editsynctracemode=3

\starttext

\chapter{Test}

\processfile{tufte}

\startitemize
\item first
\item second
\stopitemize

\processfile{tufte}

\startlines
some local \editsync sync and \editsynctext{here}a marked point
some local \editsync sync and \editsyncnode{here}a marked point
\stoplines

\stoptext