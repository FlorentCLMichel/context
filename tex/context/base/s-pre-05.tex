%D \module
%D   [      file=s-pre-05,
%D        version=1998.12.12,
%D          title=\CONTEXT\ Style File,
%D       subtitle=Presentation Environment 5,
%D         author=Hans Hagen,
%D           date=\currentdate,
%D      copyright={PRAGMA / Hans Hagen \& Ton Otten}]
%C
%C This module is part of the \CONTEXT\ macro||package and is
%C therefore copyrighted by \PRAGMA. See mreadme.pdf for 
%C details. 

\usemodule[pre-general] 

%D As all styles sofar, this one has the same structuring 
%D commands. 

\startmode[asintended] \setupbodyfont[lbr] \stopmode

\setupbodyfont[14.4pt]

\setupcolors [state=start]

\definecolor [BackgroundColor] [s=.95] 
\definecolor [OrnamentColor]   [r=.6,g=.7,b=.8]

\setuppapersize
  [S6][S6]

\setuplayout
  [width=430pt,
   height=400pt,
   header=0pt,
   footer=0pt,
   margin=0pt,
   backspace=25pt,
   topspace=25pt,
   rightedgedistance=20pt,
   rightedge=110pt]

\setupinteractionscreen
  [option=max]

\setupbackgrounds
  [state=repeat]

\setupbackgrounds
  [page]
  [backgroundcolor=white]

\setupbackgrounds
  [text][text]
  [background={HashFrameA,NextPage},
   backgroundoffset=20pt]

\defineoverlay
  [HashFrameA]
  [\useMPgraphic{HashFrameA}]

\defineoverlay
  [HashFrameB]
  [\useMPgraphic{HashFrameB}]

\setupinteraction
  [state=start,
   menu=on,
   color=OrnamentColor,
   contrastcolor=OrnamentColor]

%D Watch how we use a list alternative that matches the 
%D menu.

\setupinteractionmenu
  [right]
  [background=HashFrameB,
   style=smallbold,
   frame=off,
   offset=10pt,
   height=35pt, 
   before=,
   after=,
   inbetween=\endgraf,
   width=\rightedgewidth]

\startinteractionmenu[right]
  \placelist
    [Topic]
    [criterium=all,
     alternative=right,
     maxwidth=.8\rightedgewidth,
     interaction=all,
     before=,
     after=]
  \vfill
  \setupinteractionmenu
    [right]
    [height=30pt]
  \but [CloseDocument] Close \\ 
\stopinteractionmenu

\setupwhitespace
  [big]

\setupblank
  [big]

%D \macros
%D   {TitlePage}
%D
%D Now the main layout and navigational definitions are
%D done, it makes sense to define and tune some structuring
%D commands. First we build the titlepage.

\defineoverlay [TitleGraphic] [\useMPgraphic{TitleGraphic}]
\defineoverlay [NextPage]     [\overlaybutton{forward}]

\def\StartTitlePage%
  {\setupbackgrounds[page][background={color,TitleGraphic,NextPage}]
   \setupbackgrounds[text][text][background=]
   \setupinteraction[menu=off]
   \setupinteractionbar[state=stop]
   \setuplayout[width=550pt,rightedge=0pt]
   \startstandardmakeup
   \switchtobodyfont[24pt]
   \bfd\setupinterlinespace
   \setupalign[middle]
   \vfil
   \let\\=\vfil}

\def\StopTitlePage%
  {\vfil\vfil\vfil
   \stopstandardmakeup
   \setuplayout[width=430pt,rightedge=110pt]
   \setupinteraction[menu=on]
   \setupinteractionbar[state=start]
   \setupbackgrounds[page][background=color]
   \setupbackgrounds[text][text][background={HashFrameA,NextPage}]}

\def\TitlePage#1%
  {\StartTitlePage#1\StopTitlePage}

%D \macros 
%D   {Topics,Subjects}
%D
%D Since the lists are in the menu, we don't honor list 
%D placement macros. 

\def\Topics#1{}
\def\Subjects{}

%D \macros
%D   {Topic, Nopic, Subject} 
%D
%D Since t his style is meant for rather flat structured 
%D documents, only \type {\Topic} makes sense.

\definehead [Topic]   [chapter]
\definehead [Nopic]   [title]
\definehead [Subject] [section]

\setuphead
  [Topic, Nopic]
  [after={\blank[3*medium]},
   number=no,
   style=\tfb,
   page=yes,
   alternative=middle]

\setuphead
  [Subject]
  [after=\blank,
   number=no,
   page=yes,
   continue=no,
   style=\tfa]

%D We use only one kind of base graphic, which is sligthly 
%D tuned for the different usage. 

\startMPinclusions
  def random_hash_frame (expr width, height, offset, linewidth ) =  

    def delta = ((uniformdeviate .5offset) + .25offset) enddef ;
    x1 := offset ; y1 := offset ; x2 := width-offset ; y2 := height-offset ;

    drawoptions(withpen pencircle scaled linewidth withcolor \MPcolor{BackgroundColor}) ; 
    fill z1--(x2,y1)--z2--(x1,y2)--cycle ;

    drawoptions(withpen pencircle scaled linewidth withcolor \MPcolor{OrnamentColor}) ; 
    draw (x1-delta,y1)--(x2+delta,y1) ;
    draw (x2,y1-delta)--(x2,y2+delta) ;
    draw (x2+delta,y2)--(x1-delta,y2) ;
    draw (x1,y2+delta)--(x1,y1-delta) ;

    drawoptions(); 
    setbounds currentpicture to unitsquare xscaled width yscaled height ;
  enddef ; 
\stopMPinclusions

\startuseMPgraphic{HashFrameA}
  random_hash_frame(OverlayWidth,OverlayHeight,15pt,2pt) ;
\stopuseMPgraphic

\startuseMPgraphic{HashFrameB}
  random_hash_frame(OverlayWidth,OverlayHeight, 5pt,2pt) ;
\stopuseMPgraphic

\startuseMPgraphic{TitleGraphic}
  for i=1 upto 300 : 
    offset := uniformdeviate 10pt ; 
    width  := 2*offset + 30pt + uniformdeviate 30pt ; 
    height := 2*offset + 10pt + uniformdeviate 10pt ; 
    addto currentpicture also 
      image(random_hash_frame(width,height,offset,1pt)) shifted  
      (uniformdeviate OverlayWidth, uniformdeviate OverlayHeight) ; 
  endfor ;  
\stopuseMPgraphic

\endinput
