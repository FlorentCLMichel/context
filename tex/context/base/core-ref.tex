%D \module
%D   [       file=core-ref,
%D        version=1998.01.15,
%D          title=\CONTEXT\ Core Macros,
%D       subtitle=Cross Referencing,
%D         author=Hans Hagen,
%D           date=\currentdate,
%D      copyright={PRAGMA / Hans Hagen \& Ton Otten}]
%C
%C This module is part of the \CONTEXT\ macro||package and is
%C therefore copyrighted by \PRAGMA. See mreadme.pdf for
%C details.

\writestatus{loading}{Context Core Macros / Cross Referencing}

% todo : unknown/illegal reference no arg
% todo : +n pages check on 'samepage' (contrastcolor)

% Makes more sense to build action data first, especially now
% openaction etc are supported.
%
% \definespecial\doexecuteactionchain w h
% \definespecial\dosetgotolocation
% \definespecial\dosetexecuteJScode
% ...
%
% complication: what when direct? Two calls!

% I considered to change / simplify
%
% rt!page -> \definereference
% rt!list -> \definereference
% rt!exec -> \definereference
%
% but for the moment will not do so, if only because
% the current implementation permits us to determine
% the page state and is also more efficient

% the code is rather fuzzy (and will be redone some day); this is
% due to the chaining (collect secondary and then hook that into
% the primary etc

\unprotect

\startmessages  dutch  library: references
  title: verwijzingen
      1: onbekende verwijzing --
      2: dubbele verwijzing -- op pagina --
      3: type verwijzing -- onbekend
      4: verboden verwijzing --
     21: document -- geladen
     22: document -- is niet interactief
     23: onduidelijke verwijzing -- (prefix=--)
\stopmessages

\startmessages  english  library: references
  title: references
      1: unknown reference --
      2: duplicate reference -- on page --
      3: unknown reference type --
      4: illegal reference --
     21: document -- loaded
     22: document -- is not interactive
     23: obscure reference -- (prefix=--)
\stopmessages

\startmessages  german  library: references
  title: referenzen
      1: unbekannte Referenz --
      2: doppelte Referenz -- auf Seite --
      3: unbekannte Referenz Typ --
      4: illegale Referenz --
     21: Dokument -- geladen
     22: Dokument -- ist nicht aktiv
     23: Obskure Referenz -- (Prefix=--)
\stopmessages

\startmessages  czech  library: references
  title: reference
      1: neznama reference --
      2: duplicitni reference -- na strane --
      3: neznamy typ reference --
      4: nedovolena reference --
     21: dokument -- nacten
     22: dokument -- neni interaktivni
     23: obskurni (nejasna) reference -- (prefix=--)
\stopmessages

\startmessages  italian  library: references
  title: riferimenti
      1: riferimento sconosciuto --
      2: riferimento duplicato -- a pagina --
      3: riferimento di tipo sconosciuto --
      4: riferimento illecito --
     21: documento -- caricato
     22: il documento -- non � interattivo
     23: riferimento ambiguo -- (prefisso=--)
\stopmessages

\startmessages  norwegian  library: references
  title: referanser
      1: ukjent referanse --
      2: duplikat referanse -- p� side --
      3: ukjent referansetype --
      4: ulovlig referanse --
     21: dokument -- er lest inn
     22: dokument -- er ikke interaktivt
     23: obskur referanse -- (Prefix=--)
\stopmessages

\startmessages  romanian  library: references
  title: referinte
      1: referinta necunoscuta --
      2: referinta duplicat -- la pagina --
      3: tip necunoscut de referinta --
      4: referinta eronata --
     21: documentul -- este incarcat
     22: documentul -- nu este interactiv
     23: referinta obscura -- (prefix=--)
\stopmessages

\startmessages  french  library: references
  title: references
      1: r�ference -- inconnue
      2: r�ference -- dupliqu�e � la page --
      3: type -- de r�ference inconnu
      4: r�ference -- inconnue
     21: document -- charg�
     22: le document -- n'est pas interactif
     23: reference -- ind�termin� (pr�fixe=--)
\stopmessages

%D This module deals with referencing. In \CONTEXT\ referencing
%D is one of the core features, although at a first glance
%D probably nobody will notice. This is good, because
%D referencing should be as hidden as possible.
%D
%D In paper documents, referencing comes down to cross
%D referencing, but in their interactive counterparts, is also
%D involves navigation. Many features implemented here are
%D therefore closely related to navigation.
%D
%D Many \CONTEXT\ commands can optionally be fed with a
%D reference. Such a reference, when called upon, returns the
%D number of a figure, table, chapter etc, a piece of text, or
%D a pagenumber.
%D
%D There are three ways of defining a reference:
%D
%D \starttyping
%D \pagereference[here]
%D \textreference[here]{some text}
%D \stoptyping
%D
%D the third alternative combines them in:
%D
%D \starttyping
%D \reference[here]{some text}
%D \stoptyping

\def\textreference {\dosingleargument\dotextreference}
\def\pagereference {\dosingleargument\dopagereference}
\def\reference     {\dosingleargument\doreference    }

%D These are implemented in a low level form as:
%D
%D \starttyping
%D \def\dotextreference[#1]{\rawtextreference\s!txt{#1}} % #2
%D \def\dopagereference[#1]{\rawpagereference\s!pag{#1}}
%D \def\doreference    [#1]{\rawreference    \s!ref{#1}} % #2
%D \stoptyping
%D
%D or without expansion problems:

\def\dotextreference[#1]#2%
  {\bgroup
   \def\asciia{#1}%
   \convertexpanded\@@rfexpansion{#2}\asciib
   \@EA\rawtextreference\@EA\s!txt\@EA\asciia\@EA{\asciib}%
   \egroup}

\def\dopagereference[#1]%
  {\rawpagereference\s!pag{#1}}

\def\doreference[#1]#2%
  {\bgroup
   \def\asciia{#1}%
   \convertexpanded\@@rfexpansion{#2}\asciib
   \@EA\rawreference\@EA\s!ref\@EA\asciia\@EA{\asciib}%
   \egroup}

%D Actually there is not much difference between a text and a
%D full reference, but it's the concept that counts. The low
%D level implementation is:

\def\rawreference#1#2#3%
  {\bgroup
   \the\everyreference
   \makesectionformat
   \writereference{#2}
     {\sectionformat\sectionseparator\sectionseparator\noexpand\pagenumber}
     {\noexpand\realfolio}{#3}%
   \egroup}

\def\rawpagereference#1#2%
  {\bgroup
   \the\everyreference
   \makesectionformat
   \writereference{#2}
     {\sectionformat\sectionseparator\sectionseparator\noexpand\pagenumber}
     {\noexpand\realfolio}{}%
   \egroup}

\def\rawtextreference#1#2#3%
  {\bgroup
   \the\everyreference
   \writereference{#2}
     {}{\noexpand\realfolio}{#3}%
   \egroup}


%D As we can see, these macros depend on three other ones,
%D \type {\makesectionformat}, that generated \type
%D {\sectionformat}, \type {\pagenumber}. The not yet used
%D argument \type{#1} is a tag that specifies the type of
%D reference.

%D \macros
%D   {everyreference}
%D
%D For rather tricky purposes, one can assign sanitizing
%D macros to \type{\everyreference}.

\newevery \everyreference \relax

%D This is really needed, since for instance Polish has a
%D different alphabet and needs accented entries in registers.

\appendtoks
 %\def\dohandleaccent #1#2{\string#1\string#2}%
 %\def\dohandlecommand  #1{\string#1}%
  \cleanupfeatures
\to \everyreference

%D Why do we have to write down references? \TEX, and therefore
%D \CONTEXT\ is a batch processing system. During the
%D typesetting process, pages are shipped out, which means that
%D especially forward references are not yet known when the
%D page is typeset. That's why we always need a second (and
%D sometimes even a third) pass to get the cross references
%D right. The same goes for lists and other pagenumber
%D dependant data.
%D
%D Therefore, during a pass, \CONTEXT\ writes the references to
%D a the utility file. The next macro does the job and
%D generates entries like:
%D
%D \starttyping
%D \mainreference{prefix}{reference}{page}{realpage}{text}
%D \stoptyping
%D
%D We did not yet discuss prefixing. Especially in interactive
%D documents, it's not always easy to keep track of duplicate
%D references. The prefix mechanism, which we will describe
%D later on, solves this problem. By (automatically) adding a
%D prefix one keeps references local, but the global ones in
%D view. To enable this feature, we explictly split the prefix
%D from the reference.
%D
%D A former implementation used \type{\removesubstring} to get
%D rid of the don't||use||a||prefix signal (\type{-:}), but the
%D next one proved to be more than twice as fast.

\let\referenceprefix=\empty
\let\lastreference  =\empty

%D The last reference is saved in a macro named \type
%D {\lastreference} (indeed). To keep track of the order of
%D references, later we will see for what purpose, we maintain
%D a counter.

\newcount\crossreferencenumber \crossreferencenumber=1

\def\writereference#1#2#3#4%
  {\ifreferencing
     \edef\!!stringa{#1}%
     \ifx\!!stringa\empty \else
       \def\dowritereference##1%
         {\xdef\lastreference{##1}%
          \@EA\dodowritereference\lastreference\empty\empty\end{#2}{#3}{#4}}%
       \rawprocesscommalist[\!!stringa]\dowritereference
     \fi
   \fi}

%D Beware: \type {#2} gobbles space in references so that
%D \typ {a nice ref} becomes \typ {anice ref}.

\def\dodowritereference#1#2#3\end#4#5#6%
  {\bgroup
   \global\advance\crossreferencenumber \plusone\relax
   \if#1-\if#2:%
     \let\referenceprefix\empty
     \xdef\lastreference{#3}%
   \else
     % \xdef\lastreference{#1#2#3}% here we loose the space
   \fi\else
     % \xdef\lastreference{#1#2#3}% here we loose the space
   \fi
   \ifx\lastreference\empty \else
     \doiffirstreferenceoccurance\lastreference
       {\thisisdestination{\referenceprefix\lastreference}}%
     \referentieinfo>\lastreference
     \edef\dododowritereference
       {\writeutilitycommand
          {\mainreference{\referenceprefix}{\lastreference}{#4}{#5}{#6}}}%
     \dododowritereference
   \fi
   \egroup}

%D We will implement \type {\doiffirstreferenceoccurance}
%D later on.

%D When (during a second pass over the document) references are
%D loaded, they are saved in a macro, one for each reference.
%D In practice this comes to giving \type {\mainreference} a
%D appropriate meaning and loading the utility file.

% \def\setreferences
%   {\def\mainreference##1##2##3##4##5%
%      {\doifundefinedelse{\r!cross\fileprefix##1##2}
%         {\setglobalcrossreference{##1##2}{##3}{##4}{##5}}
%         {\showmessage\m!references{2}{[##1][##2],##4}}}}

% \def\setreferences
%   {\the\everyreference % we're grouped anyway
%    \def\mainreference##1##2##3##4##5%
%      {\doifundefinedelse{\r!cross\fileprefix##1##2}
%         {\setglobalcrossreference{##1##2}{##3}{##4}{##5}}
%         {\ifcase0##4\else\showmessage\m!references{2}{[##1][##2],##4}\fi}}}

%D For a long time the only way to access an external file was
%D to use the file prefix (\type {somefile::}. However, when
%D you split up a document, redefining the references may be
%D such a pain, that another approach is feasible. By setting
%D the \type {autofile} variable to \type {yes} or \type
%D {page}, you can access the reference directly. The latter
%D case nills the prefix method, thereby saving some memory.
%D
%D \starttabulate[||||]
%D \NC filename::tag \NC page(filename::pnum) \NC tag     \NC\NR
%D \NC   $\star$     \NC                      \NC         \NC\NR
%D \NC   $\star$     \NC $\star$              \NC $\star$ \NC\NR
%D \NC               \NC $\star$              \NC         \NC\NR
%D \stoptabulate

\chardef\autocrossfilereferences=0

% \def\setreferences % some day, filename will be stored in ref record
%   {\the\everyreference % we're grouped anyway
%    \def\mainreference##1##2##3##4##5%
%      {\doifundefinedelse{\r!cross\fileprefix##1##2}
%         {\ifcase\autocrossfilereferences
%            \setglobalcrossreference{##1##2}{##3}{##4}{##5}%
%          \or
%            \setglobalcrossreference{##1##2}{##3}{##4}{##5}%
%            \doifundefinedelse{\r!cross##1##2}
%              {\expanded{\definereference[##1##2][\fileprefix##1##2]}}
%              {\showmessage\m!references2{[##1][##2],##4 (auto \currentutilityfilename)}}%
%          \or
%            \doifundefinedelse{\r!cross##1##2}
%              {\expanded{\definereference[##1##2][\noexpand\v!pagina(\fileprefix##4)]}}
%              {\showmessage\m!references2{[##1][##2],##4 (auto \currentutilityfilename)}}%
%          \fi}
%         {\ifcase0##4\else
%            \showmessage\m!references2{[##1][##2],##4 (\currentutilityfilename)}%
%          \fi}}}

% optimizing this one makes sense when we have docs with 10K
% refs.

\def\setreferences% some day, filename will be stored in ref record
  {\the\everyreference % we're grouped anyway
   \def\mainreference##1##2##3##4##5% can be made faster by indirect calls
     {\ifundefined{\r!cross\fileprefix##1##2}% \ifcsname\r!cross\fileprefix##1##2\endcsname
        \ifcase\autocrossfilereferences
          \setglobalcrossreference{##1##2}{##3}{##4}{##5}%
        \or
          \setglobalcrossreference{##1##2}{##3}{##4}{##5}%
          \doifundefinedelse{\r!cross##1##2}
            {\expanded{\definereference[##1##2][\fileprefix##1##2]}}
            {\showmessage\m!references2{[##1][##2],##4 (auto \currentutilityfilename)}}%
        \or
          \doifundefinedelse{\r!cross##1##2}
            {\expanded{\definereference[##1##2][\noexpand\v!page(\fileprefix##4)]}}
            {\showmessage\m!references2{[##1][##2],##4 (auto \currentutilityfilename)}}%
        \fi
      \else\ifcase0##4\else
        \showmessage\m!references2{[##1][##2],##4 (\currentutilityfilename)}%
      \fi\fi}}

\def\resetreferences
  {\let\mainreference\gobblefivearguments}

\resetreferences

%D Here we see another kind of prefix surface: \type
%D {\fileprefix}. This prefix enables us to use references from
%D different files in one document. This is no really useful in
%D paper documents, but many interactive ones cannot do
%D without.

\let\fileprefix=\empty

%D Loading references is done using the normal utility file
%D handling macros. The \type{\hbox} trick prevents spaces
%D creeping in (references are set globally anyway).

\newtoks\everycheckreferences

%D When we load references, the file name is stored in a
%D list.

\let\loadedreferences\empty

%D We only load references ones.

\newconditional\jobreferencesloaded

%D This token list is expanded after the references are loaded.
%D This hook can be used to initialize mechanisms that depend
%D on the reference mechsnism. An example can be found in the
%D field module.

\def\checkreferences
  {\bgroup
   \let\fileprefix\empty
   \global\let\checkreferences\relax
   \usereferences[\jobname]%
   \checkrealpage
   \egroup
   \the\everycheckreferences}

\def\usereferences[#1]%
  {\startnointerference
   \checkreferences
   \doifparentfileelse{#1}
     {\ifconditional\jobreferencesloaded\else
        \doutilities{references}{#1}\empty\relax\relax
        \global\settrue\jobreferencesloaded
      \fi}
     {\ExpandBothAfter\doifnotinset{#1}{\loadedreferences}
        {\doutilities{references}{#1}\empty\relax\relax
         \ifx\fileprefix\empty\else
           \doglobal\addtocommalist{#1}\loadedreferences
         \fi}}
   \stopnointerference}

%D As mentioned we will also use the cross reference mechanism
%D for navigational purposes. The main reason for this is that
%D we want to treat both categories alike:
%D
%D \starttyping
%D \goto{go back}[PreviousJump]
%D \goto{colofon}[colofon page]
%D \stoptyping
%D
%D Here \type{PreviousJump} is handled by the viewer, while the
%D \type{colofon page} reference is, apart from hyperlinking, a
%D rather normal reference.
%D
%D We already saw that cross refences are written to and read
%D from a file. The pure navigational ones don't need to be
%D written to file, but both for fast processing and
%D transparant integration, they are saved internally as a sort
%D of reference. We can easily distinguish such system
%D references from real cross reference ones by their tag:

\chardef\rt!cross=0  % even means possible page reference
\chardef\rt!done =1
\chardef\rt!page =2  % and is used in \checkrealreferencepage
\chardef\rt!exec =3
\chardef\rt!list =4  % to determine the page state

%D We also use the odd/even characteristic to determine the
%D page state.

%D Here the \type{\rt!exec} tags a viewer specific navigational
%D reference, while for instance \type{\rt!page} gives fast
%D access to for instance the previous or next page. The
%D counter serves a purpose to be explained later. We use a
%D token register to prevent expansion of the text component,
%D which can contain all kind of \TEX\ commands.

\newcount\crossreferenceorder

% these are used often so we sped them up

\def\setlocalcrossreference#1#2#3#4%
  {\scratchtoks{#4}%
   \@EA\edef\csname\r!cross\fileprefix#1\endcsname
     {\rt!cross{#2}{#3}{\the\scratchtoks}{0}}}

\def\setglobalcrossreference#1#2#3#4%
  {\scratchtoks{#4}%
   \global\advance\crossreferenceorder 1
   \@EA\xdef\csname\r!cross\fileprefix#1\endcsname
     {\rt!cross{#2}{#3}{\the\scratchtoks}{\the\crossreferenceorder}}}

\def\setlocalsystemreference#1#2#3%
  {\@EA\edef\csname\r!cross\fileprefix#2\endcsname{#1{#3}}}

\def\setglobalsystemreference#1#2#3%
  {\@EA\xdef\csname\r!cross\fileprefix#2\endcsname{#1{#3}}}

% Maybe I need this some day.

% \def\copycrossreference#1#2#3% file from to
%   {\bgroup
%    \def\fileprefix{#1::}%
%    \def\rt!cross##1##2##3##4%
%      {\setxvalue{\r!cross\fileprefix#3}%
%         {\noexpand\rt!cross{##1}{##2}{##3}{##4}}}%
%    \getvalue{\r!cross\fileprefix#2}%
%    \egroup}

\def\copycrossreference#1#2#3% file from to / slow
  {\bgroup
   \doifelse{#1}{}
     {\let\fileprefix\empty}
     {\def\fileprefix{#1::}}%
   \def\rt!cross##1##2##3##4%
     {\setxvalue{\r!cross\fileprefix#3}%
        {\noexpand\rt!cross{##1}{##2}{##3}{##4}}}%
   \getvalue{\r!cross\fileprefix#2}%
   \egroup}

%D References from other files are defined globally without
%D ordering data. The first definition, the one without
%D \type{#1}, is used as a signal that references are defined.

\def\setoutercrossreference#1#2#3#4%
  {\toks0={#4}%
   \@EA\xdef\csname\r!cross\fileprefix  \endcsname{\rt!cross{}{}{1}{0}}%
   \@EA\xdef\csname\r!cross\fileprefix#1\endcsname{\rt!cross{#2}{#3}{\the\toks0}{0}}}

%D In practice accessing a reference comes down to:
%D
%D \startitemize[packed]
%D \item  checking the validity
%D \item  determining the type
%D \item  filtering the content
%D \stopitemize
%D
%D We'll deal with the last two steps first. References are
%D saved in the general format:
%D
%D \starttyping
%D {\referenceclass{realpage}{page}{text}}
%D {\referenceclass{type}{data}}
%D \stoptyping
%D
%D When we filter the content, next macros are set when we
%D meet a normal cross reference:

\let\currentrealreference      =\empty
\let\currentpagereference      =\empty
\let\currenttextreference      =\empty
\let\currentsubtextreference   =\empty
\let\currentsubsubtextreference=\empty

%D System references only have one component:

\let\currentdatareference=\empty

%D Because internally a reference comes in two disguises, one
%D with four arguments and one with only two, we need a two
%D step filter.

%\def\getreferenceelements#1%
%  {\edef\referenceelements{\getvalue{\r!cross\referenceprefix#1}}%
%   \expandafter\dogetreferenceelements\referenceelements{}{}{}{}}
%
%\def\getreferenceelements#1% only one level expansion permitted!
%  {\@EA\@EA\@EA\dogetreferenceelements\csname\r!cross\referenceprefix#1\endcsname{}{}{}{}}

\def\getreferenceelements#1% only one level expansion permitted!
  {\@EA\@EA\@EA\dogetreferenceelements\csname\r!cross\referenceprefix#1\endcsname\empty\empty\empty\empty}

%D In the following step, the \type{\ifx#1} test is needed
%D because we can access this macro directly, and therefore
%D \type{#1} can be an undefined reference (in fact, this hack
%D was needed for the line numbering mechanism).
%D
%D We already introduced a few counters. Here we see why we
%D need those. The discrepancy between the cross reference
%D definition order (determined by the utility file) and the
%D moment the reference is defined in the text, is a measure
%D for it's forward character. This enables references like
%D {\em as we will see later on}.

\chardef\currentreferencetype=0

\newif\ifforwardreference

\newif\ifrealreferencepage

\def\docheckrealreferencepage#1%
  {\doifnumberelse{#1}
     {\ifnum#1=\realpageno
        \realreferencepagetrue
      \else
        \realreferencepagefalse
      \fi}
     {\realreferencepagefalse}}

\def\currentfolioreference{0}

\def\dogetreferenceelements#1#2#3#4#5%
  {\chardef\currentreferencetype=\ifx#1\relax0\else#1\fi\relax
   \ifnum\currentreferencetype<2
     \edef\currentpagereference{#2}%
     \ifx\currentpagereference   \empty
       \let\currentfolioreference\folio
     \else
       \def \currentpagereference {\referencepagenumber[#2]}%
       \edef\currentfolioreference{\dosplitofffoliopart[#2]}%
     \fi
     \edef\currentrealreference{#3}%
     \settextreferences#4\end
     \ifnum0#5<\crossreferencenumber
       \forwardreferencetrue
     \else
       \forwardreferencefalse
     \fi
   \else
     \edef\currentrealreference {#3}%
     \def \currentdatareference {#2}%
     \let \currentfolioreference\folio
     \settextreferences#4\end
     \forwardreferencefalse
   \fi
   \ifodd\currentreferencetype
     \realreferencepagefalse
   \else
     \docheckrealreferencepage\currentrealreference
     \ifrealreferencepage \else
       \docheckrealreferencepage\currentdatareference
     \fi
   \fi}

\ifx\referencepagenumber\undefined

  \def\referencepagenumber[#1]{?}

\fi

%D Text references can contain more than one entry and
%D therefore we check for
%D
%D \starttyping
%D {entry}
%D \stoptyping
%D
%D or
%D
%D \starttyping
%D {{entry}{entry}{entry}}
%D \stoptyping
%D
%D and split accordingly.

\def\settextreferences
  {\futurelet\next\dosettextreferences}

\def\dosettextreferences
  {\ifx\next\bgroup
     \expandafter\dotriplegroupempty\expandafter\dodosettextreferences
   \else
     \expandafter\donosettextreferences
   \fi}

\def\donosettextreferences#1\end
  {\def\currenttextreference{#1}%
   \let\currentsubtextreference\empty
   \let\currentsubsubtextreference\empty}

\def\dodosettextreferences#1#2#3#4\end
  {\def\currenttextreference{#1}%
   \def\currentsubtextreference{#2}%
   \def\currentsubsubtextreference{#3}}

%D When inside this testing macro we can savely use:

\def\doifforwardreferenceelse#1#2%
  {\ifforwardreference#1\else#2\fi}

%D Duplicate references are reported while loading the utility
%D file. To prevent problems with document viewers cq.
%D preprocessors, one can enable a (bit time consuming) check.

\newif\ifcheckduplicatereferences

%D The next rather dirty trick is needed to preserve the
%D meaning of the original cross reference. In fact,
%D \type{\rt!cross} is toggled to \type{\rt!done}.

% \def\doiffirstreferenceoccurance#1#2%
%   {\ifcheckduplicatereferences
%      \doifundefinedelse{\r!cross\referenceprefix#1}
%        {#2}
%        {\getreferenceelements{#1}%
%         \ifnum\currentreferencetype=\rt!cross
%           #2%
%           \bgroup
%           \def\rt!cross##1##2##3##4%
%             {\setgvalue{\r!cross\referenceprefix#1}%
%                {\rt!done{##1}{##2}{##3}{##4}}}%
%           \getvalue{\r!cross\referenceprefix#1}%
%           \egroup
%         \fi}%
%    \else
%      #2%
%    \fi}

\def\rt!crossdone#1#2#3#4{\rt!done{#1}{#2}{#3}{#4}}

\def\dohandleduplicatereference#1%
  {\bgroup
   \let\rt!cross\rt!crossdone
   \@EA\xdef\csname\r!cross\referenceprefix#1\endcsname
     {\csname\r!cross\referenceprefix#1\endcsname}%
   \egroup}

\def\checkfirstreferenceoccurance#1#2% etex
  {\ifcsname\r!cross\referenceprefix#1\endcsname
     \getreferenceelements{#1}%
     \ifnum\currentreferencetype=\rt!cross
       \dohandleduplicatereference{#1}%
       #2%
     \fi
   \else
     #2%
   \fi}

\def\doiffirstreferenceoccurance
  {\ifcheckduplicatereferences
     \@EA\checkfirstreferenceoccurance
   \else
     \@EA\secondoftwoarguments
   \fi}

%D We still have to test for the existence of a reference, but
%D before we come to that, we first look into the way a
%D reference can be accessed. It will be no surprise that
%D references can come in several forms.
%D
%D Cross references appear as numbers (figure~1.1, chapter~2)
%D or pagenumbers (page~2, page 3--2), and are called with
%D \type{\in} and \type{\at}. In interactive documents we also
%D have \type{\goto}, \type{\button} and alike. These are more
%D versatile and look like:
%D
%D \starttyping
%D \goto[reference]
%D \goto[outer reference::]
%D \goto[outer reference::inner reference]
%D \goto[operation(argument)]
%D \goto[operation(action{argument,argument})]
%D \goto[action]
%D \goto[action{argument}]
%D \stoptyping
%D
%D The first one is a normal reference, the second and third
%D are references to a file or \URL. The brace delimited
%D references for instance refer to a \JAVASCRIPT. The last
%D example shows that we can pass arguments to the actions.
%D
%D When we split off the components of such a reference, the
%D results are available in:
%D
%D \starttyping
%D \currentreferencespecial
%D \currentreferenceoperation
%D \currentreferencearguments
%D \currentinnerreference
%D \currentouterreference
%D \currentfullreference
%D \stoptyping
%D
%D Splitting a reference is done by:
%D
%D \starttyping
%D \splitofffullreference {reference}
%D \splitoffreference     {reference}
%D \stoptyping
%D
%D The second alternative can be used in a second stage
%D splitoff and only handles \type{::}.

\newif\ifreferencefound

\let\currentfullreference     =\empty
\let\currentreferencespecial  =\empty
\let\currentreferenceoperation=\empty
\let\currentreferencearguments=\empty
\let\currentouterreference    =\empty
\let\currentinnerreference    =\empty

\def\splitofffullreference#1%
  {\edef\currentfullreference{#1}%
   \@EA\dosplitofffullreference\currentfullreference\empty(\relax)\empty\end}

\def\dosplitofffullreference#1(#2#3)#4#5\end
  {\ifx#2)%
     \let\currentreferenceoperation\empty
     \let\currentreferencearguments\empty
     \let\currentinnerreference    \empty
     \dodosplitofffullreferenceA#1::::\empty\end
       \currentouterreference\currentreferencespecial
   \else\ifx#2\relax
     \let\currentreferencespecial  \empty
     \let\currentreferenceoperation\empty
     \let\currentreferencearguments\empty
     \dodosplitofffullreferenceA#1::::\empty\end
       \currentouterreference\currentinnerreference
   \else
     \dosplitoffreferenceoperation#2#3{}\end
     \let\currentinnerreference\empty
     \dodosplitofffullreferenceB#1::::\empty\end
       \currentouterreference\currentreferencespecial
   \fi\fi}

\def\dosplitoffreferenceoperation#1#%
  {\def\currentreferenceoperation{#1}%
   \dodosplitoffreferenceoperation}

\def\dodosplitoffreferenceoperation#1#2\end
  {\def\currentreferencearguments{#1}}

\def\dodosplitofffullreferenceA#1::#2::#3#4\end#5#6%
  {\if#3:%
     \dosetfullreferenceA#5#1{}\edef#6{#2}%
   \else
     \dosetfullreferenceA#6#1{}\let#5\empty
   \fi}

\def\dosetfullreferenceA#1#2#%
  {\edef#1{#2}%
   \def\currentreferencearguments}

\def\dodosplitofffullreferenceB#1::#2::#3#4\end#5#6%
  {\if#3:%
     \edef#5{#1}\edef#6{#2}%
   \else
     \let#5\empty\edef#6{#1}%
   \fi}

\def\splitoffreference#1%
  {\expandafter\dodosplitofffullreferenceB#1::::\empty\end
     \currentouterreference\currentinnerreference}

%D Although the previous split macros have a multistep
%D character, there performance is quite reasonable.
%D
%D For debugging purposes we provide a showcase macro:

\long\def\dodoshowcurrentreference#1\from#2\with#3%
  {\convertcommand#2\to\ascii
   \edef\currentreferenceshow{\currentreferenceshow/#1/\ascii/#3/}}

\long\def\doshowcurrentreference#1%
  {\edef\currentreferenceshow{/\ifreferencefound+\else-\fi/#1}%
   \dodoshowcurrentreference ful\from\currentfullreference     \with#1%
   \dodoshowcurrentreference spe\from\currentreferencespecial  \with#1%
   \dodoshowcurrentreference ope\from\currentreferenceoperation\with#1%
   \dodoshowcurrentreference arg\from\currentreferencearguments\with#1%
   \dodoshowcurrentreference out\from\currentouterreference    \with#1%
   \dodoshowcurrentreference inn\from\currentinnerreference    \with#1}

\def\showcurrentreference%
  {\bgroup\tttf\doshowcurrentreference\par\currentreferenceshow\egroup}

%D We use this visualizer to demonstrate the way references are
%D split.
%D
%D \hbox{\splitofffullreference{rr}\showcurrentreference}
%D \hbox{\splitofffullreference{pp{rr}}\showcurrentreference}
%D \hbox{\splitofffullreference{pp(qq)}\showcurrentreference}
%D \hbox{\splitofffullreference{pp(qq{aa,bb})}\showcurrentreference}
%D \hbox{\splitofffullreference{ff::}\showcurrentreference}
%D \hbox{\splitofffullreference{ff::rr}\showcurrentreference}
%D \hbox{\splitofffullreference{ff::pp()}\showcurrentreference}
%D \hbox{\splitofffullreference{ff::pp(qq)}\showcurrentreference}
%D \hbox{\splitofffullreference{ff::pp(qq{aa})}\showcurrentreference}

%D Now we've come to the promissed testing step. As we can
%D see, this macro does bit more than testing: it also resolves
%D the reference. This means that whenever we test for the
%D existance of a reference at an outer level, we have all the
%D relevant properties of that reference avaliable inside the
%D true branche~(\type{#2}).
%D
%D The prefix has to do with localizing references. When a
%D prefix is set, looking for a reference comes to looking for
%D the prefixed one, and when not found, looking for the non
%D prefixed one. Consider for instance the prefix set to
%D \type{sidetrack}.
%D
%D \starttyping
%D \pagereference[important]
%D \pagereference[unimportant]
%D \setupreferencing[prefixprefix=sidetrack]
%D \pagereference[important]
%D \stoptyping
%D
%D results in saving (writing) the references
%D
%D \starttyping
%D ...{}{important}
%D ...{}{unimportant}
%D ...{sidetrack}{important}...
%D \stoptyping
%D
%D Now when we call for \type{unimportant}, we will indeed get
%D the pagenumber associated to this reference. But when we
%D call for \type{important}, while the prefix is still set, we
%D will get the pagenumber bound to the prefixed one.
%D
%D {\em Some day, when processing time and memory are no longer
%D performance factors, we will introduce multi||level
%D prefixes.}
%D
%D Before we start analyzing, I introduce a general
%D definition macro. Consider:
%D
%D \starttyping
%D \goto{do}[JS(My_Script{"test",123}),titlepage]
%D \stoptyping
%D
%D This can also be achieved by:
%D
%D \starttyping
%D \definereference[startup][JS(My_Script{"test",123}),titlepage]
%D \goto{do}[REF(startup)]
%D \stoptyping
%D
%D Now is this is a handy feature or not?
%D
%D \showsetup{definereference}
%D
%D We can trace references by setting the next switch to
%D true.

\newif\iftracereferences

\let\tracereferences\tracereferencestrue

\def\specialREFidentifier{REF}

\def\dodefinereference[#1][#2]%
  {\ifsecondargument
     \doifelsenothing{#2}
       {\resetreference[#1]}%
       {\@EA\gdef\csname\specialREFidentifier#1\endcsname{#2}}%
   \else\iffirstargument
     \resetreference[#1]%
   \fi\fi}

\def\definereference%
  {\dodoubleempty\dodefinereference}

\def\resetreference[#1]%
  {\global\letbeundefined{\specialREFidentifier#1}}

\beginTEX

% \def\dodoifreferencefoundelse#1#2#3%
%   {\checkreferences
%    \bgroup
%      \let\unharmedreferenceprefix=\referenceprefix
%      \bgroup
%      \splitofffullreference{#1}%
%      \ifx\currentreferencespecial\specialREFidentifier
%        \@EA\ifx\csname\specialREFidentifier\currentreferenceoperation\endcsname\relax
%          \global\referencefoundfalse
%        \else
%          \global\referencefoundtrue
%          \@EA\@EA\@EA\splitofffullreference\@EA\@EA\@EA % 1 level
%             {\csname\specialREFidentifier\currentreferenceoperation\endcsname}%
%        \fi
%      \else
%        \global\referencefoundtrue
%      \fi
%      \ifreferencefound
%        \ifx\currentreferencespecial\empty
%          \ifx\currentouterreference\empty
%            \docheckinnerreference
%            \ifreferencefound \else
%              \checkglobalfilereferences
%            \fi
%          \else
%            \docheckouterreference
%          \fi
%          \ifreferencefound
%            \ifx\currentreferencearguments\empty
%              \getreferenceelements\currentfullreference
%            \else
%              \getreferenceelements\currentinnerreference
%            \fi
%          \fi
%        \else
%          \docheckspecialreference
%        \fi
%      \fi
%      \ifreferencefound \else
%        \let\referenceprefix\empty
%        \@EA\ifx\csname\specialREFidentifier#1\endcsname\relax\else
%          \@EA\@EA\@EA\splitofffullreference\@EA\@EA\@EA % 1 level
%             {\csname\specialREFidentifier#1\endcsname}%
%          \ifx\currentreferencespecial\empty
%            \ifx\currentouterreference\empty
%              \docheckinnerreference
%              \ifreferencefound \else
%                \checkglobalfilereferences
%              \fi
%            \else
%              \docheckouterreference
%            \fi
%            \ifreferencefound
% %             \getreferenceelements\currentfullreference
%              \ifx\currentreferencearguments\empty
%                \getreferenceelements\currentfullreference
%              \else
%                \getreferenceelements\currentinnerreference
%              \fi
%            \fi
%          \else
%            \docheckspecialreference
%          \fi
%        \fi
%      \fi
%      \iftracereferences
%        \doshowcurrentreference\space
%        \writestatus{\m!references}{\currentreferenceshow}%
%      \fi
%      \ifreferencefound#2\else#3\fi
%      \egroup
%    \egroup}

\newcount\nofexpandedreferences

\def\dodoexpandreferences#1REF(#2#3)#4\relax
  {\ifx#2\relax
     \@EA\ifx\csname\specialREFidentifier#1\endcsname\relax
       \global\advance\nofexpandedreferences \plusone
       \@EA\xdef\csname REF::\number\nofexpandedreferences\endcsname{#1}%
     \else
       \edef\expandedreference{\csname\specialREFidentifier#1\endcsname,}%
     \fi
   \else
     \@EA\ifx\csname\specialREFidentifier#2#3\endcsname\relax
       % not set
     \else
       \edef\expandedreference{\csname\specialREFidentifier#2#3\endcsname,}%
     \fi
   \fi}

\endTEX

\beginETEX \ifcsname

% \def\dodoifreferencefoundelse#1#2#3%
%   {\checkreferences
%    \bgroup
%      \let\unharmedreferenceprefix=\referenceprefix
%      \bgroup
%      \splitofffullreference{#1}%
%      \ifx\currentreferencespecial\specialREFidentifier
%        \ifcsname\specialREFidentifier\currentreferenceoperation\endcsname
%          \global\referencefoundtrue
%          \@EA\@EA\@EA\splitofffullreference\@EA\@EA\@EA % 1 level
%             {\csname\specialREFidentifier\currentreferenceoperation\endcsname}%
%        \else
%          \global\referencefoundfalse
%        \fi
%      \else
%        \global\referencefoundtrue
%      \fi
%      \ifreferencefound
%        \ifx\currentreferencespecial\empty
%          \ifx\currentouterreference\empty
%            \docheckinnerreference
%            \ifreferencefound \else
%              \checkglobalfilereferences
%            \fi
%          \else
%            \docheckouterreference
%          \fi
%          \ifreferencefound
%            \ifx\currentreferencearguments\empty
%              \getreferenceelements\currentfullreference
%            \else
%              \getreferenceelements\currentinnerreference
%            \fi
%          \fi
%        \else
%          \docheckspecialreference
%        \fi
%      \fi
%      \ifreferencefound \else
%        \let\referenceprefix=\empty
%        \ifcsname\specialREFidentifier#1\endcsname
%          \@EA\@EA\@EA\splitofffullreference\@EA\@EA\@EA % 1 level
%             {\csname\specialREFidentifier#1\endcsname}%
%          \ifx\currentreferencespecial\empty
%            \ifx\currentouterreference\empty
%              \docheckinnerreference
%              \ifreferencefound \else
%                \checkglobalfilereferences
%              \fi
%            \else
%              \docheckouterreference
%            \fi
%            \ifreferencefound
% %             \getreferenceelements\currentfullreference
%              \ifx\currentreferencearguments\empty
%                \getreferenceelements\currentfullreference
%              \else
%                \getreferenceelements\currentinnerreference
%              \fi
%            \fi
%          \else
%            \docheckspecialreference
%          \fi
%        \fi
%      \fi
%      \iftracereferences
%        \doshowcurrentreference\space
%        \writestatus{\m!references}{\currentreferenceshow}%
%      \fi
%      \ifreferencefound#2\else#3\fi
%      \egroup
%    \egroup}

\newcount\nofexpandedreferences

\def\dodoexpandreferences#1REF(#2#3)#4\relax
  {\ifx#2\relax
     \ifcsname\specialREFidentifier#1\endcsname
       \edef\expandedreference{\csname\specialREFidentifier#1\endcsname,}%
     \else
       \global\advance\nofexpandedreferences \plusone
       \@EA\xdef\csname REF::\number\nofexpandedreferences\endcsname{#1}%
     \fi
   \else
     \ifcsname\specialREFidentifier#2#3\endcsname
       \edef\expandedreference{\csname\specialREFidentifier#2#3\endcsname,}%
     \else
       % not set
     \fi
   \fi}

\endETEX

\def\doexpandreferences#1,%
  {\if]#1\else
     \let\expandedreference\empty
     \dodoexpandreferences#1REF(\relax)\relax
     \@EAEAEA\doexpandreferences\@EA\expandedreference
   \fi}

\def\expandreferences#1%
  {\global\nofexpandedreferences\zerocount
   \doexpandreferences#1,],}

\def\dodoifreferencefoundelse#1%
  {\@EA\splitofffullreference\@EA{#1}%
   \ifx\currentreferencespecial\empty
     \ifx\currentouterreference\empty
       \docheckinnerreference
       \ifreferencefound \else
         \checkglobalfilereferences
       \fi
     \else
       \docheckouterreference
     \fi
     \ifreferencefound
       \ifx\currentreferencearguments\empty
         \getreferenceelements\currentfullreference
       \else
         \getreferenceelements\currentinnerreference
       \fi
     \fi
   \else
     \docheckspecialreference
   \fi
   \iftracereferences
     \doshowcurrentreference\space
     \writestatus\m!references\currentreferenceshow
   \fi}

%D Although this can be considered a hack, we provide the
%D option to locate unknown references in other (loaded) files.
%D This can be dangerous, since there can be conflicting
%D definitions.

\newconditional\autoglobalfilereferences

\def\checkglobalfilereferences% sloooow
  {\ifconditional\autoglobalfilereferences
%     \processcommacommand[\loadedreferences]\docheckglobalfilereference
      \rawprocesscommalist[\loadedreferences]\docheckglobalfilereference
   \fi}

\def\docheckglobalfilereference#1%
  {\ifundefined{\r!cross#1::\currentinnerreference}\else
     \def\currentouterreference{#1}%
     \edef\currentfullreference%
       {\currentouterreference::\currentinnerreference}%
     \global\referencefoundtrue
     \quitcommalist
   \fi}

%D For most situations, we could use:
%D
%D \starttyping
%D \let\doifreferencefoundelse=\dodoifreferencefoundelse
%D \stoptyping
%D
%D But when we also want to support chained references, we need
%D some more. Such a chained reference is defined as:
%D
%D \starttyping
%D \goto{somewhere}[JS(somescript),nextpage,JS(anotherscript)]
%D \stoptyping
%D
%D Actually supporting chains is up to the special driver. Here
%D we only provide the hooks.

%D \macros
%D   {ifenablereferencechains}
%D
%D First we provide a switch to turn this mechanism off:

\newif\ifenablereferencechains \enablereferencechainstrue

%D We don't use the general commalist processing macros,
%D because we don't want to pay a speed penalty.

\newif\ifsecondaryreference
\newcount\nofsecondaryreferences

% Aanpassen: eerst alle refs scannen en componenten opslaan in
% lijst, dan de chain doorlopen. Momenteel mag alleen laatste
% laatste undefined zijn, eigenlijk moet dat overal kunnen met
% 'geen' zonder melding. Is wel trager. Dus niet.
%
% \def\doifreferencefoundelse#1#2#3% \@EA niet nodig
%   {\doresetgotowhereever
%    \global\secondaryreferencefalse
%    {\the\everyreference\xdef\primaryreference{#1}}%
%    \global\nofsecondaryreferences=0
%    \def\dopreprocessreference##1,%
%      {\if]##1\else
%         \ifsecondaryreference
%           \ifenablereferencechains \iflocation
%             {\the\everyreference\xdef\secondaryreference{##1}}%
%            %\global\advance\nofsecondaryreferences by 1
%             \@EA\dodoifreferencefoundelse\@EA{\secondaryreference}
%               {\global\advance\nofsecondaryreferences by 1
%                #2}
%               {%\global\advance\nofsecondaryreferences by -1
%                \dostartnoto#3\dostopnoto}%
%           \fi \fi
%         \else
%           {\the\everyreference\xdef\primaryreference{##1}}%
%           \global\secondaryreferencetrue
%         \fi
%         \expandafter\dopreprocessreference
%       \fi}%
%    \dopreprocessreference#1,],%
%    \global\secondaryreferencefalse
%    \@EA\dodoifreferencefoundelse\@EA{\primaryreference}{#2}{#3}%
%    \doresetgotowhereever} % to prevent problems with direct goto's

% \def\doifreferencefoundelse#1#2#3% REF \cs
%   {\checkreferences
%    \bgroup
%      \the\everyreference
%      \let\referenceprefix\empty
%      \expandreferences{#1}%
%    \egroup
%    \doresetgotowhereever
%    \global\nofsecondaryreferences 0
%    \ifcase\nofexpandedreferences\relax % #1 can be number -)
%      % no ref
%    \or
%      % one ref
%    \or
%      % two refs
%      \ifenablereferencechains \iflocation
%        \global\secondaryreferencetrue
%        \xdef\secondaryreference{\csname REF::2\endcsname}%
%        \bgroup
%          \let\unharmedreferenceprefix\referenceprefix
%          %\bgroup
%            \dodoifreferencefoundelse\secondaryreference
%            \global\nofsecondaryreferences 1
%            \ifreferencefound
%              #2%
%            \else
%              \dostartnoto#3\dostopnoto
%            \fi
%          %\egroup
%        \egroup
%      \fi \fi
%    \else
%      % more than two refs
%      \ifenablereferencechains \iflocation
%        \global\secondaryreferencetrue
%        \scratchcounter2
%        \loop
%          \xdef\secondaryreference{\csname REF::\number\scratchcounter\endcsname}%
%          \bgroup
%            \let\unharmedreferenceprefix\referenceprefix
%            %\bgroup
%              \dodoifreferencefoundelse\secondaryreference
%              \global\advance\nofsecondaryreferences 1
%              \ifreferencefound
%                #2%
%              \else
%                \global\advance\nofsecondaryreferences -1
%                \dostartnoto#3\dostopnoto
%              \fi
%            %\egroup
%          \egroup
%          \ifnum\scratchcounter<\nofexpandedreferences\relax
%            \advance\scratchcounter 1
%        \repeat
%      \fi \fi
%    \fi
%    \global\secondaryreferencefalse
%    \xdef\primaryreference{\csname REF::1\endcsname}%
%    \bgroup
%      \let\unharmedreferenceprefix\referenceprefix
%      %\bgroup
%        \dodoifreferencefoundelse\primaryreference
%        \ifreferencefound#2\else#3\fi
%      %\egroup
%    \egroup
%    \doresetgotowhereever} % to prevent problems with direct goto's

\def\doifreferencefoundelse#1#2#3% REF \cs
  {\checkreferences
   % first we collect the secondary ones
   \bgroup
     \the\everyreference
     \let\referenceprefix\empty
     \expandreferences{#1}%
   \egroup
   \doresetgotowhereever
   \global\nofsecondaryreferences \zerocount
   \ifcase\nofexpandedreferences\relax % #1 can be number -)
     % no ref
   \or
     % one ref
   \or
     % two refs
     \ifenablereferencechains \iflocation
       \global\secondaryreferencetrue
       \xdef\secondaryreference{\csname REF::2\endcsname}%
       % test: \global\letcscsname\secondaryreference\csname REF::2\endcsname
       \bgroup
         %%\let\doifreferencefoundelse\localdoifreferencefoundelse
         \let\unharmedreferenceprefix\referenceprefix
         \dodoifreferencefoundelse\secondaryreference
         \ifreferencefound
           \global\nofsecondaryreferences \plusone
           #2%
         \else
           \dostartnoto#3\dostopnoto
         \fi
       \egroup
     \fi \fi
   \else
     % more than two refs
     \ifenablereferencechains \iflocation
       \global\secondaryreferencetrue
       \scratchcounter\plustwo
       \loop
         \xdef\secondaryreference{\csname REF::\number\scratchcounter\endcsname}%
         % test: \global\letcscsname\secondaryreference\csname REF::\number\scratchcounter\endcsname
         \bgroup
           %%\let\doifreferencefoundelse\localdoifreferencefoundelse
           \let\unharmedreferenceprefix\referenceprefix
           \dodoifreferencefoundelse\secondaryreference
           \ifreferencefound
             \global\advance\nofsecondaryreferences \plusone
             #2%
           \else
             \dostartnoto#3\dostopnoto
           \fi
         \egroup
         \ifnum\scratchcounter<\nofexpandedreferences\relax
           \advance\scratchcounter \plusone
       \repeat
     \fi \fi
   \fi
   \global\secondaryreferencefalse
   \xdef\primaryreference{\csname REF::1\endcsname}%
   % test: \global\letcscsname\secondaryreference\csname REF::1\endcsname
   \bgroup
     % now we handle the primary one
     %%\let\doifreferencefoundelse\localdoifreferencefoundelse
     \let\unharmedreferenceprefix\referenceprefix
     \dodoifreferencefoundelse\primaryreference
     \ifreferencefound#2\else#3\fi
   \egroup
   \doresetgotowhereever} % to prevent problems with direct goto's

%D The following local redefinition permits the usage of
%D nested \type {\doifreferencefoundelse}; see for an
%D example the local test for file|/|url references. This is
%D a fuzzy part of this mechanism and a result of the choice
%D to let speed prevail over beauty in resolving chained
%D references with symbolic (defined) references.

% \def\localdoifreferencefoundelse#1#2#3% beware, this feeds back the (...) ref
%   {\dodoifreferencefoundelse{#1}%
%    \ifreferencefound#2\else#3\fi}

\def\localdoifreferencefoundelse#1%
  {\dodoifreferencefoundelse{#1}%
   \ifreferencefound\@EA\firstoftwoarguments\else\@EA\secondoftwoarguments\fi}

%D Somewhere else we will properly define \type {\dostartgoto};
%D the noto alternative takes care of undefined references in
%D a sequence

\ifx\dostartnoto\undefined

  \def\dostartnoto#1\dostopnoto
    {\ifsecondaryreference\else{#1}\fi}

\fi

\ifx\dostartgoto\undefined

  \def\dostartgoto\data#1\start#2\stop#3\dostopgoto
    {\ifsecondaryreference\else{#1}\fi}

\fi

%D As one can see, while processing the references, the first
%D one is handled last. While scanning the second and following
%D ones, we increment a counter and set a boolean to true.

%D The next fast one permits rather raw references with
%D \type{()}'s and is used in the object reference mechanism.

\def\doifrawreferencefoundelse#1#2#3%
  {\checkreferences
   \bgroup
   \edef\currentfullreference{#1}%
   \ifundefined{\r!cross\currentfullreference}%
     \global\referencefoundfalse #3%
   \else\getreferenceelements\currentfullreference
     \global\referencefoundtrue  #2%
   \fi
   \egroup}

%D The inner case is simple. Only two cases have to be taken
%D care of:
%D
%D \starttyping
%D \goto{some text}[reference]
%D \goto{some text}[prefix:reference]
%D \stoptyping

%\def\docheckinnerreference%
%  {\global\let\predefinedreference=\currentinnerreference
%   \ifx\currentreferencearguments\empty
%     \doifdefinedelse{\r!cross\referenceprefix\currentfullreference}
%       {\global\referencefoundtrue}
%       {\let\referenceprefix\empty
%        \doifdefinedelse{\r!cross\currentfullreference}
%          {\global\referencefoundtrue}
%          {\global\referencefoundfalse}}%
%   \else % [SomeThing{with,me}]
%     \let\referenceprefix\empty
%     \doifdefinedelse{\r!cross\currentinnerreference}
%       {\global\referencefoundtrue}
%       {\global\referencefoundfalse}%
%   \fi
%   \doifpredefinedreferenceelse{\global\referencefoundfalse}{}}

\beginETEX

\def\docheckinnerreference
  {\global\let\predefinedreference\currentinnerreference
   \ifx\currentreferencearguments\empty
     \ifcsname\r!cross\referenceprefix\currentfullreference\endcsname
       \global\referencefoundtrue
     \else
       \let\referenceprefix\empty
       \ifcsname\r!cross\currentfullreference\endcsname
         \global\referencefoundtrue
       \else
         \global\referencefoundfalse
       \fi
     \fi
   \else % [SomeThing{with,me}]
     \let\referenceprefix\empty
     \ifcsname\r!cross\currentinnerreference\endcsname
       \global\referencefoundtrue
     \else
       \global\referencefoundfalse
     \fi
   \fi
   \doifpredefinedreferenceelse{\global\referencefoundfalse}\donothing}

\endETEX

\beginTEX

\def\docheckinnerreference
  {\global\let\predefinedreference\currentinnerreference
   \ifx\currentreferencearguments\empty
     \@EA\ifx\csname\r!cross\referenceprefix\currentfullreference\endcsname\relax
       \let\referenceprefix\empty
       \@EA\ifx\csname\r!cross\currentfullreference\endcsname\relax
         \global\referencefoundfalse
       \else
         \global\referencefoundtrue
       \fi
     \else
       \global\referencefoundtrue
     \fi
   \else % [SomeThing{with,me}]
     \let\referenceprefix\empty
     \@EA\ifx\csname\r!cross\currentinnerreference\endcsname\relax
       \global\referencefoundfalse
     \else
       \global\referencefoundtrue
     \fi
   \fi
   \doifpredefinedreferenceelse{\global\referencefoundfalse}\donothing}

\endTEX

%D References to other files however are treated strict or
%D tolerant, depending on their loading and availability:
%D
%D \starttyping
%D \useexternaldocument[somefile][filename][a nice description]
%D
%D \goto{checked   reference}[somefile::reference]
%D \goto{unchecked reference}[somefile::]
%D \goto{unchecked reference}[anotherfile::reference]
%D \stoptyping
%D
%D Here we use the dummy reference \type{somefile::} set in
%D \type{\setouterreference} as a signal that indeed references
%D are defined for the outer file.

% \def\docheckouterreference%
%   {\let\referenceprefix=\empty
%    \let\unharmedreferenceprefix=\empty
%    \xdef\predefinedreference%
%      {\currentouterreference::\currentinnerreference}%
%    \ifx\innerreference\empty
%      \global\referencefoundtrue % no checking done
%    \else
%      \doifdefinedelse{\r!cross\currentouterreference::}
%        {\doifdefinedelse{\r!cross\currentfullreference}
%           {\global\referencefoundtrue}
%           {\global\referencefoundfalse}}
%        {\global\referencefoundtrue}% no checking done
%    \fi
%    \doifpredefinedreferenceelse{\global\referencefoundfalse}{}}

\newif\ifstrictouterreferences \strictouterreferencesfalse

\beginTEX

\def\dodocheckouterreference
  {\@EA\ifx\csname\specialREFidentifier\currentfullreference\endcsname\relax
     \ifstrictouterreferences
       \global\referencefoundfalse
     \else
       % already \global\referencefoundtrue % no checking done
     \fi
   \else
     \@EA\@EA\@EA\splitofffullreference\@EA\@EA\@EA % 1 level
        {\csname\specialREFidentifier\currentfullreference\endcsname}%
     \docheckouterreference
   \fi}

\def\docheckouterreference
  {\let\referenceprefix\empty
   \let\unharmedreferenceprefix\empty
   \xdef\predefinedreference
     {\currentouterreference::\currentinnerreference}%
   \ifx\innerreference\empty
     \global\referencefoundtrue % no checking done
   \else
     \@EA\ifx\csname\r!cross\currentouterreference::\endcsname\relax
       \ifstrictouterreferences
         \global\referencefoundfalse
       \else
         \global\referencefoundtrue % no checking done
       \fi
     \else
       \ifcsname\r!cross\currentfullreference\endcsname
         \global\referencefoundtrue
       \else
         \dodocheckouterreference
       \fi
     \fi
   \fi
   \doifpredefinedreferenceelse{\global\referencefoundfalse}\donothing}

\endTEX

\beginETEX \ifcsname

\def\dodocheckouterreference
  {\ifcsname\specialREFidentifier\currentfullreference\endcsname
     \@EA\@EA\@EA\splitofffullreference\@EA\@EA\@EA % 1 level
        {\csname\specialREFidentifier\currentfullreference\endcsname}%
     \docheckouterreference
   \else\ifstrictouterreferences
     \global\referencefoundfalse
   \else
     % already \global\referencefoundtrue % no checking done
   \fi\fi}

\def\docheckouterreference
  {\let\referenceprefix\empty
   \let\unharmedreferenceprefix\empty
   \xdef\predefinedreference
     {\currentouterreference::\currentinnerreference}%
   \ifx\innerreference\empty
     \global\referencefoundtrue % no checking done
   \else
     \ifcsname\r!cross\currentouterreference::\endcsname
       \ifcsname\r!cross\currentfullreference\endcsname
         \global\referencefoundtrue
       \else
         \dodocheckouterreference
       \fi
     \else
       \ifstrictouterreferences
         \global\referencefoundfalse
       \else
         \global\referencefoundtrue % no checking done
       \fi
     \fi
   \fi
   \doifpredefinedreferenceelse{\global\referencefoundfalse}\donothing}

\endETEX

%D Special references are only tested when some test routine is
%D defined.

\def\docheckspecialreference
  {\let\referenceprefix\empty
   \let\unharmedreferenceprefix\empty
   \xdef\predefinedreference
     {\currentreferencespecial::\currentreferenceoperation}%
   \executeifdefined{\s!do:\v!test:\currentreferencespecial}%
     {\global\referencefoundtrue\gobbletwoarguments}%
     {\global\referencefoundtrue}{\global\referencefoundfalse}%
   \doifpredefinedreferenceelse{\global\referencefoundfalse}\donothing}

%D An unknown reference is reported on the screen, in the log
%D file and, when enabled, in the left margin of the text.

\def\reportreferenceerror#1#2%
  {\bgroup
   \the\everyreference % cleanup : etc in french
   \ifinpagebody\else
     \doifconcepttracing
       {\doifsomething{#2}
          {\inleft
             {\infofont
              \scratchdimen\leftmarginwidth
              \advance\scratchdimen -2em
              \doboundtext{#2}\scratchdimen{..}->}}}%
   \fi
   \doifpredefinedreferenceelse
     \donothing
     {\predefinereference
      \showmessage\m!references{#1}{[\unharmedreferenceprefix][#2]}}%
   \egroup}

\def\unknownreference{\reportreferenceerror1}
\def\illegalreference{\reportreferenceerror4}

%D Although not actually needed, we default the unharmed
%D reference prefix to the normal one.

\def\unharmedreferenceprefix{\referenceprefix}

%D When a reference is not found, we typeset a placeholder
%D (two glyphs are often enough to represent the reference
%D text).

\def\dummyreference{{\tttf ??}}

%D To prevent repetitive messages concerning a reference
%D being defined, we set such an unknown reference to an empty
%D one after the first encounter.

\let\predefinedreference\s!unknown

% \def\predefinereference
%   {\setgvalue{\r!cross\predefinedreference}{\rt!cross{}{}{}{}}%
%    \setgvalue{\r!cross\unharmedreferenceprefix\predefinedreference}{\rt!cross{}{}{}{}}}

\def\dummypredefinedreference{\r!cross{}{}{}{}}

\def\predefinereference
  {\global\@EA\let\csname\r!cross\predefinedreference    \endcsname\dummypredefinedreference
   \global\@EA\let\csname\r!cross\unharmedreferenceprefix\endcsname\dummypredefinedreference} % bug!

%D Testing on existance then becomes:

% \def\checkpredefinedreference#1#2#3#4% #3 can expand tricky
%   {{#1}{#2}{#4}}
%
% \def\doifpredefinedreferenceelse#1#2% \referenceprefix added
%   {\bgroup
%    \let\rt!cross\checkpredefinedreference
%    \doifelse{\getvalue{\r!cross\referenceprefix\predefinedreference}}{{}{}{}}
%      {\egroup#1}{\egroup#2}}

\def\doifpredefinedreferenceelse % \referenceprefix added
  {\@EA\ifx\csname\r!cross\referenceprefix\predefinedreference\endcsname\dummypredefinedreference
     \expandafter\firstoftwoarguments
   \else
     \expandafter\secondoftwoarguments
   \fi}

%D Sometimes we want to temporary put a reference out of
%D order. An example can be found in the menu macros.
%D
%D \starttyping
%D \doifreferencepermittedelse{reference}{set}{true}{false}
%D \stoptyping
%D
%D The second argument can be a comma seperated list.

% this will be replaced by a more general mechanis

% \def\doifreferencepermittedelse#1#2#3#4% ref set found notfound
%   {\doifreferencefoundelse{#1}
%      {\ifx\currentinnerreference\empty
%         \ifx\currentouterreference\empty
%           #3%
%         \else
%           \doifinstringelse{\currentouterreference::}{#2}{#4}{#3}%
%         \fi
%       \else\ifx\currentouterreference\empty
%         \doifinstringelse{\currentinnerreference}{#2}{#4}{#3}%
%       \else
%         \doifinstringelse{\currentouterreference::\currentinnerreference}{#2}{#4}{#3}%
%       \fi\fi}
%      {#4\unknownreference{#1}}}

\let\permittedreferences\empty

\def\doifreferencepermittedelse#1#2#3% ref found notfound
  {\doifreferencefoundelse{#1}
     {\donetrue
      \ifx\permittedreferences\empty \else
        \docheckifreferencepermitted{#1}%
      \fi
      \ifdone#2\else#3\fi}
     {#3\unknownreference{#1}}}

\def\docheckifreferencepermitted#1%
  {\ifx\currentinnerreference\empty
     \ifx\currentouterreference\empty \else
       \doifinstring{\currentouterreference::}\permittedreferences\donefalse
     \fi
   \else\ifx\currentouterreference\empty
     \doifinstring{\currentinnerreference}\permittedreferences\donefalse
   \else
     \doifinstring{\currentouterreference::\currentinnerreference}\permittedreferences\donefalse
   \fi\fi}

%D Apart from cross references supplied by the user, \CONTEXT\
%D generates cross references itself. Most of them are not
%D saved as a reference, but stored with their source, for
%D instance a list or an index entry. Such automatically
%D generated, for the user invisible, references are called
%D {\em internal references}. The user supplied ones are
%D labeled as {\em external references}.
%D
%D A second important characteristic is that when we want to
%D support different backends (viewers), we need to support
%D named destinations as well as page numbers. I invite readers
%D to take a glance at the special driver modules to understand
%D the fine points of this. As a result we will deal with {\em
%D locations} as well as {\em real page numbers}. We explictly
%D call this pagenumber a real one, because it is independant
%D of the page numbering scheme used in the document.
%D
%D One of the reasons for \CONTEXT\ being the first \TEX\ base
%D macropackage to support sophisticated interactive \PDF\
%D files, lays in the mere fact that real page numbers are
%D available in most two pass data, like references, list data
%D and index entries.
%D
%D We will speak of \type{thisis...} when we are marking a
%D location, and \type{goto...} when we point to such a
%D location. The latter one can be seen as a hyperlink to the
%D former one. In the next macros one we use constructs like:
%D
%D \starttyping
%D \dostart...
%D \dostop...
%D \stoptyping
%D
%D Such macros are used to invoke the relevant specials from
%D the special driver modules (see \type{spec-ini}). The flag
%D \type{\iflocation} signals if we're in interactive mode.

\def\thisisdestination#1% destination
  {\iflocation \ifusepagedestinations \else
     \dostartthisislocation{#1}\dostopthisislocation
   \fi \fi}

\def\thisisrealpage#1% pagenumber
  {\iflocation
     \dostartthisisrealpage{#1}\dostopthisisrealpage
   \fi}

%D The previous tho macros were easy ones, opposite to their
%D counterparts. A common component in these is:
%D
%D \starttyping
%D \dostartgoto
%D   \data  {..}
%D   \start {..}
%D   \stop  {..}
%D \dostopgoto
%D \stoptyping
%D
%D Here data can be whatever needs highlighting, e.g. {\em
%D figure 2.4}, and the start and stop entries handle the
%D specials. The two \DIMENSIONS\ \type{\buttonwidth} and
%D \type{\buttonheight} have to be set when handling the
%D data~(\type{#2}).

\ifx\buttonheight\undefined \newdimen\buttonheight \fi
\ifx\buttonwidth \undefined \newdimen\buttonwidth  \fi

\def\gotodestination#1#2#3#4#5% url file destination page data
  {\iflocation
     \ifusepagedestinations
       \gotorealpage{#1}{#2}{#4}{#5}%
     \else
       \dostartgoto
         \data
           {#5}%
         \start
           \the\everyreference % we are (have to be) grouped !
           \dostartgotolocation\buttonwidth\buttonheight{#1}{#2}{#3}{#4}%
         \stop
           \dostopgotolocation
       \dostopgoto
     \fi
   \else
     {#5}%
   \fi}

\def\gotorealpage#1#2#3#4% url file page data
  {\iflocation
     \dostartgoto
       \data
         {#4}%
       \start
         \dostartgotorealpage\buttonwidth\buttonheight{#1}{#2}{#3}%
       \stop
         \dostopgotorealpage
     \dostopgoto
   \else
     {#4}%
   \fi}

%D \macros
%D   {setreferencefilename}
%D
%D This command can be used in the special drivers to
%D uppercase filenames. This is needed when one wants to
%D produce \CDROM's conforming to ISO9660. We consider is the
%D savest to enable this feature by default. We cannot handle
%D uppercase here, since the suffix is handled in the special
%D driver. Conversion is taken care of by:
%D
%D \starttyping
%D \setreferencefilename somefilename\to\SomeFileName
%D \stoptyping

% \def\setreferencefilename#1\to#2%
%   {\doifelse{\@@converteerfile}{\v!ja} % boolean is sneller
%      {\uppercasestring#1\to#2}
%      {\edef#2{#1}}}

\chardef\referencefilecase=0

\def\setreferencefilename#1\to#2%
  {\ifcase\referencefilecase
     \edef#2{#1}%
   \or
     \uppercasestring#1\to#2%
   \or
     \lowercasestring#1\to#2%
   \else
     \edef#2{#1}%
   \fi}

%D Internal references can best be set using the next few
%D macros. Setting such references to unique values is
%D completely up to the macros that call them.
%D
%D \starttyping
%D \thisissomeinternal{tag}{identifier}
%D \gotosomeinternal  {tag}{identifier}{pagenumber}{text}
%D \stoptyping

\def\thisissomeinternal#1#2%  tag reference
  {\doifsomething{#2}{\thisisdestination{#1:#2}}}

\def\gotosomeinternal#1#2% #3#4
  {\gotodestination\empty\empty{#1:#2}}

%D An automatic mechanism is provided too:
%D
%D \starttyping
%D \thisisnextinternal{tag}
%D \gotonextinternal  {tag}{number}{pagenumber}{text}
%D \stoptyping
%D
%D The first macro increments a counter. The value of this
%D counter is available in the macro \type{\nextinternalreference}
%D and should be saved somewhere (for instance in a file) for
%D future reference. The second argument of
%D \type {\gotonextinternal} takes such a saved number. One can
%D turn on tracing these references, in which case the
%D references are a bit more verbose.

\newcount\locationcount

\newif\iftraceinternalreferences
\newif\ifinternalnamedreferences \internalnamedreferencestrue

\def\nextinternalreference
  {\the\locationcount}

% \def\thisisnextinternal#1%
%   {\global\advance\locationcount \plusone
%    \thisisdestination{\s!aut\iftraceinternalreferences:#1\fi:\nextinternalreference}}
%
% \def\gotonextinternal#1#2% #3#4
%   {\gotodestination{}{}{\s!aut\iftraceinternalreferences:#1\fi:#2}}

\def\thisisnextinternal#1%
  {\global\advance\locationcount \plusone
   \ifinternalnamedreferences
     \thisisdestination{\s!aut\iftraceinternalreferences:#1\fi:\nextinternalreference}%
   \fi}

\def\gotonextinternal#1#2#3#4%
  {\ifinternalnamedreferences
     \gotodestination\empty\empty{\s!aut\iftraceinternalreferences:#1\fi:#2}{#3}{#4}%
   \else
     \gotorealpage\empty\empty{#3}{#4}%
   \fi}

%D We already went through a lot of problems to sort out what
%D kind of reference we're dealing with. Sorting out the user
%D supplied cross references (show/goto this or that) as well
%D as user supplied system references (invoke this or that) is
%D already taken care of in the test routine, but we still have
%D to direct the request to the right (first) routine.

\def\gotolocation% #1#2%
  {\ifx\currentreferencespecial\empty
     \ifx\currentouterreference\empty
       \ifnum\currentreferencetype<2
         \@EA\@EAEAEA\@EA\gotoinnerlocation
       \else
         \@EA\@EAEAEA\@EA\gotosystemlocation
       \fi
     \else
       \@EAEAEA\gotoouterlocation
     \fi
   \else
     \@EA\gotospeciallocation
   \fi} % {#1}{#2}

%D An inner reference refers to some place in the document
%D itself.

\def\gotoinnerlocation#1% #2%
  {\gotodestination\empty\empty
     {\referenceprefix\currentinnerreference}\currentrealreference} % {#2}

%D The outer location refers to another document, specified as
%D file or \URL.

\def\gotoouterlocation#1#2% % page checken!
  {\bgroup
   \let\referenceprefix\empty
   \setouterlocation\currentouterreference
   \ifx\currentinnerreference\empty
     \gotorealpage
       \otherURL\otherfile1{#2}%
   \else
     \gotodestination
       \otherURL\otherfile\currentinnerreference\currentrealreference{#2}%
   \fi
   \egroup}

%D Special locations are those that are accessed by saying
%D things like:
%D
%D \starttyping
%D \goto{calculate total}[JS(summarize{10,23,56}]
%D \stoptyping
%D
%D After several intermediate steps this finally arrives at
%D the next macro and expands into (simplified):
%D
%D \starttyping
%D \gotoJSlocation{total{summarize{10,23,56}}}{calculate total}
%D \stoptyping
%D
%D The first argument is the full reference, the second one
%D is the text, in some kind of manipulated form. In practice
%D we split references, so we get:
%D
%D \starttyping
%D \gotoJSlocation{summarize{10,23,56}}{calculate}
%D \gotoJSlocation{summarize{10,23,56}}{total}
%D \stoptyping
%D
%D where \type{calculate} and \type{total} are colored, boxed
%D or whatever \type{\goto} is told to do.
%D
%D The macro \type{\gotoJSlocation} can use \type
%D {\currentreferenceoperation} (in our example
%D \type{summarize}) and \type{\currentreference} (here
%D being \type {10,23,56}) to perform its task.

\def\gotospeciallocation
  {\executeifdefined{goto\currentreferencespecial location}\gobbleoneargument}

%D Such special macros can be defined by:

\def\definespeciallocation#1%
  {\setvalue{goto#1location}}

%D The associated test is to be defined by:

\def\definespecialtest#1%
  {\setvalue{\s!do:\v!test:#1}}

%D This \type{\def} alike macro is to be used as:
%D
%D \starttyping
%D \definespeciallocation{JS}#1#2{... #1 ... #2 ...}
%D \stoptyping
%D
%D In module \type {java-ini} one can see that \type
%D {\gotoJSlocation} looks much like the previous goto
%D definitions.

%D A system location is not always a location, but for the
%D consistency we also consider actions as such.

\def\gotosystemlocation
  {\csname\r!syst\the\currentreferencetype\endcsname}

\def\definesystemreferencehandler#1#2%
  {\setgvalue{\r!syst\the#1}{#2}}

%D In this module we define three system references: one for
%D handling navigational, viewer specific, commands, another
%D for jumping to special pages, like the first or last one,
%D and a third reference for linking tree like lists, like
%D tables of contents. The latter two adapt themselves to the
%D current state.

\definesystemreferencehandler \rt!exec \handleexecreference
\definesystemreferencehandler \rt!page \handlepagereference
\definesystemreferencehandler \rt!list \handlelistreference

\def\handleexecreference#1%
  {\checkexecutecommand\currentdatareference\currentreferencearguments
   \executecommand\currentdatareference\currentreferencearguments}

\def\handlepagereference#1%
  {\gotorealpage\empty\empty\currentdatareference}

\def\handlelistreference#1% is deze nog echt nodig?
  {\gotodestination\empty\empty\currentdatareference{\getvalue{\currentdatareference}}}

%D \macros
%D   {setexecutecommandcheck}
%D
%D In case a command action needs to do some checking in
%D advance, one can assign an check function by:
%D
%D \starttyping
%D \setexecutecommandcheck{startsound}\checksoundtrack
%D \stoptyping

\def\setexecutecommandcheck#1#2% #2 permits \first \second
  {\setvalue{\s!do:\s!do:#1}{#2}}

\beginETEX

\def\checkexecutecommand#1#2% evt geen #1 en #2
  {\ifx#2\empty \else \ifcsname\s!do:\s!do:#1\endcsname
     \@EA\let\@EA\docheckexecutecommand\csname\s!do:\s!do:#1\endcsname
     \rawprocesscommalist[#2]\docheckexecutecommand
   \fi \fi }

\endETEX

\beginTEX

\def\checkexecutecommand#1#2% evt geen #1 en #2
  {\ifx#2\empty \else \@EA\ifx\csname\s!do:\s!do:#1\endcsname\relax \else
     \@EA\let\@EA\docheckexecutecommand\csname\s!do:\s!do:#1\endcsname
     \rawprocesscommalist[#2]\docheckexecutecommand
   \fi \fi }

\endTEX

%D Command references (in dutch, english, german of
%D whatever interface language) are translated into a bit
%D shorter reference (\type{close}) and passed to the
%D special driver (using \type{\executecommand}).

% better: [action(name)] and \definereference[name][action(name)]

\setglobalsystemreference \rt!exec \v!CloseDocument    {close}
\setglobalsystemreference \rt!exec \v!ExitViewer       {exit}
\setglobalsystemreference \rt!exec \v!FirstPage        {first}
\setglobalsystemreference \rt!exec \v!LastPage         {last}
\setglobalsystemreference \rt!exec \v!NextJump         {forward}
\setglobalsystemreference \rt!exec \v!NextPage         {next}
\setglobalsystemreference \rt!exec \v!PauseMovie       {pausemovie}
\setglobalsystemreference \rt!exec \v!PauseSound       {pausesound}
\setglobalsystemreference \rt!exec \v!PauseRendering   {pauserendering}
\setglobalsystemreference \rt!exec \v!PreviousJump     {backward}
\setglobalsystemreference \rt!exec \v!PreviousPage     {previous}
\setglobalsystemreference \rt!exec \v!PrintDocument    {print}
\setglobalsystemreference \rt!exec \v!SaveForm         {exportform}
\setglobalsystemreference \rt!exec \v!LoadForm         {importform}
\setglobalsystemreference \rt!exec \v!ResetForm        {resetform}
\setglobalsystemreference \rt!exec \v!ResumeMovie      {resumemovie}
\setglobalsystemreference \rt!exec \v!ResumeSound      {resumesound}
\setglobalsystemreference \rt!exec \v!ResumeRendering  {resumerendering}
\setglobalsystemreference \rt!exec \v!SaveDocument     {save}
\setglobalsystemreference \rt!exec \v!SaveNamedDocument{savenamed}
\setglobalsystemreference \rt!exec \v!OpenNamedDocument{opennamed}
\setglobalsystemreference \rt!exec \v!SearchDocument   {search}
\setglobalsystemreference \rt!exec \v!SearchAgain      {searchagain}
\setglobalsystemreference \rt!exec \v!StartMovie       {startmovie}
\setglobalsystemreference \rt!exec \v!StartSound       {startsound}
\setglobalsystemreference \rt!exec \v!StartRendering   {startrendering}
\setglobalsystemreference \rt!exec \v!StopMovie        {stopmovie}
\setglobalsystemreference \rt!exec \v!StopSound        {stopsound}
\setglobalsystemreference \rt!exec \v!StopRendering    {stoprendering}
\setglobalsystemreference \rt!exec \v!SubmitForm       {submitform}
\setglobalsystemreference \rt!exec \v!ToggleViewer     {toggle}
\setglobalsystemreference \rt!exec \v!ViewerHelp       {help}
\setglobalsystemreference \rt!exec \v!HideField        {hide}
\setglobalsystemreference \rt!exec \v!ShowField        {show}
\setglobalsystemreference \rt!exec \v!GotoPage         {gotopage}
\setglobalsystemreference \rt!exec \v!GotoPage         {gotopage}
\setglobalsystemreference \rt!exec \v!Query            {query}
\setglobalsystemreference \rt!exec \v!QueryAgain       {queryagain}
\setglobalsystemreference \rt!exec \v!FitWidth         {fitwidth}
\setglobalsystemreference \rt!exec \v!FitHeight        {fitheight}

%D Executing the command looks alike the previous goto macros.

\def\executecommand#1#2#3%
  {\iflocation
     \dostartgoto
       \data
         {#3}%
       \start
         \dostartexecutecommand\buttonwidth\buttonheight{#1}{#2}%
       \stop
         \dostopexecutecommand
     \dostopgoto
   \else
     {#3}%
   \fi}

%D We could have done without the short tags and thereby saving
%D some tokens, but the current approach leaves room for future
%D extensions.

%D It is possible to disable the writing of references to the
%D utility file by setting:

\newif\ifreferencing \referencingtrue

%D One can also activate an automatic prefix mechanism. By
%D setting the \type{\prefix} variable to \type{+}, the prefix
%D is incremented, when set to \type{-} or empty, the prefix is
%D reset. Other values become the prefix.

\newcount\prefixcounter

%D These settings are accomplished by:
%D
%D \showsetup{setupreferencing}
%D
%D In interactive documents verbose references don't always
%D make sense (what is a page number in an unnumbered
%D document). By setting the \type{interaction} variable, one
%D can influences the way interactive references are set.

% \newif\ifreferencestrut % some day an option

\def\setupreferencing
  {\dosingleargument\dosetupreferencing}

\def\dosetupreferencing[#1]%
  {\getparameters
     [\??rf]
     [\c!prefix=\s!unknown,#1]%
   \processaction
     [\@@rfstate]
     [  \v!stop=>\referencingfalse,
       \v!start=>\referencingtrue]%
   \processaction
     [\@@rfinteraction]
     [   \v!all=>\let\dowantedreference\docompletereference,
       \v!label=>\let\dowantedreference\dolabelonlyreference,
        \v!text=>\let\dowantedreference\dotextonlyreference,
      \v!symbol=>\let\dowantedreference\dosymbolreference]%
   \chardef\autocrossfilereferences\zerocount
   \processaction
     [\@@rfautofile]
     [  \v!yes=>\chardef\autocrossfilereferences\plusone,
       \v!page=>\chardef\autocrossfilereferences\plustwo]%
   \chardef\referencefilecase\zerocount
   \processaction[\@@rfconvertfile]
     [  \v!yes=>\chardef\referencefilecase\plusone,
        \v!big=>\chardef\referencefilecase\plusone,
      \v!small=>\chardef\referencefilecase\plustwo]%
  %\doifelse\@@rfstrut\v!yes % some day an option
  %  \referencetruttrue\referencestrutfalse
   \setupreferenceprefix[\@@rfprefix]%
   \doifelse\@@rfglobal\v!yes
     {\settrue \autoglobalfilereferences}
     {\setfalse\autoglobalfilereferences}}

\def\incrementreferenceprefix{+}
\def\decrementreferenceprefix{-}

\def\setupreferenceprefix[#1]%
  {\edef\@@rfprefix{#1}%
   \ifx\@@rfprefix\empty
     \let\referenceprefix\empty
   \else\ifx\@@rfprefix\incrementreferenceprefix
     \advance\prefixcounter \plusone % should be global
     \edef\referenceprefix{\the\prefixcounter:}%
     \let\@@rfprefix\s!unknown
   \else\ifx\@@rfprefix\decrementreferenceprefix
     \let\referenceprefix\empty
     \let\@@rfprefix\s!unknown
   \else\ifx\@@rfprefix\s!unknown
     % forget about it
   \else
     \edef\referenceprefix{\@@rfprefix:}%
   \fi\fi\fi\fi}

%D \macros
%D   {handlereferenceactions,
%D    collectreferenceactions}
%D
%D Sometimes we need to pass the actions connected to
%D references to variables instead of rectangular areas on
%D which one can click. The next macro collects the actions
%D and passes them to a handle. This is a rather dreadfull
%D hack!
%D
%D \starttyping
%D \handlereferenceactions{references}\handle
%D \stoptyping
%D
%D So, \type {\handle} does the final job, which in for
%D instance the \PDF\ drivers comes down to doing something
%D with \type {\lastPDFaction}.

\newif\ifcollectreferenceactions

\def\handlereferenceactions#1#2%
  {\doifsomething{#1}
     {\bgroup
      \collectreferenceactionstrue
      \@EA\doifreferencefoundelse\@EA{#1}
        {\gotolocation{#1}{}\ifsecondaryreference\else#2\fi}
        {\unknownreference{#1}}%
      \egroup}}

%D The most straightforward way of retrieving references is
%D using \type{\ref}. Consider the reference:
%D
%D \startbuffer
%D \reference[my ref]{{Look}{Here}{I am}}
%D \stopbuffer
%D
%D \typebuffer
%D
%D \getbuffer
%D
%D We can ask for upto five reference components:
%D
%D \startbuffer
%D user page reference:  \ref[p][my ref]
%D text reference:       \ref[t][my ref]
%D real page reference:  \ref[r][my ref]
%D sub text reference:   \ref[s][my ref]
%D extra text reference: \ref[e][my ref]
%D \stopbuffer
%D
%D \typebuffer
%D
%D And get back:
%D
%D \startlines
%D \getbuffer
%D \stoplines

\def\ref{\dodoubleargument\doref}

\def\reftypep{\currentpagereference}
\def\reftypet{\currenttextreference}
\def\reftyper{\currentrealreference}
\def\reftypes{\currentsubtextreference}
\def\reftypee{\currentsubsubtextreference}

\def\doref[#1][#2]%
  {\ifsecondargument
     \doifreferencefoundelse{#2}
       {\doifundefinedelse{reftype#1}{\reftypep}{\getvalue{reftype#1}}}
       {\unknownreference{#2}\dummyreference}%
   \else
     \dummyreference
   \fi}

%D We can typeset a reference using \type{\in}, \type{\at} and
%D \type{\about} and goto specific locations using
%D \type{\goto}. The last one does not make that much sense in
%D a paper document. To complicate things, \PLAIN\ \TEX\ also
%D implements an \type {\in} but fortunately that one only
%D makes sense in math mode.

%D --- VANAF HIER NOG VERENGELSEN ---

%\let\donormalin  =\in
%\let\donormalover=\over % about/oppassen beter nederlands dan engels!
%
%\def\in%
%  {\ifmmode
%     \expandafter\donormalin
%   \else
%     \expandafter\doinatreference\expandafter\currenttextreference
%   \fi}

% we need to bypass math tokens

% \let\normalover \over

\definecommand in    {\doinatreference\currenttextreference}
\definecommand at    {\doinatreference\currentpagereference}
\definecommand about {\dospecialabout}
\definecommand from  {\dospecialfrom}
\definecommand over  {\dospecialabout} % needed here, else math problems

\unexpanded\def\dospecialabout[#1]%
  {\dontleavehmode
   \bgroup
   \def\thecurrentsubtextreference
     {\limitatetext\currentsubtextreference\@@rfwidth\unknown}%
  %\leaveoutervmode % no
   \@@rfleft
   \doifreferencefoundelse{#1}
     {\let\crlf\space
      \let\\\space
      \let\dogotofixed\dogotospace
      \dogotospace{\thecurrentsubtextreference}[#1]}
     {\unknownreference{#1}\dummyreference}%
   \@@rfright
   \referentieinfo{<}{#1}%
   \egroup}

%D Typesetting the reference is a bit more complicated than one
%D would at first sight expect. This is due to the fact that we
%D distinguish three (five) alternative calls:
%D
%D \placefigure
%D   [here][three calls]
%D   {Three alternatives reference calls.}
%D   {\startcombination[1*3]
%D      {\framed{\type{ \in }}} {a}
%D      {\framed{\type{ \at }}} {b}
%D      {\framed{\type{\goto}}} {c}
%D    \stopcombination}
%D
%D \startbuffer
%D \in figure[fig:three calls]
%D \in{figure}[fig:three calls]
%D \in figure a[fig:three calls]
%D \in{figure}{a}[fig:three calls]
%D figure~\in[fig:three calls]
%D \stopbuffer
%D
%D \typebuffer
%D
%D This turns up as:
%D
%D \startlines
%D \getbuffer
%D \stoplines
%D
%D The dual \type{{}} results in a split reference. In a
%D document meant for paper, one is tempted to use the last
%D (most straightforward) alternative. When a document is also
%D meant voor electronic distribution, the former alternatives
%D have preference, because everything between the \type{\in}
%D and~\type{[} becomes active (and when asked for, typeset
%D in a different color and typeface).

\def\doinatreference#1%
  {\doifnextcharelse[% {[}
     {\dodoinatreference{#1}{}}
     {\dodoinatreference{#1}}}

\def\dodoinatreference#1%
  {\def\dododoinatreference{\dodododoinatreference{#1}}%
   \futurelet\next\dododoinatreference}

% overloaded
%
% \def\dodododoinatreference#1#2#3[#4]%
%   {\ifx\next\bgroup
%      \dododododoinatreference{#1\ignorespaces#3}{#2}[#4]%
%    \else
%      \dododododoinatreference{#1}{#2#3}[#4]%
%    \fi}

%D We arrived at the last step. Before we do the typesetting,
%D we forget all previous (paragraph bound) settings and make
%D sure that we remain in horizontal mode. Next we choose
%D among the several representations.

% overloaded
%
% \def\dododododoinatreference#1#2[#3]%
%   {\dontleavehmode
%    \bgroup
%    \forgetall
%   %\leaveoutervmode
%    \doifreferencefoundelse{#3}
%      {\bgroup
%       \let\ignorespaces\empty % rather dirty but ok
%       \doifelsenothing{#1}
%         {\egroup\dosymbolreference{#1}{#2}[#3]}
%         {\egroup\dowantedreference{#1}{#2}[#3]}}
%      {\dounknownreference{#1}{#2}[#3]}%
%    \referentieinfo{<}{#3}%
%    \egroup}

%D The previously discussed setup macro lets us specify the
%D representation of references. A symbol reference does not
%D show the specific data, like the number of a figure, but
%D shows one of: \hbox {$^\goforwardcharacter$
%D $^\gobackwardcharacter$ $^\gonowherecharacter$}, depending
%D on the direction to go.

\def\dosymbolreference#1#2[#3]%
  {\bgroup
   \setupsymbolset[\@@iasymbolset]%
   \removelastskip
   \ifx\currentreferencespecial\empty
     \ifx\currentouterreference\empty
       \ifnum0\currentrealreference=\zerocount
         \ifhmode\strut\high{\symbol[\v!nowhere]}\fi
       \else\ifnum0\currentrealreference>\realpageno
         \dodosymbolreference{#2}{\high{\symbol[\v!next]}}%
       \else\ifnum0\currentrealreference<\realpageno
         \dodosymbolreference{#2}{\high{\symbol[\v!previous]}}%
       \else
         \ifhmode\strut\high{\symbol[\v!nowhere]}\fi
       \fi\fi\fi
     \else
       \gotoouterlocation{#3}{\showlocation{\high{\symbol[\v!somewhere]}}}%
     \fi
   \else
     \gotospeciallocation{#3}{\showlocation{\high{\symbol[\v!somewhere]}}}%
   \fi
   \egroup}

\def\dodosymbolreference#1#2%
  {#1\hbox{\gotorealpage\empty\empty\currentrealreference
    {\dolocationattributes\??ia\c!style\c!color{#2}}}}

%D The other alternatives just conform their names: only the
%D label, only the text, or the label and the text.

\def\dounknownreference#1#2[#3]%
  {\unknownreference{#3}\dotextprefix{#2}\dummyreference}%

% \def\docompletereference#1#2[#3]%
%   {\doifsomespaceelse{#2}
%      {\doifsomething{#2}{\dogotospace{#2}[#3]~}\dogotofixed{#1}[#3]}
%      {\dogotofixed{\dotextprefix{#2}#1}[#3]}}
%
% better:

% \def\docompletereference#1#2[#3]%
%   {\iflocationsplit
%       \doifsomespaceelse{#2}
%         {\dogotospace{\dotextprefix{#2}#1}[#3]}
%         {\dogotofixed{\dotextprefix{#2}#1}[#3]}%
%    \else
%      \dogotofixed{\dotextprefix{#2}#1}[#3]%
%    \fi}
%
% more efficient

\def\docompletereference#1#2[#3]%
  {\iflocationsplit
     \doifsomespaceelse{#2}\dogotospace\dogotofixed{\dotextprefix{#2}#1}[#3]%
   \else
     \dogotofixed{\dotextprefix{#2}#1}[#3]%
   \fi}

\def\dolabelonlyreference#1#2[#3]%
  {\doifsomespaceelse{#2}
     {\doifsomething{#2}{\dogotospace{#2}[#3]}}
     {\dogotofixed{\dotextprefix{#2}}[#3]}}

\def\dotextonlyreference#1#2[#3]%
  {\dotextprefix{#2}\dogotofixed{#1}[#3]}

\let\dowantedreference=\docompletereference

%D \macros
%D   {definereferenceformat}
%D
%D The next few macros were made for for David Arnold and Taco
%D Hoekwater. They can be used for predefining reference
%D texts, and thereby stimulate efficiency.
%D
%D [more documentation will be added]
%D
%D \starttyping
%D \definereferenceformat[informula]  [left=(,right=),text=formula]
%D \definereferenceformat[informulas] [left=(,right=),text=formulas]
%D \definereferenceformat[andformula] [left=(,right=),text=and]
%D \definereferenceformat[andformulas][left=(,right=),text=and]
%D
%D \informula [b] and \informula [for:c]
%D the \informula {formulas}[b] \informula {and} [for:c]
%D the \informulas {formulas}[b] \informula {and} [for:c]
%D the \informulas [b] \informula {en} [for:c]
%D the \informulas [b] \andformula [for:c]
%D \stoptyping
%D
%D Instead of a text, one can specify a label, which should
%D be defined with \type {\setuplabeltext}.

\def\definereferenceformat%
  {\dodoubleargument\dodefinereferenceformat}

\def\dodefinereferenceformat[#1][#2]%
  {\iffirstargument
     \getparameters[\??rf#1]
       [\c!left=,  % of the number
        \c!right=, % of the number
        \c!text=,  % before the number
        \c!label=,  % can be {left}{right}
        \c!command=\in,
        #2]%
     \unexpanded\setvalue{#1}%
       {\dontleavehmode\doexecutereferenceformat{#1}}%
   \fi}

\def\noexecutelabelreferenceformat#1%
  {\doifvaluesomething{\??rf#1\c!text}
     {\gdef\textofreference{\csname\??rf#1\c!text\endcsname}}%
   \csname\??rf#1\c!command\endcsname}

\def\doexecutelabelreferenceformat#1%
  {\csname\??rf#1\c!command\endcsname
     {\leftlabeltext {\csname\??rf#1\c!label\endcsname}}%
     {\rightlabeltext{\csname\??rf#1\c!label\endcsname}}}

\def\doexecutereferenceformat#1%
  {\gdef\leftofreference {\csname\??rf#1\c!left \endcsname}%
   \gdef\rightofreference{\csname\??rf#1\c!right\endcsname}%
   \global\let\textofreference\empty % otherwise ~ added
   \doifelsevaluenothing{\??rf#1\c!label}
     \noexecutelabelreferenceformat\doexecutelabelreferenceformat{#1}}

\let\leftofreference \relax
\let\rightofreference\relax
\let\textofreference \relax

\def\dodododoinatreference#1#2#3[#4]% \removeunwantedspaces added june 2004
  {\ifx\next\bgroup                 % but removed later, fails on metafun
     \dododododoinatreference
      % fails on metafun  {\leftofreference#1\ignorespaces#3\removeunwantedspaces\rightofreference}{#2}[#4]%
       {\leftofreference#1\ignorespaces#3\rightofreference}{#2}[#4]%
   \else
     \dododododoinatreference
       {\leftofreference#1\rightofreference}{#2#3}[#4]%
   \fi}

% \def\dododododoinatreference#1#2[#3]%
%   {\bgroup
%    \forgetall
%    \postponefootnotes
%    \leaveoutervmode
%    \doifreferencefoundelse{#3}
%      {\bgroup
%       \let\ignorespaces    \empty % rather dirty trick, but ok
%       \let\leftofreference \empty % the same, again ok
%       \let\rightofreference\empty % and once more
%       \def\textofreference {#2}%  % temporary value
%       \ifx\textofreference\empty  % simple expansion
%        %\doifelsenothing{#1}
%        %  {\egroup\dosymbolreference{#1}{\textofreference}[#3]}
%        %  {\egroup\dowantedreference{#1}{\textofreference}[#3]}%
%         \doifelsenothing{#1}%
%           {\egroup\dosymbolreference}%
%           {\egroup\dowantedreference}%
%           {#1}{\textofreference}[#3]%
%       \else
%        %\doifelsenothing{#1}
%        %  {\egroup\dosymbolreference{#1}{#2}[#3]}
%        %  {\egroup\dowantedreference{#1}{#2}[#3]}%
%         \doifelsenothing{#1}%
%           {\egroup\dosymbolreference}%
%           {\egroup\dowantedreference}%
%           {#1}{#2}[#3]%
%       \fi}
%      {\dounknownreference{#1}{#2}[#3]}%
%    \referentieinfo<{#3}%
%    \global\let\leftofreference \empty
%    \global\let\rightofreference\empty
%    \global\let\textofreference \empty
%    \egroup}

\def\dododododoinatreference#1#2[#3]%
  {\dontleavehmode   % replaces \leaveoutervmode
   \bgroup
   \forgetall
   \postponefootnotes
   %\leaveoutervmode % replaced by \dontleavehmode
   \doifreferencefoundelse{#3}
     {\bgroup
      \let\ignorespaces    \empty % rather dirty trick, but ok
      \let\leftofreference \empty % the same, again ok
      \let\rightofreference\empty % and once more
      \def\textofreference {#2}%  % temporary value
      \ifx\textofreference\empty  % simple expansion
       %\doifelsenothing{#1}
       %  {\egroup\dosymbolreference{#1}{\textofreference}[#3]}
       %  {\egroup\dowantedreference{#1}{\textofreference}[#3]}%
        \doifelsenothing{#1}%
          {\egroup\dosymbolreference}%
          {\egroup\dowantedreference}%
          {#1}{\textofreference}[#3]%
      \else
       %\doifelsenothing{#1}
       %  {\egroup\dosymbolreference{#1}{#2}[#3]}
       %  {\egroup\dowantedreference{#1}{#2}[#3]}%
        \doifelsenothing{#1}%
          {\egroup\dosymbolreference}%
          {\egroup\dowantedreference}%
          {#1}{#2}[#3]%
      \fi}
     {\dounknownreference{#1}{#2}[#3]}%
   \referentieinfo<{#3}%
   \global\let\leftofreference \empty
   \global\let\rightofreference\empty
   \global\let\textofreference \empty
   \egroup}

%D In interactive documents going to a specific location is not
%D bound to cross references. The \type{\goto} commands can be
%D used to let users access another part of the document. In
%D this respect, interactive tables of contents and registers
%D can be considered goto's. Because in fact a \type{\goto} is
%D just a reference without reference specific data, the
%D previous macros are implemented using the goto
%D functionality.
%D
%D \showsetup{goto}
%D
%D One important chaacteristic is that the first argument of
%D \type{\goto} (and therefore \type{\at} and \type{\in} is
%D split at spaces. This means that, although hyphenation is
%D prevented, long references can cross line endings.

%\unexpanded\def\naar#1[#2]%
%  {\dontleavehmode
%  %\leaveoutervmode
%   \doifreferencefoundelse{#2}
%     {\doifelsenothing{#1}
%        {\dosymbolreference{}{}[#2]}
%        {\dogotospace{#1}[#2]}}
%     {\unknownreference{#2}#1}%
%   \referentieinfo{<}{#2}}

\def\dogoto#1[#2]%
  {\dontleavehmode
  %\leaveoutervmode
\bgroup
\postponefootnotes
   \doifreferencefoundelse{#2}
     {\doifelsenothing{#1}
        {\dosymbolreference{}{}[#2]}
        {\dogotospace{#1}[#2]}}
     {\unknownreference{#2}#1\relax}% \relax catches lookahead
\egroup
   \referentieinfo{<}{#2}}

\unexpanded\def\goto#1#2%
  {\dogoto{#1}#2}

\newif\ifsharesimilarreferences \sharesimilarreferencestrue
\newcount\similarreference % 0=noppes 1=create/refer 2,3,..=refer

\def\dogotospace#1[#2]%
  {\iflocationsplit
     \ifsecondaryreference\setbox0\hbox\fi % due to space insertion
       {\let\dogotospace\dogotofixed
        \iflocation
          \def\processisolatedword##1%
            {\ifisolatedwords\ifsharesimilarreferences
               \global\advance\similarreference \plusone
             \fi\fi
             \hbox{\gotolocation{#2}{##1\presetgoto}}}%
          \doattributes\??ia\c!style\c!color
            {\processisolatedwords{#1}\processisolatedword}%
        \else
          #1\relax % \relax prevents #1's next macros from gobbling \fi
        \fi}%
   \else
     \iflocation
       \doattributes\??ia\c!style\c!color
         {\gotolocation{#2}{#1\presetgoto}}%
     \else
       #1\relax % \relax prevents #1's next macros from gobbling \fi
     \fi
   \fi
   \global\similarreference\zerocount}

\def\dogotofixed#1[#2]%
  {{\iflocation
      \hbox{\gotolocation{#2}{\doattributes\??ia\c!style\c!color
        {#1\presetgoto}}}%
    \else
      #1%
    \fi}}

%D In case the auto split feature is not needed or even not
%D even wanted, \type{\gotobox} can be used.

%D --- NOG IN HANDLEIDING ---

\unexpanded\def\gotobox#1[#2]%
  {\dontleavehmode
   \bgroup
   \locationstrutfalse
  %\leaveoutervmode
   \doifreferencefoundelse{#2}
     {\dogotofixed{#1}[#2]}
     {\hbox{\unknownreference{#2}#1}}%
   \referentieinfo{<}{#2}%
   \egroup}

%D An reference to another document can be specified as a file
%D or as an \URL. Both are handled by the same mechanism and
%D can be issued by saying something like:
%D
%D \starttyping
%D \goto[dictionary::the letter a]
%D \stoptyping
%D
%D The macros that are responsible for handling these
%D references, use the next six variables:

\let\otherlabel   = \empty
\let\fileprefix   = \empty
\def\otherfile      {\jobname}
\let\otherURL     = \empty
\let\otherprefix  = \empty
\let\dowithdocdes = \empty

%D One can imagine that many references to such a dictionary
%D are made, so in most cases such a document reference in an
%D indirect one.
%D
%D \showsetup{useexternaldocument}
%D
%D For example:
%D
%D \starttyping
%D \useexternaldocument
%D   [dictionary][engldict]
%D   [The Famous English Dictionary]
%D \stoptyping
%D
%D The next macro implements these relations, and also take
%D care of loading the document specific references.

\def\useexternaldocument%
  {\dotripleargument\douseexternaldocument}

\def\douseexternaldocument[#1][#2][#3]%
  {\bgroup
   \ifsecondargument
     \doifelsenothing{#1}
       {\douseexternaldocument[#2][#2][#3]}
       {\doifelsenothing{#3}
          {\douseexternaldocument[#1][#2][#2]}
          {\doifsomething{#2}
             {\setgvalue{\v!file:::#1}{\doexternaldocument{}{#2}{#3}}% just \do
              \doif\@@rfstate\v!start
                {\doifparentfileelse{#2}
                   {\showmessage\m!references{21}{#2}}
                   {\dodouseexternaldocument{#1}{#2}}}}}}%
   \else
     \dodouseexternaldocument{#1}{#1}%
   \fi
   \egroup}

\def\dodouseexternaldocument#1#2%
  {\bgroup % prevents wrong loading of \jobname
     \def\fileprefix{#1::}%
     \let\setglobalcrossreference\setoutercrossreference
     \usereferences[#2]%
   \egroup % when called nested
   \showmessage\m!references{21}{#2}}

%D The \URL\ alternative takes four arguments:
%D
%D \showsetup{useURL}
%D
%D like:
%D
%D \starttyping
%D \useURL
%D   [dictionary][http://www.publisher.com/public][engldict]
%D   [The Famous English Dictionary]
%D \stoptyping
%D
%D Several specifications are possible:
%D
%D \starttyping
%D \useURL [id] [url] [file] [description]
%D \useURL [id] [url] [file]
%D \useURL [id] [url]
%D \stoptyping
%D
%D This time we don't load the references when no file is
%D specified. This is logical when one keeps in mind that a
%D valid \URL\ can also be a mail address.

\def\useURL
  {\bgroup
   \protectlabels
   \catcode`\#=\@@other\catcode`\%=\@@other\catcode`\/=\@@other
   \catcode`\_=\@@other\catcode`\~=\@@other\catcode`\:=\@@other
   \dodoubleempty\douseURL}

\def\douseURL[#1][#2]%
  {\egroup\doquadrupleempty\dodouseURL[#1][#2]}

\let\useurl\useURL

% \def\dodouseURL[#1][#2][#3][#4]%
%   {\iffirstargument
%      \iffourthargument\setgvalue{\v!file:::#1}{\doexternaldocument{#2}{#3}{#4}}\else
%      \ifthirdargument \setgvalue{\v!file:::#1}{\doexternaldocument{#2}{#3}{\url[#1]}}\else
%      \ifsecondargument\setgvalue{\v!file:::#1}{\doexternaldocument{#2}{}{\url[#1]}}\fi\fi\fi
%    \fi}

\def\dodouseURL[#1][#2][#3][#4]% to be redone: not too tricky redefs ad reuse
  {\iffirstargument
     \iffourthargument\setgvalue{\v!file:::#1}{\doexternaldocument{#2}{#3}{#4}}\else
     \ifthirdargument \setgvalue{\v!file:::#1}{\doexternalurl     {#2}{#3}{#1}}\else
     \ifsecondargument\setgvalue{\v!file:::#1}{\doexternalurl     {#2}{}  {#1}}\fi\fi\fi
   \fi}

\def\doexternalurl#1#2#3%
  {\bgroup
   \doifsomething\@@urstyle{\let\@@iastyle\@@urstyle\let\@@urstyle\empty}%
   \doifsomething\@@urcolor{\let\@@iacolor\@@urcolor\let\@@urcolor\empty}%
   \doexternaldocument{#1}{#2}{\url[#3]}%
   \egroup}

%D \macros
%D   {url,setupurl}
%D
%D We also have: \type{\url} for directly calling the
%D description. So we can say:
%D
%D \starttyping
%D \useURL [one] [http://www.test.nl]
%D \useURL [two] [http://www.test.nl] [] [Some Site]
%D
%D \url[one] or \from[two] or \goto{Whatever Site}[URL(two)]
%D \stoptyping
%D
%D An \URL\ can be set up with
%D
%D \showsetup{setupurl}

\def\setupurl
  {\dodoubleargument\getparameters[\??ur]}

\unexpanded\def\url[#1]% slow
  {\bgroup
   \processaction
     [\@@uralternative]
     [  \v!none=>\chardef\urlsplitmode\zerocount,
        \v!both=>\chardef\urlsplitmode\plusone,
       \v!after=>\chardef\urlsplitmode\plustwo,
      \v!before=>\chardef\urlsplitmode\plusthree]%
   \doifelse\@@urspace\v!yes
     {\setbetweenisolatedwords{\scratchskip\currentspaceskip\hskip\zeropoint\!!plus.2\scratchskip}}
     {\setbetweenisolatedwords\allowbreak}%
   \def\doexternaldocument##1##2##3{\hyphenatedurl{##1}}% awful hack
   \dostartattributes\??ur\c!style\c!color{}%
   \getvalue{\v!file:::#1}%
   \dostopattributes
   \egroup}

%D This macro is hooked into a support macro, and thereby
%D \URL's break ok, according to the setting of a switch,
%D
%D \startbuffer
%D \useURL
%D   [test]
%D   [sentence_sentence%sentence#sentence~sentence/sentence//sentence:sentence.sentence]
%D \stopbuffer
%D
%D \typebuffer
%D
%D Such an \URL\ is, depending on the settings, hyphenated as:
%D
%D \getbuffer
%D
%D \startlinecorrection
%D \hbox to \hsize
%D   {\hss\en
%D    \setupreferencing[urlalternative=both]%
%D    \vbox{\hsize.25cm\hbox{\bf both}\prewordbreak\url[test]}%
%D    \hss
%D    \setupreferencing[urlalternative=before]%
%D    \vbox{\hsize.25cm\hbox{\bf before}\prewordbreak\url[test]}%
%D    \hss
%D    \setupreferencing[urlalternative=after]%
%D    \vbox{\hsize.25cm\hbox{\bf after}\prewordbreak\url[test]}%
%D    \hss}
%D \stoplinecorrection
%D
%D By setting \type{urlspace=yes} one can get slightly better
%D spacing when using very long \URL's.

%D Many macro definitions ago we called for the auxiliary macro
%D \type {\setouterlocation} and now is the time to define this
%D one.

\newconditional\forceURLlocation

\def\setouterfilelocation#1#2#3%
  {\edef\otherURL{#1}%
   \edef\otherfile{#2}}%

\def\setouterlocation#1%
  {\ifundefined{\v!file:::#1}%
     \ifconditional\forceURLlocation
       \edef\otherURL{#1}%
       \let\otherfile\empty
     \else
       \let\otherURL\empty
       \edef\otherfile{#1}%
     \fi
   \else
     \let\doexternaldocument\setouterfilelocation % will change
     \let\doexternalurl     \setouterfilelocation % will change
     \csname\v!file:::#1\endcsname
   \fi
   \setfalse\forceURLlocation
   \doifparentfileelse\otherfile
     {\let\otherURL\empty
      \let\otherfile\empty
      \global\let\otherlabel\empty
      \let\otherprefix\empty}
     {\xdef\otherlabel{#1}%
      \edef\otherprefix{#1::}}}

%D When defining the external source of information, one can
%D also specify a suitable name (the last argument). This name
%D can be called upon with:
%D
%D \showsetup{from}
%D
%D As can be expected, this macro used \type{\goto} to
%D perform its task.

% normal version:
%
% \def\from[#1]%
%   {\dontleavehmode % added, but probably not needed
%    \bgroup
%    \doifinstringelse{::}{#1}{\docomplexuit[#1]}{\docomplexuit[#1::]}}
%
% since we need to serve active :'s we need the less readable:

\def\dospecialfrom % retest this one !
  {\dosingleempty\dodospecialfrom}

\def\dodospecialfrom[#1]%
  {\dontleavehmode % added, but probably not needed
   \bgroup
   \protectlabels % needed for active french :'s
   \iffirstargument
     \edef\!!stringa{#1}%
     \doifincsnameelse{::}\!!stringa\donothing{\edef\!!stringa{#1::}}%
     \expanded{\redospecialfrom[\!!stringa]}%
   \else
     \expanded{\nodospecialfrom[\otherlabel]}%
   \fi
   \egroup}

\def\redospecialfrom[#1::#2]%
  {\ifundefined{\v!file:::#1}%
     \tttf[#1]%
   \else
     \def\doexternaldocument##1##2##3{\goto{##3}[#1::#2]}%
     \csname\v!file:::#1\endcsname
   \fi}

\def\nodospecialfrom[#1]%
  {\ifundefined{\v!file:::#1}%
     \tttf[#1]%
   \else
     \def\doexternaldocument##1##2##3{##3}% different than ^
     \csname\v!file:::#1\endcsname
   \fi}

%D We also support:
%D
%D \starttyping
%D \goto{some text}[file(identifier{location}]
%D \stoptyping
%D
%D which is completely equivalent with
%D
%D \starttyping
%D \goto{some text}[identifier::location]
%D \stoptyping
%D
%D The fastest implementation would be:

\definespecialtest\v!file    {\setfalse\forceURLlocation\handlespecialFILEandURL}
\definespecialtest\v!URL     {\settrue \forceURLlocation\handlespecialFILEandURL}
\definespecialtest\v!url     {\settrue \forceURLlocation\handlespecialFILEandURL}

\definespeciallocation\v!file{\setfalse\forceURLlocation\handlespecialallocationFILEandURL}
\definespeciallocation\v!URL {\settrue \forceURLlocation\handlespecialallocationFILEandURL}
\definespeciallocation\v!url {\settrue \forceURLlocation\handlespecialallocationFILEandURL}

\def\handlespecialFILEandURL
  {\localdoifreferencefoundelse
     {\currentreferenceoperation::\currentreferencearguments}}

\def\handlespecialallocationFILEandURL
  {\let\currentouterreference\currentreferenceoperation
   \let\currentinnerreference\currentreferencearguments
   \let\currentreferenceoperation\empty
   \let\currentreferencearguments\empty
   \gotoouterlocation}

%D Now we have file references as special ones, it's rather
%D logical to have the viewer specific ones available in a dual
%D way too. At first glance we could do with:
%D
%D \starttyping
%D \definespeciallocation\v!action
%D   {\getreferenceelements\currentreferenceoperation
%D    \handleexecreference}
%D \stoptyping
%D
%D An better alternative, slower but error aware, is

% \definespecialtest\v!actie
%   {\localdoifreferencefoundelse\currentreferenceoperation}

\definespecialtest\v!action % rather ugly action(whatever{argument})
  {\expanded{\localdoifreferencefoundelse{\currentreferenceoperation
     \ifx\currentreferencearguments\empty\else{\currentreferencearguments}\fi}}}

\definespeciallocation\v!action
  {\handleexecreference}

%D So now we can say:
%D
%D \starttyping
%D \goto{some action}[PreviousJump]
%D \stoptyping
%D
%D as well as:
%D
%D \starttyping
%D \goto{some text}[action(PreviousJump]
%D \stoptyping

%D A special case of references are those to programs. These,
%D very system dependant references are implemented by abusing
%D some of the previous macros.
%D
%D \showsetup{setupprograms}
%D \showsetup{defineprogram}
%D \showsetup{program}
%D
%D The latter gives access to the description of the program,
%D being the last argument to the definition command.

\def\setupprograms
  {\dodoubleargument\getparameters[\??pr]}

\def\dodefineprogram[#1][#2][#3]%
  {\setgvalue{\v!program:::#1}{\doprogram{#2}{#3}}}

\def\defineprogram
  {\dotripleargument\dodefineprogram}

\def\program#1[#2]%
  {\bgroup
   \ifundefined{\v!program:::#2}%
     {\tttf[#2]}%
   \else
     \def\doprogram##1##2%
       {\goto{\doifelsenothing{#1}{##2}{#1}}[\v!program(#2)]}%
     \csname\v!program:::#2\endcsname
   \fi
   \egroup}

% needs an update: program(abc{arg})

\definespeciallocation\v!program#1#2%
  {\bgroup
   \iflocation
     \doifdefinedelse{\v!program:::\currentreferenceoperation}
       {\def\doprogram##1##2{\def\@@programfile{##1}}%
        \getvalue{\v!program:::\currentreferenceoperation}}
       {\let\@@programfile\currentreferenceoperation}%
     \convertcommand\@@programfile\to\ascii
     \dostartgoto
       \data
         {#2}%
       \start
         \dostartrunprogram\buttonwidth\buttonheight
           {\@@prdirectory\ascii}\currentreferencearguments
       \stop
         \dostoprunprogram
     \dostopgoto
   \else
     {#2}%
   \fi
   \egroup}

%D As we can see, we directly use the special reference
%D mechanism, which means that
%D
%D \starttyping
%D \goto{some text}[program(name{args})]
%D \stoptyping
%D
%D is valid.

%D The next macro provides access to the actual pagenumbers.
%D When documenting and sanitizing the original reference
%D macros, I decided to keep the present meaning as well as to
%D make this meaning available as a special reference method.
%D So now one can use:
%D
%D \starttyping
%D \gotopage{some text}[location]
%D \gotopage{some text}[number]
%D \gotopage{some text}[file::number]
%D \stoptyping
%D
%D as well as:
%D
%D \starttyping
%D \goto{some text}[page(location)]
%D \goto{some text}[page(number)]
%D \goto{some text}[file::page(number)]
%D \stoptyping
%D
%D Here location is a keyword like \type{nextpage}.
%D
%D \showsetup{gotopage}

\def\dodefinepage[#1][#2]%
  {\setvalue{\v!page:::#1}{#2}}

\def\definepage
  {\dodoubleargument\dodefinepage}

\definepage [\v!firstpage]      [\firstpage]
\definepage [\v!previouspage]      [\prevpage]
\definepage [\v!nextpage]    [\nextpage]
\definepage [\v!lastpage]     [\lastpage]
\definepage [\v!firstsubpage]   [\firstsubpage]
\definepage [\v!previoussubpage]   [\prevsubpage]
\definepage [\v!nextsubpage] [\nextsubpage]
\definepage [\v!lastsubpage]  [\lastsubpage]
\definepage [\v!first]            [\firstpage]
\definepage [\v!previous]            [\prevpage]
\definepage [\v!next]          [\nextpage]
\definepage [\v!last]           [\lastpage]
\definepage [\v!first\v!sub]      [\firstsubpage]
\definepage [\v!previous\v!sub]      [\prevsubpage]
\definepage [\v!next\v!sub]    [\nextsubpage]
\definepage [\v!last\v!sub]     [\lastsubpage]

%D Because we combine both methods, we have to take care of
%D the \type{file::page(n)} as well as \type{page(file::n)}.

% \definespeciallocation\v!pagina#1#2%
%   {\iflocation
%      \ifx\currentouterreference\empty
%        \splitoffreference\currentreferenceoperation
%      \else
%        \let\currentinnerreference=\currentreferenceoperation
%      \fi
%      \ifx\currentouterreference\empty
%        \doifnonzeropositiveelse{\currentinnerreference}
%          {}
%          {\edef\currentinnerreference{1}}%
%        \gotorealpage\empty\empty\currentinnerreference{#2}%
%      \else
%        \setouterlocation\currentouterreference
%        \doifnonzeropositiveelse{\currentinnerreference}
%          {}
%          {\doifdefinedelse{\v!pagina:::\currentinnerreference}
%             {\edef\currentinnerreference{\getvalue{\v!pagina:::\currentinnerreference}}}
%             {\edef\currentinnerreference{1}}}%
%        \gotorealpage\otherURL\otherfile\currentinnerreference{#2}%
%      \fi
%    \else
%      {#2}%
%    \fi}

% \definespeciallocation\v!pagina#1#2% page(n) page(+n) page(-n)
%   {\iflocation
%      \ifx\currentouterreference\empty
%        \splitoffreference\currentreferenceoperation
%      \else
%        \let\currentinnerreference\currentreferenceoperation
%      \fi
%      \ifx\currentouterreference\empty
%        \doifinstringelse+\currentinnerreference
%          {\scratchcounter\realpageno
%           \advance\scratchcounter \currentinnerreference
%           \edef\currentinnerreference{\the\scratchcounter}}
%          {\doifinstringelse-\currentinnerreference
%             {\scratchcounter\realpageno
%              \advance\scratchcounter \currentinnerreference
%              \edef\currentinnerreference{\the\scratchcounter}}
%             \donothing}%
%        \doifnonzeropositiveelse\currentinnerreference
%          \donothing
%          {\edef\currentinnerreference{1}}%
%        \gotorealpage\empty\empty\currentinnerreference{#2}%
%      \else
%        \setouterlocation\currentouterreference
%        \doifnonzeropositiveelse\currentinnerreference
%          \donothing
%          {\doifdefinedelse{\v!pagina:::\currentinnerreference}
%             {\edef\currentinnerreference{\getvalue{\v!pagina:::\currentinnerreference}}}
%             {\edef\currentinnerreference{1}}}%
%        \gotorealpage\otherURL\otherfile\currentinnerreference{#2}%
%      \fi
%    \else
%      {#2}%
%    \fi}

\definespeciallocation\v!page#1#2% page(n) page(+n) page(-n)
  {\iflocation
     \ifx\currentouterreference\empty
       \splitoffreference\currentreferenceoperation
     \else
       \let\currentinnerreference\currentreferenceoperation
     \fi
     \ifx\currentouterreference\empty
       \doifinstringelse+\currentinnerreference
         {\scratchcounter\realpageno
          \advance\scratchcounter \currentinnerreference
          \edef\currentinnerreference{\the\scratchcounter}}
         {\doifinstringelse-\currentinnerreference
            {\scratchcounter\realpageno
             \advance\scratchcounter \currentinnerreference
             \edef\currentinnerreference{\the\scratchcounter}}
            \donothing}%
       \doifnonzeropositiveelse\currentinnerreference
         \donothing
         {\edef\currentinnerreference{1}}%
       \docheckrealreferencepage\currentinnerreference % new
       \let\currentrealreference\currentinnerreference % handy to have this available
       \gotorealpage\empty\empty\currentinnerreference{#2}%
     \else
       \setouterlocation\currentouterreference
       \doifnonzeropositiveelse\currentinnerreference
         \donothing
         {\doifdefinedelse{\v!page:::\currentinnerreference}
            {\edef\currentinnerreference{\getvalue{\v!page:::\currentinnerreference}}}
            {\edef\currentinnerreference{1}}}%
       \gotorealpage\otherURL\otherfile\currentinnerreference{#2}%
     \fi
   \else
     {#2}%
   \fi}

\def\gotopage#1[#2]%
  {\goto{#1}[\v!page(#2)]}

%D A still very rudimentary|/|experimental forward|/|backward
%D reference mechanism is provided by the macro \type{\atpage}:
%D
%D \starttyping
%D ... \somewhere{backward text}{forward text}[someref] ...
%D ... \atpage[someref] ...
%D \stoptyping
%D
%D In future versions there will be more sophisticated

%D support, also suitable for references to floating bodies.

\unexpanded\def\somewhere#1#2#3[#4]% #3 gobbles space around #2
  {\dontleavehmode
  %\leaveoutervmode
   \doifreferencefoundelse{#4}
     {\ifforwardreference
        \doifelsenothing{#1}
          {\dosymbolreference{}{}[#4]}
          {\dogotospace{#1}[#4]}%
      \else
        \doifelsenothing{#2}
          {\dosymbolreference{}{}[#4]}
          {\dogotospace{#2}[#4]}%
      \fi}
     {\unknownreference{#4}#1/#2}%
   \referentieinfo{<}{#4}}

\unexpanded\def\atpage[#1]%
  {\dontleavehmode
  %\leaveoutervmode
   \doifreferencefoundelse{#1}
     {\ifrealreferencepage
        \ifforwardreference
          \dogotofixed{\labeltext\v!hencefore}[#1]%
        \else
          \dogotofixed{\labeltext\v!hereafter}[#1]%
        \fi
      \else
        \dogotofixed{\labeltexts\v!atpage\currentpagereference}[#1]%
      \fi}
     {\unknownreference{#1}%
      \labeltexts\v!page\dummyreference}%
   \referentieinfo{<}{#1}}

%D We can cross link documents by using:
%D
%D \showsetup{coupledocument}
%D
%D like:
%D
%D \starttyping
%D \coupledocument[print][somefile][chapter,section]
%D \stoptyping
%D
%D After which when applicable, we have available the
%D references:
%D
%D \starttyping
%D \goto{print version}[print::chapter]
%D \stoptyping
%D
%D and alike. The title placement definition macros have a
%D key \type{file}, which is interpreted as the file to jump
%D to, that is, when one clicks on the title.

\let\crossdocumentreferences\empty
\let\crossdocumentelements\empty

\newif\ifautocrossdocument

\def\docoupledocument[#1][#2][#3][#4]% is this :/- safe ?
  {\ifthirdargument
     \begingroup
     \def\dolistelement##1##2##3##4##5##6% 2=aut 6=pag / 2 goes into text ref slot
       {\global\utilitydonetrue %{Watch the braces here below!}
        \setglobalcrossreference{{##1::\@@filterblocknumberpart[##5]}}{}{##6}{##2}}%
     \def\usereferences[##1]%
      %{\setbox0\vbox{\doutilities{#3}{##1}{#3}\relax\relax}}%
       {\startnointerference
        \doutilities{#3}{##1}{#3}\relax\relax
        \stopnointerference}%
     \douseexternaldocument[#1][#2][#4]%
     \doglobal\addtocommalist{#1}\crossdocumentreferences
     \def\docommando##1%
       {\letgvalue{\??rf##1\c!state}\v!start % for fast checking
        \doglobal\addtocommalist{##1}\crossdocumentelements}%
     \processcommalist[#3]\docommando
     \ifutilitydone
       \global\autocrossdocumenttrue
     \fi
     \endgroup
   \fi}

\def\coupledocument%
  {\doquadrupleempty\docoupledocument}

%D --- STRANGE HERE, BETTER IN CORE-NAV ---

\def\checkcontrastreference#1%
  {\ifnum\currentreferencetype=\rt!page\ifnum\currentdatareference=\realpageno
     \doifdefined{#1\c!contrastcolor}
       {\setevalue{#1\c!color}{\getvalue{#1\c!contrastcolor}}}%
   \fi\fi}

\def\checkcontrastreference#1%
  {\ifnum\currentreferencetype=\rt!page\relax\ifnum\currentdatareference=\realpageno
     \copycsname#1\c!color\endcsname\csname#1\c!contrastcolor\endcsname
   \fi\fi}

%D Buttons are just what their names says: things that can be
%D clicked (pushed) on. They are similar to \type{\goto},
%D except that the text argument is not interpreted.
%D Furthermore one can apply anything to them that can be done
%D with \type{\framed}.
%D
%D \startbuffer
%D \button[width=3cm,height=1.5cm]{Exit}[ExitViewer]
%D \stopbuffer
%D
%D \typebuffer
%D
%D gives
%D
%D \getbuffer
%D
%D This command is formally specified as:
%D
%D \showsetup{button}
%D
%D The characteristics can be set with:
%D
%D \showsetup{setupbuttons}

\def\setupbuttons
  {\dodoubleargument\getparameters[\??bt]}

\definecomplexorsimpleempty\button

\def\complexbutton
  {\docomplexbutton\??bt}

\presetlocalframed[\??bt]

\long\def\docomplexbutton#1[#2]#3#4% get rid of possible space before [#4]
  {\dodocomplexbutton#1[#2]{#3}#4} % #4 == [

\def\buttonframed{\dodoubleempty\localframed[\??bt]} % goodie

\long\def\dodocomplexbutton#1[#2]#3[#4]% #3 can contain [] -> {#3} later
  {\bgroup
   \doifvalue{#1\c!state}\v!stop\locationfalse
   \iflocation
     \resetgoto
     \ConvertConstantAfter\doifelse{#3}\v!none\hphantom\hbox
       {\doifelsenothing{#4}
          {\setlocationboxnop#1[#2]{#3}[#4]}
          {\doifreferencefoundelse{#4}
             {\setlocationboxyes#1[#2]{#3}[#4]}
             {\unknownreference{#4}%
              \setlocationboxnop#1[#2]{#3}[#4]}}}%
   \fi
   \egroup}

%D Interaction buttons, in fact a row of tiny buttons, are
%D typically only used for navigational purposed. The next
%D macro builds such a row based on a specification list.
%D
%D \startbuffer
%D \interactionbuttons
%D   [width=\hsize][page,PreviousJump,ExitViewer]
%D \stopbuffer
%D
%D \typebuffer
%D
%D gives
%D
%D \getbuffer
%D
%D Apart from individual entries, one can use \type{page} and
%D \type {subpage} as shortcuts to their four associated buttons.
%D The symbols are derived from the symbols linked to the
%D entries.

% does not work well with for instance SomeRef{whatever}

\def\interactionbuttons
  {\dodoubleempty\dointeractionbuttons}

\def\dointeractionbuttons[#1][#2]% er is een verdeel macro \horizontalfractions
  {\iflocation
     % BUG: fails when frame=off; best is to rewrite this macro
     \bgroup
     \doif\@@ibstate\v!stop\locationfalse
     \iflocation
       \ifsecondargument
         \setupinteractionbar[#1]%
         \checkinteractionbar{1.5em}\v!broad\!!zeropoint % brrrrr
         \setbox2\hbox
           {\localframed[\??ib][\c!background=]{\symbol[\@@iasymbolset][\v!previouspage]}}%
         \!!heighta\ht2 % needed because we default to nothing
         \setupinteractionbar[\c!strut=\v!no]%
         \setinteractionparameter\c!width\!!zeropoint
         \!!counta\zerocount % new, was 1
         \processallactionsinset
           [#2]
           [   \v!page=>\advance\!!counta 4,
            \v!subpage=>\advance\!!counta 4,
            \s!unknown=>\advance\!!counta 1]%
         \ifdim\@@ibwidth=\zeropoint
           \!!widtha2em
           \advance\!!widtha \@@ibdistance  % new
           \!!widthb\!!counta\!!widtha
           \advance\!!widthb -\@@ibdistance % new
         \else
           \!!widtha\@@ibwidth
           \!!widthb\@@ibdistance           % new
           \multiply\!!widthb \!!counta     % new
           \advance\!!widthb -\@@ibdistance % new
           \advance\!!widtha -\!!widthb     % new
           \divide\!!widtha \!!counta
           \!!widthb\@@ibwidth
         \fi
         \def\goto##1% clash ?
           {\setnostrut
            \edef\localreference{##1}%
            \expanded{\dodocomplexbutton\??ib[\c!height=\the\!!heighta,\c!width=\the\!!widtha]}%
              {\dontleavehmode\symbol[\@@iasymbolset][\localreference]}%
              [\localreference]%
            \hss}%
         \hbox to \!!widthb
           {\processallactionsinset
              [#2]
              [      \v!page=>\goto\v!firstpage
                              \goto\v!nextpage
                              \goto\v!previouspage
                              \goto\v!lastpage,
                  \v!subpage=>\goto\v!firstsubpage
                              \goto\v!nextsubpage
                              \goto\v!previoussubpage
                              \goto\v!lastsubpage,
                  \s!unknown=>\goto\commalistelement]%
            \unskip}%
       \else
         \interactionbuttons[][#1]%
       \fi
     \fi
     \egroup
   \fi}

%D \macros
%D   {overlaybutton}
%D
%D For converience we provide:
%D
%D \starttyping
%D \overlaybutton[reference]
%D \stoptyping
%D
%D This command can be used to define overlays an/or can be
%D used in the whatevertext areas, like:
%D
%D \starttyping
%D \defineoverlay[PrevPage][\overlaybutton{PrevPage}]
%D \setupbackgrounds[page][background=PrevPage]
%D \setuptexttexts[\overlaybutton{NextPage}]
%D \stoptyping
%D
%D For practical reasons, this macro accepts square brackets
%D as well as braces.

\definecomplexorsimple\overlaybutton

\def\simpleoverlaybutton#1%
  {\complexoverlaybutton[#1]}

\def\complexoverlaybutton[#1]%
  {\iflocation
     \doifreferencefoundelse{#1}
       {\overlayfakebox  {#1}}
       {\unknownreference{#1}}%
   \fi}

\def\overlayfakebox#1%
  {\hbox
     {\setbox\scratchbox\null
      \wd\scratchbox\overlaywidth
      \ht\scratchbox\overlayheight
      \locationstrutfalse
      \gotolocation{#1}{\box\scratchbox\presetgoto}}}

%D \macros
%D   {dotextprefix}
%D
%D In previous macros we used \type {\dotextprefix} to
%D generate a space between a label and a number.
%D
%D \starttyping
%D \dotextprefix{text}
%D \stoptyping
%D
%D Only when \type {text} is not empty, a space is inserted.

\def\dotextprefix#1%
  {\bgroup
   \global\labeltextdonefalse  % this is an ugly dependancy,
   \setbox\scratchbox\hbox{#1}% to be solved some day
   \ifdim\wd\scratchbox>\zeropoint
     \unhbox\scratchbox
     \iflabeltextdone\else\@@rfseparator\fi
   \else
     \unhbox\scratchbox
   \fi
   \egroup}

%D In the next settings we see some variables that were not
%D used here and that concern the way the pagenumbers refered
%D to are typeset.

\setupreferencing
  [\c!state=\v!start,
   \c!autofile=\v!no,
   \v!part\c!number=\v!yes,
   \v!chapter\c!number=\v!no,
   \c!interaction=\v!all,
  %\c!urlalternative=\v!both,
  %\c!urlspace=\v!no,
  %\c!urlletter=,
  %\c!urlkleur=,
   \c!convertfile=\v!no,
  %\c!strut=\v!no, % some day an option
   \c!prefix=,
   \c!width=.75\makeupwidth,
   \c!left=\quotation\bgroup,
   \c!right=\egroup,
   \c!global=\v!no,
   \c!expansion=\v!no,
   \c!separator=\nonbreakablespace]

\let\nonbreakablespace=~

\setupurl
  [\c!alternative=\v!both,
   \c!space=\v!no,
   \c!style=\v!type,
   \c!color=]

\setupprograms
  [\c!directory=]

%D We cannot set up buttons (not yet, this one calls a menu macro):

% under consideration:
%
% \setupinteraction[state=start]
%
% \unprotect
%
% \chardef\rt!extern=5
%
% \definesystemreferencehandler \rt!extern \handleexecreference
%
% \definespecialtest\v!extern
%   {\expanded{\localdoifreferencefoundelse{\currentreferenceoperation
%      \ifx\currentreferencearguments\empty\else{\currentreferencearguments}\fi}}}
%
% \definespeciallocation\v!extern
%   {\handleexecreference}
%
% \def\defineexternalreference[#1]%
%   {\setglobalsystemreference\rt!extern{#1}{#1}}
%
% \protect
%
% \defineexternalreference[NewOne]
%
% \def\PDFexecuteNewOne{/SomeNewAction /SomeParameter (\argumentA)}
%
% \starttext
%
%   \goto{test}[AVDP{../../nach-dateipfad.pdf}]
%   \blank
%   \goto{test}[external(AVDP{../../nach-dateipfad.pdf})]
%   \blank
%   \goto{test}[AVDP{../../nach-dateipfad.pdf}]
%   \blank
%   \goto{test}[external(AVDP{../../nach-dateipfad.pdf})]
%   \blank
%   \goto{test}[CloseDocument]
%   \blank
%   \goto{test}[action(CloseDocument)]
%
% \stoptext

\protect \endinput
