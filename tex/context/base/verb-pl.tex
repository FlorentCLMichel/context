%D \module
%D   [       file=verb-pl,
%D        version=1997.12.22,
%D          title=\CONTEXT\ Verbatim Macros,
%D       subtitle=Pretty \PERL\ Verbatim,
%D         author=Hans Hagen,
%D           date=\currentdate,
%D      copyright={PRAGMA / Hans Hagen \& Ton Otten}]
%C
%C This module is part of the \CONTEXT\ macro||package and is
%C therefore copyrighted by \PRAGMA. See mreadme.pdf for 
%C details. 

\writestatus{loading}{Context Support Macros / Pretty PERL Verbatim}

%D When I rewrote \TEXUTIL\ in \PERL\ I decided to write a 
%D pretty \PERL\ interpreter too. In fact, this was the main 
%D reason for rewriting the pretty verbatim mechanisms to 
%D handle more that \TEX. 
%D
%D \startPL 
%D $alfa = $beta{gamma};
%D if ($alfa="delta") 
%D  { print "epsilon" }
%D \stopPL
%D 
%D When looking at the macros, watch the \type{\doglobal}'s! 
%D They're something \CONTEXT\ specific. 

\unprotect

%D \macros
%D   {prettyPLvariables}
%D
%D Identifiers and system variables get a special treatment. 
%D One can disable this option for the variables by resetting 
%D the next switch.

\doglobal\newif\ifprettyPLvariables 

\global\prettyPLvariablestrue

%D As said before, the global assignment is needed in 
%D \CONTEXT\ (for once||only loading on demand). 

\doglobal\newif\ifinPLsingle
\doglobal\newif\ifinPLdouble
\doglobal\newif\ifinPLcomment
\doglobal\newif\ifPLverbose

\newcount\PLverboseskipped

\gdef\PLverbosecorrection%
  {\ifnum\PLverboseskipped>0
     \obeyedspace
     \advance\PLverboseskipped by -1
     \@EA\PLverbosecorrection  
   \fi}

\gdef\PLsetcontrols%
  {\saveprettycontrols
   \def\obeyedspace%
     {\endPLtypesix
      \ifPLverbose
        \advance\PLverboseskipped by 1
      \else
        \oldobeyedspace
      \fi}%
   \def\flushrestofverbatimline%
     {\endPLtypesix
      \global\inPLcommentfalse}%
   \def\obeyedline%
     {\PLverbosefalse
      \PLverboseskipped=0 
      \oldobeyedline}%
   \let\obeytabs=\ignoretabs}

\gdef\JVsetcontrols%
  {\PLsetcontrols
     {\endPLtypesix
      \ifcase\JVcommentlevel \inPLcommentfalse \fi
      \PLverbosefalse
      \PLverboseskipped=0}}

\gdef\setupprettyPLtype%
  {\def\prettyidentifier{PL}%
   \PLsetvariables
   \PLsetcontrols
   \PLsethandlers
   \PLsetspecials
   \PLsetdiagnostics}

\gdef\PLsetvariables%
  {\global\inPLsinglefalse
   \global\inPLdoublefalse
   \global\inPLcommentfalse
   \global\let\PLsequence=\empty
   \global\let\PLprefix=\empty}

\gdef\PLsetspecials%
  {\setpretty`\{=10 \setpretty`\}=10
   \setpretty`\$=21 \setpretty`\@=21 \setpretty`\%=22
   \setpretty`\[=31 \setpretty`\]=31 \setpretty`\(=32 \setpretty`\)=32
   \setpretty`\==32 \setpretty`\~=32 \setpretty`\<=32 \setpretty`\>=32
   \setpretty`\-=41 \setpretty`\+=41 \setpretty`\/=41 \setpretty`\*=41
   \setpretty`\|=41 \setpretty`\`=41 \setpretty`\!=41 \setpretty`\?=41
   \setpretty`\^=41 \setpretty`\&=42
   \setpretty`\\=50
   \setpretty`\1=61 \setpretty`\2=61 \setpretty`\3=61 \setpretty`\4=61
   \setpretty`\5=61 \setpretty`\6=61 \setpretty`\7=61 \setpretty`\8=61
   \setpretty`\9=61 \setpretty`\0=61 \setpretty`\_=61
   \setpretty`\A=62 \setpretty`\B=62 \setpretty`\C=62 \setpretty`\D=62 \setpretty`\E=62
   \setpretty`\F=62 \setpretty`\G=62 \setpretty`\H=62 \setpretty`\I=62 \setpretty`\J=62
   \setpretty`\K=62 \setpretty`\L=62 \setpretty`\M=62 \setpretty`\N=62 \setpretty`\O=62
   \setpretty`\P=62 \setpretty`\Q=62 \setpretty`\R=62 \setpretty`\S=62 \setpretty`\T=62
   \setpretty`\U=62 \setpretty`\V=62 \setpretty`\W=62 \setpretty`\X=62 \setpretty`\Y=62
   \setpretty`\Z=62
   \setpretty`\a=62 \setpretty`\b=62 \setpretty`\c=62 \setpretty`\d=62 \setpretty`\e=62
   \setpretty`\f=62 \setpretty`\g=62 \setpretty`\h=62 \setpretty`\i=62 \setpretty`\j=62
   \setpretty`\k=62 \setpretty`\l=62 \setpretty`\m=62 \setpretty`\n=62 \setpretty`\o=62
   \setpretty`\p=62 \setpretty`\q=62 \setpretty`\r=62 \setpretty`\s=62 \setpretty`\t=62
   \setpretty`\u=62 \setpretty`\v=62 \setpretty`\w=62 \setpretty`\x=62 \setpretty`\y=62
   \setpretty`\z=62
   \setpretty`\;=71 \setpretty`\:=72 \setpretty`\,=71 \setpretty`\.=71
   \setpretty`\'=80
   \setpretty`\"=90
   \setpretty`\#= 0 }

\gdef\PLsethandlers%
  {\installprettyhandler 10 \PLtypeone
   \installprettyhandler 21 \PLtypetwo
   \installprettyhandler 22 \PLtypetwo
   \installprettyhandler 31 \PLtypethree
   \installprettyhandler 32 \PLtypethree
   \installprettyhandler 41 \PLtypefourone
   \installprettyhandler 42 \PLtypefourtwo
   \installprettyhandler 50 \PLtypefive
   \installprettyhandler 61 \PLtypesixone
   \installprettyhandler 62 \PLtypesixtwo
   \installprettyhandler 71 \PLtypesevenone
   \installprettyhandler 72 \PLtypeseventwo
   \installprettyhandler 80 \PLtypeeight
   \installprettyhandler 90 \PLtypenine
   \installprettyhandler  0 \PLtypezero}

\gdef\PLtypeone#1%
  {\ifinPLdouble
     \PLverbosefalse
     \ifinPLcomment
       \endPLtypesix
       \getpretty{#1}%
     \else\ifx\PLprefix\empty
       \endPLtypesix
       \getpretty{#1}%
     \else
       \endPLtypesix
       \xdef\PLprefix{\getpretty{#1}}%
       \beginofpretty[\!!prettyone]\getpretty{#1}\endofpretty
     \fi\fi
   \else
     \endPLtypesix
     \ifinPLcomment
       \getpretty{#1}%
     \else\ifinPLsingle
       \getpretty{#1}%
     \else
       \beginofpretty[\!!prettyone]\getpretty{#1}\endofpretty
     \fi\fi
   \fi}

\gdef\PLtypetwo#1%
  {\endPLtypesix
   \xdef\PLprefix{\getpretty{#1}}%
   \ifinPLcomment
     \getpretty{#1}%
   \else\ifinPLsingle
     \getpretty{#1}%
   \else\ifinPLdouble
     \getprettydata{#1}%
     \ifnum\prettytype=21
       \beginofpretty[\!!prettytwo]\getpretty{#1}\endofpretty
     \else
       \getpretty{#1}%
     \fi
   \else
     \beginofpretty[\!!prettytwo]\getpretty{#1}\endofpretty
   \fi\fi\fi}

\gdef\PLtypethree#1%
  {\endPLtypesix
   \ifinPLcomment
     \getpretty{#1}%
   \else\ifinPLsingle
     \getpretty{#1}%
   \else\ifinPLdouble
     \getprettydata{#1}%
     \ifnum\prettytype=31
       \ifx\PLprefix\empty
         \getpretty{#1}%
       \else
         \beginofpretty[\!!prettythree]\getpretty{#1}\endofpretty
       \fi
     \else
       \getpretty{#1}%
     \fi
   \else
     \beginofpretty[\!!prettythree]\getpretty{#1}\endofpretty
   \fi\fi\fi}

\gdef\PLtypefourone#1%
  {\endPLtypesix
   \ifinPLcomment
     \getpretty{#1}%
   \else\ifinPLsingle
     \getpretty{#1}%
   \else\ifinPLdouble
     \getpretty{#1}%
   \else
     \beginofpretty[\!!prettyfour]\getpretty{#1}\endofpretty
   \fi\fi\fi}

\gdef\doPLtypefourtwo#1#2%
  {\PLverbosefalse
   \getprettydata{#2}%
   \ifnum\prettytype=62
     \expandafter\PLtypetwo
   \else
     \expandafter\PLtypefourone
   \fi
   {#1}#2}

\gdef\PLtypefourtwo%
  {\handlenextnextpretty\doPLtypefourtwo\PLtypefourone}

\gdef\doPLtypefiveA#1#2%
  {\endPLtypesix
   \ifinPLcomment
     \getpretties{#1}{#2}%
   \else\ifinPLsingle
     \getprettydata{#2}%
     \ifnum\prettytype=50
       \beginofpretty[\!!prettytwo]\getpretties{#1}{#2}\endofpretty
     \else\ifnum\prettytype=80
       \beginofpretty[\!!prettytwo]\getpretties{#1}{#2}\endofpretty
     \else
       \getpretties{#1}{#2}%
     \fi\fi
   \else
     \beginofpretty[\!!prettytwo]\getpretties{#1}{#2}\endofpretty
   \fi\fi}

\gdef\doPLtypefiveB#1%
  {\endPLtypesix
   \ifinPLcomment
     \getpretty{#1}%
   \else\ifinPLsingle
     \getpretty{#1}%
   \else
     \beginofpretty[\!!prettytwo]\getpretty{#1}\endofpretty
   \fi\fi}

\gdef\PLtypefive%
  {\ifPLverbose
     \expandafter\PLtypesixtwo
   \else
     \expandafter\handlenextnextpretty\expandafter\doPLtypefiveA\expandafter\doPLtypefiveB
   \fi}

\gdef\endPLtypesix%
  {\ifx\PLsequence\empty\else
     \ifinPLcomment
       \PLsequence
     \else\ifinPLsingle
       \PLsequence
     \else\ifinPLdouble
       \PLsequence
     \else\ifPLverbose
       \beginofpretty[\!!prettytwo]\PLsequence\endofpretty
       \PLverbosefalse
     \else\ifx\PLprefix\empty
       \doifprettyidentifierelse{\PLsequence}{\PLverboses}
         {\PLverbosetrue\PLverboseskipped=3 }
         {\doifprettyidentifierelse{\PLsequence}{\PLidentifiers}
            {\beginofpretty[\!!prettytwo]%
               {\prettyidentifierfont\PLsequence}%
             \endofpretty}
            {\ifprettyPLvariables
               \doifprettyidentifierelse{\PLsequence}{\PLvariables}
                 {\beginofpretty[\!!prettytwo]%
                    {\prettyvariablefont\PLsequence}%
                  \endofpretty}
                 {\PLsequence}%
             \else
               \PLsequence
             \fi}}%
     \else\ifprettyPLvariables
       \doifprettyidentifierelse{\PLprefix\PLsequence}{\PLvariables}
         {\beginofpretty[\!!prettytwo]%
            {\prettyvariablefont\PLsequence}%
          \endofpretty}
         {\PLsequence}%
     \else
       \PLsequence
     \fi\fi\fi\fi\fi\fi
     \global\let\PLsequence=\empty
     \global\let\PLprefix=\empty
   \fi}

\gdef\PLtypesixone%
  {\ifx\PLsequence\empty
     \expandafter\getpretty
   \else
     \expandafter\PLtypesixtwo
   \fi}

\gdef\PLtypesixtwo#1%
  {\ifinPLcomment
     \getpretty{#1}%
   \else\ifinPLsingle
     \getpretty{#1}%
   \else\ifinPLdouble
     \ifx\PLprefix\empty
       \getpretty{#1}%
     \else
       \xdef\PLsequence{\PLsequence\getpretty{#1}}%
     \fi
   \else
     \xdef\PLsequence{\PLsequence\getpretty{#1}}%
   \fi\fi\fi}

\gdef\PLtypesevenone#1%
  {\endPLtypesix
   \getpretty{#1}}

\gdef\PLtypeseventwo%
  {\handlenextnextpretty\doPLtypeseventwo\PLtypesevenone}

\gdef\doPLtypeseventwo#1#2%
  {\getprettydata{#2}%
   \ifnum\prettytype=72
     \PLtypesixtwo{#1}\PLtypesixtwo{#1}%
   \else
     \PLtypesevenone{#1}\expandafter#2%
   \fi}

\gdef\PLtypeeight#1% single quote
  {\endPLtypesix
   \ifinPLcomment
     \getpretty{#1}%
   \else\ifinPLdouble
     \getpretty{#1}%
   \else\ifinPLsingle
     \global\inPLsinglefalse
     \beginofpretty[\!!prettyfour]\getpretty{#1}\endofpretty
   \else
     \global\inPLsingletrue
     \beginofpretty[\!!prettyfour]\getpretty{#1}\endofpretty
   \fi\fi\fi}

\gdef\PLtypenine#1% double quote
  {\endPLtypesix
   \ifinPLcomment
     \getpretty{#1}%
   \else\ifinPLsingle
     \getpretty{#1}%
   \else\ifinPLdouble
     \global\inPLdoublefalse
     \beginofpretty[\!!prettyone]\getpretty{#1}\endofpretty
   \else
     \global\inPLdoubletrue
     \beginofpretty[\!!prettyone]\getpretty{#1}\endofpretty
   \fi\fi\fi}

\gdef\PLtypezero%
  {\endPLtypesix
   \ifnewpretty\expandafter\handlenewpretty\fi\doPLtypezero}

\gdef\doPLtypezero#1%
  {\ifinPLcomment
     \getpretty{#1}%
   \else
     \global\inPLcommenttrue
     \PLverbosecorrection
     \ifnaturaltextext % non tested yet 
       \let\next\naturaltextext
     \else
       \def\next{\beginofpretty[\!!prettythree]\getpretty{#1}\endofpretty}%
     \fi
     \expandafter\next
   \fi}

\gdef\doreportprettyPLtype#1#2#3#4%
  {#1%
     [PERL \string#2 typing conflict, use \string#3 instead]
     \writestatus
       {PERL typing}
       {use \string#3 instead of \string#2 to enable propper visualization}%
     \global#4%
     \expandafter\doemptyverbatimline
   \fi}

\gdef\reportprettyPLtype%
  {\doreportprettyPLtype\ifinPLsingle'\'\inPLsinglefalse
   \doreportprettyPLtype\ifinPLdouble"\"\inPLdoublefalse}

\gdef\PLsetdiagnostics%
  {\let\normaldoemptyverbatimline=\doemptyverbatimline
   \def\doemptyverbatimline%
     {\normaldoemptyverbatimline
      \reportprettyPLtype}}

%D The official \PERL\ version 5 keywords:

\useprettyidentifiers \PLidentifiers \PLsetspecials
  abs accept alarm and atan atan2 bind binmode bless caller
  chdir chmod chomp chop chown chr chroot close closedir cmp
  connect continue cos crypt dbmclose dbmopen defined delete
  die do dump each echo else elsif endgrent endhostent
  endnetent endprotoend endpwent endservent eof eq eval exec
  exists exit exp fcntl fileno flock for foreach fork format
  formline getc getgrent getgrgid getgrnam gethostbyaddr
  gethostbyname gethostent getlogin getnetbyaddr getnetbyname
  getnetent getpeername getpgrp getppid getpriority
  getprotobyname getprotobynumber getprotoent getpwent
  getpwnam getpwuid getservbyname getservbyport getservent
  getsockname getsockopt glob gmtime goto grep gt hex if
  import index int integer ioctl join keys kill last lc
  lcfirst le length lib link listen local localtime log lstat
  lt map method mkdir msgctl msgget msgrcv msgsnd my ne next
  no not oct open opendir or ord pack package pipe pop pos
  print printf process prompt push quotemeta rand read readdir
  readlink recv redo ref rename require reset return reverse
  rewinddir rindex rmdir scalar seek seekdir select semctl
  semget semop send setgrent sethostent setnetent setpgrp
  setpriority setprotoent setpwent setservent setsockopt shift
  shmctl shmget shmread shmwrite shutdown sin sleep socket
  socketpair sort splice split sprintf sqrt srand stat strict
  study sub subs substr symlink syscall sysopen sysread system
  syswrite tell telldir tie tied time times times truncate uc
  ucfirst ucmask umask unary undef unless unlink unpack
  unshift untie until use utime values vars vec wait waitpid
  wantarray warn warn while write xor

%D Additional module keywords (still incomplete):

\useprettyidentifiers \PLvariables \PLsetspecials
  Getopt::Long
  $Getopt::Long::passthrough
  $Getopt::Long::autoabbrev
  &GetOptions

%D A funny hack. When prefixes by \type{TEX}, a \type{\csname}
%D is typeset as \TEX\ sequence. 

\useprettyidentifiers \PLverboses \PLsetspecials
  TEX

\protect 

\endinput 
