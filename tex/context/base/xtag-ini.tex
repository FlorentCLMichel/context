%D \module
%D   [       file=xtag-ini,
%D        version=2000.12.20,
%D          title=\CONTEXT\ XML Support,
%D       subtitle=Initialization,
%D         author=Hans Hagen,
%D           date=\currentdate,
%D      copyright={PRAGMA / Hans Hagen \& Ton Otten}]
%C
%C This module is part of the \CONTEXT\ macro||package and is
%C therefore copyrighted by \PRAGMA. See mreadme.pdf for
%C details.

\writestatus{loading}{Context XML Macros (initialization)}

%D To be sure: 

\newif\ifprocessingXML 

\beginTEX
  \writestatus{XML}{sorry, XML is only supported in (pdf)etex}
  \def\startXMLdefinitions{\gobbleuntil\stopXMLdefinitions}
  \endinput
\endTEX

%D This module is highly optimized for speed, which sometimes 
%D reads to rather unreadable code. Sorry for this. 

\beginETEX XML

  \bgroup \obeylines

    \gdef\startXMLdefinitions%
      {\bgroup\obeylines\dostartXMLdefinitions}

    \gdef\dostartXMLdefinitions #1
      {\egroup\doifsomething{#1}{\writestatus{XML}{loading #1 support}}}

    \global\let\stopXMLdefinitions\relax

  \egroup

\endETEX

% todo: XMLfullsave == also attributes 
% csnames 
% XMLelse -> elseXML

%D Remark: some hard coded character things will be replaced
%D by named glyphs as soon as the upgraded encoding modules
%D are released. At that moment, unicode support will be
%D provided in accordance with the normal support in \CONTEXT.

%D Like it or not, this module deals with angle bracketed
%D input. Processing \XML\ alike input in \CONTEXT\ has been
%D possible since 1995, and till 2000 several methods ran in
%D parallel. These were implemented in modules like the semi
%D public \type {m-sgml}.
%D
%D There is not one optimal solution for processing \XML\ data.
%D The oldest method was based on a very simple preprocessor
%D written in \MODULA\ and later \PERL: \type {<command>} was
%D converted into \type {\begSGML[command]} and optional
%D parameters were passed.
%D
%D A second method uses a \PERL\ or \XSL\ transformation
%D script that produces \CONTEXT\ commands. This method is
%D much slower, mainly because the whole document is read into
%D memory and a document tree is to be build. The advantage is
%D that processing of the resulting document is fast.
%D
%D The third method uses a basic parser written in the \TEX\
%D language, and apart from a few pitfalls, this method is
%D clean and efficient, but not always robust. Because errors
%D in the input are not catched on forehand, processing in
%D \TEX\ may fail due to errors. But, given that a document
%D can be validated on forehand, this is no big problem.
%D
%D Each method has its advantage, but especially the third
%D method puts some demands on \CONTEXT, since no interference
%D between the parser and the core commands may occur. What
%D method is used, depends on the situation.
%D
%D All three methods introduce some problems in interfacing to
%D core \CONTEXT\ functionality. This is due to the fact that
%D sometimes we want to typeset content directly, while in
%D other cases we just want to pick up data for later usage,
%D either or not using delimited arguments. And, when moving
%D data around, there is always the expansion problem to deal
%D with.
%D 
%D As can be expected, we have to map begin and end tags onto
%D \CONTEXT's start- and stopcommands. This is the easy part.
%D When we have to pass the content of an element to a macro
%D as argument, we need to do a delimited grab. Such mappings
%D are not that hard to implement and were part of \type
%D {m-sgml.tex} already. New in this core module is nested
%D grabbing. Also new here is the support for namespaces and
%D extensive attribute handling. On the other hand, recursive
%D traceback of attributes is no longer supported. Because 
%D this feature was not really used, we can hereby safely we
%D declare the \type {m-sgml.tex} module obsolete. 
%D 
%D In order to be able to incorporate \XML\ style definitions
%D into basic \TEX\ styles, we will provide some basic
%D functionality in the core itself. Some of the functionality
%D can be set up with this general command. We use a token
%D list register to handle post||setup actions. This permits
%D us to extend this command. 

\unprotect 

\newtoks\aftersetupXMLprocessing

\def\setupXMLprocessing
  {\dosingleargument\dosetupXMLprocessing}

\def\dosetupXMLprocessing[#1]%
  {\getparameters[\??xp][#1]\the\aftersetupXMLprocessing}

\protect 

%D We will be dealing with elements, which means that we have
%D to take care of \type {<this>} and \type {</that>}, but
%D also with \type {<such/>} and \type {<so />}. In some cases
%D the upper and lowercase alternatives need to be dealt with,
%D although this is not realy needed since XML is case
%D sensitive.
%D
%D We also have to handle entities, like \type {&you;} and
%D \type {&me;}. These are quite easy to deal with and need to
%D be hooked into the encoding and abbreviation mechanisms.
%D
%D And then there are the parameters to be taken care of. Here
%D we meet \type {key="value"} but also \type {key='eulav'}
%D and even the spacy \typ {key = "value"}.
%D
%D Since we have to handlers for each element and entity, we
%D will create a few namespaces. Special care has to be
%D given to preformated code.
%D
%D There are two namespace mechanisms in place: one for
%D \TEX, and one for \XML. The later mechanism permits
%D remapping and ignoring.

\unprotect

\def\@@XML            {XML:}
\def\@@XMLentity      {\@@XML ent} % &crap; 
\def\@@XMLelement     {\@@XML ele} % <this> </this> <that/>
\def\@@XMLvariable    {\@@XML var} % key="val"
\def\@@XMLvalue       {\@@XML val} % key="val"
\def\@@XMLpars        {\@@XML par} %
\def\@@XMLdata        {\@@XML dat} % mem buffer  
\def\@@XMLcode        {\@@XML cod} % named mem buffers
\def\@@XMLinstruction {\@@XML ins} % <? 
\def\@@XMLmap         {\@@XML map} % mapping on context attr 
\def\@@XMLlist        {\@@XML lst} % 
\def\@@XMLnamespace   {\@@XML nam} % namespace:element
\def\@@XMLurlspace    {\@@XML url} %
\def\@@XMLescape      {\@@XML esc} % <!
\def\@@XMLdepth       {\@@XML dep} % used to track nesting 
\def\@@XMLdopth       {\@@XML dop} % used to track nesting 
\def\@@XMLsave        {\@@XML sav} % namespace for saved elements

\newtoks\XMLtoks
\newtoks\XMLresetlist

\newif\ifignoreXMLspaces
\newif\iffixedXMLfont
\newif\iftraceXMLelements
%newif\ifprocessingXML

\newcount\XMLlevel % scratchcounter 
\newcount\XMLdepth % used here 

\newtoks\everyenableXML

%D We will implement the parser by making a few characters
%D active. For that reason we also have to save their
%D original meaning. 

%D \macros
%D   {enableXML}
%D
%D The macro \type {\enableXML} will be used to turn on the
%D parser. This means that after that, \TEX\ commands starting
%D with a backslash will not longer be read as such. There is
%D a way around this, but for convenience \TEXEXEC\ will take
%D care of processing raw \XML\ files in a transparant way.

\bgroup
\catcode`\*=\@@comment
\catcode`\.=\@@escape
.catcode`.B=.@@begingroup
.catcode`.E=.@@endgroup
.catcode`.P=.@@parameter

.catcode`.&=.@@active * .gdef.letterampersand  B.string&E
.catcode`.<=.@@active * .gdef.letterless       B.string<E
                      * .gdef.lettermore       B.string>E
.catcode`.#=.@@active * .gdef.letterhash       B.string#E
.catcode`.$=.@@active * .gdef.letterdollar     B.string$E
.catcode`.%=.@@active * .gdef.letterpercent    B.string%E
.catcode`.\=.@@active * .gdef.letterbackslash  B.string\E
.catcode`.^=.@@active * .gdef.letterhat        B.string^E
.catcode`._=.@@active * .gdef.letterunderscore B.string_E
.catcode`.{=.@@active * .gdef.letterbgroup     B.string{E
.catcode`.}=.@@active * .gdef.letteregroup     B.string}E
.catcode`.|=.@@active * .gdef.letterbar        B.string|E
.catcode`.~=.@@active * .gdef.lettertilde      B.string~E

.gdef.enableXMLexpansion
  B.def<B.doXMLelementE.let&=.doXMLentityE

.gdef.disableXMLexpansion
  B.unexpanded.def<B.doXMLelementE.let&=.doXMLentityE

* internally the # becomes two #'s (before expanding, during
* the parsing stage) which is why we let the first # gobble 
* the second one
*
* since this only takes place when reading arguments, as in 
* \startXMLdata ... cum, suis, we can take place of it 
* there; this is needed because reading from file goes wrong  
* (eating up argument) 

.unexpanded.gdef.enableXML*
  B.catcode`.!=.@@other*
   .catcode`.?=.@@other* 
   .catcode`.:=.@@other* active in french  
   .catcode`.;=.@@other* active in french  
   .catcode`.&=.@@active .let&=.doXMLentity*
   .catcode`.<=.@@active .unexpanded.def<B.doXMLelementE*
   .catcode`.>=.@@other  
   .catcode`."=.@@other*
   .catcode`./=.@@other  
   .catcode`.'=.@@other*
   *catcode`.#=.@@active .def#PP1B&tex-hash;E* gobbles its own dup 
   .catcode`.#=.@@active .def#B&tex-hash;E* 
   .catcode`.$=.@@active .def$B&tex-dollar;E*
   .catcode`.%=.@@active .def%B&tex-percent;E*
   .catcode`.\=.@@active .def\B&tex-backslash;E*
   .catcode`.^=.@@active .def^B&tex-hat;E*
   .catcode`._=.@@active .def_B&tex-underscore;E*
   .catcode`.{=.@@active .def{B&tex-leftbrace;E*
   .catcode`.}=.@@active .def}B&tex-rightbrace;E*
   .catcode`.|=.@@active .def|B&tex-bar;E*
   .catcode`.~=.@@other  *def~B&tex-tilde;E*
   .processingXMLtrue
   .the.everyenableXML
  E

* The following macro can be invokes when reading from 
* an auxiliary file. 

.unexpanded.gdef.enableXMLelements* 
  B.catcode60=.@@active * .catcode`.<=.@@active 
   .catcode62=.@@other  * .catcode`.>=.@@other
   .unexpanded.def<B.doXMLelementE*
   .processingXMLtrue
  E

.unexpanded.gdef.disableXMLelements* 
  B.catcode60=.@@active * .catcode`.<=.@@active 
   .catcode62=.@@other  * .catcode`.>=.@@other
   .let<.relax * new 
   .processingXMLfalse
  E

.global.let<.relax * new 
.global.let&.relax * new 

.egroup

\def\disableXML{\setnormalcatcodes\processingXMLfalse}

%D An element can be singular or paired. A singular element is
%D called an empty element. The following definitions are
%D equivalent:
%D
%D \starttypen
%D <eerste></eerste>  <eerste/>  <eerste />
%D \stoptypen
%D
%D Empty elements can have arguments too. Conforming the
%D standard, each key must have a value. These are separated
%D by an \type {=} sign and the value is delimited by either
%D \type {"} or \type {'}. There may be spaces around the
%D equal sign.
%D
%D \starttypen
%D <eerste a= "b" c ="d" />  <eerste a = "b" c="d"/>
%D \stoptypen
%D
%D Officially the following definition is not valid:
%D
%D \starttypen
%D <eerste>some text</eerste>  <eerste/>  <eerste />
%D \stoptypen
%D
%D Although we can handle both cases independently, this is
%D seldom needed.
%D
%D Processing instructions are identified by a~\type {?} and are
%D like empty elements.
%D
%D \starttypen
%D <?doel a="b" c="d"?> <?doel a="b" c="d" ?>
%D \stoptypen
%D
%D Comment is formatted as follows.
%D
%D \starttypen
%D <!-- comment -->
%D \stoptypen
%D
%D Verbatim code inits purest form is called \type {CDATA} and
%D is embedded in the following ugly and therefore recognizable
%D way:
%D
%D \starttypen
%D <![CDATA[
%D Dit is nogal verbatim !
%D Dit is nogal verbatim !
%D Dit is nogal verbatim !
%D ]]>
%D \stoptypen

%D The parser is implemented as a multi||step macro. Because
%D \type {!} and \type {?} should be picked up correctly, we
%D need to define a few macros in unprotected mode!
%D
%D Because \XML\ is defined with some restrictions in mind,
%D parsing the elements is not that complicated. First we have
%D to determine if we're dealing with a comment or processing
%D instruction. We need a bit of grouping because we have to
%D mess up with catcodes. We probably have to treat a few
%D more catcode and first character cases. We need to use
%D \type {\begingroup} here, otherwise we get funny spaces in
%D math. 

%D Maybe I will remove grouping here and introduce \type 
%D {\obeyXMLlines}. 

%D By using a few {\expandafter}'s we can avoid a \type {\next}
%D construction. We could speed the first char test up a bit
%D by using an installer and something \typ {\getvalue
%D {#1doXMLelement}} (todo).

\protect % we need an normal ! ?  

% \long\def\doXMLelement#1%
%   {\if#1!\expandafter                         \xparseXMLescape  \else
%    \if#1?\expandafter\expandafter\expandafter \xparseXMLprocess \else
%          \expandafter\expandafter\expandafter \xparseXMLelement  \fi\fi
%    #1}

\def\expandthree{\expandafter\expandafter\expandafter}

\long\def\doXMLelement#1%
  {\if#1!\expandafter \xparseXMLescape  \else
   \if#1?\expandthree \xparseXMLprocess \else
         \expandthree \xparseXMLelement \fi\fi
   #1}

% does it end with ? or ?> 

\long\def\xparseXMLescape  !#1    {\parseXMLescape{#1}}
\long\def\xparseXMLprocess ?#1 #2>{\parseXMLprocess{#1}{#2}}
\long\def\xparseXMLelement     #1>{\parseXMLelement #1 >}

%D The escape handler takes care of the fuzzy \type {<!} 
%D constructs. Unfortunately, we need to catch \type {<--text} 
%D too, so we need another handler: 

\long\def\xparseXMLescape !#1#2%
  {\if#1-%
     \if#2-%
       \expandafter\expandafter\expandafter\xxparseXMLescape
     \else
       \expandafter\expandafter\expandafter\xyparseXMLescape
     \fi
   \else
     \if#1[%
       \expandafter\expandafter\expandafter\xzparseXMLescape
     \else
       \expandafter\expandafter\expandafter\xyparseXMLescape
     \fi
   \fi#1#2}

\long\def\xxparseXMLescape--#1{\parseXMLescape{--}#1}
\long\def\xyparseXMLescape#1  {\parseXMLescape{#1}}
\long\def\xzparseXMLescape[#1[{\parseXMLescape{#1}}

%D Now the real work can begin.

\unprotect

\def\defineXMLescape[#1]#2%
  {\unspaceargument#1\to\ascii % get rid of {}, like in {CDATA[}
   \long\setvalue{\@@XMLescape:\ascii}{#2}}

\def\parseXMLescape#1% #2> parsing takes place in macros 
  {\executeifdefined{\@@XMLescape:#1}{\csname\@@XMLescape:\s!default\endcsname}}

%D In our case, processing instructions are only needed if
%D we want specific \CONTEXT\ support. This may be useful in
%D applications where the data is generated by an
%D application. We will implement a \CONTEXT\ code handler
%D later.

%D The processing instructions handler is implemented as
%D follows.

\long\def\defineXMLprocessor[#1]#2% 
  {\long\setvalue{\@@XMLinstruction:#1}{#2}}

% \def\parseXMLprocess#1#2% 
%   {\executeifdefined{\@@XMLinstruction:#1}\gobbleoneargument{#2}}

%D Because we support \type {.. ?>} as well as \type {.. >} 
%D end||of||pi situations, we need to clean up the ending 
%D \type {?}.

\protect 

\long\def\cleanupXMLprocess#1%
  {\docleanupXMLprocess#1 ? \relax}

\long\def\docleanupXMLprocess#1? #2\relax 
  {\def\currentXMLprocess{#1}}

\unprotect 

\def\parseXMLprocess#1#2% 
  {\cleanupXMLprocess{#2}%
  %\convertcommand\currentXMLprocess\to\ascii
  %\writestatus{xml-process}{\ascii}
   \expanded
     {\executeifdefined
        {\@@XMLinstruction:#1}
        \noexpand\gobbleoneargument
        {\expandafter\noexpand\currentXMLprocess}}}

%D One never knows: 

\let\normalparseXMLescape \parseXMLescape
\let\normalparseXMLelement\parseXMLelement
\let\normalparseXMLprocess\parseXMLprocess

%D Next we will implement the normal element handler. This
%D piece of code is complicated by the fact that we need to
%D handle namespaces.

\let\currentXMLarguments \empty
\let\currentXMLelement   \empty % name
\let\currentXMLidentifier\empty % name or name/
\let\currentXMLnamespace \empty % the (remapped) namespace
\let\originalXMLnamespace\empty % the unremapped namespace
\let\rawXMLidentifier    \empty 

\def\rawXMLnamespace
  {\ifx\currentXMLnamespace\empty\else\currentXMLnamespace:\fi}

\def\rawXMLelement{\rawXMLnamespace\rawXMLidentifier}

%D The following token list register provided the user a hook
%D for extensions.

\newtoks\everyXMLelement

%D We try to keep track of the nature of an element. This
%D flag can be used for special purposes (as in the pretty
%D printing macros).

\chardef\kindofXMLelement=0

\chardef\beginXMLtag=1
\chardef\endXMLtag  =2
\chardef\emptyXMLtag=3

%D We do a rather hard coded scan for the namespace attribute. This 
%D is needed because its value determines further namespace related 
%D actions. 

\def\openXMLargument{ /}
   
\long\def\parseXMLelement#1#2 #3>%
  {\def\currentXMLarguments{#3}% including end / 
   \if#1/% 
     \chardef\kindofXMLelement\endXMLtag
     \def\currentXMLelement{#2}%
   \else
     \docleanupXMLelement#1#2/\empty\relax
   \fi 
   \ifx\currentXMLarguments\openXMLargument
     \chardef\kindofXMLelement\emptyXMLtag
   \fi
   \@EA\splitoffXMLnamespace\currentXMLelement::\relax
   \ifcase\kindofXMLelement
     % can't happen 
   \or 
     % begintag or emptytag with arguments or space before /  
     \the\everyXMLelement % only for begin/empty tag !
     \ifx\currentXMLarguments\empty \else
       \dogetXMLarguments\rawXMLelement#3>%
     \fi
   \or
     % no arguments 
   \or
     % empty element without arguments (but possible presets) 
     \the\everyXMLelement 
   \fi 
   \ifcase\kindofXMLelement\or
     \let\currentXMLidentifier       \rawXMLidentifier
     \edef\currentXMLfullidentifier {\rawXMLelement    }%
   \or
     \edef\currentXMLidentifier    {/\rawXMLidentifier }%
     \edef\currentXMLfullidentifier{/\rawXMLelement    }%
   \or
     \edef\currentXMLidentifier     {\rawXMLidentifier/}%
     \edef\currentXMLfullidentifier {\rawXMLelement   /}%
   \fi
   \iftraceXMLelements\traceXMLelement\fi
   \executeXMLelement}

\long\def\docleanupXMLelement#1/#2#3\relax
  {\def\currentXMLelement{#1}% watch out: \empty == begin or empty tag 
   \chardef\kindofXMLelement\ifx#2\empty\beginXMLtag\else\emptyXMLtag\fi}

\def\@@traceXMLelement%
  {\originalXMLfullidentifier
   \ifx\originalXMLfullidentifier\currentXMLfullidentifier\else
     \space=>\space\currentXMLfullidentifier
   \fi
   \ifx\currentXMLarguments\empty\else
     \space\string|\space\currentXMLarguments
   \fi}

\long\def\traceXMLelement%
  {\edef\originalXMLfullidentifier{\someXMLelement\currentXMLelement}%
   \cleanupXMLarguments\writestatus{xml-element}{\@@traceXMLelement}}

%D We split off the namespace part, construct the
%D identifier, and remap the namespace if needed.

\def\splitoffXMLnamespace#1:#2:#3\relax
  {\def\rawXMLidentifier{#2}%
   \ifx\rawXMLidentifier\empty
     \let\currentXMLnamespace\empty
     \edef\rawXMLidentifier{#1}%
   \else
     \edef\currentXMLnamespace{#1}%
   \fi
   \let\originalXMLnamespace\currentXMLnamespace
   \checkXMLnamespace\rawXMLidentifier}

\def\xsplitoffXMLnamespace% fast resplit 
  {\ifcsname\@@XMLnamespace:\currentXMLnamespace\endcsname
     \csname\@@XMLnamespace:\currentXMLnamespace\endcsname
   \fi}

%D We will implement this macro later.

\let\checkXMLnamespace\gobbleoneargument % see below

%D The namespace attribute checking is part of the element
%D parser, since the value of \type {xmlns} may influence other
%D namespace mapping. 

\def\@@XMLns{xmlns}

\def\checkXMLnamespaceattr#1% xmlns:\@@XMLname="\XMLns"
  {\edef\XMLns{#1}% 
   \ifx\XMLns\empty \else
     \ifcsname\@@XMLurlspace:\XMLns\endcsname
       % get remapped namespace (from url) 
       % \edef\XMLns{\csname\@@XMLurlspace:\XMLns\endcsname}%
       \@EA\let\@EA\XMLns\csname\@@XMLurlspace:\XMLns\endcsname
       % remap this one 
       \ifx\@@XMLname\empty 
         % not watertight since no implicit grouping 
         \xautoXMLnamespace\XMLns
       \else
         \xremapXMLnamespace\@@XMLname\XMLns
         % redo namespace remapping of self if needed 
         \ifx\XMLns\currentXMLnamespace 
           % i'm still not sure if this is ok 
         \else
           \xsplitoffXMLnamespace
         \fi
       \fi
     \fi 
   \fi}

%D Although not really needed, we clean up the arguments. 

% \long\def\cleanupXMLarguments
%   {\ifnum\kindofXMLelement=\emptyXMLtag
%      \ifx\currentXMLarguments\empty \else 
%        \@EA\docleanupXMLarguments\currentXMLarguments/\empty
%      \fi
%    \fi}
% 
% \long\def\docleanupXMLarguments#1/#2\empty
%   {\edef\currentXMLarguments{#1}}
%
% we  need to be ...="/" .... /> safe

\long\def\cleanupXMLarguments
  {\ifnum\kindofXMLelement=\emptyXMLtag
     \ifx\currentXMLarguments\empty \else 
        \@EA\docleanupXMLarguments\currentXMLarguments/ \relax
     \fi
   \fi}

\long\def\docleanupXMLarguments#1/ #2\relax % space added earlier
  {\edef\currentXMLarguments{#1}}

\def\executeXMLelementA% no fallback 
  {\ifcsname\@@XMLelement:\currentXMLfullidentifier\endcsname
     \csname\@@XMLelement:\currentXMLfullidentifier\endcsname   
   \fi}

\def\executeXMLelementB% default fallback 
  {\csname \@@XMLelement:%
     \ifcsname\@@XMLelement:\currentXMLfullidentifier\endcsname
       \currentXMLfullidentifier
     \else
       \defaultXMLelement
     \fi
   \endcsname}

\def\executeXMLelementC% no namespace of default fallback
  {\csname \@@XMLelement:%
     \ifcsname\@@XMLelement:\currentXMLfullidentifier\endcsname
       \currentXMLfullidentifier
     \else\ifcsname\@@XMLelement:\currentXMLidentifier\endcsname
       \currentXMLidentifier
     \else
       \defaultXMLelement
     \fi\fi
   \endcsname}

\def\setXMLfallbackmode#1%
  {\ifcase#1\relax
     \let\executeXMLelement\executeXMLelementA
   \or
     \let\executeXMLelement\executeXMLelementB
   \or
     \let\executeXMLelement\executeXMLelementC
   \fi}

\setXMLfallbackmode2

%D An example of fall back modes is given below. 

%D Later we will implement the error handler, here we handle
%D the default case. 

\def\someXMLelement#1%
  {\ifnum\kindofXMLelement=\endXMLtag  /\fi
   #1%
   \ifnum\kindofXMLelement=\emptyXMLtag/\fi}

\def\defaultXMLelement%
  {\someXMLelement\s!default}

%D It is possible to keep track of nesting automatically, 
%D but this would kind of prohibit things like \type 
%D {\ignorespaces}. In the future we may provide an 
%D automatic depth tracking as an alternative (exclusive) 
%D mode of operation combined with space grabbing.  

\def\beginXMLelement
  {\global\advance\XMLdepth \plusone    %  1 
   \global\@EA\let\csname\@@XMLdepth:\the\XMLdepth\endcsname\currentXMLelement}

\def\endXMLelement
  {\global\advance\XMLdepth \minusone } % -1 }

\def\XMLancestor#1%
  {\ifnum\numexpr(\XMLdepth-#1)>0
     \csname\@@XMLdepth:\the\numexpr(\XMLdepth-#1)\endcsname
   \else
     \currentXMLelement
   \fi}

\def\XMLparent%
  {\XMLancestor1}

% todo: split #1 into raws en reconstruct, set current etc, push and pop   
%
% \def\beginXMLelement
%   {\dosingleempty\dobeginXMLelement}
% 
% \def\dobeginXMLelement[#1]%
%   {\global\advance\XMLdepth 1 
%    \global\@EA\let\csname\@@XMLdepth:\the\XMLdepth\endcsname\currentXMLelement
%    \global\@EA\edef\csname\@@XMLdopth:\the\XMLdepth\endcsname{\ownXMLelement}%
%    \iffirstargument\edef\ownXMLelement{#1}\fi}
% 
% \def\endXMLelement
%   {\@EA\let\@EA\ownXMLelement\csname\@@XMLdopth:\the\XMLdepth\endcsname
%    \global\advance\XMLdepth -1 }

%D \defineXMLenvironment[one] 
%D   {\beginXMLelement}
%D   {\endXMLelement}
%D 
%D \defineXMLenvironment[two] 
%D   {\beginXMLelement
%D    \starttabulate
%D    \NC parent 1 \NC \XMLparent{1} \NC \NR 
%D    \NC parent 2 \NC \XMLparent{2} \NC \NR 
%D    \NC parent 3 \NC \XMLparent{3} \NC \NR 
%D    \NC parent 4 \NC \XMLparent{4} \NC \NR 
%D    \stoptabulate}
%D   {\endXMLelement}
%D   
%D   \startbuffer
%D   <x:one> <x:two> <one> <two> </two> </one> </x:two> </x:one> 
%D   \stopbuffer
%D   
%D   fallback A: \setXMLfallbacklevel0 \processXMLbuffer
%D   fallback B: \setXMLfallbacklevel1 \processXMLbuffer
%D   fallback C: \setXMLfallbacklevel2 \processXMLbuffer

%D Here we do the namespace (re)mapping. More examples are 
%D provided in the manual. 
%D
%D \starttypen
%D \supportXMLnamespace  [test] % needed to get a namespace working 
%D \skipXMLnamespace     [test] % slow
%D \ignoreXMLnamespace   [test] % faster
%D \defineXMLenvironment [rubish:itemize] {[} {]}
%D \defineXMLenvironment [rubish:item]    {(} {)}
%D \remapXMLnamespace    [crap] [rubish]
%D \remapXMLnamespace    [con]  [context]
%D \remapXMLurlspace     [http://www.pragma-ade.com/dtd/context]  [context]
%D \autoXMLnamespace     [context] % fallback
%D \autoXMLnamespace     [whatever] % second fall back
%D \stoptypen

\newtoks\autoXMLnamespaces

%D The automatically mapped namespaces (the fallbacks so to 
%D day) are collected in a token list. 

\let\checkedXMLnamespace\empty

\def\checkXMLnamespace#1%
  {\edef\checkedXMLnamespace{#1}%
   \ifcsname\@@XMLnamespace:\currentXMLnamespace\endcsname
     \csname\@@XMLnamespace:\currentXMLnamespace\endcsname
     % forced namespace 
   \else\ifcsname\@@XMLelement:\currentXMLelement\endcsname
     % natural element 
   \else\ifcsname\@@XMLelement:\currentXMLelement/\endcsname
     % natural element 
   \else
     % locate fallback 
     \donefalse\the\autoXMLnamespaces
   \fi\fi\fi}

\def\skipXMLnamespace[#1]%
  {\letvalue{\@@XMLnamespace:#1}\doXMLskipnamespace}

\def\doXMLskipnamespace
  {\long\@EA\def\csname\@@XMLelement:\checkedXMLnamespace\endcsname
     {\getXMLgroupedignore\checkedXMLnamespace}}

\def\hideXMLnamespace[#1]%
  {\letvalue{\@@XMLnamespace:#1}\doXMLhidenamespace}

\def\doXMLhidenamespace
  {\long\@EA\def\csname\@@XMLelement:\checkedXMLnamespace\endcsname
     {\redoXMLignore\checkedXMLnamespace}}

\def\ignoreXMLnamespace[#1]%
  {\letvalue{\@@XMLnamespace:#1}\doXMLignorenamespace}

\def\doXMLignorenamespace % \let binnen def 
  {\long\@EA\def\csname\@@XMLelement:\checkedXMLnamespace\endcsname
     {\@EA\redoXMLignore\@EA{\checkedXMLnamespace}}} % EA ? 

\def\remapXMLnamespace
  {\dodoubleargument\doremapXMLnamespace}

\def\doremapXMLnamespace[#1][#2]%
  {\ifsecondargument
     \setvalue{\@@XMLnamespace:#1}{\def\currentXMLnamespace{#2}}%
   \else
     \letvalue{\@@XMLnamespace:#1}\relax
   \fi}

\def\supportXMLnamespace%
  {\dosingleargument\dosupportXMLnamespace}

\def\dosupportXMLnamespace[#1]%
  {\setvalue{\@@XMLnamespace:#1}{\def\currentXMLnamespace{#1}}}

\def\xremapXMLnamespace#1#2% fast internal one 
  {\@EA\edef\csname\@@XMLnamespace:#1\endcsname
     {\def\noexpand\currentXMLnamespace{#2}}}

\def\autoXMLnamespace[#1]%
  {\xautoXMLnamespace{#1}}

\def\xautoXMLnamespace#1% fast internal one 
  {\ifcsname\@@XMLnamespace-#1\endcsname\else
     \@EA\appendtoks\csname\@@XMLnamespace-#1\endcsname\to\autoXMLnamespaces
   \fi
   \@EA\edef\csname\@@XMLnamespace-#1\endcsname
     {\noexpand\doautoXMLnamespace{#1}}}

\def\doautoXMLnamespace#1% \done is set before list
  {\ifdone\else
     \ifcsname\@@XMLelement:#1:\checkedXMLnamespace\endcsname
       \def\currentXMLnamespace{#1}%
     \else\ifcsname\@@XMLelement:#1:\checkedXMLnamespace/\endcsname
       \def\currentXMLnamespace{#1}%
     \fi\fi
   \fi}

\def\resetXMLnamespace[#1]%
  {\letvalue{\@@XMLnamespace-#1}\gobbleoneargument
   \letvalue{\@@XMLnamespace:#1}\gobbleoneargument}

\def\remapXMLurlspace
  {\dodoubleargument\doremapXMLurlspace}

\def\doremapXMLurlspace[#1][#2]%
  {\setvalue{\@@XMLurlspace:#1}{#2}}

%D Entities needs a bit more work, as well as a connection 
%D with the encoding handlers. 

% we need to be able to do:
%
% \defineXMLentity[amp] {\FunnyAmp} \def\FunnyAmp#1;{\getXMLentity{#1}}
%
% \defineXMLentity [pound] {(why not use euro's?)}
%
% \startXMLdata
% test &amp;pound; test
% \stopXMLdata
%
% so we need an ifless implementation

% \eacute -> simplified -> e (via raw encoding)  
%         -> raw        -> eacute (via handler) 
%
% naming sucks

\newif\ifXMLrawentities     % proper fallback 
\newif\ifXMLsimpleentities  % last resort 

\def\simpleXMLencoding{raw}

\def\simplifyXMLentities%
  {\fastenableencoding\simpleXMLencoding
   \XMLsimpleentitiestrue}

\def\defineXMLentity%
  {\dodoubleempty\dodefineXMLentity}

\def\dodefineXMLentity[#1][#2]#3%
  {\ifsecondargument
     \defineXMLentities[#1]{#2}{#3}%
   \else
     \dododefineXMLentity{#1}{#3}%
   \fi}

\def\defineXMLentities[#1]#2#3%
  {\dododefineXMLentity{#1}{\ifXMLsimpleentities#2\else#3\fi}}

\def\dododefineXMLentity#1#2%
  {\unspaceargument#1\to\ascii % #1 can be {[} or so 
   \long\setvalue{\@@XMLentity:\@EA\firstofoneargument\ascii}{#2}}

%D May this wile become dodo (more in tune with rest); 
%D beware: also remapped in xtag-map. 

\def\doXMLentity#1#2;% interesting: # is now ## 
  {\if\string#1\letterhash 
     \@EA\parseXMLcharacter
   \else\ifXMLrawentities
     \@EAEAEA\firstofoneargument
   \else
     \@EAEAEA\executeXMLentity
   \fi\fi{#1#2}}

%D Here we need to get rid of the double hash and act upon the
%D number. Proper hex/oct number support can be implemented by
%D redefining \type {\executeXMLcharacter}. 

% \def\parseXMLcharacter#1% gobble the ##x 
%   {\@EA\executeXMLcharacter\@EA{\gobblethreearguments#1}} 
%
% single hash now 
% 
% \def\parseXMLcharacter#1% 
%   {\@EA\executeXMLcharacter\@EA{\gobbleoneargument#1}} 
%   
% \def\executeXMLcharacter#1% can be overloaded
%   {\ifnum"#1<256
%      \@EA\getXMLcharacter
%    \else\ifXMLrawentities
%      \@EAEAEA\firstofoneargument
%    \else
%      \@EAEAEA\unknownXMLcharacter
%    \fi\fi{\number"#1}}
% 
% \unexpanded\def\getXMLcharacter#1{\rawcharacter{#1}} 

\def\parseXMLcharacter#1% 
  {\@EA\executeXMLcharacter\gobbleoneargument#1\empty\relax} 
 
\def\executeXMLcharacter#1#2\relax 
  {\if#1x%
     \@EA\executeXMLhexcharacter
   \else
     \@EA\executeXMLdeccharacter
   \fi#1#2\relax}

% \def\executeXMLhexcharacter x#1\relax % can be overloaded
%   {\ifnum"#1<256
%      \@EA\getXMLcharacter
%    \else\ifXMLrawentities
%      \@EAEAEA\firstofoneargument
%    \else
%      \@EAEAEA\unknownXMLcharacter
%    \fi\fi{\number"#1}}

\def\executeXMLhexcharacter x#1\relax % can be overloaded
  {\uppercase{\doexecuteXMLhexcharacter#1\relax}}

\def\doexecuteXMLhexcharacter#1\relax
  {\ifnum"#1<256
     \@EA\getXMLcharacter
   \else\ifXMLrawentities
     \@EAEAEA\firstofoneargument
   \else
     \@EAEAEA\unknownXMLcharacter
   \fi\fi{\number"#1}}

% == 
%
% \def\executeXMLhexcharacter x#1\relax
%   {\uppercase{\ifnum"#1}<256
%      \@EA\getXMLcharacter
%    \else\ifXMLrawentities
%      \@EAEAEA\firstofoneargument
%    \else
%      \@EAEAEA\unknownXMLcharacter
%    \fi\fi{\uppercase{\number"#1}}}

\def\executeXMLdeccharacter#1\relax % can be overloaded
  {\ifnum#1<256
     \@EA\getXMLcharacter
   \else\ifXMLrawentities
     \@EAEAEA\firstofoneargument
   \else
     \@EAEAEA\unknownXMLcharacter
   \fi\fi{\number#1}}

\unexpanded\def\getXMLcharacter#1%
  {\ifcsname\@@XMLentity:#1\endcsname
     \@EA\getXMLentity
   \else
     \@EA\rawcharacter
   \fi{#1}}

\def\unknownXMLcharacter#1{[#1]} 

% \useXMLfilter[ent] 
% 
% \defineXMLsingular[test]{{\simplifyXMLentities\XMLpar{test}{bla}{}}}
% 
% \startXMLdata
% <test bla="&#xD3;bla&eacute;bla&tex;and$and&#xFC;ziezo&#xF99;" />
% \stopXMLdata
% 
% \defineXMLentity[45]{|it works|} % {|-|}
% 
% \startXMLdata
% text&#045;&#046;text
% text&#x045;&#x046;text
% \stopXMLdata

%D May be this will change a bit ... 

\def\executeXMLentity#1% named one 
  {\getXMLentity{#1}}

%\def\expandedXMLentity#1%
%  {\ifcsname\@@XMLentity:#1\endcsname\csname\@@XMLentity:#1\endcsname\fi} 
%
%\unexpanded\def\getXMLentity#1%
%  {\ifcsname\@@XMLentity:#1\endcsname\csname\@@XMLentity:#1\endcsname\fi} 
%
%\def\expandedXMLentity#1%
% {\csname\@@XMLentity:#1\endcsname} 
%
%\unexpanded\def\getXMLentity#1%
%  {\csname\@@XMLentity:#1\endcsname} 

\newif\ifautoXMLentities % fall back on context commands 

\def\expandedXMLentity#1%
  {\ifcsname\@@XMLentity:#1\endcsname \@EA     \execXMLentity
   \else\ifautoXMLentities            \@EAEAEA \autoXMLentity
   \else                              \@EAEAEA \crapXMLentity
   \fi\fi{#1}} 

\def\execXMLentity#1% 
  {\csname\@@XMLentity:#1\endcsname}

\def\crapXMLentity#1% 
  {\inframed[\c!offset=.1ex]{\tttf#1}}

\def\autoXMLentity#1%
  {\ifcsname#1\endcsname\csname#1\endcsname\fi}

\unexpanded\def\getXMLentity{\expandedXMLentity}

\def\doifXMLentityelse#1#2#3%
  {\ifcsname\@@XMLentity:#1\endcsname#2\else#3\fi}

% see \defineXML... commands:
%
% [key=val]          => \presetXMLarguments{element} => default key/vals
% [blabla]           => \theXMLarguments{blabla}     => user key/vals
% [blabla] [key=val] => \presetXMLarguments{element} => default key/vals
%                       \theXMLarguments{blabla}     => user key/vals
%
% <element key="val"> stored in case of [blabla] else set as \XMLpar
%
% see m-steps for an example of usage

\let\@@XMLmapmap\empty
\newif\ifXMLnamespace

\long\def\getXMLarguments#1#2%
  {\XMLnamespacefalse
   \dogetXMLarguments{#1}#2>}

\let\dosetXMLattributeA\gobbleoneargument

\def\dosetXMLattributeB#1%
  {\ifx\@@XMLspac\originalXMLnamespace 
     \@EA\def\csname\@@XMLvariable:\@@XMLclass:\@@XMLname\endcsname{#1}%
   \fi}

% \def\dosetXMLattributeC#1%
%   {\ifx\@@XMLspac\originalXMLnamespace
%      \@EA\def\csname\@@XMLvariable:\@@XMLclass:\@@XMLname\endcsname{#1}%
%    \else
%      \@EA\def\csname\@@XMLvariable:\@@XMLclass:\@@XMLspac:\@@XMLname\endcsname{#1}%
%    \fi}

\def\dosetXMLattributeC%
  {\@EA\def\csname\@@XMLvariable:\@@XMLclass\ifx\@@XMLspac
     \originalXMLnamespace\else:\@@XMLspac\fi:\@@XMLname\endcsname}

% \def\dosetXMLattributeD#1%
%   {\ifx\@@XMLspac\originalXMLnamespace
%      \@EA\def\csname\@@XMLvariable:\@@XMLclass:\currentXMLnamespace:\@@XMLname\endcsname{#1}%
%    \else
%      \@EA\def\csname\@@XMLvariable:\@@XMLclass:\@@XMLspac:\@@XMLname\endcsname{#1}%
%    \fi}

\def\dosetXMLattributeD%
  {\@EA\def\csname\@@XMLvariable:\@@XMLclass:\ifx\@@XMLspac
     \originalXMLnamespace\currentXMLnamespace\else\@@XMLspac\fi
       :\@@XMLname\endcsname}

\def\setXMLattributemode#1%
  {\ifcase#1\relax
     \let\dosetXMLattribute\dosetXMLattributeA
   \or
     \let\dosetXMLattribute\dosetXMLattributeB
   \or
     \let\dosetXMLattribute\dosetXMLattributeC
   \or
     \let\dosetXMLattribute\dosetXMLattributeD
   \fi}

\setXMLattributemode{2} % a reasonable default 

\let\@@XMLspac\empty % argumentnamespace

\long\def\dogetXMLarguments#1%
  {\XMLtoks\emptytoks
   \ifcsname\@@XMLmap:#1\endcsname
     \let\dodosetXMLargument\dodosetXMLargumentB
   \else
     \def\@@XMLclass{#1}%
     \let\dodosetXMLargument\dodosetXMLargumentA
   \fi
   \let\dodoparseXMLarguments\doparseXMLarguments
   \doparseXMLarguments}

\long\def\doparseXMLarguments#1% space goes ok
  {\if#1>%
     \let\dodoparseXMLarguments\empty
   \else\if#1=% 
     \edef\@@XMLname{\the\XMLtoks}%
     \XMLtoks\emptytoks
   \else\if#1"%
     \let\dodoparseXMLarguments\dodoparseXMLargumentsD
   \else\if#1'%
     \let\dodoparseXMLarguments\dodoparseXMLargumentsS
   \else\if#1:%
     \XMLnamespacetrue
     \edef\@@XMLspac{\the\XMLtoks}%
     \XMLtoks\emptytoks
   \else\if#1/%
     \chardef\kindofXMLelement\emptyXMLtag
   \else
     \XMLtoks\@EA{\the\XMLtoks#1}%
   \fi\fi\fi\fi\fi\fi
   \dodoparseXMLarguments}

\def\dodoparseXMLargumentsD#1"{\dosetXMLargument{#1}}
\def\dodoparseXMLargumentsS#1'{\dosetXMLargument{#1}}

\def\dosetXMLargument#1%
  {\ifXMLnamespace
     \ifx\@@XMLspac\@@XMLns
       \checkXMLnamespaceattr{#1}% xmlns:\@@XMLname="#1"
     \else
       \dosetXMLattribute{#1}% some:\@@XMLname="#1"
     \fi
     \XMLnamespacefalse
   \else\ifx\@@XMLname\@@XMLns
     \checkXMLnamespaceattr{#1}% xmlns="#1"
   \else
     \dodosetXMLargument{#1}%
   \fi\fi
   \let\dodoparseXMLarguments\doparseXMLarguments
   \dodoparseXMLarguments}

\def\dosetXMLargument#1% ugly alternative 
  {\ifXMLnamespace
     \XMLnamespacefalse
     \ifx\@@XMLspac\@@XMLns
       \@EAEAEA\checkXMLnamespaceattr % xmlns:\@@XMLname="#1"
     \else
       \@EAEAEA\dosetXMLattribute % some:\@@XMLname="#1"
     \fi
   \else\ifx\@@XMLname\@@XMLns
     \@EAEAEA\checkXMLnamespaceattr % xmlns="#1"
   \else
     \@EAEAEA\dodosetXMLargument
   \fi\fi{#1}%
   \let\dodoparseXMLarguments\doparseXMLarguments
   \dodoparseXMLarguments}

\def\dodosetXMLargumentA%
  {\@EA\def\csname\@@XMLvariable:\@@XMLclass:\@@XMLname\endcsname}

\def\dodosetXMLargumentB#1%
  {\@EA\edef\csname\@@XMLmap:\@@XMLmapmap\endcsname
      {\@EA\ifx\csname\@@XMLmap:\@@XMLmapmap\endcsname\empty\else
         \csname\@@XMLmap:\@@XMLmapmap\endcsname,%
       \fi
       \@@XMLname=#1}}

\appendtoks
  \resetXMLarguments{\rawXMLnamespace\rawXMLidentifier}%
\to \everyXMLelement

\def\resetXMLarguments#1%
  {\ifcsname\@@XMLmap:#1\endcsname
     \@EA\let\@EA\@@XMLmapmap\csname\@@XMLmap:#1\endcsname
     \@EA\let\csname\@@XMLmap:\@@XMLmapmap\endcsname\empty
   \fi}

\def\theXMLarguments#1%
  {\ifcsname\@@XMLmap:#1\endcsname\csname\@@XMLmap:#1\endcsname\fi}

\newtoks\globalscratchtoks

\def\doexpandXMLvalue#1#2%
  {\ifcsname#2\endcsname
     \bgroup
     \enableXMLexpansion
     \let\getXMLentity\expandedXMLentity
     #1% simplify maps entities back to _ and alike 
     \expanded{\global\globalscratchtoks{\csname#2\endcsname}}%
     \egroup
     \@EA\edef\csname#2\endcsname{\the\globalscratchtoks}%
   \fi}

\def\expandXMLvalue  {\doexpandXMLvalue\relax}
\def\simplifyXMLvalue{\doexpandXMLvalue\XMLsimpleentitiestrue}

\def\expandTEXpar      #1#2{\expandXMLvalue{#1\interfaced{#2}}}
\def\expandXMLpar      #1#2{\expandXMLvalue{\@@XMLvariable:#1:#2}}
\def\expandXMLarguments  #1{\expandXMLvalue{\@@XMLmap:#1}}

\def\simplifyTEXpar    #1#2{\simplifyXMLvalue{#1\interfaced{#2}}}
\def\simplifyXMLpar    #1#2{\simplifyXMLvalue{\@@XMLvariable:#1:#2}}
\def\simplifyXMLarguments#1{\simplifyXMLvalue{\@@XMLmap:#1}}

%D \startbuffer[tex]
%D \defineXMLsingular [fx:root]
%D   {\XMLNSpar{fx:root}{xml}{lang}{}
%D    \XMLpar{fx:root}{xml:lang}{}
%D    \starttabulate[||||]
%D    \HL
%D    \NC \bf mode \NC \bf call \NC \bf result \NC\NR
%D    \HL 
%D    \NC 0\NC\asciistr{\XMLpar   {fx:root} {crap} {}}       \NC dirt  \NC\NR
%D    \NC  \NC\asciistr{\XMLpar   {fx:root} {junk} {}}       \NC       \NC\NR
%D    \NC  \NC\asciistr{\XMLNSpar {fx:root} {fx}   {crap} {}}\NC       \NC\NR
%D    \NC  \NC\asciistr{\XMLNSpar {fx:root} {xml}  {lang} {}}\NC       \NC\NR
%D    \HL
%D    \NC 1\NC\asciistr{\XMLpar   {fx:root} {crap} {}}       \NC dirt  \NC\NR
%D    \NC  \NC\asciistr{\XMLpar   {fx:root} {junk} {}}       \NC junk  \NC\NR
%D    \NC  \NC\asciistr{\XMLNSpar {fx:root} {fx}   {crap} {}}\NC       \NC\NR
%D    \NC  \NC\asciistr{\XMLNSpar {fx:root} {xml}  {lang} {}}\NC       \NC\NR
%D    \HL
%D    \NC 2\NC\asciistr{\XMLpar   {fx:root} {crap} {}}       \NC dirt  \NC\NR
%D    \NC  \NC\asciistr{\XMLpar   {fx:root} {junk} {}}       \NC junk  \NC\NR
%D    \NC  \NC\asciistr{\XMLNSpar {fx:root} {fx}   {crap} {}}\NC       \NC\NR
%D    \NC  \NC\asciistr{\XMLNSpar {fx:root} {xml}  {lang} {}}\NC en    \NC\NR
%D    \HL
%D    \NC 3\NC\asciistr{\XMLpar   {fx:root} {crap} {}}       \NC dirt  \NC\NR
%D    \NC  \NC\asciistr{\XMLpar   {fx:root} {junk} {}}       \NC junk  \NC\NR
%D    \NC  \NC\asciistr{\XMLNSpar {fx:root} {fx}   {crap} {}}\NC rubish\NC\NR
%D    \NC  \NC\asciistr{\XMLNSpar {fx:root} {xml}  {lang} {}}\NC en    \NC\NR
%D    \HL
%D    \stoptabulate}
%D 
%D \remapXMLurlspace [http://www.w3.org/1999/XSL/Format] [fx]
%D \stopbuffer
%D 
%D \startbuffer[xml]
%D <fo:root xmlns:fo="http://www.w3.org/1999/XSL/Format"
%D           fo:crap="rubish"
%D           fo:junk="junk"
%D          xml:lang="en"
%D              crap="dirt" />
%D \stopbuffer
%D
%D \typebuffer[tex] \processTEXbuffer[tex] 
%D \typebuffer[xml] \processXMLbuffer[xml]

%D The previous macros were the basic parser and their working
%D is left to the imagination of the reader. These macros
%D will be improved over time. We use rather low level 
%D definitions so that the mappings will run as fast as 
%D possible. 

\bgroup \catcode`<=\@@active

\long\gdef\dododefineXMLsingular#1#2%
  {\long\setvalue{\@@XMLelement:#1/}{#2}}

\long\gdef\dododefineXMLcommand#1#2%
  {\long\setvalue{\@@XMLelement:#1/}{#2}%
   \long\setvalue{\@@XMLelement:#1}{#2}}

\long\gdef\dododefineXMLgrouped#1#2%
  {\long\setvalue{\@@XMLelement:#1}{\groupedcommand{#2}{}\bgroup}%
   \long\setvalue{\@@XMLelement:/#1}{\egroup}}

%\long\gdef\dododefineXMLargument#1#2% watch the {} around ##1
%  {\long\setvalue{\@@XMLelement:#1/}{#2{}}%
%   \long\setvalue{\@@XMLelement:#1}##1</#1>{#2{##1}}}

\long\gdef\dododefineXMLargument#1#2% watch the {} around ##1
  {\long\setvalue{\@@XMLelement:#1/}{#2{}}%
   \long\setvalue{\@@XMLelement:#1}{\redoXMLargument{#1}{#2}}}

\long\gdef\redoXMLargument#1#2%
  {\long\@EA\def\@EA\next\@EA##\@EA1\@EA<\@EA/\currentXMLelement>{#2{##1}}%
   \next}

%\long\gdef\dododefineXMLignore#1%
%  {\long\setvalue{\@@XMLelement:#1/}{}%
%   \long\setvalue{\@@XMLelement:#1}##1</#1>{}}

\long\gdef\dododefineXMLignore#1%
  {\long\setvalue{\@@XMLelement:#1/}{}%
   \long\setvalue{\@@XMLelement:#1}{\redoXMLignore{#1}}}

\long\gdef\redoXMLignore#1%
  {\long\@EA\def\@EA\next\@EA##\@EA1\@EA<\@EA/\currentXMLelement>{}%
   \next}

%\long\gdef\dododefineXMLpickup#1#2#3%
%  {\long\setvalue{\@@XMLelement:#1/}{#2#3}%
%   \long\setvalue{\@@XMLelement:#1}##1</#1>{#2##1#3}}

\long\gdef\dododefineXMLpickup#1#2#3%
  {\long\setvalue{\@@XMLelement:#1/}{#2#3}%
   \long\setvalue{\@@XMLelement:#1}{\redoXMLpickup{#1}{#2}{#3}}}

\long\gdef\redoXMLpickup#1#2#3%
  {\long\@EA\def\@EA\next\@EA##\@EA1\@EA<\@EA/\currentXMLelement>{#2##1#3}%
   \next}

\long\gdef\dododefineXMLenvironment#1#2#3%
  {\long\setvalue{\@@XMLelement:#1/}{#2#3}% 
   \long\setvalue{\@@XMLelement:#1}{#2}%
   \long\setvalue{\@@XMLelement:/#1}{#3}}

%\long\gdef\dododefineXMLsave#1%
%  {\letvalue{\@@XMLdata:#1}\longempty
%   \long\setvalue{\@@XMLelement:#1/}{\long\setvalue{\@@XMLdata:#1}{}}%
%   \long\setvalue{\@@XMLelement:#1}##1</#1>{\long\setvalue{\@@XMLdata:#1}{##1}}}

\long\gdef\dododefineXMLsave#1%
  {\letvalue{\@@XMLdata:#1}\longempty
   \long\setvalue{\@@XMLelement:#1/}{\long\setvalue{\@@XMLdata:#1}{}}%
   \long\setvalue{\@@XMLelement:#1}{\redoXMLsave{#1}}}

\gdef\redoXMLsave#1%
  {\long\@EA\def\@EA\next\@EA##\@EA1\@EA<\@EA/\currentXMLelement>%
     {\long\@EA\def\csname\@@XMLdata:#1\endcsname{##1}}%
   \next}

\long\gdef\dododefineXMLgsave#1%
  {\letvalue{\@@XMLdata:#1}\longempty
   \long\setvalue{\@@XMLelement:#1/}{\long\setgvalue{\@@XMLdata:#1}{}}%
   \long\setvalue{\@@XMLelement:#1}{\redoXMLgsave{#1}}}

\gdef\redoXMLgsave#1%
  {\long\@EA\def\@EA\next\@EA##\@EA1\@EA<\@EA/\currentXMLelement>%
     {\long\@EA\gdef\csname\@@XMLdata:#1\endcsname{##1}}%
   \next}

%\long\gdef\dododefineXMLenvironmentsave#1#2#3%
%  {\letvalue{\@@XMLdata:#1}\longempty
%   \long\setvalue{\@@XMLelement:#1/}{#2\long\setvalue{\@@XMLdata:#1}{}#3}%
%   \long\setvalue{\@@XMLelement:#1}##1</#1>{#2\long\setvalue{\@@XMLdata:#1}{##1}#3}}

\long\gdef\dododefineXMLenvironmentsave#1#2#3%
  {\letvalue{\@@XMLdata:#1}\longempty
   \long\setvalue{\@@XMLelement:#1/}{#2\long\setvalue{\@@XMLdata:#1}{}#3}%
   \long\setvalue{\@@XMLelement:#1}{\redoXMLenvironmentsave{#1}{#2}{#3}}}

\gdef\redoXMLenvironmentsave#1#2#3%
  {\long\@EA\def\@EA\next\@EA##\@EA1\@EA<\@EA/\currentXMLelement>%
     {#2\long\@EA\def\csname\@@XMLdata:#1\endcsname{##1}#3}%
   \next}

\long\gdef\dododefineXMLprocess#1%
  {\long\setvalue{\@@XMLelement:#1/}{}%
   \long\setvalue{\@@XMLelement:#1}{}%
   \long\setvalue{\@@XMLelement:/#1}{}}

\long\gdef\dododefineXMLnestedenvironment#1#2#3%
  {\long\setvalue{\@@XMLelement:#1}{\getXMLgroupedenvironment{#1}{#2}{#3}}}

\long\gdef\dododefineXMLnestedargument#1#2%
  {\long\setvalue{\@@XMLelement:#1}{\getXMLgroupedargument{#1}{#2}}}

\egroup

%D The high level definition macros.

\def\defineXMLsingular       {\dotripleempty\dodefineXMLsingular}
\def\defineXMLcommand        {\dotripleempty\dodefineXMLcommand}
\def\defineXMLgrouped        {\dotripleempty\dodefineXMLgrouped}
\def\defineXMLargument       {\dotripleempty\dodefineXMLargument}
\def\defineXMLignore         {\dotripleempty\dodefineXMLignore}
\def\defineXMLpickup         {\dotripleempty\dodefineXMLpickup}
\def\defineXMLenvironment    {\dotripleempty\dodefineXMLenvironment}
\def\defineXMLsave           {\dotripleempty\dodefineXMLsave}
\def\defineXMLenvironmentsave{\dotripleempty\dodefineXMLenvironmentsave}
\def\defineXMLprocess        {\dotripleempty\dodefineXMLprocess}

% push is (not yet) a real push, so:

\def\defineXMLpush           {\dotripleempty\dodefineXMLsave}
\def\defineXMLenvironmentpush{\dotripleempty\dodefineXMLenvironmentsave}

% goes for all types

\def\defineXMLnested           {\dotripleempty\dodefineXMLnestedenvironment}
\def\defineXMLnestedenvironment{\dotripleempty\dodefineXMLnestedenvironment}
\def\defineXMLnestedargument   {\dotripleempty\dodefineXMLnestedargument}

\long\def\dodefineXMLsingular[#1][#2][#3]#4%
  {\defineXMLmethod\dododefineXMLsingular{#1}{#2}{#3}{#4}{}}

\long\def\dodefineXMLcommand[#1][#2][#3]#4%
  {\defineXMLmethod\dododefineXMLcommand{#1}{#2}{#3}{#4}{}}

\long\def\dodefineXMLgrouped[#1][#2][#3]#4%
  {\defineXMLmethod\dododefineXMLgrouped{#1}{#2}{#3}{#4}{}}

\long\def\dodefineXMLargument[#1][#2][#3]#4%
  {\defineXMLmethod\dododefineXMLargument{#1}{#2}{#3}{#4}{}}

\long\def\dodefineXMLignore[#1][#2][#3]%
  {\defineXMLmethod\dododefineXMLignore{#1}{#2}{#3}{}{}}

\long\def\dodefineXMLpickup[#1][#2][#3]#4#5%
  {\defineXMLmethod\dododefineXMLpickup{#1}{#2}{#3}{#4}{#5}}

\long\def\dodefineXMLenvironment[#1][#2][#3]#4#5%
  {\defineXMLmethod\dododefineXMLenvironment{#1}{#2}{#3}{#4}{#5}}

\long\def\dodefineXMLsave[#1][#2][#3]%
  {\defineXMLmethod\dododefineXMLsave{#1}{#2}{#3}{}{}}

\long\def\dodefineXMLenvironmentsave[#1][#2][#3]#4#5%
  {\defineXMLmethod\dododefineXMLenvironmentsave{#1}{#2}{#3}{#4}{#5}}

\long\def\dodefineXMLprocess[#1][#2][#3]%
  {\defineXMLmethod\dododefineXMLprocess{#1}{#2}{#3}{}{}}

\long\def\dodefineXMLnestedenvironment[#1][#2][#3]#4#5%
  {\defineXMLmethod\dododefineXMLnestedenvironment{#1}{#2}{#3}{#4}{#5}}

\long\def\dodefineXMLnestedargument[#1][#2][#3]#4%
  {\defineXMLmethod\dododefineXMLnestedargument{#1}{#2}{#3}{#4}{}}

% [key=val]          => \presetXMLarguments{element} => default key/vals
% [blabla]           => \theXMLarguments{blabla}     => user key/vals
% [blabla] [key=val] => \presetXMLarguments{element} => default key/vals
%                       \theXMLarguments{blabla}     => user key/vals

% command element [map] [parlst] begin end

\long\def\defineXMLmethod#1#2#3#4#5#6% 
  {\ifsecondargument
     \setXMLarguments{#2}{#3}{#4}%
   \else
     \resetXMLarguments{#2}% new 
   \fi
   #1{#2}{#5}{#6}}

%D Arguments (attributes) \unknown

\long\def\setXMLarguments#1#2#3% element [tag] settings
  {\doifassignmentelse{#2}
     {\long\setvalue{\@@XMLpars:#1}{\getrawparameters[\@@XMLvariable:#1:][#2]}}
     {\long\setvalue{\@@XMLmap :#1}{#2}% later we can init vars by this name
      \doifsomething{#3}{\long\setvalue{\@@XMLpars:#1}{\getrawparameters[#2][#3]}}}}

\def\presetXMLarguments#1%
  {\csname\@@XMLpars:\rawXMLnamespace#1\endcsname} % == \getvalue{}

\prependtoks
  \presetXMLarguments\rawXMLidentifier
\to \everyXMLelement

\long\def\longempty{}
\long\def\longspace{ }

\def\doifelseXMLdata#1% always empty at start [gets a long assignment]
  {\@EA\ifx\csname\@@XMLdata:#1\endcsname\longempty
     \expandafter\secondoftwoarguments
   \else
     \expandafter\firstoftwoarguments
   \fi} 

\def\doifXMLdata#1% always empty at start [gets a long assignment]
  {\@EA\ifx\csname\@@XMLdata:#1\endcsname\longempty
     \expandafter\gobbleoneargument
   \else
     \expandafter\firstofoneargument
   \fi} 

\let\doifXMLdataelse\doifelseXMLdata 

\def\doifelseXMLempty#1% 
  {\@EA\ifx\csname\@@XMLdata:#1\endcsname\longempty
     \expandafter\firstoftwoarguments
   \else\@EA\ifx\csname\@@XMLdata:#1\endcsname\longspace
     \expandthree\firstoftwoarguments
   \else
     \expandthree\secondoftwoarguments
   \fi\fi} 

% test case: 
%
% \defineXMLenvironmentsave[test]
%    {}
%    {\message{[\XMLflush{test}]}
%     \message{\doifelseXMLdata {test}{}{no  }data}
%     \message{/}
%     \message{\doifelseXMLempty{test}{}{not }empty}
%     \wait}
% 
% \startXMLdata
% <test>xxx</test>
% <test></test>
% <test> </test>
% <test>   </test>
% <test>       </test>
% <test>   x    </test>
% \stopXMLdata

\def\XMLflush#1% one level
  {\csname\@@XMLdata:#1\endcsname}

\def\XMLflushdata#1% see m-steps for usage
  {\@EA\ifx\csname\@@XMLdata:#1\endcsname\longempty\else
     %\@EAEAEA\XMLdata\@EA\@EA\@EA{\csname\@@XMLdata:#1\endcsname}%
     \@EA\XMLdata\csname\@@XMLdata:#1\endcsname
   \fi}

\let\XMLpop    \XMLflush

\let\XMLpopdata\XMLflushdata

\def\XMLappend#1#2% let to empty expands to nothing -) 
  {\long\@EA\edef\csname\@@XMLdata:#1\endcsname
     {\csname\@@XMLdata:#1\endcsname#2}}

\def\XMLprepend#1#2% let to empty expands to nothing -) 
  {\long\@EA\edef\csname\@@XMLdata:#1\endcsname
     {#2\csname\@@XMLdata:#1\endcsname}}

\def\XMLerase#1%
  {\@EA\let\csname\@@XMLdata:#1\endcsname\longempty}

\def\XMLassign#1%
  {\long\@EA\def\csname\@@XMLdata:#1\endcsname}

\def\dontparseXMLelement#1>{}

\def\simplifyXMLelements{\let\parseXMLelement\dontparseXMLelement}

\def\defXMLstring#1#2%
  {\bgroup
   \enableXMLexpansion
   \simplifyXMLelements
   \let\getXMLentity\firstofoneargument
   \XMLrawentitiestrue
   \xdef\@@XML@@string{\csname\@@XMLdata:#2\endcsname}%
   \egroup
   \convertcommand\@@XML@@string\to#1}

\def\defXMLclean#1#2%
  {\bgroup
   \enableXMLexpansion
   \simplifyXMLelements
   \simplifyXMLentities
   \let\getXMLentity\expandedXMLentity % should this go in \simplify ? 
   \xdef\@@XML@@string{\csname\@@XMLdata:#2\endcsname}%
   \egroup
   \convertcommand\@@XML@@string\to#1}

\def\XMLshow#1%
  {\showvalue{\@@XMLdata:#1}}

\def\XMLunspace#1% kan sneller 
  {\@EA\ifx\csname\@@XMLdata:#1\endcsname\longempty\else
     \long\@EA\edef\csname\@@XMLdata:#1\endcsname
       {\@EAEAEA\dounspaced\csname\@@XMLdata:#1\endcsname\end}%
   \fi}

\chardef\asciispacecode=32

\def\defXMLlowerclean#1% lowercase ! evt tzt upper too
  {\bgroup
   \lccode`\#\asciispacecode
   \lccode`\$\asciispacecode
   \lccode`\%\asciispacecode
   \lccode`\\\asciispacecode
   \lccode`\^\asciispacecode
   \lccode`\_\asciispacecode
   \lccode`\{\asciispacecode
   \lccode`\}\asciispacecode
   \lccode`\|\asciispacecode
   \lccode`\~\asciispacecode
   \@EA\lowercase\@EA{\@EA\xdef\@EA#1\@EA{#1}}%
   \egroup}

\def\processXMLparelse#1#2#3#4%
  {\processaction
     [\XMLpar{#1}{#2}{}]
     [#3,\s!unknown=>{#4},\s!default={#4}]}

%D We can pick up key|/|value pairs, but we still need a way
%D to process these.

\def\mapXMLvalue#1#2#3% td align center -> middle
  {\setvalue{\@@XMLvalue:#1:#2:#3}}

\def\XMLvar#1#2#3% td align center
  {\ifcsname\@@XMLvariable:#1:#2\endcsname
     \XMLval{#1}{#2}{\csname\@@XMLvariable:#1:#2\endcsname}%
   \else
     \XMLval{#1}{#2}{#3}% evt inline code
   \fi}

% \def\XMLvar#1#2#3% td align center
%   {\XMLval{#1}{#2}{\ifcsname\@@XMLvariable:#1:#2\endcsname
%      \csname\@@XMLvariable:#1:#2\endcsname\else#3\fi}}

\def\XMLval#1#2#3%
  {\ifcsname\@@XMLvalue:#1:#2\endcsname
     \csname\@@XMLvalue:#1:#2\endcsname
   \else
     #3%
   \fi}

\def\XMLpar#1#2#3%
  {\ifcsname\@@XMLvariable:#1:#2\endcsname
     \csname\@@XMLvariable:#1:#2\endcsname
   \else
     #3%
   \fi}

\def\XMLNSpar#1#2#3#4% element namespace name default
  {\ifcsname\@@XMLvariable:#1:#2:#3\endcsname
     \csname\@@XMLvariable:#1:#2:#3\endcsname
   \else
     #4%
   \fi}

\def\setXMLpar#1#2%
  {\@EA\def\csname\@@XMLvariable:#1:#2\endcsname}

\def\TEXpar#1#2% 
  {\csname#1\interfaced{#2}\endcsname}

\let\texXMLpar\TEXpar % soon obsolete 

\let\XMLtex\TEXpar

% handy one 

\def\XMLtyp#1#2#3%
  {\ifcsname\@@XMLvariable:#1:#2\endcsname
     \@EA\convertcommand\csname\@@XMLvariable:#1:#2\endcsname\to\ascii
   \else
     \convertargument#3\to\ascii
   \fi
   \ascii}

\defineXMLsingular [begingroup]        {\begingroup}
\defineXMLsingular [endgroup]          {\endgroup}
\defineXMLsingular [gobblespacetokens] {\gobblespacetokens}
\defineXMLsingular [disableXML]        {\disableXML}

\def\XMLstr#1%
  {{\enableXML\scantokens{#1}\unskip}}

\def\XMLstr#1% test
  {\scantokens{\begingroup\enableXML#1<endgroup/>\gobblespacetokens}}

\def\XMLstrpar#1#2#3% test
  {\ifcsname\@@XMLvariable:#1:#2\endcsname
     \scantokens\@EAEAEA{\@EA\begingroup\@EA\enableXML
       \csname\@@XMLvariable:#1:#2\endcsname<endgroup/>}%
   \else
     \scantokens{\begingroup\enableXML#3<endgroup/>}%
   \fi}

\def\doifXMLvarelse#1#2% geen etex, \relax too
  {\ifcsname\@@XMLvariable:#1:#2\endcsname     
%\expandafter\ifx\csname\@@XMLvariable:#1:#2\endcsname\relax
%  \message{#1 #2 -> relax}\wait 
%\fi
     \expandafter\ifx\csname\@@XMLvariable:#1:#2\endcsname\empty
       \@EAEAEA\secondoftwoarguments
     \else
       \@EAEAEA\firstoftwoarguments
     \fi
   \else
     \@EA\secondoftwoarguments
   \fi}

\def\doifXMLvar#1#2% geen etex, \relax too
  {\ifcsname\@@XMLvariable:#1:#2\endcsname     
     \expandafter\ifx\csname\@@XMLvariable:#1:#2\endcsname\empty
       \@EAEAEA\gobbleoneargument
     \else
       \@EAEAEA\firstofoneargument
     \fi
   \else
     \@EA\gobbleoneargument
   \fi}

\def\doifXMLvalelse#1#2% geen etex, \relax too
  {\ifcsname\@@XMLvalue:#1:#2\endcsname     
     \expandafter\ifx\csname\@@XMLvalue:#1:#2\endcsname\empty
       \@EAEAEA\secondoftwoarguments
     \else
       \@EAEAEA\firstoftwoarguments
     \fi
   \else
     \@EA\secondoftwoarguments
   \fi}

\let\doifXMLparelse\doifXMLvarelse
\let\doifXMLpar    \doifXMLvar

\bgroup \catcode`<=\@@active

\long\gdef\ignoreuntilXMLelement#1<{<}
\long\gdef\grabuntilXMLelement  #1<\to#2{\def#2{#1}<}

\egroup

%D Saves tokens and typing.

\def\XMLownvar        {\XMLvar        {\rawXMLnamespace\rawXMLidentifier}}
\def\XMLownval        {\XMLval        {\rawXMLnamespace\rawXMLidentifier}}
\def\XMLownpar        {\XMLpar        {\rawXMLnamespace\rawXMLidentifier}}
\def\XMLownstrpar     {\XMLstrpar     {\rawXMLnamespace\rawXMLidentifier}}
\def\doifXMLownvarelse{\doifXMLvarelse{\rawXMLnamespace\rawXMLidentifier}}
\def\doifXMLownvalelse{\doifXMLvalelse{\rawXMLnamespace\rawXMLidentifier}}
\def\doifXMLownparelse{\doifXMLparelse{\rawXMLnamespace\rawXMLidentifier}}

\def\ownXMLelement{\rawXMLnamespace\rawXMLidentifier}

\def\XMLop#1% ownpar 
  {\csname\@@XMLvariable:\ownXMLelement:#1\endcsname}

\def\XMLtp#1% texpar 
  {\csname\@@XMLmapmap\interfaced{#1}\endcsname}

\def\doifelseXMLop#1{\doifelse{\XMLop{#1}}}
\def\doifXMLop    #1{\doif    {\XMLop{#1}}}
\def\doifnotXMLop #1{\doifnot {\XMLop{#1}}}

\def\doifelsenothingXMLop#1{\doifelsenothing{\XMLop{#1}}}
\def\doifsomethingXMLop  #1{\doifsomething  {\XMLop{#1}}}
\def\doifnothingXMLop    #1{\doifnothing    {\XMLop{#1}}}

\def\doifelseXMLtp#1{\doifelse{\XMLtp{#1}}}
\def\doifXMLtp    #1{\doif    {\XMLtp{#1}}}
\def\doifnotXMLtp #1{\doifnot {\XMLtp{#1}}}

\def\doifelsenothingXMLtp#1{\doifelsenothing{\XMLtp{#1}}}
\def\doifsomethingXMLtp  #1{\doifsomething  {\XMLtp{#1}}}
\def\doifnothingXMLtp    #1{\doifnothing    {\XMLtp{#1}}}

\def\XMLflushself{\csname\@@XMLdata:\ownXMLelement\endcsname}

\def\XMLta      {\theXMLarguments\@@XMLmapmap}
\def\getXMLta   {\expanded{\getparameters[\@@XMLmapmap][\XMLta]}}
\def\expandXMLta{\expandXMLarguments\@@XMLmapmap}
\def\expandXMLtp{\expandTEXpar\@@XMLmapmap} % #1

\def\defXMLop#1#2{\@EA\let\@EA#1\csname\@@XMLvariable:\ownXMLelement:#2\endcsname}
\def\defXMLtp#1#2{\@EA\let\@EA#1\csname\@@XMLmapmap\interfaced{#2}\endcsname}

%D ...

\def\protectXMLdata
  {\catcode`\^^I\@@space
   \catcode`\^^M\@@space
   \catcode`\^^L\@@space
   \catcode`\#\@@other}

\long\def\startXMLcode
  {\begingroup
   \protectXMLdata
   \dostartXMLcode}
   
\long\def\dostartXMLcode[#1] #2 \stopXMLcode
  {\@EA\gdef\csname\@@XMLcode:#1\endcsname{\startXMLdata#2\stopXMLdata}%
   \endgroup}

\def\getXMLcode[#1]% \expandXMLcode
  {\csname\@@XMLcode:#1\endcsname}

% \long\def\startXMLdata#1\stopXMLdata%
%   {\begingroup\enableXML\scantokens{#1}\endgroup}
%
% \defineXMLentity[tex-backslash] {\catchXMLpar}
%
% \def\catchXMLpar#1#2#3
%   {\if#1p\if#2a\if#3r\ifmmode\else\endgraf\fi
%    \else\texescape\fi\else\texescape\fi\else\texescape\fi}

\long\def\startXMLdata
  {\begingroup
   \protectXMLdata
   \dostartXMLdata}

\long\def\dostartXMLdata#1\stopXMLdata
  {\enableXML\scantokens{#1<gobblespacetokens/>}%
   \endgroup
   \ifhmode\unskip\unskip\fi}

% suboptimal:
%
% \unexpanded\def\XMLdata#1% % \unexpanded added 22/5/2001
%   {\begingroup
%    \enableXML\scantokens{#1}\ifhmode\unskip\unskip\fi
%    \endgroup}
%
% better but does not work in tables:
%
% \unexpanded\def\XMLdata#1% % grouping changed 20/5/2001
%   {\scantokens{\begingroup\enableXML#1<endgroup/>\gobblespacetokens}}
%
% currently:

\unexpanded\def\XMLdata % # safe 
  {\begingroup
   \protectXMLdata
   \doXMLdata}

\def\doXMLdata#1%
   {\enableXML
    \scantokens{#1<gobblespacetokens/>}%
    \endgroup}

%D

\def\bXMLs{\ifignoreXMLspaces\ignorespaces\fi}
\def\eXMLs{\ifignoreXMLspaces\ifhmode\unskip\fi\fi}

\protect

% \defineXMLcommand{placeindex/}
%   {\placeindex[criterium=all]}
%
% \defineXMLargument{index}
%   {\index[\XMLvar{index}{key}{}]}

%D Here we implement the handling of preformatted code.

\unprotect

\def\startXMLpreformatted#1%
  {\startpacked
   #1%
   \fixedXMLfonttrue
   \obeylines
   \obeyspaces
   \setbox\scratchbox=\hbox{x}%
   \edef\obeyedspace{\noindent\noexpand\kern\the\wd\scratchbox}}

\def\stopXMLpreformatted#1%
  {\stoppacked}

%D

\def\XMLinput{\enableXML\input} \global\let\inputXML\XMLinput

% options

\def\processXMLfile       #1{\enableXML\processfile{#1}}
\def\processXMLfilegrouped#1{{\enableXML\processfile{#1}\relax\ifmmode\else\par\fi}}

%D \type
%D   {processXMLbuffer}
%D
%D For illustrative purposes, we need to be able to reuse
%D definitions, which is why we implement a buffer processor
%D here. The macro \type {\processXMLbuffer} behaves like
%D any buffer processor.

\def\processXMLbuffer%
  {\dosingleempty\doprocessXMLbuffer}

\def\doprocessXMLbuffer[#1]%
  {\doifelsenothing{#1}
     {\doprocessXMLbuffer[\jobname]}
     {\begingroup
      \def\dodoprocessXMLbuffer##1%
        {\enableXML\processXMLfile{\TEXbufferfile{##1}}}%
      \processcommalist[#1]\dodoprocessXMLbuffer
      \endgroup}}

%D Loading specific modules takes place with \type
%D {\useXMLfilters}.

% todo: flag 

\def\useXMLfilter[#1]%
  {\processcommalist[#1]\douseXMLfilter}

\def\douseXMLfilter#1%
  {\doifundefined{\c!file\f!xtagprefix#1}
     {\setvalue{\c!file\f!xtagprefix#1}{}%
      \makeshortfilename[\f!xtagprefix#1]%
      \writestatus{xml}{loading module #1}% will be \showmessage
      \startreadingfile
      \readsysfile{\shortfilename}{}{}%
      \stopreadingfile}}

%D Temporarily here.

\newtoks\groupedtoks
\newif\ifcollectXMLgrouped

\bgroup \catcode`\<=\@@active

\newtoks\XMLgtoks

\long\unexpanded\gdef\getXMLgroupedenvironment#1#2#3%
  {\collectXMLgroupedtrue
   \XMLgtoks{#2}%
   \long\def\dodogetgrouped{\@EA\the\@EA\XMLgtoks\the\groupedtoks#3}%
   \getXMLgrouped{#1}}

\long\unexpanded\gdef\getXMLgroupedargument#1#2%
  {\collectXMLgroupedtrue
   \XMLgtoks{#2}%
   \long\def\dodogetgrouped{\@EA\the\@EA\XMLgtoks\@EA{\the\groupedtoks}}%
   \getXMLgrouped{#1}}

\long\unexpanded\gdef\getXMLgroupedignore#1%
  {\collectXMLgroupedfalse
   \let\dodogetgrouped\relax
   \getXMLgrouped{#1}}

%\long\gdef\docountXMLgrouped#1\end#2\end % @@ => \relax\relax
%  {\long\def\dosplitstring##1#1##2@@##3\end%
%     {\def\ascii{##2}%
%      \ifx\ascii\empty \else
%        \advance\scratchcounter 1
%        \dosplitstring##2@@#1@@\end
%      \fi}%
%   \dosplitstring#2@@#1@@\end}

\long\gdef\docountXMLgrouped#1\end#2\end % 1 relax is enough since it's
  {\long\def\dosplitstring##1#1##2\relax\relax##3\end % another regime 
     {\def\ascii{##2}%
      \ifx\ascii\empty \else
        \advance\scratchcounter \plusone
        \dosplitstring##2\relax\relax#1\relax\relax\end
      \fi}%
   \dosplitstring#2\relax\relax#1\relax\relax\end}

\long\unexpanded\gdef\getXMLgrouped#1% #1 kan weg % klopt dit nu? 
  {\groupedtoks\emptytoks
   \scratchcounter=0
   \edef\theXMLnamespace
     {\ifx\originalXMLnamespace\empty\else\originalXMLnamespace:\fi
      \currentXMLidentifier}%
   \expanded{\long\noexpand\def\noexpand\dogetgrouped####1\noexpand</\currentXMLelement>}%
     {\ifcollectXMLgrouped\appendtoks##1\to\groupedtoks\fi
      \@EA\docountXMLgrouped\@EA<\theXMLnamespace>\end##1\end
      \@EAEAEA\docountXMLgrouped\@EA\@EA\@EA<\@EA\theXMLnamespace\space \end##1\end
      \ifcase\scratchcounter
        \let\dogetgrouped\dodogetgrouped
      \else
        \advance\scratchcounter \minusone 
        \ifcollectXMLgrouped\@EA\appendtoks\@EA<\@EA/\currentXMLelement>\to\groupedtoks\fi
      \fi
      \dogetgrouped}%
   \dogetgrouped}

\egroup

% interesting and fully expandable

\def\XMLownifequalelse#1#2%
  {\@EAEAEA\@@ifequal\csname\@@XMLvariable:\ownXMLelement:#1\endcsname\relax\@@and#2\relax\@@then}

% \def\XMLifequalelse#1#2#3%
%   {\@EAEAEA\@@ifequal\csname\@@XMLvariable:#1:#2\endcsname\relax\@@and#3\relax\@@then}

\def\XMLifequalelse#1#2%
  {\ifcsname\@@XMLvariable:#1:#2\endcsname
     \@EAEAEA\doXMLifequalelse\@EA\@EA\csname\@@XMLvariable:#1:#2\endcsname
   \else
     \@EA\secondoftwoarguments
   \fi}

\def\doXMLifequalelse#1#2%
  {\@EA\@@ifequal#1\relax\@@and#2\relax\@@then}

%D \starttypen 
%D \defineXMLenvironment[test][a=1]
%D   {\XMLownifequalelse{a}{2}{YES}{NO}}
%D   {}
%D 
%D \defineXMLenvironment[test][a=1]
%D   {\XMLifequalelse{test}{a}{1}{YES}{NO}}
%D   {}
%D 
%D \startXMLdata
%D <test a="1">test</test>
%D \stopXMLdata
%D \stoptypen  

\def\XMLyes#1{\XMLownifequalelse{#1}{yes}{#1}{}}

\def\expifequalelse#1#2%
  {\@@ifequal#1\relax\relax\@@and#2\relax\relax\@@then}

\def\@@ifequal#1#2\@@and#3%
  {\ifx#1\relax
     \ifx#3\relax
       \@EAEAEA\@@if@@equal@@true
     \else
       \@EAEAEA\@@if@@equal@@false
     \fi
   \else
     \ifx#3\relax
       \@EAEAEAEAEAEA\@@if@@equal@@false
     \else\ifx#1#3%
       % go on 
     \else
       \@EAEAEAEAEAEA\@@if@@equal@@false
     \fi\fi
   \fi
   \@@ifequal#2\@@and} 

\long\def\@@if@@equal@@true #1\@@then#2#3{#2}
\long\def\@@if@@equal@@false#1\@@then#2#3{#3}

%D new stuff :

\def\partialexpanded#1%
   {\let\notexpanded\noexpand
    \edef\@@expanded{\noexpand#1}%
    \let\notexpanded\empty
    \@@expanded}

\protect \endinput
