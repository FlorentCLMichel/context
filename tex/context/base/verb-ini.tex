%D \module
%D   [       file=verb-ini,
%D        version=1997.12.22,
%D          title=\CONTEXT\ Verbatim Macros,
%D       subtitle=Initialization,
%D         author=Hans Hagen,
%D           date=\currentdate,
%D      copyright={PRAGMA / Hans Hagen \& Ton Otten}]
%C
%C This module is part of the \CONTEXT\ macro||package and is
%C therefore copyrighted by \PRAGMA. Non||commercial use is
%C granted.

%D Because this module is quite independant of system macros,
%D it can be used as a stand||alone verbatim environment.
%D
%D This is a sort of second release of \type{supp-ver} and
%D therefore differs in some aspects from the implementation
%D published in the \MAPS. The first change concern
%D optimization of breaks, that is, the first and last two
%D lines of verbatim blocks are kept together. The second
%D adaption is due to the fact that I wanted to support pretty
%D printing not only for \TEX\ sources, but also for \PERL,
%D \METAPOST\ and probably more. The \JAVASCRIPT\ module is 
%D closely related to \PERL, so we will not mention that one 
%D again.

\ifx \undefined \writestatus \input supp-mis.tex \fi

%D Verbatim typesetting, especially of \TEX\ sources, is a
%D non||trivial task. This is a direct results of the fact that
%D characters can have \CATCODES\ other than~11 and such
%D characters needs a special treatment. What for instance is
%D \TEX\ supposed to do when it encounters a \type{$} or an
%D \type{#}? This module deals with these matters.

\writestatus{loading}{Context Verbatim Macros / Initialization}

%D The verbatim environment has some features, like coloring
%D \TEX\ text, seldom found in other environments. Especially
%D when the output of \TEX\ is viewed on an electronic medium,
%D coloring has a positive influence on the readability of
%D \TEX\ sources, so we found it very acceptable to dedicate
%D half of this module to typesetting \TEX\ specific character
%D sequences in color. In this module we'll also present some
%D macro's for typesetting inline, display and file verbatim.
%D The macro's are capable of handling \TAB\ too.
%D
%D This module shows a few tricks that are often overseen by
%D novice, like the use of the \TEX\ primitive \type{\meaning}.
%D First I'll show in what way the users are confronted with
%D verbatim typesetting. Because we want to be able to test for
%D symmetry and because we hate the method of closing down the
%D verbatim mode with some strange active character, we will
%D use the following construction for display verbatim:
%D
%D \starttypen
%D \starttyping
%D The Dutch word 'typen' stands for 'typing', therefore in the Dutch version
%D one will not find the word 'verbatim'.
%D \stoptyping
%D \stoptypen
%D
%D In \CONTEXT\ files can be typed with \type{\typefile} and
%D inline verbatim can be accomplished with \type{\type}. This
%D last command comes in many flavors:
%D
%D \starttypen
%D We can say \type<<something>> or \type{something}. The first one is a bit
%D longer but also supports slanted typing, which accomplished by typing
%D \type<<a <<slanted>> word>>. We can also use commands to enhance the text
%D \type<<with <</bf boldfaced>> text>>. Just to be complete, we decided
%D to accept also \LaTeX\ alike verbatim, which means that \type+something+
%D and \type|something| are valid commands too. Of course we want the grouped
%D alternatives to process \type{hello {\bf big} world} with braces.
%D \stoptypen
%D
%D In the core modules, we will build this support on top of
%D this module. There these commands can be tuned with
%D accompanying setup commands. There we can enable commands,
%D slanted typing, control spaces, \TAB||handling and (here we
%D are:) coloring. We can also setup surrounding white space
%D and indenting. Here we'll only show some examples.

\unprotect

%D \macros
%D   {verbatimfont}
%D   {}
%D
%D When we are typesetting verbatim we use a non||proportional
%D (mono spaced) font. Normally this font is available by
%D calling \type{\tt}. In \CONTEXT\ this command does a
%D complete font||style switch. There we could have stuck with
%D \type{\tttf}.

\ifx \undefined \verbatimfont \def\verbatimfont {\tt} \fi

%D \macros
%D   {obeyedspace, obeyedtab, obeyedline, obeyedpage}
%D   {}
%D
%D We have followed Knuth in naming macros that make \SPACE,
%D \NEWLINE\ and \NEWPAGE\ active and assigning them
%D \type{\obeysomething}, but first we set some default values.

\def\obeyedspace {\hbox{ }}
\def\obeyedtab   {\obeyedspace}
\def\obeyedline  {\par}
\def\obeyedpage  {\vfill\eject}

%D \macros
%D   {controlspace,setcontrolspaces}
%D   {}
%D
%D First we define \type{\obeyspaces}. When we want visible
%D spaces (control spaces) we only have to adapt the definition
%D of \type{\obeyedspace} to:

\def\controlspace {\hbox{\char32}}

\bgroup
\catcode`\ =\@@active
\gdef\obeyspaces{\catcode`\ =\@@active\def {\obeyedspace}}
\gdef\setcontrolspaces{\catcode`\ =\@@active\def {\controlspace}}
\egroup

%D \macros
%D   {obeytabs, obeylines, obeypages,
%D    ignoretabs, ignorelines, ignorepages}
%D   {}
%D
%D Next we take care of \NEWLINE\ and \NEWPAGE\ and because we
%D want to be able to typeset listings that contain \TAB, we
%D have to handle those too. Because we have to redefine the
%D \NEWPAGE\ character locally, we redefine the meaning of
%D this (often already) active character.

\catcode`\^^L=\@@active \def^^L{\par}

%D The following indirect definitions enable us to implement 
%D all kind of \type{\obeyed} handlers.

\bgroup

\catcode`\^^I=\@@active
\catcode`\^^M=\@@active
\catcode`\^^L=\@@active

\gdef\obeytabs    {\catcode`\^^I=\@@active\def^^I{\obeyedtab}}
\gdef\obeylines   {\catcode`\^^M=\@@active\def^^M{\obeyedline}}
\gdef\obeypages   {\catcode`\^^L=\@@active\def^^L{\obeyedpage}}

\gdef\ignoretabs  {\catcode`\^^I=\@@active\def^^I{\obeyedspace}}
\gdef\ignorelines {\catcode`\^^M=\@@active\def^^M{\obeyedspace}}
\gdef\ignorepages {\catcode`\^^L=\@@active\def^^L{\obeyedline}}

\egroup

%D \macros
%D   {obeycharacters}
%D   {}
%D
%D We also predefine \type{\obeycharacters}, which will
%D enable us to implement character||specific behavior, like
%D colored verbatim.

\let\obeycharacters=\relax

%D \macros
%D   {settabskips}
%D   {}
%D
%D The macro \type{\settabskip} can be used to enable tab
%D handling. Processing tabs is sometimes needed when one
%D processes a plain \ASCII\ listing. Tab handling slows down
%D verbatim typesetting considerably.

\bgroup

\catcode`\^^I=\@@active

\gdef\settabskips%
  {\let\processverbatimline=\doprocesstabskipline
   \catcode`\^^I=\@@active
   \let^^I=\doprocesstabskip}

\egroup

%D \macros
%D   {processinlineverbatim}
%D   {}
%D
%D Although the inline verbatim commands presented here will be
%D extended and embedded in the core modules of \CONTEXT,
%D they can be used separately. Both grouped and character
%D alternatives are provided but \type{<<} and nested
%D braces are implemented in the core module. This commands
%D takes one argument: the closing command.
%D
%D \starttypen
%D \processinlineverbatim{\closingcommand}
%D \stoptypen
%D
%D One can define his own verbatim commands, which can be very
%D simple:
%D
%D \starttypen
%D \def\Verbatim {\processinlineverbatim\relax}
%D \stoptypen
%D
%D or a bit more more complex:
%D
%D \starttypen
%D \def\GroupedVerbatim%
%D   {\bgroup
%D    \dosomeusefullthings
%D    \processinlineverbatim\egroup}
%D \stoptypen
%D
%D Before entering inline verbatim mode, we take care of the
%D unwanted \TAB, \NEWLINE\ and \NEWPAGE\ characters and
%D turn them into \SPACE. We need the double \type{\bgroup}
%D construction to keep the closing command local.

\def\setupinlineverbatim%
  {\verbatimfont
   \let\obeytabs=\ignoretabs
   \let\obeylines=\ignorelines
   \let\obeypages=\ignorepages
   \setupcopyverbatim}

\def\doprocessinlineverbatim%
  {\ifx\next\bgroup
     \setupinlineverbatim
     \catcode`\{=\@@begingroup
     \catcode`\}=\@@endgroup
     \def\next{\let\next=}%
   \else
     \setupinlineverbatim
     \def\next##1{\catcode`##1=\@@endgroup}%
   \fi
   \next}

\def\processinlineverbatim#1%
  {\bgroup
   \localcatcodestrue % TeX processes paragraph's
   \def\endofverbatimcommand{#1\egroup}%
   \bgroup
   \aftergroup\endofverbatimcommand
   \futurelet\next\doprocessinlineverbatim}

%D The closing command is executed afterwards as an internal
%D command and therefore should not be given explicitly when
%D typesetting inline verbatim.

%D \macros
%D   {processingverbatim}
%D   {}
%D
%D Typesetting a file in most cases results in more than one
%D page. Because we don't want problems with files that are
%D read in during the construction of the page, we set
%D \type{\ifprocessingverbatim}, so the output routine can
%D adapt its behavior. Originally we used
%D \type{\scratchread}, but because we want to support nesting,
%D we decided to use a separate input file.

\newif\ifprocessingverbatim

%D \macros
%D   {optimizeverbatim}
%D
%D One day, a collegue asked me why I didn't prevent breaking
%D after a first or before a last verbatim line. At first sight
%D I thought of using the two pass mechanism, but because we're
%D already keeping track of individual lines, a more direct
%D solution is possible: we just keep track of in what line
%D we are. One can turn this feature off.

\newif\ifoptimizeverbatim \optimizeverbatimtrue

%D Before we implement display and file verbatim, we define
%D some macros that deal with typesetting the individual lines.
%D We keep track of the status by means of a character
%D specification. This status variable tells us if we're
%D skipping a first line or placing a first or last line.
%D The next few examples show us where breaks are inserted.
%D
%D \bgroup
%D
%D \def\doverbatimnobreak%
%D   {\nobreak\hrule width 10cm\par\penalty500} % == \nobreak
%D
%D \def\doverbatimgoodbreak%
%D   {\nobreak\hrule width 3cm \par\penalty100} % >> \goodbreak
%D
%D \starttypen
%D test
%D test
%D \stoptypen
%D
%D or
%D
%D \starttypen
%D test
%D test
%D test
%D test
%D \stoptypen
%D
%D or
%D
%D \starttypen
%D test
%D
%D test
%D test
%D test
%D \stoptypen
%D
%D \egroup
%D
%D The long ones are \type{\nobreaks} and the short ones 
%D \type{\goodbreaks}. And this is how it's done:

\def\doverbatimnobreak%
  {\ifoptimizeverbatim\par\penalty500\fi} 

\def\doverbatimgoodbreak%
  {\ifoptimizeverbatim\par\penalty100\fi} 

\def\doflushverbatimline%
  {\expandafter\dodoverbatimline\expandafter{\savedverbatimline}}

\let\handleverbatimline=\relax

\def\initializeverbatimline%
  {\global\let\savedverbatimline=\empty
   \ifskipfirstverbatimline
     \global\chardef\verbatimstatus=0
   \else
     \global\chardef\verbatimstatus=1
   \fi}

\def\presetemptyverbatimline%
  {\ifcase\verbatimstatus
     \global\chardef\verbatimstatus=1
   \or
   \or
     % \ifoptimizeverbatim\else\doemptyverbatimline\fi
   \or
     \doflushverbatimline
     \global\let\savedverbatimline=\empty
     \doemptyverbatimline
     \global\chardef\verbatimstatus=2
   \else
     \doverbatimnobreak
     \doflushverbatimline
     \global\let\savedverbatimline=\empty
     \doemptyverbatimline
     \global\chardef\verbatimstatus=2
   \fi}

\def\presetnormalverbatimline%
  {\ifcase\verbatimstatus
     \global\chardef\verbatimstatus=2
   \or
     \global\chardef\verbatimstatus=3
   \or
     \global\chardef\verbatimstatus=3
   \or
     \doflushverbatimline
     \global\chardef\verbatimstatus=4
   \or
     \doverbatimnobreak
     \doflushverbatimline
     \global\chardef\verbatimstatus=5
   \or
     \doverbatimgoodbreak
     \doflushverbatimline
   \fi
   \global\let\savedverbatimline=\verbatimline}

\def\presetlastverbatimline%
  {\ifcase\verbatimstatus \or \or \or
     \doflushverbatimline
   \else
     \doverbatimnobreak
     \doflushverbatimline
   \fi}

%D \macros
%D   {skipfirstverbatimline}
%D
%D By default the rest of the first line is ignored. We can
%D turn this feature off by saying:
%D
%D \starttypen
%D \skipfirstverbatimlinefalse
%D \stoptypen

\newif\ifskipfirstverbatimline \skipfirstverbatimlinetrue

%D \macros
%D   {processdisplayverbatim}
%D   {}
%D
%D We can define a display verbatim environment with the
%D command \type{\processdisplayverbatim} in the following way:
%D
%D \starttypen
%D \processdisplayverbatim{\closingcommand}
%D \stoptypen
%D
%D \noindent For instance, we can define a simple command like:
%D
%D \starttypen
%D \def\BeginVerbatim {\processdisplayverbatim{EndVerbatim}}
%D \stoptypen
%D
%D \noindent But we can also do more advance things like:
%D
%D \starttypen
%D \def\BeginVerbatim {\bigskip \processdisplayverbatim{\EndVerbatim}}
%D \def\EndVerbatim   {\bigskip}
%D \stoptypen
%D
%D When we compare these examples, we see that the backslash in
%D the closing command is optional. One is free in actually
%D defining a closing command. If one is defined, the command
%D is executed after ending verbatim mode.

\def\processdisplayverbatim#1%
  {\par
   \bgroup
   \escapechar=-1
   \xdef\verbatimname{\string#1}%
   \egroup
   \def\endofdisplayverbatim{\csname\verbatimname\endcsname}%
   \bgroup
   \parindent\!!zeropoint
   \ifdim\lastskip<\parskip
     \removelastskip
     \vskip\parskip
   \fi
   \parskip\!!zeropoint
   \processingverbatimtrue
   \linepartrue
   \expandafter\let\csname\verbatimname\endcsname=\relax
   \edef\endofverbatimcommand{\csname\verbatimname\endcsname}%
   \edef\endofverbatimcommand{\meaning\endofverbatimcommand}%
   \verbatimfont
   \setupcopyverbatim
   \initializeverbatimline
   \copyverbatimline}

%D We save the closing sequence in \type{\endofverbatimcommand}
%D in such a way that it can be compared on a line by line
%D basis. For the conversion we use \type{\meaning}, which
%D converts the line to non||expandable tokens. We reset
%D \type{\parskip}, because we don't want inter||paragraph
%D skips to creep into the verbatim source. Furthermore we
%D \type{\relax} the line||processing macro while getting the
%D rest of the first line. The initialization command
%D \type{\setupcopyverbatim} does just what we expect it to do:
%D it assigns all characters \CATCODE~11. Next we switch to
%D french spacing and call for obeyance.

\def\setupcopyverbatim%
  {\uncatcodecharacters
   \frenchspacing
   \obeyspaces
   \obeytabs
   \obeylines
   \obeypages
   \obeycharacters}

%D \macros
%D   {eightbitcharacters,
%D    setcatcodes,uncatcodespecials,
%D    uncatcodecharacters}
%D   {}
%D
%D As its name says, \type{\uncatcodecharacters} resets the
%D \CATCODE\ of characters. When we use an upper bound of
%D 127 or 255, depending in \type{\ifeightbitcharacters}. By
%D counting down, we only have to use one counter. The
%D macro \type{\setcatcodes} can be uses to set alternative
%D values. The macro \type{\resetspecialcharacters} resets
%D characters with special meanings. This macro is not used
%D in the verbatim macros, but is best defined in this module.

\def\doprocesscatcodes#1%
  {\ifeightbitcharacters
     \scratchcounter=255
   \else
     \scratchcounter=127
   \fi
   \loop
     \savecatcode
     #1\relax
     \advance\scratchcounter by -1
     \ifnum\scratchcounter>-1
   \repeat
   \let\savecatcode=\relax
   \let\restorecatcodes=\dorestorecatcodes}

\def\uncatcodespecials%
  {\doprocesscatcodes
     {\ifnum\catcode\scratchcounter=\@@letter\relax\else
        \catcode\scratchcounter=\@@other
      \fi}%
   \catcode`\   =\@@space
   \catcode`\^^L=\@@ignore
   \catcode`\^^M=\@@endofline
   \catcode`\^^?=\@@ignore}

\def\setcatcodes#1%
  {\doprocesscatcodes
     {\catcode\scratchcounter=#1}}

\def\uncatcodecharacters%
  {\setcatcodes\@@letter}

%D \macros
%D   {localcatcodes,
%D    restorecatcodes,
%D    beginrestorecatcodes,endrestorecatcodes}
%D   {}
%D
%D We're not finished dealing \CATCODES\ yet. In \CONTEXT\ we
%D use only one auxiliary file, which deals with tables of
%D contents, registers, two pass tracking, references etc. This
%D file, as well as files concerning graphics, is processed when
%D needed, which can be in the mid of typesetting verbatim.
%D However, when reading in data in verbatim mode, we should
%D temporary restore the normal \CATCODES, and that's exactly
%D what the next macros do. Saving the catcodes can be
%D disabled by saying \type{\localcatcodestrue}.

%D The previous macros call for \type{\savecatcode}, which is
%D implemented as:

\newif\iflocalcatcodes

\def\savecatcode%
  {\iflocalcatcodes \else
     \expandafter\chardef\csname @@cc@@\the\scratchcounter\endcsname
       =\catcode\scratchcounter
   \fi}

%D It's counterpart is:

\def\restorecatcode%
  {\expandafter\catcode\expandafter\scratchcounter\expandafter=
     \csname @@cc@@\the\scratchcounter\endcsname}

%D When we want to restore \CATCODES\ we call for
%D \type{\restorecatcodes}, which default to \type{\relax}

\let\restorecatcodes=\relax

%D or when we've saves things calls for:

\def\dorestorecatcodes%
  {\iflocalcatcodes \else
     \doprocesscatcodes\restorecatcode
   \fi}

%D We also provide an alternative, that forces grouping
%D when needed. An application of this macros can be found in
%D buffering data.

\def\beginrestorecatcodes%
  {\ifx\restorecatcodes\relax
     \let\endrestorecatcodes=\relax
   \else
     \bgroup
     \restorecatcodes
     \let\beginrestorecatcodes=\bgroup
     \let\endrestorecatcodes  =\egroup
   \fi}

%D The main copying routine of display verbatim does an
%D ordinary string||compare on the saved closing command and
%D the current line. The space after \type{#1} in the
%D definition of \type{\next} is essential! As a result of
%D using \type{\obeylines}, we have to use \type{%}'s after
%D each line but none after the first \type{#1}.

{\obeylines%
 \long\gdef\copyverbatimline#1
   {\def\next{#1 }%
    \gdef\verbatimline{#1}%
    \ifx\next\emptyspace%
      \presetemptyverbatimline%
    \else%
      \edef\next{\meaning\next}%
      \ifx\next\endofverbatimcommand%
        \presetlastverbatimline%
        \def\copyverbatimline{\egroup\endofdisplayverbatim}%
      \else%
        \presetnormalverbatimline%
      \fi%
    \fi%
    \handleverbatimline%
    \copyverbatimline}}

%D The actual typesetting of a line is done by a separate
%D macro, which enables us to implement \TAB\ handling. The
%D \type{\do} and \type{\dodo} macros take care of the
%D preceding \type{\parskip}, while skipping the rest of the
%D first line. The \type{\relax} is used as an signal.

%D \macros
%D   {iflinepar}
%D   {}
%D
%D A careful reader will see that \type{\linepar} is reset.
%D This boolean can be used to determine if the current line is
%D the first line in a pseudo paragraph and this boolean is set
%D after each empty line. The \type{\relax} can be used to
%D determine the end of the line when one implements a scanner
%D routine.

\newif\iflinepar

\long\def\dodoverbatimline#1%
  {\leavevmode\the\everyline\strut\processverbatimline{#1\relax}%
   \EveryPar{}%
   \lineparfalse
   \obeyedline\par}

%D \macros
%D   {obeyemptylines,verbatimbaselineskip}
%D   {}
%D
%D Empty lines in verbatim can lead to white space on top of
%D a new page. Because this is not what we want, we turn
%D them into vertical skips. This default behavior can be
%D overruled by:
%D
%D \starttypen
%D \obeyemptylines
%D \stoptypen
%D
%D Although it would cost us only a few lines of code, we
%D decided not to take care of multiple empty lines. When a
%D (display) verbatim text contains more successive empty
%D lines, this probably suits some purpose. When applicable,
%D one can set the verbatim baselineskip.

\bgroup
\catcode`\^^L=\@@active  \gdef\emptypage  {^^L}
\catcode`\^^M=\@@active  \gdef\emptyline  {^^M}
                         \gdef\emptyspace { }
\egroup

\def\verbatimbaselineskip% We don't use \let here!
  {\baselineskip}

\def\doemptyverbatimline%
  {\vskip\verbatimbaselineskip
   {\setbox0=\hbox{\the\everyline}}%
   \linepartrue}

\def\obeyemptylines%
  {\def\doemptyverbatimline{\doverbatimline}}

%D \TEX\ does not offer \type{\everyline}, which is a direct
%D result of its advanced multi||pass paragraph typesetting
%D mechanism. Because in verbatim mode paragraphs and lines are
%D more or less equal, we can easily implement our own simple
%D \type{\everyline} support.

%D \macros
%D   {EveryPar, EveryLine}
%D   {}
%D
%D In this module we've reserved \type{\everypar} for the
%D things to be done with paragraphs and \type{\everyline} for
%D line specific actions. In \CONTEXT\ however, we use
%D \type{\everypar} for placing side- and columnfloats,
%D inhibiting indentation and some other purposes. In verbatim
%D mode, every line becomes a paragraph, which means that
%D \type{\everypar} is executed frequently. To be sure, the
%D user specific use of both \type{\everyline} and
%D \type{\everypar} is implemented by means of
%D \type{\EveryLine} and \type{\EveryPar}.
%D
%D We still have to take care of the \TAB. A \TAB\ takes eight
%D spaces and a \SPACE\ normally has a width of 0.5~em. Because
%D we can be halfway a tabulation, we must keep track of the
%D position. This takes time, especially when we print complete
%D files, therefore we \type{\relax} this mechanism by default.

\def\doprocesstabskip%
  {\obeyedspace % \hskip.5em  or  \hbox to .5em{}
   \ifdone
     \advance\scratchcounter by 1
     \let\next=\doprocesstabskip
     \donefalse
   \else\ifnum\scratchcounter>7
     \let\next=\relax
   \else
     \advance\scratchcounter 1
     \let\next=\doprocesstabskip
   \fi\fi
   \next}

\def\dodoprocesstabskipline#1#2\endoftabskipping%
  {\ifnum\scratchcounter>7
     \scratchcounter=1
     \donetrue
   \else
     \advance\scratchcounter 1
     \donefalse
   \fi
   \ifx#1\relax
     \let\next=\relax
   \else
     \def\next{#1\dodoprocesstabskipline#2\endoftabskipping}%
   \fi
   \next}

\let\endoftabskipping=\relax

\def\processverbatimline#1{#1} % remove the fake grouping

\def\doprocesstabskipline#1%
  {\bgroup
   \scratchcounter=1
   \dodoprocesstabskipline#1\relax\endoftabskipping
   \egroup}

%D \macros
%D   {processfileverbatim}
%D   {}
%D
%D The verbatim typesetting of files is done on a bit different
%D basis. This time we don't check for a closing command, but
%D look for \EOF\ and when we've met, we make sure it does not
%D turn into an empty line.
%D
%D \starttypen
%D \processfileverbatim{filename}
%D \stoptypen
%D
%D We reserve a dedicated file handle.

\newread\verbatiminput

\def\processfileverbatim#1%
  {\par
   \bgroup
   \parindent\!!zeropoint
   \ifdim\lastskip<\parskip
     \removelastskip
     \vskip\parskip
   \fi
   \parskip\!!zeropoint
   \processingverbatimtrue
   \linepartrue
   \uncatcodecharacters
   \verbatimfont
   \frenchspacing
   \obeyspaces
   \obeytabs
   \obeylines
   \obeypages
   \obeycharacters
   \openin\verbatiminput=#1
   \skipfirstverbatimlinefalse
   \initializeverbatimline
   \def\readline%
     {\read\verbatiminput to \verbatimline
      \ifeof\verbatiminput
        \presetlastverbatimline
        \let\readline=\relax
      \else\ifx\verbatimline\emptyline
        \presetemptyverbatimline
      \else\ifx\verbatimline\emptypage
        \presetemptyverbatimline
      \else
        \presetnormalverbatimline
      \fi\fi\fi
      \handleverbatimline
      \readline}%
   \ifeof\verbatiminput \else
     \expandafter\readline
   \fi
   \closein\verbatiminput
   \egroup
   \ignorespaces}

%D One can use the different \type{\obeysomething} commands to
%D influence the behavior of these macro's. We use for instance
%D \type{\obeycharacters} for making \type{/} an active
%D character when we want to include typesetting commands.

%D The next part of this module deals with pretty printing. The
%D best way to understand how pretty verbatim typeseting works
%D is to take a look at the output produced by the \TEX, \PERL\
%D and \METAPOST\ modules first. Each of these modules has a
%D few setup macros that tag the individual characters with a
%D number that itself is associated to a interpretation macro.
%D A previous implementation linked characters (after making
%D them active) directly to such interpreters, but the more
%D indirect way makes it possible to inspect the next
%D character(s) without much expansion problems and/or
%D increasing run time.

%D By the way, \TEX\ defines \type{\+} as an outer macro, so we 
%D have to redefine this one to keep ourselves out of complaints.

\def\+{\tabalign}

%D Just to keep things consistant and to speed up some macros a
%D but, we define a few private constants.

\def\!!PRETTY      {PRETTY}

\def\!!prettyone   {prettyone}
\def\!!prettytwo   {prettytwo}
\def\!!prettythree {prettythree}
\def\!!prettyfour  {prettyfour}

%D The first step in defining a pretty interpreter is to assign
%D each character that needs special attention a number, like:
%D
%D \starttypen
%D \setpretty \`A 21
%D \stoptypen
%D
%D Here the macro \type{\setpretty} makes the character
%D \type{A} active, and sets it meaning to the auxiliary macro
%D \type{\handleprettytoken}. This auxiliary macro takes the
%D character code (a number) and the interpretation number. The
%D three step implementation uses the \TEX book \type{~} trick.

\def\dodosetpretty%
  {\!!countb=\uccode`~\relax
   \catcode\!!countb=\@@active
   \uppercase{\edef~{\noexpand\handleprettytoken{\the\!!counta}{\the\!!countb}}}}

\def\dosetpretty%
  {\afterassignment\dodosetpretty\!!counta}

\def\setpretty%
  {\afterassignment\dosetpretty\uccode`~=}

%D The macro \type{\handleprettytoken} is rather trivial and
%D calls for an interpreter macro.

\def\handleprettytoken#1%
  {\getvalue{\!!PRETTY#1}}

%D This interpreter is installed by saying
%D
%D \starttypen
%D \installprettyhandler 21 \SOMEprettyone
%D \stoptypen

\def\installprettyhandler#1 #2%
  {\letvalue{\!!PRETTY#1}#2}

%D Such an interpreter gets the character number:
%D
%D \starttypen
%D \def\SOMEprettyone#1{...\getpretty{#1}...}
%D \stoptypen
%D
%D where \type{\getpretty} equals \type{\char}. We can't
%D use \type{\let} here because we have to get rid of the
%D braces.

\def\getpretty#1%
  {\char#1}

%D Sometimes the action depends on the next token. This token
%D can be passed to the macro \type{\getprettydata}, that sets
%D \type{\prettytype} to the interpreter code. The character
%D code is saved in \type{\prettychar}.

\def\getprettydata#1%
  {\bgroup
   \global\chardef\prettytype=0
   \global\chardef\prettychar=0
   \def\handleprettytoken##1##2%
     {\global\chardef\prettytype=##1\relax
      \global\chardef\prettychar=##2\relax}%
   \setbox0=\hbox{\global\chardef\prettychar=0#1}% expands #1 and ignores space
   \egroup}

%D If needed the current and next token can be handled alongside:

\def\getpretties#1#2%
  {\char#1\bgroup\let\handleprettytoken=\getsecondpretty#2\egroup}

\def\getsecondpretty#1#2%
  {\char#2}

%D When needed, one can reassign an interpreter by using
%D \type{\newpretty} and its associates.
%D
%D \bgroup
%D \steltypenin[file][optie=tex,palet=colorpretty]
%D
%D \startbuffer
%D \bgroup
%D \catcode`\|=\@@escape %%\|\
%D \catcode`\\=\@@active %%\\+
%D |gdef|dohandlenewpretty#1%
%D   {|def|dodohandlenewpretty##1%
%D      {|getprettydata{#1}%
%D       |let|oldprettychar=|prettychar
%D       |getprettydata{##1}%
%D       |ifnum|oldprettychar=|prettychar
%D         |def|dododohandlenewpretty####1%
%D           {|getprettydata{\}%
%D            |let|oldprettytype=|prettytype
%D            |getprettydata{####1}%
%D            |ifnum|prettytype=|oldprettytype
%D              |let|next=|newpretty
%D            |else
%D              |def|next{|newprettycommand{#1}##1####1}%
%D            |fi
%D            |next}%
%D         |let|next=|dododohandlenewpretty
%D       |else
%D         |def|next{|newprettycommand{#1}##1}%
%D       |fi
%D       |next}%
%D    |def|donohandlenewpretty##1%
%D      {|newprettycommand{#1}##1}%
%D    |handlenextnextpretty|dodohandlenewpretty|donohandlenewpretty}
%D |egroup
%D \stopbuffer
%D
%D {\newprettytrue\typebuffer}
%D
%D In this example we see that the colors differ from what we
%D expect, but conform the definitions in the macro. This kind
%D of recoloring can be achieved by saying:
%D
%D \typebuffer
%D
%D We'll show some more examples:
%D
%D \startbuffer
%D %%\R{Red
%D %%\G\Green
%D %%\B[Blue
%D \stopbuffer
%D
%D {\newprettytrue\typebuffer}
%D
%D Watch the green \type{\Green}! This lines are specified as:
%D
%D \typebuffer
%D
%D When needed, one can use grouping.
%D
%D \startbuffer
%D {yes} %%\ B %%\{[ %%\}]
%D {no}
%D {no}  %%\ E
%D {yes}
%D \stopbuffer
%D
%D \typebuffer
%D
%D This leads to:
%D
%D {\newprettytrue\typebuffer}
%D
%D I won't explain the details of this mechanism. Those who
%D want to build their own pretty interpreters have to close
%D read the source anyway.
%D
%D Last we show an example of mixed pretty typesetting:
%D
%D \startbuffer
%D \ziezo{test}          %%\ P   ##\ B##\ T % enter PERL mode       %%\ E
%D if $test eq "test"    ##\ B   ##\ B##\ T % begin group (\bgroup) %%\ E
%D if $test eq "test";   ##\ T   %%\ B%%\ T % enter TEX mode        %%\ E
%D \ziezo{test}          %%\ M   %%\ B%%\ T % enter METAPOST mode   %%\ E
%D draw (0,0)--(10,10);  %%\ E   ##\ B##\ T % end group (\egroup)   %%\ E
%D if $test eq "test";
%D \stopbuffer
%D
%D \typebuffer
%D
%D Here we use \type{%%\ T}, \type{%%\ P} and \type{%%\ M} for
%D switching between \TEX, \PERL\ and \METAPOST\ mode.
%D
%D {\newprettytrue\typebuffer}
%D
%D \egroup
%D 
%D Don't forget to set:

\newif\ifnewpretty

\def\installnewpretty%
  {\def\doinstallnewpretty##1%
     {\setvalue{NP::\the\scratchcounter}{##1}}%
   \afterassignment\doinstallnewpretty\scratchcounter=`}

%D Let's define the options we used here:

\let\prettyend  =\relax
\def\prettybegin{\bgroup\let\prettyend=\egroup}

\installnewpretty B \prettybegin
\installnewpretty E \prettyend

\installnewpretty J \setupprettyJVtype
\installnewpretty M \setupprettyMPtype
\installnewpretty P \setupprettyPLtype
\installnewpretty T \setupprettyTEXtype

%D This means that when the interpreter modules support this
%D mechanism, by default we have four options at available.

\def\newpretty#1%
  {\getprettydata{#1}%
   \ifnum\prettytype=0
     \expandafter\nonewpretty
   \else
     \expandafter\donewpretty
   \fi
   #1}

\def\nonewpretty#1#2%
  {\getprettydata{#2}%
   \getvalue{NP::\the\prettychar}}

\def\donewpretty#1%
  {\bgroup
   \def\handleprettytoken##1##2##3%
     {\getprettydata{##3}%
      \egroup
      \ifnum\prettytype>0
        \setpretty##2=\prettytype\relax
      \fi}%
   #1}

%D When implementing new pretty macros, one only needs to
%D define something like:
%D
%D \starttypen
%D \def\TEXtypezero%
%D   {\handlenewpretty\TEXtypethree}
%D \stoptypen
%D
%D Here the number states the category (in our examples the
%D backslash). The second argument takes care of normal
%D situations.

\def\handlenewpretty#1%
  {\let\newprettycommand=#1%
   \handlenextnextpretty\dohandlenewpretty\newprettycommand}

%D The previous shown implementation only interprets single
%D comments, but the final one also accepts double ones. The
%D main reason for this is that in \JAVA\ we have to deal with
%D \type{//}. Personally I prefer the double \type because
%D is stands out and is more symmetrical with the double
%D slash. 

\bgroup
\catcode`\|=\@@escape
\catcode`\\=\@@active
|gdef|dohandlenewpretty#1%
  {|def|dodohandlenewpretty##1%
     {|getprettydata{#1}%
      |let|oldprettychar=|prettychar
      |getprettydata{##1}%
      |ifnum|oldprettychar=|prettychar
        |def|dododohandlenewpretty####1%
          {|getprettydata{\}%
           |let|oldprettytype=|prettytype
           |getprettydata{####1}%
           |ifnum|prettytype=|oldprettytype
             |let|next=|newpretty
           |else
             |def|next{|newprettycommand{#1}##1####1}%
           |fi
           |next}%
        |let|next=|dododohandlenewpretty
      |else
        |def|next{|newprettycommand{#1}##1}%
      |fi
      |next}%
   |def|donohandlenewpretty##1%
     {|newprettycommand{#1}##1}%
   |handlenextnextpretty|dodohandlenewpretty|donohandlenewpretty}
|egroup

%D The `nextnext' macro we used in the previous definition
%D looks ahead. This is needed because individual lines are
%D handles by macro's and the next character can be something
%D that ends the line and/or does not belong to the verbatim
%D data.

\def\handlenextnextpretty#1#2#3%
  {\def\dohandlenextnextpretty%
     {\ifx\next\bgroup       % something {}
        \let\next=#2%
      \else\ifx\next\relax   % end of line / signal
        \let\next=#2%
      %\else\ifx\next\egroup % end of line / no signal
      %  \let\next=#2%
      %\else\ifx\next\par    % end of line / \par
        \let\next=#2%
      \else
        \let\next=#1%
      \fi\fi%\fi\fi
      \next{#3}}%
   \futurelet\next\dohandlenextnextpretty}

%D The pretty interpreters can (and will) change the meaning of
%D some controls. To enable them calling the originals we save
%D their meanings and to enable nesting we permit this only
%D once.

\def\saveprettycontrols%
  {\global\let\oldobeyedspace = \obeyedspace
   \global\let\oldobeyedline  = \obeyedline
   \global\let\oldobeyedpage  = \obeyedpage
   \let\saveprettycontrols    = \relax}

%D The \PERL\ and \METAPOST\ interpreters will also handle
%D reserved words. Sets of identifiers are defined like:
%D
%D \starttypen
%D \useprettyidentifiers \MODULAidentifiers \MODULAsetspecials
%D   if then else elsif case while do repeat until ...
%D \stoptypen
%D
%D New entries can be added to existing sets by repeatedly
%D using this command.

\def\useprettyidentifiers#1#2% \variable \presetcatcodes
  {\bgroup
   \ifx#1\undefined
     \global\let#1=\empty
   \fi
   \def\handleprettytoken##1##2{\char##2}%
   #2\relax
   \def\douseprettyidentifiers##1\par%
     {\xdef#1{\space#1\space ##1}%
      \egroup}%
   \douseprettyidentifiers}

%D We can test om identifiers with:

\def\doifprettyidentifierelse#1%
  {\doifincsnameelse{\space#1\space}}

%D \macros 
%D   {prettyidentifierfont,prettyvariablefont}
%D
%D When one want to typeset identifiers and system variables 
%D in a different typeface, one has to set the next two 
%D variables.  

\let\prettyidentifierfont=\relax
\let\prettyvariablefont  =\relax

%D The interpreter macros are loaded at run time. The main
%D reason lays in the fact that we don't want to have the
%D identifier lists hard coded in the format file. To prevent
%D repetitive loading, one should load the modules before the
%D first call to the macros.
%D
%D \starttypen
%D \input verb-tex.tex
%D \input verb-pl.tex
%D \input verb-mp.tex
%D \input verb-jv.tex
%D \stoptypen
%D
%D In \CONTEXT\ we follow a different thread, and therefore all
%D macros in the additional modules use \type{\gdef}'s and
%D \type{\doglobal}'s. Manipulating \type{\globaldef} is
%D possible but leads to fuzzy situations.

%D The rest of this module is dedicated to non \CONTEXT\ users
%D and shows an example of an verbatim environment based on the
%D previous macros.
%D
%D The macro's can be used to construct the commands we
%D mentioned in the beginning of this documentation. We leave
%D this to the fantasy of the reader and only show some \PLAIN\
%D \TEX\ alternatives for display verbatim and listings. We
%D define three commands for typesetting inline text, display
%D text and files verbatim. The inline alternative also accepts
%D user supplied delimiters.
%D
%D \starttypen
%D \type{text}
%D
%D \starttyping
%D ... verbatim text ...
%D \stoptyping
%D
%D \typefile{filename}
%D \stoptypen
%D
%D We can turn on the options by:
%D
%D \starttypen
%D \controlspacetrue
%D \verbatimtabstrue
%D \prettyverbatimtrue
%D \stoptypen
%D
%D Here is the implementation:

\newif\ifcontrolspace   
\newif\ifverbatimtabs   
\newif\ifprettyverbatim 

\ifCONTEXT \else 

  \def\presettyping%
    {\ifcontrolspace
       \let\obeyspace=\setcontrolspace
     \fi
     \ifverbatimtabs
       \let\obeytabs=\settabskips
     \fi
     \ifprettyverbatim
       \let\obeycharacters=\setupprettytype
     \fi}

  \def\type%
    {\bgroup
     \presettyping
     \processinlineverbatim{\egroup}}

  \def\starttyping%
    {\bgroup
     \presettyping
     \processdisplayverbatim{\stoptyping}}

  \def\stoptyping%
    {\egroup}

  \def\typefile#1%
    {\bgroup
     \presettyping
     \processfileverbatim{#1}%
     \egroup}

\fi

%D In \CONTEXT\ users say things like:
%D
%D \starttypen
%D \definetyping [TEX]  [option=TEX]
%D \definetyping [MP]   [option=MP]
%D \definetyping [PL]   [option=PL]
%D \definetyping [JV]   [option=JV]
%D \stoptypen
%D
%D or
%D
%D \starttypen
%D \setuptyping[file][option=color]
%D \stoptypen
%D
%D That way \CONTEXT\ selects the appropriate interpreter
%D itself, but more on that in another module. In other
%D packages one can define:

\ifCONTEXT \else

  \input verb-tex
  \input verb-mp
  \input verb-pl
  \input verb-jv

  \def\startTEX%
    {\bgroup \everypar{}%
     \let\obeycharacters=\setupprettyTEXtype
     \processdisplayverbatim{\stopTEX}}

  \def\startMP%
    {\bgroup \everypar{}%
     \let\obeycharacters=\setupprettyMPtype
     \processdisplayverbatim{\stopMP}}

  \def\startPL%
    {\bgroup \everypar{}%
     \let\obeycharacters=\setupprettyPLtype
     \processdisplayverbatim{\stopPL}}

  \def\startJV%
    {\bgroup \everypar{}%
     \let\obeycharacters=\setupprettyJVtype
     \processdisplayverbatim{\stopJV}}

  \let\stopTEX=\egroup
  \let\stopMP =\egroup
  \let\stopPL =\egroup
  \let\stopJV =\egroup

\fi

%D This following poor mans implementation of color is based on
%D PostScript. One can of course use grayscales too. In the
%D core modules these macros are redefined to using the color
%D mechanism present in \CONTEXT.

\ifCONTEXT \else

  \def\setcolorverbatim%
    {\def\prettyone   {.9 .0 .0 }       % red
     \def\prettytwo   {.0 .8 .0 }       % green
     \def\prettythree {.0 .0 .9 }       % blue
     \def\prettyfour  {.8 .8 .6 }       % yellow
     \def\beginofpretty[##1]%
       {\special{ps:: \csname##1\endcsname setrgbcolor}}
     \def\endofpretty%
       {\special{ps:: 0 0 0 setrgbcolor}}} % black

  \def\setgrayverbatim%
    {\def\prettyone   {.30 }            % gray
     \def\prettytwo   {.45 }            % gray
     \def\prettythree {.60 }            % gray
     \def\prettyfour  {.75 }            % gray
     \def\beginofpretty[##1]%
       {\special{ps:: \csname##1\endcsname setgray}}
     \def\endofpretty%
       {\special{ps:: 0 setgray}}}         % black

\fi

%D One can redefine these two commands after loading this
%D module. When available, one can also use appropriate
%D font||switch macro's. We default to color.

\ifCONTEXT \else \setcolorverbatim \fi

%D Here come the commands that are responsible for entering and
%D leaving the two states. As we can see, they've got much in
%D common.

%D The previous version of this module was published in the
%D \MAPS\ of the dutch \TEX\ users group \NTG. In that article,
%D the verbatim part of the text was typeset with the following
%D commands for the examples:
%D
%D \starttypen
%D \def\starttypen% We simplify the \ConTeXt\ macro.
%D   {\bgroup
%D    \everypar{} % We disable some troublesome mechanisms.
%D    \advance\leftskip by 1em
%D    \processdisplayverbatim{\stoptypen}}
%D
%D \let\stoptypen=\egroup
%D \stoptypen
%D
%D We also used:
%D
%D \starttypen
%D \def\startdefinition%
%D   {\bgroup
%D    \everypar{} % Again we disable some troublesome mechanisms.
%D    \let\obeycharacters=\setupprettyTEXtype % See verb-tex.tex!
%D    \EveryPar{\showparagraphcounter}%
%D    \EveryLine{\showlinecounter}%
%D    \verbatimbodyfont
%D    \processdisplayverbatim{\stopdefinition}}
%D
%D \def\stopdefinition%
%D   {\egroup}
%D \stoptypen
%D
%D And because we have both \type{\EveryPar} and
%D \type{\EveryLine} available, we can implement a dual
%D numbering mechanism:
%D
%D \starttypen
%D \newcount\paragraphcounter
%D \newcount\linecounter
%D
%D \def\showparagraphcounter%
%D   {\llap
%D      {\bgroup
%D       \counterfont
%D       \hbox to 4em
%D         {\global\advance\paragraphcounter by 1
%D          \hss \the\paragraphcounter \hskip2em}%
%D       \egroup
%D       \hskip1em}}
%D
%D \def\showlinecounter%
%D   {\llap
%D      {\bgroup
%D       \counterfont
%D       \hbox to 2em
%D         {\global\advance\linecounter by 1
%D          \hss \the\linecounter}%
%D       \egroup
%D       \hskip1em}}
%D \stoptypen
%D
%D One may have noticed that the \type{\EveryPar} is only
%D executed once, because we consider each piece of verbatim
%D as one paragraph. When one wants to take the empty lines
%D into account, the following assignments are appropriate:
%D
%D \starttypen
%D \EveryLine
%D   {\iflinepar
%D      \showparagraphcounter
%D    \fi
%D    \showlinecounter}
%D \stoptypen
%D
%D In this case, nothing has to be assigned to \type{\EveryPar},
%D maybe except of just another extra numbering scheme. The
%D macros used to typeset this documentation are a bit more
%D complicated, because we have to take take 'long' margin
%D lists into account. When such a list exceeds the previous
%D pargraph we postpone placement of the paragraph number till
%D there's room. This way so it does not clash with the margin
%D words.

%D Normally such commands have to be embedded in a decent setup
%D structure, where options can be set at will.
%D
%D Now let's summarize the most important commands.
%D
%D \starttypen
%D \processinlineverbatim{\closingcommand}
%D \processdisplayverbatim{\closingcommand}
%D \processfileverbatim{filename}
%D \stoptypen
%D
%D We can satisfy our own specific needs with the following
%D interfacing macro's:
%D
%D \starttypen
%D \obeyspaces  \obeytabs  \obeylines  \obeypages  \obeycharacters
%D \stoptypen
%D
%D We can influence the verbatim environment with the following
%D macro:
%D
%D \starttypen
%D \obeyemptylines
%D \stoptypen
%D
%D Some needs are fulfilled already with:
%D
%D \starttypen
%D \setcontrolspace  \settabskips
%D \stoptypen
%D
%D lines can be enhanced with ornaments using:
%D
%D \starttypen
%D \everypar  \everyline  \iflinepar
%D \stoptypen
%D
%D and pretty verbatim support is implemented by:
%D
%D \starttypen
%D \beginofpretty[#1] ... \endofpretty
%D \stoptypen
%D
%D and some setup macro, like:
%D
%D \starttypen
%D \setupprettyIDENTIFIERtype
%D \stoptypen
%D
%D The color support macro can be redefined by the user. The
%D parameter \type{#1} can be one of the four 'fixed'
%D identifiers {\em prettyone}, {\em prettytwo}, {\em
%D prettythree} and {\em prettyfour}. We have implemented a
%D more or less general PostScript color support mechanism,
%D using \type{specials}. One can toggle between color and
%D grayscale with:
%D
%D \starttypen
%D \setgrayverbatim  \setcolorverbatim
%D \stoptypen

%D \macros
%D   {permitshiftedendofverbatim}
%D   {}
%D
%D We did not mention one drawback of the mechanism described
%D here. The closing command must start at the first position
%D of the line. In \CONTEXT\ we will not have this drawback,
%D because we can test if the end command is a substring of the
%D current line. The testing is done by two of the support
%D macros, which of course are not available in a stand alone
%D application of this module.

\ifCONTEXT 

  \let\doifendofverbatim=\doifelse

  \def\permitshiftedendofverbatim%
    {\let\doifendofverbatim=\doifinstringelse}

  \def\processdisplayverbatim#1%
    {\par
     \bgroup
     \escapechar=-1
     \xdef\verbatimname{\string#1}%
     \egroup
     \def\endofdisplayverbatim{\csname\verbatimname\endcsname}%
     \bgroup
     \parindent\!!zeropoint
     \ifdim\lastskip<\parskip
       \removelastskip
       \vskip\parskip
     \fi
     \parskip\!!zeropoint
     \processingverbatimtrue
     \expandafter\let\csname\verbatimname\endcsname=\relax
     \expandafter\convertargument\csname\verbatimname\endcsname
       \to\endofverbatimcommand
     \verbatimfont
     \setupcopyverbatim
     \initializeverbatimline
     \copyverbatimline}

  {\obeylines%
   \long\gdef\copyverbatimline#1
     {\convertargument#1 \to\next%
      \gdef\verbatimline{#1}%
      \ifx\next\emptyspace%
        \presetemptyverbatimline%
      \else%
        \doifendofverbatim{\endofverbatimcommand}{\next}%
          {\presetlastverbatimline%
           \def\copyverbatimline{\egroup\endofdisplayverbatim}}%
          {\presetnormalverbatimline}%
      \fi%
      \handleverbatimline%
      \copyverbatimline}}

\fi

\protect

\endinput
