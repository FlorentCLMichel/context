%D \module
%D   [       file=supp-mpe,
%D        version=1999.07.10,
%D          title=\CONTEXT\ Support Macros,
%D       subtitle=METAPOST Special Extensions,
%D         author=Hans Hagen,
%D           date=\currentdate,
%D      copyright={PRAGMA / Hans Hagen \& Ton Otten}]
%C
%C This module is part of the \CONTEXT\ macro||package and is
%C therefore copyrighted by \PRAGMA. See mreadme.pdf for
%C details.

%D This module is still experimental and deals with some
%D extensions to \METAPOST. When using \POSTSCRIPT\ output,
%D these extensions can be supplied by means of proper
%D preamble definitions, but when producing \PDF\ we have to
%D set up the appropriate datastructures ourselves. It acts as
%D a plug in into \type {supp-pdf}. As soon as we need more
%D extensions, we will generalize these macro. Some
%D definitions will move to the special drivers.

\writestatus{loading}{MetaPost Special Extensions}

%D We implement extensions by using the \METAPOST\ special
%D mechanism. Opposite to \TEX's specials, the \METAPOST\ ones
%D are flushed before or after the graphic data, but thereby
%D are no longer connected to a position.
%D
%D We implement specials by overloading the \type {fill}
%D operator. By counting the fills, we can let the converter
%D treat the appropriate fill in a special way. The
%D specification of the speciality can have two forms,
%D determined by the setting of a boolean variable:
%D
%D \starttypen
%D _inline_specials_ := false ; % comment like code (default)
%D _inline_specials_ := true  ; % command like code
%D \stoptypen
%D
%D When the specification is embedded as comment, it looks
%D like:
%D
%D \starttypen
%D %%MetaPostSpecial <size> <data> <number> <identifier>
%D \stoptypen
%D
%D The in||line alternative is more tuned for \POSTSCRIPT,
%D since it permits us to define a macro \type {special}.
%D
%D \starttypen
%D inline  : <data> <number> <identifier> <size> special
%D \stoptypen
%D
%D The \type {identifier} determines what to do, and the data
%D can be used to accomplish this. A type~2 shading function
%D has identifier~2. Alltogether, the number of parameters is
%D specified in \type {size}. The \type {number} is the number
%D of the fill that needs the special treatment. For a type~2
%D and~3 shaded fill, the datablock contains the following
%D data:
%D
%D \starttypen
%D from to n inner_r g b x y        outer_r g b x y
%D from to n inner_r g b x y radius outer_r g b x y radius
%D \stoptypen
%D
%D The implementation below saves the data on the stack in
%D a way similar to the macros in \type {supp-pdf.tex}, and
%D just overload a few already defined handlers. That way,
%D the existing macros are still generic. \voetnoot {Actually,
%D the macros here are just as generic.}
%D
%D Currently the only extension concerns shading, which is
%D accomplished by handling yet another value of \type
%D {\finiMPpath}. The recource disctionary is stored and
%D later picked up by the general \CONTEXT\ figure inclusion
%D macros.

\unprotect

%D The \type {%%MetaPostSpecials: version.revision signal} line
%D triggers this module into handling color specifications kind
%D of special. We need this safeguard for non||special
%D usage.

\chardef\MPspecialversion  = 0  % specials when >1
\chardef\MPspecialrevision = 0  % specials when >1
\chardef\MPspecialsignal   = 0  % passed on by graphic

\chardef\inlineMPspecials  = 1  % only needed for stack resetting

%D This macro handles the special definitions that are
%D passed as comment.

\def\dohandleMPspecialcomment#1
  {\setMPargument{#1}%
   \advance\scratchcounter -1
   \ifcase\scratchcounter\relax
     \handleMPspecialcommand
     \donetrue
     \doresetMPstack
     \let\handleMPsequence\dohandleMPsequence
     \expandafter\handleMPsequence
   \else
     \expandafter\dohandleMPspecialcomment
   \fi}

\def\handleMPspecialcomment #1 % number of arguments
  {\doresetMPstack
   \scratchcounter=#1\relax
   \ifcase\scratchcounter % when zero, inline shading is used
     \chardef\inlineMPspecials1
     \let\handleMPsequence\dohandleMPsequence
     \expandafter\handleMPsequence
   \else
     \chardef\inlineMPspecials0
     \expandafter\dohandleMPspecialcomment
   \fi}

%D When defined inline, we use another macro to handle the
%D definitions. Actually, this macro is called by the
%D previous ones.

\def\handleMPspecialcommand%
  {\ifcase\inlineMPspecials\or
     \advance\nofMParguments -1 % pop the size
   \fi
   \ifundefined\MPspecial
     \message{[unknown \MPspecial]}%
   \else
     \csname\MPspecial\endcsname
   \fi
   \ifcase\inlineMPspecials
     \doresetMPstack % 0
   \else
     \resetMPstack   % 1
   \fi}

%D This macro triggers special support. Currently, the
%D version and revision number are not used. Any version number
%D greater than zero will enable special support.

\def\handleMPspecialscomment #1.#2 #3 % version.revision signal
  {\doresetMPstack
   \chardef\MPspecialversion #1
   \chardef\MPspecialrevision#2
   \chardef\MPspecialsignal  #3
   \let\handleMPsequence\dohandleMPsequence
   \handleMPsequence}

%D In case of \PDF, we need to prepare resourcs.

\newtoks\MPstartresources
\newtoks\MPstopresources

\def\startMPresources%
  {\the\MPstartresources
   \ifx\currentPDFresources\empty\else
     \message{[unused resources]}%
   \fi
   \global\let\currentPDFresources\empty}

\def\stopMPresources%
  {\let\currentPDFresources\empty
   \the\MPstopresources}

%D Since colors are not subjected to transformations, we can
%D only use colors as signal. In our case, we use a dummy colored
%D path with a red color component of \type {0.n}, so \type
%D {0.001} is the first path and \type {0.010} the tenth. Since
%D \METAPOST strips trailing zeros, we have to padd the string.

\newif\ifMPcmyk

\ifx\normalhandleMPrgbcolor\undefined % in case we reload this module

  \let\normalhandleMPrgbcolor \handleMPrgbcolor
  \let\normalhandleMPcmykcolor\handleMPcmykcolor
  \let\normalhandleMPgraycolor\handleMPgraycolor

\fi

%D When we are using \CONTEXT, we will fall back to the
%D better color conversion routines. This also has the advantage
%D that we don't have to parse and convert the file. In this
%D alternative, \type {\!MP} is not (yet) supported. Because
%D we can (for efficiency reasons) turn off strokecolor,
%D something we cannot do in \METAPOST\ converted code.

\newif\ifPDFMPstrokecolor \PDFMPstrokecolortrue

%D [This code should move to meta-ini.]

\ifCONTEXT % we can use this for a better xgstate handling

  \def\checkPDFMPstrokecolor%
    {\ifPDFMPstrokecolor \PDFstrokecolortrue \fi}

  \def\normalhandleMPrgbcolor
    {{\checkPDFMPstrokecolor\execcolorR\gMPa1:\gMPa2:\gMPa3:0:0\od}}

  \def\normalhandleMPcmykcolor
    {{\checkPDFMPstrokecolor\execcolorC\gMPa1:\gMPa2:\gMPa3:\gMPa4:0:0\od}}

  \def\normalhandleMPgraycolor
    {{\checkPDFMPstrokecolor\execcolorS\gMPa1:0:0\od}}

\fi

% In the previous macros we use the special drivers. A more
% direct approach would have been:
%
% \def\doPDFstartrgbcolormode#1#2#3%
%   {\PDFcode{#1 #2 #3 rg #1 #2 #3 RG}}
%
% \def\doPDFstartcmykcolormode#1#2#3#4%
%   {\PDFcode{#1 #2 #3 #4 k #1 #2 #3 #4 K}}
%
% \def\doPDFstartgraycolormode#1%
%   {\PDFcode{#1 g #1 G}}
%
% \appendtoks
%   \let\dostartrgbcolormode \doPDFstartrgbcolormode
%   \let\dostartcmykcolormode\doPDFstartcmykcolormode
%   \let\dostartgraycolormode\doPDFstartgraycolormode
% \to \everyMPtoPDFconversion

%D Now we can handle special color signals. We only do this
%D when special are detected.

\def\MPrgbnumber#1{\expandafter\doMPrgbnumber#1000.0000\relax}
\def\doMPrgbnumber#1.#2#3#4#5\relax{#2#3#4}

%D The naive case looks like:
%D
%D \starttypen
%D \def\handleMPrgbcolor%
%D   {\setMPcolor
%D    \ifcase\MPspecialversion
%D      \resetMPcolor\normalhandleMPrgbcolor
%D    \else\ifnum\MPrgbnumber\lastMPrvalue=\MPspecialsignal
%D      % consider it to be a signal
%D    \else
%D      \resetMPcolor\normalhandleMPrgbcolor
%D    \fi\fi}
%D \stoptypen
%D
%D However, since we want \CMYK\ support, we will use the
%D following implementation:

% \def\setMPcolor
%   {\edef\lastMPrvalue{\gMPa1}%
%    \edef\lastMPgvalue{\gMPa2}%
%    \edef\lastMPbvalue{\gMPa3}}
%
% speed up:

\def\setMPcolor%
  {\edef\lastMPrvalue{\csname\@@MP01\endcsname}%
   \edef\lastMPgvalue{\csname\@@MP02\endcsname}%
   \edef\lastMPbvalue{\csname\@@MP03\endcsname}}

\def\zeroMPrgbvalue{0.0}

\def\resetMPcolor
  {\let\lastMPrvalue\zeroMPrgbvalue
   \let\lastMPgvalue\zeroMPrgbvalue
   \let\lastMPbvalue\zeroMPrgbvalue}

\resetMPcolor

% \def\handleMPrgbcolor%
%  {\ifcase\MPspecialversion
%     \normalhandleMPrgbcolor
%   \else
%     \setMPcolor
%     \ifnum\MPrgbnumber\lastMPrvalue=\MPspecialsignal\relax
%       \ifMPcmyk \interceptMPcmykcolor \fi
%     \else
%       \resetMPcolor\normalhandleMPrgbcolor
%     \fi
%   \fi}

\def\@@MPSK{@MPSK@}

\def\interceptMPcmykcolor % todo : \ifMPcmyk
  {\ifcase\MPrgbnumber\lastMPgvalue
     % cannot happen
   \or
     % 1 == cmyk color spec
     \ifMPcmyk \dointerceptMPcmykcolor \fi
   \or
     % 2 == rgb transparency
     \invokeMPtransparencyspecial
   \else
     % 3 == cmyk transparency
     \ifMPcmyk \invokeMPtransparencyspecial \fi
   \fi}

\def\dointerceptMPcmykcolor % can be used often, so sped up 
  {\revokeMPtransparencyspecial
   \@EA\ifx\csname\@@MPSK\number\MPrgbnumber\lastMPbvalue\endcsname\relax\else
     \@EA\@EA\@EA\setMPcmyk\csname\@@MPSK\number\MPrgbnumber\lastMPbvalue\endcsname
     \normalhandleMPcmykcolor
   \fi}

\def\handleMPrgbcolor
  {\resetMPcolor 
   \ifcase\MPspecialversion
     \normalhandleMPrgbcolor
   \else
     \setMPcolor
     \ifnum\MPrgbnumber\lastMPrvalue=\MPspecialsignal\relax
       \interceptMPcmykcolor
     \else
       \revokeMPtransparencyspecial 
       \normalhandleMPrgbcolor
     \fi
   \fi}

\def\handleMPgraycolor
  {\resetMPcolor 
   \ifcase\MPspecialversion \else \revokeMPtransparencyspecial \fi
   \normalhandleMPgraycolor}

\def\handleMPcmykcolor
  {\resetMPcolor 
   \ifcase\MPspecialversion \else \revokeMPtransparencyspecial \fi
   \normalhandleMPcmykcolor}

%D Specials are define and recalled using:

\def\MPspecial
  {MP special \gMPs\nofMParguments}

\def\defineMPspecial#1#2%
  {\setvalue{MP special #1}{#2}}

%D The path processing macro is slightly extended.

\newtoks \invokeMPspecials

\def\finishMPpath%
  {\PDFcode
     {\ifcase\finiMPpath W n\or S\or f\or B\else W n\fi
      \extraMPpathcode}}

\def\processMPpath%
  {\checkMPpath % !
   \ifcase\nofMPsegments\else
     \let\extraMPpathcode\empty
     \ifcase\MPspecialversion\else
       \ifnum\MPrgbnumber\lastMPrvalue=\MPspecialsignal
         \ifnum\MPrgbnumber\lastMPgvalue>10 % really needed
           \scratchcounter\MPrgbnumber\lastMPbvalue
           \edef\currentMPspecial{\the\scratchcounter}%
           \ifnum\finiMPpath=2 % to outer level
             \the\invokeMPspecials
           \fi
         \fi
       \fi
     \fi
     \flushMPpath
     \closeMPpath
     \finishMPpath
   \fi
   \let\handleMPsequence\dohandleMPsequence
   \resetMPstack
   \nofMPsegments0
   \handleMPsequence}

%D Shading is an example of a more advanced graphic feature,
%D but users will seldom encounter those complications. Here
%D we only show a few simple examples, but many other
%D alternatives are possible by setting up the functions built
%D in \PDF\ in the appropriate way.
%D
%D Shading has to do with interpolation between two or more
%D points or user supplied ranges. In \PDF, the specifications
%D of a shade has to be encapsulated in objects and passed on
%D as resources. This is a \PDF\ level 1.3. feature. One can
%D simulate three dimensional shades as well and define simple
%D functions using a limited set of \POSTSCRIPT\ primitives.
%D Given the power of \METAPOST\ and these \PDF\ features, we
%D can achieve superb graphic effects.
%D
%D Since everything is hidden in \TEX\ and \METAPOST\ graphics,
%D we can stick to high level \CONTEXT\ command, as shown in
%D the following exmples.
%D
%D \startbuffer
%D \startuniqueMPgraphic{CircularShade}
%D   path  p ; p := unitsquare xscaled \overlaywidth yscaled \overlayheight ;
%D   circular_shade(p,0,.2red,.9red) ;
%D \stopuniqueMPgraphic
%D
%D \startuniqueMPgraphic{LinearShade}
%D   path  p ; p := unitsquare xscaled \overlaywidth yscaled \overlayheight ;
%D   linear_shade(p,0,.2blue,.9blue) ;
%D \stopuniqueMPgraphic
%D
%D \startuniqueMPgraphic{DuotoneShade}
%D   path  p ; p := unitsquare xscaled \overlaywidth yscaled \overlayheight ;
%D   linear_shade(p,2,.5green,.5red) ;
%D \stopuniqueMPgraphic
%D \stopbuffer
%D
%D \typebuffer
%D
%D \haalbuffer
%D
%D These graphics can be hooked into the overlay mechanism,
%D which is available in many commands.
%D
%D \startbuffer
%D \defineoverlay[demo 1][\uniqueMPgraphic{CircularShade}]
%D \defineoverlay[demo 2][\uniqueMPgraphic  {LinearShade}]
%D \defineoverlay[demo 3][\uniqueMPgraphic {DuotoneShade}]
%D \stopbuffer
%D
%D \typebuffer
%D
%D \haalbuffer
%D
%D These backgrounds can for instance be applied to \type
%D {\framed}:
%D
%D \startbuffer
%D \setupframed[breedte=3cm,hoogte=2cm,kader=uit]
%D \startcombinatie[3*1]
%D   {\framed[achtergrond=demo 1]{\bfd \white Demo 1}} {}
%D   {\framed[achtergrond=demo 2]{\bfd \white Demo 2}} {}
%D   {\framed[achtergrond=demo 3]{\bfd \white Demo 3}} {}
%D \stopcombinatie
%D \stopbuffer
%D
%D \typebuffer
%D
%D \startregelcorrectie
%D \haalbuffer
%D \stopregelcorrectie
%D
%D There are a few more alternatives, determined by the second
%D parameter passed to \type {circular_shade} and alike.
%D
%D \def\SomeShade#1#2#3#4#5%
%D   {\startuniqueMPgraphic{Shade-#1}
%D      width := \overlaywidth ;
%D      height := \overlayheight ;
%D      path p ; p := unitsquare xscaled width yscaled height ;
%D      #2_shade(p,#3,#4,#5) ;
%D    \stopuniqueMPgraphic
%D    \defineoverlay[Shade-#1][\uniqueMPgraphic{Shade-#1}]%
%D    \framed[achtergrond=Shade-#1,breedte=2cm,hoogte=2cm,kader=uit]{}}
%D
%D \startregelcorrectie
%D \startcombinatie[5*1]
%D   {\SomeShade{10}{circular}{0}{.3blue}{.9blue}} {circular 0}
%D   {\SomeShade{11}{circular}{1}{.3blue}{.9blue}} {circular 1}
%D   {\SomeShade{12}{circular}{2}{.3blue}{.9blue}} {circular 2}
%D   {\SomeShade{13}{circular}{3}{.3blue}{.9blue}} {circular 3}
%D   {\SomeShade{14}{circular}{4}{.3blue}{.9blue}} {circular 4}
%D \stopcombinatie
%D \stopregelcorrectie
%D
%D \blanko
%D
%D \startregelcorrectie
%D \startcombinatie[5*1]
%D   {\SomeShade{20}{circular}{0}{.9green}{.3green}} {circular 0}
%D   {\SomeShade{21}{circular}{1}{.9green}{.3green}} {circular 1}
%D   {\SomeShade{22}{circular}{2}{.9green}{.3green}} {circular 2}
%D   {\SomeShade{23}{circular}{3}{.9green}{.3green}} {circular 3}
%D   {\SomeShade{24}{circular}{4}{.9green}{.3green}} {circular 4}
%D \stopcombinatie
%D \stopregelcorrectie
%D
%D \blanko
%D
%D \startregelcorrectie
%D \startcombinatie[4*1]
%D   {\SomeShade{30}{linear}{0}{.3red}{.9red}} {linear 0}
%D   {\SomeShade{31}{linear}{1}{.3red}{.9red}} {linear 1}
%D   {\SomeShade{32}{linear}{2}{.3red}{.9red}} {linear 2}
%D   {\SomeShade{33}{linear}{3}{.3red}{.9red}} {linear 3}
%D \stopcombinatie
%D \stopregelcorrectie
%D
%D These macros closely cooperate with the \METAPOST\ module
%D \type {mp-spec.mp}, which is part of the \CONTEXT\
%D distribution.
%D
%D The low level (\PDF) implementation is based on the \TEX\
%D based \METAPOST\ to \PDF\ converter. Shading is supported
%D by overloading the \type {fill} operator as implemented
%D earlier. In \PDF\ type~2 and~3 shading functions are
%D specified in terms of:
%D
%D \starttabulatie[|Tl|l|]
%D \NC /Domain \NC sort of meeting range \NC \NR
%D \NC /C0     \NC inner shade \NC \NR
%D \NC /C1     \NC outer shade \NC \NR
%D \NC /N      \NC smaller values, bigger inner circles \NC \NR
%D \stoptabulatie

\newcount\currentPDFshade  % 0  % global (document wide) counter
\let\currentMPshades\empty

\def\startMPshading#1%
  {\edef\currentMPspecial{\gMPs{#1}}}

\def\stopMPshading%
  {\global\advance\currentPDFshade 1
   \setxvalue{obj:Sh:\currentMPspecial}%
     {/Sh\the\currentPDFshade\space\the\pdflastobj\space0 R }%
   \setxvalue{mps:Sh:\currentMPspecial}%
     {\the\currentPDFshade}}

\appendtoks
  \global\let\currentMPshades\empty
\to \MPstartresources

\appendtoks
   \ifx\currentMPshades\empty \else
     \xdef\currentPDFresources{\currentPDFresources
       /Shading <<\currentMPshades>>}%
   \fi
\to \MPstopresources

\def\invokeMPshadespecial%
  {\doifdefined{mps:Sh:\currentMPspecial}
     {\edef\currentMPshade{\getvalue{obj:Sh:\currentMPspecial}}%
      \doifinstringelse{\currentMPshade}{\currentMPshades}
        {}{\xdef\currentMPshades{\currentMPshades\currentMPshade}}%
      \def\extraMPpathcode{/Sh\getvalue{mps:Sh:\currentMPspecial} sh Q}%
      \chardef\finiMPpath=0
      \PDFcode{q /Pattern cs}}}

\appendtoks \invokeMPshadespecial \to \invokeMPspecials

%D We need to convert the \CMYK\ specials into colors, because
%D we have to do it twice, we define a macro. 

\def\checkMPshadingcolor#1#2#3#4#5%
  {\edef\tempMPrvalue{\csname\@@MP0#1\endcsname}%
   \edef\tempMPgvalue{\csname\@@MP0#2\endcsname}%
   \edef\tempMPbvalue{\csname\@@MP0#3\endcsname}%
   \edef#5%
     {\ifx\tempMPrvalue\tempMPgvalue
        \ifx\tempMPrvalue\tempMPbvalue
          \ifx\tempMPgvalue\tempMPbvalue
            \tempMPbvalue
          \fi
        \fi   
      \fi}%
   \edef#4%
     {\ifnum\MPrgbnumber\tempMPrvalue=\MPspecialsignal\space
        \ifMPcmyk\getvalue{\@@MPSK\number\MPrgbnumber\tempMPbvalue}\fi
      \fi}}

%D We also need to make sure that we have two \RGB\ or 
%D \CMYK colors, since we have to set the colorspace. 

\def\setMPshadingcolors#1#2#3#4#5#6%
  {\checkMPshadingcolor{#1}{#2}{#3}\MPshadeA\MPshadeAA
   \checkMPshadingcolor{#4}{#5}{#6}\MPshadeB\MPshadeBB
   \ifx\MPshadeA\empty 
     \ifx\MPshadeB\empty 
       \edef\MPshadeA{\gMPs{#1} \gMPs{#2} \gMPs{#3}}%
       \edef\MPshadeB{\gMPs{#4} \gMPs{#5} \gMPs{#6}}%
       \def \MPshadeC{RGB}%   
     \else
       \ifx\MPshadeAA\empty
         \let\MPshadeA\MPcmykWhite
         \let\MPshadeB\MPcmykBlack
       \else
         \scratchdimen=1pt\advance\scratchdimen -\MPshadeAA pt
         \edef\MPshadeA{0 0 0 \withoutpt\the\scratchdimen}%
       \fi
       \def\MPshadeC{CMYK}%   
     \fi
   \else
     \ifx\MPshadeB\empty 
       \ifx\MPshadeBB\empty
         \let\MPshadeA\MPcmykWhite
         \let\MPshadeB\MPcmykBlack
       \else
         \scratchdimen=1pt\advance\scratchdimen -\MPshadeBB pt
         \edef\MPshadeB{0 0 0 \withoutpt\the\scratchdimen}%
       \fi
     \fi
     \def\MPshadeC{CMYK}%   
   \fi}

%D The reason why this macro is a bit complicates is that we
%D handle black and white situations (otherwise we would have
%D to use \CMYK\ b/w in case of a \CMYK\ shade). 

\def\MPcmykBlack{0 0 0 0} 
\def\MPcmykWhite{0 0 0 1} 

%D Here are the special handlers: 

\defineMPspecial{30}
  {\startMPshading{14}% type 2
   \setMPshadingcolors{4}{5}{6}{9}{10}{11}%
   \immediate\pdfobj
     {<</FunctionType 2
        /Domain [\gMPs1 \gMPs2]
        /C0 [\MPshadeA]
        /C1 [\MPshadeB]
        /N \gMPs3>>}%
   \immediate\pdfobj
     {<</ShadingType 2
        /ColorSpace /Device\MPshadeC\space
        /Function \the\pdflastobj\space 0 R
        /Coords [\gMPs7 \gMPs8 \gMPs{12} \gMPs{13}]
        /Extend [true true]>>}%
   \stopMPshading}

\defineMPspecial{31}
  {\startMPshading{16}% type 3
   \setMPshadingcolors{4}{5}{6}{10}{11}{12}%
   \immediate\pdfobj
     {<</FunctionType 2
        /Domain [\gMPs1 \gMPs2]
        /C0 [\MPshadeA]
        /C1 [\MPshadeB]
        /N \gMPs3>>}%
   \immediate\pdfobj
     {<</ShadingType 3
        /ColorSpace /Device\MPshadeC\space
        /Function \the\pdflastobj\space 0 R
        /Coords [\gMPs7 \gMPs8 \gMPs9 \gMPs{13} \gMPs{14} \gMPs{15}]
        /Extend [true true]>>}%
   \stopMPshading}

%D Figure inclusion is kind of strange to \METAPOST, but when
%D Santiago Muelas started discussing this with me, I was able
%D to cook up a solution using specials.

\def\invokeMPfigurespecial%
  {\getvalue{mps:gr:\currentMPspecial}} % or \relax

\appendtoks \invokeMPfigurespecial \to \invokeMPspecials

\defineMPspecial{10}
  {\setxvalue{mps:gr:\gMPs8}%
     {\noexpand\handleMPfigurespecial
        {\gMPs1}{\gMPs2}{\gMPs3}{\gMPs4}{\gMPs5}{\gMPs6}{\gMPs7}%
      \noexpand\setxvalue{mps:gr:\gMPs8}{}}}

% \def\handleMPfigurespecial#1#2#3#4#5#6#7%
%   {\vbox to 0pt
%      {\vss
%       \hbox to 0pt
%         {\pdfliteral{q #1 #2 #3 #4 #5 #6 cm}%
%          \pdfimage width 1bp height 1bp {#7}% maybe 10 is better
%          \pdfliteral{Q}%
%          \hss}}}
%
% better, since xform reuse

\def\handleMPfigurespecial#1#2#3#4#5#6#7% todo : combine with ext fig 
  {\vbox to \zeropoint
     {\vss
      \hbox to \zeropoint
        {\ifcase\pdfoutput\or % will be hooked into the special driver
           \doiffileelse{#7}
             {\doifundefinedelse{mps:x:#7}
                {\immediate\pdfximage width 1bp height 1bp {#7}%
                 \setxvalue{mps:x:#7}{\pdfrefximage\the\pdflastximage}}%
                {\message{[reusing figure #7]}}%
              \pdfliteral{q #1 #2 #3 #4 #5 #6 cm}%
              \rlap{\getvalue{mps:x:#7}}%
              \pdfliteral{Q}}
             {\message{[unknown figure #7]}}%
         \fi
         \hss}}}

%D An example of using both special features is the
%D following.
%D
%D \starttypen
%D \startMPpage
%D   externalfigure "hakker1b.png" scaled 22cm rotated  10 shifted (-2cm,0cm);
%D   externalfigure "hakker1b.png" scaled 10cm rotated -10 ;
%D   externalfigure "hakker1b.png" scaled  7cm rotated  45 shifted (8cm,12cm) ;
%D   path p ; p := unitcircle xscaled 15cm yscaled 20cm;
%D   path q ; q := p rotatedaround(center p,90) ;
%D   path r ; r := buildcycle(p,q) ; clip currentpicture to r ;
%D   path s ; s := boundingbox currentpicture enlarged 5mm ;
%D   picture c ; c := currentpicture ; currentpicture := nullpicture ;
%D   circular_shade(s,0,.2red,.9red) ;
%D   addto currentpicture also c ;
%D \stopMPpage
%D \stoptypen

%D This is some experimental hyperlink driver that I wrote
%D for Mark Wicks.

\defineMPspecial{20}
  {\setxvalue{mps:hl:\gMPs6}%
     {\noexpand\handleMPhyperlink
        {\gMPs1}{\gMPs2}{\gMPs3}{\gMPs4}{\gMPs5}%
      \noexpand\setxvalue{mps:hl:\gMPs6}{}}}

\def\handleMPhyperlink#1#2#3#4#5%
  {%\ifcase\pdfoutput\or
     \setbox\scratchbox=\hbox
       {\setbox\scratchbox=\null
        \scratchdimen=#1bp\scratchdimen=-\scratchdimen
        \advance\scratchdimen#3bp
        \wd\scratchbox=\scratchdimen
        \scratchdimen=#2bp\scratchdimen=-\scratchdimen
        \advance\scratchdimen#4bp
        \ht\scratchbox=\scratchdimen
        \incolorfalse
        \gotobox{\box\scratchbox}[#5]}%
     \setbox\scratchbox=\hbox
       {\scratchdimen\MPxoffset  bp \advance\scratchdimen#1bp
        \hskip\scratchdimen
        \scratchdimen=\MPyoffset bp \advance\scratchdimen#2bp
        \raise\scratchdimen\box\scratchbox}%
     \smashbox\scratchbox
     \box\scratchbox
   }%\fi}

\def\invokeMPhyperlinkspecial%
  {\getvalue{mps:hl:\currentMPspecial}} % or \relax

\appendtoks \invokeMPhyperlinkspecial \to \invokeMPspecials

%D Special number~1 is dedicated to \CMYK\ support. If you
%D want to know why: look at this:
%D
%D \startbuffer[mp]
%D   fill fullcircle xyscaled (3cm,1cm) withcolor \MPcolor{test} ;
%D \stopbuffer
%D
%D \startbuffer[cmyk]
%D \startcombinatie[4*1]
%D   {\definecolor[test][c=1,y=.3,k=.3] \processMPbuffer[mp]} {c=1 y=.3 k=.3}
%D   {\definecolor[test][c=.9,y=.15]    \processMPbuffer[mp]} {c=.9 y=.15}
%D   {\definecolor[test][c=.25,y=.8]    \processMPbuffer[mp]} {c=.25 y=.8}
%D   {\definecolor[test][c=.45,y=.1]    \processMPbuffer[mp]} {c=.45 y=.1}
%D \stopcombinatie
%D \stopbuffer
%D
%D \plaatsfiguur
%D   {\CMYK\ support disabled,
%D    conversion to \RGB.}
%D   {\setupcolors[cmyk=nee,status=start]\haalbuffer[cmyk]}
%D 
%D \plaatsfiguur
%D   {\CMYK\ support enabled,
%D    no support in \METAPOST.}
%D   {\setupcolors[cmyk=ja,mpcmyk=nee,status=start]\haalbuffer[cmyk]}
%D 
%D \plaatsfiguur
%D   {\CMYK\ support enabled,
%D    no conversion to \RGB,
%D    support in \METAPOST}
%D   {\setupcolors[cmyk=ja,status=start]\haalbuffer[cmyk]}

\defineMPspecial{1}
  {\ifMPcmyk
     \setxvalue{\@@MPSK\gMPs1}{\gMPs2 \gMPs3 \gMPs4 \gMPs5 }%
   \fi}
 
\def\setMPcmyk#1 #2 #3 #4 %
  {\setvalue{\@@MP01}{#1}%
   \setvalue{\@@MP02}{#2}%
   \setvalue{\@@MP03}{#3}%
   \setvalue{\@@MP04}{#4}}

%D This special (number 50) passes positions to a tex file.
%D This method uses a two||pass approach an (mis|)|used the
%D context positioning macros. In \type {core-pos} we will
%D implement the low level submacro needed.
%D
%D \startbuffer
%D \definelayer[test]
%D
%D \setlayer
%D   [test]
%D   [x=\MPx{somepos-1},y=\MPy{somepos-1}]
%D   {Whatever we want here!}
%D
%D \setlayer
%D   [test]
%D   [x=\MPx{somepos-2},y=\MPy{somepos-2}]
%D   {Whatever we need there!}
%D
%D \startuseMPgraphic{oeps}
%D   draw fullcircle scaled 6cm withcolor red ;
%D   register ("somepos-1",1cm,2cm,center currentpicture) ;
%D   register ("somepos-2",4cm,3cm,(-1cm,-2cm)) ;
%D \stopuseMPgraphic
%D
%D \framed[background=test,offset=overlay]{\useMPgraphic{oeps}}
%D \stopbuffer
%D
%D \typebuffer
%D
%D Here the width and height are not realy used, but one can
%D imagine situations where tex has to work with values
%D calculated by \METAPOST.
%D
%D \startregelcorrectie
%D \getbuffer
%D \stopregelcorrectie
%D
%D Later we will implement a more convenient macro:
%D
%D \starttypen
%D \setMPlayer [test] [somepos-1] {Whatever we want here!}
%D \setMPlayer [test] [somepos-2] {Whatever we need there!}
%D \stoptypen

\ifx\dosavepositionwhd\undefined
  \let\dosavepositionwhd\gobblesevenarguments
\fi

\defineMPspecial{50} % x y width height label
  {\bgroup
   \scratchdimen\MPllx bp \scratchdimen-\scratchdimen
   \advance\scratchdimen\gMPs1bp
   \edef\x{\number\scratchdimen}%
   \scratchdimen\gMPs2bp \scratchdimen-\scratchdimen
   \advance\scratchdimen\MPury bp
   \edef\y{\number\scratchdimen}%
   \scratchdimen\gMPs3bp
   \edef\w{\number\scratchdimen}%
   \scratchdimen\gMPs4bp
   \edef\h{\number\scratchdimen}%
   \dosavepositionwhd{\gMPs5}{0}{\x}{\y}{\w}{\h}{0}%
   \egroup}

%D Transparency support used specials 60 (rgb) and 61
%D (cmyk).
%D
%D \startbuffer
%D u := 2cm ; path p ; p := fullcircle scaled u shifted (u/4,0);
%D
%D fill p rotated  90 withcolor transparent(1,.5,yellow) ;
%D fill p rotated 210 withcolor transparent(1,.5,green) ;
%D fill p rotated 330 withcolor transparent(1,.5,blue) ;
%D \stopbuffer
%D
%D \typebuffer
%D
%D \startregelcorrectie \processMPbuffer \stopregelcorrectie
%D
%D One can also communicate colors between \CONTEXT\ and
%D \METAPOST:
%D
%D \startbuffer
%D \definecolor[tcyan]   [c=1,k=.2,t=.5]
%D \definecolor[tmagenta][m=1,k=.2,t=.5]
%D \definecolor[tyellow] [y=1,k=.2,t=.5]
%D \stopbuffer
%D
%D \typebuffer \getbuffer
%D
%D \startbuffer
%D u := 2cm ; path p ; p := fullcircle scaled u shifted (u/4,0);
%D
%D fill p rotated  90 withcolor \MPcolor{tcyan} ;
%D fill p rotated 210 withcolor \MPcolor{tmagenta} ;
%D fill p rotated 330 withcolor \MPcolor{tyellow} ;
%D \stopbuffer
%D
%D \startregelcorrectie \processMPbuffer \stopregelcorrectie
%D
%D We save all the three components needed in one macro,
%D just to save hash space.

\def\@@MPST{@MPST@}

\def\assignMPStransparency#1#2#3%
  {\edef\PDFtransparencyidentifier{#1}%
   \edef\PDFtransparencyreference {#2}%
   \edef\PDFtransparencycolorspecs{#3}}

\defineMPspecial{2}
  {\edef\currentMPspecial{\gMPs6}%
   \presetPDFtransparency{\gMPs1}{\gMPs2}%
   \setevalue{\@@MPST\currentMPspecial}% was \setxvalue, bug !
     {\noexpand\assignMPStransparency
        {\PDFtransparencyidentifier}%
        {\PDFtransparencyreference}%
        {\gMPs3 \gMPs4 \gMPs5 rg \gMPs3 \gMPs4 \gMPs5 RG}}}

\defineMPspecial{3}
  {\edef\currentMPspecial{\gMPs7}%
   \presetPDFtransparency{\gMPs1}{\gMPs2}%
   \setevalue{\@@MPST\currentMPspecial}% was \setxvalue, bug !
     {\noexpand\assignMPStransparency
        {\PDFtransparencyidentifier}%
        {\PDFtransparencyreference}%
        {\gMPs3 \gMPs4 \gMPs5 \gMPs6 k \gMPs3 \gMPs4 \gMPs5 \gMPs6 K}}}

%D For efficiency reasons, we fall back on the allocation
%D mechanisms already present. For use within \MPTOPDF, we
%D provide a fall back routine.

\let\currentMPtransparencies\empty

% \def\invokeMPtransparencyspecial
%   {\doifdefined{\@@MPST\currentMPspecial}
%      {\getvalue{\@@MPST\currentMPspecial}%
%       \doifinstringelse
%         {\PDFtransparencyidentifier}{\currentMPtransparencies}
%         {}
%         {\xdef\currentMPtransparencies
%            {\currentMPtransparencies
%             \PDFtransparencyidentifier\space
%             \PDFtransparencyreference\space}}%
%       \def\extraMPpathcode{ Q}%
%       \PDFcode{q
%         \PDFtransparencycolorspecs\space
%         \PDFtransparencyidentifier\space gs}}}

% this one triggers a new graphic state

\def\invokeMPtransparencyspecial
  {\scratchcounter\MPrgbnumber\lastMPbvalue
   \edef\currentMPspecial{\the\scratchcounter}%
   \ifundefined{\@@MPST\currentMPspecial}\else
     \getvalue{\@@MPST\currentMPspecial}%
     \doifinstringelse\PDFtransparencyidentifier\currentMPtransparencies
       \donothing\registerMPtransparencyresource % slow
     \PDFcode
       {\PDFtransparencycolorspecs\space
        \PDFtransparencyidentifier\space gs}%
     % potential optimization
     % \setevalue{\@@MPST\currentMPspecial}%
     %   {\PDFcode
     %      {\PDFtransparencycolorspecs\space
     %       \PDFtransparencyidentifier\space gs}}%
     % \getvalue{\@@MPST\currentMPspecial}%
     \let\revokeMPtransparencyspecial\dorevokeMPtransparencyspecial
   \fi}

% this one does a reset

\let\revokeMPtransparencyspecial\relax

\appendtoks 
  \revokeMPtransparencyspecial
\to \MPstopresources

\def\dorevokeMPtransparencyspecial % only called if state is set
  {\ifx\PDFtransparencyresetidentifier\empty\else
     \doifinstringelse\PDFtransparencyresetidentifier\currentMPtransparencies
       \donothing\registerMPtransparencyresetresource
     \PDFcode{\PDFtransparencyresetidentifier\space gs}%
     % potential optimization
     % \def\dorevokeMPtransparencyspecial
     %   {\PDFcode
     %      {\PDFtransparencycolorspecs\space
     %       \PDFtransparencyidentifier\space gs}}%
     \let\dorevokeMPtransparencyspecial\dodorevokeMPtransparencyspecial
     \let\revokeMPtransparencyspecial\relax % invoke sets it
   \fi}

% and this one a simplified reset

% \def\dodorevokeMPtransparencyspecial % used after first invocation
%   {\getvalue{\@@MPST0}%
%    \PDFcode{\PDFtransparencyidentifier\space gs}}

\def\dodorevokeMPtransparencyspecial % used after first invocation
  {\PDFcode{\PDFtransparencyresetidentifier\space gs}%
   \let\revokeMPtransparencyspecial\relax} % invoke sets it

% add a resource entry

\def\registerMPtransparencyresource
  {\xdef\currentMPtransparencies
     {\currentMPtransparencies
      \PDFtransparencyidentifier\space
      \PDFtransparencyreference\space}}

\def\registerMPtransparencyresetresource
  {\xdef\currentMPtransparencies
     {\currentMPtransparencies
      \PDFtransparencyresetidentifier\space
      \PDFtransparencyresetreference\space}}

\appendtoks \invokeMPtransparencyspecial \to \invokeMPspecials

\ifCONTEXT \else

  \def\@@MPSTN{@MPSTN@}
  \def\@@MPSTO{@MPSTO@}

  \newcount\PDFcurrenttransparency

  \let\PDFtransparencyresetidentifier\empty
  \let\PDFtransparencyresetreference \empty

  \let\PDFtransparencyidentifier\empty
  \let\PDFtransparencyreference \empty

  \def\initializePDFtransparency
    {\global\let\initializePDFtransparency\relax
     \presetPDFtransparency{1}{1}%
     \xdef\PDFtransparencyresetidentifier{/Tr0}%
     \xdef\PDFtransparencyresetreference{\the\pdflastobj\space 0 R}}

  \def\presetPDFtransparency#1#2%
    {\initializePDFtransparency
     \@EA\ifx\csname\@@MPSTO#1:#2\endcsname\relax
       \global\advance\PDFcurrenttransparency 1
       \immediate\pdfobj{\PDFtransparencydictionary{#1}{#2}{}}%
       \setxvalue{\@@MPSTN#1:#2}{\the\PDFcurrenttransparency}%
       \setxvalue{\@@MPSTO#1:#2}{\the\pdflastobj}%
     \fi
     \edef\PDFtransparencyidentifier{/Tr\getvalue{\@@MPSTN#1:#2}}%
     \edef\PDFtransparencyreference{\getvalue{\@@MPSTO#1:#2} 0 R}}

     \def\PDFtransparencydictionary#1#2#3% type fraction extras
       {<</Type /ExtGState 
          /ca #2 /CA #2
          /BM /\ifcase#1 Normal\or Normal\or Multiply\or Screen\or 
               Overlay\or SoftLight\or HardLight\or ColorDodge\or 
               ColorBurn\or Darken\or Lighten\or Difference\or 
               Exclusion\else Compatible\fi
          #3>>}
     
\fi

\appendtoks
  \global\let\currentMPtransparencies\empty
\to \MPstartresources

\appendtoks
  \ifx\currentMPtransparencies\empty \else
    \xdef\currentPDFresources{\currentPDFresources
       /ExtGState <<\currentMPtransparencies>>}%
  \fi
\to \MPstopresources

%D In all cases, we need to keep track of the resources
%D used.

\protect \endinput
