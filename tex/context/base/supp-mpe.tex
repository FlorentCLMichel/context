%D \module
%D   [       file=supp-mpe,
%D        version=1999.07.10,
%D          title=\CONTEXT\ Support Macros,
%D       subtitle=METAPOST Special Extensions,
%D         author=Hans Hagen,
%D           date=\currentdate,
%D      copyright={PRAGMA / Hans Hagen \& Ton Otten}]
%C
%C This module is part of the \CONTEXT\ macro||package and is
%C therefore copyrighted by \PRAGMA. See mreadme.pdf for 
%C details. 

%D This module is still experimental and deals with some
%D extensions to \METAPOST. When using \POSTSCRIPT\ output,
%D these extensions can be supplied by means of proper
%D preamble definitions, but when producing \PDF\ we have to
%D set up the appropriate datastructures ourselves. It acts as
%D a plug in into \type {supp-pdf}. As soon as we need more 
%D extensions, we will generalize these macro. 

\writestatus{loading}{MetaPost Special Extensions}

%D We implement extensions by using the \METAPOST\ special
%D mechanism. Opposite to \TEX's specials, the \METAPOST\ ones
%D are flushed before or after the graphic data, but thereby
%D are no longer connected to a position. 
%D 
%D We implement specials by overloading the \type {fill}
%D operator. By counting the fills, we can let the converter
%D treat the appropriate fill in a special way. The
%D specification of the speciality can have two forms,
%D determined by the setting of a boolean variable: 
%D
%D \starttypen 
%D _inline_shading_ := false ; % comment like code (default)
%D _inline_shading_ := true  ; % command like code 
%D \stoptypen 
%D 
%D When the specification is embedded as comment, it looks 
%D like: 
%D
%D \starttypen 
%D %%MetaPostSpecial <size> <data> <number> <identifier> 
%D \stoptypen 
%D
%D The in||line alternative is more tuned for \POSTSCRIPT, 
%D since it permits us to define a macro \type {special}.
%D
%D \starttypen 
%D inline  : <data> <number> <identifier> <size> special
%D \stoptypen 
%D 
%D The \type {identifier} determines what to do, and the data
%D can be used to accomplish this. A type~2 shading function
%D has identifier~2. Alltogether, the number of parameters is
%D specified in \type {size}. The \type {number} is the number
%D of the fill that needs the special treatment. For a type~2
%D and~3 shaded fill, the datablock contains the following
%D data: 
%D 
%D \starttypen 
%D from to n inner_r g b x y        outer_r g b x y        
%D from to n inner_r g b x y radius outer_r g b x y radius 
%D \stoptypen
%D 
%D The implementation below, saves the data on the stack in 
%D a way similar to the macros in \type {supp-pdf.tex}, and 
%D just overload a few already defined handlers. That way, 
%D the existing macros are still generic. \voetnoot {Actually, 
%D the macros here are just as generic.}  
%D 
%D Currently the only extension concerns shading, which is
%D accomplished by handling yet another value of \type
%D {\finiMPpath}. The recource disctionary is stored and 
%D later picked up by the general \CONTEXT\ figure inclusion 
%D macros. 

\unprotect 

\newcount\currentPDFshade      % global count 
\newcount\currentMPshade       % local count 
\newcount\currentMPfill        % local count 
\chardef\inlineMPspecials=0    % only needed for stack resetting 
\let\currentMPshades\empty

\def\dohandleMPspecialcomment#1 
  {\setMPargument{#1}%
   \advance\scratchcounter by -1 
   \ifcase\scratchcounter 
     \handleMPspecialcommand
     \donetrue
     \doresetMPstack
     \let\handleMPsequence=\dohandleMPsequence
     \expandafter\handleMPsequence
   \else
     \expandafter\dohandleMPspecialcomment
   \fi}

\def\handleMPspecialcomment #1 % number of arguments 
  {\doresetMPstack
   \scratchcounter=#1\relax
   \ifcase\scratchcounter % when zero, inline shading is used 
     \chardef\inlineMPspecials=1
     \let\handleMPsequence=\dohandleMPsequence
     \expandafter\handleMPsequence
   \else
     \chardef\inlineMPspecials=0
     \expandafter\dohandleMPspecialcomment
   \fi}

\def\startMPresources%
  {\global\let\currentMPshades\empty
   \global\currentMPfill=0
   \global\currentMPshade=0
   \ifx\currentPDFresources\empty\else
     \message{unused resources before shade \the\currentPDFshade}%
   \fi
   \global\let\currentPDFresources\empty}

\def\stopMPresources%
  {\ifx\currentMPshades\empty
     \global\let\currentPDFresources\empty
   \else
     \xdef\currentPDFresources%
       {/Shading << \currentMPshades >>}%
   \fi}

\def\startMPshading#1%
  {\edef\currentMPspecial{\gMPs{#1}}}

\def\stopMPshading%
  {\global\advance\currentPDFshade by 1 
   \global\advance\currentMPshade by 1 
   \setevalue{mps:Sh:\currentMPspecial}% non global !!
     {\the\currentPDFshade}% 
   \xdef\currentMPshades%
     {\currentMPshades/Sh\the\currentPDFshade\space\the\pdflastobj\space0 R }}

\def\processMPpath% 
  {\global\advance\currentMPfill by 1 
   \ifnum\finiMPpath=2 \ifx\currentMPshades\empty \else
     \doifdefined{mps:Sh:\the\currentMPfill}
       {\chardef\finiMPpath=4 \PDFcode{q /Pattern cs}}%
   \fi \fi
   \flushMPpath
   \closeMPpath
   \PDFcode
     {\ifcase\finiMPpath 
        W n\or S\or f\or B\or W n /Sh\getvalue{mps:Sh:\the\currentMPfill} sh Q%
      \fi}%
   \let\handleMPsequence=\dohandleMPsequence
   \resetMPstack
   \nofMPsegments=0
   \handleMPsequence}

\def\handleMPspecialcommand%
  {\ifcase\inlineMPspecials\or
     \advance\nofMParguments by -1 % pop the size 
   \fi
   \doifundefinedelse{\MPspecial}
     {\message{[unknown \MPspecial]}} 
     {\getvalue{\MPspecial}}% 
   \ifcase\inlineMPspecials 
     \doresetMPstack % 0 
   \else             
     \resetMPstack   % 1 
   \fi}

\def\MPspecial%
  {MP special \gMPs\nofMParguments}

\def\defineMPspecial#1#2%
  {\setvalue{MP special #1}{#2}}

%D Shading is an example of a more advanced graphic feature,
%D but users will seldom encounter those complications. Here
%D we only show a few simple examples, but many other
%D alternatives are possible by setting up the functions built
%D in \PDF\ in the appropriate way. 
%D 
%D Shading has to do with interpolation between two or more
%D points or user supplied ranges. In \PDF, the specifications
%D of a shade has to be encapsulated in objects and passed on
%D as resources. This is a \PDF\ level 1.3. feature. One can
%D simulate three dimensional shades as well and define simple
%D functions using a limited set of \POSTSCRIPT\ primitives.
%D Given the power of \METAPOST\ and these \PDF\ features, we
%D can achieve superb graphic effects. 
%D 
%D Since everything is hidden in \TEX\ and \METAPOST\ graphics,
%D we can stick to high level \CONTEXT\ command, as shown in 
%D the following exmples.
%D 
%D \startbuffer
%D \startuniqueMPgraphic{CircularShade}
%D   path  p ; p := unitsquare xscaled \overlaywidth yscaled \overlayheight ;
%D   circular_shade(p,0,.2red,.9red) ;
%D \stopuniqueMPgraphic
%D 
%D \startuniqueMPgraphic{LinearShade}
%D   path  p ; p := unitsquare xscaled \overlaywidth yscaled \overlayheight ;
%D   linear_shade(p,0,.2blue,.9blue) ;
%D \stopuniqueMPgraphic
%D 
%D \startuniqueMPgraphic{DuotoneShade}
%D   path  p ; p := unitsquare xscaled \overlaywidth yscaled \overlayheight ;
%D   linear_shade(p,2,.5green,.5red) ;
%D \stopuniqueMPgraphic
%D \stopbuffer
%D 
%D \typebuffer 
%D 
%D \haalbuffer
%D 
%D These graphics can be hooked into the overlay mechanism, 
%D which is available in many commands.
%D 
%D \startbuffer
%D \defineoverlay[demo 1][\uniqueMPgraphic{CircularShade}]
%D \defineoverlay[demo 2][\uniqueMPgraphic  {LinearShade}]
%D \defineoverlay[demo 3][\uniqueMPgraphic {DuotoneShade}]
%D \stopbuffer
%D 
%D \typebuffer 
%D 
%D \haalbuffer
%D 
%D These backgrounds can for instance be applied to \type 
%D {\framed}: 
%D 
%D \startbuffer
%D \setupframed[breedte=3cm,hoogte=2cm,kader=uit]
%D \startcombinatie[3*1]
%D   {\framed[achtergrond=demo 1]{\bfd \white Demo 1}} {}
%D   {\framed[achtergrond=demo 2]{\bfd \white Demo 2}} {}
%D   {\framed[achtergrond=demo 3]{\bfd \white Demo 3}} {}
%D \stopcombinatie
%D \stopbuffer
%D 
%D \typebuffer 
%D 
%D \startregelcorrectie
%D \haalbuffer
%D \stopregelcorrectie
%D 
%D There are a few more alternatives, determined by the second 
%D parameter passed to \type {circular_shade} and alike.
%D 
%D \def\SomeShade#1#2#3#4#5% 
%D   {\startuniqueMPgraphic{Shade-#1}
%D      width := \overlaywidth ;
%D      height := \overlayheight ;
%D      path p ; p := unitsquare xscaled width yscaled height ;
%D      #2_shade(p,#3,#4,#5) ;
%D    \stopuniqueMPgraphic
%D    \defineoverlay[Shade-#1][\uniqueMPgraphic{Shade-#1}]%
%D    \framed[achtergrond=Shade-#1,breedte=2cm,hoogte=2cm,kader=uit]{}}
%D 
%D \startregelcorrectie
%D \startcombinatie[5*1]
%D   {\SomeShade{10}{circular}{0}{.3blue}{.9blue}} {circular 0}  
%D   {\SomeShade{11}{circular}{1}{.3blue}{.9blue}} {circular 1}  
%D   {\SomeShade{12}{circular}{2}{.3blue}{.9blue}} {circular 2}  
%D   {\SomeShade{13}{circular}{3}{.3blue}{.9blue}} {circular 3}  
%D   {\SomeShade{14}{circular}{4}{.3blue}{.9blue}} {circular 4}  
%D \stopcombinatie
%D \stopregelcorrectie
%D 
%D \blanko
%D 
%D \startregelcorrectie
%D \startcombinatie[5*1]
%D   {\SomeShade{20}{circular}{0}{.9green}{.3green}} {circular 0}  
%D   {\SomeShade{21}{circular}{1}{.9green}{.3green}} {circular 1}  
%D   {\SomeShade{22}{circular}{2}{.9green}{.3green}} {circular 2}  
%D   {\SomeShade{23}{circular}{3}{.9green}{.3green}} {circular 3}  
%D   {\SomeShade{24}{circular}{4}{.9green}{.3green}} {circular 4}  
%D \stopcombinatie
%D \stopregelcorrectie
%D 
%D \blanko
%D 
%D \startregelcorrectie
%D \startcombinatie[4*1]
%D   {\SomeShade{30}{linear}{0}{.3red}{.9red}} {linear 0}  
%D   {\SomeShade{31}{linear}{1}{.3red}{.9red}} {linear 1}  
%D   {\SomeShade{32}{linear}{2}{.3red}{.9red}} {linear 2}  
%D   {\SomeShade{33}{linear}{3}{.3red}{.9red}} {linear 3}  
%D \stopcombinatie
%D \stopregelcorrectie
%D 
%D These macros closely cooperate with the \METAPOST\ module
%D \type {mp-spec.mp}, which is part of the \CONTEXT\
%D distribution. 
%D 
%D The low level (\PDF) implementation is based on the \TEX\ 
%D based \METAPOST\ to \PDF\ converter. Shading is supported 
%D by overloading the \type {fill} operator as implemented 
%D earlier. In \PDF\ type~2 and~3 shading functions are 
%D specified in terms of: 
%D 
%D \starttabulatie[|Tl|l|]
%D \NC /Domain \NC sort of meeting range \NC \NR 
%D \NC /C0     \NC inner shade \NC \NR 
%D \NC /C1     \NC outer shade \NC \NR 
%D \NC /N      \NC smaller values, bigger inner circles \NC \NR 
%D \stoptabulatie

\defineMPspecial{2} 
  {\startMPshading{14}% type 2 
   \immediate\pdfobj 
     {<</FunctionType 2
        /Domain [\gMPs1 \gMPs2]          
        /C0 [\gMPs4 \gMPs5 \gMPs6]       
        /C1 [\gMPs9 \gMPs{10} \gMPs{11}] 
        /N \gMPs3>>}%                       
   \immediate\pdfobj 
     {<</ShadingType 2
        /ColorSpace /DeviceRGB
        /Function \the\pdflastobj\space 0 R
        /Coords [\gMPs7 \gMPs8 \gMPs{12} \gMPs{13}]  
        /Extend [true true]>>}%
   \stopMPshading}

\defineMPspecial{3} 
  {\startMPshading{16}% type 3 
   \immediate\pdfobj 
     {<</FunctionType 2
        /Domain [\gMPs1 \gMPs2]          
        /C0 [\gMPs4 \gMPs5 \gMPs6] 
        /C1 [\gMPs{10} \gMPs{11} \gMPs{12}] 
        /N \gMPs3>>}%                      
   \immediate\pdfobj 
     {<</ShadingType 3
        /ColorSpace /DeviceRGB
        /Function \the\pdflastobj\space 0 R
        /Coords [\gMPs7 \gMPs8 \gMPs9 \gMPs{13} \gMPs{14} \gMPs{15}]  
        /Extend [true true]>>}%
   \stopMPshading}

\protect \endinput 
