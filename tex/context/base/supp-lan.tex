%D \module
%D   [       file=supp-lan,
%D        version=1997.03.20,
%D          title=\CONTEXT\ Support Macros,
%D       subtitle=Language Options,
%D         author=Hans Hagen,
%D           date=\currentdate,
%D      copyright={PRAGMA / Hans Hagen \& Ton Otten}]
%C
%C This module is part of the \CONTEXT\ macro||package and is
%C therefore copyrighted by \PRAGMA. Non||commercial use is
%C granted.

%D \gdef\starttest%
%D   {\blanko
%D    \noindent
%D    \halign\bgroup\tt##\hskip2em&##\hskip2em&##\cr}
%D
%D \gdef\stoptest%
%D   {\egroup
%D    \blanko}
%D
%D \gdef\test#1%
%D   {\convertargument#1\to\ascii\ascii&\hyphenatedword{#1}&#1\cr}

%D One of \TEX's strong points in building paragraphs is the way
%D hyphenations are handled. Although for real good hyphenation
%D of non||english languages some extensions to the program are
%D needed, fairly good results can be reached with the standard
%D mechanisms and an additional macro, at least in Dutch.

\unprotect

%D \CONTEXT\ originates in the wish to typeset educational
%D materials, especially in a technical environment. In
%D production oriented environments, a lot of compound words
%D are used. Because the Dutch language poses no limits on
%D combining words, we often favor putting dashes between those
%D words, because it facilitates reading, at least for those
%D who are not that accustomed to it.
%D
%D In \TEX\ compound words, separated by a hyphen, are not
%D hyphenated at all. In spite of the multiple pass paragraph
%D typesetting this can lead to parts of words sticking into
%D the margin. The solution lays in saying
%D \type{spoelwater||terugwinunit} instead of
%D \type{spoelwater-terugwinunit}. By using a one character
%D command like \type{|}, delimited by the same character
%D \type{|}, we get ourselves both a decent visualization (in
%D \TEXEDIT\ and colored verbatim we color these commands
%D yellow) and an efficient way of combining words.
%D
%D The sequence \type{||} simply leads to two words connected by
%D a hyphen. Because we want to distinguish such a hyphen from
%D the one inserted when \TEX\ hyphenates a word, we use a bit
%D longer one.
%D
%D \hyphenation {spoel-wa-ter te-rug-win-unit}
%D
%D \starttest
%D \test {spoelwater||terugwinunit}
%D \stoptest
%D
%D As we already said, the \type{|} is a command. This commands
%D accepts an optional argument before it's delimiter, which is
%D also a \type{|}.
%D
%D \hyphenation {po-ly-meer che-mie}
%D
%D \starttest
%D \test {polymeer|*|chemie}
%D \stoptest
%D
%D Arguments like \type{*} are not interpreted and inserted
%D directly, in contrary to arguments like:
%D
%D \starttest
%D \test {polymeer|~|chemie}
%D \test {|(|polymeer|)|chemie}
%D \test {polymeer|(|chemie|)| }
%D \stoptest
%D
%D Although such situations seldom occur |<|we typeset thousands
%D of pages before we encountered one that forced us to enhance
%D this mechanism|>| we also have to take care of comma's.
%D
%D \hyphenation {uit-stel-len}
%D
%D \starttest
%D \test {op||, in|| en uitstellen}
%D \stoptest
%D
%D The next special case (concerning quotes) was brought to my
%D attention by Piet Tutelaers, one of the driving forces
%D behind rebuilding hyphenation patterns for the dutch
%D language.\voetnoot{In 1996 the spelling of the dutch
%D language has been slightly reformed which made this topic
%D actual again.} We'll also take care of this case.
%D
%D \starttest
%D \test {AOW|'|er}
%D \test {cd|'|tje}
%D \test {ex|-|PTT|'|er}
%D \test {rock|-|'n|-|roller}
%D \stoptest
%D
%D Tobias Burnus pointed out that I should also support
%D something like
%D
%D \starttest
%D \test {well|_|known}
%D \stoptest
%D 
%D to strees the compoundness of hyphenated words. 
%D
%D Of course we also have to take care of the special case: 
%D
%D \starttest
%D \test {text||color and ||font}
%D \stoptest

%D \macros
%D   {installdiscretionaries}
%D   {}
%D
%D The mechanism described here is one of the older inner parts
%D of \CONTEXT. The most recent extensions concerns some
%D special cases as well as the possibility to install other
%D characters as delimiters. The prefered way of specifying
%D compound words is using \type{||}, which is installed by:
%D
%D \starttypen
%D \installdiscretionaries || -
%D \stoptypen
%D
%D Some alternative definitions are:
%D
%D \startbuffer
%D \installdiscretionaries ** -
%D \installdiscretionaries ++ -
%D \installdiscretionaries // -
%D \installdiscretionaries ~~ -
%D \stopbuffer
%D
%D \typebuffer
%D
%D after which we can say:
%D
%D \bgroup
%D \haalbuffer
%D \starttest
%D \test {test**test**test}
%D \test {test++test++test}
%D \test {test//test//test}
%D \test {test~~test~~test}
%D \stoptest
%D \egroup

%D \macros
%D   {compoundhyphen,
%D    beginofsubsentence,endofsubsentence}
%D   {}
%D
%D Now let's go to the macros. First we define some variables.
%D In the main \CONTEXT\ modules these can be tuned by a setup
%D command. Watch the (maybe) better looking compound hyphen.

\def\compoundhyphen     {{-}\kern-.25ex{-}}
\def\beginofsubsentence {---}
\def\endofsubsentence   {---}

%D The last two variables are needed for subsentences
%D |<|like this one|>| which we did not yet mention.
%D
%D We want to enable breaking but at the same time don't want
%D compound characters like |-| or || to be separated from the
%D words. \TEX\ hackers will recognise the next two macro's:

\def\prewordbreak  {\penalty10000\hskip0pt\relax}
\def\postwordbreak {\penalty0\prewordbreak}

%D We first show the original implementation, which only
%D supports \type{|} as command and delimiter. Before
%D activating \type{|} we save it's value:
%D
%D \starttypen
%D \edef\domathmodediscretionary{\string|}
%D \stoptypen
%D
%D after which we're ready to define it's meaning to:
%D
%D \starttypen
%D \catcode`\|=\@@active
%D
%D \unexpanded\def|%
%D   {\ifmmode
%D      \expandafter\domathmodediscretionary
%D    \else
%D      \expandafter\dotextmodediscretionary
%D    \fi}
%D \stoptypen
%D
%D We need a two stage \type{\futurelet} because we want to
%D look ahead for both the compound character definition and
%D the (optional) comma that follows it, and because we want to
%D prevent that \TEX\ puts this comma on the next line. We use
%D \type{\next} for easy and fast checking of the argument, we
%D save this argument (which can consist of more tokens) and
%D also save the character following the \type{|#1|} in
%D \type{\nextnext}.
%D
%D \starttypen
%D \def\dotextmodediscretionary%
%D   {\bgroup
%D    \futurelet\next\dodotextmodediscretionary}
%D
%D \def\dodotextmodediscretionary#1|%
%D   {\def\betweendiscretionaries{#1}%
%D    \futurelet\nextnext\dododotextmodediscretionary}
%D \stoptypen
%D
%D The main macro consists of quite some \type{\ifx} tests
%D while \type{\checkafterdiscretionary} handles the commas.
%D We show the simplified version here:
%D
%D \starttypen
%D \def\dododotextmodediscretionary%
%D   {\let\nextnextnext=\egroup
%D    \ifx     |\next
%D      \checkafterdiscretionary
%D      \prewordbreak\hbox{\compoundhyphen\nextnext}\allowbreak
%D    \else\ifx=\next
%D      \prewordbreak\compoundhyphen
%D    \else\ifx~\next
%D      \discretionary{-}{}{\thinspace}\postwordbreak
%D    \else\ifx(\next
%D      \prewordbreak\discretionary{}{(-}{(}\prewordbreak
%D    \else\ifx)\next
%D      \prewordbreak\discretionary{-)}{}{)}\prewordbreak
%D    \else\ifx'\next
%D      \prewordbreak\discretionary{-}{}{'}\postwordbreak
%D    \else
%D      \checkafterdiscretionary
%D      \prewordbreak\hbox{\betweendiscretionaries\nextnext}\allowbreak
%D    \fi\fi\fi\fi\fi\fi
%D    \nextnextnext}
%D
%D \def\checkafterdiscretionary%
%D   {\ifx,\nextnext
%D      \def\nextnextnext{\afterassignment\egroup\let\next=}%
%D    \else
%D      \let\nextnext=\relax
%D    \fi}
%D \stoptypen
%D 
%D Handling \type{(} and \type{)} is a a bit special, because
%D \TEX\ sees them as decent hyphenation points, according to
%D their \type{\lccode} being non||zero. For the same reason,
%D later on in this module we cannot manipulate the
%D \type{\lccode} but take the \type{\uccode}.

%D The most recent implementation is more advanced. As
%D demonstrated we can install delimiters, like:
%D
%D \starttypen
%D \installdiscretionaries || \compoundhyphen
%D \stoptypen
%D
%D This time we have to use a bit more clever way of saving the
%D math mode specification of the character we're going to
%D make active. We also save the user supplied compound hyphen.
%D We show the a bit more traditional implementation first.
%D
%D \starttypen
%D \def\installdiscretionaries#1%
%D   {\catcode`#1\@@other
%D    \expandafter\doinstalldiscretionaries\string#1}
%D
%D \def\doinstalldiscretionaries#1%
%D   {\setvalue{mathmodediscretionary#1}{#1}%
%D    \catcode`#1\@@active
%D    \dodoinstalldiscretionaries}
%D
%D \def\dodoinstalldiscretionaries#1#2%
%D   {\setvalue{textmodediscretionary\string#1}{#2}%
%D    \unexpanded\def#1{\discretionarycommand#1}}
%D \stoptypen
%D
%D A bit more \CATCODE\ and character trickery enables us to
%D discard the two intermediate steps. This trick originates
%D on page~394 of the \TEX book, in the appendix full of
%D dirty tricks. The second argument has now become redundant,
%D but I decided to reserve it for future use. At least it
%D remembers us of the symmetry.

\def\installdiscretionaries#1#2#3%
  {\setvalue{mathmodediscretionary\string#1}{\char`#1}%
   \setvalue{textmodediscretionary\string#1}{#3}%
   \catcode`#1=\@@active
   \scratchcounter=\the\uccode`~
   \uccode`~=`#1
   \uppercase{\unexpanded\def~{\discretionarycommand~}}%
   \uccode`~=\scratchcounter}

\def\dohandlemathmodebar#1%
  {\getvalue{mathmodediscretionary\string#1}}

\def\discretionarycommand%
  {\ifmmode
     \expandafter\dohandlemathmodebar
   \else
     \expandafter\dotextmodediscretionary
   \fi}

%D Although adapting character codes and making characters
%D active can interfere with other features of macropackages,
%D normally there should be no problems with things like:
%D
%D \starttypen
%D \installdiscretionary || +
%D \installdiscretionary ++ =
%D \stoptypen
%D
%D The real work is done by the next set of macros. We have
%D to use a double \type{\futurelet} because we have to take
%D following characters into account.

\def\dotextmodediscretionary#1%
  {\bgroup
   \def\dodotextmodediscretionary##1#1%
     {\def\betweendiscretionary{##1}%
      \futurelet\nextnext\dododotextmodediscretionary}%
   \let\discretionarycommand=#1%
   \def\textmodediscretionary{\getvalue{textmodediscretionary\string#1}}%
   \futurelet\next\dodotextmodediscretionary}

\def\dododotextmodediscretionary%
  {\let\nextnextnext=\egroup
   \ifx\discretionarycommand\next
     \checkafterdiscretionary
     \bgroup
       \checkbeforediscretionary
       \prewordbreak\hbox{\textmodediscretionary\nextnext}\allowbreak
     \egroup
   \else\ifx=\next
     \prewordbreak\textmodediscretionary
   \else\ifx~\next
     \prewordbreak\discretionary{-}{}{\thinspace}\postwordbreak
   \else\ifx_\next
     \prewordbreak\discretionary{\textmodediscretionary}
       {\textmodediscretionary}{\textmodediscretionary}\prewordbreak
   \else\ifx(\next
     \ifdim\lastskip>\!!zeropoint\relax
       (\prewordbreak
     \else
       \prewordbreak\discretionary{}{(-}{(}\prewordbreak
     \fi
   \else\ifx)\next
     \ifx\nextnext\blankspace
       \prewordbreak)\relax
     \else
       \prewordbreak\discretionary{-)}{}{)}\prewordbreak
     \fi
   \else\ifx'\next
     \prewordbreak\discretionary{-}{}{'}\postwordbreak
   \else\ifx<\next
     \beginofsubsentence\prewordbreak\beginofsubsentencespacing
   \else\ifnum\uccode`>=\nextuccode
     \endofsubsentencespacing\prewordbreak\endofsubsentence
   \else
     \checkafterdiscretionary
     \bgroup
       \checkbeforediscretionary
       \prewordbreak\hbox{\betweendiscretionary\nextnext}\allowbreak
     \egroup
   \fi\fi\fi\fi\fi\fi\fi\fi\fi
   \nextnextnext}

\def\checkbeforediscretionary%
  {\setbox0=\lastbox
   \ifdim\wd0=\!!zeropoint
     \let\postwordbreak=\prewordbreak
   \fi
   \box0\relax}

\def\checkafterdiscretionary%
  {\ifx,\nextnext
     \def\nextnextnext{\afterassignment\egroup\let\next=}%
   \else
     \let\nextnext=\relax
   \fi}

%D The macro \type{\checkbeforediscretionary} takes care of 
%D loners like \type{||word}, while it counterpart 
%D \type{\checkafterdiscretionary} is responsible for handling 
%D the comma. 

%D \macros
%D   {beginofsubsentencespacing,endofsubsentencespacing}
%D   {}
%D 
%D In the previous macros we provided two hooks which can be 
%D used to support nested sub||sentences. In \CONTEXT\ these 
%D hooks are used to insert a small space when needed. 

\let\beginofsubsentencespacing=\relax
\let\endofsubsentencespacing  =\relax

%D Before we show some more tricky alternative, we first install
%D the mechanism:

\installdiscretionaries || \compoundhyphen

%D One of the drawbacks of this mechanism is that characters can
%D be made active afterwards. The next alternative can be used
%D in such situations.  This time we don't compare the arguments
%D directly but use the \type{\uccode}'s instead. \TEX\
%D initializes these codes of the alphabetics glyphs to their
%D uppercase counterparts. Normally the other characters remain
%D zero. If so, we can use the \type{\uccode} as a signal.

%D \macros
%D   {enableactivediscretionaries}
%D   {}
%D
%D The more advanced mechanism is activated by calling:
%D
%D \starttypen
%D \enableactivediscretionaries
%D \stoptypen
%D
%D which is defined as:

\def\enableactivediscretionaries%
  {\uccode`'=`'\relax \uccode`~=`~\relax \uccode`_=`_\relax
   \uccode`(=`(\relax \uccode`)=`)\relax \uccode`==`=\relax
   \uccode`<=`<\relax \uccode`>=`>\relax
   \let\dotextmodediscretionary     = \activedotextmodediscretionary
   \let\dododotextmodediscretionary = \activedododotextmodediscretionary}

%D We only have to redefine two macros. While saving the
%D \type{\uccode} in a macro we have to take care of empty
%D arguments, like in \type{||}.

\def\activedotextmodediscretionary#1%
  {\bgroup
   \def\dodotextmodediscretionary##1#1%
     {\def\betweendiscretionary{##1}%
      \def\nextuccode####1####2\relax%
        {\ifcat\noexpand####1\noexpand\relax
           \edef\nextuccode{0}%
         \else
           \edef\nextuccode{\the\uccode`####1}%
         \fi}%
      \nextuccode##1@\relax
      \futurelet\nextnext\dododotextmodediscretionary}%
   \let\discretionarycommand=#1%
   \def\textmodediscretionary{\getvalue{textmodediscretionary\string#1}}%
   \futurelet\next\dodotextmodediscretionary}

%D This time we use \type{\ifnum}:

\def\activedododotextmodediscretionary%
  {\let\nextnextnext=\egroup
   \ifx\discretionarycommand\next
     \checkafterdiscretionary
     \bgroup
       \checkbeforediscretionary
       \prewordbreak\hbox{\textmodediscretionary\nextnext}\allowbreak
     \egroup
   \else\ifnum\uccode`==\nextuccode
     \prewordbreak\textmodediscretionary
   \else\ifnum\uccode`~=\nextuccode
     \prewordbreak\discretionary{-}{}{\thinspace}\postwordbreak
   \else\ifnum\uccode`_=\nextuccode
     \prewordbreak\discretionary{\textmodediscretionary} 
       {\textmodediscretionary}{\textmodediscretionary}\prewordbreak
   \else\ifnum\uccode`(=\nextuccode
     \ifdim\lastskip>\!!zeropoint\relax
       (\prewordbreak
     \else
       \prewordbreak\discretionary{}{(-}{(}\prewordbreak
     \fi
   \else\ifnum\uccode`)=\nextuccode
     \ifx\nextnext\blankspace
       \prewordbreak)\relax
     \else
       \prewordbreak\discretionary{-)}{}{)}\prewordbreak
     \fi
   \else\ifnum\uccode`'=\nextuccode
     \prewordbreak\discretionary{-}{}{'}\postwordbreak
   \else\ifnum\uccode`<=\nextuccode
     \beginofsubsentence\prewordbreak\beginofsubsentencespacing
   \else\ifnum\uccode`>=\nextuccode
     \endofsubsentencespacing\prewordbreak\endofsubsentence
   \else 
     \checkafterdiscretionary
     \bgroup
       \checkbeforediscretionary
       \prewordbreak\hbox{\betweendiscretionary\nextnext}\allowbreak
     \egroup
   \fi\fi\fi\fi\fi\fi\fi\fi\fi
   \nextnextnext}

%D Now we can safely do things like: \enableactivediscretionaries
%D
%D \starttypen
%D \catcode`<=\@@active  \def<{hello there}
%D \catcode`>=\@@active  \def>{hello there}
%D \catcode`(=\@@active  \def({hello there}
%D \catcode`)=\@@active  \def){hello there}
%D \stoptypen
%D
%D In normal day||to||day production of texts this kind of
%D activation is seldom used.\voetnoot{In the \CONTEXT\ manual
%D the \type{<} and \type{>} are made active and used for some
%D cross||reference trickery.} If so, we have to take care of
%D the math mode explicitly, just like we did when making
%D \type{|} active. It can be confusing too, especially when we
%D load macropackages afterwards that make use of \type{<} in
%D \type{\ifnum} or \type{\ifdim} statements.

%D \macros
%D   {installcompoundcharacter}
%D   {}
%D
%D When Tobias Burnus started translating the dutch manual of
%D \PPCHTEX\ into german, he suggested to let \CONTEXT\ support
%D the \type{german.sty} method of handling compound
%D characters, especially the umlaut. This package is meant for
%D use with \PLAIN\ \TEX\ as well as \LATEX. 
%D 
%D I decided to implement compound character support as
%D versatile as possible. As a result one can define his own
%D compound character support, like: 
%D 
%D \starttypen
%D \installcompoundcharacter "a {\"a}
%D \installcompoundcharacter "e {\"e}
%D \installcompoundcharacter "i {\"i}
%D \installcompoundcharacter "u {\"u}
%D \installcompoundcharacter "o {\"o}
%D \installcompoundcharacter "s {\SS} 
%D \stoptypen
%D
%D or even 
%D 
%D \starttypen
%D \installcompoundcharacter "ck {\discretionary {k-}{k}{ck}}
%D \installcompoundcharacter "ff {\discretionary{ff-}{f}{ff}}
%D \stoptypen
%D 
%D The support is not limited to alphabetic characters, so the 
%D next definition is also valid. 
%D 
%D \starttypen
%D \installcompoundcharacter ". {.\doifnextcharelse{\spacetoken}{}{\kern.125em}}
%D \stoptypen
%D 
%D The implementation looks familiar and uses the same tricks as
%D mentioned earlier in this module. We take care of two 
%D arguments, which complicates things a bit.  

\def\@nc@{@nc@} % normal character 
\def\@cc@{@cc@} % compound character 
\def\@cs@{@cs@} % compound characters 

\def\installcompoundcharacter #1#2#3 #4% {{#4}} keeps move local
  {\setvalue{\@nc@\string#1}{\char`#1}%
   \def\!!stringa{#3}%
   \ifx\!!stringa\empty
     \setvalue{\@cc@\string#1\string#2}{{#4}}% 
   \else
     \setvalue{\@cs@\string#1\string#2\string#3}{{#4}}%
   \fi
   \catcode`#1=\@@active
   \scratchcounter=\the\uccode`~
   \uccode`~=`#1
   \uppercase{\unexpanded\def~{\handlecompoundcharacter~}}%
   \uccode`~=\scratchcounter}

%D In handling the compound characters we have to take care of
%D \type{\bgroup} and \type{\egroup} tokens, so we end up with
%D a multi||step interpretation macro. We look ahead for a
%D \type{\bgroup}, \type{\egroup} or \type{\blankspace}. Being
%D no user of this mechanism, the credits for testing them goes
%D to Tobias Burnus, the first german user of \CONTEXT.
%D
%D We define these macros as \type{\long} because we can
%D expect \type{\par} tokens. We need to look into the future
%D with \type{\futurelet} to prevent spaces from
%D disappearing.

\def\handlecompoundcharacter#1%
  {\def\dohandlecompoundcharacter%
     {\ifx\next\bgroup
       %\def\next{\dodohandlecompoundcharacter#1}%    % handle "{ee} -> \"ee
       %\let\next=\relax                              % forget "{ee} -> ee
        \def\next{\handlecompoundcharacterone#1}%     % ignore "{ee} -> "ee
      \else\ifx\next\egroup
        \let\next=\relax
      \else\ifx\next\blankspace
        \let\next=\relax
      \else
        \def\next{\dodohandlecompoundcharacter#1}%
      \fi\fi\fi
      \next}%
   \futurelet\next\dohandlecompoundcharacter}

\def\dodohandlecompoundcharacter#1#2%
  {\def\dododohandlecompoundcharacter%
     {\ifx\next\bgroup
        \def\next{\handlecompoundcharacterone#1#2}%
      \else\ifx\next\egroup
        \def\next{\handlecompoundcharacterone#1#2}%
      \else\ifx\next\blankspace
        \def\next{\handlecompoundcharacterone#1#2}%
      \else
        \def\next{\handlecompoundcharactertwo#1#2}%
      \fi\fi\fi
      \next}%
   \futurelet\next\dododohandlecompoundcharacter}

%D Besides taken care of the grouping and space tokens, we have
%D to deal with three situations. First we look if the next
%D character equals the first one, if so, then we just insert
%D the original. Next we look if indeed a compound character is
%D defined. We either execute the compound character or just
%D insert the first. So we have
%D
%D \starttypen
%D <key><key>  <key><known>  <key><unknown>  
%D \stoptypen
%D
%D In later modules we will see how these commands are used.

\long\def\handlecompoundcharacterone#1#2%
  {\ifx#1#2%
     \def\next{\getvalue{\@nc@\string#1}\getvalue{\@nc@\string#2}}%
   \else
     \expandafter\ifx\csname\@cc@\string#1\string#2\endcsname\relax
       \def\next{\getvalue{\@nc@\string#1}#2}%
     \else
       \def\next{\getvalue{\@cc@\string#1\string#2}}%
     \fi
   \fi
   \next}

\long\def\handlecompoundcharactertwo#1#2#3%
  {\ifx#1#2%
     \def\next{\getvalue{\@nc@\string#1}\getvalue{\@nc@\string#2}#3}%
   \else
     \@EA\ifx\csname\@cs@\string#1\string#2\string#3\endcsname\relax
       \expandafter\ifx\csname\@cc@\string#1\string#2\endcsname\relax
         \def\next{\getvalue{\@nc@\string#1}#2#3}%
       \else
         \def\next{\getvalue{\@cc@\string#1\string#2}#3}%
       \fi
     \else
       \def\next{\getvalue{\@cs@\string#1\string#2\string#3}}%
     \fi
   \fi
   \next}

%D \macros
%D  {midworddiscretionary}
%D
%D If needed, one can add a discretionary hyphen using \type
%D {\midworddiscretionary}. This macro does the same as 
%D \PLAIN\ \TEX's \type {\-}, but, like the ones implemented 
%D earlier, this one also looks ahead for spaces and grouping
%D tokens. 
 
\def\domidworddiscretionary%
  {\ifx\next\blankspace\else
   \ifx\next\bgroup    \else
   \ifx\next\egroup    \else
     \discretionary{-}{}{}%
   \fi\fi\fi}

\def\midworddiscretionary%
  {\futurelet\next\domidworddiscretionary}

\protect

\endinput
