%D \module
%D   [       file=supp-pdf,
%D        version=1998.10.15,
%D          title=\CONTEXT\ Support Macros,
%D       subtitle=\METAPOST\ to \PDF\ conversion,
%D         author=Hans Hagen,
%D           date=\currentdate,
%D      copyright={PRAGMA / Hans Hagen \& Ton Otten}]
%C
%C This module is part of the \CONTEXT\ macro||package and is
%C therefore copyrighted by \PRAGMA. Non||commercial use is
%C granted.

%D These macros are written as generic as possible. Some
%D general support macro's are loaded from a small module
%D especially made for non \CONTEXT\ use. In this module I
%D use a matrix transformation macro written by Tanmoy
%D Bhattacharya. Thanks to extensive testing of Sebastian
%D Ratz I was able to complete this module within reasonable
%D time. First we take care of non||\CONTEXT\ use:

\ifx \undefined \writestatus           \input supp-mis.tex \relax \fi
\ifx \undefined \convertPDFtoPDF \else \expandafter \endinput     \fi 

%D This module handles some \PDF\ conversion and insertions
%D topics. By default, the macros use the \PDFTEX\ primitive
%D \type{\pdfliteral} when available. 

\writestatus{loading}{Context Support Macros / PDF}

\unprotect

\ifx\pdfliteral\undefined
  \def\PDFcode#1{\message{[ignored pdfliteral: #1]}}
\else
  \let\PDFcode=\pdfliteral
\fi

%D \macros
%D   {convertPDFtoPDF}
%D
%D \PDFTEX\ supports verbatim inclusion of \PDF\ code. The
%D following macro takes care of inserting externally defined
%D illustrations in \PDF\ format. According to a suggestion
%D Tanmoy Bhattacharya posted to the \PDFTEX\ mailing list, we
%D first skip lines until \type{stream} is reached and then
%D copy lines until \type{endstream} is encountered. This
%D scheme only works with vectorized graphics in which no
%D indirect references to objects are used. Bitmaps also don't
%D work. Interpreting their specifications is beyond the
%D current implementation.
%D
%D \starttypen
%D \convertPDFtoPDF
%D   {filename}
%D   {x scale} {y scale}
%D   {x offset } {y offset}
%D   {width} {height}
%D \stoptypen
%D
%D When the scales are set to~1, the last last four values
%D are the same as the bounding box, e.g.
%D
%D \starttypen
%D \convertPDFtoPDF{mp-pra-1.pdf} {1} {1}{-1bp}{-1bp}{398bp}{398bp}
%D \convertPDFtoPDF{mp-pra-1.pdf}{.5}{.5} {0bp} {0bp}{199bp}{199bp}
%D \stoptypen
%D
%D Keep in mind, that this kind of copying only works for
%D pure and valid pdf code (without fonts).

%D The scanning and copying is straightforward and quite fast.
%D To speed up things we use two constants.

\def\@@PDFstream@@    {stream}
\def\@@PDFendstream@@ {endstream}

%D \macros
%D   {PDFmediaboxprefered}
%D
%D If needed, the macros can scan for the mediabox that
%D specifies the dimensions and offsets of the graphic. When
%D we say:
%D
%D \starttypen
%D \PDFmediaboxpreferedtrue
%D \stoptypen
%D
%D the mediabox present in the file superseded the user
%D specified, already scaled and calculated offset and
%D dimensions. Beware: the user supplied values are not the
%D bounding box ones!

\newif\ifPDFmediaboxprefered

\def\setPDFboundingbox#1#2#3#4#5#6%
  {\dimen0=#1\dimen0=#5\dimen0
   \ScaledPointsToBigPoints{\number\dimen0}\PDFxoffset
   \dimen0=#3\dimen0=#5\dimen0
   \xdef\PDFwidth{\the\dimen0}%
   \dimen0=#2\dimen0=#6\dimen0
   \ScaledPointsToBigPoints{\number\dimen0}\PDFyoffset
   \dimen0=#4\dimen0=#6\dimen0
   \xdef\PDFheight{\the\dimen0}%
   \global\let\PDFxoffset=\PDFxoffset
   \global\let\PDFyoffset=\PDFyoffset}

\def\setPDFmediabox#1[#2 #3 #4 #5]#6\done%
  {\dimen2=#2bp\dimen2=-\dimen2
   \dimen4=#3bp\dimen4=-\dimen4
   \dimen6=#4bp\advance\dimen6 by \dimen2
   \dimen8=#5bp\advance\dimen8 by \dimen4
   \setPDFboundingbox{\dimen2}{\dimen4}{\dimen6}{\dimen8}\PDFxscale\PDFyscale}

\def\checkPDFmediabox#1/MediaBox#2#3\done%
  {\ifx#2\relax \else
     \message{mediabox}%
     \setPDFmediabox#2#3\done
   \fi}

%D We use the general macro \type{\doprocessfile} and feed this
%D with a line handling macro that changed it's behavior when
%D the stream operators are encountered.

\def\handlePDFline%
  {\ifx\@@PDFstream@@\fileline
     \let\doprocessPDFline=\copyPDFobject
     \startPDFtoPDF
   \else\ifPDFmediaboxprefered
     \expandafter\checkPDFmediabox\fileline/MediaBox\relax\done
   \fi\fi}

\def\copyPDFobject%
  {\ifx\@@PDFendstream@@\fileline
     \ifPDFmediaboxprefered
       \let\doprocessPDFline=\findPDFmediabox
     \else
       \let\doprocessPDFline=\relax
     \fi
   \else
     \advance\scratchcounter by 1
     \PDFcode{\fileline}%
   \fi}

\def\findPDFmediabox%
  {\expandafter\checkPDFmediabox\fileline/MediaBox\relax\done}

%D The main conversion macro wraps the \PDF\ codes in a box
%D that is output as an object. The graphics are embedded
%D in~\type{q} and~\type{Q} and are scaled and positioned using
%D one transform call (\type{cm}). This saves some additional
%D scaling.

%D \starttypen
%D \def\startPDFtoPDF%
%D   {\setbox0=\vbox\bgroup
%D      \message{[PDF to PDF \PDFfilename}%
%D      \forgetall
%D      \scratchcounter=0
%D      \let\stopPDFtoPDF=\dostopPDFtoPDF}
%D 
%D \def\dostopPDFtoPDF%
%D   {\ifnum\scratchcounter<0 \scratchcounter=1 \fi
%D    \message{(\the\scratchcounter\space lines)]}%
%D    \egroup
%D    \wd0=\PDFwidth
%D    \vbox to \PDFheight
%D      {\forgetall
%D       \vfill
%D       \PDFcode{q}%
%D       \PDFcode{1 0 0 1 \PDFxoffset\space \PDFyoffset\space cm}%
%D       \PDFcode{\PDFxscale\space 0 0 \PDFyscale\space 0 0 cm}%
%D       \box0
%D       \PDFcode{Q}}}
%D 
%D \def\stopPDFtoPDF%
%D    {\message{[PDF to PDF \PDFfilename\space not found]}}
%D 
%D \def\convertPDFtoPDF#1#2#3#4#5#6#7%
%D   {\bgroup
%D    \def\PDFfilename{#1}%
%D    \def\PDFxscale  {#2}%
%D    \def\PDFyscale  {#3}%
%D    \setPDFboundingbox{#4}{#5}{#6}{#7}{1}{1}%
%D    \uncatcodespecials
%D    \endlinechar=-1
%D    \let\doprocessPDFline=\handlePDFline
%D    \doprocessfile\scratchread\PDFfilename\doprocessPDFline
%D    \stopPDFtoPDF
%D    \egroup}

\def\convertPDFtoPDF#1#2#3#4#5#6#7%
  {\message{[PDF to PDF use \string\PDFcode instead]}%
   \vbox{use the direct method instead}}

%D \macros 
%D   {dogetPDFmediabox}
%D
%D The next macro can be used to find the mediabox of a \PDF\
%D illustration. 
%D
%D \starttypen
%D \dogetPDFmediabox
%D   {filename}
%D   {new dimen}{new dimen}{new dimen}{new dimen} 
%D \stoptypen
%D
%D Beware of dimen clashes: this macro uses the 5~default  
%D scratch registers! When no file or mediabox is found, the 
%D dimensions are zeroed. 

\def\dogetPDFmediabox#1#2#3#4#5%
  {\bgroup
   \def\PDFxscale{1}%
   \def\PDFyscale{1}%
   \uncatcodespecials
   \endlinechar=-1
   \def\checkPDFtypepage##1/Type /Page##2##3\done%
     {\ifx##2\relax 
      \else\if##2s% accept /Page and /Pages 
        \let\doprocessPDFline=\findPDFmediabox
      \else
        \let\doprocessPDFline=\findPDFmediabox
      \fi\fi}%
   \def\findPDFtypepage%
     {\expandafter\checkPDFtypepage\fileline/Type /Page\relax\done}%
   \def\checkPDFmediabox##1/MediaBox##2##3\done%
     {\ifx##2\relax \else
        \setPDFmediabox##2##3\done
        \fileprocessedtrue
      \fi}%
   \def\findPDFmediabox%
     {\expandafter\checkPDFmediabox\fileline/MediaBox\relax\done}%
   \let\doprocessPDFline=\findPDFtypepage
   \doprocessfile\scratchread{#1}\doprocessPDFline
   \egroup
   \ifx\PDFxoffset\undefined
     #2=\!!zeropoint   #3=\!!zeropoint   #4=\!!zeropoint #5=\!!zeropoint
   \else
     #2=\PDFxoffset bp #3=\PDFyoffset bp #4=\PDFwidth    #5=\PDFheight
   \fi}

%D \macros
%D   {convertMPtoPDF}
%D
%D The next set of macros implements \METAPOST\ to \PDF\
%D conversion. Because we want to test as fast as possible, we
%D first define the \POSTSCRIPT\ operators that \METAPOST\
%D uses. We don't define irrelevant ones, because these are
%D skipped anyway.

\def \PScurveto          {curveto}
\def \PSlineto           {lineto}
\def \PSmoveto           {moveto}
\def \PSshowpage         {showpage}
\def \PSnewpath          {newpath}
\def \PSfshow            {fshow}
\def \PSclosepath        {closepath}
\def \PSfill             {fill}
\def \PSstroke           {stroke}
\def \PSclip             {clip}
\def \PSrlineto          {rlineto}
\def \PSsetlinejoin      {setlinejoin}
\def \PSsetlinecap       {setlinecap}
\def \PSsetmiterlimit    {setmiterlimit}
\def \PSsetgray          {setgray}
\def \PSsetrgbcolor      {setrgbcolor}
\def \PSsetcmykcolor     {setcmykcolor}
\def \PSsetdash          {setdash}
\def \PSgsave            {gsave}
\def \PSgrestore         {grestore}
\def \PStranslate        {translate}
\def \PSscale            {scale}
\def \PSconcat           {concat}
\def \PSdtransform       {dtransform}

\def \PSnfont            {nfont}

\def \PSBoundingBox      {BoundingBox:}
\def \PSHiResBoundingBox {HiResBoundingBox:}
\def \PSExactBoundingBox {ExactBoundingBox:}
\def \PSPage             {Page:}

%D By the way, the \type {setcmykcolor} operator is not
%D output by \METAPOST\ but can result from converting the
%D \kap{RGB} color specifications, as implemented in
%D \type{supp-mps}.

%D In \POSTSCRIPT\ arguments precede the operators. Due to the
%D fact that in some translations we need access to those
%D arguments, as well as that sometimes we have to skip them,
%D we stack them up. The stack is one||dimensional for non path
%D operators and two||dimensional for operators inside a path.
%D This is because we have to save the whole path for
%D (optional) postprocessing. Values are pushed onto the stack
%D by:
%D
%D \starttypen
%D \setMPargument {value}
%D \stoptypen
%D
%D They can be retrieved by the short named macros:
%D
%D \starttypen
%D \gMPa {number}
%D \sMPa {number}
%D \stoptypen
%D
%D When scanning a path specification, we also save the
%D operator, using
%D
%D \starttypen
%D \setMPkeyword {n}
%D \stoptypen
%D
%D The path drawing operators are coded for speed: \type{clip},
%D \type{stroke}, \type{fill} and \type{fillstroke} become
%D 1, 2, 3 and~4.
%D
%D When processing the path this code can be retrieved
%D using
%D
%D \starttypen
%D \getMPkeyword{n}
%D \stoptypen
%D
%D When setting an argument, the exact position on the stack
%D depend on the current value of the \COUNTERS\
%D \type{\nofMPsegments} and \type{\nofMParguments}.

\newcount\nofMPsegments
\newcount\nofMParguments

%D These variables hold the coordinates. The argument part of
%D the stack is reset by:
%D
%D \starttypen
%D \resetMPstack
%D \stoptypen
%D
%D We use the prefix \type{@@MP} to keep the stack from
%D conflicting with existing macros. To speed up things bit
%D more, we use the constant \type{\@@MP}.

\def\@@MP{@@MP}

\def\setMPargument#1%
  {\advance\nofMParguments by 1
   \expandafter\def
     \csname\@@MP\the\nofMPsegments\the\nofMParguments\endcsname%
     {\do#1}}

\def\gMPa#1%
  {\csname\@@MP0#1\endcsname}

\def\gMPs#1%
  {\csname\@@MP\the\nofMPsegments#1\endcsname}

\def\setMPkeyword#1
  {\expandafter\def\csname\@@MP\the\nofMPsegments0\endcsname{#1}%
   \advance\nofMPsegments by 1
   \nofMParguments=0\relax}

\def\getMPkeyword#1%
  {\csname\@@MP#10\endcsname}

%D When we reset the stack, we can assume that all further
%D comment is to be ignored as well as handled in strings.
%D By redefining the reset macro after the first call, we
%D save some run time.

\def\resetMPstack%
  {\catcode`\%=\@@active
   \let\handleMPgraphic=\handleMPendgraphic
   \def\resetMPstack{\nofMParguments=0 }%
   \resetMPstack}

%D The arguments are saved with the preceding command
%D \type{\do}. By default this command expands to nothing, but
%D when we deal with strings it's used to strip off the
%D \type{(} and \type{)}.
%D
%D Strings are kind of tricky, because characters can be
%D passed verbatim \type{(hello)}, by octal number
%D \type{(\005)} or as command \type{(\()}. We therefore
%D cannot simply ignore \type{(} and \type{)}, the way we do
%D with \type{[} and \type{]}. Another complication is that
%D strings may contain characters that normally have a
%D special meaning in \TEX, like \type{$} and \type{{}}.
%D
%D A previous solution made \type{\} an active character and
%D let it look ahead for a number or character. W ehad to
%D abandon this scheme because of the need for verbatim
%D support. The next solution involved some \CATCODE\
%D trickery but works well.

\def\octalMPcharacter#1#2#3%
  {\char'#1#2#3\relax}

\bgroup
\catcode`\|=\@@comment
\catcode`\%=\@@active
\catcode`\[=\@@active
\catcode`\]=\@@active
\catcode`\{=\@@active
\catcode`\}=\@@active
\catcode`B=\@@begingroup
\catcode`E=\@@endgroup
\gdef\ignoreMPspecials|
  B\def%BE|
   \def[BE|
   \def]BE|
   \def{BE|
   \def}BEE
\gdef\obeyMPspecials|
  B\def%B\char 37\relax E|
   \def[B\char 91\relax E|
   \def]B\char 93\relax E|
   \def{B\char123\relax E|
   \def}B\char125\relax EE
\gdef\setMPspecials|
  B\setnaturalcatcodes
   \catcode`\%=\@@active
   \catcode`\[=\@@active
   \catcode`\]=\@@active
   \catcode`\{=\@@active
   \catcode`\}=\@@active
   \def\(B\char40\relax     E|
   \def\)B\char41\relax     E|
   \def\\B\char92\relax     E|
   \def\0B\octalMPcharacter0E|
   \def\1B\octalMPcharacter1E|
   \def\2B\octalMPcharacter2E|
   \def\3B\octalMPcharacter3E|
   \def\4B\octalMPcharacter4E|
   \def\5B\octalMPcharacter5E|
   \def\6B\octalMPcharacter6E|
   \def\7B\octalMPcharacter7E|
   \def\8B\octalMPcharacter8E|
   \def\9B\octalMPcharacter9EE
\egroup

%D We use the comment symbol as a sort of trigger:

\bgroup
\catcode`\%=\@@active
\gdef\startMPscanning{\let%=\startMPconversion}
\egroup

%D In earlier versions we used the sequence
%D
%D \starttypen
%D \expandafter\handleMPsequence\input filename\relax
%D \stoptypen
%D
%D Persistent problems in \LATEX\ however forced us to use a
%D different scheme. Every \POSTSCRIPT\ file starts with a
%D \type{%}, so we temporary make this an active character
%D that starts the scanning and redefines itself. (The problem
%D originates in the redefinition by \LATEX\ of the
%D \type{\input} primitive.)

\def\startMPconversion%
  {\catcode`\%=\@@ignore
   \ignoreMPspecials
   \handleMPsequence}

%D Here comes the main loop. Most arguments are numbers. This
%D means that they can be recognized by their \type{\lccode}.
%D This method saves a lot of processing time. We could
%D speed up the conversion by handling the \type{path}
%D seperately.

\def\dohandleMPsequence#1#2 %
  {\ifdone
     \ifnum\lccode`#1=0
       \setMPargument{#1#2}%
     \else
       \edef\somestring{#1#2}%
       \ifx\somestring\PSmoveto
         \edef\lastMPmoveX{\gMPa1}%
         \edef\lastMPmoveY{\gMPa2}%
         \pdfliteral{\!MP{\gMPa1} \!MP{\gMPa2} m}%
         \resetMPstack
       \else\ifx\somestring\PSnewpath
         \let\handleMPsequence=\handleMPpath
       \else\ifx\somestring\PSgsave
         \pdfliteral{q}%
         \resetMPstack
       \else\ifx\somestring\PSgrestore
         \pdfliteral{Q}%
         \resetMPstack
       \else\ifx\somestring\PSdtransform  % == setlinewidth
         \let\handleMPsequence=\handleMPdtransform
       \else\ifx\somestring\PSconcat
         \pdfliteral{\gMPa1 \gMPa2 \gMPa3 \gMPa4 \gMPa5 \gMPa6 cm}%
         \resetMPstack
       \else\ifx\somestring\PSsetrgbcolor
         \pdfliteral{\!MP{\gMPa1} \!MP{\gMPa2} \!MP{\gMPa3} rg 
                     \!MP{\gMPa1} \!MP{\gMPa2} \!MP{\gMPa3} RG}%
         \resetMPstack
       \else\ifx\somestring\PSsetcmykcolor
         \pdfliteral{\!MP{\gMPa1} \!MP{\gMPa2} \!MP{\gMPa3} \!MP{\gMPa4} k 
                     \!MP{\gMPa1} \!MP{\gMPa2} \!MP{\gMPa3} \!MP{\gMPa4} K}% 
         \resetMPstack
       \else\ifx\somestring\PSsetgray
         \pdfliteral{\!MP{\gMPa1} g \!MP{\gMPa1} G}%
         \resetMPstack
       \else\ifx\somestring\PStranslate
         \pdfliteral{1 0 0 1 \gMPa1 \gMPa2 cm}%
         \resetMPstack
       \else\ifx\somestring\PSsetdash
         \handleMPsetdash
         \resetMPstack
       \else\ifx\somestring\PSsetlinejoin
         \pdfliteral{\gMPa1 j}%
         \resetMPstack
       \else\ifx\somestring\PSsetmiterlimit
         \pdfliteral{\gMPa1 M}%
         \resetMPstack
       \else\ifx\somestring\PSfshow
         \handleMPfshow
         \resetMPstack
       \else\ifx\somestring\PSsetlinecap
         \pdfliteral{\gMPa1 J}%
         \resetMPstack
       \else\ifx\somestring\PSrlineto
         \pdfliteral{\!MP{\lastMPmoveX} \!MP{\lastMPmoveY} l S}%
         \resetMPstack
       \else\ifx\somestring\PSscale
         \pdfliteral{\gMPa1 0 0 \gMPa2 0 0 cm}%
         \resetMPstack
       \else
         \handleMPgraphic{#1#2}%
       \fi\fi\fi\fi\fi\fi\fi\fi\fi\fi\fi\fi\fi\fi\fi\fi\fi
     \fi
   \else
     \edef\somestring{#1#2}%
     \handleMPgraphic{#1#2}%
   \fi
   \handleMPsequence}

%D Beginning and ending the graphics is taken care of by the
%D macro \type{\handleMPgraphic}, which is redefined when
%D the first graphics operator is met.

\def\handleMPendgraphic#1%
  {\ifx\somestring\PSshowpage
     \let\handleMPsequence=\finishMPgraphic
   \else
     \setMPargument{#1}%
   \fi}

\def\handleMPbegingraphic#1%
  {\ifx\somestring\PSBoundingBox
     \def\handleMPsequence{\handleMPboundingbox1}%
   \else\ifx\somestring\PSHiResBoundingBox
     \def\handleMPsequence{\handleMPboundingbox2}%
   \else\ifx\somestring\PSExactBoundingBox
     \def\handleMPsequence{\handleMPboundingbox3}%
   \else\ifx\somestring\PSshowpage
     \let\handleMPsequence=\finishMPgraphic
   \else\ifx\somestring\PSPage
     \let\handleMPsequence=\handleMPpage
   \else
     \setMPargument{#1}% kan weg
   \fi\fi\fi\fi\fi}

\let\handleMPgraphic=\handleMPbegingraphic

%D We check for three kind of bounding boxes: the normal one
%D and two high precission ones:
%D
%D \starttypen
%D BoundingBox: llx lly ucx ucy
%D HiResBoundingBox: llx lly ucx ucy
%D ExactBoundingBox: llx lly ucx ucy
%D \stoptypen
%D
%D The original as well as the recalculated dimensions are 
%D saved for later use.

\newif\ifskipemptyMPgraphic \skipemptyMPgraphicfalse

\chardef\currentMPboundingbox=0

\def\handleMPboundingbox#1#2 #3 #4 #5
  {\ifnum#1>\currentMPboundingbox
     \xdef\MPllx{#2}\xdef\MPlly{#3}%
     \xdef\MPurx{#4}\xdef\MPury{#5}%
     \dimen0=#2pt\dimen0=-\MPxscale\dimen0
     \dimen2=#3pt\dimen2=-\MPyscale\dimen2
     \xdef\MPxoffset{\withoutpt{\the\dimen0}}%
     \xdef\MPyoffset{\withoutpt{\the\dimen2}}%
     \dimen0=#2bp\dimen0=-\dimen0
     \dimen2=#3bp\dimen2=-\dimen2
     \advance\dimen0 by #4bp
     \dimen0=\MPxscale\dimen0
     \xdef\MPwidth{\the\dimen0}%
     \advance\dimen2 by #5bp
     \dimen2=\MPyscale\dimen2
     \xdef\MPheight{\the\dimen2}%
     \chardef\currentMPboundingbox=#1\relax
   \fi
   \nofMParguments=0
   \let\handleMPsequence=\dohandleMPsequence
   \let\next=\handleMPsequence
   \ifskipemptyMPgraphic
     \ifdim\MPheight=\!!zeropoint\relax\ifdim\MPwidth=\!!zeropoint\relax
       \def\next{\endinput\finishMPgraphic}%
     \fi\fi
   \fi
   \next}

%D We use the \type{page} comment as a signal that
%D stackbuilding can be started.

\def\handleMPpage #1 #2
  {\nofMParguments=0
   \donetrue
   \let\handleMPsequence=\dohandleMPsequence
   \handleMPsequence}

%D \METAPOST\ draws it dots by moving to a location and
%D invoking \type{0 0 rlineto}. This operator is not
%D available in \PDF. Our solution is straightforward: we draw
%D a line from $(current\_x, current\_y)$ to itself. This
%D means that the arguments of the preceding \type{moveto} have
%D to be saved.

\def\lastMPmoveX{0}
\def\lastMPmoveY{0}

%D These saved coordinates are also used when we handle the
%D texts. Text handling proved to be a bit of a nuisance, but
%D finaly I saw the light. It proved that we also had to
%D take care of \type{(split arguments)}.

\def\handleMPfshow%
  {\setbox0=\hbox
     {\obeyMPspecials
      \edef\size{\gMPa{\the\nofMParguments}}%
      \ifx\size\PSnfont % round font size (to pt) 
        \advance\nofMParguments by -1
        \expandafter\scratchdimen\gMPa{\the\nofMParguments}pt
        \ifdim\scratchdimen<1pt
          \def\size{1pt}%
        \else
          \advance\scratchdimen by .5pt 
          \def\size##1.##2\relax{\def\size{##1pt}}% 
          \expandafter\size\the\scratchdimen\relax
        \fi
      \else
        \edef\size{\size bp}%
      \fi
      \advance\nofMParguments by -1
      \font\temp=\gMPa{\the\nofMParguments} at \size
      \advance\nofMParguments by -1
      \temp
      \ifnum\nofMParguments=1
        \def\do(##1){##1}%
        \gMPa1%
      \else
        \scratchcounter=1
        \def\do(##1{\if##1 \char32\else##1\fi}%
        \gMPa{\the\scratchcounter}\space
        \def\do{}%
        \loop
          \advance\scratchcounter by 1
          \ifnum\scratchcounter<\nofMParguments
            \gMPa{\the\scratchcounter}\space
        \repeat
        \def\do##1){\if##1 \char32\else##1\fi}%
        \gMPa{\the\scratchcounter}%
      \fi
      \unskip}%
   \dimen0=\lastMPmoveY bp
   \advance\dimen0 by \ht0
   \ScaledPointsToBigPoints{\number\dimen0}\lastMPmoveY
   \PDFcode{n q 1 0 0 1 \lastMPmoveX\space\lastMPmoveY\space cm}%
   \dimen0=\ht0
   \advance\dimen0 by \dp0
   \box0
   \vskip-\dimen0
   \PDFcode{Q}}

%D Most operators are just converted and keep their
%D arguments. Dashes however need a bit different treatment,
%D otherwise \PDF\ viewers complain loudly. Another
%D complication is that one argument comes after the \type{]}.
%D When reading the data, we simple ignore the array boundary
%D characters. We save ourselves some redundant newlines and
%D at the same time keep the output readable by packing the
%D literals.

\def\handleMPsetdash%
  {\bgroup
   \def\somestring{[}%
   \scratchcounter=1
   \loop
   \ifnum\scratchcounter<\nofMParguments
     \edef\somestring{\somestring\space\gMPa{\the\scratchcounter}}%
     \advance\scratchcounter by 1
   \repeat
   \edef\somestring{\somestring]\gMPa{\the\scratchcounter} d}%
   \PDFcode{\somestring}%
   \egroup}

%D The \type{setlinewidth} commands look a bit complicated. There are
%D two alternatives, that alsways look the same. As John Hobby
%D says:
%D
%D \startsmaller
%D \starttypen
%D x 0 dtransform exch truncate exch idtransform pop setlinewidth
%D 0 y dtransform truncate idtransform setlinewidth pop
%D \stoptypen
%D
%D These are just fancy versions of \type{x setlinewidth} and
%D \type{y setlinewidth}. The \type{x 0 ...} form is used if
%D the path is {\em primarily vertical}. It rounds the width
%D so that vertical lines come out an integer number of pixels
%D wide in device space. The \type{0 y ...} form does the same
%D for paths that are {\em primarily horizontal}. The reason
%D why I did this is Knuth insists on getting exactly the
%D widths \TEX\ intends for the horizontal and vertical rules
%D in \type{btex...etex} output. (Note that PostScript scan
%D conversion rules cause a horizontal or vertical line of
%D integer width $n$ in device space to come out $n+1$ pixels
%D wide, regardless of the phase relative to the pixel grid.)
%D \stopsmaller
%D
%D The common operator in these sequences is \type{dtransform},
%D so we can use this one to trigger setting the linewidth.

\def\handleMPdtransform%
  {\ifdim\gMPa1pt>\!!zeropoint
     \PDFcode{\gMPa1 w}%
     \def\next##1 ##2 ##3 ##4 ##5 ##6 {\handleMPsequence}%
   \else
     \PDFcode{\gMPa2 w}%
     \def\next##1 ##2 ##3 ##4 {\handleMPsequence}%
   \fi
   \let\handleMPsequence=\dohandleMPsequence
   \resetMPstack
   \next}

%D The most complicated command is \type{concat}. \METAPOST\
%D applies this operator to \type{stoke}. At that moment the
%D points set by \type{curveto} and \type{moveto}, are already
%D fixed. In \PDF\ however the \type{cm} operator affects the
%D points as well as the pen (stroke). Like more \PDF\
%D operators, \type{cm} is a defined in a bit ambiguous way.
%D The only save route for non||circular penshapes, is saving
%D teh path, recalculating the points and applying the
%D transformation matrix in such a way that we can be sure
%D that its behavior is well defined. This comes down to
%D inverting the path and applying \type{cm} to that path as
%D well as the pen. This all means that we have to save the
%D path.

%D In \METAPOST\ there are three ways to handle a path $p$:
%D
%D \starttypen
%D draw p;  fill p;  filldraw p;
%D \stoptypen
%D
%D The last case outputs a \type{gsave fill grestore} before
%D \type{stroke}. Handling the path outside the main loops
%D saves about 40\% run time.\voetnoot{We can save some more by
%D following the \METAPOST\ output routine, but for the moment
%D we keep things simple.} Switching between the main loop and
%D the path loop is done by means of the recursely called
%D macro \type{\handleMPsequence}.

\def\handleMPpath%
  {\chardef\finiMPpath=0
   \let\closeMPpath=\relax
   \let\flushMPpath=\flushnormalMPpath
   \resetMPstack
   \nofMPsegments=1
   \let\handleMPsequence=\dohandleMPpath
   \dohandleMPpath}

%D Most paths are drawn with simple round pens. Therefore we've
%D split up the routinein two.

\def\flushnormalMPpath%
  {\scratchcounter=\nofMPsegments
   \nofMPsegments=1
   \loop
     \expandafter\ifcase\getMPkeyword{\the\nofMPsegments}\relax
       \pdfliteral{\!MP{\gMPs1} \!MP{\gMPs2} l}%
     \or
       \pdfliteral{\!MP{\gMPs1} \!MP{\gMPs2} \!MP{\gMPs3} \!MP{\gMPs4} \!MP{\gMPs5} \!MP{\gMPs6} c}%
     \or
       \pdfliteral{\!MP{\lastMPmoveX} \!MP{\lastMPmoveY} l S}%
     \or
       \edef\lastMPmoveX{\gMPs1}%
       \edef\lastMPmoveY{\gMPs2}%
       \pdfliteral{\!MP{\lastMPmoveX} \!MP{\lastMPmoveY} m}%
     \fi
     \advance\nofMPsegments by 1\relax
   \ifnum\nofMPsegments<\scratchcounter
   \repeat}

\def\flushconcatMPpath%
  {\scratchcounter=\nofMPsegments
   \nofMPsegments=1
   \loop
     \expandafter\ifcase\getMPkeyword{\the\nofMPsegments}\relax
       \doMPconcat{\gMPs1}\a{\gMPs2}\b%
       \pdfliteral{\!MP{\a} \!MP{\b} l}%
     \or
       \doMPconcat{\gMPs1}\a{\gMPs2}\b%
       \doMPconcat{\gMPs3}\c{\gMPs4}\d%
       \doMPconcat{\gMPs5}\e{\gMPs6}\f%
       \pdfliteral{\!MP{\a} \!MP{\b} \!MP{\c} \!MP{\d} \!MP{\e} \!MP{\f} c}%
     \or
       \bgroup
       \noMPtranslate
       \doMPconcat\lastMPmoveX\a\lastMPmoveY\b%
       \pdfliteral{\!MP{\a} \!MP{\b} l S}%
       \egroup
     \or
       \edef\lastMPmoveX{\gMPs1}%
       \edef\lastMPmoveY{\gMPs2}%
       \doMPconcat\lastMPmoveX\a\lastMPmoveY\b%
       \pdfliteral{\!MP{\a} \!MP{\b} m}%
     \fi
     \advance\nofMPsegments by 1\relax
   \ifnum\nofMPsegments<\scratchcounter
   \repeat}

%D The transformation of the coordinates is handled by one of
%D the macros Tanmoy posted to the \PDFTEX\ mailing list.
%D I rewrote and optimized the original macro to suit the other
%D macros in this module.
%D
%D \starttypen
%D \doMPconcat {x position} \xresult {y position} \yresult
%D \stoptypen
%D
%D By setting the auxiliary \DIMENSIONS\ \type{\dimen0} upto
%D \type{\dimen10} only once per path, we save over 20\% run
%D time. Some more speed was gained by removing some parameter
%D passing. These macros can be optimized a bit more by using
%D more constants. There is however not much need for further
%D optimization because penshapes usually are round and
%D therefore need no transformation. Nevertheless we move the
%D factor to the outer level and use bit different \type{pt}
%D removal macro. Although the values represent base points,
%D we converted them to pure points, simply because those can
%D be converted back.

\def\MPconcatfactor{256}

\def\doMPreducedimen#1
  {\count0=\MPconcatfactor
   \advance\dimen#1 \ifdim\dimen#1>\!!zeropoint .5\else -.5\fi\count0
   \divide\dimen#1 \count0\relax}

\def\doMPexpanddimen#1
  {\multiply\dimen#1 \MPconcatfactor\relax}

\def\presetMPconcat%
  {\dimen 0=\gMPs1 pt \doMPreducedimen 0    % r_x
   \dimen 2=\gMPs2 pt \doMPreducedimen 2    % s_x
   \dimen 4=\gMPs3 pt \doMPreducedimen 4    % s_y
   \dimen 6=\gMPs4 pt \doMPreducedimen 6    % r_y
   \dimen 8=\gMPs5 pt \doMPreducedimen 8    % t_x
   \dimen10=\gMPs6 pt \doMPreducedimen10 }  % t_y

\def\presetMPscale%
  {\dimen 0=\gMPs1 pt \doMPreducedimen 0 
   \dimen 2=\!!zeropoint
   \dimen 4=\!!zeropoint
   \dimen 6=\gMPs2 pt \doMPreducedimen 6 
   \dimen 8=\!!zeropoint 
   \dimen10=\!!zeropoint}  

\def\noMPtranslate% use this one grouped
  {\dimen 8=\!!zeropoint                    % t_x
   \dimen10=\!!zeropoint}                   % t_y

\def\doMPconcat#1#2#3#4%
  {\dimen12=#1 pt \doMPreducedimen12        % p_x
   \dimen14=#3 pt \doMPreducedimen14        % p_y
   %
   \dimen16  \dimen 0
   \multiply \dimen16  \dimen 6
   \dimen20  \dimen 2
   \multiply \dimen20  \dimen 4
   \advance  \dimen16 -\dimen20
   %
   \dimen18  \dimen12
   \multiply \dimen18  \dimen 6
   \dimen20  \dimen14
   \multiply \dimen20  \dimen 4
   \advance  \dimen18 -\dimen20
   \dimen20  \dimen 4
   \multiply \dimen20  \dimen10
   \advance  \dimen18  \dimen20
   \dimen20  \dimen 6
   \multiply \dimen20  \dimen 8
   \advance  \dimen18 -\dimen20
   %
   \multiply \dimen12 -\dimen 2
   \multiply \dimen14  \dimen 0
   \advance  \dimen12  \dimen14
   \dimen20  \dimen 2
   \multiply \dimen20  \dimen 8
   \advance  \dimen12  \dimen20
   \dimen20  \dimen 0
   \multiply \dimen20  \dimen10
   \advance  \dimen12 -\dimen20
   %
   \doMPreducedimen16
   \divide   \dimen18  \dimen16 \doMPexpanddimen18
   \divide   \dimen12  \dimen16 \doMPexpanddimen12
   %
   \edef#2{\withoutpt{\the\dimen18}}%       % p_x^\prime
   \edef#4{\withoutpt{\the\dimen12}}}       % p_y^\prime

%D The following explanation of the conversion process was
%D posted to the \PDFTEX\ mailing list by Tanmoy. The original
%D macro was part of a set of macro's that included sinus and
%D cosinus calculation as well as scaling and translating. The
%D \METAPOST\ to \PDF\ conversion however only needs
%D transformation.

%D \start \switchnaarkorps [ss]
%D
%D Given a point $(U_x, U_y)$ in user coordinates, the business
%D of \POSTSCRIPT\ is to convert it to device space. Let us say
%D that the device space coordinates are $(D_x, D_y)$. Then, in
%D \POSTSCRIPT\ $(D_x, D_y)$ can be written in terms of
%D $(U_x, U_y)$ in matrix notation, either as
%D
%D \plaatsformule
%D   \startformule
%D   \pmatrix{D_x&D_y&1\cr} = \pmatrix{U_x&U_y&1\cr}
%D                            \pmatrix{s_x&r_x&0\cr
%D                                     r_y&s_y&0\cr
%D                                     t_x&t_y&1\cr}
%D   \stopformule
%D
%D or
%D
%D \plaatsformule
%D   \startformule
%D    \pmatrix{D_x\cr D_y\cr 1} = \pmatrix{s_x&r_y&t_x\cr
%D                                         r_x&s_y&t_y\cr
%D                                         0  &0  &1  \cr}
%D                                \pmatrix{U_x\cr
%D                                         U_y\cr
%D                                         1  \cr}
%D   \stopformule
%D
%D both of which is a shorthand for the same set of equations:
%D
%D \plaatsformule
%D   \startformule
%D      D_x = s_x U_x + r_y U_y + t_x
%D   \stopformule
%D
%D \plaatsformule
%D   \startformule
%D      D_y = r_x U_x + s_y U_y + t_y
%D   \stopformule
%D
%D which define what is called an `affine transformation'.
%D
%D \POSTSCRIPT\ represents the `transformation matrix' as a
%D six element matrix instead of a $3\times 3$ array because
%D three of the elements are always~0, 0 and~1. Thus the above
%D transformation is written in postscript as $[s_x\, r_x\,
%D r_y\, s_y\, t_x\, t_y]$. However, when doing any
%D calculations, it is useful to go back to the original
%D matrix notation (whichever: I will use the second) and
%D continue from there.
%D
%D As an example, if the current transformation matrix is
%D $[s_x\, r_x\, r_y\, s_y\, t_x\, t_y]$ and you say \typ{[a b
%D c d e f] concat}, this means:
%D
%D \startsmaller
%D Take the user space coordinates and transform them to an
%D intermediate set of coordinates using array $[a\, b\, c\, d\,
%D e\, f]$ as the transformation matrix.
%D
%D Take the intermediate set of coordinates and change them to
%D device coordinates using array $[s_x\, r_x\, r_y\, s_y\, t_x\, t_y]$
%D as the transformation matrix.
%D \stopsmaller
%D
%D Well, what is the net effect? In matrix notation, it is
%D
%D \plaatsformule
%D   \startformule
%D     \pmatrix{I_x\cr I_y\cr 1\cr} = \pmatrix{a&c&e\cr
%D                                             b&d&f\cr
%D                                             0&0&1\cr}
%D                                    \pmatrix{U_x\cr
%D                                             U_y\cr
%D                                             1  \cr}
%D   \stopformule
%D
%D \plaatsformule
%D   \startformule
%D     \pmatrix{D_y\cr D_y\cr 1\cr} = \pmatrix{s_x&r_y&t_x\cr
%D                                             r_x&s_y&t_y\cr
%D                                             0  &0  &1  \cr}
%D                                    \pmatrix{I_x\cr
%D                                             I_y\cr
%D                                             1  \cr}
%D   \stopformule
%D
%D where $(I_x, I_y)$ is the intermediate coordinate.
%D
%D Now, the beauty of the matrix notation is that when there is
%D a chain of such matrix equations, one can always compose
%D them into one matrix equation using the standard matrix
%D composition law. The composite matrix from two matrices can
%D be derived very easily: the element in the $i$\hoog{th}
%D horizontal row and $j$\hoog{th} vertical column is
%D calculated by`multiplying' the $i$\hoog{th} row of the first
%D matrix and the $j$\hoog{th} column of the second matrix (and
%D summing over the elements). Thus, in the above:
%D
%D \plaatsformule
%D   \startformule
%D   \pmatrix{D_x\cr D_y\cr 1} = \pmatrix{s_x^\prime&r_y^\prime&t_x^\prime\cr
%D                                        r_x^\prime&s_y^\prime&t_y^\prime\cr
%D                                        0         &0         &0        \cr}
%D                               \pmatrix{U_x\cr
%D                                        U_y\cr
%D                                        1  \cr}
%D   \stopformule
%D
%D with
%D
%D \plaatsformule
%D   \startformule
%D      \eqalign
%D        {s_x^\prime & = s_x a + r_y b       \cr
%D         r_x^\prime & = r_x a + s_y b       \cr
%D         r_y^\prime & = s_x c + r_y d       \cr
%D         s_y^\prime & = r_x c + s_y d       \cr
%D         t_x^\prime & = s_x e + r_y f + t_x \cr
%D         t_y^\prime & = r_x e + s_y f + t_y \cr}
%D   \stopformule

%D In fact, the same rule is true not only when one is going
%D from user coordinates to device coordinates, but whenever
%D one is composing two `transformations' together
%D (transformations are `associative'). Note that the formula
%D is not symmetric: you have to keep track of which
%D transformation existed before (i.e.\ the equivalent of
%D $[s_x\, r_x\, r_y\, s_y\, t_x\, t_y]$) and which was
%D specified later (i.e.\ the equivalent of $[a\, b\, c\, d\,
%D e\, f]$). Note also that the language can be rather
%D confusing: the one specified later `acts earlier',
%D converting the user space coordinates to intermediate
%D coordinates, which are then acted upon by the pre||existing
%D transformation. The important point is that order of
%D transformation matrices cannot be flipped (transformations
%D are not `commutative').
%D
%D Now what does it mean to move a transformation matrix
%D before a drawing? What it means is that given a point
%D $(P_x, P_y)$ we need a different set of coordinates
%D $(P_x^\prime, P_y^\prime)$ such that if the transformation
%D acts on $(P_x^\prime, P_y^\prime)$, they produce $(P_x,
%D P_y)$. That is we need to solve the set of equations:
%D
%D \plaatsformule
%D   \startformule
%D     \pmatrix{P_x\cr P_y\cr 1\cr} = \pmatrix{s_x&r_y&t_x\cr
%D                                             r_x&s_y&t_y\cr
%D                                             0  &0  &1  \cr}
%D                                    \pmatrix{P_x^\prime\cr
%D                                             P_y^\prime\cr
%D                                             1         \cr}
%D   \stopformule
%D
%D Again matrix notation comes in handy (i.e. someone has
%D already solved the problem for us): we need the inverse
%D transformation matrix. The inverse transformation matrix can
%D be calculated very easily: it is
%D
%D \plaatsformule
%D   \startformule
%D     \pmatrix{P_x^\prime\cr P_y^\prime\cr 1\cr} =
%D        \pmatrix{s_x^\prime&r_y^\prime&t_x^\prime\cr
%D                 r_x^\prime&s_y^\prime&t_y^\prime\cr
%D                 0  &0  &1  \cr}
%D        \pmatrix{P_x\cr
%D                 P_y\cr
%D                 1         \cr}
%D   \stopformule
%D
%D where, the inverse transformation matrix is given by
%D
%D \plaatsformule
%D   \startformule
%D     \eqalign
%D       {D          & = s_x s_y - r_x r_y           \cr
%D        s_x^\prime & = s_y / D                     \cr
%D        s_y^\prime & = s_x / D                     \cr
%D        r_x^\prime & = - r_x / D                   \cr
%D        r_y^\prime & = - r_y / D                   \cr
%D        t_x^\prime & = ( - s_y t_x + r_y t_y ) / D \cr
%D        t_y^\prime & = (   r_x t_x - s_x t_y ) / D \cr}
%D   \stopformule
%D
%D And you can see that when expanded out, this does
%D give the formulas:
%D
%D \plaatsformule
%D   \startformule
%D     P_x^\prime = { { s_y(p_x-t_x) + r_y(t_y-p_y) } \over
%D                    { s_x*s_y-r_x*r_y } }
%D   \stopformule
%D
%D \plaatsformule
%D   \startformule
%D     P_y^\prime = { { s_x(p_y-t_y) + r_x(t_x-p_x) } \over
%D                    { s_x*s_y-r_x*r_y } }
%D   \stopformule
%D
%D The code works by representing a real number by converting
%D it to a dimension to be put into a \DIMENSION\ register: 2.3 would
%D be represented as 2.3pt for example. In this scheme,
%D multiplying two numbers involves multiplying the \DIMENSION\
%D registers and dividing by 65536. Accuracy demands that the
%D division be done as late as possible, but overflow
%D considerations need early division.
%D
%D Division involves dividing the two \DIMENSION\ registers and
%D multiplying the result by 65536. Again, accuracy would
%D demand that the numerator be multiplied (and|/|or the
%D denominator divided) early: but that can lead to overflow
%D which needs to be avoided.
%D
%D If nothing is known about the numbers to start with (in
%D concat), I have chosen to divide the 65536 as a 256 in each
%D operand. However, in the series calculating the sine and
%D cosine, I know that the terms are small (because I never
%D have an angle greater than 45 degrees), so I chose to
%D apportion the factor in a different way.
%D
%D \stop
%D
%D The path is output using the values saved on the stack. If
%D needed, all coordinates are recalculated.

\def\processMPpath%
  {\flushMPpath
   \closeMPpath
   \PDFcode{\ifcase\finiMPpath W n\or S\or f\or B\fi}%
   \let\handleMPsequence=\dohandleMPsequence
   \resetMPstack
   \nofMPsegments=0
   \handleMPsequence}

%D In \PDF\ the \type{cm} operator must precede the path
%D specification. We therefore can output the \type{cm} at
%D the moment we encounter it.

\def\handleMPpathconcat%
  {\presetMPconcat
   \PDFcode{\gMPs1 \gMPs2 \gMPs3 \gMPs4 \gMPs5 \gMPs6 cm}%
   \resetMPstack}

\def\handleMPpathscale%
  {\presetMPscale
   \PDFcode{\gMPs1 0 0 \gMPs2 0 0 cm}%
   \resetMPstack}

%D This macro interprets the path and saves it as compact as
%D possible.

\def\dohandleMPpath#1#2 %
  {\ifnum\lccode`#1=0
     \setMPargument{#1#2}%
   \else
     \def\somestring{#1#2}%
     \ifx\somestring\PSlineto
       \setMPkeyword0
     \else\ifx\somestring\PScurveto
       \setMPkeyword1
     \else\ifx\somestring\PSrlineto
       \setMPkeyword2
     \else\ifx\somestring\PSmoveto
       \setMPkeyword3
     \else\ifx\somestring\PSclip
       \let\handleMPsequence=\processMPpath
     \else\ifx\somestring\PSgsave
       \chardef\finiMPpath=3
     \else\ifx\somestring\PSgrestore
     \else\ifx\somestring\PSfill
       \ifnum\finiMPpath=0
         \chardef\finiMPpath=2
         \let\handleMPsequence=\processMPpath
       \fi
     \else\ifx\somestring\PSstroke
       \ifnum\finiMPpath=0
         \chardef\finiMPpath=1
       \fi
       \let\handleMPsequence=\processMPpath
     \else\ifx\somestring\PSclosepath
       \def\closeMPpath{\PDFcode{h}}%
     \else\ifx\somestring\PSconcat
       \let\flushMPpath=\flushconcatMPpath
       \handleMPpathconcat
     \else\ifx\somestring\PSscale
       \let\flushMPpath=\flushconcatMPpath
       \handleMPpathscale
     \fi\fi\fi\fi\fi\fi\fi\fi\fi\fi\fi\fi
   \fi
   \handleMPsequence}

%D The main conversion command is
%D
%D \starttypen
%D \convertMPtoPDF {filename} {x scale} {y scale}
%D \stoptypen
%D
%D The dimensions are derived from the bounding box. So we
%D only have to say:
%D
%D \starttypen
%D \convertMPtoPDF{mp-pra-1.eps}{1}{1}
%D \convertMPtoPDF{mp-pra-1.eps}{.5}{.5}
%D \stoptypen

\def\processMPtoPDFfile% file xscale yscale 
  {\bgroup
   \let\finishMPgraphic=\egroup
   \doprocessMPtoPDFfile}

\ifx\deleteMPgraphic\undefined
  \def\deleteMPgraphic#1{}
\fi

\def\doprocessMPtoPDFfile#1#2#3% file xscale yscale 
  {\setMPspecials
   \catcode`\^^M=\@@endofline
   \startMPscanning
   \let\do=\empty
   \xdef\MPxscale{#2}%
   \xdef\MPyscale{#3}%
   \donefalse
   \let\handleMPsequence=\dohandleMPsequence
   \message{[MP to PDF #1]}%
   \input#1\relax
   \deleteMPgraphic{#1}}

\def\convertMPtoPDF#1#2#3%
  {\bgroup
   \setbox0=\vbox\bgroup
     \forgetall
     \offinterlineskip
     \PDFcode{q}%
     \doprocessMPtoPDFfile{#1}{#2}{#3}}

\def\finishMPgraphic%
  {\PDFcode{Q}%
   \egroup
   \wd0=\MPwidth
   \vbox to \MPheight
     {\forgetall
      \vfill
      \PDFcode{q \MPxscale\space 0 0 \MPyscale\space
        \MPxoffset\space \MPyoffset\space cm}%
      \box0
      \PDFcode{Q}}%
   \egroup}

%D \macros
%D   {twodigitMPoutput}
%D
%D We can limit the precission to two digits after the comma 
%D by saying: 
%D
%D \starttypen
%D \twodigitMPoutput
%D \stoptypen
%D
%D This option only works in \CONTEXT\ combined with \ETEX. 

\def\twodigitMPoutput%
  {\let\!MP\twodigitrounding}

\def\!MP#1{#1}

%D This kind of conversion is possible because \METAPOST\
%D does all the calculations. Converting other \POSTSCRIPT\
%D files would drive both me and \TEX\ crazy.

\protect

\endinput
