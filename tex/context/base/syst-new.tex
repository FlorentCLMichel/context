%D \module
%D   [       file=supp-new,
%D        version=1997.01.03,
%D          title=\CONTEXT\ Support Macros,
%D       subtitle=New Ones,
%D         author=Hans Hagen,
%D           date=\currentdate,
%D      copyright={PRAGMA / Hans Hagen \& Ton Otten}]
%C
%C This module is part of the \CONTEXT\ macro||package and is
%C therefore copyrighted by \PRAGMA. See mreadme.pdf for 
%C details. 

\unprotect

% \buildarray[test][aa,bb,cc,dd,ee,ff]
% \setarrayelement{test}{1}{qq}
% \arrayelement{test}{1}
% \arraylength{test}
% 
% \def\buildarray[#1][#2]%
%   {\scratchcounter=0  
%    \def\docommando##1%
%      {\advance\scratchcounter by 1   
%       \setvalue{@@aa#1\the\scratchcounter}{##1}}%
%    \processcommalist[#2]\docommando
%    \setevalue{@@aa#1}{\the\scratchcounter}}%
% 
% \def\setarrayelement#1#2{\setvalue{@@aa#1#2}}
% \def\arrayelement   #1#2{\getvalue{@@aa#1#2}}
% \def\arraylength      #1{\getvalue{@@aa#1}}

% \newsignal\junksignal
% 
% \def\setjunksignal%
%   {\ifhmode
%      \hskip\junksignal
%      \let\removejunkspaces\doremovejunkspaces
%    \else
%      \let\removejunkspaces\relax
%    \fi}
% 
% \def\doremovejunkspaces%
%   {\doloop{\ifdim\lastskip=\junksignal\unskip\else\exitloop\fi}}

% \def\doifnonzeropositiveelse#1#2#3%
%   {\bgroup
%    \setbox\scratchbox=\hbox{\scratchcounter=0#1}%
%    \ifdim\wd\scratchbox=\!!zeropoint\egroup#2\else\egroup#3\fi}
%
% \def\dodoifnonzeropositiveelse#1#2\end#3#4%
%   {\egroup
%    \ifx#1\relax#3\else#4\fi}
%
% \def\doifnonzeropositiveelse#1%
%   {\bgroup
%    \afterassignment\dodoifnonzeropositiveelse\scratchcounter=0#1\relax\empty\end}

\def\dodoifnonzeropositiveelse#1#2\end#3#4%
  {\ifx#1\relax\ifcase\scratchcounter\egroup#4\else\egroup#3\fi\else\egroup#4\fi}

\def\doifnonzeropositiveelse#1%
  {\bgroup\afterassignment\dodoifnonzeropositiveelse\scratchcounter=0#1\relax\empty\end}

% here ? 

\def\dosetrawvalue#1#2#3%
  {\setvalue{#1#2}{#3}}

\def\getrawparameters%
  {\dogetparameters\dosetrawvalue}

\def\splitskip#1%
  {\scratchskip=0pt plus 1pt minus 1pt
   \advance\scratchskip by #1
   \expandafter\SPLITSKIP\the\scratchskip}

{\catcode`\.=\@@other
 \catcode`\p=\@@other
 \catcode`\l=\@@other
 \catcode`\u=\@@other
 \catcode`\s=\@@other
 \catcode`\m=\@@other
 \catcode`\i=\@@other
 \catcode`\n=\@@other
 \catcode`\t=\@@other
 \gdef\SPLITSKIP#1pt plus #2pt minus #3pt%
   {\DOSPLITSKIP#1 #2 #3 }}

\def\DOSPLITSKIP#1 #2 #3
  {\dimen0=#1pt\dimen2=#2pt\dimen4=#3pt
   \advance\dimen2 by -1pt%
   \advance\dimen4 by -1pt}

% \def\minimaxskip#1#2%
%   {\splitskip#2\relax
%    \scratchdimen=#2\relax
%    #2=\scratchdimen   
%    \advance#2 by #1\relax}
% 
% \def\maximizeskip%
%   {\minimaxskip{-\dimen4}}
% 
% \def\maximizeskip%
%   {\minimaxskip{\dimen2}}
% 
% \def\maximizespacing%
%   {\maximizeskip\blankoskipamount
%    \maximizeskip\parskip
%    \maximizeskip\tussenwit
%    \maximizeskip\baselineskip
%    \maximizeskip\bigskipamount
%    \maximizeskip\medskipamount
%    \maximizeskip\smallskipamount}

\newcount\modcounter

\def\DoMod #1by#2to#3%
  {\modcounter=#1\relax
   \divide\modcounter by #2\relax
   \multiply\modcounter by #2\relax
   #3=#1\relax
   \advance#3 by -\modcounter}

\def\DoDiv #1by#2to#3%
  {#3=#1\relax
   \divide#3 by #2\relax}

\def\dounprotected#1\par%
  {#1\protect}

\def\unprotected%
  {\unprotect\dounprotected}

%D Standaard kan een spatie (zoals ~) uitrekken. Dit is in
%D overzichten niet altijd de bedoeling, vandaar:

\def\fixedspace%
  {\hskip\fontdimen2\font\relax}

%\def\ExpandSecondAfter#1#2#3%
%  {\!!toksa={#2}%
%   \edef\!!stringa{#3}%
%   \edef\expanded%
%     {\noexpand#1{\the\!!toksa}{\!!stringa}}%
%   \expanded}
%
%\def\ExpandThirdAfter#1#2#3#4%
%  {\!!toksa={#2}%
%   \!!toksb={#3}%
%   \edef\!!stringa{#4}%
%   \edef\expanded%
%     {\noexpand#1{\the\!!toksa}{\the\!!toksb}{\!!stringa}}%
%   \expanded}

%\def\indirect#1#2#3%
%  {\@EA#1\@EA#2\@EA{\@EA#3\csname\s!do\string#2\endcsname}%
%   \@EA#1\csname\s!do\string#2\endcsname}
%
%\def\doubleemptied#1#2#3%
%  {\indirect#1#2\dodoublempty}
%
%\indirect\def\stelietsin\dodoubleempty[#1][#2]%
%  {...}
%
%\doubleemptied\def\stelietsin[#1][#2]%
%  {...}

% in mult-set
%
%\def\defaultsetup{def}
%
%\def\selectdefaultsetup#1#2%
%  {\writestatus{setup}{choose #1 setupfile}%
%   \bgroup
%   \endlinechar=-1
%   \global\read16 to \usersetup
%   \egroup
%   \ifx\usersetup\empty
%     \let\usersetup=\defaultsetup
%   \fi
%   \readfile{#2\usersetup}{}{}%
%   \writestatus{setup}{loading #1 setupfile #2\usersetup}}

\newcount\featuretest

\def\testfeature#1#2%
  {\def\dotestfeature%
     {\advance\featuretest 1
      \ifnum\featuretest<#1\relax#2\expandafter\dotestfeature\fi}%
   \retestfeature}

\def\retestfeature%
  {\message{starting feature test}\wait
   \featuretest=-1 \dotestfeature
   \message{feature test done}\wait}

%D \macros
%D   {adddimenregister,adddimenmacro}
%D
%D Instead of using numerous \type {\advance}'s, one can use
%D the next macros to add|/|subtract a series of dimensions 
%D to a register or macro.  
%D
%D \starttypen 
%D \adddimenregister 10pt 5pt \papierbreedte \to \somedimen
%D \adddimenmacro    10pt 5pt \papierbreedte \to \bagger
%D \stoptypen

\newdimen\dimentoaddto

\def\adddimenregister#1\to#2%
  {\bgroup
   #2=\!!zeropoint
   \dimentoaddto\!!zeropoint
   \def\docommando%
     {\advance#2 by \dimentoaddto
      \futurelet\next\dodocommando}%
   \def\dodocommando%
     {\ifx\next\relax
        \expanded{\egroup#2=\the#2}%
      \else
        \@EA\afterassignment\@EA\docommando\@EA\dimentoaddto
      \fi}%
   \docommando#1\relax}

\def\adddimenmacro#1\to#2%
  {\adddimenregister#1\to\scratchdimen
   \edef#2{\the\scratchdimen}}

%D \macros
%D   {freezedimenmacro}
%D
%D This macro is use as: 
%D
%D \starttypen 
%D \freezedimenmacro\linkermargeafstand
%D \stoptypen 

\def\freezedimenmacro#1%
  {\scratchdimen#1\edef#1{\the\scratchdimen}}

% \newcount\rawrecursecounter
% 
% \def\rawrecurselevel{\the\rawrecursecounter}%
% 
% \def\dorawrecurse#1#2%
%   {\rawrecursecounter=1\relax
%    \let\oldrecurselevel\recurselevel
%    \let\recurselevel\rawrecurselevel
%    \def\dodorawrecurse%
%      {\ifnum\rawrecursecounter>#1\relax
%         \let\recurselevel\oldrecurselevel
%       \else
%         #2\advance\rawrecursecounter by 1
%         \expandafter\dodorawrecurse
%       \fi}%
%    \dodorawrecurse}

\protect

\endinput
