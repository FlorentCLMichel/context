%D \module
%D   [       file=cont-log,
%D        version=1995.10.10,
%D          title=\CONTEXT\ Miscellaneous Macros,
%D       subtitle=\TEX\ Logos,
%D         author=J. Hagen,
%D           date=\currentdate,
%D      copyright={PRAGMA / Hans Hagen \& Ton Otten}]
%C
%C This module is part of the \CONTEXT\ macro||package and is
%C therefore copyrighted by \PRAGMA. See mreadme.pdf for
%C details.

\writestatus{loading}{ConTeXt TeX Logos}

%D The system that is used to typeset this text is called \TEX,
%D typeset with an lowered~E. From te beginning of \TEX,
%D authors of macro packages adapted this raising and lowering
%D style. In this module we define some of those logos.
%D Watch the \type{cmr} detection hack.

\unprotect

\defconvertedargument\someCMRfont{cmr} % hm, we now have lm

% \def\doifCMRfontelse#1#2%
%   {\doifinstringelse{\someCMRfont}{\fontname\font}
%      {\def\next{#1}}
%      {\def\next{#2}}%
%    \next}

\def\doifCMRfontelse
  {\doifinstringelse\someCMRfont{\fontname\font}}

\unexpanded\def\CMRkern
  {\doifCMRfontelse\kern{\scratchdimen=}}

% \def\TeX
%   {T%
%    \kern-.1667em\lower.5ex\hbox{E}%
%    \kern-.125emX}

\def\Mkern#1%
  {{\setbox\scratchbox\hbox{M}\kern#1\wd\scratchbox}}

\unexpanded\def\TeX
  {T%
   \Mkern{-.1667}\lower.5ex\hbox{E}%
   \Mkern{-.125}X}

\unexpanded\def\ConTeXt
                  {C%
   \CMRkern-.0333emo%
   \CMRkern-.0333emn%
%  \CMRkern-.1667em\TeX%
   \CMRkern-.0667em\TeX%
   \CMRkern-.0333emt}

\unexpanded\def\PPCHTeX
  {ppch\TeX}

\unexpanded\def\PRAGMA
  {Pragma ADE}

%\def\LaTeX
%  {L%
%   \kern-.30em\raise.3ex\hbox{\txx A}%
%   \kern-.18em\TeX}

\unexpanded\def\LaTeX % requested by erik frambach
  {{\setbox\scratchbox\hbox{L}%
    \scratchdimen\ht\scratchbox
    \setbox\scratchbox\hbox{\txx A}%
    L\kern-.55\wd\scratchbox
    \raise\scratchdimen\hbox{\lower\ht\scratchbox\copy\scratchbox}%
    \kern-.2\wd\scratchbox\TeX}}

\unexpanded\def\TaBlE
  {T%
   \kern-.27em\lower.5ex\hbox{A}%
   \kern-.18emB%
   \kern-.1em\lower.5ex\hbox{L}%
   \kern-.075emE}

\unexpanded\def\PiCTeX
  {P%
   \kern-.12em\lower.5ex\hbox{I}%
   \kern-.075em C%
   \kern-.11em\TeX}

\def\AMSswitch#1%
  {$\cal\ifdim\bodyfontsize>1.1em\scriptstyle\fi#1$}

\unexpanded\def\AmSTeX
  {\AMSswitch A%
   \kern-.1667em\lower.5ex\hbox{\AMSswitch M}%
   \kern-.125em\AMSswitch S%
   -\TeX}

\unexpanded\def\LamSTeX
  {L%
   \kern-.4em\raise.3ex\hbox{\AMSswitch A}%
   \kern-.25em\lower.4ex\hbox{\AMSswitch M}%
   \kern-.1em{\AMSswitch S}%
   -\TeX}

\unexpanded\def\AmSLaTeX
  {\AMSswitch A%
   \kern-.1667em\lower.5ex\hbox{\AMSswitch M}%
   \kern-.125em\AMSswitch S%
   -\LaTeX}

%D Alternative \CONTEXT\ logo, first Idris S.~Hamid's version:
%D
%D \def\Context
%D   {{\sc C\kern -.0667emo\kern -.0667emn\kern -.0549emt\kern
%D    -.1667em\lower.5ex\hbox {e}\kern -.125emx\kern -.0549emt}}
%D
%D I changed this into one that adapts itself:

\unexpanded\def\Context
  {{C\kern -.0667em\getscaledglyph{.8}\empty{O\kern -.0667emN\kern
   -.0549emT\doifitalicelse{\kern-.1em}{\kern-.1667em}\lower.5ex\hbox
   {E}\doifitalicelse\empty{\kern-.11em}X\kern-.055emT}}}

%D The \METAFONT\ and \METAPOST\ logos adapt themselves to the
%D current fontsize, an ugly but usefull hack.

% rather hard coded
%
% \loadmapfile[original-base.map] % \loadmapfile[original-vogel-symbol]
%
% \unexpanded\def\setMFPfont
%   {\font\logofont=logo%
%      \ifnum\fam=\bffam\c!bf\else
%      \ifnum\fam=\slfam\c!sl\else
%      \ifnum\fam=\itfam\c!sl\else
%      \ifnum\fam=\bsfam\c!bf\else
%      \ifnum\fam=\bifam\c!bf\else
%      \fi\fi\fi\fi\fi
%      10 at \currentfontscale\bodyfontsize
%    \logofont}
%
% or:
%
% \definefontsynonym [MetaLogo]            [logo10]
% \definefontsynonym [MetaLogoBold]        [logobf10]
% \definefontsynonym [MetaLogoSlanted]     [logosl10]
% \definefontsynonym [MetaLogoItalic]      [logosl10]
% \definefontsynonym [MetaLogoBoldSlanted] [logobf10]
% \definefontsynonym [MetaLogoBoldtalic]   [logobf10]
%
% \loadmapfile[original-base.map] % \loadmapfile[original-vogel-symbol]
%
% \def\setMFPfont{\symbolicfont{MetaLogo}}

\let\logofont\nullfont

\loadmapfile[original-base.map]

\unexpanded\def\setMFPfont% more sensitive for low level changes
  {\font\logofont=logo%
     \ifx\fontalternative\c!bf\else
     \ifx\fontalternative\c!it\else
     \ifx\fontalternative\c!sl\else
     \ifx\fontalternative\c!bi\else
     \ifx\fontalternative\c!bs\else
     \fi\fi\fi\fi\fi
     10 at \currentfontscale\bodyfontsize
   \logofont}

%\unexpanded\def\MetaFont%
%  {\hbox{\setMFPfont METAFONT}}
%
%\unexpanded\def\MetaPost%
%  {\hbox{\setMFPfont METAPOST}}

\def\MetaHyphen% there is no hyphenchar in this font
  {\discretionary{\vrule\!!height.33em\!!depth-.27em\!!width.33em}{}{}}

\unexpanded\def\MetaFont
  {{\setMFPfont META\MetaHyphen FONT}}

\unexpanded\def\MetaPost
  {{\setMFPfont META\MetaHyphen POST}}

\unexpanded\def\MetaFun
  {MetaFun}

%D \macros
%D  {TEX, METAFONT, METAPOST, METAFUN,
%D   PICTEX, TABLE,
%D   CONTEXT, PPCHTEX,
%D   AMSTEX, LATEX, LAMSTEX}
%D
%D We define the funny written ones as well as th eless
%D error prone upper case names (in \CONTEXT\ we tend to
%D write all user defined commands, like abbreviations, in
%D uppercase.)

\unexpanded\def\METAFONT {\MetaFont}
\unexpanded\def\METAPOST {\MetaPost}
\unexpanded\def\PPCHTEX  {\PPCHTeX}
\unexpanded\def\CONTEXT  {\ConTeXt}
\unexpanded\def\METAFUN  {\MetaFun}

\unexpanded\def\TEX      {\TeX}
\unexpanded\def\LATEX    {\LaTeX}
\unexpanded\def\PICTEX   {\PiCTeX}
\unexpanded\def\TABLE    {\TaBlE}
\unexpanded\def\AMSTEX   {\AmSTeX}
\unexpanded\def\LAMSTEX  {\LamSTeX}
\unexpanded\def\INRSTEX  {inrs\TeX}

%D And this is how they show up: \TeX, \MetaFont, \MetaPost,
%D \PiCTeX, \TaBlE, \ConTeXt, \PPCHTeX, \AmSTeX, \LaTeX,
%D \LamSTeX.

% \def\TEXEDIT  {\TeX edit}
% \def\TEXFORM  {\TeX form}
% \def\TEXADRES {\TeX adres}
% \def\TEXSPELL {\TeX spell}
% \def\TEXUTIL  {\TeX util}
% \def\TEXEXEC  {\TeX exec}

%D Some placeholders:

\unexpanded\def\eTeX   {\mathematics{\varepsilon}-\TeX}
\unexpanded\def\pdfTeX {pdf\TeX}
\unexpanded\def\pdfeTeX{pdfe-\TeX}
\unexpanded\def\luaTeX {lua\TeX}
\unexpanded\def\metaTeX{meta\TeX}
\unexpanded\def\XeTeX  {X\lower.5ex\hbox{\kern-.15em\mirror{E}}\kern-.1667em\TeX}

% Better, since lm has a mirrored E (don't ask me why)

% \unexpanded\def\XeTeX
%   {X\lower.5ex
%    \hbox
%      {\kern-.15em
%       \ifx\XeTeXcharglyph\undefined
%         \mirror{E}%
%       \else\ifcase\XeTeXcharglyph"018E\relax
%         \mirror{E}%
%       \else
%         \char"018E%
%       \fi}%
%    \kern-.1667em \TeX}

% Adapted from a patch by Mojca:

\def\@XeTeX@
  {\setbox\scratchbox\hbox{E}%
   \raise\dimexpr\ht\scratchbox+\dp\scratchbox\relax\hbox{\rotate[\c!rotation=180]{\box\scratchbox}}}

\ifnum\texengine=\pdftexengine

    \unexpanded\def\XeTeX
      {X\lower.5ex
       \hbox
         {\kern-.15em
          \ifx\fontalternative\c!bf\mirror{E}\else
          \ifx\fontalternative\c!it  \@XeTeX@\else
          \ifx\fontalternative\c!sl  \@XeTeX@\else
          \ifx\fontalternative\c!bi  \@XeTeX@\else
          \ifx\fontalternative\c!bs  \@XeTeX@\else
                                   \mirror{E}\fi\fi\fi\fi\fi}%
       \kern-.1667em \TeX}

\else

    \unexpanded\def\XeTeX
      {X\lower.5ex
       \hbox
         {\kern-.15em
          \iffontchar\font"018E\relax
            \char"018E%
          \else
            \ifx\fontalternative\c!bf\mirror{E}\else
            \ifx\fontalternative\c!it  \@XeTeX@\else
            \ifx\fontalternative\c!sl  \@XeTeX@\else
            \ifx\fontalternative\c!bi  \@XeTeX@\else
            \ifx\fontalternative\c!bs  \@XeTeX@\else
                                     \mirror{E}\fi\fi\fi\fi\fi
          \fi}%
       \kern-.1667em \TeX}

\fi

\let\ETEX   \eTeX
\let\PDFTEX \pdfTeX
\let\PDFETEX\pdfeTeX
\let\LUATEX \luaTeX
\let\LuaTeX \luaTeX
\let\XETEX  \XeTeX

\unexpanded\def\MkApproved
  {\dontleavehmode\rotate
     [\c!rotation={\ifnum\texengine=\luatexengine\ctxlua{tex.write(45-45*\the\luatexversion/100)}\else0\fi},
      \c!align=\v!middle,
      \c!foregroundstyle=\v!type,
      \c!foregroundcolor=darkred,
      \c!frame=\v!on,
      \c!offset=1ex,
      \c!background=\v!color,
      \c!backgroundcolor=lightgray,
      \c!framecolor=darkred,
      \c!rulethickness=2pt]
     {Mk\ifnum\texengine=\luatexengine IV\else II\fi\\approved}}


% \unexpanded\def\luaTeX
%   {\dontleavehmode\begingroup
%    Lua%
%    \setbox0\hbox{oT}%
%    \setbox2\hbox{o\kern0ptT}%
%    \ifdim\wd0=\wd2
%      \setbox0\hbox dir TRT{To}%
%      \setbox2\hbox{T\kern0pto}%
%      \hskip\dimexpr\wd0-\wd2\relax
%    \fi
%    \TeX
%    \endgroup}
%
% a further iteration from the list, patched again

% \ifx\fontalternative\c!it -\else
% \ifx\fontalternative\c!sl -\else
% \ifx\fontalternative\c!bi -\else
% \ifx\fontalternative\c!bs -\fi\fi\fi\fi

\def\LuaTeX
  {\dontleavehmode
   \begingroup
     Lua%
     % hope for kerning, try aT
     \setbox0\hbox{aT}%
     \setbox2\hbox{a\kern\zeropoint T}%
     \ifdim\wd0=\wd2 % kerns can go two ways
       % no aT kerning, try oT as a is not symmetrical
       \setbox0\hbox{oT}%
       \setbox2\hbox{o\kern\zeropoint T}%
       \ifdim\wd0=\wd2 % kerns can go two ways
         % no aT and oT kerning, try To
         \setbox0\hbox{To}%
         \setbox2\hbox{T\kern\zeropoint o}%
         % maybe we need to compensate for the angle (sl/it/bs/bi)
       \fi
       \ifdim\wd0=\wd2\else
         \kern\dimexpr\wd0-\wd2\relax
       \fi
     \fi
     \TeX
   \endgroup}

\let\luaTeX \LuaTeX
\let\LUATEX \LuaTeX

\protect \endinput
