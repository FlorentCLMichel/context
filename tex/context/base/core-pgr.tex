%D \module
%D   [       file=core-pgr, % split off core-pos
%D        version=1999.08.01,
%D          title=\CONTEXT\ Core Macros,
%D       subtitle=Positioning Support,
%D         author=Hans Hagen,
%D           date=\currentdate,
%D      copyright={PRAGMA / Hans Hagen \& Ton Otten}]
%C
%C This module is part of the \CONTEXT\ macro||package and is
%C therefore copyrighted by \PRAGMA. See mreadme.pdf for
%C details.

\writestatus{loading}{Context Positioning Grapics}

%D Before we come to graphics support, we have to make sure of
%D the reference point on the page. The next macro does so and
%D is hooked into the page building routine.

\unprotect

% in the future, the depth of tail will reflect page depth

\ifx\textheight\undefined \def\textheight{\vsize} \fi

%D The next macros so some housekeeping.

\def\pageanchor{page:0}
\def\textanchor{text:\realfolio}
\def\headanchor{head:\realfolio} % virtual position
\def\tailanchor{tail:\realfolio} % virtual position

%D Anchors:

\def\dopresetpositionanchors % also mkii
  {\bgroup
   \!!dimena\ifdim\topskip>\strutht\topskip\else\strutht\fi
   \!!dimenb\dimexpr\MPy\textanchor+\MPh\textanchor-\!!dimena\relax
   \!!dimenc\dimexpr\MPy\textanchor+\strutdp\relax
   \!!dimend\MPx\textanchor
   \!!dimene\MPw\textanchor
   \replacepospxywhd\headanchor\realfolio\!!dimend\!!dimenb\!!dimene\!!dimena\strutdp
   \replacepospxywhd\tailanchor\realfolio\!!dimend\!!dimenc\!!dimene\strutht \strutdp
   \egroup}

\def\presetpositionanchors% compatibility hack (still needed?)
  {\ifpositioning
     \dopresetpositionanchors
   \fi}

%D The first version of this module implemented head and tail
%D anchors. Currently we stick to just one anchor and derive
%D the head and tail anchors from this one.

\def\showanchor#1%
  {\expanded{\writestatus{#1}
     {\MPp{#1}\string|\MPx{#1}\string|\MPy{#1}\string|%
      \MPw{#1}\string|\MPh{#1}\string|\MPd{#1}}}}

%D We set these anchors before and after each page.

\appendtoks \presetpositionanchors \to \beforeeverypage
\appendtoks \presetpositionanchors \to \aftereverypage

% todo: change with each page size change

\def\registerpageposition#1% this one is flushed first !
  {\ifpositioning\ifcase\realpageno\or
     \ifdim\printpaperheight=\paperheight\else
       \ifdim\printpaperwidth=\paperwidth\else
         \setbox#1\hbox{\hpos\pageanchor{\box#1}}%
       \fi
     \fi
   \fi\fi}

\def\placepositionanchors % todo : depth pagebox
  {\ifpositioning
     \setbox\scratchbox\vbox to \textheight
       {\simpletopskipcorrection
        \hbox{\strut\dopositionaction\headanchor}%
        \vfill
        \hbox{\strut\dopositionaction\tailanchor}}%
     \dp\scratchbox\zeropoint
     \wd\scratchbox\makeupwidth % not \zeropoint, else wrong text backgrounds
     \hpos\textanchor{\box\scratchbox}%
   \else
     \vskip\textheight
   \fi}

%D \macros
%D   {positionoverlay,startpositionoverlay}
%D
%D As long as we're dealing with graphics it makes much sense
%D to use the available overlay mechanism. For this purpose, we
%D define some dedicated overlay extensions.
%D
%D \startbuffer[sample]
%D \defineoverlay [sample] [\positionoverlay{sample}]
%D
%D \startpositionoverlay{sample}
%D   \setMPpositiongraphic{A-1}{connectcenter}{from=A-1,to=A-2}
%D \stoppositionoverlay
%D \stopbuffer
%D
%D \typebuffer[sample]
%D
%D \startbuffer[graphic]
%D \startMPpositiongraphic{connectcenter}
%D   path pa, pb ; pair ca, cb ;
%D   initialize_box(\MPpos{\MPvar{from}}) ; pa := pxy ; ca := cxy ;
%D   initialize_box(\MPpos{\MPvar{to}}) ; pb := pxy ; cb := cxy ;
%D   draw pa withcolor red ;
%D   draw pb withcolor red ;
%D   draw ca -- cb withcolor blue ;
%D   anchor_box(\MPanchor{\MPvar{from}}) ;
%D \stopMPpositiongraphic
%D \stopbuffer
%D
%D We can best demonstrate this in an example, say:
%D
%D \startbuffer[text]
%D \framed
%D   [backgroundachtergrond=sample,align=middle,width=7cm]
%D   {We want to connect \hpos {A-1} {this} word with its
%D    grammatical cousin \hpos {A-2} {that}.}
%D \stopbuffer
%D
%D \typebuffer[text]
%D
%D \startlinecorrection
%D %\getbuffer[graphic,sample,text]
%D \stoplinecorrection
%D
%D The graphic is defined in the following way, using some
%D macros defined in an auxiliary \METAPOST\ module that is
%D preloaded.
%D
%D \typebuffer[graphic]

\def\MPanchoridentifier{mpa} % {mp-anchor}
\def\MPoverlayposprefix{MO::}

% obsolete and wrong anyway
%
% \long\def\defineMPpositiongraphic#1%
%   {\long\setvalue{\MPoverlayposprefix#1}}

%D The rest of the definitions concerning such overlays may
%D look complicated,

\let\currentpositionoverlay\empty

%D Position actions are automatically executed when a position
%D is set.

\let\MPanchornumber\realfolio

\def\positionoverlay#1% the test prevents too many redundant positions
  {\ifpositioning     % in (not used) text* position layers
     \vbox to \overlayheight
       {\doifpositionactionelse{#1::\MPanchoridentifier}%
          {\edef\MPanchorid{#1::\MPanchoridentifier:\MPanchornumber}%
           \edef\MPanchor##1{\MPpos{\MPanchorid}}%
           \the\everyinsertpositionaction
           \copyposition{#1::\MPanchoridentifier}{#1::\MPanchoridentifier:\MPanchornumber}%
           \hpos
             {#1::\MPanchoridentifier:\MPanchornumber}%
            % this is ok
            %{\hbox to \overlaywidth{\dopositionaction{#1::\MPanchoridentifier}\hss}}}%
            % but this one prevents cyclic runs due to
            % rounding errors
             {\setbox\scratchbox\hbox to \overlaywidth
                {\dopositionaction{#1::\MPanchoridentifier}\hss}%
              \ht\scratchbox\overlayheight
              \dp\scratchbox\zeropoint
              \box\scratchbox}}%
          {\hbox to \overlaywidth{\hss}}%
        \vfill}%
   \fi}

\def\startpositionoverlay#1%
  {\iftrialtypesetting % we don't want redundant entries in the list
     \@EA\gobbleuntil\@EA\stoppositionoverlay
   \else
     \def\currentpositionoverlay{#1}%
   \fi}

\def\stoppositionoverlay
  {\let\currentpositionoverlay\empty}

\def\resetpositionoverlay#1%
  {\dosetpositionaction{#1::\MPanchoridentifier::}{}}

%D Here the complication has to do with collecting actions
%D for later execution. This collection is especially handy
%D when we want to move actions to a specific layer.
%D Such series of actions are stored in a macro (the one
%D with the funny \type {++}) which is cleaned up after each
%D invocation.

\newtoks\everycleanpositionaction
\newtoks\everyinsertpositionaction

\def\cleanuppositionaction#1% not in trialtypesetting
  {\ifundefined{\POSactionprefix#1++}\else
     \the\everycleanpositionaction
     \iflocalpositioning
       \letgvalue{\POSactionprefix#1++}\empty
     \else
       \setxvalue{\POSactionprefix#1++}{\getvalue{\POSactionprefix#1++}}%
     \fi
   \fi}

\def\handlepositionaction#1\with#2\on#3%
  {\bgroup
   \ifx\currentpositionoverlay\empty
     \edef\!!stringa{#3}% no layer, just pos itself as anchor
   \else
     \edef\!!stringa{\currentpositionoverlay::\MPanchoridentifier}%
   \fi
   \edef\!!stringc{\POSactionprefix\!!stringa++}%
   \expanded{\dosetpositionaction{\!!stringa}{\noexpand\getvalue{\!!stringc}}}%
   \global\let#1\relax
   \edef\!!stringb{\executeifdefined\!!stringc\empty}%
   \setxvalue\!!stringc{\!!stringb#1#2}%
   \egroup}

%D The indirectness enables us redefine macros for special
%D purposes, like a cleanup.

\def\handlepositionboxes#1#2#3%
  {\handlepositionaction\dohandlepositionboxes\with{#1}{#2}{#3}\on{#2}}

\def\doinsertpositionboxes#1#2#3% pos tag setups
  {\ifnum\MPp{#1}=\realpageno\relax
     % \doifdefined{\MPoverlayposprefix#1}
     %   {\getvalue{\MPoverlayposprefix#1}{#1}{#2}{#3}}%
     \executeifdefined{\MPoverlayposprefix#1}%
       \gobblethreearguments{#1}{#2}{#3}%
   \fi}

\appendtoks
  \let\dohandlepositionboxes\doinsertpositionboxes % was handle ?
\to \everyinsertpositionaction

\def\docleanpositionboxes#1#2#3% pos tag setups
  {\ifnum\MPp{#1}<\realpageno \else
     \noexpand \dohandlepositionboxes{#1}{#2}{#3}% reinsert
   \fi}

\appendtoks
  \let\dohandlepositionboxes\docleanpositionboxes
\to \everycleanpositionaction

%D A position graphic is a normal (non||reused) \METAPOST\
%D graphic, used immediately, with zero dimensions, so that a
%D sequence of them does not harm.

\newbox\positiongraphicbox

\def\startMPpositiongraphic % id setups
  {\dodoublegroupempty\dostartMPpositiongraphic}

\long\def\dostartMPpositiongraphic#1#2#3\stopMPpositiongraphic
  {\long\setgvalue{MPG:#1}% tag list mpcode
     {\useMPpositiongraphic{#1}{#2}{#3}}}

\let\stopMPpositiongraphic\relax

\def\prepareMPpositionvariables
  {\ifundefined{\@@meta self}\setvalue{\@@meta self}{\currentposition}\fi
   \ifundefined{\@@meta from}\setvalue{\@@meta from}{\currentposition}\fi}

\newif\ifcollectMPpositiongraphics \collectMPpositiongraphicstrue

\long\def\useMPpositiongraphic#1#2#3%
  {\bgroup
   \prepareMPvariables{#2}%
   \prepareMPpositionvariables
   \enableincludeMPgraphics
   \ifcollectMPpositiongraphics
     \expanded{\startMPdrawing#3\noexpand\stopMPdrawing}%
     \global\MPdrawingdonetrue
   \else
     \startMPgraphic#3\stopMPgraphic
     \loadMPgraphic{\MPgraphicfile.\the\currentMPgraphic}{}%
     \deallocateMPslot\currentMPgraphic
     \placeMPgraphic
   \fi
   \egroup}

% Now we need a adapted action handler:

\def\dopositionaction#1% test saves hash entry in etex
  {\ifundefined{\POSactionprefix#1::}\else
     \ifnum\MPp{#1}>\zerocount % new
       \bgroup
       \setbox\scratchbox\hbox
       \bgroup
       \traceposstring\clap\red{<#1>}%
       \the\everyinsertpositionaction
       \the\everypositionaction
       \ifcollectMPpositiongraphics
         % can save a lot of run time
         \pushMPdrawing
         \MPshiftdrawingtrue
         \resetMPdrawing
         \getvalue{\POSactionprefix#1::}%
         \ifMPdrawingdone
           \getMPdrawing
         \fi
         \resetMPdrawing
         \popMPdrawing
       \else
         \getvalue{\POSactionprefix#1::}%
       \fi
       \cleanuppositionaction{#1}%
       \egroup                % smashed is really needed else
       \smashedbox\scratchbox % we get problems with too big
       \egroup                % overlays (s-pre-0x.tex)
     \else
       % shouldn't happen too often
       \traceposstring\clap\cyan{<#1>}%
     \fi
   \fi}

\def\MPpositiongraphic
  {\dodoublegroupempty\doMPpositiongraphic}

\def\doMPpositiongraphic#1#2% tag setups
  {\bgroup
   \def\@@meta{#1:}%
   \setupMPvariables[#2]%
   \prepareMPpositionvariables
   \MPshiftdrawingtrue
   \def\doMPpositiongraphic##1##2%
     {{% new, see (techniek)
       \def\@@meta{##1:}%
       \setupMPvariables[#2,##2]%
       \prepareMPpositionvariables
       % and needed
       \getvalue{MPG:##1}}}% temp hack
   \setbox\positiongraphicbox\hbox
     {\ignorespaces
      \executeifdefined{MPM:#1}{\executeifdefined{MPG:#1}\donothing}%
      \removelastspace}%
   \smashbox\positiongraphicbox
   \box\positiongraphicbox
   \egroup}

\long\def\startMPpositionmethod#1#2\stopMPpositionmethod
  {\long\setgvalue{MPM:#1}{#2}} % todo: var list here

\let\stopMPpositionmethod\relax

%D Simple one position graphics.

\def\setMPpositiongraphic
  {\dotriplegroupempty\dosetMPpositiongraphic}

\def\dosetMPpositiongraphic#1#2#3% pos tag vars
  {\ifx\currentpositionoverlay\empty
     \dosetpositionaction{#1}{\MPpositiongraphic{#2}{#3}}%
   \else % silly can be one
     \handlepositiongraphics{#1}{#2}{#3}%
   \fi}

\def\handlepositiongraphics#1#2#3% combine with boxes
  {\handlepositionaction\dohandleMPpositiongraphic\with{#1}{#2}{#3}\on{#2}}

\def\doinsertMPpositiongraphic#1#2#3% pos tag setups
  {\ifnum\MPp{#1}=\realpageno\relax % extra saveguard
     \def\currentposition{#1}\MPpositiongraphic{#2}{#3}%
   \fi}

\appendtoks
  \let\dohandleMPpositiongraphic\doinsertMPpositiongraphic
\to \everyinsertpositionaction

\def\docleanMPpositiongraphic#1#2#3% pos tag setups
  {\ifnum\MPp{#1}<\realpageno \else
     \noexpand \dohandleMPpositiongraphic{#1}{#2}{#3}%
   \fi}

\appendtoks
  \let\dohandleMPpositiongraphic\docleanMPpositiongraphic
\to \everycleanpositionaction

%D Graphics that span two positions.

\def\setMPpositiongraphicrange
  {\doquadruplegroupempty\dosetMPpositiongraphicrange}

\def\dosetMPpositiongraphicrange#1#2#3#4% bpos epos tag vars
  {\ifx\currentpositionoverlay\empty
     \dosetpositionaction{#1}{\MPpositiongraphic{#3}{#4}}%
   \else
     \handlepositiongraphicsrange{#1}{#2}{#3}{#4}%
   \fi}

\def\handlepositiongraphicsrange#1#2#3#4%
  {\handlepositionaction\dohandleMPpositiongraphicrange\with{#1}{#2}{#3}{#4}\on{#2}}

\def\doinsertMPpositiongraphicrange#1#2#3#4% pos pos tag setups
  {\ifnum\MPp{#1}\MPp{#2}>\zerocount
     \iflocalpositioning
       \donetrue
     \else
       \donefalse
       \ifnum\MPp{#1}=\realpageno
         \donetrue
       \else\ifnum\MPp{#2}=\realpageno
         \donetrue
       \else\ifnum\MPp{#1}<\realpageno\relax\ifnum\MPp{#2}>\realpageno
         \donetrue
       \fi\fi\fi\fi
     \fi
     \ifdone
       \def\currentposition{#1}\MPpositiongraphic{#3}{#4}%
     \fi
   \fi}

\appendtoks
  \let\dohandleMPpositiongraphicrange\doinsertMPpositiongraphicrange
\to \everyinsertpositionaction

\def\docleanMPpositiongraphicrange#1#2#3#4% pos tag setups
  {\ifnum\MPp{#2}<\realpageno \else
     \noexpand \dohandleMPpositiongraphicrange{#1}{#2}{#3}{#4}%
   \fi}

\appendtoks
  \let\dohandleMPpositiongraphicrange\docleanMPpositiongraphicrange
\to \everycleanpositionaction

% will be overloaded, and/or code below moved to core-box

\defineoverlay[\v!text-2][\positionoverlay{\v!text-2}]
\defineoverlay[\v!text-1][\positionoverlay{\v!text-1}]
\defineoverlay[\v!text+1][\positionoverlay{\v!text+1}]
\defineoverlay[\v!text+2][\positionoverlay{\v!text+2}]

%D The auxiliary \METAPOST\ macros are defined by default,
%D by saying:

\startMPextensions
  if unknown context_core : input mp-core.mp ; fi ;
\stopMPextensions

%D Some of these macros are pretty clever but too complicated
%D to be nice. When things are kind of stable I'll clean up
%D this mess.

%D THIS NEEDS A CLEANUP

\setupMPvariables
  [mpos:box]
  [linecolor=blue,
   linewidth=\linewidth,
   fillcolor=lightgray,
   filloffset=\!!zeropoint]

\startMPpositiongraphic{mpos:box}{fillcolor,linecolor,linewidth}
  initialize_box(\MPpos{\MPvar{self}}) ;
  boxfillcolor  := \MPvar{fillcolor} ;
  boxlinecolor  := \MPvar{linecolor} ;
  boxlinewidth  := \MPvar{linewidth} ;
  boxfilloffset := \MPvar{filloffset} ;
  draw_box ;
  anchor_box(\MPanchor{\MPvar{self}}) ;
\stopMPpositiongraphic

\setupMPvariables
  [mpos:area]
  [linecolor=blue,
   linewidth=\linewidth,
   fillcolor=lightgray,
   filloffset=\!!zeropoint]

\startMPpositiongraphic{mpos:area}{fillcolor,linecolor,linewidth}
  initialize_area(\MPpos{b:\MPvar{self}},\MPpos{e:\MPvar{self}}) ;
  boxfillcolor  := \MPvar{fillcolor} ;
  boxlinecolor  := \MPvar{linecolor} ;
  boxlinewidth  := \MPvar{linewidth} ;
  boxfilloffset := \MPvar{filloffset} ;
  draw_area ;
  anchor_area(\MPanchor{b:\MPvar{self}}) ;
\stopMPpositiongraphic

%D This is already cleaned up.

% gridtype = 1 => baseline
% gridtype = 2 => betweenline

\setupMPvariables
  [mpos:par]
  [mp=mpos:par:shape,
   gridtype=0,
   linetype=1,
   filltype=1,
   dashtype=0,
  %snaptops=true, % not that nice: true/false
   gridcolor=red,
   linecolor=blue,
   fillcolor=lightgray,
   filloffset=\!!zeropoint,
   linewidth=\linewidth,
   gridwidth=\linewidth,
   gridshift=\!!zeropoint,
   lineradius=.5\bodyfontsize,
   dashtype=1]

\startuseMPgraphic{mpos:par:shape}
  \iftracepositions show_par \else draw_par \fi ;
\stopuseMPgraphic

\startuseMPgraphic{mpos:par:setup}
   boxgridtype   := \MPvar{gridtype} ;
   boxlinetype   := \MPvar{linetype} ;
   boxfilltype   := \MPvar{filltype} ;
   boxdashtype   := \MPvar{dashtype} ;
   boxgridcolor  := \MPvar{gridcolor} ;
   boxlinecolor  := \MPvar{linecolor} ;
   boxfillcolor  := \MPvar{fillcolor} ;
   boxfilloffset := \MPvar{filloffset} ;
   boxlinewidth  := \MPvar{linewidth} ;
   boxgridwidth  := \MPvar{gridwidth} ;
   boxgridshift  := \MPvar{gridshift} ;
   boxlineradius := \MPvar{lineradius} ;
  %snap_multi_par_tops := \MPvar{snaptops} ;
\stopuseMPgraphic

\startuseMPgraphic{mpos:par:extra}
  % user stuff, like:
  % snap_multi_par_tops := false ;
\stopuseMPgraphic

\ifx\MPparcounter\undefined \newcounter\MPparcounter \fi

\def\MPself     {\MPvar{self}}
\def\MPbself    {b:\MPself}
\def\MPeself    {e:\MPself}
\def\MPwself    {w:\MPself}
\def\MPparanchor{p:\MPparcounter}

\def\MPl#1{\MPplus{#1}20}
\def\MPr#1{\MPplus{#1}30}

\startMPpositionmethod{mpos:par} %%%%%%%%%%% will become obsolete
  \edef\MPparcounter{\MPv\MPbself{1}{0}}%
  \doifpositionelse\MPwself
    {\startMPpositiongraphic{mpos:par}%
         {fillcolor,filloffset,linecolor,gridcolor,linewidth,gridwidth,gridshift,lineradius}
       initialize_area_par(\MPpos\MPbself,
                           \MPpos\MPeself,
                           \MPpos\MPwself) ;
       \includeMPgraphic{mpos:par:setup} ;
       \includeMPgraphic{mpos:par:extra} ;
       \includeMPgraphic{\MPvar{mp}} ;
       anchor_par(\MPanchor\MPbself) ;
     \stopMPpositiongraphic}
    {\startMPpositiongraphic{mpos:par}%
         {fillcolor,filloffset,linecolor,gridcolor,linewidth,gridwidth,gridshift,lineradius}
       initialize_par(\MPpos\MPbself,
                      \MPpos\MPeself,
                      \MPpos\textanchor,
                      \MPpos\MPparanchor,
                      \MPvv \MPparanchor{0pt,0pt,0pt,0pt,0,0pt}) ;
       \includeMPgraphic{mpos:par:setup} ;
       \includeMPgraphic{mpos:par:extra} ;
       \includeMPgraphic{\MPvar{mp}} ;
       anchor_par(\MPanchor\MPbself) ;
     \stopMPpositiongraphic}%
  \MPpositiongraphic{mpos:par}{}%
\stopMPpositionmethod

%D The next alternative works in columnsets :

%  \iftracepositions show\else draw\fi_multi_pars ;

\startuseMPgraphic{mpos:par:columnset}
  \iftracepositions show_multi_pars \else draw_multi_pars \fi ;
\stopuseMPgraphic

\startMPpositionmethod{mpos:par:columnset}
  \edef\MPparcounter{\MPv\MPbself{1}{0}}%
  \startMPpositiongraphic{mpos:par}%
    {fillcolor,filloffset,linecolor,gridcolor,%
     linewidth,gridwidth,lineradius}
    \includeMPgraphic{mpos:par:setup} ;
    \includeMPgraphic{mpos:par:extra} ;
    prepare_multi_pars(\MPpos\MPbself,\MPpos\MPeself,\MPpos\MPwself,
      \MPpos\MPparanchor,\MPvv\MPparanchor{0pt,0pt,0pt,0pt,0,0pt}) ;
    relocate_multipars(-\MPxy\MPanchorid) ; % inside layerpos
    \includeMPgraphic{\MPvar{mp}} ;
  \stopMPpositiongraphic
  \MPpositiongraphic{mpos:par}{}%
\stopMPpositionmethod

%D \starttyping
%D \setupbackground
%D   [test]
%D   [mp=mpos:par:columnset,
%D    method=mpos:par:columnset]
%D \stoptyping

%D We need to treat floats in a special way.

\startMPinitializations
  local_multi_par_area:=\iflocalpositioning true\else false\fi;
\stopMPinitializations

\def\textbackgroundoverlay#1%
  {\iflocalpositioning\v!local\else\v!text\fi#1}

\newcounter\localpositionnumber

\def\MPanchornumber
  {\iflocalpositioning\localpositionnumber\else\realfolio\fi}

%D So far for the trickery.

\newcounter\textbackgrounddepth

\appendtoks
  \expanded{\savecurrentvalue\noexpand\totalnofparbackgrounds{\number\nofparbackgrounds}}%
\to \everybye

\appendtoks
  \initializeparbackgrounds
\to \everystarttext

\ifx\totalnofparbackgrounds\undefined \newcounter\totalnofparbackgrounds \fi
\ifx\nofparbackgrounds     \undefined \newcount  \nofparbackgrounds      \fi

\def\initializeparbackgrounds
  {\ifcase\totalnofparbackgrounds\else
     \global\positioningtrue
     \global\positioningpartrue
   \fi}

\unexpanded\def\starttextbackground
  {\bgroup
   \increment\textbackgrounddepth
   \dodoubleempty\dostarttextbackground}

\let\dodostarttextbackground\relax
\let\dodostoptextbackground \relax

\def\currentparbackground{pbg:0}
\def\nextparbackground   {pbg:1}

\def\btbanchor{b:\currentparbackground}
\def\etbanchor{e:\currentparbackground}

\def\nextbtbanchor{b:\nextparbackground}
\def\nextetbanchor{e:\nextparbackground}

\def\textbackgroundparameter#1%
  {\csname\??td\currenttextbackground#1\endcsname}

\let\backgroundvariable\textbackgroundparameter % will become obsolete

% \definetextbackground[more][state=start,backgroundcolor=red] % location=paragraph
% \definetextbackground[test][state=start,backgroundcolor=green]
%
% \page \placefigure[left]{}{}
%
% \starttextbackground[test]
%   \readfile{ward}{}{}
%   \starttextbackground[more]
%     \readfile{ward}{}{}
%   \stoptextbackground
%   \readfile{ward}{}{}
% \stoptextbackground
%
% \page \placefigure[right]{}{}
%
% \starttextbackground[test]
%   \readfile{ward}{}{}
%   \starttextbackground[more]
%     \readfile{ward}{}{}
%   \stoptextbackground
%   \readfile{ward}{}{}
% \stoptextbackground

\def\dostarttextbackground[#1][#2]%
  {\checktextbackgrounds
   \def\currenttextbackground{#1}%
   \global\advance\nofparbackgrounds\plusone
   \edef\currentparbackground{pbg:\number\nofparbackgrounds}%
%    \bgroup
%      \advance\nofparbackgrounds\plusone
%      \xdef\nextparbackground{pbg:\number\nofparbackgrounds}%
%    \egroup
   \xdef\nextparbackground{pbg:\number\numexpr\nofparbackgrounds+\plusone\relax}% still xdef ?
   % todo : \synchonizepositionpage{b:\currentparbackground}{s:\currentparbackground}%
   \setuptextbackground[#1][#2]%
   \let\dodostarttextbackground\relax
   \let\dodostoptextbackground \relax
   \doif{\textbackgroundparameter\c!state}\v!start{\dopresettextbackground{#1}}%
   \dodostarttextbackground}

% todo \backgroundvariable\c!variant

\def\dopresettextbackground#1% todo: \backgroundparameter
  {\ExpandFirstAfter\processaction % \EFA niet echt nodig
     [\textbackgroundparameter\c!location]
     [   \v!text=>\let\dodostarttextbackground\dostarttextbackgroundtxt
                  \let\dodostoptextbackground \dostoptextbackgroundtxt,
    \v!paragraph=>\let\dodostarttextbackground\dostarttextbackgroundpar
                  \let\dodostoptextbackground \dostoptextbackgroundpar,
         \v!none=>\let\dodostarttextbackground\relax
                  \let\dodostoptextbackground \relax]%
   \ifx\dodostarttextbackground\dostarttextbackgroundpar % untested
     \ifnum\textbackgrounddepth>\plusone % new
       \let\dodostarttextbackground\dostarttextbackgroundtxt
       \let\dodostoptextbackground \dostoptextbackgroundtxt
     \fi
   \fi
   \doifelse{\textbackgroundparameter\c!frame}\v!on
     {\doifelse{\textbackgroundparameter\c!corner}\v!round
        {\setvalue{\??td#1\c!frame}{2}}
        {\setvalue{\??td#1\c!frame}{1}}}
     {\setvalue{\??td#1\c!frame}{0}}%
   \doifelse{\textbackgroundparameter\c!background}\v!color
     {\setvalue{\??td#1\c!background}{1}}
     {\setvalue{\??td#1\c!background}{0}}%
  %\startpositionoverlay{\v!text\getvalue{\??td#1\c!level}}%
   \startpositionoverlay{\textbackgroundoverlay{\textbackgroundparameter\c!level}}%
     \expanded
       {\setMPpositiongraphicrange % moet veel efficienter
          {\btbanchor}% {b:\currentparbackground}%
          {\etbanchor}% {e:\currentparbackground}%
          {\textbackgroundparameter\c!method}%
          {self=\currentparbackground,
           mp=\textbackgroundparameter\c!mp,
           gridtype=\textbackgroundparameter\c!alternative,
           filltype=\textbackgroundparameter\c!background,
           linetype=\textbackgroundparameter\c!frame,
           dashtype=\textbackgroundparameter{dash}, %  to be internationalized
           gridcolor=\textbackgroundparameter\c!framecolor,
           linecolor=\textbackgroundparameter\c!framecolor,
           fillcolor=\textbackgroundparameter\c!backgroundcolor,
           filloffset=\textbackgroundparameter\c!backgroundoffset,
           gridwidth=\textbackgroundparameter\c!rulethickness,
           gridshift=\textbackgroundparameter\c!voffset,
           linewidth=\textbackgroundparameter\c!rulethickness,
           lineradius=\textbackgroundparameter\c!radius}}%
   \stoppositionoverlay}

\def\stoptextbackground
  {\dodostoptextbackground
   \carryoverpar\egroup}

\def\starttextbackgroundmanual
  {\dostartattributes{\??td\currenttextbackground}\c!style\c!color\empty
   \fpos\currentparbackground\ignorespaces}

\def\stoptextbackgroundmanual
  {\tpos\currentparbackground
   \dostopattributes}

\def\dostarttextbackgroundtxt
  {\ifvmode \dontleavehmode \fi % was leavevmode, brrr
   \dostartattributes{\??td\currenttextbackground}\c!style\c!color\empty
   \fpos\currentparbackground\ignorespaces}

\def\dostoptextbackgroundtxt
  {\tpos\currentparbackground
   \dostopattributes}

% keep this simple one, it's used in prikkels and alike
%
% \def\dostarttextbackgroundpar
%   {\endgraf % new
%    \getvalue{\??td\currenttextbackground\c!before}%
%    \noindent\fpos\currentparbackground\ignorespaces
%    \bgroup
%    \nobreak \vskip-\lineheight \nobreak
%    \doassignsomeskip\getvalue{\??td\currenttextbackground\c!topoffset}\to\scratchskip
%    \kern\scratchskip\nobreak
%    \dosetleftskipadaption{\getvalue{\??td\currenttextbackground\c!leftoffset}}%
%    \advance\leftskip\leftskipadaption
%    \dosetleftskipadaption{\getvalue{\??td\currenttextbackground\c!rightoffset}}%
%    \advance\rightskip\leftskipadaption
%    \dostartattributes{\??td\currenttextbackground}\c!style\c!color{}%
%    \nowhitespace
%    \seteffectivehsize
%    \par}
%
% \def\dostoptextbackgroundpar
%   {\par
%    \dostopattributes
%    \doassignsomeskip\getvalue{\??td\currenttextbackground\c!bottomoffset}\to\scratchskip
%    \kern\scratchskip\nobreak
%    \nobreak \vskip-\lineheight \nobreak
%    \nowhitespace
%    \egroup
%    \nobreak \noindent \strut \hfill \kern\zeropoint \tpos\currentparbackground
%    \endgraf % new
%    \getvalue{\??td\currenttextbackground\c!after}}

\def\dostarttextbackgroundpar
  {\endgraf % new
   \textbackgroundparameter\c!before
   \doassignsomeskip\textbackgroundparameter\c!topoffset\to\scratchskip
   \xdef\textbackgroundskip{\the\scratchskip}%
   \noindent
   \ifgridsnapping
     \ifdim\textbackgroundskip>\zeropoint
       \struttedbox{\hbox{\raise\textbackgroundskip\hbox{\fpos\currentparbackground}}}%
     \else
      \fpos\currentparbackground
     \fi
   \else
     \fpos\currentparbackground
   \fi
   \bgroup
   \endgraf % we need a vertical nobreak - 29/06/2004
   \nobreak \vskip-\lineheight \nobreak
   \doassignsomeskip\textbackgroundparameter\c!topoffset\to\scratchskip
   \ifgridsnapping \else \ifdim\textbackgroundskip>\zeropoint
     \kern\textbackgroundskip\nobreak
   \fi \fi
   \dosetleftskipadaption{\textbackgroundparameter\c!leftoffset}%
   \advance\leftskip\leftskipadaption
   \dosetleftskipadaption{\textbackgroundparameter\c!rightoffset}%
   \advance\rightskip\leftskipadaption
   % new
   \dosetraggedcommand{\textbackgroundparameter\c!align}%
   \raggedcommand
   %
   \dostartattributes{\??td\currenttextbackground}\c!style\c!color\empty
   \nowhitespace
\nobreak % new per 23/04/2006 (else potential break when whitespace)
   \seteffectivehsize
   \doinhibitblank %   \blank[\v!disable]% new
   \par}

\def\dostoptextbackgroundpar
  {\par
   \removelastskip % new
   \dostopattributes
   \doassignsomeskip\textbackgroundparameter\c!bottomoffset\to\scratchskip
   \ifdim\lastskip>\zeropoint
     \advance\scratchskip-\lastskip
   \fi
   \xdef\textbackgroundskip{\the\scratchskip}%
   \ifgridsnapping \else \ifdim\textbackgroundskip>\zeropoint
     \kern\scratchskip\nobreak
   \fi \fi
   \nobreak \vskip-\lineheight \nobreak
   \nowhitespace
   \egroup
\bgroup \forgeteverypar % NOT REALLY NEEDED, SAVES HASH/MEM
   \nobreak \noindent \strut \hfill \kern\zeropoint
   \ifgridsnapping % experimental, pascal (todo: topoffset in same way)
     \ifdim\textbackgroundskip>\zeropoint
       \struttedbox{\hbox{\lower\textbackgroundskip\hbox{\tpos\currentparbackground}}}%
     \else
       \tpos\currentparbackground
     \fi
   \else
     \tpos\currentparbackground
   \fi
\egroup
   \endgraf % new
   \textbackgroundparameter\c!after}

\let\textparpages \!!zeropoint
\let\textparheight\!!zeropoint
\let\textparwidth \!!zeropoint

\def\calculatetextpardimensions
  {\docalculatetextpardimensions\btbanchor    \etbanchor    \MPparanchor}

\def\calculatenexttextpardimensions
  {\docalculatetextpardimensions\nextbtbanchor\nextetbanchor\relax}

\def\docalculatetextpardimensions#1#2#3%
  {\scratchcounter\MPp#2%\etbanchor
   \advance\scratchcounter-\MPp#1%\btanchor
   \edef\textparpages{\the\scratchcounter}%
   \ifcase\scratchcounter
     % one page
     \scratchdimen         \MPy#1%\btanchor
     \advance\scratchdimen-\MPy#2%\etbanchor
   \else
     % two or more pages
     \scratchdimen         \MPy#1%\btanchor
     \advance\scratchdimen-\MPy#2%\etbanchor
     \advance\scratchdimen-\MPy\textanchor
     \advance\scratchdimen \MPy\textanchor
     \advance\scratchdimen \MPh\textanchor\relax
     \ifcase\scratchcounter>2 \ifnum\scratchcounter<5
       % more pages
       \scratchdimen\textheight
       \advance\scratchcounter \minusone
       \multiply\scratchdimen \scratchcounter
     \else
       % keep'm small
       \scratchdimen5\textheight
     \fi \fi
   \fi
   \edef\textparheight{\the\scratchdimen}%
   \ifcase\scratchcounter
     % one page
     \scratchdimen         \MPx#2%\etbanchor
     \advance\scratchdimen-\MPx#1%\btanchor
   \else
     % two or more pages / maybe also hang
     \ifx#3\relax
       \scratchdimen\makeupwidth % \textwidth
     \else
       \scratchdimen\MPw\MPparanchor
       \advance\scratchdimen-\MPl\MPparanchor
       \advance\scratchdimen-\MPr\MPparanchor
     \fi
   \fi
   \edef\textparwidth{\the\scratchdimen}}

\def\mintextparheight{4\lineheight}

\def\dontsplitnexttextbackground % dangerous but useful
  {\ifdim\pagetotal>\textheight \else
     \ifdim\pagegoal=\maxdimen \else
      \calculatenexttextpardimensions
      % too tricky
      % \scratchdimen=\textparheight
      % \advance\scratchdimen\pagetotal\relax
      % \ifdim\scratchdimen>\pagegoal
      % \page
      % \fi
      \ifdim\textparheight>\zeropoint
        \ifdim\textparheight>\mintextparheight\else
          \page % option
        \fi
      \fi
    \fi
   \fi}

\def\definetextbackground
  {\dodoubleempty\dodefinetextbackground}

\def\dodefinetextbackground[#1][#2]%
  {\ifsecondargument % why ?
     \copyparameters[\??td#1][\??td]
       [\c!state,\c!location,\c!alternative,\c!mp,\c!method,
        \c!background,\c!backgroundcolor,\c!corner,\c!level,
        \c!backgroundoffset,\c!before,\c!after,\c!align,dash, % dash not yet internationalized
        \c!radius,\c!frame,\c!framecolor,\c!rulethickness,\c!voffset,
        \c!leftoffset,\c!rightoffset,\c!topoffset,\c!bottomoffset]%
     \getparameters[\??td#1][#2]%
     \doifvalue{\??td#1\c!state}\v!start\checktextbackgrounds
     \unexpanded\setvalue{#1}%
       {\groupedcommand{\starttextbackground[#1]}{\stoptextbackground}}%
     \setvalue{\e!start#1}{\starttextbackground[#1]}%
     \setvalue{\e!stop #1}{\stoptextbackground}%
   \fi}

\def\setuptextbackground
  {\dodoubleargument\dosetuptextbackground}

\def\dosetuptextbackground[#1][#2]%
  {\ifsecondargument
     \doifelsenothing{#1}
       {\dodosetuptextbackground{#2}\empty}
       {\processcommalist[#1]{\dodosetuptextbackground{#2}}}%
   \else
     \dodosetuptextbackground{#1}\empty
   \fi}

\def\dodosetuptextbackground#1#2%
  {\getparameters[\??td#2][#1]%
   \def\currenttextbackground{#2}%
   \doifvalue{\??td#2\c!state}\v!start\checktextbackgrounds}

\let\currenttextbackground\empty

\def\checktextbackgrounds
  {\ifproductionrun
     \enabletextarearegistration
     \enablehiddenbackground
   \fi}

\setuptextbackground
  [\c!mp=mpos:par:columnset,      % buggy: mpos:par:shape
   \c!method=mpos:par:columnset, %
   \c!state=\v!start,
   \c!location=\v!text,
   \c!leftoffset=\!!zeropoint, % 1em,
   \c!rightoffset=\textbackgroundparameter\c!leftoffset,
   \c!topoffset=\!!zeropoint,  % \v!medium,
   \c!bottomoffset=\textbackgroundparameter\c!topoffset,
   \c!level=-1,
   \c!alternative=0,
   \c!align=,
   dash=0, % to be internationalized
   \c!background=\v!color,
   \c!backgroundcolor=lightgray,
   \c!backgroundoffset=\!!zeropoint,
   \c!corner=\v!rectangular,
   \c!radius=.5\bodyfontsize,
   \c!voffset=\!!zeropoint,
   \c!frame=\v!on,
   \c!framecolor=blue,
   \c!rulethickness=\linewidth]

%D As an example we define a grid background:

\definetextbackground
  [\v!grid]
  [\c!state=\v!stop,
   \c!location=\v!paragraph,
   \c!frame=\v!off,
   \c!framecolor=red,
   \c!background=,
   \c!alternative=1]

\ifx\basegrid\undefined \else \letvalue\v!grid=\basegrid \fi

% lelijk, aanpassen, opties

\setupMPvariables
  [mpos:connect]
  [linecolor=red,
   linewidth=1pt]

\setupMPvariables
  [mpos:encircle]
  [fillcolor=lightgray,
   filloffset=\!!zeropoint,
   linecolor=blue,
   linewidth=1pt]

\startuseMPgraphic{mpos:common:ec}
  path pa ; pair ca ; color lc ; numeric lw ;
  lw := \MPvar{linewidth} ;
  lc := \MPvar{linecolor} ;
  initialize_box(\MPpos{\MPvar{self}}) ;
  pa := pxy ; ca := cxy ; pa := boundingbox pa enlarged 2lw ;
  pa := llcorner pa...lrcorner pa...urcorner pa...ulcorner pa...cycle ;
  drawoptions (withpen pencircle scaled lw withcolor lc) ;
\stopuseMPgraphic

\startMPpositiongraphic{mpos:encircle}{linecolor,fillcolor,linewidth}
  \includeMPgraphic{mpos:common:ec}
  fill pa withcolor \MPvar{fillcolor} ; draw pa ;
  anchor_box(\MPanchor{\MPvar{self}}) ;
\stopMPpositiongraphic

\startMPpositiongraphic{mpos:connect}{linecolor,linewidth}
  path pb, pc ; pair cb, cc ;
  \includeMPgraphic{mpos:common:ec}
  initialize_box(\MPpos{\MPvar{to}}) ;
  pb := pxy ; cb := cxy ; pb := boundingbox pb enlarged 2lw ;
  pb := llcorner pb...lrcorner pb...urcorner pb...ulcorner pb...cycle ;
  pc := ca {up} .. {down} cb ;
  cc := (pc intersection_point pa) ;
  if intersection_found :
    pc := pc cutbefore cc ;
    cc := (pc intersection_point pb) ;
    if intersection_found :
      pc := pc cutafter cc ;
      drawarrow pc ; drawarrow reverse pc ;
    fi ;
  fi ;
  anchor_box(\MPanchor{\MPvar{self}}) ;
\stopMPpositiongraphic

%D \macros
%D   {stackposdown, stackposup, stackposleft,stackposright}
%D
%D A non graphic example of the use of positioning, is to stack
%D text in for instance the margin.
%D
%D \stackposdown \inleft {some text}The text \type {some text}
%D goes into the left margin, and \stackposdown \inleft {some
%D more}\type {some more} as well. When they overlap, they
%D will not touch.
%D
%D Here we said \type {\stackposdown \inleft{some text}}. Instead
%D of \stackposleft \inleft {one}stacking \stackposleft \inleft
%D {two}vertically, one can stack horizontally by \stackposleft
%D \inleft {three}using \type {\stackposleft}.
%D
%D We can go in all four directions, using \type {\stackposdown},
%D \type {\stackposup}, \type {\stackposleft} and \type
%D {\stackposright}.

\def\stackposdistance{.5em}

\newcounter\currentautopos

\def\POSstackprefix{stack:}

\def\dostackposbox#1#2%
  {\dowithnextbox
     {#2{\let\previousautopos\currentautopos
         \doglobal\increment\currentautopos
         \hpos{\POSstackprefix\currentautopos}
           {\doifoverlappingelse{\POSstackprefix\currentautopos}{\POSstackprefix\previousautopos}
              {#1}
              {\flushnextbox}}}}%
     \hbox}

\def\stackposup   {\dostackposbox{\raise\lineheight\flushnextbox}}
\def\stackposdown {\dostackposbox{\lower\lineheight\flushnextbox}}
\def\stackposleft {\dostackposbox{\copy\nextbox\hskip\nextboxwd\hskip\stackposdistance}}
\def\stackposright{\dostackposbox{\hskip\stackposdistance\hskip\nextboxwd\flushnextbox}}

%D \macros
%D   {stackeddown}
%D
%D However, a better implementation is possible with the
%D following macro. We now have an extra key \type {stack} for
%D margin settings. When set to \type {yes}, this macro comes
%D into action.

% \def\stackeddown
%   {\dowithnextbox
%      {\doglobal\increment\currentautopos
%       \hpos{\POSstackprefix\currentautopos}
%         {\edef\next
%            {\nextboxht\the\nextboxht
%             \nextboxdp\the\nextboxdp
%             \nextboxwd\the\nextboxwd}%
%          \let\previousautopos\currentautopos
%          \scratchdimen\zeropoint
%          \doloop
%            {\decrement\previousautopos
%             \doifoverlappingelse{\POSstackprefix\currentautopos}{\POSstackprefix\previousautopos}
%               {\advance\scratchdimen\MPh{\POSstackprefix\previousautopos}%
%                \advance\scratchdimen\MPd{\POSstackprefix\previousautopos}}%
%               \exitloop
%             \ifnum\previousautopos<\zerocount\exitloop\fi}%
%         % \ifdim\scratchdimen>\zeropoint
%            \setbox\nextbox\iftracepositions\@EA\ruledhbox\else\@EA\hbox\fi
%              {\lower\scratchdimen\flushnextbox}%
%            \next
%         % \fi
%          \flushnextbox}}}

% Because there can be many stacked items in a line and successive lines, we
% play dirty and adapt the position and height of the current node so that
% this becomes visible to a next pass.
%
% \startbuffer
% \inleft {test 1} test 1 \inleft {test 2} test 2 \endgraf
% \inleft {test 3} test 3
% \stopbuffer
% \getbuffer \typebuffer \flushstatus \page
%
% \startbuffer
% \inleft {test 1} test 1 \inleft {test 2} test 2 \inleft {test 3} test 3 \endgraf
% \inleft {test 4} test 4
% \stopbuffer
% \getbuffer \typebuffer \flushstatus \page
%
% \startbuffer
% \inleft {test 1} test 1 \endgraf
% \inleft {test 2} test 2 \endgraf
% \inleft {test 3} test 3
% \stopbuffer
% \getbuffer \typebuffer \flushstatus \page
%
% \startbuffer
% \inleft {test 1\\test 1} test 1 \inleft {test 2} test 2 \endgraf
% \inleft {test 3} test 3
% \stopbuffer
% \getbuffer \typebuffer \flushstatus \page
%
% \startbuffer
% \inleft {test 1\\test 1\\test 1\\test 1\\test 1} test 1 \endgraf
% test 2 \endgraf
% \inleft {test 3} test 3
% \stopbuffer
% \getbuffer \typebuffer \flushstatus \page
%
% \startbuffer
% \inleft{test 1} test \inleft{test 2} test \inleft{test 3\\test 3} test
% \stopbuffer
% \getbuffer \typebuffer \flushstatus \page
%
% \startbuffer
% \inleft{test 1\\test 1\\test 1} test \inleft{test 2\\test 2} test \inleft{test 3\\test 3\\test 3} test \endgraf
% \inleft{test 1\\test 1\\test 1} test \inleft{test 2\\test 2} test \inleft{test 3\\test 3\\test 3} test
% \stopbuffer
% \getbuffer \typebuffer \flushstatus \page

\newdimen\laststackvmove

\def\stackeddown
  {\bgroup
   % this macro assumes a few things and is meant to work for margin notes
   \dowithnextbox
     {\doglobal\increment\currentautopos
      \global\laststackvmove\zeropoint
      \hpos{\POSstackprefix\currentautopos}
        {\edef\next
           {\nextboxht\the\nextboxht
            \nextboxdp\the\nextboxdp
            \nextboxwd\the\nextboxwd}%
         \let\previousautopos\currentautopos
         \scratchdimen\zeropoint
         \scratchcounter\zerocount
         \doloop
           {\decrement\previousautopos\relax
            \edef\currentposition {\POSstackprefix\currentautopos}%
            \edef\previousposition{\POSstackprefix\previousautopos}%
            \ifnum\MPp\currentposition=\MPp\previousposition\relax
              %\registerstatus{doing \currentautopos/\previousautopos}%
              \doifoverlappingelse\currentposition\previousposition
                {\scratchskip         \MPy\currentposition
                 \advance\scratchskip-\MPy\previousposition
                 \advance\scratchskip-\MPd\currentposition  % untested
                 \advance\scratchskip \MPd\previousposition % untested
                 \advance\scratchskip \MPh\currentposition
                 % todo: also take depth into account
                 \relax
                 \ifdim\scratchskip<\scratchdimen
                   %\registerstatus{no \the\scratchskip}%
                 \else
                   %\registerstatus{yes \the\scratchskip}%
                   \scratchdimen\scratchskip
                 \fi}%
                \donothing % {\registerstatus{next}}%
              \ifnum\previousautopos<\zerocount\exitloop\fi
            \else
              \exitloop
            \fi}%
         \ifdim\scratchdimen=\zeropoint \else
           \bgroup
           \edef\currentposition{\POSstackprefix\currentautopos}%
           \scratchskip\scratchdimen
           \advance\scratchskip\MPh\currentposition
           \scratchdimen-\scratchdimen
           \advance\scratchdimen\MPy\currentposition
           %\registerstatus{old \currentautopos: \MPy\currentposition/\MPh\currentposition}%
           \expanded{\replacepospxywhd
             {\currentposition}{\MPp\currentposition}{\MPx\currentposition}{\the\scratchdimen}%
             {\MPw\currentposition}{\the\scratchskip}{\MPd\currentposition}}%
           %\registerstatus{new \currentautopos: \MPy\currentposition/\MPh\currentposition}%
           \egroup
           \global\laststackvmove\scratchdimen % new
           \setbox\nextbox\iftracepositions\@EA\ruledhbox\else\@EA\hbox\fi
             {\lower\scratchdimen\flushnextbox}%
           \next
           %\registerstatus{\strut}%
         \fi
         \flushnextbox}%
        \egroup}}

%D The next hack make sure that margin texts near faulty
%D strutted lines are handled ok.

\newif\ifrepositionmarginbox \repositionmarginboxtrue

\newcount\currentmarginpos

\def\dopositionmarginbox#1%
  {\bgroup
   \ifrepositionmarginbox
     \global\advance\currentmarginpos\plusone
     \setposition{\s!margin:\number\currentmarginpos}%
     \ifcase\marginrepositionmethod
       % nothing
     \or
       % nothing
     \or
       % stack / page check yet untested
       \scratchdimen\MPy{\s!margin:\number\currentmarginpos}%
       \global\advance\currentmarginpos\plusone
       \advance\scratchdimen -\MPy{\s!margin:\number\currentmarginpos}%
       \advance\scratchdimen -\strutdp
       % new
       \setbox#1\hbox
         {\hskip-\MPx{\s!margin:\number\currentmarginpos}%
          \hskip\MPx{head:\realfolio}%
          \box#1}%
       % so far
       \setbox#1\hbox
         {\setposition{\s!margin:\number\currentmarginpos}%
          \raise\scratchdimen\box#1}%
     \or
       % move up
       \ifnum\MPp{p:\number\parposcounter}=\MPp{\s!margin:\number\currentmarginpos}\relax
         \scratchdimen\dimexpr\MPy{p:\number\parposcounter}-\MPy{\s!margin:\number\currentmarginpos}\relax
         \expanded{\setbox#1\hbox{\raise\scratchdimen\box#1}\ht#1\the\ht#1\dp#1\the\dp#1}%
       \fi
     \or
       % move up, assume end of par
       \ifnum\MPp{p:\number\parposcounter}=\MPp{\s!margin:\number\currentmarginpos}\relax
         \getnoflines\margincontentheight
         \advance\noflines\minusone
         \scratchdimen\noflines\lineheight
       \else
         \scratchdimen\dimexpr\MPy{p:\number\parposcounter}-\MPy{\s!margin:\number\currentmarginpos}\relax
       \fi
       \expanded{\setbox#1\hbox{\raise\scratchdimen\box#1}\ht#1\the\ht#1\dp#1\the\dp#1}%
     \fi
     \dp#1\zeropoint
     \ht#1\zeropoint
   \fi
   \graphicvadjust{\box#1}%
   \egroup}

\chardef\marginrepositionmethod\plusone   % sidemethod
\chardef\margincontentmethod   \plusthree % textmethod  % beware: 1 = old method
\chardef\marginpagecheckmethod \plusone   % splitmethod

%D For a right menu, a sequence of calls to \type
%D {right_menu_button} is generated.
%D
%D \starttyping
%D right_menu_button (n, p, s=0/1/2, x, y, w, h, d) ;
%D \stoptyping
%D
%D Here, n is the number of the button, s a status variable,
%D while the rest is positional info. The status variable is
%D 0, 1 or~2: not found, found and found but current page.

% 0=not found 1=found 2=current page

% geen leeg

\newtoks\MPmenutoks

\def\MPmenubuttons#1{\the\MPmenutoks}

\appendtoks \global\MPmenutoks\emptytoks \to \everyshipout

% 0=notfound  1=found  2=currentpage

\def\do@@amposition#1#2#3%
  {\doifelsevalue{\??am#1\c!position}\v!yes
     {\doglobal\increment\currentamposition
      \doifnumberelse{#2}
        {\docheckrealreferencepage{#2}%
         \global\chardef\currentamrealpage\ifrealreferencepage2\else1\fi}
        {\doifreferencefoundelse{#2}
           {\global\chardef\currentamrealpage\ifrealreferencepage2\else1\fi}
           {\global\chardef\currentamrealpage0}}% % not found
      \expanded
        {\doglobal\noexpand\appendtoks
%            #1_menu_button(\currentamposition,\the\currentamrealpage,\noexpand\MPpos{#1:\currentamposition}) ;
           #1_menu_button(\currentamposition,\the\currentamrealpage,\MPpos{#1:\currentamposition}) ;
         \to \MPmenutoks}%
      \hpos{#1:\currentamposition}{#3}}
     {#3}}

\def\do@@ammenuposition#1%
  {\ifnum\currentamposition>0
     \dowithnextbox{\hpos{menu:#1:\realfolio}{\flushnextbox}}\hbox
   \fi}

%D \macros
%D   {GFC, GTC, GSC}
%D
%D The next macros extend tables and tabulation with
%D backgrounds and position related features. Areas are
%D specified with symbolic names, and symbolic references to
%D the graphics involved. Each table has its own namespace.

\newcounter\noftabpositions
\newtoks   \posXCtoks

\def\tbPOSprefix
  {tbp:\noftabpositions:}

\def\tablepos
  {\scratchtoks\posXCtoks
   \global\posXCtoks\emptytoks
   \the\scratchtoks}

\let\tabulatepos\tablepos

\def\dodododoGSC[#1:#2]%
  {\remappositionframed{#2}{\tbPOSprefix#1}%
   \bpos{\tbPOSprefix#1}%
   \doglobal\appendtoks\@EA\epos\@EA{\tbPOSprefix#1}\to\posXCtoks}

\def\dododoGSC[#1:#2:#3]%
  {\doglobal\appendtoks\dodododoGSC[#1:#2]\to\posXCtoks\NC}

\def\dodoGSC[#1]%
  {\def\docommand##1{\dododoGSC[##1:##1]}%
   \processcommalist[#1]\docommand}

\def\dodododoGFC[#1:#2:#3]%
  {\remappositionframed{#2}{\tbPOSprefix#1}%
   \bpos{\tbPOSprefix#1}}

\def\dododoGFC[#1]%
  {\def\docommand##1{\dodododoGFC[##1:##1]}%
   \processcommalist[#1]\docommand}

\def\dodoGFC[#1]%
  {\doglobal\appendtoks\dododoGFC[#1]\to\posXCtoks\NC}

\def\dododododoGTC[#1:#2]%
  {\epos{\tbPOSprefix#1}}

\def\dodododoGTC[#1]%
  {\def\docommand##1{\dododododoGTC[##1:##1]}%
   \processcommalist[#1]\docommand}

\def\dododoGTC[#1]%
  {\doglobal\appendtoks\dodododoGTC[#1]\to\posXCtoks}

\def\dodoGTC[#1]%
  {\doglobal\appendtoks\dododoGTC[#1]\to\posXCtoks\NC}

\def\dodododoXC[#1#2]%
  {\if#1>\dodoGFC  [#2:#2]\else
   \if#1+\dodoGFC  [#2:#2]\else
   \if#1<\dodoGTC  [#2:#2]\else
   \if#1-\dodoGTC  [#2:#2]\else
   \if#1=\dodoGSC  [#2:#2]\else
         \dodoGSC[#1#2:#1#2]\fi\fi\fi\fi\fi}

\def\dododoXC#1%
  {\dodododoXC[#1]}

\def\dodoXC[#1]%
  {{\let\NC\relax\processcommalist[#1]\dododoXC}}

\def\doGSC[#1]{\iffirstargument\dodoGSC[#1]\else\expandafter\NC\fi}
\def\doGFC[#1]{\iffirstargument\dodoGFC[#1]\else\expandafter\NC\fi}
\def\doGTC[#1]{\iffirstargument\dodoGTC[#1]\else\expandafter\NC\fi}
\def\doXC [#1]{\iffirstargument\dodoXC [#1]\else\expandafter\fi\NC}

\def\tbGSC{\dosingleempty\doGSC}
\def\tbGFC{\dosingleempty\doGFC}
\def\tbGTC{\dosingleempty\doGTC}
\def\tbXC {\dosingleempty\doXC }

%D The amount of code to support tables and tabulation is
%D rather minimalistic.

\let\tabulatepos\tablepos

\def\tabulatenormalpos
  {\hss\tabulatepos\hss}

\def\tabulateequalpos
  {\setbox\scratchbox\hbox{\tabulateEQ}%
   \hbox to \wd\scratchbox{\hss\kern\zeropoint\tabulatepos\hss}%
   \hskip-\wd\scratchbox
   \box\scratchbox}

\def\tabulatenormalcolumn#1% overloaded
  {&\iftabulateequal\tabulateequalpos\else\tabulatenormalpos\fi
   &\global\chardef\tabulatetype#1&}

\def\tabulateequalcolumn#1% overloaded
  {&\tabulateequalpos
   &\global\chardef\tabulatetype#1&}

\appendtoks
  \doglobal\increment\noftabpositions
\to \everytabulate

%D In order to prevent potential clashes with abbreviations,
%D postpone the mapping.

\appendtoks
  \let\GSC\tbGSC \let\GFC\tbGFC \let\GTC\tbGTC \let\XC\tbXC
\to \everytabulate

%D \macros
%D   {definepositionframed}
%D
%D The next example show how to provide backgrounds to table
%D cells. First we define some framed backgrounds.
%D
%D \startbuffer
%D \definepositionframed[x][background=color,backgroundcolor=red]
%D \definepositionframed[y][background=color,backgroundcolor=green]
%D \definepositionframed[z][background=color,backgroundcolor=blue]
%D \stopbuffer
%D
%D \typebuffer
%D
%D % \getbuffer
%D
%D \startbuffer
%D \starttabulate[|c|c|c|]
%D \GFC[f:x] this is a small  \NC       table            \NC     in which we \NC \FR
%D \NC       will demonstrate \GFC[g:z] that this        \GTC[g] positioning \NC \MR
%D \GSC[e:y] mechanism also   \GTC[f]   works quite well \NC     in tables   \NC \LR
%D \stoptabulate
%D \stopbuffer
%D
%D The table itself defines three areas (a, b and~c) using
%D these frames.
%D
%D \typebuffer
%D % \getbuffer
%D
%D Tables (based on \TABLE) are supported by:

\def\normalTABLEsimplebar {\unskip\!ttRightGlue&\tablepos&}                       % |
\def\normalTABLEcomplexbar{\unskip\!ttRightGlue&\omit\tablepos\!ttAlternateVrule} % \|
\def\normalTABLEquote     {\unskip\!ttRightGlue&\omit\tablepos&}                  % "

\appendtoks
  \doglobal\increment\noftabpositions
\to \everytable

%D Since we don't want nameclashes:

\appendtoks
  \let\GSC\tbGSC \let\GFC\tbGFC \let\GTC\tbGTC \let\XC\tbXC
\to \everytable

%D In the previous example, we could have provided an overlay to
%D the framed definition. A more direct approach is demonstrated
%D below:
%D
%D \startbuffer
%D \def\cw#1{\color[white]{#1}}
%D
%D \startMPpositiongraphic{tableshade}
%D   initialize_area(\MPpos{\MPvar{from}},\MPpos{\MPvar{to}}) ;
%D   color c ; c := \MPvar{color} ;
%D   linear_shade(pxy,0,.4c,.9c) ;
%D   anchor_area(\MPanchor{\MPvar{from}}) ;
%D \stopMPpositiongraphic
%D
%D \setMPpositiongraphic{b:x}{tableshade}{from=b:x,to=e:x,color=red}
%D \setMPpositiongraphic{b:y}{tableshade}{from=b:y,to=e:y,color=green}
%D \setMPpositiongraphic{b:z}{tableshade}{from=b:z,to=e:z,color=blue}
%D \stopbuffer
%D
%D \typebuffer \getbuffer
%D
%D The definition of the table looks about the same as the
%D previous one:
%D
%D \startbuffer
%D \starttable[|c|c|c|]
%D \GFC[b:z] \cw{this is a small}  \NC       \cw{table}            \NC     in which we      \NC \FR
%D \NC       \cw{will demonstrate} \GFC[c:y] \cw{that this}        \GTC[c] \cw{positioning} \NC \MR
%D \GSC[a:x] \cw{mechanism also}   \GTC[b]   \cw{works quite well} \NC     in tables        \NC \LR
%D \stoptable
%D \stopbuffer
%D
%D \typebuffer
%D
%D \getbuffer

% \definepositionframed[w][background=color,backgroundcolor=yellow]
% \definepositionframed[x][background=color,backgroundcolor=red]
% \definepositionframed[y][background=color,backgroundcolor=green]
% \definepositionframed[z][background=color,backgroundcolor=blue]
%
% \starttabulate[|c|c|c|]
% \NC this is a small  \NC table            \NC in which we \NC \FR
% \NC will demonstrate \NC that this        \NC positioning \NC \MR
% \NC mechanism also   \NC works quite well \NC in tables   \NC \LR
% \stoptabulate
%
% \starttabulate[|c|c|c|]
% \GFC[f:x] this is a small      \GTC      table            \NC     in which we \NC \FR
% \NC       will demonstrate     \GFC[g:z] that this        \GTC[g] positioning \NC \MR
% \GSC[e:y] mechanism also       \GTC[f]   works quite well \NC     in tables   \NC \LR
% \stoptabulate
%
% \starttabulate[|c|c|c|]
% \GFC[f:x,d:w] this is a small  \GTC[d]   table            \NC     in which we \NC \FR
% \NC           will demonstrate \GFC[g:z] that this        \GTC[g] positioning \NC \MR
% \GSC[e:y]     mechanism also   \GTC[f]   works quite well \NC     in tables   \NC \LR
% \stoptabulate
%
% \starttabulate[|c|c|c|]
% \XC[+f:x] this is a small  \XC      table            \NC     in which we \NC \FR
% \NC      will demonstrate  \XC[+g:z] that this       \XC[-g] positioning \NC \MR
% \XC[=e:y] mechanism also   \XC[-f]  works quite well \NC     in tables   \NC \LR
% \stoptabulate
%
% \starttabulate[|c|c|c|]
% \XC[+f:x,+d:w] this is a small  \XC[-d]   table            \NC     in which we \NC \FR
% \NC            will demonstrate \XC[+g:z] that this        \XC[-g] positioning \NC \MR
% \XC[=e:y]       mechanism also  \XC[-f]   works quite well \NC     in tables   \NC \LR
% \stoptabulate

% evt                  [b:x]
%
% \definepositionframed[x][background=color,fillcolor=red]
% \definepositionframed[y][background=color,fillcolor=green]
% \definepositionframed[z][background=color,fillcolor=blue]

\def\remappositionframed#1#2% from to
  {\copyposition{b:#1}{b:#2}%
   \copyposition{e:#1}{e:#2}%
   \dosetpositionaction{b:#2}{\dopositionaction{b:#1}}}

\def\definepositionframed
  {\dodoubleargument\dodefinepositionframed}

\def\dodefinepositionframed[#1][#2]%
  {\dosetpositionaction{b:#1}{\dopositionframed[#1][#2]}}

\def\positionframed
  {\dodoubleempty\dopositionframed}

\def\dopositionframed[#1][#2]%
  {\bgroup
   \setbox\scratchbox\hbox
     {\dimen0=\MPx{e:#1}%
      \advance\dimen0 -\MPx{b:#1}%
      \dimen2=\MPy{b:#1}%
      \advance\dimen2 -\MPy{e:#1}%
      \advance\dimen2 \MPd{e:#1}%
      \lower\dimen2\hbox
        {\advance\dimen2 \MPh{b:#1}%
         \framed
           [\c!width=\dimen0,\c!height=\dimen2,
            \c!offset=\v!overlay,#2]{}}}%
   \smashedbox\scratchbox
   \egroup}

% \def\sethdistances#1%
%   {\hbox{\lpos{ml:#1}\hpos{mh:#1}{\strut}\rpos{mr:#1}}}
%
% \def\gethdistances#1%
%   {\scratchdimen\MPx{mh:#1}%
%    \advance\scratchdimen -\MPx{ml#1}%
%    \edef\lefthdistance{\the\scratchdimen}%
%    \scratchdimen\MPx{mr:#1}%
%    \advance\scratchdimen -\MPx{mh:#1}%
%    \edef\righthdistance{\the\scratchdimen}}

\protect \endinput

% todo 1: shift down option

% \startuseMPgraphic{mpos:par:columnset}
%   \iftracepositions show_multi_pars \else draw_multi_pars \fi ;
%   path p ; p := boundingbox currentpicture ;
%   currentpicture := currentpicture shifted (0,-StrutDepth/2) ;
%   setbounds currentpicture to p ;
% \stopuseMPgraphic

\definetextbackground[underline]    [location=text,alternative=1,background=,frame=off]
\definetextbackground[overstrike]   [location=text,alternative=2,background=,frame=off]
\definetextbackground[exlines]      [location=text,alternative=3,background=,frame=off]
\definetextbackground[strikethrough][location=text,alternative=4,background=,frame=off]

\definestartstop [underline]
  [before={\starttextbackground[underline]},
   after=\stoptextbackground]

\definestartstop
  [overstrike]
  [before={\starttextbackground[overstrike]},
   after=\stoptextbackground]

\definestartstop
  [exlines]
  [before={\starttextbackground[exlines]},
   after=\stoptextbackground]

\definestartstop
  [strikethrough]
  [before={\starttextbackground[strikethrough]},
   after=\stoptextbackground]

\starttext
    \startunderline     \input tufte \stopunderline      \blank
    \startoverstrike    \input tufte \stopoverstrike     \blank
    \startexlines       \input tufte \stopexlines        \blank
    \startstrikethrough \input tufte \stopstrikethrough  \blank
\stoptext
