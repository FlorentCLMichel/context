%D \module
%D   [       file=cont-sys,
%D        version=1995.10.10,
%D          title=\CONTEXT\ Miscellaneous Macros,
%D       subtitle=System Specific Setups,
%D         author=Hans Hagen,
%D           date=\currentdate,
%D      copyright={PRAGMA / Hans Hagen \& Ton Otten}]
%C
%C This module is part of the \CONTEXT\ macro||package and is
%C therefore copyrighted by \PRAGMA. See mreadme.pdf for 
%C details. 

\unprotect

% Here you can take care of overloading some (style) 
% defaults. What goes here, depends on your local system. 
%
% The following commands sets the default font encoding:
%
% \setupencoding [\s!default=ec] 
%
% You can let \CONTEXT\ load the map files for \PDFTEX. 
%
% \autoloadmapfilestrue
%
% If you use the more verbose naming scheme, uncomment this: 
%
% \usetypescript [map] [default,\defaultencoding] 
%
% or : 
%
% \usetypescript [map] [all] 
%
% In case you have set psfonts.map already, you can comment 
% the following lines. Beware: pdftex uses the fontname 
% (second entry on map file lines) for (not so) clever 
% remapping, so in case of troubles, remove the names (is 
% safe)!   
%
% \preloadmapfile [original-ams-cmr]
% \preloadmapfile [original-ams-euler]
% \preloadmapfile [il2-ams-cmr]
% \preloadmapfile [pl0-ams-cmr]
%
% If you want the default berry names (ec and 8r only): 
%
% \usetypescript [berry] [\defaultencoding]
%
% Overload Lucida by Times cum suis:
%
% \definetypescriptsynonym [lbr] [pos] 
%
% Compensate for missing files:
%
% \definefontsynonym [gbhei]   [gbsong]
% \definefontsynonym [gbheisl] [gbsong]
% \definefontsynonym [gbheisl] [gbsong]
%
% The already loaded map file list can be reset with:
%
% \forgetmapfiles
%
% Setting up a global figure path: 
%
% \setupexternalfigures [\c!gebied={e:/fig/eps,t:/mine/figs}]
%
% Loading a specific special driver: 
%
% \setupoutput [dviwindo]
%
% Enabling \CONTEXT\ navigation symbols as well as \euro's. 

\usesymbols [nav,mvs]  

\setupsymbolset [navigation 1] % not that clever 

\setupinteraction [\c!symboolset=navigation 1]

% Changing language defaults: 
% 
% \setuplanguage
%   [nl]
%   [\c!leftquote=\upperleftsinglesixquote,
%    \c!leftquotation=\upperleftdoublesixquote]

% Loading local preferences, for example
%
% \input prag-gen % company styles 
% \input prag-log % more company styles 
%
% Enabling run time \METAPOST\ (also enable \write18 in 
% texmf.cnf):

\runMPgraphicstrue 
\runMPTEXgraphicstrue

\recycleMPslotstrue 

% This saves some runtime, but needs a format, which you can
% make with 'texexec --make --alone metafun'. Make sure that 
% the mem files are moved to the used web2c path (locate with 
% 'kpsewhich plain.mem').

\useMETAFUNformattrue

% Enabling nested pretty printing:

\newprettytrue 

% This can be a way to get things working on system with 
% name clashes. (Some \TeX's tend do search system wide.)

\protectbufferstrue

% You can enable a rigurous figure searching, but normally 
% this is not really needed and even annoying. 
%
% \runutilityfiletrue 

% So far. 

\protect \endinput
