%D \module
%D   [       file=core-01d,
%D        version=1997.03.31,
%D          title=\CONTEXT\ Core Macros,
%D       subtitle=1D (to be split),
%D         author=Hans Hagen,
%D           date=\currentdate,
%D      copyright={PRAGMA / Hans Hagen \& Ton Otten}]
%C
%C This module is part of the \CONTEXT\ macro||package and is
%C therefore copyrighted by \PRAGMA. Non||commercial use is
%C granted.

\writestatus{loading}{Context Core Macros (d)}

\unprotect

\startmessages  dutch  library: interactions
  title: interactie
      1: aspect ratio -- x -- (b x h)
      2: actief
      3: niet actief
      4: geen paginasynchronisatie (--) in hmode
\stopmessages

\startmessages  english  library: interactions
  title: interaction
      1: aspect ratio -- x -- (b x h)
      2: active
      3: inactive
      4: no pagesynchronisation (--) in hmode
\stopmessages

\startmessages  german  library: interactions
  title: Interaktion
      1: Aspekt des Verhaeltnis -- x -- (B x H)
      2: aktiv
      3: inaktiv
      4: keine Seitensynchronisation (--) im hmode
\stopmessages

\startmessages  dutch  library: versions
  title: versie
      1: er mankeert een @+
      2: markeren pagina's
      3: geselecteerde pagina's: --
\stopmessages

\startmessages  english  library: versions
  title: version
      1: missing @+
      2: marking pages
      3: selected pages: --
\stopmessages

\startmessages  german  library: versions
  title: Version
      1: fehlendes @+
      2: Erstelle Seiten
      3: Ausgewaehlte Seiten: --
\stopmessages

%I n=Interacteren
%I c=\stelinteractiein,\stelinteractiemenuin
%I c=\definieerinteractiemenu,\startinteractiemenu
%I c=\blokkeerinteractiemenu,\geefinteractiemenuvrij
%I c=\stelinteractieschermin,\scherm
%I
%I Overzichten en verwijzingen kunnen worden voorzien van
%I voor DVIWINDO en Acrobat betekenisvolle codes. De volgende
%I commando's zijn (voorlopig) beschikbaar:
%I
%I  \stelinteractiein[status=,menu=,letter=,kleur=,strut=,
%I    breedte=,hoogte=,diepte=,pagina=]
%I
%I Hierbij hebben 'letter' en 'kleur' betrekking op in de tekst
%I gemarkeerde woorden waar een verwijzing achter zit. Status
%I kan 'start' of 'stop' zijn en het 'menu' kan 'aan' of 'uit'
%I staan. Met 'pagina=ja' dwingen we pagina verwijzingen af.
%I
%I Zonodig kunnen een 'titel', 'subtitel', 'auteur' en 'datum'
%I worden ingesteld.
%P
%I Een menu wordt gedefinieerd met het commando:
%I
%I   \definieerinteractiemenu[naam][plaats][instellingen]
%I
%I waarbij in plaats van de instellingen de naam van een
%I reeds gedefinieerd menu kan worden opgegeven.
%I
%I Mogelijke plaatsen zijn 'boven', 'onder', 'links' en
%I 'rechts'.
%P
%I De inhoud van een menu wordt gedefinieerd met:
%I
%I   \definieerinteractiemenu[naam][inhoud]
%I
%I Een voorbeeld van zo'n definitie is:
%I
%I   \definieerinteractiemenu
%I     [links]
%I     [{inhoud[inhoud]},
%I      {formules[formules]},
%I      {index[index]}]
%I
%I De accolades zijn essentieel! De verwijzingen dienen met de
%I daarvoor gebruikelijke commando's te worden aangemaakt.
%I Verwijzingen zoals [inhoud] en [index] zijn automatisch
%I beschikbaar.
%P
%I Naast [inhoud], [index] en andere voor de hand liggende
%I verwijzingen zijn beschikbaar:
%I
%I   [eerstepagina]
%I   [vorigepagina]
%I   [volgendepagina]
%I   [laatstepagina]
%P
%I Een alternatieve manier om menu's te definieren levert
%I het commando
%I
%I   \startinteractiemenu
%I   \stopinteractiemenu
%I
%I Hierbinnen zijn de volgende commando's mogelijk:
%I
%I \raw tekst      \\  : ongeformatteerde tekst
%I \but[ref] tekst \\  : interactief menu item
%I \nop            \\  : dummy menu item
%I \txt tekst      \\  : niet interactief menu item (omlijnd)
%I \rul tekst      \\  : niet interactief menu item
%I \com commandos  \\  : commando's
%P
%I Instellingen kunnen ook apart plaatsvinden met:
%I
%I   \stelinteractiemenuin[naam][voor=,na=,tussen=,breedte=,
%I     kader=,letter=,status=,achtergrond=,achtergrondkleur=,
%I     achtergrondraster=]
%I
%I Aan 'voor', 'na' en 'tussen' kunnen commando's worden
%I toegekend, zoals: \hfill, \blanko en \hskip1em. Bij
%I 'breedte' kan een maat, 'passend' of 'ruim' worden
%I meegegeven. Het kader kan 'aan' of 'uit' staan.
%I
%I De status kan 'start', 'stop', 'leeg' of 'geen' zijn. De
%I instelling 'leeg' geldt slechts een pagina. De instelling
%I 'geen' heeft alleen zin bij meerdere menus naast cq. boven
%I elkaar.
%P
%I Het is mogelijk een menu te (de)blokkeren. Dit gebeurt met
%I behulp van het referentie-mechanisme. Het commando luidt:
%I
%I   \blokkeerinteractiemenu[plaats][verwijzingen]
%I   \geefinteractiemenuvrij[plaats][verwijzingen]
%I
%I De verwijzingen moeten worden gescheiden door comma's.
%I Wanneer geen verwijzingen worden meegegeven, wordt de
%I blokkade opgeheven.
%P
%I Eventueel kunnen tussen menuitems commando's worden
%I opgenomen,
%I
%I   \definieerinteractiemenu
%I     [links]
%I     [{alfa[eerste]},
%I      {beta[tweede]},
%I      {{\blanko[forceer,3*groot]}},
%I      {gamma[derde]},
%I      {{\vfil}},
%I      {omega[laatste]}]
%I
%I De extra {} zijn nodig, omdat anders onduidelijkheid is
%I of het commando's zijn of verwijzingen.
%P
%I De afmetingen van het interactiescherm kunnen worden
%I ingesteld met het commando:
%I
%I   \stelinteractieschermin[breedte=,hoogte=,rugwit=,
%I     kopwit=,optie=]
%I
%I Dit is alleen nodig als men het interactieprogramma wil
%I dwingen de tekst op een afwijkend papierformaat weer te
%I geven. Als hoogte en breedte kan 'passend' worden
%I opgegeven, in dat geval wordt uitgegaan van de
%I instellingen bij \stellayoutin.
%I
%I Er kan (in acrobat) met een vol scherm worden opgestart
%I met optie=max.
%P
%I Er zijn meerdere menu's naast elkaar mogelijk. Eerst
%I wordt een menu gedefinieerd met:
%I
%I   \definieerinteractiemenu[naam][plaats][instellingen]
%I
%I De afstand tussen menus wordt ingesteld met 'afstand'.
%I Als afstand=overlay dan vallen menus over elkaar.

\def\setupinteractionscreens%
  {}

\def\complexstelinteractieschermin[#1]%
  {\getparameters[\??sc][#1]%
   \def\setupinteractionscreens% met a, b en \number
     {\bgroup
      \doifelse{\@@scbreedte}{\v!passend}
        {\!!widtha=\linkerrandbreedte
         \advance\!!widtha by \linkerrandafstand
         \advance\!!widtha by \paginascheiding
         \advance\!!widtha by \linkermargebreedte
         \advance\!!widtha by \linkermargeafstand
         \ifdim\rugwit>\!!widtha\ifdim\rugwit>\!!zeropoint\relax
           \advance\rugwit by -\!!widtha
         \fi\fi
         \advance\!!widtha by \zetbreedte
         \advance\!!widtha by \rechtermargeafstand
         \advance\!!widtha by \rechtermargebreedte
         \advance\!!widtha by \paginascheiding
         \advance\!!widtha by \rechterrandafstand
         \advance\!!widtha by \rechterrandbreedte
         \scratchdimen=\@@scrugwit
         \advance\scratchdimen by \@@scrugoffset
         \advance\!!widtha by 2\scratchdimen}
        {\!!widtha=\@@scbreedte}%
      \doifelse{\@@schoogte}{\v!passend}
        {\!!heighta=\bovenhoogte
         \advance\!!heighta by \bovenafstand
         \ifdim\kopwit>\!!heighta\ifdim\kopwit>\!!zeropoint\relax
           \advance\kopwit by -\!!heighta
         \fi\fi
         \advance\!!heighta by \zethoogte
         \advance\!!heighta by \onderafstand
         \advance\!!heighta by \onderhoogte
         \scratchdimen=\@@sckopwit
         \advance\scratchdimen by \@@sckopoffset
         \advance\!!heighta by 2\scratchdimen}
        {\!!heighta=\@@schoogte}%
      \doifelse{\@@scoptie}{\v!max}
        {\doif{\@@lyplaats}{\v!midden}
           {\scratchdimen=\printpapierhoogte
            \advance\scratchdimen by -\papierhoogte
            \divide\scratchdimen by 2
            \advance\kopoffset by \scratchdimen
            \scratchdimen=\printpapierbreedte
            \advance\scratchdimen by -\papierbreedte
            \divide\scratchdimen by 2
            \advance\rugoffset by \scratchdimen}%
         \!!counte=1}
        {\!!counte=0}%
      \showmessage{\m!interactions}{1} % niet waterdicht
        {\@EA\withoutpt\the\!!widtha,\@EA\withoutpt\the\!!heighta}%
      \dosetupscreen
        {\number\rugoffset}{\number\kopoffset}
        {\number\!!widtha}{\number\!!heighta}
        {\the\!!counte}%
      \dosetupidentity
        {\@@iatitel}
        {\@@iasubtitel}
        {\@@iaauteur}
        {CONTEXT / PRAGMA ADE / HASSELT NL / pragma@pi.net / \jobname.tex}
        {\@@iadatum}
      \egroup}}

\def\simplestelinteractieschermin%
  {\setupinteractionscreens}

\def\stelinteractieschermin%
  {\complexorsimple{stelinteractieschermin}}

% Bookmarks zijn eigenlijk niet nodig omdat \TeX uitstekend
% zelf inhoudsopgaven kan genereren.
%
% \acrobatacrobatbookmark[verwijzing]{tekst}
%
% \def\acrobatbookmark[#1]#2%
%   {\doifreferencefoundelse{#1}
%      {\bgroup
%         \special
%           {postscript
%             [/Page \currentrealreference\normalspace
%              /View [ /Fit ]
%              /Title (#2)
%              /OUT
%             pdfmark}%
%       \egroup}%
%      {\unknownreference{#1}}}

%D Due to requests I finally decided to support bookmarks, a
%D driver dependant way of showing tables of content. The most
%D simple way of support is hooking bookmark generation into
%D the existing list mechanisms. That way users can generate
%D bookmarks automatically, although its entirely valid to add
%D bookmarks by defining alternative ones. These will be added
%D at the appropriate place in the list. 

% \hoofdstuk{het eerste hoofdstuk}
%
% \bookmark {de eerste bookmark} % optional overuled hoofdstuk
%
% .... text ....
%
% \placebookmarks [hoofdstuk,paragraaf,subparagraaf,subsubparagraaf,mylist]
%                 [open list]
%
% \bookmark[mylist]{whatever}

\def\@@bookmark {bm::}
\def\@@booklevel{bl::}
\def\@@bookcount{bc::}

\definieerlijst[\@@bookmark]

\appendtoks\flushpostponedbookmark\to\everypar
\appendtoks\flushpostponedbookmark\to\neverypar

\let\flushpostponedbookmark\relax

\def\simplebookmark#1%
  {\ifx\flushpostponedbookmark\relax \else
     \bgroup
     \convertargument#1\to\ascii
     \writestatus{system}{clashing bookmarks: \ascii}% ECHTE MESSAGE MAKEN 
     \egroup
   \fi
   \gdef\flushpostponedbookmark%
     {\global\let\flushpostponedbookmark\relax
      \schrijfnaarlijst[\@@bookmark]{}{#1}}}

\def\complexbookmark[#1]#2%
  {\schrijfnaarlijst[#1]{}{#2}}

\definecomplexorsimple\bookmark

%\def\insertbookmark[#1]#2%
%  {\bgroup
%   \doifreferencefoundelse{#1}
%     {\doinsertbookmark{0}{0}{#2}{\currentrealreference}}{1}
%     {\unknownreference{#1}}%
%   \egroup}

\newif\iftracebookmarks \tracebookmarksfalse

\def\placebookmarks%
  {\dodoubleempty\doplacebookmarks}

\def\doplacebookmarks[#1][#2]%
  {\iflocation
     \iffirstargument
       \bgroup
       \ifsecondargument
         \edef\openbookmarklist{#2}%
       \else
         \let\openbookmarklist=\empty
       \fi
       \global\let\bookmarklevellist=\empty
       \def\bookmarklevelcount{0}%
       \doprocessbookmarks[#1]\dogetbookmarkelement
       \dolijstelement{}{}{}{}{}{}% needed to finish the first pass
       \doprocessbookmarks[#1]\doputbookmarkelement
       \flushbookmark
       \egroup
     \else
       \expanded{\placebookmarks\@EA[\getvalue{\??ih\v!inhoud\c!lijst}]}%
     \fi
   \fi}

\def\doprocessbookmarks[#1]#2%
  {\let\dolijstelement=#2\relax
   \scratchcounter=0
   \def\docommando##1%
     {\advance\scratchcounter by 1   
      \getlistlevel[##1]\listlevel{\the\scratchcounter}%
      \setxvalue{\@@bookcount\the\scratchcounter}{1}%
      \setxvalue{\@@booklevel##1}{\listlevel}}%
   \processcommalist[#1]\docommando
   \setxvalue{\@@bookcount0}{1}%
   \global\chardef\currentbookmarklevel=0
   \global\chardef\previousbookmarklevel=0
   \doutilities{#1,\@@bookmark}{\jobname}{#1}{}{}}

\def\dodogetbookmarkelement#1#2#3#4#5#6%
  {%\doifsomething{#1}
   %  {\global\chardef\currentbookmarklevel=\getvalue{\@@booklevel#1}\relax}%
   \doifelsenothing{#1}
     {\global\chardef\currentbookmarklevel=0\relax}
     {\global\chardef\currentbookmarklevel=\getvalue{\@@booklevel#1}\relax}%
   \ifnum\currentbookmarklevel>\previousbookmarklevel
     \setxvalue{\@@bookcount\the\currentbookmarklevel}{1}\relax
   \else\ifnum\currentbookmarklevel<\previousbookmarklevel
     \bgroup
     \!!counta=\previousbookmarklevel
     \doloop
       {\let\bookmarktag=\empty 
        \!!countb=\!!counta
        \advance\!!countb by -1
        \dorecurse{\!!countb} 
          {\edef\bookmarktag% 
             {\bookmarktag\getvalue{\@@bookcount\recurselevel}:}}%
        \edef\bookmarklevelcount%
          {\getvalue{\@@bookcount\the\!!counta}}% 
        \xdef\bookmarklevellist%
          {\bookmarklevellist/\bookmarktag:\bookmarklevelcount/}%
        \advance\!!counta by -1
        \ifnum\!!counta=\currentbookmarklevel
          \exitloop
        \fi}%
     \egroup
     \@EA\doglobal\@EA\increment\csname \@@bookcount\the\currentbookmarklevel\endcsname\relax
   \else
     \@EA\doglobal\@EA\increment\csname \@@bookcount\the\previousbookmarklevel\endcsname\relax
   \fi\fi
   \global\chardef\previousbookmarklevel=\currentbookmarklevel}

\def\getbookmarklevelcount%
  {\@EA\def\@EA\docommando\@EA[\@EA##\@EA1\@EA/\bookmarktag:##2/##3]%
     {\def\bookmarklevelcount{##2}}%
   \@EA\@EA\@EA\docommando\@EA\@EA\@EA[\@EA\bookmarklevellist\@EA/\bookmarktag:0/]}

\def\dodoputbookmarkelement#1#2#3#4#5#6%
  {%\doifsomething{#1}
   %  {\global\chardef\currentbookmarklevel=\getvalue{\@@booklevel#1}\relax}%
   \doifelsenothing{#1}
     {\global\chardef\currentbookmarklevel=0\relax}
     {\global\chardef\currentbookmarklevel=\getvalue{\@@booklevel#1}\relax}%
   \ifnum\currentbookmarklevel>\previousbookmarklevel
     \setxvalue{\@@bookcount\the\currentbookmarklevel}{1}\relax
   \else\ifnum\currentbookmarklevel<\previousbookmarklevel
     \@EA\doglobal\@EA\increment\csname \@@bookcount\the\currentbookmarklevel\endcsname\relax
   \else
     \@EA\doglobal\@EA\increment\csname \@@bookcount\the\previousbookmarklevel\endcsname\relax
   \fi\fi
   \let\bookmarktag=\empty 
   \!!countb\currentbookmarklevel
   \dorecurse{\!!countb} 
     {\edef\bookmarktag% 
        {\bookmarktag\getvalue{\@@bookcount\recurselevel}:}}%
   \getbookmarklevelcount 
   \iftracebookmarks
     \bgroup
     \par 
     \bookmarktag\quad
     \dorecurse{\currentbookmarklevel}{\quad}\unskip#1\quad
     (\bookmarklevelcount)\quad
     \egroup
   \fi
   \global\chardef\previousbookmarklevel=\currentbookmarklevel
   \insertsomebookmark{#1}{\the\currentbookmarklevel}{\bookmarklevelcount}{#4}{#6}}

\def\dogetbookmarkelement#1#2#3#4#5#6%
  {\doifnot{#1}{\@@bookmark}
     {\dodogetbookmarkelement{#1}{#2}{#3}{#4}{#5}{#6}}}     

\def\doputbookmarkelement#1#2#3#4#5#6%
  {\doifelse{#1}{\@@bookmark}
     {\localbookmark{#4}}
     {\flushbookmark
      \dodoputbookmarkelement{#1}{#2}{#3}{#4}{#5}{#6}}}     

\let\flushbookmark=\relax
\let\localbookmark=\gobbleoneargument

\def\insertsomebookmark#1#2#3#4#5% 
  {\gdef\flushbookmark%
     {\doinsertsomebookmark{#1}{#2}{#3}{#4}{#5}{g}}%
   \gdef\localbookmark##1%
     {\doinsertsomebookmark{#1}{#2}{#3}{##1}{#5}{l}}}

\def\doinsertsomebookmark#1#2#3#4#5#6%
  {\global\utilitydonetrue
   \global\let\localbookmark=\gobbleoneargument
   \global\let\flushbookmark=\relax            
   \doifinstringelse{#1}{\openbookmarklist}
     {\chardef\openbookmark=1}
     {\chardef\openbookmark=0}%
   \iftracebookmarks(#6: #4)\quad(\the\openbookmark)\par\fi
   \doinsertbookmark{#2}{#3}{#4}{#5}{\openbookmark}}

% \startinteractiemenu[rechts]
%   \but [eerste]  eerste  \\
%   \txt hello world       \\
%   \but [tweede]  tweede  \\
%   \nop                   \\
%   \but [tweede]  tweede  \\
%   \rul whow              \\
%   \but [tweede]  tweede  \\
%   \raw hello world       \\
%   \but [tweede]  tweede  \\
%   \com \vfill            \\
%   \but [derde]   derde   \\
% \stopinteractiemenu

\newif\iflocationmenupermitted

\def\testinteractiemenu#1%
   {\iflocation
      \doifelse{\@@iamenu}{\v!aan}
        {\doifelsevalue{\??am#1\c!status}{\v!start}
           {\global\locationmenupermittedtrue}
           {\global\locationmenupermittedfalse}}
        {\global\locationmenupermittedfalse}%
    \else
      \global\locationmenupermittedfalse
    \fi}

\def\doblokkeerinteractiemenu[#1][#2][#3]%
  {\def\dodoblokkeerinteractiemenu##1%
     {\doifelse{#3}{}
        {\setevalue{\??am##1\c!blokkade}{}}
        {\edef\interactieblokkade{\getvalue{\??am##1\c!blokkade}}
         \def\docommando####1%
           {#1{####1}{\interactieblokkade}}% #1 = \remove or \add
         \processcommalist[#3]\docommando
         \setevalue{\??am##1\c!blokkade}{\interactieblokkade}}}%
   \processcommalist[#2]\dodoblokkeerinteractiemenu}

\def\blokkeerinteractiemenu%
  {\dotripleempty\doblokkeerinteractiemenu[\addtocommalist]}

\def\geefinteractiemenuvrij%
  {\dotripleempty\doblokkeerinteractiemenu[\removefromcommalist]}

% ja   : kader/achtergrond met tekst
% leeg : kader/achtergrond maar geen tekst
% nee  : alleen ruimte reserveren
% geen : helemaal weglaten

\newif\iflocationdummy
\newif\ifskippedmenuitem

\def\dosetlocationbox#1[#2]#3#4%
  {\global\skippedmenuitemfalse
   \setbox\locationbox=\hbox
     {% anders cyclische aanroep !
      \iflocationdummy
        \edef\locationboxborder{\getvalue{#1\c!kader}}%
        \edef\locationboxbackground{\getvalue{#1\c!achtergrond}}%
      \else
        \edef\locationboxborder{\v!uit}%
        \edef\locationboxbackground{}%
      \fi
      \localframed[#1]
        [\c!kader=\locationboxborder,\c!achtergrond=\locationboxbackground,#2]
        {\dolocationattributes{#1}{#3}}}%
   \hbox{#4{\copy\locationbox}}}

\def\setlocationboxyes#1[#2]#3[#4]%
  {\ifx\currentouterreference\empty \else
     \let\currentrealreference=\empty
   \fi
   \doifelse{\currentrealreference}{\realfolio}%
     {\ifcase\getvalue{#1\c!zelfdepagina}
        \bgroup
        \locationdummytrue
        \setvalue{#1\c!kleur}{}%
        \dosetlocationbox{#1}[#2,\c!dummy=\v!nee]{#3}{\gotolocation{#4}}%
        \egroup
      \or
        \locationdummytrue
        \dosetlocationbox{#1}[#2,\c!dummy=\v!ja]{#3}{\gotolocation{#4}}%
      \or
        \locationdummyfalse
        \dosetlocationbox{#1}[#2,\c!dummy=\v!ja]{#3}{\gotolocation{#4}}%
      \or
        \locationdummyfalse
        \global\skippedmenuitemtrue
      \fi}
     {\locationdummytrue
      \dosetlocationbox{#1}[#2,\c!dummy=\v!nee]{#3}{\gotolocation{#4}}}}

\def\setlocationboxnop#1[#2]#3[#4]%
  {\ifcase\getvalue{#1\c!onbekendeverwijzing}
     \locationdummytrue
     \dosetlocationbox{#1}[#2,\c!dummy=\v!nee]{#3}{}%
   \or
     \locationdummytrue
     \dosetlocationbox{#1}[#2,\c!dummy=\v!ja]{#3}{}%
   \or
     \locationdummyfalse
     \dosetlocationbox{#1}[#2,\c!dummy=\v!ja]{#3}{}%
   \or
     \locationdummyfalse
     \global\skippedmenuitemtrue
   \fi}

\def\setlocationbox#1[#2]#3[#4]%
  {\doifinstringelse{#4}{\getvalue{#1\c!blokkade}}
     {\setlocationboxnop{#1}[#2]{#3}[#4]}
     {\doifreferencefoundelse{#4}
        {\setlocationboxyes{#1}[#2]{#3}[#4]}
        {\setlocationboxnop{#1}[#2]{#3}[#4]%
         \unknownreference{#4}}}}

\def\dodosetlocationcommanditem#1#2#3[#4]#5\\%
  {\bgroup
   \leavevmode
   \doifelse{#5}{[]}
     {\doifinstringelse{=}{#4}
        {#3}
        {\setlocationbox{\??am#1}[]{#3}[#4]}}
     {#3}%
   \ifskippedmenuitem \else
     \getvalue{\??am#1#2}%
   \fi
   \egroup}

\def\dosetlocationcommanditem#1#2#3%
  {\dodosetlocationcommanditem{#1}{#2}#3[]\\}

\def\setlocationnop#1[#2]#3%
  {\localframed[#1][#2]{#3}}

\def\executeamboxcommands#1#2#3#4#5%
  {\processaction
     [\getvalue{\??am#1\c!dummy}]
     [  \v!ja=>\chardef\handleunknownmenuitem=0\relax,
      \v!leeg=>\chardef\handleunknownmenuitem=1\relax,
       \v!nee=>\chardef\handleunknownmenuitem=2\relax]%
   \getvalue{\??am#1#3}%
   \ifextendedmenu
     \setamboxcommands{#1}{#4}%
     #2%
   \else
     \def\dolocationcommand##1%
       {\dosetlocationcommanditem{#1}{#4}{##1}}%
     \processcommalist[#2]\dolocationcommand
   \fi
   \unskip
   \getvalue{\??am#1#5}}

\def\setamboxcommands#1#2%
  {\def\raw##1\\%
     {\bgroup
      \leavevmode
      ##1%
      \ifskippedmenuitem \else
        \getvalue{\??am#1#2}%
      \fi
      \egroup}%
   \def\but[##1]##2\\%
     {\raw\setlocationbox{\??am#1}[]{##2}[##1]\\}%
   \def\nop##1\\%
     {\raw\phantom{\localframed[\??am#1][]{}}\\}%
   \def\txt##1\\%
     {\raw\localframed[\??am#1][\c!kader=\v!uit]{##1\unskip}\\}%
   \def\rul##1\\%
     {\raw\localframed[\??am#1][]{##1\unskip}\\}%
   \def\com##1\\%
     {##1}}

\def\@@amhbox#1#2#3#4%
  {\testinteractiemenu{#3}%
   \iflocationmenupermitted
     \bgroup
     \def\dolocationcommand##1%
       {\dosetlocationcommanditem{#3}{##1}}%
     \dimen0=\zetbreedte
     \advance\dimen0 by \pagebackgroundhoffset
     \advance\dimen0 by \pagebackgroundhoffset
\advance\dimen0 by -\getvalue{\??am#3\c!linkeroffset}%
\advance\dimen0 by -\getvalue{\??am#3\c!rechteroffset}%
     \setbox0=\hbox to \dimen0
       {\forgetall
        \executeamboxcommands{#3}{#4}\c!links\c!midden\c!rechts}%
     \wd0=\zetbreedte
% geen \ht=#2 setting (yet)
     \hskip-\pagebackgroundhoffset
\hskip \getvalue{\??am#3\c!linkeroffset}%
     \box0\relax
     \egroup
   \else
     #1\relax
   \fi}

\def\@@amvbox#1#2#3#4% don't change skipping, this one works!
  {\testinteractiemenu{#3}%
   \iflocationmenupermitted
     \bgroup
     \dimen0=\teksthoogte
     \advance\dimen0 by \pagebackgroundvoffset
     \advance\dimen0 by \pagebackgroundvoffset
     \advance\dimen0 by \pagebackgrounddepth
\advance\dimen0 by -\getvalue{\??am#3\c!bovenoffset}%
\advance\dimen0 by -\getvalue{\??am#3\c!onderoffset}%
     \setbox0=\vbox to \dimen0
       {\forgetall                     % Voor't geval de afstand
        \stelblankoin[\v!standaard]%   % (tijdelijk) is aangepast.
        \hsize#2\relax
        \executeamboxcommands{#3}{#4}\c!voor\c!tussen\c!na}%
     \setbox0=\vbox
       {\vskip-\pagebackgroundvoffset
\vskip\getvalue{\??am#3\c!bovenoffset}%
        \ht0=\!!zeropoint
        \box0
        \vskip\pagebackgroundvoffset}% overbodig 
     \ht0=\teksthoogte
     \wd0=#2\relax
     \box0
     \egroup
   \else
     #1\relax
   \fi}

\setvalue{\??am\s!do\v!rechts}%
  {\@@amvbox{\dodummypageskip\v!rechts}\rechterrandbreedte}

\setvalue{\??am\s!do\v!links}%
  {\@@amvbox{\dodummypageskip\v!links}\linkerrandbreedte}

\setvalue{\??am\s!do\v!boven}%
  {\@@amhbox{\dodummypageskip\v!boven}\bovenhoogte}

\setvalue{\??am\s!do\v!onder}%
  {\@@amhbox{\dodummypageskip\v!onder}\onderhoogte}

\def\dointeractiemenu#1#2%
  {\getvalue{\??am\s!do\getvalue{\??am#1\c!plaats}}{#1}{#2}}

\def\interactiemenu[#1]%
  {\donottest\getvalue{\??am\c!menu#1}}

\def\horizontaalinteractiemenu#1#2#3#4%
  {\dimen2=\!!zeropoint
   \setbox0=\hbox
     {\def\docommando##1%
        {\doifnotvalue{\??am##1\c!status}{\v!geen}
           {\hskip\dimen2
            \setbox2=\hbox to #2
              {\getvalue{\??am##1#3}\interactiemenu[##1]\getvalue{\??am##1#4}}%
            \doifelsevalue{\??am##1\c!afstand}{\v!overlay}
              {\dimen2=\!!zeropoint
               \wd2=\!!zeropoint}%
              {\dimen2=\getvalue{\??am##1\c!afstand}}%
            \box2}}%
     \startinteractie
     \processcommacommand[\getvalue{\??am#1}]\docommando
     \stopinteractie}%
   \wd0=#2\relax
   \box0\relax}

\def\vertikaalinteractiemenu#1#2#3#4%
  {\dimen2=\!!zeropoint
   \setbox0=\vbox
     {\def\docommando##1%
        {\doifnotvalue{\??am##1\c!status}{\v!geen}
           {\vskip\dimen2
            \setbox2=\vbox to #2
              {\getvalue{\??am##1#3}\interactiemenu[##1]\getvalue{\??am##1#4}}%
            \doifelsevalue{\??am##1\c!afstand}{\v!overlay}
              {\dimen2=\!!zeropoint
               \offinterlineskip
               \dp2=\!!zeropoint
               \ht2=\!!zeropoint}%
              {\dimen2=\getvalue{\??am##1\c!afstand}}%
            \box2}}%
      \startinteractie
      \processcommacommand[\getvalue{\??am#1}]\docommando
      \stopinteractie}%
   \ht0=#2\relax
   \dp0=\!!zeropoint
   \box0\relax}

\def\interactiemenus[#1]%
  {\iflocation
     \processaction
       [#1]
       [  \v!links=>\horizontaalinteractiemenu\v!links\linkerrandbreedte\c!links\c!rechts,
         \v!rechts=>\horizontaalinteractiemenu\v!rechts\rechterrandbreedte\c!links\c!rechts,
          \v!boven=>\vertikaalinteractiemenu\v!boven\bovenhoogte\c!voor\c!na,
          \v!onder=>\vertikaalinteractiemenu\v!onder\onderhoogte\c!voor\c!na]%
   \else
     \dodummypageskip{#1}%
   \fi}

\setvalue{\??am\v!links}{}
\setvalue{\??am\v!rechts}{}
\setvalue{\??am\v!boven}{}
\setvalue{\??am\v!onder}{}

\newif\ifextendedmenu

\def\dodefinieerinteractiemenu[#1][#2][#3]%
  {\ConvertToConstant\doifelse{#3}{}
     {\setvalue{\??am\c!menu#1}%
        {\extendedmenufalse\dointeractiemenu{#1}{#2}}}
     {\setvalue{\??am\c!menu#1}%
        {\extendedmenufalse\dointeractiemenu{#1}{}}%
      \presetlocalframed[\??am#1]%
      \setvalue{\??am#1\c!blokkade}{}%
      \edef\!!stringe{\getvalue{\??am#2}}%
      \addtocommalist{#1}\!!stringe
      \letvalue{\??am#2}=\!!stringe
      \doifnot{#1}{#2}
        {\copyparameters[\??am#1][\??am#2]
           [\c!links,\c!midden,\c!rechts,\c!voor,\c!na,\c!tussen,
            \c!breedte,\c!hoogte,\c!afstand,\c!offset,\c!kader,
            \c!achtergrond,\c!achtergrondkleur,\c!achtergrondraster,
            \c!letter,\c!kleur,\c!zelfdepagina,\c!onbekendeverwijzing,
            \c!linkeroffset,\c!rechteroffset,\c!bovenoffset,\c!onderoffset]}%
      \ConvertToConstant\doifinstringelse{=}{#3}
        {\getparameters[\??am#1][\c!plaats=#2,#3]}%
        {\doifnot{#2}{#3}
           {\copyparameters[\??am#1][\??am#3]
              [\c!links,\c!midden,\c!rechts,\c!voor,\c!na,\c!tussen,
               \c!breedte,\c!hoogte,\c!afstand,\c!offset,\c!kader,
               \c!achtergrond,\c!achtergrondkleur,\c!achtergrondraster,
               \c!letter,\c!kleur,\c!zelfdepagina,\c!onbekendeverwijzing,
               \c!linkeroffset,\c!rechteroffset,\c!bovenoffset,\c!onderoffset]}%
         \getparameters[\??am#1][\c!plaats=#2]}}}

\def\definieerinteractiemenu%
  {\dotripleempty\dodefinieerinteractiemenu}

\def\startinteractiemenu[#1]#2\stopinteractiemenu%
  {\setvalue{\??am\c!menu#1}%
     {\extendedmenutrue\dointeractiemenu{#1}{#2}}}

\def\dododostelinteractiemenuin#1%
  {\processaction
     [\getvalue{#1}]
     [     \v!ja=>\setvalue{#1}{0},
         \v!leeg=>\setvalue{#1}{1},
          \v!nee=>\setvalue{#1}{2},
         \v!geen=>\setvalue{#1}{3},
      \s!default=>\setvalue{#1}{1}]}

\def\dodostelinteractiemenuin[#1][#2]%
  {\def\docommando##1%
     {\getparameters[\??am##1][#2]
      \dododostelinteractiemenuin{\??am##1\c!onbekendeverwijzing}%
      \dododostelinteractiemenuin{\??am##1\c!zelfdepagina}}%
   \processcommalist[#1]\docommando}

\def\dostelinteractiemenuin[#1][#2]%
  {\ConvertToConstant\doifinstringelse{=}{#2}
     {\dodostelinteractiemenuin[#1][#2]}
     {\dodefinieerinteractiemenu[#1][#2][]}}

\def\stelinteractiemenuin%
  {\dodoubleargument\dostelinteractiemenuin}

% \scherm moet worden als \pagina

\def\simplescherm%  zou niet nodig moeten zijn
  {\iflocation
     \pagina[\v!ja]%
   \fi}

\def\complexscherm[#1]%
  {\iflocation
     \pagina[#1]%
   \fi}

\def\scherm%
  {\complexorsimple{scherm}}

%I n=Figuren++
%I c=\startfiguur,\refereer,\markeer,\toelichting
%I
%I Het is mogelijk naar een plaats in een figuur te verwijzen.
%I Het commando \gebruikfiguur[][][] wordt in dat geval vervangen
%I door het commando:
%I
%I   \startfiguur[naam][file][factor=]%
%I     ...
%I     \refereer(x1,y1)(h1,b1)[referentie1]%
%I     ...
%I     \markeer(xn,yn)(hn,bn)[referentien]%
%I     ...
%I   \stopfiguur
%P
%I Wanneer als optie bij \gebruikexternefiguren 'test' wordt
%I meegegegeven, wordt een testpagina met een rasterverdeling
%I gegenereerd. Met x en y wordt de linkerbovenhoek ingesteld,
%I met h en b de hoogte en de breedte van het te markeren deel
%I van de figuur.
%I
%I Standaard is de hoogte verdeeld in 24 stukken. Wil men een
%I andere indeling, dan kan men met \gebruikexternefiguren
%I de 'xmax' of 'ymax' instellen (aanbevolen: 10 - 50).
%I
%I De hokjes waarop geklikt kan worden kunnen zichtbaar worden
%I gemaakt met 'hokjes=aan'.
%P
%I Om duidelijk te kunnen maken achter welke delen van een
%I figuur iets zit, kan een kleurenbalk worden opgenomen.
%I
%I    ...
%I    \kleurenbalk[...,...,...]
%I    ...
%I
%I (deze tekst moet nog worden aangevuld)
%P
%I Het is mogelijk in een tekst toelichtingen op te nemen.
%I
%I    ...
%I     \toelichting(x1,y1)(h1,b1)[instellingen]{tekst}
%I    ...
%I
%I De instellingen zijn optioneel en komen overeen met die van
%I het commando \omlijnd.

\newcount\efreference
\newdimen\efxsteps
\newdimen\efysteps

\def\calculateefsteps%
  {\ifnum0\@@exxmax=0
     \ifnum0\@@exymax=0
       \def\@@exymax{24}%
     \fi
     \efysteps=\fighei \divide\efysteps by \@@exymax
     \efxsteps=\efysteps
     \dimen0=\figwid
     \advance\dimen0 by \efysteps
     \divide\dimen0 by \efysteps
     \edef\@@exxmax{\number\dimen0}%
   \else
     \efxsteps=\figwid \divide\efxsteps by \@@exxmax    
     \efysteps=\fighei \divide\efysteps by \@@exymax
   \fi}

\def\efcomment#1(#2,#3)#4(#5,#6)%    {kader}(x,y)(h,b)[...]{tekst}
  {\def\complexefdocomment[##1]##2%
     {\positioneer(#2,#3)%
        {\setnostrut
         \framed
           [\c!breedte=#5\efxsteps,
            \c!hoogte=#6\exysteps,
            \c!offset=\v!geen,
            \c!kader=#1,
            ##1]%
           {##2}}}%
   \complexorsimpleempty{efdocomment}}

\def\efnocomment(#1,#2)#3(#4,#5)%    (x,y)(h,b)[...]{tekst}
  {\def\complexefdonocomment[##1]##2{}%
   \complexorsimpleempty{efdonocomment}}

\def\efdomarker(#1,#2)#3#4%    (h,b){kader}{tekst}
  {\framed
     [\c!breedte=#1\efxsteps,
      \c!hoogte=#2\efysteps,
      \c!offset=\v!geen,
      \c!kader=#3]%
     {#4}}

\def\efmark(#1,#2)#3(#4,#5)#6[#7]%
  {\advance\efreference by 1
   \positioneer(#1,#2)
     {\hbox{\the\efreference}}%
   \positioneer(#1,#2)
     {\doifreferencefoundelse{\s!vwb:#7}
        {\gotosomeinternal
           {\s!vwb}{#7}{\currentrealreference}
           {\efdomarker(#4,#5){\v!aan}{\thisissomeinternal{\s!vwa}{#7}}}}
        {[?]}}}

\def\effiguur#1%
  {\positioneer(0,0){\naam{#1}}}

\def\eftext#1(#2,#3)#4(#5,#6)#7[#8]%
  {\advance\efreference by 1
   \hbox
     {\thisissomeinternal{\s!vwb}{#8}%
      \doifreferencefoundelse{\s!vwa:#8}
        {\gotosomeinternal
           {\s!vwa}{#8}{\currentrealreference}
           {\hbox to 1.5em{\the\efreference\presetgoto\hfill}}}
        {\hbox to 1.5em{[?]\hfill}}%
      ~#1 (#2,#3) (#5,#6) [#8]\hfill}%
   \endgraf}

\def\efdoarea(#1,#2)#3#4%    (h,b){kader}{tekst}
  {\start
     \setnostrut
     \framed
       [\c!breedte=#1\efxsteps,
        \c!hoogte=#2\efysteps,     
        \c!offset=\!!zeropoint,
        \c!kader=#3]
       {#4}%
   \stop}

\def\efgoto(#1,#2)#3[#4]%    (h,b)kader[ref]
  {\setbox0=\vbox{\efdoarea(#1,#2)#3{}}%
   \naarbox{\copy0}[#4]}

\def\efthisis(#1,#2)#3[#4]%
  {\pagereference[#4]}
   %\doifreferencefoundelse{#4}
   %  {\efdoarea(#1,#2){#3}{\thisisinternal{#4}}}
   %  {\unknownreference{#4}}}

\newbox\colorbarbox

\def\makecolorbar[#1]%
  {\def\docommando##1%
     {\color[##1]
        {\blackrule
           [\c!breedte=2em,
            \c!hoogte=1ex,
            \c!diepte=\!!zeropoint]}%
      \endgraf}%
   \global\setbox\colorbarbox=\vbox
     {\forgetall
      \processcommalist[#1]\docommando}%
   \global\setbox\colorbarbox=\vbox
     {\hskip2em\box\colorbarbox}%
   \global\wd\colorbarbox=\!!zeropoint}

\def\plaatsstartfiguur[#1][#2][#3]#4\plaatsstopfiguur%
  {\hbox
     {\gebruikexternfiguur[#1][#2][#3]%
      \berekenexternfiguur[#1][#2]%
      \calculateefsteps  
      \startpositioneren
        \def\refereer(##1,##2)##3(##4,##5)##6[##7]%
          {\positioneer(##1,##2)
             {\efgoto(##4,##5){\@@exhokjes}[##7]}}%
        \def\markeer(##1,##2)##3(##4,##5)##6[##7]%
          {\positioneer(##1,##2)
             {\efthisis(##4,##5){\@@exhokjes}[##7]}}%
        \def\toelichting%
          {\efnocomment}%
        \def\kleurenbalk##1[##2]%
          {}%
        \positioneer(0,0){\effiguur{#1}}%
        \linewidth=1pt
        \stelpositionerenin
          [\c!eenheid=pt,
           \c!xschaal=\withoutpt{\the\efxsteps},
           \c!yschaal=\withoutpt{\the\efysteps},
           \c!factor=1]%
        \ignorespaces#4%
        \def\refereer(##1,##2)##3(##4,##5)##6[##7]%
          {}%
        \let\markeer=\refereer
        \def\toelichting%
          {\efcomment\v!nee}%
        \def\kleurenbalk##1[##2]%
          {\makecolorbar[##2]}%
        \ignorespaces#4%
      \stoppositioneren
      \box\colorbarbox}}

% De onderstaande macro mag niet zondermeer worden aangepast
% en is afgestemd op gebruik in de handleiding.

\def\teststartfiguur[#1][#2][#3]#4\teststopfiguur%
  {\begingroup
     \gebruikexternfiguur[#1][#2][\c!bfactor=\v!max]%
     \def\refereer%
       {\efmark}%
     \def\markeer%
       {\efmark}%
     \def\toelichting%
       {\efcomment\v!ja}%
     \def\kleurenbalk##1[##2]%
       {}%
     \efreference=0
     \setbox0=\vbox
       {\hsize240pt
        \startpositioneren
          \berekenexternfiguur[#1][#2]%
          \calculateefsteps
          \positioneer(0,0)
            {\effiguur{#1}}%
          \positioneer(0,0)
            {\rooster
               [\c!nx=\@@exxmax, 
                \c!dx=\withoutpt{\the\efxsteps}, 
                \c!ny=\@@exymax, 
                \c!dy=\withoutpt{\the\efysteps}, 
                \c!xstap=1,
                \c!ystap=1,
                \c!schaal=1,
                \c!offset=\v!nee,
                \c!eenheid=pt]}%
          \stelpositionerenin%
            [\c!eenheid=pt,
             \c!xschaal=\withoutpt{\the\efxsteps},
             \c!yschaal=\withoutpt{\the\efysteps},            
             \c!factor=1]%
          \linewidth=1pt
          \ignorespaces#4\relax
        \stoppositioneren
        \vfill}%
     \efreference=0
     \def\refereer%
       {\eftext{$\rightarrow$}}%
     \def\markeer%
       {\eftext{$\leftarrow$}}%
     \def\toelichting%
       {\efnocomment}%
     \def\kleurenbalk##1[##2]%
       {}%
     \setbox2=\vbox
       {{\tfa\doifelsenothing{#1}{#2}{#1}}
        \blanko
        \tfxx#4
        \vfilll}%
     \ifdim\ht0>\ht2
       \ht2=\ht0
     \else
       \ht0=\ht2
     \fi
     \hbox
       {\hskip3em 
        \vtop{\vskip12pt\box0\vskip6pt}%
        \vtop{\vskip12pt\box2\vskip6pt}}%
   \endgroup}

\def\dostartfiguur[#1][#2][#3]#4\stopfiguur%
  {\doifelse{\@@exoptie}{\v!test}
     {\teststartfiguur[#1][#2][#3]#4\teststopfiguur%
      \def\@@exhokjes{\v!aan}}
     {\def\@@exhokjes{\v!uit}}%
   \setvalue{#1}%
     {\plaatsstartfiguur[#1][#2][#3]#4\plaatsstopfiguur}}

\def\startfiguur%
  {\dotripleargument\dostartfiguur}

%I n=Springen
%I c=\naar,\button,\menubutton,\stelbuttonsin
%I
%I Een woord kan woren gemarkeerd met:
%I
%I   \naar{woord}[referentie]
%I
%I Voorlopig zijn daarnaast beschikbaar:
%I
%I   \button[instellingen]{woord}[referentie]
%I   \menubutton[plaats][instellingen]{woord}[referentie]
%I
%I waarbij de [instellingen] facultatief zijn en {geen} in
%I plaats van {woord} kan worden opgegeven. De instellingen komen,
%I voor zover relevant, overeen met die van \omlijnd. Vaste
%I instellingen vinden plaats met:
%I
%I   \stelbuttonsin[breedte=,hoogte=,offset=,kader=,
%I     achtergrond=,achtergrondkleur=,achtergrondraster=,
%I     kleur=]
%P
%I De elders reeds beschreven commando's \op en \in zijn in
%I geval van interactie ook als volgt te gebruiken:
%I
%I   \in{woord}[referentie]
%I   \op{woord}[referentie]
%I
%I In dat geval wordt 'tekst~referentie' net zo weergegeven
%I als
%I
%I   \naar{woord}[referentie]     % woorden
%I   \naarbox{woord}[referentie]  % robuust
%I
%I opgeroepen verwijzing.

\def\domenubutton[#1][#2]#3[#4]%
  {\bgroup
   \locationdummytrue
   \iffirstargument
     \ifsecondargument
       \setlocationbox{\??am#1}[#2]{#3}[#4]%
     \else
       \ConvertToConstant\doifinstringelse{=}{#1}
         {\setlocationbox{\??bt}[#1]{#3}[#4]}
         {\setlocationbox{\??am#1}[]{#3}[#4]}%
     \fi
   \else
     \setlocationbox{\??bt}[]{#3}[#4]%
   \fi
   \egroup}

\unexpanded\def\menubutton%
  {\dodoubleempty\domenubutton}

\def\domenubox[#1][#2]#3%
  {\bgroup
   \def\setlocationbox##1[##2]##3[##4]%
     {\localframed[##1][##2]{\dolocationattributes{##1}{##3}}}%
   \domenubutton[#1][#2]#3[]%
   \egroup}

\def\menubox%
  {\dodoubleempty\domenubox}

%I n=Externe files
%I c=\gebruikexterndocument,\uit
%I c=\startsynchroniseer,\stopsynchronisatie,\synchroniseer,
%I c=\stelsynchronisatiein,\stelsynchronisatiebalkin,\synchronisatiebalk
%I
%I Mits ondersteund door het interactieprogramma, kan naar
%I een andere file worden gesprongen. Zo'n file moet eerst
%I worden gedefinieerd met:
%I
%I   \gebruikexterndocument [naam] [file] [omschrijving]
%I
%I Een verwijzing naar een andere file ziet er als volgt
%I uit:
%I
%I   \in{tekst}[naam::verwijzing]
%I   \op{tekst}[naam::verwijzing]
%I   \naar{tekst}[naam::verwijzing]
%I
%I De naam van het andere document kan worden opgeroepen met
%I
%I   \uit[naam]
%I
%I terwijl \uit zonder [naam] de bij \in, \op of \naar
%I behorende omschrijving oproept.
%P
%I Documenten kunnen worden gesynchroniseerd. Dat wil zeggen
%I dat gesprongen kan worden naar hetzelfde punt in een anders
%I vormgegeven tekst. Een synchronisatiepunt wordt aangegeven
%I met:
%I
%I   \sychroniseer
%I
%I Een synchronisatie kan uit de pas lopen, bijvoorbeeld als
%I we in een kop een synchronisatiebalk oproepen. Wanneer vindt
%I immers precies de synchronisatie plaats? Vandaar de wat
%I meer betrouwbare menier van aangeven:
%I
%I   \startsynchronisatie    % voor de kop (nu mag balk)
%I   \stopsynchronisatie     % eind van de kop (fixeert plaats)
%P
%I Synchronisatie kan worden ingesteld met:
%I
%I   \stelsynchronisatiein[status=]
%I
%I Het synchronisatiemechanisme is in geval van interactieve
%I teksten standaard niet actief.
%P
%I Er kan een synchronisatiebalk worden opgeroepen waarmee
%I naar een ander document kan worden gesprongen.
%I
%I   \synchronisatiebalk[naam][instellingen]
%I
%I De instellingen kunnen ook apart worden opgegeven:
%I
%I   \stelsynchronisatiebalkin[variant=,breedte=,
%I     letter=,kleur=,achtergrond=,achtergrondraster=,
%I     achtergrondkleur=]
%I
%I Mogelijke varianten zijn 'lokaal' en 'pagina'. In het
%I eerste geval wordt slechts een mogelijkheid geboden, in het
%I tweede geval zijn tot drie mogelijkheden mogelijk. Deze
%I zijn te vergelijken met markeringen.

% Hier volgen de synchronisatiemacro's:

\def\syncprefix{sync}
\def\syncmarker{syncmark}

\definieermarkering[\syncmarker]
\stelmarkeringin[\syncmarker][\c!expansie=\v!ja]

\newcounter\synccounter

\newif\ifsynchronisation

\def\startsynchronisatie%
  {\iflocation\ifsynchronisation
     \doglobal\increment\synccounter
   \fi\fi}

\def\stopsynchronisatie%
  {\iflocation\ifsynchronisation
     %\thisisdestination{\syncprefix:\synccounter}%
     \paginareferentie[\syncprefix:\synccounter]%
     \ifvmode
       \markeer[\syncmarker]{\synccounter}%
     \else
       \showmessage{\m!interactions}{4}{\synccounter}%
     \fi
   \fi\fi}

\def\synchroniseer%
  {\startsynchronisatie
   \stopsynchronisatie}

\def\dostelsynchronisatiein[#1]%
  {\getparameters[\??sy][#1]%
   \doifelse{\@@systatus}{\v!start}
     {\synchronisationtrue}
     {\synchronisationfalse}}

\def\stelsynchronisatiein%
  {\dosingleargument\dostelsynchronisatiein}

\def\definieersynchronisatie%
  {\dosingleargument\dodefinieersynchronisatie}

\def\stelsynchronisatiebalkin%
  {\dodoubleargument\getparameters[\??ba]}

\presetlocalframed[\??ba]

\setvalue{synchronisatie\v!pagina}[#1]%
  {\bgroup
   \stelinteractiein[\c!breedte=\!!zeropoint]%
   \setbox0=\hbox
     {\localframed[\??ba][]%
        {\dolocationattributes{\??ba}{\strut\@@batekst}}}%
   \mindermeldingen
   \def\onder%
     {\leaders\hrule\!!depth1ex\!!height-.5ex\hfil}%
   \def\boven##1##2##3%
     {\dimen0=\wd0\relax
      \divide\dimen0 by 3\relax
      \multiply\dimen0 by ##2\relax
      \dimen2=.25em\relax
      \advance\dimen0 by -##3\dimen2\relax
      %\gotodestination
      %  {}{#1}{\syncprefix:##1}{}
      %  {\hbox to \dimen0{\color[\locationcolor\@@bakleur]{\onder}}}}%
      \naarbox 
        {\hbox to \dimen0{\color[\locationcolor\@@bakleur]{\onder}}}%
        [#1::\syncprefix:##1]}%
   \hbox
     {\prefetchmark[\syncmarker]%
      \def\check##1##2%
        {\edef##2{\fetchmark[\syncmarker][##1]}%
         \ifnum0##2=0 \def##2{1}\fi}%
      \check\v!vorige\top
      \check\v!eerste\first
      \check\v!laatste\bot
      \setbox2=\hbox to \wd0
        {\ifnum\top=\first\relax
           \ifnum\first=\bot\relax
             \boven\first30\relax
           \else
             \boven\first21\hss\boven\bot11\relax
           \fi
         \else
           \ifnum\first=\bot\relax
             \boven\top11\hss\boven\first21\relax
           \else
             \boven\top11\hss\boven\first11\hss\boven\bot11\relax
           \fi
         \fi}%
      \wd2=\!!zeropoint\box2
      \box0\relax}%
   \egroup}

\setvalue{synchronisatie\v!lokaal}[#1]%
  {\bgroup
   \stelinteractiein[\c!breedte=\!!zeropoint]%
   \def\blackrule{\hbox{\vrule\!!height.5em\!!width.5em}}%
   %\gotodestination
   %   {}{##1}{\syncprefix:#1}{0}
   %   {\color[\locationcolor\@@bakleur]{\blackrule}}%
   \naarbox % 
     {\color[\locationcolor\@@bakleur]{\blackrule}}%
     [#1::\syncprefix:\synccounter]%
   \egroup}

\def\synchronisatiebalk[#1][#2]%
  {\iflocation\ifsynchronisation
     \bgroup
     \stelsynchronisatiebalkin
       [\c!tekst=\getvalue{doc:des:#1},#2]%
     \getvalue{synchronisatie\@@bavariant}[#1]%
     \egroup
   \fi\fi}

%I n=Interactiebalk
%I c=\interactiebalk,\stelinteractiebalkin
%I
%I Het volgende commando genereert een interactiebalk. Pas op:
%I de waarde van \realpageno staat niet echt vast.
%I
%I   \interactiebalk[variant=,breedte=,hoogte=,diepte=,
%I     achtergrond=,achtergrondkleur=,achtergrondraster=,
%I     kader=,stap=]
%I
%I   \interactiebalk[reset]
%I
%I   \stelinteractiebalkin[...]
%I
%I Mogelijke stappen zijn 'klein', 'middel', 'groot' en 'n';
%I beschikbare varianten zijn:
%I
%I   a  meter  (scrollbar)
%I   b  jumper (symbool: begin terug vooruit eind)
%I   c  jumper (balk: begin terug vooruit eind)
%I   d  pagina (subpaginas vast formaat)
%I   e  pagina (subpaginas vrij formaat)
%I   f  pagina (subpaginas grote aantallen)
%I   g  pagina (subpaginas jumper)

% Dit is leuke toepassing van glue!

\newbox\meterbox

\newif\ifbalksymbool

\def\doganaareenpagina#1#2#3% nog checken !
  {\checkreferences  % nodig ??
   \iflocation
     \ifnum#3=\realpageno
       {#2}
     \else
       \doifelsenothing{#1}
         {\hbox{\gotorealpage{}{}{#3}{#2}}}
         {\hbox{\gotorealpage{}{}{#3}{\dolocationattributes{#1}{#2}}}}%
     \fi
   \else
     {#2}%
   \fi}

\def\interactiebalka%
  {\iflocation
     \bgroup
     \stelinteractiein[\c!breedte=\!!zeropoint]%
     \setupblackrules[\c!hoogte=\v!max,\c!diepte=\v!max]%
     \!!widthb=\@@ibbreedte\relax
     \advance\!!widthb by -2.75em\relax
     \!!widtha=\!!widthb\relax
     \divide\!!widtha by \lastpage\relax
     \bgroup
       \advance\realpageno by -1\relax
       \ifvoid\meterbox
         \bgroup
         \processaction
           [\@@ibstap]
           [   \v!klein=>\dimen0=.25em\relax,
              \v!middel=>\dimen0=.5em\relax,
               \v!groot=>\dimen0=1em\relax,
             \s!unknown=>\dimen0=\!!widtha]%
         \ifdim\!!widtha<\dimen0\relax
           \!!counta=\dimen0\relax
           \!!countb=\!!widtha
           \divide\!!counta by \!!countb
         \else
           \!!counta=\@@ibstap\relax
         \fi
         \!!widtha=\!!counta\!!widtha
         \setbox0=\hbox{\blackrule[\c!breedte=\!!widtha]}%
         \global\setbox\meterbox=\hbox to \!!widthb
           {\hss
            \for \teller=1 \to \lastpage \step \!!counta \do
              {\gotorealpage{}{}{\teller}{\copy0}}%
            \hss}%
         \global\wd\meterbox=\!!zeropoint\relax
         \egroup
       \fi
     \egroup
     \noindent
     \strut
     \hbox to \@@ibbreedte
       {\mindermeldingen
        \setupblackrules[\c!breedte=1em]%
        \doganaareenpagina\??ib\blackrule\firstpage
        \hss
        \color[middlegray]{\copy\meterbox}%
        \hbox to \!!widthb
          {\ifdim\!!widtha<1em\relax
             \!!widtha=1em\relax
           \fi
           \setupblackrules[\c!breedte=\!!widtha]%
           \ifnum\realpageno>1\relax
             \!!counta=\realpageno
             \advance\!!counta by -2\relax
             \hskip\!!zeropoint\!!plus\!!counta sp\relax % cm gives overflow
             \doganaareenpagina\??ib\blackrule\prevpage
           \fi
           \color[\@@ibcontrastkleur]{\blackrule[\c!breedte=.5em]}%
           \ifnum\realpageno<\lastpage\relax
             \doganaareenpagina\??ib\blackrule\nextpage
             \!!counta=\lastpage\relax
             \advance\!!counta by -\realpageno
             \advance\!!counta by -1\relax
             \hskip\!!zeropoint\!!plus\!!counta sp\relax % cm gives overflow
           \fi}%
        \hss
        \doganaareenpagina\??ib\blackrule\lastpage}%
     \egroup
   \fi}

\presetlocalframed[\??ib]

\def\interactiebalkc%
  {\iflocation
     \ifnum\lastpage>1\relax
       \hbox to \@@ibbreedte
         {\setupblackrules[\c!hoogte=\v!max,\c!diepte=\v!max]%
          \def\goto##1%
            {\doganaareenpagina{}{\blackrule[\c!breedte=##1]}}%
          \dimen0=\@@ibbreedte\relax
          \advance\dimen0 by -4em\relax
          \!!counta=\lastpage
          \advance\!!counta by -1
          \divide\dimen0 by \!!counta
          \!!counta=\realpageno
          \advance\!!counta by -1\relax
          \!!widtha=\!!counta\dimen0\relax
          \!!countb=\lastpage
          \advance\!!countb by -\realpageno
          \!!widthb=\!!countb\dimen0\relax
          \startcolor[\locationcolor\@@ibkleur]%
          \goto{1em}\firstpage
          \hss
          \goto{\!!widtha}\prevpage
          \color[\@@ibcontrastkleur]{\blackrule[\c!breedte=1em]}%
          \goto{\!!widthb}\nextpage
          \hss
          \goto{1em}\lastpage
          \stopcolor}%
     \fi
   \fi}

\def\interactiebalkd%
  {\iflocation\ifshowingsubpage
     \ifnum\nofsubpages>1\relax
       \hbox
       \bgroup
       \stelinteractiein[\c!breedte=\!!zeropoint]%
       \ifbalksymbool % beter: 3 chars assign en 3*box
         \setbox0=\hbox{\gobackwardcharacter}%
         \setbox2=\hbox{\gotosomewherecharacter}%
         \setbox4=\hbox{\goforwardcharacter}%
       \else
         \setbox0=\hbox
           {\vrule
              \!!height\@@ibhoogte
              \!!depth\@@ibdiepte
              \!!width\@@ibbreedte}%
         \setbox2=\copy0
         \setbox4=\copy0
       \fi
       \startcolor[\locationcolor\@@ibkleur]%
       \for\teller=1\to\nofsubpages\step1\do
         {\bgroup
          \increment(\teller,\firstsubpage)\relax
          \decrement\teller\relax
          \ifnum\teller<\realpageno\relax
            \gotorealpage{}{}{\teller}{\copy0}\relax
          \else\ifnum\teller=\realpageno\relax
            \color
              [\@@ibcontrastkleur]
              {\gotorealpage{}{}{\teller}{\copy2}}%
          \else
            \gotorealpage{}{}{\teller}{\copy4}\relax
          \fi\fi
          \egroup
          \hskip\@@ibafstand}%
       \unskip
       \stopcolor
       \egroup
     \fi
   \fi\fi}

\def\interactiebalke%  KAN WORDEN GECOMBINEERD MET D
  {\iflocation\ifshowingsubpage
     \ifnum\nofsubpages>1\relax
       \bgroup
       \!!widthb=\@@ibafstand
       \multiply\!!widthb by \nofsubpages
       \advance\!!widthb by -\@@ibafstand % (n-1)
       \!!widtha=\@@ibbreedte
       \advance\!!widtha by -\!!widthb
       \divide\!!widtha by \nofsubpages\relax
       \ifdim\!!widtha<\@@ibafstand\relax
         \interactiebalkf
       \else
         \stelinteractiein[\c!breedte=\!!zeropoint]%
         \noindent
         \hbox to \@@ibbreedte
           \bgroup
             \ifbalksymbool
               \setbox0=\hbox{\gobackwardcharacter}%
               \setbox2=\hbox{\gotosomewherecharacter}%
               \setbox4=\hbox{\goforwardcharacter}%
             \else
               \setbox0=\hbox
                  {\vrule
                     \!!height\@@ibhoogte
                     \!!depth\@@ibdiepte
                     \!!width\!!widtha}%
               \setbox2=\copy0
               \setbox4=\copy0
             \fi
             \startcolor[\locationcolor\@@ibkleur]%
             \for\teller=1\to\nofsubpages\step1\do
               {\bgroup
                \increment(\teller,\firstsubpage)\relax
                \decrement\teller\relax
                \ifnum\teller<\realpageno\relax
                  \gotorealpage{}{}{\teller}{\copy0}\relax
                \else\ifnum\teller=\realpageno\relax
                  \color
                    [\@@ibcontrastkleur]
                    {\gotorealpage{}{}{\teller}{\copy2}}%
                \else
                  \gotorealpage{}{}{\teller}{\copy4}\relax
                \fi\fi
                \egroup
                \hss}%
             \unskip
             \stopcolor
           \egroup
       \fi
       \egroup
     \fi
   \fi\fi}

\def\interactiebalkf%  !! KAN WORDEN GECOMBINEERD MET D !!
  {\iflocation\ifshowingsubpage
     \ifnum\nofsubpages>1\relax
       \stelinteractiein[\c!breedte=\!!zeropoint]%
       \noindent
       \hbox to \@@ibbreedte
       \bgroup
       \!!countb=0
       \loop
         \advance\!!countb by 1\relax
         \!!countc=\nofsubpages
         \divide\!!countc by \!!countb
         \advance\!!countc by 1
         \!!widthb=\@@ibafstand
         \multiply\!!widthb by \!!countc
         \advance\!!widthb by -\@@ibafstand
         \!!widtha=\@@ibbreedte
         \advance\!!widtha by -\!!widthb
         \divide\!!widtha by \!!countc
         \ifdim\!!widtha<\@@ibafstand\relax
       \repeat
       \ifbalksymbool
         \setbox0=\hbox{\gobackwardcharacter}%
         \setbox2=\hbox{\gotosomewherecharacter}%
         \setbox4=\hbox{\goforwardcharacter}%
       \else
         \setbox0=\hbox
           {\vrule
              \!!height\@@ibhoogte
              \!!depth\@@ibdiepte
              \!!width\!!widtha}%
         \setbox2=\hbox
           {\dimen0=\@@ibhoogte
            \dimen2=\@@ibdiepte
            \vrule
              \!!height.5\dimen0
              \!!depth.5\dimen2
              \!!width\!!widtha}%
         \ht2=\ht0
         \dp2=\dp0
         \setbox4=\copy0
       \fi
       \def\goto##1##2%
         {\ifnum##1=\realpageno\relax
            \color
              [\@@ibcontrastkleur]
              {\gotorealpage{}{}{##1}{##2}}%
          \else
            \gotorealpage{}{}{##1}{##2}\relax
          \fi}%
       \startcolor[\locationcolor\@@ibkleur]%
       \!!countc=\realpageno \advance\!!countc by -2\relax
       \!!countd=\realpageno \advance\!!countd by 2\relax
       \for\teller=\firstsubpage\to\lastsubpage\step1\do
         {\!!doneafalse
          \!!donebfalse
          \ifnum\teller=\firstsubpage\relax \!!doneatrue \fi
          \ifnum\teller=\lastsubpage\relax  \!!doneatrue \fi
          \ifnum\teller>\!!countc \ifnum\teller<\!!countd \!!doneatrue \fi\fi
          \advance\!!countf by 1\relax
          \ifnum\!!countf=\!!countb\relax \!!donebtrue \fi
          \if!!donea   % nog eens checken
            \!!countf=0
            \goto\teller{\copy2}%
            \hss
          \else\if!!doneb
            \!!countf=0
            \ifnum\teller<\realpageno\relax
              \goto\teller{\copy0}%
            \else\ifnum\teller>\realpageno\relax
              \goto\teller{\copy2}%
            \else
              \goto\teller{\copy4}%
            \fi\fi
            \hss
          \fi\fi}%
       \unskip
       \stopcolor
       \egroup
     \fi
   \fi\fi}

\def\interactiebalkb%
  {\ifnum\lastpage>\firstpage\relax
     \interactiebuttons
       [\v!eerste  \v!pagina,
        \v!vorige  \v!pagina,
        \v!volgende\v!pagina,
        \v!laatste \v!pagina]%
   \fi}

\def\interactiebalkg%
  {\ifnum\lastsubpage>\firstsubpage\relax
     \interactiebuttons
       [\v!eerste  \v!sub\v!pagina,
        \v!vorige  \v!sub\v!pagina,
        \v!volgende\v!sub\v!pagina,
        \v!laatste \v!sub\v!pagina]%
   \fi}

\def\complexinteractiebalk[#1]%
  {\doifelse{#1}{\v!reset}%
    {\global\setbox\meterbox=\box\voidb@x}%
    {\bgroup
       \iflocation
         \checksubpages % goes wrong / loads \numberofpages too
         \getparameters[\??ib][#1]%
         \doif{\@@ibstatus}{\v!start}
           {\startinteractie
            \processaction
              [\@@ibvariant]
              [d=>{\getparameters[\??ib][\c!breedte=.5em,\c!hoogte=.5em,#1]},
               e=>{\getparameters[\??ib][\c!hoogte=.5em,\c!afstand=.2em,#1]},
               f=>{\getparameters[\??ib][\c!hoogte=.5em,\c!afstand=.2em,#1]}]%
            \doifelse{\@@ibsymbool}{\v!ja}
              {\balksymbooltrue}
              {\balksymboolfalse}%
            \getvalue{interactiebalk\@@ibvariant}%
            \stopinteractie}%
       \fi
     \egroup}}

\unexpanded\def\interactiebalk%
  {\complexorsimpleempty{interactiebalk}}

\def\stelinteractiebalkin%
  {\dodoubleargument\getparameters[\??ib]}

%I n=Profielen
%I c=\definieerprofiel,\startprofiel,\volgprofiel
%I
%I Er kunnen een gericht leesprofiel worden gedefinieerd.
%I Daartoe worden delen van de tekst gemerkt met commando:
%I
%I  \startprofiel[label,label,...]
%I  \stopprofiel
%I
%I Een profiel wordt vervolgens samengesteld uit gemerkte
%I teksten:
%I
%I   \definieerprofiel[naam,naam][label,label,...]
%I
%I en kan worden gevolgd met:
%I
%I   \volgprofiel{tekst}[naam]
%I
%I Dit laatste commando is te vergelijken met \naar, met dat
%I verschil dat naar een serie teksten wordt gesprongen.
%P
%I Profielen kunnen worden getest. In dat geval worden de
%I begin- en eindpunten in de tekst aangegeven. De labels
%I zijn actief, zodat snel heen en weer gesprongen kan worden.
%I
%I   \stelprofielenin[optie=]
%I
%I Hierbij kan optie de waarde 'test' hebben.

% Er wordt vooralsnog uitgegaan van een symmetrische
% start-stop situatie.

\def\c!profiel!! {profiel:}  % brrr
\def\c!versie!!  {versie:}

\def\dodefinieerprofiel[#1][#2]%
  {\iflocation
     \def\dododefinieerprofiel##1%
       {\def\dodododefinieerprofiel####1%
          {\doifdefinedelse{\c!profiel!!####1}%
             {\edef\!!stringa{\getvalue{\c!profiel!!####1}}%
              \setevalue{\c!profiel!!####1}{\!!stringa,##1}}%
             {\setevalue{\c!profiel!!####1}{##1}}}%
        \processcommalist[#2]\dodododefinieerprofiel}%
     \processcommalist[#1]\dododefinieerprofiel
   \fi}

\def\definieerprofiel%
  {\dodoubleargument\dodefinieerprofiel}

% Als met \getpar wordt gewerkt, dan moet \next worden toegepast.

\def\profilepage{}

\let\dosetprofilepage=\relax
\let\dogetprofilepage=\relax

\def\processprofile[#1]#2#3%
  {\iflocation
     \par % needed for pdftex
     \bgroup
     \dosetprofilepage
     \dogetprofilepage
     \def\processoneprofile##1##2%
       {\ExpandBothAfter\doifinsetelse{##2}{\processedprofiles}%
          {\doifsomething{##1}{(##1)}}%
          {\addtocommalist{##2}\processedprofiles
            ##1\relax
            #3{##2}{\hsize}{\profilepage}}}%
     \def\processedprofiles{}%
     \def\doprocessprofile##1%
       {\doifelse{\@@pfoptie}{\v!test}%
          {\goodbreak\blanko\nobreak\tt[\spatie
           #2 profiel\spatie ##1:\spatie
           \doifdefinedelse{\c!profiel!!##1}%
             {\def\dodoprocessprofile####1%
                {\processoneprofile
                   {\naar{####1}[\c!profiel!!####1]}%
                   {####1}%
                 \spatie}%
              \processcommacommand
                [\getvalue{\c!profiel!!##1}]\dodoprocessprofile}%
             {- }%
           ]\nobreak\blanko}%
          {\doifdefined{\c!profiel!!##1}%
             {\def\dodoprocessprofile####1%
                {\processoneprofile{}{####1}}%
              \processcommacommand
                 [\getvalue{\c!profiel!!##1}]\dodoprocessprofile}}}%
     \processcommalist[#1]\doprocessprofile
     \egroup
     \par % needed for pdftex
   \fi}

\def\startprofiel[#1]%
  {\iflocation
     \bgroup
     \addtocommalist{#1}\actualprofile
     \def\stopprofiel%
       {\processprofile[#1]\v!stop\doendofprofile
        \egroup}%
     \DoAfterFi\processprofile[#1]\v!start\dobeginofprofile
   \fi}

\let\stopprofiel=\relax

\def\dovolgprofiel#1[#2]%
  {\iflocation
     \hbox
       {\dostartgoto
          \data
            {\dolocationattributes{\??ia}{#1\presetgoto}}%
          \start
            \dostartgotoprofile
              {\number\buttonwidth}{\number\buttonheight}
              {#2}%
          \stop
            \dostopgotoprofile
        \dostopgoto}%
   \else
     {#1}%
   \fi}

\def\volgprofiel#1[#2]%
  {\iflocation
     \doif{\@@pfoptie}{\v!test}{\pagereference[\c!profiel!!#2]}%
     \dovolgprofiel{#1}[#2]%
   \fi}

\def\stelprofielenin%
  {\dodoubleargument\getparameters[\??pf]}

% Als er nog geen tekst op de pagina staat, dan heeft het
% profiel betrekking op het bovenstaande, dus soms een vorige
% pagina! Vreemd, omdat PDF paginagewijs werkt. Gelukkig
% biedt /page een oplossing. Echter: expansie van een
% \special kan niet worden uitgesteld, zodat alleen een
% two-pass een oplossing vormt. Het onderstaande kan komen
% te vervallen als Acrobat dit ondervangt. Het scheelt een
% pass en een lijst.
%
% Er kunnen eventueel twee lijsten worden gebruikt. Een voor
% het begin (start) en een voor het eind (stop). Nu staat
% alles in een lijst.

\definetwopasslist{\s!profile}

\newcounter\currentprofile

\def\dosetprofilepage%
  {\doglobal\increment\currentprofile
   \edef\docommando%
     {\writeutilitycommand%
        {\twopassentry%
           {\s!profile}%
           {\currentprofile}%
           {\noexpand\realfolio}}}%
   \docommando}

\def\dogetprofilepage%
  {\gettwopassdata{\s!profile}%
   \let\profilepage=\twopassdata}

%I n=Versies
%I c=\definieerversie,\stelversiesin,\startversie
%I c=\selecteerversie,\markeerversie,\volgversie
%I
%I Het mechanisme om wijzigingen in een tekst te markeren en
%I selecteren sluit aan op dat van profielen.
%I
%I Een deel van de tekst kan worden gemarkeerd met ofwel
%I
%I   \startversie[nr,nr,...]
%I   \stopversie
%I
%I of tussen  @+ ...... @-. Eventueel mogen achter @+ nummers
%I worden opgenomen: @+nr,nr,... ...... @-.
%I
%I Aldus gemerkte tekst kan in een afwijkende letter worden
%I weergegeven, bijvoorbeeld \ss. Daarbij wordt de laagste
%I nog te mrkeren versie opgegeven:
%I
%I   \stelversiesin[nummer=,letter=]
%I
%I Een versienummer mag punten (.) bevatten. Deze worden
%I voor het vergelijken niet meegenomen. Oppassen dus: 1.10
%I wordt 110 en 2.2 wordt 22 en moetD@2 zijn.
%P
%I Het is mogelijk alleen de 'recente' wijzigingen te
%I verwerken. Dit gaat in twee slagen:
%I
%I   \markeerversie        (eerste slag)
%I   \selecteerversie      (tweede slag)
%I
%I Er is gekozen voor het bewust markeren, omdat het
%I markeren gaat ten koste van de omvang van de hulpfile.
%P
%I Evenals profielen, kunnen ook versie worden gevolgd. Ook
%I hier dient eerst een definitie plaats te vinden:
%I
%I   \definieerversie[naam][nummer,nummer,nummer]
%I
%I waarna gebruik kan worden gemaakt van:
%I
%I   \volgversie{tekst}[naam]
%I
%I Er is ook een combinatie (doorsnede) mogelijk van
%I profielen en versies:
%I
%I   \volgprofielversie{tekst}[naam profiel][naam versie]

\newcounter\versionlevel
\newcounter\versionorder

\newif\ifrecentversion

\let\oldatcharacter=@

\def\minimumversion{0}
\def\actualversion{0}

\def\dostelversiesin[#1]%
  {\getparameters[\??ve][#1]
   \stripcharacter.\from\@@venummer\to\minimumversion
   \setversion}

\def\stelversiesin%
  {\dosingleargument\dostelversiesin}

\definetwopasslist{\s!versionbegin}
\definetwopasslist{\s!versionend}

\def\actualprofile{}

\def\doresetpageversion%
  {\edef\docommando%
     {\writeutilitycommand%
        {\twopassentry%
           {\s!versionend}%
           {\versionorder}%
           {\noexpand\realfolio}}}%
   \docommando}

\def\dosetpageversion#1%
  {\recentversiontrue
   \doglobal\increment\versionorder\relax
   \edef\docommando%
     {\writeutilitycommand%
        {\twopassentry%
           {\s!versionbegin}%
           {\versionorder}%
           {\noexpand\realfolio}}}%
   \docommando
   \let\resetpageversion=\doresetpageversion}

\def\recentcontributions{}

\def\checkrecentcontributions%
  {\gettwopassdata{\s!versionbegin}%
   \iftwopassdatafound
     \!!counta=\twopassdata\relax
     \gettwopassdata{\s!versionend}%
     \iftwopassdatafound
       \!!countb=\twopassdata\relax
       \doglobal\increment\versionorder\relax
       \writeutilitycommand%
         {\twopassentry%
            {\s!versionbegin}%
            {\versionorder}%
            {\the\!!counta}}%
       \writeutilitycommand%
         {\twopassentry%
            {\s!versionend}%
            {\versionorder}%
            {\the\!!countb}}%
       \for\teller=\!!counta\to\!!countb\step1\do%
         {\@EA\doglobal\@EA\addtocommalist\@EA{\teller}{\recentcontributions}}%
       \let\next=\checkrecentcontributions
     \else
       \let\next=\relax
     \fi
   \else
     \let\next=\relax
   \fi
   \next}

\def\docheckpageversion%
  {\ExpandBothAfter\doifinsetelse{\realfolio}{\recentcontributions}
     {\geselecteerdtrue}%
     {\geselecteerdfalse}}

\let\setpageversion   = \gobbleoneargument
\let\resetpageversion = \relax
\let\checkpageversion = \relax

\def\complexstartversie[#1]%
  {\bgroup
   \doifelse{\actualprofile}{}%
     {\startprofiel[#1]}%
     {\startprofiel[#1,\actualprofile]}%
   \def\docomplexstartversie##1%
     {\stripcharacter.\from##1\to\actualversion
      \ifnum\versionlevel>0\relax
        \ifnum\actualversion=0\relax
          \setpageversion\actualversion   % unknown version
        \else
          \ifnum\actualversion<\minimumversion\relax
            \relax                        % old version
          \else
            \setpageversion\actualversion % new version
          \fi
        \fi
      \fi}%
   \doglobal\increment\versionlevel\relax
   \doifelsenothing{#1}
     {\docomplexstartversie{0}}%
     {\processcommalist[#1]\docomplexstartversie}}

\def\startversie%
  {\complexorsimpleempty{startversie}}

\def\stopversie%
  {\stopprofiel
   \doglobal\decrement\versionlevel
   \ifnum\versionlevel<0\relax
     \showmessage{\m!versions}{1}{}%
   \else
     \resetpageversion
     \egroup
   \fi}

\bgroup
\catcode`@=\active
\gdef\setversion%
  {\catcode`@=\active        % we can't use \@@active here
   \long\def@##1##2 %
     {\ifx##1+%
        \startversie[##2]%
      \else\ifx##1-%
        \stopversie
      \else
        \oldatcharacter##1##2 %
      \fi\fi}}
\egroup

\def\markeerversie%
  {\showmessage{\m!versions}{2}{}%
   \let\setpageversion=\dosetpageversion
   \let\resetpageversion=\relax
   \let\checkpageversion=\relax}

\def\selecteerversie%
  {\checkrecentcontributions
   \showmessage{\m!versions}{3}{\recentcontributions}%
   \let\setpageversion=\gobbleoneargument
   \let\resetpageversion=\relax
   \let\checkpageversion=\docheckpageversion
   \setversion}

\def\dodefinieerversie[#1][#2]%
  {\setvalue{\c!versie!!#1}{#2}%
   \definieerprofiel[#1][#2]}

\def\definieerversie%
  {\dodoubleargument\dodefinieerversie}

\def\volgversie%
  {\volgprofiel}

\def\volgprofielversie#1[#2][#3]%
  {\def\docommando##1%
     {\definieerprofiel[#2#3][##1]}%
   \processcommacommand[\getvalue{\c!versie!!#3}]\docommando
   \volgprofiel#1[#2#3]}

\newcounter\currentpagetransition

\definecomplexorsimple\stelpaginaovergangenin

\def\simplestelpaginaovergangenin%
  {\complexstelpaginaovergangenin[\pagetransitions]}

\def\complexstelpaginaovergangenin[#1]%  
  {\doifsomething{#1}
     {\doifelse{#1}{\v!reset}
        {\let\setpagetransition=\relax}
        {\def\setpagetransition%
           {\doifelse{#1}{\v!willekeurig}
              {\expanded{\getcommalistsize[\pagetransitions]}% 
               \getrandomnumber{\currentpagetransition}{1}{\commalistsize}%
               \expanded{\getfromcommalist[\pagetransitions][\currentpagetransition]}}
              {\doglobal\increment\currentpagetransition
               \expanded{\getfromcommalist[#1][\currentpagetransition]}}%
            \ifx\commalistelement\empty
              \doglobal\newcounter\currentpagetransition 
              \setpagetransition
            \else\iflocation
              %\message{[\commalistelement]}\wait
              \expanded{\dosetpagetransition{\commalistelement}}% 
            \fi\fi}}}}

\prependtoks \setpagetransition \to \everyshipout

% temporary here

%D \startbuffer
%D \dorecurse{10}
%D   {\horizontalpositionbar
%D      \pos\recurselevel \min1 \max10
%D      \token\framed{\recurselevel}%
%D    \\}
%D
%D \hbox to 15em
%D   {\hss
%D    \dorecurse{10}
%D      {\verticalpositionbar\pos\recurselevel\min1\max10\token\blokje\\
%D       \hss}}
%D \stopbuffer

\def\horizontalpositionbar\pos#1\min#2\max#3\token#4\\%
  {\hbox to \hsize
     {\hskip\!!zeropoint\!!plus #1\!!fill
      \hskip\!!zeropoint\!!plus-#2\!!fill
      #4\relax
      \hskip\!!zeropoint\!!plus #3\!!fill
      \hskip\!!zeropoint\!!plus-#1\!!fill}}

\def\verticalpositionbar\pos#1\min#2\max#3\token#4\\%
  {\vbox to \vsize
     {\vskip\!!zeropoint\!!plus #1\!!fill
      \vskip\!!zeropoint\!!plus-#2\!!fill
      \hbox{#4}\relax
      \vskip\!!zeropoint\!!plus #3\!!fill
      \vskip\!!zeropoint\!!plus-#1\!!fill}}

\def\horizontalgrowingbar\pos#1\min#2\max#3\height#4\depth#5\\%
  {\hbox to \hsize
     {\scratchcounter=#1\relax
      \advance\scratchcounter by -#2\relax
      \advance\scratchcounter by 1\relax
      \leaders\vrule\hskip\!!zeropoint\!!plus \scratchcounter\!!fill
      \vrule\!!width\!!zeropoint\!!height#4\!!depth#5\relax
      \hskip\!!zeropoint\!!plus #3\!!fill
      \hskip\!!zeropoint\!!plus-#1\!!fill}}

\def\verticalgrowingbar\pos#1\min#2\max#3\width#4\\%
  {\vbox to \vsize
     {\scratchcounter=#1\relax
      \advance\scratchcounter by -#2\relax
      \advance\scratchcounter by 1\relax
      \leaders\hrule\vskip\!!zeropoint\!!plus\scratchcounter\!!fill
      \hrule\!!width#4\!!height\!!zeropoint\!!depth\!!zeropoint
      \vskip\!!zeropoint\!!plus #3\!!fill
      \vskip\!!zeropoint\!!plus-#1\!!fill}}

\definieerinteractiemenu
  [\v!rechts]
  [\v!rechts]
  [\c!voor=,
   \c!na=\vfil,
   \c!tussen=\blanko,
   \c!afstand=12pt,
   \c!links=\hss,
   \c!rechts=\hss,
   \c!breedte=\rechterrandbreedte,
   \c!hoogte=\v!ruim]

\definieerinteractiemenu
  [\v!links]
  [\v!links]
  [\c!voor=,
   \c!na=\vfil,
   \c!tussen=\blanko,
   \c!afstand=12pt,
   \c!links=\hss,
   \c!rechts=\hss,
   \c!breedte=\linkerrandbreedte,
   \c!hoogte=\v!ruim]

\definieerinteractiemenu
  [\v!onder]
  [\v!onder]
  [\c!voor=\vss,
   \c!na=\vss,
   \c!midden=\hfil,
   \c!afstand=12pt,
   \c!breedte=\v!passend,
   \c!hoogte=\v!ruim]

\definieerinteractiemenu
  [\v!boven]
  [\v!boven]
  [\c!voor=\vss,
   \c!na=\vss,
   \c!midden=\hfil,
   \c!afstand=12pt,
   \c!breedte=\v!passend,
   \c!hoogte=\v!ruim]

\stelinteractiemenuin
  [\v!links,\v!rechts,\v!boven,\v!onder]
  [\c!offset=.25em,
   \c!kader=\v!aan,
   \c!achtergrond=,
   \c!achtergrondkleur=,
   \c!achtergrondraster=\@@rsraster,
   \c!letter=\@@ialetter,
   \c!kleur=\@@iakleur,
   \c!status=\v!start,
   \c!zelfdepagina=\v!ja,
   \c!onbekendeverwijzing=\v!leeg,
   \c!bovenoffset=\!!zeropoint,
   \c!onderoffset=\!!zeropoint,
   \c!linkeroffset=\!!zeropoint,
   \c!rechteroffset=\!!zeropoint]

\stelbovenin [\v!tekst] [\c!middentekst={\interactiemenus[\v!boven]}]
\stelonderin [\v!tekst] [\c!middentekst={\interactiemenus[\v!onder]}]

\def\plaatsrechterrandblok {\interactiemenus[\v!rechts]}
\def\plaatslinkerrandblok  {\interactiemenus[\v!links]}

\stelinteractieschermin
  [\c!breedte=\printpapierbreedte,
   \c!hoogte=\printpapierhoogte,
   \c!rugoffset=\!!zeropoint,
   \c!kopoffset=\!!zeropoint,
   \c!rugwit=\rugwit,
   \c!kopwit=\kopwit,
   \c!optie=\c!min]

\stelexternefigurenin
  [\c!hokjes=\v!uit,
   \c!ymax=24,
   \c!xmax=]

\stelbuttonsin
  [\c!breedte=\v!passend,
   \c!hoogte=\v!ruim,
   \c!offset=0.25em,
   \c!kader=\v!aan,
   \c!achtergrond=,
   \c!achtergrondraster=\@@rsraster,
   \c!achtergrondkleur=,
   \c!letter=\@@ialetter,
   \c!kleur=\@@iakleur,
   \c!zelfdepagina=\v!ja,
   \c!onbekendeverwijzing=\v!leeg]

\stelinteractiebalkin
  [\c!status=\v!start,
   \c!variant=a,
   \c!symbool=\v!nee,
   \c!breedte=\rechterrandbreedte,
   \c!hoogte=\v!ruim,
   \c!diepte=\!!zeropoint,
   \c!afstand=1em,
   \c!stap=1,
   \c!kleur=\@@iakleur,
   \c!contrastkleur=\@@iacontrastkleur,
   \c!kader=\v!aan,
   \c!achtergrond=,
   \c!achtergrondraster=\@@rsraster,
   \c!achtergrondkleur=]

\stelsynchronisatiebalkin
  [\c!variant=\v!pagina,
   \c!breedte=\rechterrandbreedte,
   \c!letter=\@@ialetter,
   \c!kleur=\@@iakleur,
   \c!achtergrond=,
   \c!achtergrondraster=\@@rsraster,
   \c!achtergrondkleur=]

\stelsynchronisatiein
  [\c!status=\v!stop]

\stelversiesin
  [\c!nummer=1,
   \c!letter=\ss,
   \c!kleur=]

\stelprofielenin
  [\c!optie=]

\stelprogrammasin
  [\c!gebied=]

\stelpaginaovergangenin
  [\v!reset]

\protect

\endinput
