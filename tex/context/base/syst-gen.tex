%D \module
%D   [       file=syst-gen,
%D        version=1996.03.20,
%D          title=\CONTEXT\ System Macros,
%D       subtitle=General,
%D         author=Hans Hagen,
%D           date=\currentdate,
%D      copyright={PRAGMA / Hans Hagen \& Ton Otten}]
%C
%C This module is part of the \CONTEXT\ macro||package and is
%C therefore copyrighted by \PRAGMA. See mreadme.pdf for
%C details.

% nagaan : \ifinstringelse in syst-ext.tex
% do => p! dodo pp! dododo ppp!
% todo: \donetrue etc local maken

%D The following macros are responsible for the interaction
%D with \CONTEXT. These macros have proven their use. These
%D macros are optimized as far as possible within of course,
%D the know how of the author.
%D
%D In this module we also show some of the optimizations,
%D mainly because we don't want to forget them and start doing
%D things over and over again. If showing them has a learing
%D effect for others too, we've surved another purpose too.

%D \macros
%D   {abortinputifdefined}
%D
%D Because this module can be used in a different context, we
%D want to prevent it being loaded more than once. This can be
%D done using:
%D
%D \starttyping
%D \abortinputifdefined\command
%D \stoptyping
%D
%D where \type{\command} is a command defined in the module
%D to be loaded only once.
%D
%D \starttyping
%D \def\abortinputifdefined#1%
%D   {\ifx#1\undefined
%D      \let\next=\relax
%D    \else
%D      \let\next=\endinput
%D    \fi
%D    \next}
%D \stoptyping
%D
%D This macro can be speed up in terms of speed as well as
%D memory. Because this is a nice example of a bit strange
%D command (\type{\endinput}), we spend some more lines on this.
%D
%D If we perform such actions directly, we can say:
%D
%D \starttyping
%D \ifx\somecommand\undefined
%D   \let\next=\relax
%D \else
%D   \let\next=\endinput
%D \fi
%D \next
%D \stoptyping
%D
%D We need the \type{\next} because we need to end the
%D \type{\fi}. The efficient one is:
%D
%D \starttyping
%D \ifx\somecommand\undefined
%D \else
%D   \expandafter\endinput
%D \fi
%D \stoptyping
%D
%D Because \type{\endinput} comes into action after the current
%D line, we can also say:
%D
%D \starttyping
%D \ifx\somecommand\undefined \else \endinput \fi
%D \stoptyping
%D
%D When we define a macro, we tend to use a format which
%D shows as besat as can how things are done. \TEX\ however
%D stores the definitions as a sequence of tokens, so in fact
%D we can use a formatted definition:

\def\abortinputifdefined#1%
  {\ifx#1\undefined \else
     \endinput
   \fi}

%D which also works. Keep in mind that this is entirely due to
%D the fact that \type{\endinput} after the line, i.e. at the
%D end of the macro. We therefore can burry this primitive quite
%D deep in code.

%D And because this module implements \type{\writestatus}, we
%D just say:

\abortinputifdefined\writestatus

%D Normally we tell the users what module is being loaded.
%D However, the command that is needed for this is not yet
%D defined.
%D
%D \starttyping
%D \writestatus{laden}{Context Systeem Macro's (a)}
%D \stoptyping

%D The next few macros are needed in case this module is
%D used outside \CONTEXT.

\ifx\beginTEX\undefined
  \let\beginTEX\relax\let\endTEX\relax
  \long\def\beginETEX  #1\endETEX {}
  \long\def\beginOMEGA #1\endOMEGA{}
%   \long\def\onlyTEX  #1{#1}
%   \long\def\onlyETEX #1{}
%   \long\def\onlyOMEGA#1{}
\fi

%D \macros
%D   [protecting]
%D   {protect,unprotect}
%D
%D We can shield macros from users by using some special
%D characters in their names. Some characters that are normally
%D no letters and therefore often used are: \type{@}, \type{!}
%D and \type{?}. Before and after the definition of protected
%D macros, we have to change the \CATCODE\ of these characters.
%D This is done by \type{\unprotect} and \type{\protect}, for
%D instance:
%D
%D \starttyping
%D \unprotect
%D \def\!test{test}
%D \protect
%D \stoptyping
%D
%D The defined command \type{\!test} can of course only be
%D called upon when we are in the \type{\unprotect}'ed state,
%D otherwise \TEX\ reads \type{\!} and probably complains
%D loudly about not being in math mode.
%D
%D Both commands can be used nested, but only the \CATCODE\
%D of the outermost level is saved. We make use of
%D an auxilary macro \type{\doprotect} to prevent us from
%D conflicts with existing macro's \type{\protect}. When
%D nesting deeper than one level, the system shows the
%D protection level.

\newcount\protectionlevel

\ifx\protect\undefined
  \def\protect{\message{<too much protection>}}
\else
  % a simple version is already defined
\fi

\let\normalprotect\protect % only for latex

%D Although we don't need the \type{%} after commands that
%D don't take arguments, unless lines are obeyed, I decided
%D to put it there as a reminder. I only mention this once.

\chardef\protectionthreshold=10
%chardef\protectionthreshold=0 % for testing

% \def\unprotect
%   {\ifcase\protectionlevel
%      \edef\doprotectcharacters%
%        {\catcode`\noexpand @=\the\catcode`@
%         \catcode`\noexpand !=\the\catcode`!
%         \catcode`\noexpand ?=\the\catcode`? }%
%      \let\protect\doprotect
%    \fi
%    \catcode`@=11
%    \catcode`!=11
%    \catcode`?=11
%    \advance\protectionlevel 1
%    \ifnum\protectionlevel>\protectionthreshold
%      \message{<unprotect \the\protectionlevel>}%
%    \fi}
%
% \def\doprotect%
%   {\ifnum\protectionlevel=1
%      \doprotectcharacters
%      \let\doprotectcharacters\relax
%      \let\protect\normalprotect
%    \fi
%    \ifnum\protectionlevel>\protectionthreshold
%      \message{<protect \the\protectionlevel>}%
%    \fi
%    \advance\protectionlevel -1 }

\def\saveprotectedcharacters
  {\edef\doprotectcharacters
     {\catcode`\noexpand @\the\catcode`@
      \catcode`\noexpand !\the\catcode`!
      \catcode`\noexpand ?\the\catcode`? }}

\def\setprotectedcharacters
  {\catcode`@=11
   \catcode`!=11
   \catcode`?=11 }

\def\unprotect
  {\ifcase\protectionlevel
     \saveprotectedcharacters
     \let\protect\doprotect
   \fi
   \setprotectedcharacters
   \advance\protectionlevel 1
   \ifnum\protectionlevel>\protectionthreshold
     \reportunprotection
   \fi}

\def\doprotect
  {\ifcase\protectionlevel\or
     \doprotectcharacters
     \let\doprotectcharacters\relax
     \let\protect\normalprotect
   \fi
   \ifnum\protectionlevel>\protectionthreshold
     \reportprotection
   \fi
   \advance\protectionlevel -1 }

\def\reportunprotection   {\message{<unprotect       \protectionstate>}}
\def\reportprotection     {\message{<protect         \protectionstate>}}
\def\reportprotectionstate{\message{<protectionstate \protectionstate>}}

\def\protectionstate
  {\the\protectionlevel
   \ifcase\protectionthreshold
     :\space
     @=\the\catcode`@\space\space
     !=\the\catcode`!\space\space
     ?=\the\catcode`?%
   \fi}

\ifx\everyeof\undefined

  \let\checkprotection\relax

\else

  \def\checkprotection
    {\everyeof{\writestatus{protection}{state: \protectionstate}}}

\fi

%D Now it is defined, we can make use of this very useful
%D macro.

\unprotect

%D So, now we can redefine a previously defined macro as
%D follows:

\def\setprotectedcharacters
  {\catcode`@\@@letter
   \catcode`!\@@letter
   \catcode`?\@@letter}

%D \macros
%D   {@@escape,@@begingroup,@@endgroup,@@mathshift,@@alignment,
%D    @@endofline,@@parameter,@@superscript,@@subscript,
%D    @@ignore,@@space,@@letter,@@other,@@active,@@comment}
%D
%D In \CONTEXT\ we sometimes manipulate the \CATCODES\ of
%D certain characters. Because we are not that good at numbers,
%D we introduce some symbolic names.

\chardef\@@escape      =  0
\chardef\@@begingroup  =  1
\chardef\@@endgroup    =  2
\chardef\@@mathshift   =  3
\chardef\@@alignment   =  4
\chardef\@@endofline   =  5
\chardef\@@parameter   =  6
\chardef\@@superscript =  7
\chardef\@@subscript   =  8
\chardef\@@ignore      =  9
\chardef\@@space       = 10
\chardef\@@letter      = 11
\chardef\@@other       = 12   \chardef\other  = 12
\chardef\@@active      = 13   \chardef\active = 13
\chardef\@@comment     = 14

%D \macros
%D   {normalspace}
%D
%D We often need a space as defined in \PLAIN\ \TEX. Because
%D we cannot be sure of \type{\space} is redefined, we define:

\def\normalspace{ }

%D \macros
%D   {scratchcounter,
%D    scratchdimen,scratchskip,scratchmuskip,
%D    scratchbox,
%D    scratchtoks,
%D    ifdone}
%D
%D Because we often need counters on a temporary basis, we
%D define the \COUNTER\ \type{\scratchcounter}. This is a
%D real \COUNTER, and not a pseudo one, as we will meet
%D further on. We also define some other scratch registers.

\newcount  \scratchcounter
\newdimen  \scratchdimen
\newskip   \scratchskip
\newmuskip \scratchmuskip
\newbox    \scratchbox
\newtoks   \scratchtoks
\newif     \ifdone

%D \macros
%D   {ifCONTEXT}
%D
%D In the system and support modules we sometimes show examples
%D that make use of core commands. We can skip those parts of
%D the documentation when we use another macropackage. Of
%D course we default to false.

\newif \ifCONTEXT

%D \macros
%D   {!!count, !!toks, !!dimen, !!box,
%D    !!width, !!height, !!depth, !!string, !!done}
%D
%D We define some more \COUNTERS\ and \DIMENSIONS. We also
%D define some shortcuts to the local scatchregisters~0, 2, 4,
%D 6 and~8.

\newcount\!!counta \toksdef\!!toksa=0 \dimendef\!!dimena= 0 \chardef\!!boxa =0
\newcount\!!countb \toksdef\!!toksb=2 \dimendef\!!dimenb= 2 \chardef\!!boxb =2
\newcount\!!countc \toksdef\!!toksc=4 \dimendef\!!dimenc= 4 \chardef\!!boxc =4
\newcount\!!countd \toksdef\!!toksd=6 \dimendef\!!dimend= 6 \chardef\!!boxd =6
\newcount\!!counte \toksdef\!!tokse=8 \dimendef\!!dimene= 8 \chardef\!!boxe =8
\newcount\!!countf                    \dimendef\!!dimenf=10 %skipdef\!!skipa=0
                                      \dimendef\!!dimeng=12 %skipdef\!!skipb=2
                                      \dimendef\!!dimenh=14 %skipdef\!!skipc=4
                                      \dimendef\!!dimeni=16 %skipdef\!!skipd=6
                                      \dimendef\!!dimenj=18 %skipdef\!!skipe=8
                                      \dimendef\!!dimenk=20 %skipdef\!!skipf=10

\let\!!stringa=\empty \let\!!stringb=\empty \let\!!stringc=\empty
\let\!!stringd=\empty \let\!!stringe=\empty \let\!!stringf=\empty

\newdimen\!!widtha \newdimen\!!heighta \newdimen\!!deptha
\newdimen\!!widthb \newdimen\!!heightb \newdimen\!!depthb
\newdimen\!!widthc \newdimen\!!heightc \newdimen\!!depthc

\newif\if!!donea   \newif\if!!doneb    \newif\if!!donec
\newif\if!!doned   \newif\if!!donee    \newif\if!!donef

\ifx  \z@\undefined \else \let\zeropoint\z@   \fi
\ifx \@ne\undefined \else \let\plusone  \@ne  \fi
\ifx\m@ne\undefined \else \let\minusone \m@ne \fi

\ifx\zeropoint\undefined \csname newdimen\endcsname\zeropoint \fi % plain
\ifx\plusone  \undefined \chardef\plusone  =1                 \fi
\ifx\plustwo  \undefined \chardef\plustwo  =2                 \fi
\ifx\plusthree\undefined \chardef\plusthree=3                 \fi
\ifx\plusfour \undefined \chardef\plusfour =4                 \fi
\ifx\plusfive \undefined \chardef\plusfive =5                 \fi
\ifx\minusone \undefined \count\minusone \minusone=-1         \fi

\ifx\@m\undefined \else \let\plusthousand   \@m \fi
\ifx\@M\undefined \else \let\plusttenhousand\@M \fi

\ifx\plusten        \undefined \mathchardef\plusten        =   10 \fi
\ifx\plushundred    \undefined \mathchardef\plushundred    =  100 \fi
\ifx\plusthousand   \undefined \mathchardef\plusthousand   = 1000 \fi
\ifx\plustenthousand\undefined \mathchardef\plustenthousand=10000 \fi

\newcount\zerocount

%D \macros
%D   {s!,c!,e!,p!,v!,@@,??}
%D
%D To save memory, we use constants (sometimes called
%D variables). Redefining these constants can have desastrous
%D results.

\def\v!prefix! {v!}           \def\c!prefix! {c!}
\def\s!prefix! {s!}           \def\p!prefix! {p!}

\def\s!next    {next}         \def\s!default {default}
\def\s!dummy   {dummy}        \def\s!unknown {unknown}

\def\s!do      {do}           \def\s!dodo    {dodo}

\def\s!complex {complex}      \def\s!start   {start}
\def\s!simple  {simple}       \def\s!stop    {stop}

%D \macros
%D   {@EA,@EAEA,@EAEAEA,@EAEAEAEAEAEA,expanded,startexpanded}
%D
%D When in unprotected mode, to be entered with
%D \type{\unprotect}, one can use \type{\@EA} as equivalent
%D of \type{\expandafter}.

\let\@NX\noexpand
\let\@EA\expandafter

\def\@EAEA  {\expandafter\expandafter}
\def\@EAEAEA{\expandafter\expandafter\expandafter}

\def\@EAEAEAEAEAEA{\expandafter\@EAEAEA\expandafter}

%D Sometimes we pass macros as arguments to commands that
%D don't expand them before interpretation. Such commands can
%D be enclosed with \type{\expanded}, like:
%D
%D \starttyping
%D \expanded{\setupsomething[\alfa]}
%D \stoptyping
%D
%D Such situations occur for instance when \type{\alfa} is a
%D commalist or when data stored in macros is fed to index of
%D list commands. If needed, one should use \type{\noexpand}
%D inside the argument. Later on we will meet some more clever
%D alternatives to this command.

\long\def\@@expanded{} % always long; global (less restores)

\long\def\expanded#1%
  {\long\xdef\@@expanded{\noexpand#1}\@@expanded}

%D Beware, the next one has no \type {\noexpand} before its
%D argument.

\long\def\startexpanded#1\stopexpanded % see x-fo for example
  {\long\xdef\@@expanded{#1}\@@expanded}

%D \macros
%D   {safeexpanded,everysafeexpanded}
%D
%D In addition we provide:

\newtoks\everysafeexpanded

\long\def\safeexpanded#1% why the \noexpand
  {\begingroup
   \the\everysafeexpanded\long\xdef\@@expanded{\noexpand#1}%
   \endgroup
   \@@expanded}

\def\safeedef#1#2%
  {\begingroup
   \the\everysafeexpanded\long\xdef\@@expanded{\noexpand#2}%
   \endgroup
   \let#1\@@expanded}

\def\safexdef#1#2%
  {\begingroup
   \the\everysafeexpanded\long\xdef\@@expanded{\noexpand#2}%
   \endgroup
   \global\let#1\@@expanded}

%D You can append protective measures to the token register if
%D needed, as we will do later.

%D \macros
%D   {expandoneargafter,expandtwoargsafter}
%D
%D These two commands make macros more readable by hiding a
%D lot of \type {\expandafter}'s. They expand the arguments
%D after the first command.
%D
%D \starttyping
%D \expandoneargafter \command{\abc}
%D \expandtwoargsafter\command{\abc}{\def}
%D \stoptyping
%D
%D These commands expect the arguments to be macros.

\def\expandoneargafter   #1{\@EA#1\@EA}
\def\expandtwoargsafter#1#2{\@EA\@EA\@EA#1\@EA\@EA\@EA{\@EA#2\@EA}\@EA}

%D These two do a full expansion:

\def\fullexpandoneargafter   #1#2{\long\xdef\@@expanded{\noexpand#1{#2}}\@@expanded}
\def\fullexpandtwoargsafter#1#2#3{\long\xdef\@@expanded{\noexpand#1{#2}{#3}}\@@expanded}

%D \macros
%D   {gobbleoneargument,gobble...arguments}
%D
%D The next set of macros just do nothing, except that they
%D get rid of a number of arguments.

\long\def\gobbleoneargument    #1{}
\long\def\gobbletwoarguments   #1#2{}
\long\def\gobblethreearguments #1#2#3{}
\long\def\gobblefourarguments  #1#2#3#4{}
\long\def\gobblefivearguments  #1#2#3#4#5{}
\long\def\gobblesixarguments   #1#2#3#4#5#6{}
\long\def\gobblesevenarguments #1#2#3#4#5#6#7{}
\long\def\gobbleeightarguments #1#2#3#4#5#6#7#8{}
\long\def\gobbleninearguments  #1#2#3#4#5#6#7#8#9{}
\long\def\gobbletenarguments   #1{\gobbleninearguments}

%D \macros
%D   {doifnextcharelse}
%D
%D When we started using \TEX\ in the late eighties, our
%D first experiences with programming concerned a simple shell
%D around \LATEX. The commands probably use most at \PRAGMA,
%D are the itemizing ones. One of those few shell commands took
%D care of an optional argument, that enabled us to specify
%D what kind of item symbol we wanted. Without understanding
%D anything we were able to locate a \LATEX\ macro that could
%D be used to inspect the next character.
%D
%D It's this macro that the ancester of the next one presented
%D here. It executes one of two actions, dependant of the next
%D character. Disturbing spaces and line endings, which are
%D normally interpreted as spaces too, are skipped.
%D
%D \starttyping
%D \doifnextcharelse {karakter} {then ...} {else ...}
%D \stoptyping
%D
%D This macro differs from the original in the use of \type
%D {\localnext} because we don't want clashes with \type
%D {\next}.

\long\def\doifnextcharelse#1#2#3% #1 should not be {} !
  {\let\charactertoken=#1% = needed here
   \def\!!stringa{#2}%
   \def\!!stringb{#3}%
   \futurelet\nexttoken\inspectnextcharacter}

\def\inspectnextcharacter
  {\ifx\nexttoken\blankspace
     \@EA\reinspectnextcharacter
   \else\ifx\nexttoken\charactertoken
     \@EAEAEA\!!stringa
   \else
     \@EAEAEA\!!stringb
   \fi\fi}

%\let\endoflinetoken=^^M
%
%\long\def\reinspectaftercharacter#1%
%  {\futurelet\nexttoken\inspectnextcharacter}

%D This macro uses some auxiliary macros. Although we were able
%D to program quite complicated things, I only understood these
%D after rereading the \TEX book. The trick is in using a
%D command with a one character name. Such commands differ from
%D the longer ones in the fact that trailing spaces are {\em
%D not} skipped. This enables us to indirectly define a long
%D named macro that gobbles a space.
%D
%D In the first line we define \type{\blankspace}. Next we
%D make \type{\:} equivalent to \type{\reinspect...}. This
%D one||character command is expanded before the next
%D \type{\def} comes into action. This way the space after
%D \type{\:} becomes a delimiter of the longer named
%D \type{\reinspectnextcharacter}. The chain reaction is
%D visually compatible with the next sequence:
%D
%D \starttyping
%D \expandafter\def\reinspectnextcharacter %
%D   {\futurelet\nexttoken\inspectnextcharacter}
%D \stoptyping
%D
%D However complicated it may look, I'm still glad I stumbled
%D into this construction. Saving and restoring \type {\:} is
%D needed when we use \PPCHTEX\ in \LATEX.

\let\next\:

\def\:{\let\blankspace= }  \:

\def\:{\reinspectnextcharacter}

\expandafter\def\: {\futurelet\nexttoken\inspectnextcharacter}

\let\:\next

%D \macros
%D   {setvalue,setgvalue,setevalue,setxvalue,
%D    letvalue,
%D    getvalue,
%D    resetvalue}
%D
%D \TEX's primitive \type{\csname} can be used to construct
%D all kind of commands that cannot be defined with
%D \type{\def} and \type{\let}. Every macro programmer sooner
%D or later wants macros like these.
%D
%D \starttyping
%D \setvalue   {name}{...} = \def\name{...}
%D \setgvalue  {name}{...} = \gdef\name{...}
%D \setevalue  {name}{...} = \edef\name{...}
%D \setxvalue  {name}{...} = \xdef\name{...}
%D \letvalue   {name}=\... = \let\name=\...
%D \letgvalue  {name}=\... = \global\let\name=\...
%D \getvalue   {name}      = \name
%D \resetvalue {name}      = \def\name{}
%D \stoptyping
%D
%D As we will see, \CONTEXT\ uses these commands many times,
%D which is mainly due to its object oriented and parameter
%D driven character.

\def\setvalue     #1{\expandafter \def\csname#1\endcsname}
\def\setgvalue    #1{\expandafter\gdef\csname#1\endcsname}
\def\setevalue    #1{\expandafter\edef\csname#1\endcsname}
\def\setxvalue    #1{\expandafter\xdef\csname#1\endcsname}
\def\getvalue     #1{\csname#1\endcsname}
\def\letvalue     #1{\expandafter\let\csname#1\endcsname}
\def\letgvalue    #1{\global\expandafter\let\csname#1\endcsname}
\def\resetvalue   #1{\expandafter\let\csname#1\endcsname\empty}
\def\ignorevalue#1#2{\expandafter\let\csname#1\endcsname\empty}

%D \macros
%D   {globallet,glet}
%D
%D In \CONTEXT\ of May 2000 using \type {\globallet}
%D instead of the two tokens will save us some
%D $300\times4=1200$ bytes of format file on a 32~bit
%D system. So:

\def\globallet{\global\let}  \let\glet\globallet

%D \macros
%D   {donottest,unexpanded}
%D
%D When expansion of a macro gives problems, we can precede it
%D by \type{\donottest}. It seems that protection is one of the
%D burdens of developers of packages, so maybe that's why in
%D \ETEX\ protection is solved in a more robust way.
%D
%D Sometimes prefixing the macro with \type{\donottest} leads
%D to defining an auxiliary macro, like
%D
%D \starttyping
%D \def\dosomecommand {... ... ...}
%D \def\somecommand   {\donottest\dosomecommand}
%D \stoptyping
%D
%D This double definition can be made transparant by using
%D \type{\unexpanded}, as in:
%D
%D \starttyping
%D \unexpanded\def\somecommand{... ... ...}
%D \stoptyping
%D
%D The protection mechanism uses:

\beginTEX

\def\dontprocesstest#1{==}
\def\doprocesstest  #1{#1}

\let\donottest=\doprocesstest

\endTEX

\beginETEX \detokenize

\def\donottest#1{#1} % {\detokenize{#1}}

\endETEX

%D By the way, we use a placeholder because we don't want
%D interference when testing on empty strings. Using a
%D placeholder of 8~characters increases the processing time
%D of simple \type{\doifelse} tests by about 10 \%. When we
%D process the test, we have to remove the braces and
%D therefore explictly gobble \type{#1}.

%D \macros
%D   {honorunexpanded,forceunexpanded}
%D
%D The fact that many macros have the same prefix, could have
%D a negative impact on searching in the hash table. Because
%D some simple testing does not show differences, we just use:
%D
%D \starttyping
%D \def\unexpanded#1#2%
%D   {\@EA#1\@EA#2\@EA{\@EA\donottest\csname\s!do\string#2\endcsname}%
%D    \@EA#1\csname\s!do\string#2\endcsname}
%D \stoptyping
%D
%D Well, in fact we use the bit more versatile alternative. The
%D \type {\honorunexpanded} can be used to \type {\string}
%D the protected command, which by the way is seldom needed
%D in \CONTEXT.

\beginTEX

\def\dosetunexpanded#1#2%
  {\@EA#1\@EA{\@EA#2\@EA}%
     \@EA{\@EA\donottest\csname\s!do\@EA\string\csname#2\endcsname\endcsname}%
   \@EA#1\@EA{\@EA\s!do\@EA\string\csname#2\endcsname}}

\def\docomunexpanded#1#2%
  {\@EA#1\@EA#2\@EA{\@EA\donottest\csname\s!do\string#2\endcsname}%
   \@EA#1\csname\s!do\string#2\endcsname}

\def\unexpanded#1%
  {\def\dounexpanded
     {\ifx\next\bgroup
        \@EA\dosetunexpanded
      \else
        \@EA\docomunexpanded
      \fi#1}%
   \futurelet\next\dounexpanded}

\def\honorunexpanded% for writing to a file or message
  {\def\donottest##1{\expandafter\gobblethreearguments\string##1}}

\def\forceunexpanded% for preventing expansion in \xdef
  {\def\donottest##1%
     {\expandafter\noexpand\csname\expandafter\gobblefourarguments\string##1\endcsname}}

\def\resetunexpanded%
  {\let\donottest\doprocesstest}

\endTEX

\beginETEX \protected

\let \unexpanded      \normalprotected
\let \honorunexpanded \empty % \relax
\let \forceunexpanded \empty % \relax
\let \resetunexpanded \empty % \relax

\endETEX

%D This one accepts the more direct \type{\def} and cousins
%D as well as the \CONTEXT\ specific \type{\setvalue} ones.
%D
%D And so the definition in our example turns out to be:
%D
%D \starttyping
%D \def\csname do\somecommand\endcsname{... ... ...}
%D \def\somecommand{\donottest\csname do\somecommand\endcsname}
%D \stoptyping
%D
%D In which \type{do\somecommand} is hidden from the user and
%D cannot lead to confusion. It's still permitted to define
%D auxiliary macros like \type{\dosomecommand}.
%D
%D When we are going to use e-\TEX, we'll probably end up
%D redefining some commands, but we can probably keep the
%D \type{\unexpanded} ones unchanged.

%D \macros
%D   {doifundefined,doifdefined,
%D    doifundefinedelse,doifdefinedelse,
%D    doifalldefinedelse}
%D
%D The standard way of testing if a macro is defined is
%D comparing its meaning with another undefined one, usually
%D \type{\undefined}. To garantee correct working of the next
%D set of macros, \type{\undefined} may never be defined!
%D
%D \starttyping
%D \doifundefined      {string}    {...}
%D \doifdefined        {string}    {...}
%D \doifundefinedelse  {string}    {then ...} {else ...}
%D \doifdefinedelse    {string}    {then ...} {else ...}
%D \doifalldefinedelse {commalist} {then ...} {else ...}
%D \stoptyping
%D
%D Every macroname that \TEX\ builds gets an entry in the hash
%D table, which is of limited size. It is expected that e-\TeX\
%D will offer a less memory||consuming alternative.

%D Although it will probably never be a big problem, it is good
%D to be aware of the difference between testing on a macro
%D name to be build by using \type{\csname} and
%D \type{\endcsname} and testing the \type{\name} directly.
%D
%D \starttyping
%D \expandafter\ifx\csname NameA\endcsname\relax ... \else ... \fi
%D
%D \ifx\NameB\undefined ... \else ... \fi
%D \stoptyping
%D
%D I became aware of this when I mistakenly testen the first
%D one against \type{\undefined}. When \TEX\ build a name using
%D \type{\csname} it automatically sets it to \type{\relax},
%D which is definitely not the same as \type{\undefined}. The
%D quickest way to check these things is asking \TEX\ to show
%D the meaning of the names:
%D
%D \starttyping
%D \expandafter\show\csname NameA\endcsname
%D
%D \show\NameB
%D \stoptyping
%D
%D The main reason why this never will be a big problem is that
%D when one uses the \type{\csname} way, one probably has to do
%D with some macroname that always is dealt with that way.
%D Confusion can however arise when one applies both testing
%D methods to the same macroname. By the way, the assignment
%D of \type{\relax} obeys grouping.

%D The first one gets rid of \type{#1}, but still expands to
%D something and the second one expands to \type{#1}. Because
%D we accept arguments between \type{{}}, we have to get rid
%D of one level of braces.
%D
%D Our first implementation of \type{\ifundefined} was
%D straightforward and readable:
%D
%D \starttyping
%D \def\ifundefined#1%
%D   {\expandafter\ifx\csname#1\endcsname\relax}%
%D
%D \def\doifundefinedelse#1#2#3%
%D   {\let\donottest=\dontprocesstest
%D    \ifundefined{#1}%
%D      \let\donottest=\doprocesstest#2%
%D    \else
%D      \let\donottest=\doprocesstest#3%
%D    \fi}
%D
%D \def\doifdefinedelse#1#2#3%
%D   {\doifundefinedelse{#1}{#3}{#2}}
%D
%D \def\doifundefined#1#2%
%D   {\doifundefinedelse{#1}{#2}{}}
%D
%D \def\doifdefined#1#2%
%D   {\doifundefinedelse{#1}{}{#2}}
%D
%D \def\doifalldefinedelse#1#2#3%
%D   {\begingroup
%D    \donetrue
%D    \def\checkcommand##1%
%D      {\doifundefined{##1}{\donefalse}}%
%D    \processcommalist[#1]\checkcommand
%D    \ifdone
%D      \endgroup#2%
%D    \else
%D      \endgroup#3%
%D    \fi}
%D \stoptyping
%D
%D When this module was optimized, timing showed that the
%D next alternative can be upto twice as fast, especially when
%D longer arguments are used. Watch how we reach over the
%D \type {\else} and \type {\fi}: this way they cannot get
%D into the way (and we can avoid those \type {\next}
%D hacks); I started using this method after I did some
%D speed optimization tests in the xtag modules; the
%D efficiency of such hacks depends on the length of the
%D argument etc. etc.)

\beginTEX

\def\ifundefined#1%
  {\expandafter\ifx\csname#1\endcsname\relax}

\def\p!doifundefined#1%
  {\let\donottest\dontprocesstest
   \expandafter\ifx\csname#1\endcsname\relax}

\def\doifundefinedelse#1%
  {\p!doifundefined{#1}%
     \let\donottest\doprocesstest\@EA\firstoftwoarguments
   \else
     \let\donottest\doprocesstest\@EA\secondoftwoarguments
   \fi}

\def\doifdefinedelse#1%
  {\p!doifundefined{#1}%
     \let\donottest\doprocesstest\@EA\secondoftwoarguments
   \else
     \let\donottest\doprocesstest\@EA\firstoftwoarguments
   \fi}

\def\doifundefined#1%
  {\p!doifundefined{#1}%
     \let\donottest\doprocesstest\@EA\firstofoneargument
   \else
     \let\donottest\doprocesstest\@EA\gobbleoneargument
   \fi}

\def\doifdefined#1%
  {\p!doifundefined{#1}%
     \let\donottest\doprocesstest\@EA\gobbleoneargument
   \else
     \let\donottest\doprocesstest\@EA\firstofoneargument
   \fi}

\endTEX

\beginETEX \ifcsname

\def\ifundefined#1% ongelukkige naam
  {\unless\ifcsname#1\endcsname}

\def\p!doifundefined#1%
  {\edef\p!defined{#1}%
   \unless\ifcsname\detokenize\@EA{\p!defined}\endcsname}

\def\doifundefinedelse#1%
  {\edef\p!defined{#1}%
   \ifcsname\detokenize\@EA{\p!defined}\endcsname
     \expandafter\secondoftwoarguments
   \else
     \expandafter\firstoftwoarguments
   \fi}

\def\doifdefinedelse#1%
  {\edef\p!defined{#1}%
   \ifcsname\detokenize\@EA{\p!defined}\endcsname
     \expandafter\firstoftwoarguments
   \else
     \expandafter\secondoftwoarguments
   \fi}

\def\doifundefined#1%
  {\edef\p!defined{#1}%
   \ifcsname\detokenize\@EA{\p!defined}\endcsname
     \expandafter\gobbleoneargument
   \else
     \expandafter\firstofoneargument
   \fi}

\def\doifdefined#1%
  {\edef\p!defined{#1}%
   \ifcsname\detokenize\@EA{\p!defined}\endcsname
     \expandafter\firstofoneargument
   \else
     \expandafter\gobbleoneargument
   \fi}

\endETEX

%D \macros
%D   {letbeundefined}
%D
%D Testing for being undefined comes down to testing on \type
%D {\relax} when we use \type {\csname}, but when using \type
%D {\ifx}, we test on being \type {\undefined}! In \ETEX\ we
%D have \type {\ifcsname} and that way of testing on existance
%D is not the same as the one described here. Therefore we
%D introduce:

\beginTEX

\def\letbeundefined#1%
  {\expandafter\let\csname#1\endcsname\relax}

\endTEX

\beginETEX \undefined

\def\letbeundefined#1% potential stack buildup when used \global
  {\expandafter\let\csname#1\endcsname\undefined}

\endETEX

%D Beware, being \type {\undefined} in \ETEX\ means that the macro
%D {\em is} defined!

%D Before we start using this variant, we used another one,
%D which is even a bit faster. This one looked like:
%D
%D \starttyping
%D \def\p!doifundefined%
%D   {\begingroup
%D    \let\donottest=\dontprocesstest
%D    \ifundefined}
%D
%D \def\doifundefinedelse#1#2#3%
%D   {\p!doifundefined{#1}%
%D      \endgroup#2%
%D    \else
%D      \endgroup#3%
%D    \fi}
%D \stoptyping
%D
%D A even more previous version used \type{\bgroup} and
%D \type {\egroup}. In math mode however, \type{$1{x}2$} differs
%D from \type{$1x2$}. This can been seen when one compares the
%D output of:
%D
%D \starttyping
%D $\kern10pt\showthe\lastkern$
%D $\kern10pt{\showthe\lastkern}$
%D $\kern10pt\begingroup\showthe\lastkern\endgroup$
%D \stoptyping
%D
%D Also in math mode, one can better use \type {\begingroup}
%D and companion instead of \type {\bgroup}.
%D
%D When we were developing the scientific units module, we
%D encountered different behavior in text and math mode, which
%D was due to this grouping subtilities. We therefore decided
%D to use \type{\begingroup} instead of \type{\bgroup}.
%D Later, when we had optimized some macro's the grouped
%D solution turned out to be unsafe when typesetting this
%D documentation, especially when using \type{\globaldefs}.
%D
%D We still have to define \type{\doifalldefinedelse}. Watch
%D the use of grouping, which garantees local use of the
%D boolean \type{\ifdone}.

\beginTEX

\def\docheckonedefined#1%
  {\ifundefined{#1}%
     \donefalse
   \fi}

\def\doifalldefinedelse#1%
  {\begingroup
   \let\donottest\dontprocesstest
   \donetrue
   \processcommalist[#1]\docheckonedefined
   \ifdone
     \endgroup\let\donottest\doprocesstest
     \expandafter\firstoftwoarguments
   \else
     \endgroup\let\donottest\doprocesstest
     \expandafter\secondoftwoarguments
   \fi}

\endTEX

\beginETEX \ifcsname

\def\docheckonedefined#1%
  {\unless\ifcsname#1\endcsname
     \donefalse
   \fi}

\def\doifalldefinedelse#1%
  {\begingroup
   \donetrue \processcommalist[#1]\docheckonedefined
   \ifdone
     \endgroup\expandafter\firstoftwoarguments
   \else
     \endgroup\expandafter\secondoftwoarguments
   \fi}

\endETEX

%D \macros
%D   {doif,doifelse,doifnot,
%D    donottest}
%D
%D Programming in \TEX\ differs from programming in procedural
%D languages like \MODULA. This means that one --- well, let me
%D speek for myself --- tries to do the things in the well
%D known way. Therefore the next set of \type{\ifthenelse}
%D commands were between the first ones we needed. A few years
%D later, the opposite became true: when programming in
%D \MODULA, I sometimes miss handy things like grouping,
%D runtime redefinition, expansion etc. While \MODULA\ taught
%D me to structure, \TEX\ taught me to think recursive.
%D
%D \starttyping
%D \doif     {string1} {string2} {...}
%D \doifnot  {string1} {string2} {...}
%D \doifelse {string1} {string2} {then ...}{else ...}
%D \stoptyping
%D
%D When expansion gives problems, we can precede the
%D troublemaker with \type{\donottest}.
%D
%D This implementatie does not use the construction which is
%D more robust for nested conditionals.
%D
%D \starttyping
%D \ifx\!!stringa\!!stringb
%D   \def\next{#3}%
%D \else
%D   \def\next{#4}%
%D \fi
%D \next
%D \stoptyping
%D
%D In practice, this alternative is at least 20\% slower than
%D the alternative used here. The few cases in which we
%D really need the \type{\next} construction, often need some
%D other precautions and or adaptions too.

\beginTEX

% \long\def\doif#1#2#3%
%   {\let\donottest\dontprocesstest
%    \edef\!!stringa{#1}%
%    \edef\!!stringb{#2}%
%    \let\donottest\doprocesstest
%    \ifx\!!stringa\!!stringb
%      #3%
%    \fi}
%
% \long\def\doifnot#1#2#3%
%   {\let\donottest\dontprocesstest
%    \edef\!!stringa{#1}%
%    \edef\!!stringb{#2}%
%    \let\donottest\doprocesstest
%    \ifx\!!stringa\!!stringb
%    \else
%      #3%
%    \fi}
%
% \long\def\doifelse#1#2#3#4%
%   {\let\donottest\dontprocesstest
%    \edef\!!stringa{#1}%
%    \edef\!!stringb{#2}%
%    \let\donottest\doprocesstest
%    \ifx\!!stringa\!!stringb
%      #3%
%    \else
%      #4%
%    \fi}

%D Slightly faster on big arguments, as well as \type
%D {\next} avoiding:

\long\def\doif#1#2%
  {\let\donottest\dontprocesstest
   \edef\!!stringa{#1}%
   \edef\!!stringb{#2}%
   \let\donottest\doprocesstest
   \ifx\!!stringa\!!stringb
     \expandafter\firstofoneargument
   \else
     \expandafter\gobbleoneargument
   \fi}

\long\def\doifnot#1#2%
  {\let\donottest\dontprocesstest
   \edef\!!stringa{#1}%
   \edef\!!stringb{#2}%
   \let\donottest\doprocesstest
   \ifx\!!stringa\!!stringb
     \expandafter\gobbleoneargument
   \else
     \expandafter\firstofoneargument
   \fi}


\long\def\doifelse#1#2%
  {\let\donottest\dontprocesstest
   \edef\!!stringa{#1}%
   \edef\!!stringb{#2}%
   \let\donottest\doprocesstest
   \ifx\!!stringa\!!stringb
     \expandafter\firstoftwoarguments
   \else
     \expandafter\secondoftwoarguments
   \fi}

\endTEX

\beginETEX \protected

% \long\def\doif#1#2#3%
%   {\edef\!!stringa{#1}\edef\!!stringb{#2}%
%    \ifx\!!stringa\!!stringb#3\fi}
%
% \long\def\doifnot#1#2#3%
%   {\edef\!!stringa{#1}\edef\!!stringb{#2}%
%    \unless\ifx\!!stringa\!!stringb#3\fi}
%
% \long\def\doifelse#1#2#3#4%
%   {\edef\!!stringa{#1}\edef\!!stringb{#2}%
%    \ifx\!!stringa\!!stringb#3\else#4\fi}

%D Slightly faster on big arguments, as well as \type
%D {\next} avoiding:

\long\def\doif#1#2%
  {\edef\!!stringa{#1}\edef\!!stringb{#2}%
   \ifx\!!stringa\!!stringb
     \expandafter\firstofoneargument
   \else
     \expandafter\gobbleoneargument
   \fi}

\long\def\doifnot#1#2%
  {\edef\!!stringa{#1}\edef\!!stringb{#2}%
   \ifx\!!stringa\!!stringb
     \expandafter\gobbleoneargument
   \else
     \expandafter\firstofoneargument
   \fi}

\long\def\doifelse#1#2%
  {\edef\!!stringa{#1}\edef\!!stringb{#2}%
   \ifx\!!stringa\!!stringb
     \expandafter\firstoftwoarguments
   \else
     \expandafter\secondoftwoarguments
   \fi}

\endETEX

%D One could wonder why we don't follow the the same approach
%D as in \type{\doifdefined} c.s.\ and use \type{\begingroup}
%D and \type{\endgroup}. In this case, this alternative is
%D slower, which is probably due to the fact that more meanings
%D need to be restored.
%D
%D The in terms of memory more efficient alternative using a
%D auxiliary macro also proved to be slower, so we definitely
%D did not choose for:
%D
%D \starttyping
%D \def\p!doifelse#1#2%
%D   {\let\donottest=\dontprocesstest
%D    \edef\!!stringa{#1}%
%D    \edef\!!stringb{#2}%
%D    \let\donottest=\doprocesstest
%D    \ifx\!!stringa\!!stringb}
%D
%D \long\def\doif#1#2#3%
%D   {\p!doifelse{#1}{#2}#3\fi}
%D
%D \long\def\doifnot#1#2#3%
%D   {\p!doifelse{#1}{#2}\else#3\fi}
%D
%D \long\def\doifelse#1#2#3#4%
%D   {\p!doifelse{#1}{#2}#3\else#4\fi}
%D \stoptyping
%D
%D Optimizations like this are related of course to the
%D bottlenecks in \TEX. It seems that restoring saved meanings
%D and passing arguments takes some time.

%D \macros
%D   {doifempty,doifemptyelse,doifnotempty}
%D
%D We complete our set of conditionals with:
%D
%D \starttyping
%D \doifempty     {string} {...}
%D \doifnotempty  {string} {...}
%D \doifemptyelse {string} {then ...} {else ...}
%D \stoptyping
%D
%D This time, the string is not expanded.

\long\def\doifemptyelse#1%
  {\def\!!stringa{#1}%
   \ifx\!!stringa\empty
     \expandafter\firstoftwoarguments
   \else
     \expandafter\secondoftwoarguments
   \fi}

\long\def\doifempty#1%
  {\def\!!stringa{#1}%
   \ifx\!!stringa\empty
     \expandafter\firstofoneargument
   \else
     \expandafter\gobbleoneargument
   \fi}

\long\def\doifnotempty#1%
  {\def\!!stringa{#1}%
   \ifx\!!stringa\empty
     \expandafter\gobbleoneargument
   \else
     \expandafter\firstofoneargument
   \fi}

%D \macros
%D   {doifinset,doifnotinset,doifinsetelse}
%D
%D We can check if a string is present in a comma separated
%D set of strings. Depending on the result, some action is
%D taken.
%D
%D \starttyping
%D \doifinset     {string} {string,...} {...}
%D \doifnotinset  {string} {string,...} {...}
%D \doifinsetelse {string} {string,...} {then ...} {else ...}
%D \stoptyping
%D
%D The second argument is the comma separated set of strings.
%D
%D \starttyping
%D \long\def\doifinsetelse#1#2#3#4%
%D   {\doifelse{#1}{}
%D      {#4}
%D      {\donefalse
%D       \def\p!checkiteminset##1%
%D         {\doif{#1}{##1}
%D            {\donetrue
%D             \let\p!checkiteminset=\gobbleoneargument}}%
%D       \processcommalist[#2]\p!checkiteminset
%D       \ifdone
%D         #3%
%D       \else
%D         #4%
%D       \fi}}
%D
%D \long\def\doifinset#1#2#3%
%D   {\doifinsetelse{#1}{#2}{#3}{}}
%D
%D \long\def\doifnotinset#1#2#3%
%D   {\doifinsetelse{#1}{#2}{}{#3}}
%D \stoptyping
%D
%D Because this macro is called quite often we've spent some
%D time optimizing it. This time, the gain in speed is due to
%D (1)~defining an external auxiliary macro, (2)~not calling
%D any other macros and (3)~minimizing the passing of
%D arguments. The gain in speed is impressive.

% \def\p!dodocheckiteminset#1%
%   {\edef\!!stringb{#1}%
%    \ifx\!!stringa\!!stringb
%      \donetrue
%      \let\p!docheckiteminset\gobbleoneargument
%    \fi}
%
% \beginTEX
%
% \def\p!doifinsetelse#1#2%
%   {\let\donottest\dontprocesstest
%    \donefalse
%    \edef\!!stringa{#1}%
%    \ifx\!!stringa\empty
%    \else
%      \let\p!docheckiteminset\p!dodocheckiteminset
%      \processcommalist[#2]\p!docheckiteminset
%    \fi
%    \let\donottest\doprocesstest
%    \ifdone}
%
% \endTEX
%
% \beginETEX \protected
%
% \def\p!doifinsetelse#1#2%
%   {\donefalse
%    \edef\!!stringa{#1}%
%    \ifx\!!stringa\empty
%    \else
%      \let\p!docheckiteminset\p!dodocheckiteminset
%      \processcommalist[#2]\p!docheckiteminset
%    \fi
%    \ifdone}
%
% \endETEX

% then we had:
%
% \def\p!docheckiteminset#1%
%   {\edef\!!stringb{#1}%
%    \ifx\!!stringa\!!stringb
%      \donetrue
%      \expandafter\quitcommalist
%    \fi}
%
% \beginTEX
%
% \def\p!doifinsetelse#1#2%
%   {\let\donottest\dontprocesstest
%    \donefalse
%    \edef\!!stringa{#1}%
%    \ifx\!!stringa\empty
%    \else
%      \processcommalist[#2]\p!docheckiteminset
%    \fi
%    \let\donottest\doprocesstest
%    \ifdone}
%
% \endTEX
%
% % can be sped up with processnext...
%
% \beginETEX \protected
%
% \def\p!doifinsetelse#1#2%
%   {\donefalse
%    \edef\!!stringa{#1}%
%    \ifx\!!stringa\empty
%    \else
%      \processcommalist[#2]\p!docheckiteminset
%    \fi
%    \ifdone}
%
% \endETEX
%
% \long\def\doifinsetelse#1#2%
%   {\p!doifinsetelse{#1}{#2}%
%      \expandafter\firstoftwoarguments
%    \else
%      \expandafter\secondoftwoarguments
%    \fi}
%
% \long\def\doifinset#1#2%
%   {\p!doifinsetelse{#1}{#2}%
%      \expandafter\firstofoneargument
%    \else
%      \expandafter\gobbleoneargument
%    \fi}
%
% \long\def\doifnotinset#1#2%
%   {\p!doifinsetelse{#1}{#2}%
%      \expandafter\gobbleoneargument
%    \else
%      \expandafter\firstofoneargument
%     \fi}
%
% now we have

\def\p!docheckiteminset#1%
  {\edef\!!stringb{#1}%
   \ifx\!!stringa\!!stringb
     \donetrue
     \expandafter\quitcommalist
   \fi}

\beginTEX

\def\p!doifinsetelse#1#2#3#4%
  {\let\donottest\dontprocesstest
   \donefalse
   \edef\!!stringa{#3}%
   \ifx\!!stringa\empty
   \else
     \processcommalist[#4]\p!docheckiteminset
   \fi
   \let\donottest\doprocesstest
   \ifdone\expandafter#1\else\expandafter#2\fi}

\endTEX

\beginETEX \protected

\def\p!doifinsetelse#1#2#3#4%
  {\donefalse
   \edef\!!stringa{#3}%
   \ifx\!!stringa\empty
   \else
     \processcommalist[#4]\p!docheckiteminset
   \fi
   \ifdone\expandafter#1\else\expandafter#2\fi}

\endETEX

\long\def\doifinsetelse
  {\p!doifinsetelse\firstoftwoarguments\secondoftwoarguments}

\long\def\doifinset
  {\p!doifinsetelse\firstofoneargument\gobbleoneargument}

\long\def\doifnotinset
  {\p!doifinsetelse\gobbleoneargument\firstofoneargument}

%D \macros
%D   {doifcommon,doifnotcommon,doifcommonelse}
%D
%D Probably the most time consuming tests are those that test
%D for overlap in sets of strings.
%D
%D \starttyping
%D \doifcommon     {string,...} {string,...} {...}
%D \doifnotcommon  {string,...} {string,...} {...}
%D \doifcommonelse {string,...} {string,...} {then ...} {else ...}
%D \stoptyping
%D
%D We show the slower alternative first, because it shows us
%D how things are done.
%D
%D \starttyping
%D \long\def\doifcommonelse#1#2#3#4%
%D   {\donefalse
%D    \def\p!docommoncheck##1%
%D      {\def\p!dodocommoncheck####1%
%D         {\doif{####1}{##1}
%D            {\donetrue
%D             \def\commalistelement{##1}%
%D             \let\p!docommoncheck=\gobbleoneargument
%D             \let\p!dodocommoncheck=\gobbleoneargument}}%
%D       \processcommalist[#2]\p!dodocommoncheck}%
%D    \processcommalist[#1]\p!docommoncheck
%D    \ifdone
%D      #3%
%D    \else
%D      #4%
%D    \fi}
%D
%D \long\def\doifcommon#1#2#3%
%D   {\doifcommonelse{#1}{#2}{#3}{}}
%D
%D \long\def\doifnotcommon#1#2#3%
%D   {\doifcommonelse{#1}{#2}{}{#3}}
%D \stoptyping
%D
%D The processing time is shortened by getting the auxiliary
%D macro to the outermost level and using less \type{\edef}'s.
%D Sometimes it makes more sence to define local macro's not
%D only because this way we can be sure that they are not
%D redefined, but also because it shows the dependance. In
%D compiled languages, this is no problem at all. It can even
%D save us bytes and processing time. In interpreted languages
%D like \TEX\ it nearly always slows down processing.

% \def\p!dododocommoncheck#1%
%   {\edef\!!stringb{#1}%
%    \ifx\!!stringa\!!stringb
%      \donetrue
%      \let\p!docommoncheck\gobbleoneargument
%      \let\p!dodocommoncheck\gobbleoneargument
%    \fi}
%
% \beginTEX
%
% \def\p!doifcommonelse#1#2%
%   {\donefalse
%    \let\donottest\dontprocesstest
%    \let\p!dodocommoncheck\p!dododocommoncheck
%    \def\p!docommoncheck##1%
%      {\edef\!!stringa{##1}%
%       \def\commalistelement{##1}%
%       \processcommalist[#2]\p!dodocommoncheck}%
%    \processcommalist[#1]\p!docommoncheck
%    \let\donottest\doprocesstest
%    \ifdone}
%
% \endTEX
%
% \beginETEX \protected
%
% \def\p!doifcommonelse#1#2%
%   {\donefalse
%    \let\p!dodocommoncheck\p!dododocommoncheck
%    \def\p!docommoncheck##1%
%      {\edef\!!stringa{##1}%
%       \def\commalistelement{##1}%
%       \processcommalist[#2]\p!dodocommoncheck}%
%    \processcommalist[#1]\p!docommoncheck
%    \ifdone}
%
% \endETEX

% \def\p!dodocommoncheck#1%
%   {\edef\!!stringb{#1}%
%    \ifx\!!stringa\!!stringb
%      \donetrue
%      \expandafter\quitprevcommalist
%    \fi}
%
% \beginTEX
%
% \def\p!doifcommonelse#1#2%
%   {\donefalse
%    \let\donottest\dontprocesstest
%    \def\p!docommoncheck##1%
%      {\edef\!!stringa{##1}%
%       \def\commalistelement{##1}% no let to stringa
%       \processcommalist[#2]\p!dodocommoncheck}%
%    \processcommalist[#1]\p!docommoncheck
%    \let\donottest\doprocesstest
%    \ifdone}
%
% \endTEX
%
% \beginETEX \protected
%
% \def\p!doifcommonelse#1#2%
%   {\donefalse
%    \def\p!docommoncheck##1%
%      {\edef\!!stringa{##1}%
%       \def\commalistelement{##1}%
%       \processcommalist[#2]\p!dodocommoncheck}%
%    \processcommalist[#1]\p!docommoncheck
%    \ifdone}
%
% \endETEX
%
% \long\def\doifcommonelse#1#2%  % #3#4%
%   {\p!doifcommonelse{#1}{#2}%  % #3\else#4\fi}
%      \expandafter\firstoftwoarguments
%    \else
%      \expandafter\secondoftwoarguments
%    \fi}
%
% \long\def\doifcommon#1#2%
%   {\p!doifcommonelse{#1}{#2}%
%      \expandafter\firstofoneargument
%    \else
%      \expandafter\gobbleoneargument
%    \fi}
%
% \long\def\doifnotcommon#1#2%
%   {\p!doifcommonelse{#1}{#2}%
%      \expandafter\gobbleoneargument
%    \else
%      \expandafter\firstofoneargument
%    \fi}

\def\p!dodocommoncheck#1%
  {\edef\!!stringb{#1}%
   \ifx\!!stringa\!!stringb
     \donetrue
     \expandafter\quitprevcommalist
   \fi}

\beginTEX

\def\p!doifcommonelse#1#2#3#4%
  {\donefalse
   \let\donottest\dontprocesstest
   \def\p!docommoncheck##1%
     {\edef\!!stringa{##1}%
      \def\commalistelement{##1}% no let to stringa
      \processcommalist[#4]\p!dodocommoncheck}%
   \processcommalist[#3]\p!docommoncheck
   \let\donottest\doprocesstest
   \ifdone\expandafter#1\else\expandafter#2\fi}

\endTEX

\beginETEX \protected

\def\p!doifcommonelse#1#2#3#4%
  {\donefalse
   \def\p!docommoncheck##1%
     {\edef\!!stringa{##1}%
      \def\commalistelement{##1}%
      \processcommalist[#4]\p!dodocommoncheck}%
   \processcommalist[#3]\p!docommoncheck
   \ifdone\expandafter#1\else\expandafter#2\fi}

\endETEX

\def\doifcommonelse
  {\p!doifcommonelse\firstoftwoarguments\secondoftwoarguments}

\def\doifcommon
  {\p!doifcommonelse\firstofoneargument \gobbleoneargument}

\def\doifnotcommon
  {\p!doifcommonelse\gobbleoneargument  \firstofoneargument}

%D \macros
%D   {processcommalist,processcommacommand,quitcommalist,
%D    processcommalistwithparameters}
%D
%D We've already seen some macros that take care of comma
%D separated lists. Such list can be processed with
%D
%D \starttyping
%D \processcommalist[string,string,...]\commando
%D \stoptyping
%D
%D The user supplied command \type{\commando} receives one
%D argument: the string. This command permits nesting and
%D spaces after commas are skipped. Empty sets are no problem.
%D
%D \startbuffer
%D \def\dosomething#1{(#1)}
%D
%D 1: \processcommalist [\hbox{$a,b,c,d,e,f$}] \dosomething \par
%D 2: \processcommalist [{a,b,c,d,e,f}]        \dosomething \par
%D 3: \processcommalist [{a,b,c},d,e,f]        \dosomething \par
%D 4: \processcommalist [a,b,{c,d,e},f]        \dosomething \par
%D 5: \processcommalist [a{b,c},d,e,f]         \dosomething \par
%D 6: \processcommalist [{a,b}c,d,e,f]         \dosomething \par
%D 7: \processcommalist []                     \dosomething \par
%D 8: \processcommalist [{[}]                  \dosomething \par
%D \stopbuffer
%D
%D \typebuffer
%D
%D Before we show the result, we present the macro's:

\newcount\commalevel

\def\dododoprocesscommaitem
  {\csname\s!next\the\commalevel\endcsname}

%\def\dodoprocesscommaitem%
%  {\ifx\nexttoken\blankspace
%     \let\nextcommaitem\redoprocesscommaitem
%   \else\ifx\nexttoken]%
%     \let\nextcommaitem\gobbleoneargument
%   \else
%     \let\nextcommaitem\dododoprocesscommaitem
%   \fi\fi
%   \nextcommaitem}
%
% faster ?

\def\dodoprocesscommaitem
  {\ifx\nexttoken\blankspace
     \@EA\redoprocesscommaitem
   \else\ifx\nexttoken]%
     \@EAEAEA\gobbleoneargument
   \else
     \@EAEAEA\dododoprocesscommaitem
   \fi\fi}

\def\doprocesscommaitem
  {\futurelet\nexttoken\dodoprocesscommaitem}

%D Empty arguments are not processed. Empty items (\type{,,})
%D however are treated. We have to check for the special case
%D \type{[{a,b,c}]}.
%D
%D \starttyping
%D \def\processcommalist[%
%D   {\futurelet\nexttoken\docheckcommaitem}
%D
%D \def\docheckcommaitem%
%D   {\ifx\nexttoken]%
%D      \let\nextcommaitem\gobbletwoarguments
%D    \else\ifx\nexttoken\bgroup
%D      \let\nextcommaitem\doprocesscommalistA
%D    \else
%D      \let\nextcommaitem\doprocesscommalistB
%D    \fi\fi
%D    \nextcommaitem}
%D
%D \def\doprocesscommalistA#1#2]#3%
%D   {\global\advance\commalevel 1
%D    \long\expandafter\def\csname\s!next\the\commalevel\endcsname##1,%
%D      {#3{##1}\doprocesscommaitem}%
%D    \doprocesscommaitem{#1}#2,]\relax
%D    \global\advance\commalevel -1 }
%D
%D \def\doprocesscommalistB#1]#2%
%D   {\global\advance\commalevel 1
%D    \long\expandafter\def\csname\s!next\the\commalevel\endcsname##1,%
%D      {#2{##1}\doprocesscommaitem}%
%D    \doprocesscommaitem#1,]\relax
%D    \global\advance\commalevel -1 }
%D \stoptyping
%D
%D However, this is not a the most straightforward solution!
%D We can misuse one of \TEX's hidden features, and prepend
%D and remove a \type {\relax}. By the way, although it
%D involves less testing, this cleaner alternative is not
%D faster.

\def\processcommalist[%
  {\futurelet\nexttoken\docheckcommaitem}

\def\docheckcommaitem
  {\ifx\nexttoken]%
     \expandafter\gobblethreearguments
   \else
     \expandafter\doprocesscommalist
   \fi
   \relax} % this one preserved the next {}

\def\doprocesscommalist#1]#2%
  {\global\advance\commalevel \plusone
   \long\expandafter\def\csname\s!next\the\commalevel\endcsname##1,%
     {#2{##1}\doprocesscommaitem}%
   \@EA\dodoprocesscommaitem\gobbleoneargument#1,]\relax
   \global\advance\commalevel \minusone }

%D One way of quitting a commalist halfway is:

\def\quitcommalist
  {\begingroup\let\doprocesscommaitem\doquitcommalist}

\def\doquitcommalist#1]%
  {\endgroup}

\def\quitprevcommalist
  {\begingroup\let\doprocesscommaitem\doquitprevcommalist}

\def\doquitprevcommalist#1]%
  {\let\doprocesscommaitem\doquitcommalist}

%D The hack we used for checking the next character
%D \type {\doifnextcharelse} is also used here.

\def\:{\redoprocesscommaitem}

\expandafter\def\: {\futurelet\nexttoken\dodoprocesscommaitem}

%D The previous examples lead to:
%D
%D \getbuffer

%D When a list is saved in a macro, we can use a construction
%D like:
%D
%D \starttyping
%D \expandafter\processcommalist\expandafter[\list]\command
%D \stoptyping
%D
%D Such solutions suit most situations, but we wanted a bit
%D more.
%D
%D \starttyping
%D \processcommacommand[string,\stringset,string]\commando
%D \stoptyping
%D
%D where \type{\stringset} is a predefined set, like:
%D
%D \starttyping
%D \def\first{aap,noot,mies}
%D \def\second{laatste}
%D
%D \processcommacommand[\first]\message
%D \processcommacommand[\first,second,third]\message
%D \processcommacommand[\first,between,\second]\message
%D \stoptyping
%D
%D Commands that are part of the list are expanded, so the
%D use of this macro has its limits.

\def\processcommacommand[#1]%
  {\expanded{\processcommalist[#1]}}

%D The argument to \type{\command} is not delimited. Because
%D we often use \type{[]} as delimiters, we also have:
%D
%D \starttyping
%D \processcommalistwithparameters[string,string,...]\command
%D \stoptyping
%D
%D where \type{\command} looks like:
%D
%D \starttyping
%D \def\command[#1]{... #1 ...}
%D \stoptyping

\def\processcommalistwithparameters[#1]#2%
  {\def\docommand##1{#2[##1]}%
   \processcommalist[#1]\docommand}

%D \macros
%D   {processaction,
%D    processfirstactioninset,
%D    processallactionsinset}
%D
%D \CONTEXT\ makes extensive use of a sort of case or switch
%D command. Depending of the presence of one or more provided
%D items, some actions is taken. These macros can be nested
%D without problems.
%D
%D \starttyping
%D \processaction           [x]     [a=>\a,b=>\b,c=>\c]
%D \processfirstactioninset [x,y,z] [a=>\a,b=>\b,c=>\c]
%D \processallactionsinset  [x,y,z] [a=>\a,b=>\b,c=>\c]
%D \stoptyping
%D
%D We can supply both a \type{default} action and an action
%D to be undertaken when an \type{unknown} value is met:
%D
%D \starttyping
%D \processallactionsinset
%D   [x,y,z]
%D   [      a=>\a,
%D          b=>\b,
%D          c=>\c,
%D    default=>\default,
%D    unknown=>\unknown{... \commalistelement ...}]
%D \stoptyping
%D
%D When \type{#1} is empty, this macro scans list \type{#2} for
%D the keyword \type{default} and executed the related action
%D if present. When \type{#1} is non empty and not in the list,
%D the action related to \type{unknown} is executed. Both
%D keywords must be at the end of list \type{#2}. Afterwards,
%D the actually found keyword is available in
%D \type{\commalistelement}. An advanced example of the use of
%D this macro can be found in \PPCHTEX, where we completely
%D rely on \TEX\ for interpreting user supplied keywords like
%D \type{SB}, \type{SB1..6}, \type{SB125} etc.
%D
%D Even a quick glance at the macros below show some overlap,
%D which means that more efficient alternatives are possible.
%D Because these macro's are very sensitive to subtle changes,
%D we've decided to present the readable originals first
%D Maybe these these macros look complicated, but this is a
%D direct result of the support of nesting. Protection is only
%D applied in \type{\processaction}.
%D
%D \starttyping
%D \newcount\processlevel
%D
%D \def\processaction[#1]#2[#3]%
%D   {\doifelse{#1}{}
%D      {\def\p!compareprocessaction[##1=>##2]%
%D         {\edef\!!stringa{##1}%
%D          \ifx\!!stringa\s!default
%D            \def\commalistelement{#1}%
%D            ##2%
%D          \fi}}
%D      {\let\donottest=\dontprocesstest
%D       \edef\!!stringb{#1}%
%D       \let\donottest=\doprocesstest
%D       \def\p!compareprocessaction[##1=>##2]%
%D         {\edef\!!stringa{##1}%
%D          \ifx\!!stringa\!!stringb
%D            \def\commalistelement{#1}%
%D            ##2%
%D            \let\p!doprocessaction=\gobbleoneargument
%D          \else\ifx\!!stringa\s!unknown
%D            \def\commalistelement{#1}%
%D            ##2%
%D          \fi\fi}}%
%D    \def\p!doprocessaction##1%
%D      {\p!compareprocessaction[##1]}%
%D    \processcommalist[#3]\p!doprocessaction}
%D
%D \def\processfirstactioninset[#1]#2[#3]%
%D   {\doifelse{#1}{}
%D      {\processaction[][#3]}
%D      {\def\p!compareprocessaction[##1=>##2][##3]%
%D         {\edef\!!stringa{##1}%
%D          \edef\!!stringb{##3}%
%D          \ifx\!!stringa\!!stringb
%D            \def\commalistelement{##3}%
%D            ##2%
%D            \let\p!doprocessaction=\gobbleoneargument
%D            \let\p!dodoprocessaction=\gobbleoneargument
%D          \else\ifx\!!stringa\s!unknown
%D            \def\commalistelement{##3}%
%D            ##2%
%D          \fi\fi}%
%D       \def\p!doprocessaction##1%
%D         {\def\p!dodoprocessaction####1%
%D            {\p!compareprocessaction[####1][##1]}%
%D          \processcommalist[#3]\p!dodoprocessaction}%
%D       \processcommalist[#1]\p!doprocessaction}}
%D
%D \def\processallactionsinset[#1]#2[#3]%
%D   {\doifelse{#1}{}
%D      {\processaction[][#3]}
%D      {\advance\processlevel by 1
%D       \def\p!compareprocessaction[##1=>##2][##3]%
%D         {\edef\!!stringa{##1}%
%D          \edef\!!stringb{##3}%
%D          \ifx\!!stringa\!!stringb
%D            \def\commalistelement{##3}%
%D            ##2%
%D            \let\p!dodoprocessaction=\gobbleoneargument
%D          \else\ifx\!!stringa\s!unknown
%D            \def\commalistelement{##3}%
%D            ##2%
%D          \fi\fi}%
%D       \setvalue{\s!do\the\processlevel}##1%
%D         {\def\p!dodoprocessaction####1%
%D            {\p!compareprocessaction[####1][##1]}%
%D          \processcommalist[#3]\p!dodoprocessaction}%
%D       \processcommalist[#1]{\getvalue{\s!do\the\processlevel}}%
%D       \advance\processlevel by -1 }}
%D \stoptyping
%D
%D The gain of speed in the (again) next implementation is
%D around 20\%, depending on the application.

\newcount\processlevel

\def\p!compareprocessactionA[#1=>#2][#3]%
  {\edef\!!stringb{#1}%
   \ifx\!!stringb\s!default
     \let\commalistelement\empty
     #2%
   \fi}

% \def\p!compareprocessactionB[#1=>#2][#3]%
%   {\expandedaction\!!stringb{#1}%
%    \ifx\!!stringa\!!stringb
%      \def\commalistelement{#3}%
%      #2%
%      \let\p!doprocessaction\gobbleoneargument
%    \else
%      \edef\!!stringb{#1}%
%      \ifx\!!stringb\s!unknown
%        \def\commalistelement{#3}% beware of loops
%        #2%
%      \fi
%    \fi}

% met \quitcommalist tot meer dan 25\% sneller

\def\p!compareprocessactionB[#1=>#2][#3]%
  {\expandedaction\!!stringb{#1}%
   \ifx\!!stringa\!!stringb
     \def\commalistelement{#3}%
     #2%
     \expandafter\quitcommalist
   \else
     \edef\!!stringb{#1}%
     \ifx\!!stringb\s!unknown
       \def\commalistelement{#3}% beware of loops
       #2%
     \fi
   \fi}

\beginTEX

\def\processaction[#1]#2[#3]%
  {\let\donottest\dontprocesstest
   \expandedaction\!!stringa{#1}%
   \let\donottest\doprocesstest
   \ifx\!!stringa\empty
     \let\p!compareprocessaction\p!compareprocessactionA
   \else
     \let\p!compareprocessaction\p!compareprocessactionB
   \fi
   \def\p!doprocessaction##1%
     {\p!compareprocessaction[##1][#1]}%
   \processcommalist[#3]\p!doprocessaction
   \expandactions}

\endTEX

\beginETEX \protected

\def\processaction[#1]#2[#3]% faster version follows
  {\expandedaction\!!stringa{#1}%
   \ifx\!!stringa\empty
     \let\p!compareprocessaction\p!compareprocessactionA
   \else
     \let\p!compareprocessaction\p!compareprocessactionB
   \fi
   \def\p!doprocessaction##1%
     {\p!compareprocessaction[##1][#1]}%
   \processcommalist[#3]\p!doprocessaction
   \expandactions}

\endETEX

% \def\p!compareprocessactionC[#1=>#2][#3]%
%   {\expandedaction\!!stringa{#1}%
%    \expandedaction\!!stringb{#3}%
%    \ifx\!!stringa\!!stringb
%      \def\commalistelement{#3}%
%      #2%
%      \let\p!doprocessaction\gobbleoneargument
%      \let\p!dodoprocessaction\gobbleoneargument
%    \else
%      \edef\!!stringa{#1}%
%      \ifx\!!stringa\s!unknown
%        \def\commalistelement{#3}%
%        #2%
%      \fi
%    \fi}

\def\p!compareprocessactionC[#1=>#2][#3]%
  {\expandedaction\!!stringa{#1}%
   \expandedaction\!!stringb{#3}%
   \ifx\!!stringa\!!stringb
     \def\commalistelement{#3}%
     #2%
     \expandafter\quitprevcommalist
   \else
     \edef\!!stringa{#1}%
     \ifx\!!stringa\s!unknown
       \def\commalistelement{#3}%
       #2%
     \fi
   \fi}

\def\processfirstactioninset[#1]#2[#3]% faster version follows
  {\expandedaction\!!stringa{#1}%
   \ifx\!!stringa\empty
     \processaction[][#3]%
   \else
     \def\p!doprocessaction##1%
       {\def\p!dodoprocessaction####1%
          {\p!compareprocessactionC[####1][##1]}%
        \processcommalist[#3]\p!dodoprocessaction}%
     \processcommalist[#1]\p!doprocessaction
   \fi
   \expandactions}

% \def\p!compareprocessactionD[#1=>#2][#3]%
%   {\expandedaction\!!stringa{#1}%
%    \expandedaction\!!stringb{#3}%
%    \ifx\!!stringa\!!stringb
%      \def\commalistelement{#3}%
%      #2%
%      \let\p!dodoprocessaction\gobbleoneargument
%    \else
%      \edef\!!stringa{#1}%
%      \ifx\!!stringa\s!unknown
%        \def\commalistelement{#3}%
%        #2%
%      \fi
%    \fi}

\def\p!compareprocessactionD[#1=>#2][#3]%
  {\expandedaction\!!stringa{#1}%
   \expandedaction\!!stringb{#3}%
   \ifx\!!stringa\!!stringb
     \def\commalistelement{#3}%
     #2%
     \expandafter\quitcommalist
   \else
     \edef\!!stringa{#1}%
     \ifx\!!stringa\s!unknown
       \def\commalistelement{#3}%
       #2%
     \fi
   \fi}

\def\doprocessallactionsinset
  {\csname\s!do\the\processlevel\endcsname}

\def\processallactionsinset[#1]#2[#3]% faster version follows
  {\expandedaction\!!stringa{#1}%
   \ifx\!!stringa\empty
     \processaction[][#3]%
   \else
     \advance\processlevel \plusone
     \expandafter\def\csname\s!do\the\processlevel\endcsname##1%
       {\def\p!dodoprocessaction####1%
          {\p!compareprocessactionD[####1][##1]}%
        \processcommalist[#3]\p!dodoprocessaction}%
     \processcommalist[#1]\doprocessallactionsinset
     \advance\processlevel \minusone
   \fi
   \expandactions}

%D We can speed up these macros a bit when we use a dedicated
%D commalist processor, one that avoids passing the (often)
%D big action list.

\beginTEX

\def\processaction[#1]#2[%
  {\let\donottest\dontprocesstest
   \expandedaction\!!stringa{#1}%
   \let\donottest\doprocesstest
   \ifx\!!stringa\empty
     \let\p!compareprocessaction\p!compareprocessactionA
   \else
     \let\p!compareprocessaction\p!compareprocessactionB
   \fi
   \def\p!doprocessaction##1%
     {\p!compareprocessaction[##1][#1]}%
   \processnextcommalist\relax\expandactions\p!doprocessaction[}

\endTEX

\beginETEX

\def\processaction[#1]#2[%
  {\expandedaction\!!stringa{#1}%
   \ifx\!!stringa\empty
     \let\p!compareprocessaction\p!compareprocessactionA
   \else
     \let\p!compareprocessaction\p!compareprocessactionB
   \fi
   \def\p!doprocessaction##1%
     {\p!compareprocessaction[##1][#1]}%
   \processnextcommalist\relax\expandactions\p!doprocessaction[}

\endETEX

\def\processfirstactionsinset[#1]%
  {\expandedaction\!!stringa{#1}%
   \ifx\!!stringa\empty
     \expandafter\processaction
   \else
     \expandafter\processfirstactionsinsetindeed
   \fi
   [#1]}

\def\processfirstactioninsetindeed[#1]#2[#3]%
  {\def\p!doprocessaction##1%
     {\def\p!dodoprocessaction####1%
        {\p!compareprocessactionC[####1][##1]}%
      \processcommalist[#3]\p!dodoprocessaction}%
   \processcommalist[#1]\p!doprocessaction
   \expandactions}

\def\processallactionsinset[#1]%
  {\expandedaction\!!stringa{#1}%
   \ifx\!!stringa\empty
     \expandafter\processaction
   \else
     \expandafter\processallactionsinsetindeed
   \fi
   [#1]}

\def\processallactionsinsetindeed[#1]#2[#3]%
  {\advance\processlevel \plusone
   \expandafter\def\csname\s!do\the\processlevel\endcsname##1%
     {\def\p!dodoprocessaction####1%
        {\p!compareprocessactionD[####1][##1]}%
      \processcommalist[#3]\p!dodoprocessaction}%
   \processcommalist[#1]\doprocessallactionsinset
   \advance\processlevel \minusone
   \expandactions}

\def\processnextcommalist#1#2#3[#4#5]%
  {#1%
   \let\nexttoken#4%
   \global\advance\commalevel \plusone
   \long\expandafter\def\csname\s!next\the\commalevel\endcsname##1,%
     {#3{##1}\doprocesscommaitem}%
   \dodoprocesscommaitem#4#5,]\relax
   \global\advance\commalevel \minusone
   #2}

%D I do have an even faster version (saving 3 sec on a 13
%D sec run for 50K invocations, but normally we don't have
%D that many calls and that alternative uses more macros and is
%D even less readable. What we did add, was \type {\@EA}, so
%D that we can pass a command.

%D \macros
%D   {unexpandedprocessaction,
%D    unexpandedprocessfirstactioninset,
%D    unexpandedprocessallactionsinset}
%D
%D Now what are those expansion commands doing there. Well,
%D sometimes we want to compare actions that may consist off
%D commands (i.e. are no constants). In such occasions we can
%D use the a bit slower alternatives:

\def\unexpandedprocessfirstactioninset{\dontexpandactions\processfirstactioninset}
\def\unexpandedprocessaction          {\dontexpandactions\processaction}
\def\unexpandedprocessallactionsinset {\dontexpandactions\processallactionsinset}

%D By default we expand actions:

\def\expandactions{\let\expandedaction\edef} \expandactions

%D But when needed we convert the strings to meaningful
%D sequences of characters.

\def\unexpandedaction#1>{}

\def\noexpandedaction#1#2%
  {\def\convertedargument{#2}%
   \@EA\edef\@EA#1\@EA{\@EA\unexpandedaction\meaning\convertedargument}}

\def\dontexpandactions%
  {\let\expandedaction\noexpandedaction}

%D \macros
%D   {getfirstcharacter, firstcharacter, remainingcharacters, doiffirstcharacter}
%D
%D Sometimes the action to be undertaken depends on the
%D next character. This macro get this character and puts it in
%D \type{\firstcharacter}.
%D
%D \starttyping
%D \getfirstcharacter {string}
%D \stoptyping
%D
%D A two step expansion is used to prevent problems with
%D complicated arguments, for instance arguments that
%D consist of two or more expandable tokens.

\def\dogetfirstcharacter#1#2\relax
  {\def\firstcharacter{#1}%
   \def\remainingcharacters{#2}}

\def\getfirstcharacter#1%
  {\edef\!!stringa{#1}%
   \expandafter\dogetfirstcharacter\!!stringa\relax}

\def\doiffirstcharelse#1#2% char string
% kort (maar onleesbaar)
% {\expanded{\dogetfirstcharacter#2}\\\doifelse{#1}\firstcharacter}
% korter (en begrijpelijk))
  {\getfirstcharacter{#2}\doifelse{#1}\firstcharacter}
% snel (maar zelden gebruikt, dus niet zo belangrijk)
% {\getfirstcharacter{#2}%
%  \edef\!!stringa{#1}%
%  \ifx\!!stringa\firstcharacter
%    \expandafter\firstoftwoarguments
%  \else
%    \expandafter\secondoftwoarguments
%  \fi}

%D \macros
%D   {doifinstringelse, doifincsnameelse}
%D
%D We can check for the presence of a substring in a given
%D sequence of characters.
%D
%D \starttyping
%D \doifinsetelse {substring} {string} {then ...} {else ...}
%D \stoptyping
%D
%D An application of this command can be found further on.
%D Like before, we first show some alternatives, like the one
%D we started with:
%D
%D \starttyping
%D \long\def\p!doifinstringelse#1#2#3#4%
%D   {\def\pp!doifinstringelse##1#1##2##3\war%
%D      {\if##2@%
%D         #4%
%D       \else
%D         #3%
%D       \fi}%
%D    \pp!doifinstringelse#2#1@@\war}
%D
%D \def\doifinstringelse%
%D   {\ExpandBothAfter\p!doifinstringelse}
%D \stoptyping
%D
%D After this we came to:
%D
%D \starttyping
%D \def\p!doifinstringelse#1#2%
%D   {\def\pp!doifinstringelse##1#1##2##3\war%
%D      {\if##2@}%
%D    \pp!doifinstringelse#2#1@@\war}
%D
%D \def\doifinstringelse#1#2#3#4%
%D   {\ExpandBothAfter\p!doifinstringelse{#1}{#2}%
%D      #4%
%D    \else
%D      #3%
%D    \fi}
%D \stoptyping
%D
%D Sometimes the second argument is passed as a macro. By
%D postponing the expansion of this macro, we gain quite some
%D run time, simply because the less tokens we pass, the faster
%D \TEX\ runs. So finally the definition became:

\long\def\doifinstringelse#1%
  {\edef\@@@instring{#1}% expand #1 here
   \ifx\@@@instring\empty
     \@EA\thirdofthreearguments
   \else
     \@EA\dodoifinstringelse
   \fi}

\long\def\dodoifinstringelse#1%
  {\p!doifinstringelse\@@@instring{#1}%
     \@EA\firstoftwoarguments
   \else
     \@EA\secondoftwoarguments
   \fi}

\long\def\doifinstring#1%%
  {\edef\@@@instring{#1}% expand #1 here
   \ifx\@@@instring\empty
     \@EA\gobbletwoarguments
   \else
     \@EA\dodoifinstring
   \fi}

\long\def\dodoifinstring#1%
  {\p!doifinstringelse\@@@instring{#1}%
     \@EA\firstofoneargument
   \else
     \@EA\gobbleoneargument
   \fi}

%D \starttyping
%D \beginTEX
%D
%D \long\def\p!doifinstringelse#1#2%
%D   {\long\def\pp!doifinstringelse##1#1##2##3\war%
%D      {\csname if\if##2@fals\else tru\fi e\endcsname}%
%D    \expanded{\pp!doifinstringelse#2#1@@\noexpand\war}} % expand #2 here
%D
%D \endTEX
%D
%D \beginETEX \unless
%D
%D \long\def\p!doifinstringelse#1#2%
%D   {\long\def\pp!doifinstringelse##1#1##2##3\war%
%D      {\unless\if##2@}%
%D    \expanded{\pp!doifinstringelse#2#1@@\noexpand\war}} % expand #2 here
%D
%D \endETEX
%D \stoptyping
%D
%D And then \unknown\ after a couple of years, we ran into a
%D situation where \type {##2} was something \type {{bla}}. So
%D finally we need to use an auxiliary macro, otherwise we get
%D funny strings in the output.

% \long\def\p!doifinstringelse#1#2% ##2 can be {abc}
%   {\long\@EA\def\@EA\pp!doifinstringelse\@EA##\@EA1#1##2##3\war % expand #1 here
%      {\ppp!doifinstringelse##2\war}%
%    \expanded{\pp!doifinstringelse#2#1@@\noexpand\war}} % expand #2 here
%
% \beginTEX
%
% \def\ppp!doifinstringelse#1#2\war%
%   {\csname if\ifx#1@fals\else tru\fi e\endcsname}%
%
% \endTEX
%
% \beginETEX \unless
%
% \def\ppp!doifinstringelse#1#2\war%
%   {\unless\ifx#1@}
%
% \endETEX

\beginETEX

\long\def\p!doifinstringelse#1#2% ##2 can be {abc}
  {\long\@EA\def\@EA\pp!doifinstringelse\@EA##\@EA1#1##2##3\war % expand #1 here
     {\unless\if##2@}%
%  \expanded{\pp!doifinstringelse#2#1@@\noexpand\war}} % expand #2 here
   \expanded{\pp!doifinstringelse#2#1}@@\war} % expand #2 here

\endETEX

\beginTEX

\long\def\p!doifinstringelse#1#2% ##2 can be {abc}
  {\long\@EA\def\@EA\pp!doifinstringelse\@EA##\@EA1#1##2##3\war % expand #1 here
     {\csname if\if##2@fals\else tru\fi e\endcsname}%
  %\expanded{\pp!doifinstringelse#2#1@@\noexpand\war}} % expand #2 here
   \expanded{\pp!doifinstringelse#2#1}@@\war} % expand #2 here

\endTEX

%D The next alternative proved to be upto twice as fast on
%D tasks like checking reserved words in pretty verbatim
%D typesetting! This is mainly due to the fact that passing
%D (expanded) strings is much slower that passing a macro.
%D
%D \starttyping
%D \doifincsnameelse {substring} {\string} {then ...} {else ...}
%D \stoptyping
%D
%D Where \type{\doifinstringelse} does as much expansion as
%D possible, the latter alternative does minimal (one level)
%D expansion.

\beginTEX

\long\def\p!doifincsnameelse#1#2%
  {\long\def\pp!doifincsnameelse##1#1##2##3\war
     {\csname if\if##2@fals\else tru\fi e\endcsname}%
   \@EA\pp!doifincsnameelse#2#1@@\war}

\endTEX

\beginETEX \unless

\long\def\p!doifincsnameelse#1#2%
  {\long\def\pp!doifincsnameelse##1#1##2##3\war
     {\unless\if##2@}%
   \@EA\pp!doifincsnameelse#2#1@@\war}

\endETEX

\long\def\doifincsnameelse#1#2%                   % #3#4%
  {\edef\@@@instring{#1}%
   \@EA\p!doifincsnameelse\@EA{\@@@instring}{#2}% % #3\else#4\fi}
     \expandafter\firstoftwoarguments
   \else
     \expandafter\secondoftwoarguments
   \fi}

%D \macros
%D   {doifnumberelse}
%D
%D The next macro executes a command depending of the outcome
%D of a test on numerals. This is probably one of the fastest
%D test possible, exept from a less robust 10||step
%D \type{\if}||ladder or some tricky \type{\lcode} checking.
%D
%D \starttyping
%D \doifnumberelse {string} {then ...} {else ...}
%D \stoptyping
%D
%D The macro accepts \type{123}, \type{abc}, \type{{}},
%D \type{\getal} and \type{\the\count...}. This macro is a
%D rather dirty one.
%D
%D \starttyping
%D \long\def\doifnumberelse#1#2#3%
%D   {\begingroup\donefalse
%D    \ifcase1#1\or\or\or\or\or\or\or\or\or\else\donetrue\fi
%D    \ifdone\endgroup#2\else\endgroup#3\fi}
%D \stoptyping
%D
%D Or better:
%D
%D \starttyping
%D \long\def\doifnumberelse#1%
%D   {\begingroup\donefalse
%D    \ifcase1#1\or\or\or\or\or\or\or\or\or\else\donetrue\fi
%D    \ifdone
%D      \endgroup\expandafter\firstoftwoarguments
%D    \else
%D      \endgroup\expandafter\secondoftwoarguments
%D    \fi}
%D \stoptyping
%D
%D A previous implementation was:
%D
%D \starttyping
%D \long\def\doifnumberelse#1#2#3%
%D   {\getfirstcharacter{#1}%
%D    \@EA\p!doifinstringelse\@EA{\firstcharacter}{1234567890}%
%D      #2%
%D    \else
%D      #3%
%D    \fi}
%D \starttyping
%D
%D And before we had \type{\p!doifinstringelse} available, we
%D used:
%D
%D \starttyping
%D \def\doifnumberelse#1%
%D   {\getfirstcharacter{#1}%
%D    \rawdoifinsetelse{\firstcharacter}{1,2,3,4,5,6,7,8,9,0}}
%D \stoptyping
%D
%D The implementation using \type {\ifcase} is much faster, but
%D the next one is not, not even when testing milion calls.
%D
%D \starttyping
%D \newif\ifitsanumber
%D
%D \long\def\isitanumber#1%
%D   {\itsanumberfalse
%D    \ifcase1#1\or\or\or\or\or\or\or\or\or\else\itsanumbertrue\fi}
%D
%D \long\def\doifnumberelse#1#2#3%
%D   {\isitanumber{#1}\ifitsanumber#2\else#3\fi}
%D \stoptyping
%D
%D After a while the next evolved and this one is the one we
%D will use. This one is some 5\% faster than the group/done
%D one (partly because it does not have to pass arguments).
%D Even more important is that this alternative is fully
%D expandable!

\long\def\doifnumberelse#1% does not accept counters
  {\ifcase0\ifcase1#1\or\or\or\or\or\or\or\or\or\else1\fi\space
     \expandafter\secondoftwoarguments
   \else
     \expandafter\firstoftwoarguments
   \fi}

%D \macros
%D   {makerawcommalist,
%D    rawdoinsetelse,
%D    rawprocesscommalist,
%D    rawprocessaction}
%D
%D Some of the commands mentioned earlier are effective but
%D slow. When one is desperately in need of faster alternatives
%D and when the conditions are predictable safe, the \type{\raw}
%D alternatives come into focus. A major drawback is that
%D they do not take \type{\c!constants} into account, simply
%D because no expansion is done. This is no problem with
%D \type{\rawprocesscommalist}, because this macro does not
%D compare anything. Expandable macros are permitted as search
%D string.
%D
%D \starttyping
%D \makerawcommalist[string,string,...]\stringlist
%D \rawdoifinsetelse{string}{string,...}{...}{...}
%D \rawprocesscommalist[string,string,...]\commando
%D \rawprocessaction[x][a=>\a,b=>\b,c=>\c]
%D \stoptyping
%D
%D Spaces embedded in the list, for instance after commas,
%D spoil the search process. The gain in speed depends on the
%D length of the argument (the longer the argument, the less
%D we gain).
%D
%D The slow alternative looks like:
%D
%D \starttyping
%D \def\makerawcommalist[#1]#2%
%D   {\def\appendtocommalist##1%
%D      {\doifelse{#2}{}
%D         {\edef#2{##1}}
%D         {\edef#2{#2,##1}}}%
%D    \def#2{}%
%D    \processcommalist[#1]\appendtocommalist}
%D \stoptyping
%D
%D But we prefer:

% \appendtocommalist is defined in syst-ext

\def\makerawcommalist[#1]#2% use \processnext ... here
  {\def\domakerawcommalist##1% we don't expand ##1
     {\ifx#2\empty
        \def#2{##1}%
      \else
        \@EA\def\@EA#2\@EA{#2,##1}%
      \fi}%
   \let#2\empty
   \processcommalist[#1]\domakerawcommalist}

\def\rawprocesscommaitem#1,#2% #2 eats up preceding space
  {\if]#1\else
     \csname\s!next\the\commalevel\endcsname{#1}%
     \expandafter\rawprocesscommaitem
   \fi#2}

\def\rawprocesscommalist[#1]#2% accepteert ook [\cs]
  {\global\advance\commalevel \plusone
   \expandafter\let\csname\s!next\the\commalevel\endcsname#2%
   \expandafter\rawprocesscommaitem#1,],% \relax
   \global\advance\commalevel \minusone }

\def\rawprocesscommacommand[#1]% not really needed
  {\expanded{\rawprocesscommalist[#1]}}

% \def\rawdoifinsetelse#1#2{\doifinstringelse{,#1,}{,#2,}}
% \def\rawdoifinset    #1#2{\doifinstring    {,#1,}{,#2,}}

\def\@@rawempty{,,}

\long\def\rawdoifinsetelse#1%
  {\edef\@@@instring{,#1,}% expand #1 here
   \ifx\@@@instring\@@rawempty
     \@EA\thirdofthreearguments
   \else
     \@EA\rawdodoifinsetelse
   \fi}

\long\def\rawdodoifinsetelse#1%
  {\p!doifinstringelse\@@@instring{,#1,}%
     \@EA\firstoftwoarguments
   \else
     \@EA\secondoftwoarguments
   \fi}

\long\def\rawdoifinset#1%
  {\edef\@@@instring{,#1,}% expand #1 here
   \ifx\@@@instring\@@rawempty
     \@EA\gobbletwoarguments
   \else
     \@EA\rawdodoifinset
   \fi}

\long\def\rawdodoifinset#1%%
  {\p!doifinstringelse\@@@instring{,#1,}%
     \@EA\firstofoneargument
   \else
     \@EA\gobbleoneargument
   \fi}

%D Some more raw material:

\def\p!rawprocessaction[#1][#2]%
  {\def\pp!rawprocessaction##1,#1=>##2,##3\war%
     {\if##3@\else
        \def\!!processaction{##2}%
      \fi}%
   \pp!rawprocessaction,#2,#1=>,@\war}

\def\rawprocessaction[#1]#2[#3]%
  {\edef\!!stringa{#1}%
   \edef\!!stringb{undefined}% better \!!undefined
   \let\!!processaction\!!stringb
   \ifx\!!stringa\empty
     \@EA\p!rawprocessaction\@EA[\s!default][#3]%
   \else
      \expandafter\p!rawprocessaction\expandafter[\!!stringa][#3]%
      \ifx\!!processaction\!!stringb
        \@EA\p!rawprocessaction\@EA[\s!unknown][#3]%
      \fi
   \fi
   \ifx\!!processaction\!!stringb
   \else
     \!!processaction
   \fi}

% not needed
%
% \def\rawprocessallactionsinset[#1]#2[#3]%
%   {\def\docommando##1%
%      {\rawprocessaction[##1][#3]}%
%    \processcommalist[#1]\docommando}

%D When we process the list \type{a,b,c,d,e}, the raw routine
%D takes over 30\% less time, when we feed $20+$ character
%D strings we gain about 20\%. Alternatives which use
%D \type{\futurelet} perform worse. Part of the speedup is
%D due to the \type{\let} and \type{\expandafter} in the test.

% %D \macros
% %D   {processunexpandedcommalist}
% %D
% %D When processing commalists, the arguments are expanded. The
% %D main reason for doing so lays in the fact that these
% %D macros are used for interfacing. The next alternative can be used
% %D for
% %D
% %D \starttyping
% %D \processunexpandedcommalist
% %D   [\alfa\beta,\gamma,\delta\epsilon]
% %D   \handleitem
% %D \stoptyping
% %D
% %D This time nesting is not supported.
%
% %\def\processunexpandedcommaitem#1,%
% %  {\if]\noexpand#1%
% %     \let\nextcommaitem\relax
% %   \else
% %     \handleunexpandedcommaitem{#1}%
% %     \let\nextcommaitem\processunexpandedcommaitem
%   \fi
%   \nextcommaitem}
%
% faster:
%
% \def\processunexpandedcommaitem#1,%
%   {\if]\noexpand#1\else
%      \handleunexpandedcommaitem{#1}%
%      \expandafter\processunexpandedcommaitem
%    \fi}
%
% \def\processunexpandedcommalist[#1]#2%
%   {\def\handleunexpandedcommaitem{#2}%
%    \processunexpandedcommaitem#1,],}% \relax}
%
% %D Or faster:
%
% \def\processunexpandedcommaitem#1,%
%   {\if]\noexpand#1\else
%      \handleunexpandedcommaitem{#1}%
%      \expandafter\processunexpandedcommaitem
%    \fi}

%D \macros
%D   {dosetvalue,dosetevalue,dosetgvalue,docopyvalue,doresetvalue,
%D    dogetvalue}
%D
%D When we are going to do assignments, we have to take
%D multi||linguality into account. For the moment we keep
%D things simple and single||lingual.
%D
%D \starttyping
%D \dosetvalue   {label}    {variable}   {value}
%D \dosetevalue  {label}    {variable}   {value}
%D \dosetgvalue  {label}    {variable}   {value}
%D \docopyvalue  {to label} {from label} {variable}
%D \doresetvalue {label}    {variable}
%D \stoptyping
%D
%D These macros are in fact auxiliary ones and are not meant
%D for use outside the assignment macros.

\def\dosetvalue#1#2% #3
  {\@EA\def\csname#1#2\endcsname} % {#3}}

\def\dosetevalue#1#2% #3
  {\@EA\edef\csname#1#2\endcsname} % {#3}}

\def\dosetgvalue#1#2% #3
  {\@EA\gdef\csname#1#2\endcsname} % {#3}}

\def\doresetvalue#1#2%
  {\@EA\let\csname#1#2\endcsname\empty}

\def\doignorevalue#1#2#3%
  {\@EA\let\csname#1#2\endcsname\empty}

\def\docopyvalue#1#2#3%
  {\@EA\def\csname#1#3\endcsname{\csname#2#3\endcsname}}

%D \macros
%D   {doassign,undoassign,doassignempty}
%D
%D Assignments are the backbone of \CONTEXT. Abhorred by the
%D concept of style file hacking, we took a considerable effort
%D in building a parameterized system. Unfortunately there is a
%D price to pay in terms of speed. Compared to other packages
%D and taking the functionality of \CONTEXT\ into account, the
%D total size of the format file is still very acceptable. Now
%D how are these assignments done.
%D
%D Assignments can be realized with:
%D
%D \starttyping
%D \doassign[label][variable=value]
%D \undoassign[label][variable=value]
%D \stoptyping
%D
%D and:
%D
%D \starttyping
%D \doassignempty[label][variable=value]
%D \stoptyping
%D
%D Assignments like \type{\doassign} are compatible with:
%D
%D \starttyping
%D \def\labelvariable{value}
%D \stoptyping
%D
%D We do check for the presence of an \type{=} and loudly
%D complain of it's missed. We will redefine this macro later
%D on, when a more advanced message mechanism is implemented.

\newif\iferrorisfatal

\def\waitonfatalerror
  {\iferrorisfatal\wait\fi}

\def\showassignerror#1#2%
  {\writestatus{setup}{missing or ungrouped '=' after '#1' in line #2}%
   \waitonfatalerror}

%\def\p!doassign#1[#2][#3=#4=#5]%
%  {\let\donottest\dontprocesstest
%   \edef\!!stringa{#5}%
%   \let\!!stringb\relax
%   \let\donottest\doprocesstest
%   \ifx\!!stringa\!!stringb
%     \showassignerror{#3}%
%   \else
%     #1{#2}{#3}{#4}%
%   \fi}

\def\p!doassign#1[#2][#3=#4=#5]%
  {\ifx\empty#3\else  % and definitely not \ifx#3\empty
     \ifx\relax#5%
       \showassignerror{#3}{\the\inputlineno\space(#2)}%
     \else
       #1{#2}{#3}{#4}%
     \fi
   \fi}

\def\doassign  [#1][#2]{\p!doassign\dosetvalue  [#1][#2==\relax]}
\def\doeassign [#1][#2]{\p!doassign\dosetevalue [#1][#2==\relax]}
\def\undoassign[#1][#2]{\p!doassign\doresetvalue[#1][#2==\relax]}

\def\doassignempty[#1][#2=#3]%
  {\ifundefined{#1#2}\dosetvalue{#1}{#2}{#3}\fi}

%D \macros
%D   {getparameters,geteparameters,getgparameters,
%D    forgetparameters}
%D
%D Using the assignment commands directly is not our
%D ideal of user friendly interfacing, so we take some further
%D steps.
%D
%D \starttyping
%D \getparameters    [label] [...=...,...=...]
% %D \forgetparameters [label] [...=...,...=...]
%D \stoptyping
%D
%D Again, the label identifies the category a variable
%D belongs to. The second argument can be a comma separated
%D list of assignments.
%D
%D \starttyping
%D \getparameters
%D   [demo]
%D   [alfa=1,
%D    beta=2]
%D \stoptyping
%D
%D is equivalent to
%D
%D \starttyping
%D \def\demoalfa{1}
%D \def\demobeta{2}
%D \stoptyping
%D
%D
%D In the pre||multi||lingual stadium \CONTEXT\ took the next
%D approach. With
%D
%D \starttyping
%D \def\??demo {@@demo}
%D \def\!!alfa {alfa}
%D \def\!!beta {beta}
%D \stoptyping
%D
%D calling
%D
%D \starttyping
%D \getparameters
%D   [\??demo]
%D   [\!!alfa=1,
%D    \!!beta=2]
%D \stoptyping
%D
%D lead to:
%D
%D \starttyping
%D \def\@@demoalfa{1}
%D \def\@@demobeta{2}
%D \stoptyping
%D
%D Because we want to be able to distinguish the \type{!!}
%D pre||tagged user supplied variables from internal
%D counterparts, we will introduce a slightly different tag in
%D the multi||lingual modules. There we will use \type{c!} or
%D \type{v!}, depending on the context.
%D
%D By calling \type{\p!doassign} directly, we save ourselves
%D some argument passing and gain some speed. Whatever
%D optimizations we do, this  command will always be one of the
%D bigger bottlenecks.
%D
%D The alternative \type{\geteparameters} --- it's funny to
%D see that this alternative saw the light so lately --- can be
%D used to do expanded assigments.

\def\dogetparameters#1[#2]#3[#4]%
  {\def\p!dogetparameter##1%
     {\p!doassign#1[#2][##1==\relax]}%
   \processcommalist[#4]\p!dogetparameter}

\def\getparameters   {\dogetparameters\dosetvalue}
\def\geteparameters  {\dogetparameters\dosetevalue}
\def\getgparameters  {\dogetparameters\dosetgvalue}
\def\forgetparameters{\dogetparameters\doignorevalue}

\let\getexpandedparameters=\geteparameters

%D This one is slightly faster:

\def\dogetparameters#1[#2]#3[#4%
  {\if\noexpand#4]%
     \expandafter\gobbleoneargument
   \else
     \def\p!dogetparameter##1{\p!doassign#1[#2][##1==\relax]}%
     \expandafter\xdogetparameters
   \fi#4}

\def\xdogetparameters#1]%
  {\processcommalist[#1]\p!dogetparameter}

%D The next alternative is much faster but also uglier. Because
%D in \XML\ processing we will probably set much more parameters
%D than normally we need this faster one.

\def\dogetparameters#1[#2]#3[#4%
  {\if\noexpand#4]%
     \expandafter\gobbleoneargument
   \else
     \def\p!dogetparameter{\p!doassign#1#2}%
     \expandafter\xdogetparameters
   \fi#4}

\def\xdogetparameters#1]%
  {\xprocesscommaitem#1,],\@relax@}

% \long\def\xprocesscommaitem#1,#2% #2 takes space before ,
%   {\if]#1%
%      \expandafter\gobbleoneargument
%    \else
%      \p!dogetparameter\@relax@#1==\@relax@
%      \expandafter\xprocesscommaitem
%    \fi#2}

\long\def\xprocesscommaitem#1,#2% #2 takes space before ,
  {\if,#1,% dirty trick for testing #1=empty
     \@EA\xprocesscommaitem
   \else\if]#1%
     \@EAEAEA\gobbleoneargument
   \else
     \p!dogetparameter\@relax@#1==\empty\@relax@
     \@EAEAEA\xprocesscommaitem
   \fi\fi#2}

%D Here we use a slightly different assignment macro:

% \def\p!doassign#1#2\@relax@#3=#4=#5\@relax@
%   {\ifx\@relax@#5\@EA\xshowassignerror\else\@EA#1\fi{#2}{#3}{#4}}

% \def\p!doassign#1#2\@relax@#3=#4=#5\@relax@
%   {\ifx#5\empty\@EA\xshowassignerror\else\@EA#1\fi{#2}{#3}{#4}}

\def\p!doassign#1#2\@relax@#3=#4=#5#6\@relax@
  {\ifx#5\empty
     \@EA\xshowassignerror
   \else\ifx#5=%
     \@EAEAEA#1%
   \else
     \@EAEAEA\xshowassignerror
   \fi\fi
   {#2}{#3}{#4}}

\def\xshowassignerror#1#2#3%
  {\showassignerror{#2}{\the\inputlineno\space(#1)}}

%D Now we also have to change the other macros that depend
%D on this low level one.

% \def\doassign  [#1][#2]{\p!doassign\dosetvalue  #1\@relax@#2==\@relax@}
% \def\doeassign [#1][#2]{\p!doassign\dosetevalue #1\@relax@#2==\@relax@}
% \def\undoassign[#1][#2]{\p!doassign\doresetvalue#1\@relax@#2==\@relax@}

\def\doassign  [#1][#2]{\p!doassign\dosetvalue  #1\@relax@#2==\empty\@relax@}
\def\doeassign [#1][#2]{\p!doassign\dosetevalue #1\@relax@#2==\empty\@relax@}
\def\undoassign[#1][#2]{\p!doassign\doresetvalue#1\@relax@#2==\empty\@relax@}

%D When someone asked on the mailing list if it's possible to
%D use the current value of a parameter, Taco posted a small module. His
%D method had the disadvantage of making all assignments expanded and thereby
%D fragile. The following alternative uses a prefix.

%D \macros{currentvalue}
%D
%D Just in case a \type{\getparameter} argument itself ends up
%D inside a \type{\write} or other expandable location, our
%D new macro needs a default value.

\let\currentvalue\empty

\def\p!n!doassign#1#2\@relax@#3=#4=#5#6\@relax@% normal
  {\ifx#5\empty
     \@EA\xshowassignerror
   \else\ifx#5=%
     \@EAEAEA#1%
   \else
     \@EAEAEA\xshowassignerror
   \fi\fi
   {#2}{#3}{#4}}

\beginTEX

\def\p!e!doassign#1#2\@relax@#3=#4=#5#6\@relax@
  {\ifx#5\empty
     \@EA\xshowassignerror
   \else\ifx#5=%
     \@EA\ifx\csname#2#3\endcsname\relax
       \let\currentvalue\empty
     \else
       \@EA\let\@EA\currentvalue\csname#2#3\endcsname
     \fi
     \@EAEAEA#1%
   \else
     \@EAEAEA\xshowassignerror
   \fi\fi
   {#2}{#3}{#4}}

\endTEX

\beginETEX

\def\p!e!doassign#1#2\@relax@#3=#4=#5#6\@relax@
  {\ifx#5\empty
     \@EA\xshowassignerror
   \else\ifx#5=%
     \ifcsname#2#3\endcsname
       \@EA\let\@EA\currentvalue\csname#2#3\endcsname
     \else
       \let\currentvalue\empty
     \fi
     \@EAEAEA#1%
   \else
     \@EAEAEA\xshowassignerror
   \fi\fi
   {#2}{#3}{#4}}

\endETEX

%D We default to:

\let\p!doassign\p!n!doassign

%D And set:

\let\currentvalue\empty

%D \macros {expandparameters}
%D
%D Example usage:
%D
%D \startbuffer
%D \getparameters[taco][name=taco]
%D \convertcommand\taconame\to\ascii \ascii
%D \expandparameters \getparameters[taco][name=\currentvalue\space hoekwater]
%D \convertcommand\taconame\to\ascii \ascii
%D \getparameters[taco][name=\currentvalue\space hoekwater]
%D \convertcommand\taconame\to\ascii \ascii
%D \stopbuffer
%D
%D \typebuffer
%D \startlines
%D \getbuffer
%D \stoplines

%D Here we hook in the code (beware, this is the optimized get **):

\def\xdoget@n@parameters#1]%
  {\xprocesscommaitem#1,],\@relax@}

\def\xdoget@e@parameters#1]%
  {\let\dosetnvalue\dosetvalue
   \let\dosetvalue\dosetevalue
   \let\p!doassign\p!e!doassign
   \xprocesscommaitem#1,],\@relax@
   \let\p!doassign\p!n!doassign
   \let\dosetvalue\dosetnvalue
   \let\xdogetparameters\xdoget@n@parameters
   \let\currentvalue\empty}

\let\xdogetparameters\xdoget@n@parameters % **

\def\expandparameters{\let\xdogetparameters\xdoget@e@parameters}

%D \macros
%D   {getemptyparameters}
%D
%D Sometimes we explicitly want variables to default to an
%D empty string, so we welcome:
%D
%D \starttyping
%D \getemptyparameters [label] [...=...,...=...]
%D \stoptyping

\def\getemptyparameters[#1]#2[#3]%
  {\def\p!dogetemptyparameter##1%
     {\doassignempty[#1][##1]}%
   \processcommalist[#3]\p!dogetemptyparameter}

%D \macros
%D   {copyparameters}
%D
%D Some \CONTEXT\ commands take their default setups from
%D others. All commands that are able to provide backgounds
%D or rules around some content, for instance default to the
%D standard command for ruled boxes. Is situations like this
%D we can use:
%D
%D \starttyping
%D \copyparameters [to-label] [from-label] [name1,name2,...]
%D \stoptyping
%D
%D For instance
%D
%D \starttyping
%D \copyparameters
%D   [internal][external]
%D   [alfa,beta]
%D \stoptyping
%D
%D Leads to:
%D
%D \starttyping
%D \def\internalalfa {\externalalfa}
%D \def\internalbeta {\externalbeta}
%D \stoptyping
%D
%D By using \type{\docopyvalue} we've prepared this command
%D for use in a multi||lingual environment.

\def\copyparameters[#1]#2[#3]#4[#5]%
  {\doifnot{#1}{#3}
     {\def\docopyparameter{\docopyvalue{#1}{#3}}%
     %\def\docopyparameter##1{\docopyvalue{#1}{#3}{##1}}%
      \processcommalist[#5]\docopyparameter}}

%D \macros
%D   {doifassignmentelse}
%D
%D A lot of \CONTEXT\ commands take optional arguments, for
%D instance:
%D
%D \starttyping
%D \dothisorthat[alfa,beta]
%D \dothisorthat[first=foo,second=bar]
%D \dothisorthat[alfa,beta][first=foo,second=bar]
%D \stoptyping
%D
%D Although a combined solution is possible, we prefer a
%D seperation. The next command takes care of propper
%D handling of such multi||faced commands.
%D
%D \starttyping
%D \doifassignmentelse {...} {then ...} {else ...}
%D \stoptyping

% not robust
%
% \def\doifassignmentelse%
%   {\doifinstringelse{=}}
%
% readable
%
% \def\doifassignmentelse#1%
%   {\convertargument#1\to\ascii
%    \doifinstringelse{=}{\ascii}}

\def\doifassignmentelse#1%
  {\convertargument#1\to\ascii
   \doifinstringelse=\ascii}

%D \macros
%D   {ifparameters,checkparameters}
%D
%D A slightly different one is \type{\checkparameters}, which
%D also checks on the presence of a~\type{=}.
%D
%D The boolean \type{\ifparameters} can be used afterwards.
%D Combining both in one \type{\if}||macro would lead to
%D problems with nested \type{\if}'s.
%D
%D \starttyping
%D \checkparameters[argument]
%D \stoptyping

\newif\ifparameters

\def\p!checkparameters#1=#2#3\war%
  {\if#2@\parametersfalse\else\parameterstrue\fi}

\def\checkparameters[#1]%
  {\p!checkparameters#1=@@\war}

%D \macros
%D   {getfromcommalist,getfromcommacommand,
%D    commalistelement,
%D    getcommalistsize,getcommacommandsize}
%D
%D It's possible to get an element from a commalist or a
%D command representing a commalist.
%D
%D \starttyping
%D \getfromcommalist    [string] [n]
%D \getfromcommacommand [string,\strings,string,...] [n]
%D \stoptyping
%D
%D The difference betwee the two of them is the same as the
%D difference between \type{\processcomma...}. The found string
%D is stored in \type{\commalistelement}.
%D
%D We can calculate the size of a comma separated list by
%D using:
%D
%D \starttyping
%D \getcommalistsize    [string,string,...]
%D \getcommacommandsize [string,\strings,string,...]
%D \stoptyping
%D
%D Afterwards, the length is available in the macro
%D \type{\commalistsize} (not a \COUNTER).

\newcount\commalistcounter

\def\commalistsize{0}

\def\p!dogetcommalistsize#1%
  {\advance\commalistcounter\plusone}

\def\getcommalistsize#1]% don't loose [{#1}]
  {\commalistcounter\zerocount
   \processcommalist#1]\p!dogetcommalistsize   % was [{#1}]
   \edef\commalistsize{\the\commalistcounter}}

\def\getcommacommandsize[#1]%
  {\edef\commacommand{#1}%
   \scratchtoks\expandafter{\expandafter[\commacommand]}%
   \expandafter\getcommalistsize\the\scratchtoks }

% to be tested first
%
% \def\getcommacommandsize[#1]%
%   {\expanded{\getcommalistsize[#1]}}

% \def\p!dogetfromcommalist#1%
%   {\advance\commalistcounter \minusone
%    \ifcase\commalistcounter
%      \def\commalistelement{#1}%
%      \begingroup\def\doprocesscommaitem##1]{\endgroup}%
%    \fi}

\def\p!dogetfromcommalist#1%
  {\advance\commalistcounter \minusone
   \ifcase\commalistcounter
     \def\commalistelement{#1}%
     \expandafter\quitcommalist
   \fi}

\def\getfromcommalist[#1]#2[#3]%
  {\let\commalistelement\empty
   \commalistcounter#3\relax
   \processcommalist[#1]\p!dogetfromcommalist}

% \def\getfromcommacommand[#1]% why so complicated, still needed?
%   {\edef\commacommand{#1}%
%    \toks0=\expandafter{\expandafter[\commacommand]}%
%    \expandafter\getfromcommalist\the\toks0 }

\def\getfromcommacommand[#1]%
  {\expanded{\getfromcommalist[#1]}}

%D Because 0, 1 and~2 are often asked for, we optimize this
%D macro for those cases. The indirect call however slows
%D down the other cases.
%D
%D \starttyping
%D \def\p!dogetfirstfromcommalist    [#1,#2]{\def\commalistelement{#1}}
%D \def\p!dogetsecondfromcommalist[#1,#2,#3]{\def\commalistelement{#2}}
%D \let\p!dogetotherfromcommalist=\getfromcommalist
%D
%D \def\getfromcommalist[#1]#2[#3]% optimized for 0,1,2
%D   {\ifcase#3\relax
%D      \let\commalistelement\empty
%D    \or
%D      \p!dogetfirstfromcommalist[#1,]%
%D    \or
%D      \p!dogetsecondfromcommalist[#1,,]%
%D    \else
%D      \p!dogetotherfromcommalist[#1][#3]%
%D    \fi}
%D \stoptyping
%D
%D Even worse, this alternative does not strip preceding
%D spaces, which is what we want. So, we stick to the slow
%D alternative.

%D Watertight (and efficient) solutions are hard to find, due
%D to the handling of braces during parameters passing and
%D scanning. Nevertheless:
%D
%D \startbuffer
%D \def\dosomething#1{(#1=\commalistsize) }
%D
%D \getcommalistsize [\hbox{$a,b,c,d,e,f$}] \dosomething 1
%D \getcommalistsize [{a,b,c,d,e,f}]        \dosomething 1
%D \getcommalistsize [{a,b,c},d,e,f]        \dosomething 4
%D \getcommalistsize [a,b,{c,d,e},f]        \dosomething 4
%D \getcommalistsize [a{b,c},d,e,f]         \dosomething 4
%D \getcommalistsize [{a,b}c,d,e,f]         \dosomething 4
%D \getcommalistsize []                     \dosomething 0
%D \getcommalistsize [{[}]                  \dosomething 1
%D \stopbuffer
%D
%D \typebuffer
%D
%D reports:
%D
%D \getbuffer

%D \macros
%D   {dogetcommalistelement,dogetcommacommandelement}
%D
%D For low level (fast) purposes, we can also use the next
%D alternative, which can handle 8~elements at most.
%D
%D \starttyping
%D \dogetcommalistelement1\from a,b,c\to\commalistelement
%D \stoptyping

\def\dodogetcommalistelement#1\from#2,#3,#4,#5,#6,#7,#8\to#9%
  {\edef#9{\ifcase#1\relax\or#2\or#3\or#4\or#5\or#6\or#7\or#8\fi}}

% maybe better:
%
%  {\@EA\edef\@EA#9\@EA{\ifcase#1\relax\or#2\or#3\or#4\or#5\or#6\or#7\or#8\fi}}

\def\dogetcommalistelement#1\from#2\to%
  {\dodogetcommalistelement#1\from#2,,,,,,\to}

% check sources

\def\dogetcommacommandelement#1\from#2\to%
  {\@EA\dodogetcommalistelement\@EA#1\@EA\from#2,,,,,,\to}

%D \macros
%D   {dosingleargument,dodoubleargument,dotripleargument,
%D    doquadrupleargument,doquintupleargument,dosixtupleargument,
%D    doseventupleargument}
%D
%D When working with delimited arguments, spaces and
%D lineendings can interfere. The next set of macros uses
%D \TEX' internal scanner for grabbing everything between
%D arguments. Forgive me the funny names.
%D
%D \starttyping
%D \dosingleargument\commando    = \commando[#1]
%D \dodoubleargument\commando    = \commando[#1][#2]
%D \dotripleargument\commando    = \commando[#1][#2][#3]
%D \doquadrupleargument\commando = \commando[#1][#2][#3][#4]
%D \doquintupleargument\commando = \commando[#1][#2][#3][#4][#5]
%D \dosixtupleargument\commando  = \commando[#1][#2][#3][#4][#5][#6]
%D \doseventupleargument\command= \commando[#1][#2][#3][#4][#5][#6][#7]
%D \stoptyping
%D
%D These macros are used in the following way:
%D
%D \starttyping
%D \def\dosetupsomething[#1][#2]%
%D   {... #1 ... #2 ...}
%D
%D \def\setupsomething%
%D   {\dodoubleargument\dosetupsomething}
%D \stoptyping
%D
%D The implementation can be surprisingly simple and needs no
%D further explanation, like:
%D
%D \starttyping
%D \def\dosingleargument#1[#2]%
%D   {#1[#2]}
%D \def\dotripleargument#1[#2]#3[#4]#5[#6]%
%D   {#1[#2][#4][#6]}
%D \def\doquintupleargument#1%
%D   {\def\dodoquintupleargument[##1]##2[##3]##4[##5]##6[##7]##8[##9]%
%D      {#1[##1][##3][##5][##7][##9]}%
%D    \dodoquintupleargument}
%D \stoptyping
%D
%D Because \TEX\ accepts 9~arguments at most, we have to use
%D two||step solution when getting five or more arguments.
%D
%D When developing more and more of the real \CONTEXT, we
%D started using some alternatives that provided empty
%D arguments (in fact optional ones) whenever the user failed
%D to supply them. Because this more complicated macros enable
%D us to do some checking, we reimplemented the non||empty
%D ones.

\def\dosingleargument    {\chardef\expectedarguments 1 \dosingleempty    }
\def\dodoubleargument    {\chardef\expectedarguments 2 \dodoubleempty    }
\def\dotripleargument    {\chardef\expectedarguments 3 \dotripleempty    }
\def\doquadrupleargument {\chardef\expectedarguments 4 \doquadrupleempty }
\def\doquintupleargument {\chardef\expectedarguments 5 \doquintupleempty }
\def\dosixtupleargument  {\chardef\expectedarguments 6 \dosixtupleempty  }
\def\doseventupleargument{\chardef\expectedarguments 7 \doseventupleempty}

%D \macros
%D   {iffirstagument,ifsecondargument,ifthirdargument,
%D    iffourthargument,iffifthargument,ifsixthargument,
%D    ifseventhargument}
%D
%D We use some signals for telling the calling macros if all
%D wanted arguments are indeed supplied by the user.

\newif\iffirstargument
\newif\ifsecondargument
\newif\ifthirdargument
\newif\iffourthargument
\newif\iffifthargument
\newif\ifsixthargument
\newif\ifseventhargument

%D \macros
%D   {dosingleempty,dodoubleempty,dotripleempty,
%D    doquadrupleempty,doquintupleempty,dosixtupeempty,
%D    doseventupleempty}
%D
%D The empty argument supplying macros mentioned before, look
%D like:
%D
%D \starttyping
%D \dosingleempty    \command
%D \dodoubleempty    \command
%D \dotripleempty    \command
%D \doquadrupleempty \command
%D \doquintupleempty \command
%D \dosixtupleempty  \command
%D \doseventupleempty\command
%D \stoptyping
%D
%D So \type{\dodoubleempty} leades to:
%D
%D \starttyping
%D \command[#1][#2]
%D \command[#1][]
%D \command[][]
%D \stoptyping
%D
%D Depending of the generousity of the user. Afterwards one can
%D use the \type{\if...argument} boolean. For novice: watch
%D the stepwise doubling of \type{#}'s

% idea: \ignorespaces afterwards

\chardef\noexpectedarguments=0
\chardef\expectedarguments  =0

\def\showargumenterror#1#2%
  {\writestatus{system}{#1 argument(s) expected in line #2}}

% \long\def\dogetargument#1#2#3#4% redefined in mult-ini
%   {\doifnextcharelse{#1}
%      {\let\expectedarguments\noexpectedarguments
%       #3\dodogetargument}
%      {\ifnum\expectedarguments>\noexpectedarguments
%         \showargumenterror{\expectedarguments}
%       \fi
%       \let\expectedarguments\noexpectedarguments
%       #4\dodogetargument#1#2}}
%
% less to pass

\def\doshowargumenterror
  {\ifnum\expectedarguments>\noexpectedarguments
     \showargumenterror{\number\expectedarguments}{\number\inputlineno}%
   \fi
   \noshowargumenterror}

\def\noshowargumenterror
  {\let\expectedarguments\noexpectedarguments}

% \long\def\dogetargument#1#2#3#4%
%  {\doifnextcharelse#1%
%     {\noshowargumenterror#3\dodogetargument}
%     {\doshowargumenterror#4\dodogetargument#1#2}}
%
% faster ?

\long\def\dogetargument#1#2#3#4%
  {\let\charactertoken=#1%
   \def\!!stringa{\noshowargumenterror#3\dodogetargument}%
   \def\!!stringb{\doshowargumenterror#4\dodogetargument#1#2}%
   \futurelet\nexttoken\inspectnextcharacter}

%\def\getsingleempty#1#2#3%
%  {\def\dodogetargument%
%     {#3}%
%   \dogetargument#1#2\firstargumenttrue\firstargumentfalse}
%
%\def\getdoubleempty#1#2#3%
%  {\def\dodogetargument#1##1#2%
%     {\def\dodogetargument%
%        {#3#1##1#2}%
%      \dogetargument#1#2\secondargumenttrue\secondargumentfalse}%
%   \dogetargument#1#2\firstargumenttrue\firstargumentfalse}
%
%\def\gettripleempty#1#2#3%
%  {\def\dodogetargument#1##1#2%
%     {\def\dodogetargument#1####1#2%
%        {\def\dodogetargument%
%           {#3#1##1#2%
%              #1####1#2}%
%         \dogetargument#1#2\thirdargumenttrue\thirdargumentfalse}%
%      \dogetargument#1#2\secondargumenttrue\secondargumentfalse}%
%   \dogetargument#1#2\firstargumenttrue\firstargumentfalse}
%
%\def\getquadrupleempty#1#2#3%
%  {\def\dodogetargument#1##1#2%
%     {\def\dodogetargument#1####1#2%
%        {\def\dodogetargument#1########1#2%
%           {\def\dodogetargument%
%              {#3#1##1#2%
%                 #1####1#2%
%                 #1########1#2}%
%            \dogetargument#1#2\fourthargumenttrue\fourthargumentfalse}%
%         \dogetargument#1#2\thirdargumenttrue\thirdargumentfalse}%
%      \dogetargument#1#2\secondargumenttrue\secondargumentfalse}%
%   \dogetargument#1#2\firstargumenttrue\firstargumentfalse}
%
%\def\getquintupleempty#1#2#3%
%  {\def\dodogetargument#1##1#2%
%     {\def\dodogetargument#1####1#2%
%        {\def\dodogetargument#1########1#2%
%           {\def\dodogetargument#1################1#2%
%              {\def\dodogetargument%
%                 {#3#1##1#2%
%                    #1####1#2%
%                    #1########1#2%
%                    #1################1#2}%
%               \dogetargument#1#2\fifthargumenttrue\fifthargumentfalse}%
%            \dogetargument#1#2\fourthargumenttrue\fourthargumentfalse}%
%         \dogetargument#1#2\thirdargumenttrue\thirdargumentfalse}%
%      \dogetargument#1#2\secondargumenttrue\secondargumentfalse}%
%   \dogetargument#1#2\firstargumenttrue\firstargumentfalse}
%
%\def\getsixtupleempty#1#2#3%
%  {\def\dodogetargument#1##1#2%
%     {\def\dodogetargument#1####1#2%
%        {\def\dodogetargument#1########1#2%
%           {\def\dodogetargument#1################1#2%
%              {\def\dodogetargument#1################################1#2%
%                 {\def\dodogetargument%
%                    {#3#1##1#2%
%                       #1####1#2%
%                       #1########1#2%
%                       #1################1#2%
%                       #1################################1#2}%
%                 \dogetargument#1#2\sixthargumenttrue\sixthargumentfalse}%
%               \dogetargument#1#2\fifthargumenttrue\fifthargumentfalse}%
%            \dogetargument#1#2\fourthargumenttrue\fourthargumentfalse}%
%         \dogetargument#1#2\thirdargumenttrue\thirdargumentfalse}%
%      \dogetargument#1#2\secondargumenttrue\secondargumentfalse}%
%   \dogetargument#1#2\firstargumenttrue\firstargumentfalse}
%
%\def\getseventupleempty#1#2#3%
%  {\def\dodogetargument#1##1#2%
%     {\def\dodogetargument#1####1#2%
%        {\def\dodogetargument#1########1#2%
%           {\def\dodogetargument#1################1#2%
%              {\def\dodogetargument#1################################1#2%
%                 {\def\dodogetargument#1###############################%
%                                        ################################1#2%
%                    {\def\dodogetargument%
%                       {#3#1##1#2%
%                          #1####1#2%
%                          #1########1#2%
%                          #1################1#2%
%                          #1################################1#2%
%                          #1###############################%
%                            ################################1#2}%
%                   \dogetargument#1#2\seventhargumenttrue\seventhargumentfalse}%
%                 \dogetargument#1#2\sixthargumenttrue\sixthargumentfalse}%
%               \dogetargument#1#2\fifthargumenttrue\fifthargumentfalse}%
%            \dogetargument#1#2\fourthargumenttrue\fourthargumentfalse}%
%         \dogetargument#1#2\thirdargumenttrue\thirdargumentfalse}%
%      \dogetargument#1#2\secondargumenttrue\secondargumentfalse}%
%   \dogetargument#1#2\firstargumenttrue\firstargumentfalse}

\def\getsingleempty#1#2#3%
  {\def\dodogetargument%
     {#3}%
   \dogetargument#1#2\firstargumenttrue\firstargumentfalse}

\def\getdoubleempty#1#2#3%
  {\def\dodogetargument#1##1#2%
     {\def\dodogetargument%
        {#3#1{##1}#2}%
      \dogetargument#1#2\secondargumenttrue\secondargumentfalse}%
   \dogetargument#1#2\firstargumenttrue\firstargumentfalse}

\def\gettripleempty#1#2#3%
  {\def\dodogetargument#1##1#2%
     {\def\dodogetargument#1####1#2%
        {\def\dodogetargument%
           {#3#1{##1}#2%
              #1{####1}#2}%
         \dogetargument#1#2\thirdargumenttrue\thirdargumentfalse}%
      \dogetargument#1#2\secondargumenttrue\secondargumentfalse}%
   \dogetargument#1#2\firstargumenttrue\firstargumentfalse}

\def\getquadrupleempty#1#2#3%
  {\def\dodogetargument#1##1#2%
     {\def\dodogetargument#1####1#2%
        {\def\dodogetargument#1########1#2%
           {\def\dodogetargument%
              {#3#1{##1}#2%
                 #1{####1}#2%
                 #1{########1}#2}%
            \dogetargument#1#2\fourthargumenttrue\fourthargumentfalse}%
         \dogetargument#1#2\thirdargumenttrue\thirdargumentfalse}%
      \dogetargument#1#2\secondargumenttrue\secondargumentfalse}%
   \dogetargument#1#2\firstargumenttrue\firstargumentfalse}

\def\getquintupleempty#1#2#3%
  {\def\dodogetargument#1##1#2%
     {\def\dodogetargument#1####1#2%
        {\def\dodogetargument#1########1#2%
           {\def\dodogetargument#1################1#2%
              {\def\dodogetargument%
                 {#3#1{##1}#2%
                    #1{####1}#2%
                    #1{########1}#2%
                    #1{################1}#2}%
               \dogetargument#1#2\fifthargumenttrue\fifthargumentfalse}%
            \dogetargument#1#2\fourthargumenttrue\fourthargumentfalse}%
         \dogetargument#1#2\thirdargumenttrue\thirdargumentfalse}%
      \dogetargument#1#2\secondargumenttrue\secondargumentfalse}%
   \dogetargument#1#2\firstargumenttrue\firstargumentfalse}

\def\getsixtupleempty#1#2#3%
  {\def\dodogetargument#1##1#2%
     {\def\dodogetargument#1####1#2%
        {\def\dodogetargument#1########1#2%
           {\def\dodogetargument#1################1#2%
              {\def\dodogetargument#1################################1#2%
                 {\def\dodogetargument%
                    {#3#1{##1}#2%
                       #1{####1}#2%
                       #1{########1}#2%
                       #1{################1}#2%
                       #1{################################1}#2}%
                 \dogetargument#1#2\sixthargumenttrue\sixthargumentfalse}%
               \dogetargument#1#2\fifthargumenttrue\fifthargumentfalse}%
            \dogetargument#1#2\fourthargumenttrue\fourthargumentfalse}%
         \dogetargument#1#2\thirdargumenttrue\thirdargumentfalse}%
      \dogetargument#1#2\secondargumenttrue\secondargumentfalse}%
   \dogetargument#1#2\firstargumenttrue\firstargumentfalse}

\def\getseventupleempty#1#2#3%
  {\def\dodogetargument#1##1#2%
     {\def\dodogetargument#1####1#2%
        {\def\dodogetargument#1########1#2%
           {\def\dodogetargument#1################1#2%
              {\def\dodogetargument#1################################1#2%
                 {\def\dodogetargument#1################################%
                                        ################################1#2%
                    {\def\dodogetargument%
                       {#3#1{##1}#2%
                          #1{####1}#2%
                          #1{########1}#2%
                          #1{################1}#2%
                          #1{################################1}#2%
                          #1{################################%
                             ################################1}#2}%
                   \dogetargument#1#2\seventhargumenttrue\seventhargumentfalse}%
                 \dogetargument#1#2\sixthargumenttrue\sixthargumentfalse}%
               \dogetargument#1#2\fifthargumenttrue\fifthargumentfalse}%
            \dogetargument#1#2\fourthargumenttrue\fourthargumentfalse}%
         \dogetargument#1#2\thirdargumenttrue\thirdargumentfalse}%
      \dogetargument#1#2\secondargumenttrue\secondargumentfalse}%
   \dogetargument#1#2\firstargumenttrue\firstargumentfalse}

\def\dosingleempty    {\getsingleempty    []}
\def\dodoubleempty    {\getdoubleempty    []}
\def\dotripleempty    {\gettripleempty    []}
\def\doquadrupleempty {\getquadrupleempty []}
\def\doquintupleempty {\getquintupleempty []}
\def\dosixtupleempty  {\getsixtupleempty  []}
\def\doseventupleempty{\getseventupleempty[]}

%D Because some of these are called quite often, we will now
%D replace the more general version by alternatives tuned for
%D speed.

\def\dosingleempty#1%
  {\noshowargumenterror % \relax % prevents lookahead, brr
   \doifnextcharelse[%
     {\firstargumenttrue#1}
     {\dosinglefakeempty#1}}

\def\dodoubleempty#1%
  {\noshowargumenterror % \relax % prevents lookahead, brr
   \doifnextcharelse[%
     {\dodoubletestempty#1}
     {\dodoublefakeempty#1}}

\def\dotripleempty#1%
  {\noshowargumenterror % \relax % prevents lookahead, brr
   \doifnextcharelse[%
     {\dotripletestempty#1}
     {\dotriplefakeempty#1}}

\def\dosinglefakeempty#1%
  {\firstargumentfalse#1[]}

\def\dodoublefakeempty#1%
  {\firstargumentfalse\secondargumentfalse#1[][]}

\def\dotriplefakeempty#1%
  {\firstargumentfalse\secondargumentfalse\thirdargumentfalse#1[][][]}

\long\def\dodoubletestempty#1[#2]%
  {\firstargumenttrue
   \doifnextcharelse[%
     {\secondargumenttrue #1[{#2}]}
     {\secondargumentfalse#1[{#2}][]}}

\long\def\dotripletestempty#1[#2]%
  {\firstargumenttrue
   \doifnextcharelse[%
     {\dotripletestemptyx #1[{#2}]}
     {\secondargumentfalse
      \thirdargumentfalse #1[{#2}][][]}}

\long\def\dotripletestemptyx#1[#2][#3]%
  {\secondargumenttrue
   \doifnextcharelse[%
     {\thirdargumenttrue #1[{#2}][{#3}]}
     {\thirdargumentfalse#1[{#2}][{#3}][]}}

%D \macros
%D   {dosingleargumentwithset,
%D    dodoubleargumentwithset,dodoubleemptywithset,
%D    dotripleargumentwithset,dotripleemptywithset}
%D
%D These maybe too mysterious macros enable us to handle more
%D than one setup at once.
%D
%D \starttyping
%D \dosingleargumentwithset \command[#1]
%D \dodoubleargumentwithset \command[#1][#2]
%D \dotripleargumentwithset \command[#1][#2][#3]
%D \dodoubleemptywithset    \command[#1][#2]
%D \dotripleemptywithset    \command[#1][#2][#3]
%D \stoptyping
%D
%D The first macro calls \type{\command[##1]} for each string
%D in the set~\type{#1}. The second one calls for
%D \type{\commando[##1][#2]} and the third, well one may guess.
%D These commands support constructions like:
%D
%D \starttyping
%D \def\dodefinesomething[#1][#2]%
%D   {\getparameters[\??xx#1][#2]}
%D
%D \def\definesomething%
%D   {\dodoubleargumentwithset\dodefinesomething}
%D \stoptyping
%D
%D Which accepts calls like:
%D
%D \starttyping
%D \definesomething[alfa,beta,...][variable=...,...]
%D \stoptyping
%D
%D Now a whole bunch of variables like \type{\@@xxalfavariable}
%D and \type{\@@xxbetavariable} is defined.

% \def\dosingleargumentwithset#1%
%   {\def\dodosinglewithset[##1]%
%      {\def\dododosinglewithset####1{#1[####1]}%
%       \processcommalist[##1]\dododosinglewithset}%
%    \dosingleargument\dodosinglewithset}%

\def\dodoublewithset#1#2%
  {\def\dododoublewithset[##1][##2]%
     {\doifsomething{##1}
        {\def\dodododoublewithset####1{#2[####1][##2]}%
         \processcommalist[##1]\dodododoublewithset}}%
   #1\dododoublewithset}

\def\dodoubleemptywithset   {\dodoublewithset\dodoubleempty}
\def\dodoubleargumentwithset{\dodoublewithset\dodoubleargument}

\def\dotriplewithset#1#2%
  {\def\dodotriplewithset[##1][##2][##3]%
     {\doifsomething{##1}
        {\def\dododotriplewithset####1{#2[####1][##2][##3]}%
         \processcommalist[##1]\dododotriplewithset}}%
   #1\dodotriplewithset}

\def\dotripleemptywithset   {\dotriplewithset\dotripleempty}
\def\dotripleargumentwithset{\dotriplewithset\dotripleargument}

%D \macros
%D   {strippedcsname}
%D
%D The next macro can be very useful when using \type{\csname}
%D like in:
%D
%D \starttyping
%D \csname if\strippedcsname\something\endcsname
%D \stoptyping
%D
%D This expands to \type{\ifsomething}.

\def\strippedcsname%
  {\expandafter\gobbleoneargument\string}

%D \macros
%D   {complexorsimple,complexorsimpleempty}
%D
%D Setups can be optional. A command expecting a setup is
%D prefixed by \type{\complex}, a command without one gets the
%D prefix \type{\simple}. Commands like this can be defined by:
%D
%D \starttyping
%D \complexorsimple\command
%D \stoptyping
%D
%D When \type{\command} is followed by a \type{[setup]}, then
%D
%D \starttyping
%D \complexcommand [setup]
%D \stoptyping
%D
%D executes, else we get
%D
%D \starttyping
%D \simplecommand
%D \stoptyping
%D
%D An alternative for \type{\complexorsimple} is:
%D
%D \starttyping
%D \complexorsimpleempty {command}
%D \stoptyping
%D
%D Depending on the presence of \type{[setup]}, this one
%D leads to one of:
%D
%D \starttyping
%D \complexcommando [setup]
%D \complexcommando []
%D \stoptyping
%D
%D Many \CONTEXT\ commands started as complex or simple ones,
%D but changed into more versatile (more object oriented) ones
%D using the \type{\get..argument} commands.

% This method is needed when traditional tex is used with
% the efficient definition (marked **) below.

% an old one:
%
% \def\setnameofcommand#1% handles {abc} as well as \abc
%   {\begingroup
%    \escapechar=-1
%    \globaldefs=0 % pretty important!
%    \xdef\nameofcommand{\string#1}%
%    \endgroup}
%
% \def\complexorsimple#1%
%   {\setnameofcommand{#1}%
%    \doifnextcharelse{[}
%      {\firstargumenttrue \getvalue{\s!complex\nameofcommand}}
%      {\firstargumentfalse\getvalue{\s!simple \nameofcommand}}}
%
% \def\complexorsimpleempty#1%
%   {\setnameofcommand{#1}%
%    \doifnextcharelse{[}
%      {\firstargumenttrue \getvalue{\s!complex\nameofcommand}}
%      {\firstargumentfalse\getvalue{\s!complex\nameofcommand}[]}}
%
% a newer one:

\def\complexorsimple#1%
  {% \relax % prevents lookahead, brrr
   \doifnextcharelse[%
     {\firstargumenttrue \csname\s!complex\strippedcsname#1\endcsname}
     {\firstargumentfalse\csname\s!simple \strippedcsname#1\endcsname}}

\def\complexorsimpleempty#1%
  {% \relax % prevents lookahead, brrr
   \doifnextcharelse[%
     {\firstargumenttrue \csname\s!complex\strippedcsname#1\endcsname}
     {\firstargumentfalse\csname\s!complex\strippedcsname#1\endcsname[]}}

%D \macros
%D   {definecomplexorsimple,definecomplexorsimpleempty}
%D
%D The previous commands are used that often that we found it
%D worthwile to offer two more alternatives. Watch the build
%D in protection.

% See earlier. Because we don't want \type {\simple..} and
% \type {\complex..} commands to show up in expansions, we need
% to pass them as \type {simple..} and \type {complex..}.

% \beginTEX
%
% \def\definewithnameofcommand#1#2%
%   {\setnameofcommand{#2}%
%    \@EA\def\@EA#2\@EA{\@EA\donottest\@EA#1\@EA{\nameofcommand}}}
%
% \def\definecomplexorsimple%
%   {\definewithnameofcommand\complexorsimple}
%
% \def\definecomplexorsimpleempty%
%   {\definewithnameofcommand\complexorsimpleempty}
%
% \endTEX
%
% \beginETEX \protected
%
% \def\definecomplexorsimple#1%
%   {\normalprotected\def#1{\complexorsimple#1}}
%
% \def\definecomplexorsimpleempty#1%
%   {\normalprotected\def#1{\complexorsimpleempty#1}}
%
% \endETEX

% However, since this one uses an idirect method, things go
% okay (at the cost of extra macros).

% \def\definecomplexorsimple#1%
%   {\unexpanded\def#1{\complexorsimple#1}}
%
% \def\definecomplexorsimpleempty#1%
%   {\unexpanded\def#1{\complexorsimpleempty#1}}
%
% faster, since no \strippedcsname needed in call, but more spacy

\def\docomplexorsimple#1#2%
  {\doifnextcharelse[{\firstargumenttrue#1}{\firstargumentfalse#2}}

\def\docomplexorsimpleempty#1%
  {\doifnextcharelse[{\firstargumenttrue#1}{\firstargumentfalse#1[]}}

\def\definecomplexorsimple#1%
  {\unexpanded\edef#1%
     {\noexpand\docomplexorsimple
        \@EA\noexpand\csname\s!complex\strippedcsname#1\endcsname
        \@EA\noexpand\csname\s!simple \strippedcsname#1\endcsname}}

\def\definecomplexorsimpleempty#1%
  {\unexpanded\edef#1%
     {\noexpand\docomplexorsimpleempty
        \@EA\noexpand\csname\s!complex\strippedcsname#1\endcsname}}

%D These commands are called as:
%D
%D \starttyping
%D \definecomplexorsimple\command
%D \stoptyping
%D
%D Of course, we must have available
%D
%D \starttyping
%D \def\complexcommand[#1]{...}
%D \def\simplecommand     {...}
%D \stoptyping
%D
%D Using this construction saves a few string now and then.

%D \macros
%D   {definestartstopcommand}
%D
%D Those who get the creeps of expansion may skip the next
%D one. It's one of the most recent additions and concerns
%D \type{\start}||\type{\stop} pairs with complicated
%D arguments.
%D
%D We won't go into details here, but the general form of
%D this using this command is:
%D
%D \starttyping
%D \definestartstopcommand\somecommand\e!specifier{arg}{arg}%
%D   {do something with arg}
%D \stoptyping
%D
%D This expands to something like:
%D
%D \starttyping
%D \def\somecommand arg \startspecifier arg \stopspecifier%
%D   {do something with arg}
%D \stoptyping
%D
%D The arguments can be anything reasonable, but double
%D \type{#}'s are needed in the specification part, like:
%D
%D \starttyping
%D \definestartstopcommand\somecommand\e!specifier{[##1][##2]}{##3}%
%D   {do #1 something #2 with #3 arg}
%D \stoptyping
%D
%D which becomes:
%D
%D \starttyping
%D \def\somecommand[#1][#2]\startspecifier#3\stopspecifier%
%D   {do #1 something #2 with #3 arg}
%D \stoptyping
%D
%D We will see some real applications of this command in the
%D core modules.

\def\definestartstopcommand#1#2#3#4% can be done with \expanded ot better, toks
  {\def\!stringa{#3}%              % but let's keep this unused one crappy
   \def\!stringb{\e!start#2}%
   \def\!stringc{#4}%
   \def\!stringd{\e!stop#2}%
   \@EA\@EA\@EA\@EA\@EA\@EA\@EA\@EA\@EA\@EA\@EA\@EA\@EA\@EA\@EA
   \def\@EA\@EA\@EA\@EA\@EA\@EA\@EA\@EA\@EA\@EA\@EA\@EA\@EA\@EA\@EA
   #1\@EA\@EA\@EA\@EA\@EA\@EA\@EA
   \!stringa\@EA\@EA\@EA
   \csname\@EA\@EA\@EA\!stringb\@EA\@EA\@EA\endcsname\@EA
   \!stringc
   \csname\!stringd\endcsname}

%D \macros
%D   {dosinglegroupempty,dodoublegroupempty,dotriplegroupempty,
%D    doquadruplegroupempty, doquintuplegroupempty}
%D
%D We've already seen some commands that take care of
%D optional arguments between \type{[]}. The next two commands
%D handle the ones with \type{{}}. They are called as:
%D
%D \starttyping
%D \dosinglegroupempty    \ineedONEargument
%D \dodoublegroupempty    \ineedTWOarguments
%D \dotriplegroupempty    \ineedTHREEarguments
%D \doquadruplegroupempty \ineedFOURarguments
%D \doquintuplegroupempty \ineedFIVEarguments
%D \stoptyping
%D
%D where \type{\ineedONEargument} takes one and the others
%D two and three arguments. These macro's were first needed in
%D \PPCHTEX.
%D
%D \starttyping
%D \def\dogetgroupargument#1#2% redefined in mult-ini
%D   {\def\nextnextargument%
%D      {\ifx\nextargument\bgroup
%D         \let\expectedarguments\noexpectedarguments
%D         \def\nextargument{#1\dodogetargument}%
%D       %\else\ifx\nextargument\lineending % this can be an option
%D       %  \def\nextargument{\begingroup\def\\ {\endgroup\dogetgroupargument#1#2}\\}%
%D       %\else\ifx\nextargument\blankspace % but it may never be default
%D       %  \def\nextargument{\begingroup\def\\ {\endgroup\dogetgroupargument#1#2}\\}%
%D       \else
%D         \ifnum\expectedarguments>\noexpectedarguments
%D           \writestatus
%D              {setup}
%D              {\the\expectedarguments\space argument(s) expected
%D               in line \the\inputlineno\space}%
%D         \fi
%D         \let\expectedarguments\noexpectedarguments
%D         \def\nextargument{#2\dodogetargument{}}%
%D       \fi%\fi\fi                 % so let's get rid of it
%D       \nextargument}%
%D    \futurelet\nextargument\nextnextargument}
%D \stoptyping
%D
%D In order to catch \type {\nextargument}'s that expand to
%D \type {\if} and friends, in practice we will use a
%D slightly more complicated macro.

\let\normalif         \if
\let\normalifcat      \ifcat
\let\normalifnum      \ifnum
\let\normalifdim      \ifdim
\let\normalifodd      \ifodd
\let\normalifvmode    \ifvmode
\let\normalifhmode    \ifhmode
\let\normalifmmode    \ifmmode
\let\normalifinner    \ifinner
\let\normalifvoid     \ifvoid
\let\normalifhbox     \ifhbox
\let\normalifvbox     \ifvbox
\let\normalifx        \ifx
\let\normalifeof      \ifeof
\let\normaliftrue     \iftrue
\let\normaliffalse    \iffalse
\let\normalifcase     \ifcase
\let\normalifdefined  \ifdefined
\let\normalifcsname   \ifcsname
\let\normaliffontchar \iffontchar
\let\normalifincsname \ifincsname
\let\normalifprimitive\ifprimitive
\let\normalifabsnum   \ifabsnum
\let\normalifabsdim   \ifabsdim

\let\normalelse       \else
\let\normalor         \or
\let\normalfi         \fi

\newtoks \everyrobusttest

\everyrobusttest
  {\let\if         \relax
   \let\ifcat      \relax
   \let\ifnum      \relax
   \let\ifdim      \relax
   \let\ifodd      \relax
   \let\ifvmode    \relax
   \let\ifhmode    \relax
   \let\ifmmode    \relax
   \let\ifinner    \relax
   \let\ifvoid     \relax
   \let\ifhbox     \relax
   \let\ifvbox     \relax
   \let\ifx        \relax
   \let\ifeof      \relax
   \let\iftrue     \relax
   \let\iffalse    \relax
   \let\ifcase     \relax
   \let\ifdefined  \relax
   \let\ifcsname   \relax
   \let\iffontchar \relax
   \let\ifincsname \relax
   \let\ifprimitive\relax
   \let\ifabsnum   \relax
   \let\ifabsdim   \relax
   \let\else       \relax
   \let\or         \relax
   \let\fi         \relax}

\def\beginrobusttest
  {\begingroup
   \the\everyrobusttest}

\let\endrobusttest\endgroup

%D We can add additional definitions later when we have defined
%D \type {\appendtoks}.

\def    \permitspacesbetweengroups{\chardef\@@permitspacesbetweengroups=0 }
\def\dontpermitspacesbetweengroups{\chardef\@@permitspacesbetweengroups=1 }

\dontpermitspacesbetweengroups

\def\dogetgroupargument#1#2%
  {\def\nextnextargument%
     {\normalifx\nextargument\bgroup
        \endrobusttest
        \noshowargumenterror
        \def\nextargument{#1\dodogetargument}%
      \normalelse
        \normalifcase\@@permitspacesbetweengroups
          \normalifx\nextargument\lineending
            \endrobusttest
            \def\nextargument{\begingroup\def\\ {\endgroup\dogetgroupargument#1#2}\\}%
          \normalelse\normalifx\nextargument\blankspace
            \endrobusttest
            \def\nextargument{\begingroup\def\\ {\endgroup\dogetgroupargument#1#2}\\}%
          \normalelse
            \endrobusttest
            \doshowargumenterror
            \def\nextargument{#2\dodogetargument{}}%
          \normalfi\normalfi
        \normalelse
          \endrobusttest
          \doshowargumenterror
          \def\nextargument{#2\dodogetargument{}}%
        \normalfi
      \normalfi
      \nextargument}%
   \beginrobusttest
   \futurelet\nextargument\nextnextargument}

\def\dosinglegroupempty#1%
  {\def\dodogetargument%
     {\dontpermitspacesbetweengroups
      #1}%
   \dogetgroupargument\firstargumenttrue\firstargumentfalse}

\def\dodoublegroupempty#1%
  {\def\dodogetargument##1%
     {\def\dodogetargument%
        {\dontpermitspacesbetweengroups
         #1{##1}}%
      \dogetgroupargument\secondargumenttrue\secondargumentfalse}%
   \dogetgroupargument\firstargumenttrue\firstargumentfalse}

\def\dotriplegroupempty#1%
  {\def\dodogetargument##1%
     {\def\dodogetargument####1%
        {\def\dodogetargument%
           {\dontpermitspacesbetweengroups
            #1{##1}{####1}}%
         \dogetgroupargument\thirdargumenttrue\thirdargumentfalse}%
      \dogetgroupargument\secondargumenttrue\secondargumentfalse}%
   \dogetgroupargument\firstargumenttrue\firstargumentfalse}

\def\doquadruplegroupempty#1%
  {\def\dodogetargument##1%
     {\def\dodogetargument####1%
        {\def\dodogetargument########1%
           {\def\dodogetargument%
              {\dontpermitspacesbetweengroups
               #1{##1}{####1}{########1}}%
            \dogetgroupargument\fourthargumenttrue\fourthargumentfalse}%
         \dogetgroupargument\thirdargumenttrue\thirdargumentfalse}%
      \dogetgroupargument\secondargumenttrue\secondargumentfalse}%
   \dogetgroupargument\firstargumenttrue\firstargumentfalse}

\def\doquintuplegroupempty#1%
  {\def\dodogetargument##1%
     {\def\dodogetargument####1%
        {\def\dodogetargument########1%
           {\def\dodogetargument################1%
             {\def\dodogetargument%
                {\dontpermitspacesbetweengroups
                 #1{##1}{####1}{########1}{################1}}%
              \dogetgroupargument\fifthargumenttrue\fifthargumentfalse}%
            \dogetgroupargument\fourthargumenttrue\fourthargumentfalse}%
         \dogetgroupargument\thirdargumenttrue\thirdargumentfalse}%
      \dogetgroupargument\secondargumenttrue\secondargumentfalse}%
   \dogetgroupargument\firstargumenttrue\firstargumentfalse}

%D These macros can explictly take care of spaces, which means
%D that the next definition and calls are valid:
%D
%D \starttyping
%D \def\test#1#2#3{[#1#2#3]}
%D
%D \dotriplegroupempty\test {a}{b}{c}
%D \dotriplegroupempty\test {a}{b}
%D \dotriplegroupempty\test {a}
%D \dotriplegroupempty\test
%D \dotriplegroupempty\test {a} {b} {c}
%D \dotriplegroupempty\test {a} {b}
%D \dotriplegroupempty\test
%D   {a}
%D   {b}
%D \stoptyping
%D
%D And alike.

%D \macros
%D   {firstofoneargument, firstoftwoarguments, firstofthreearguments
%D    secondoftwoarguments, secondofthreearguments,
%D    thirdofthreearguments}
%D
%D The next six macros (dedicated to Taco) can conveniently
%D used to select arguments. Their names explain their
%D functionality.

\long\def\firstofoneargument            #1{#1}
\long\def\firstoftwoarguments         #1#2{#1}
\long\def\firstofthreearguments     #1#2#3{#1}
\long\def\firstoffourarguments    #1#2#3#4{#1}
\long\def\firstoffivearguments  #1#2#3#4#5{#1}
\long\def\secondoftwoarguments        #1#2{#2}
\long\def\secondofthreearguments    #1#2#3{#2}
\long\def\secondoffourarguments   #1#2#3#4{#2}
\long\def\secondoffivearguments #1#2#3#4#5{#2}
\long\def\thirdofthreearguments     #1#2#3{#3}
\long\def\thirdoffourarguments    #1#2#3#4{#3}
\long\def\thirdoffivearguments  #1#2#3#4#5{#3}
\long\def\fourthoffourarguments   #1#2#3#4{#4}
\long\def\fourthoffivearguments #1#2#3#4#5{#4}
\long\def\fifthoffivearguments  #1#2#3#4#5{#5}

%D \macros
%D   {globalletempty,letempty,letvalueempty,letgvalueempty}
%D
%D Trivial:

\def\letempty             #1{\let#1\empty}
\def\globalletempty#1{\global\let#1\empty}

\def\letvalueempty        #1{\expandafter\let\csname#1\endcsname\empty}
\def\letgvalueempty#1{\global\expandafter\let\csname#1\endcsname\empty}

%D \macros
%D   {wait}
%D
%D The next macro hardly needs explanation. Because no
%D nesting is to be expected, we can reuse \type{\wait} within
%D \type{\wait} itself.

\def\wait
  {\begingroup
   \read16 to \wait
   \endgroup}

%D \macros
%D   {writestring,writeline,
%D    writestatus,statuswidth}
%D
%D Maybe one didn't notice, but we've already introduced a
%D macro for showing messages. In the multi||lingual modules,
%D we will also introduce a mechanism for message passing. For
%D the moment we stick to the core macros:
%D
%D \starttyping
%D \writestring {string}
%D \writeline
%D \writestatus {category} {message}
%D \stoptyping
%D
%D Messages are formatted. One can provide the maximum with
%D of the identification string with the macro \type
%D {\statuswidth}.

% \chardef\statuswidth=15
%
% \def\writestring%
%   {\immediate\write16}
%
% \def\writeline%
%   {\writestring{}}
%
% \def\dosplitstatus#1#2\end%
%   {\ifx#1?%
%      \loop
%        \advance\scratchcounter by 1
%        \ifnum\scratchcounter<\statuswidth
%          \edef\messagecontentA{\messagecontentA\space}%
%      \repeat
%    \else
%      \advance\scratchcounter by 1
%      \ifnum\scratchcounter<\statuswidth
%        \edef\messagecontentA{\messagecontentA#1}%
%      \fi
%      \dosplitstatus#2\end
%    \fi}
%
% \def\writestatus#1#2%
%   {\begingroup
%    \let\messagecontentA=\empty
%    \edef\messagecontentB{#2}%  maybe it's \the\scratchcounter
%    \scratchcounter=0
%    \expandafter\dosplitstatus#1?\end
%    \writestring{\messagecontentA\space:\space\messagecontentB}%
%    \endgroup}

\chardef\statuswidth=15
\chardef\statuswrite=16

\newtoks\everywritestring

\def\writestring#1%
  {\begingroup
   \the\everywritestring
   \immediate\write\statuswrite{#1}%
   \endgroup}

\def\writeline
  {\immediate\write\statuswrite{}}

%D First we present the normal \TEX\ variant, later we will
%D show the \ETEX-way.

\beginTEX

\newcount\statuscounter

\def\dosplitstatus#1%
  {\advance\statuscounter \minusone
   \ifcase\statuscounter
     \expandafter\nosplitstatus
   \else
     \scratchtoks\@EA{\the\scratchtoks#1}%
     \expandafter\dosplitstatus
   \fi}

\def\nosplitstatus#1\end
  {}

\def\writestatus#1#2%
  {\begingroup
   \scratchtoks\emptytoks
   \statuscounter\statuswidth
   \expandafter\dosplitstatus#1%
     \space\space\space\space\space\space\space
     \space\space\space\space\space\space\space
     \space\space\space\space\space\space\end
   \@EA\writestring\@EA{\the\scratchtoks\space:\space#2}%
   \endgroup}

\endTEX

%D Because we're grouped, we could have initialized at forehand:
%D
%D \starttyping
%D \statuscounter\statuswidth
%D \stoptyping
%D
%D The next implementation saves only some 10 words of format
%D memory, but we hardly gain any speed.
%D
%D \starttyping
%D \def\dosplitstatus#1#2#3#4#5#6#7#8#9%
%D   {#1#2#3#4#5#6#7#8#9\dodosplitstatus}
%D
%D \def\dodosplitstatus#1#2#3#4#5#6\end
%D   {#1#2#3#4#5}
%D
%D \def\writestatus#1#2%
%D   {\writestring
%D      {\expandafter\dosplitstatus#1%
%D         \space\space\space\space\space
%D         \space\space\space\space\space
%D         \space\space\space\space\space\end
%D         \space:\space#2}}
%D \stoptyping

%D Okay then, more obscure but slightly faster: no split grabs
%D the do split part and skipping the else branch has to happen
%D anyway, so:

\beginTEX

\def\dosplitstatus#1%
  {\advance\statuscounter \minusone
   \ifcase\statuscounter
     \expandafter\nosplitstatus
   \fi
   \scratchtoks\@EA{\the\scratchtoks#1}%
   \dosplitstatus}

\endTEX

%D The next (\ETEX\ specific) variant is twice as fast in 5/40
%D situations, only gains some speed when multiple runs of large docs
%D are done; fully expandable, no statuscounter needed, no restore (due
%D to grouping) needed etc.

\beginETEX \numexpr

\def\writestatus#1#2%
  {\writestring{\expandafter\dosplitstatus\expandafter\statuswidth#1%
     \space\space\space\space\space\space\space
     \space\space\space\space\space\space\space
     \space\space\space\space\space\space\end
     \space:\space#2}}

\def\dosplitstatus#1#2%
  {\ifcase#1 \expandafter\nosplitstatus\fi#2%
   \expandafter\dosplitstatus\expandafter{\the\numexpr#1+\minusone\relax}}

\def\nosplitstatus#1\end
  {}

\endETEX

%D \macros
%D   {emptytoks}
%D
%D For this we need an empty token register, analogous
%D to \type {\empty}.

\newtoks\emptytoks

%D \macros
%D   {debuggerinfo}
%D
%D For debugging purposes we can enhance macros with the
%D next alternative. Here \type{debuggerinfo} stands for both
%D a macro accepting two arguments and a boolean (in fact a
%D few macro's too).

\newif\ifdebuggerinfo

\def\debuggerinfo#1#2%
  {\ifdebuggerinfo
     \writestatus{debugger}{#1:: #2}%
   \fi}

%D Finally we do what from now on will be done at the top of
%D the files: we tell the user what we are loading.

\writestatus{loading}{Context System Macros / General}

%D Well, the real final command is the one that resets the
%D unprotected characters \type{@}, \type{?} and \type{!}.

\protect \endinput
