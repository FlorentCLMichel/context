% nagaan : \ifinstringelse in syst-ext.tex
% do => p! dodo pp! dododo ppp!

%D \module 
%D   [       file=syst-gen,
%D        version=1996.3.20,
%D          title=\CONTEXT\ System Macros,
%D       subtitle=General,
%D         author=Hans Hagen,
%D           date=\currentdate,
%D      copyright={PRAGMA / Hans Hagen \& Ton Otten}]
%C
%C This module is part of the \CONTEXT\ macro||package and is
%C therefore copyrighted by \PRAGMA. Non||commercial use is 
%C granted. 

%D The following macros are responsible for the interaction
%D with \CONTEXT. These macros have proven their use. These
%D macros are optimized as far as possible within of course,
%D the know how of the author. 
%D 
%D In this module we also show some of the optimizations, 
%D mainly because we don't want to forget them and start doing 
%D things over and over again. If showing them has a learing 
%D effect for others too, we've surved another purpose too.  

%D \macros
%D   {abortinputifdefined}
%D
%D Because this module can be used in a different context, we
%D want to prevent it being loaded more than once. This can be
%D done using:
%D
%D \starttypen
%D \abortinputifdefined\command
%D \stoptypen
%D
%D where \type{\command} is a command defined in the module
%D to be loaded only once.
%D 
%D \starttypen
%D \def\abortinputifdefined#1%
%D   {\ifx#1\undefined
%D      \let\next=\relax
%D    \else
%D      \let\next=\endinput
%D    \fi
%D    \next}
%D \stoptypen
%D
%D This macro can be speed up in terms of speed as well as 
%D memory. Because this is a nice example of a bit strange 
%D command (\type{\endinput}), we spend some more lines on this.
%D
%D If we perform such actions directly, we can say: 
%D 
%D \starttypen 
%D \ifx\somecommand\undefined
%D   \let\next=\relax
%D \else
%D   \let\next=\endinput
%D \fi
%D \next
%D \stoptypen
%D 
%D We need the \type{\next} because we need to end the 
%D \type{\fi}. The efficient one is: 
%D 
%D \starttypen 
%D \ifx\somecommand\undefined
%D \else
%D   \expandafter\endinput 
%D \fi
%D \stoptypen
%D
%D Because \type{\endinput} comes into action after the current 
%D line, we can also say:
%D 
%D \starttypen 
%D \ifx\somecommand\undefined \else \endinput \fi
%D \stoptypen
%D
%D When we define a macro, we tend to use a format which 
%D shows as besat as can how things are done. \TEX\ however 
%D stores the definitions as a sequence of tokens, so in fact
%D we can use a formatted definition: 

\def\abortinputifdefined#1%
  {\ifx#1\undefined \else 
     \endinput 
   \fi}

%D which also works. Keep in mind that this is entirely due to 
%D the fact that \type{\endinput} after the line, i.e. at the 
%D end of the macro. We therefore can burry this primitive quite
%D deep in code. 

%D And because this module implements \type{\writestatus}, we
%D just say:

\abortinputifdefined\writestatus

%D Normally we tell the users what module is being loaded.
%D However, the command that is needed for this is not yet
%D defined.
%D
%D \starttypen
%D \writestatus{laden}{Context Systeem Macro's (a)}
%D \stoptypen

%D \macros
%D   [beschermen]
%D   {protect,unprotect}
%D
%D We can shield macros from users by using some special
%D characters in their names. Some characters that are normally
%D no letters and therefore often used are: \type{@}, \type{!}
%D and \type{?}. Before and after the definition of protected
%D macros, we have to change the \CATCODE\ of these characters.
%D This is done by \type{\unprotect} and \type{\protect}, for
%D instance:
%D
%D \starttypen
%D \unprotect
%D \def\!test{test}
%D \protect
%D \stoptypen
%D
%D The defined command \type{\!test} can of course only be
%D called upon when we are in the \type{\unprotect}'ed state,
%D otherwise \TEX\ reads \type{\!} and probably complains
%D loudly about not being in math mode.
%D
%D Both commands can be used nested, but only the \CATCODE\
%D of the outermost level is saved. We make use of
%D an auxilary macro \type{\doprotect} to prevent us from
%D conflicts with existing macro's \type{\protect}. When
%D nesting deeper than one level, the system shows the
%D protection level.

\newcount\protectionlevel

\ifx\protect\undefined
  \def\protect{\message{<too much protection>}}
\fi

\let\normalprotect=\protect

%D Although we don't need the \type{%} after commands that
%D don't take arguments, unless lines are obeyed, I decided 
%D to put it there as a reminder. I only mention this once. 

\def\unprotect%
  {\ifcase\protectionlevel
     \edef\doprotectcharacters%
       {\catcode`\noexpand @=\the\catcode`@\relax
        \catcode`\noexpand !=\the\catcode`!\relax
        \catcode`\noexpand ?=\the\catcode`?\relax}%
     \catcode`@=11
     \catcode`!=11
     \catcode`?=11
     \let\protect\doprotect
   \fi
   \advance\protectionlevel 1
   \ifnum\protectionlevel>1
     \message{<unprotect \the\protectionlevel>}%
   \fi}

\def\doprotect%
  {\ifnum\protectionlevel=1
     \doprotectcharacters
     \let\protect\normalprotect
   \fi
   \ifnum\protectionlevel>1
     \message{<protect \the\protectionlevel>}%
   \fi
   \advance\protectionlevel -1\relax}

%D Now it is defined, we can make use of this very useful
%D macro.

\unprotect

%D \macros
%D   {@@escape,@@begingroup,@@endgroup,@@mathshift,@@alignment,
%D    @@endofline,@@parameter,@@superscript,@@subscript,
%D    @@ignore,@@space,@@letter,@@other,@@active,@@comment}
%D
%D In \CONTEXT\ we sometimes manipulate the \CATCODES\ of
%D certain characters. Because we are not that good at numbers,
%D we introduce some symbolic names.

\chardef\@@escape      =  0
\chardef\@@begingroup  =  1
\chardef\@@endgroup    =  2
\chardef\@@mathshift   =  3
\chardef\@@alignment   =  4
\chardef\@@endofline   =  5
\chardef\@@parameter   =  6
\chardef\@@superscript =  7
\chardef\@@subscript   =  8
\chardef\@@ignore      =  9
\chardef\@@space       = 10
\chardef\@@letter      = 11
\chardef\@@other       = 12   \chardef\other  = 12
\chardef\@@active      = 13   \chardef\active = 13
\chardef\@@comment     = 14

%D \macros
%D   {normalspace}
%D
%D We often need a space as defined in \PLAIN\ \TEX. Because 
%D we cannot be sure of \type{\space} is redefined, we define: 

\def\normalspace{ }

%D \macros
%D   {scratchcounter,
%D    scratchdimen,scratchskip,scratchmuskip,
%D    scratchbox,
%D    scratchtoks,
%D    ifdone}
%D
%D Because we often need counters on a temporary basis, we
%D define the \COUNTER\ \type{\scratchcounter}. This is a
%D real \COUNTER, and not a pseudo one, as we will meet
%D further on. We also define some other scratch registers.

\newcount  \scratchcounter
\newdimen  \scratchdimen
\newskip   \scratchskip
\newmuskip \scratchmuskip
\newbox    \scratchbox
\newtoks   \scratchtoks
\newif     \ifdone

%D \macros
%D   {ifCONTEXT}
%D 
%D In the system and support modules we sometimes show examples
%D that make use of core commands. We can skip those parts of
%D the documentation when we use another macropackage. Of 
%D course we default to false. 

\newif \ifCONTEXT

%D \macros
%D   {!!count, !!toks, !!dimen, !!box, 
%D    !!width, !!height, !!depth, !!string, !!done} 
%D
%D We define some more \COUNTERS\ and \DIMENSIONS. We also 
%D define some shortcuts to the local scatchregisters~0, 2, 4,
%D 6 and~8.

\newcount\!!counta \toksdef\!!toksa=0 \dimendef\!!dimena=0 \chardef\!!boxa=0
\newcount\!!countb \toksdef\!!toksb=2 \dimendef\!!dimenb=2 \chardef\!!boxb=2
\newcount\!!countc \toksdef\!!toksc=4 \dimendef\!!dimenc=4 \chardef\!!boxc=4
\newcount\!!countd \toksdef\!!toksd=6 \dimendef\!!dimend=6 \chardef\!!boxd=6
\newcount\!!counte \toksdef\!!tokse=8 \dimendef\!!dimene=8 \chardef\!!boxe=8
\newcount\!!countf

\let\!!stringa=\empty \let\!!stringb=\empty \let\!!stringc=\empty
\let\!!stringd=\empty \let\!!stringe=\empty \let\!!stringf=\empty

\newdimen\!!widtha \newdimen\!!heighta \newdimen\!!deptha 
\newdimen\!!widthb \newdimen\!!heightb \newdimen\!!depthb 

\newif\if!!donea   \newif\if!!doneb    \newif\if!!donec

%D \macros
%D   {s!,c!,e!,p!,v!,@@,??}
%D
%D To save memory, we use constants (sometimes called
%D variables). Redefining these constants can have desastrous
%D results.

\def\v!prefix! {v!}           \def\c!prefix! {c!}
\def\s!prefix! {s!}           \def\p!prefix! {p!} 

\def\s!next    {next}         \def\s!default {default}
\def\s!dummy   {dummy}        \def\s!unknown {unknown}

\def\s!do      {do}           \def\s!dodo    {dodo}

\def\s!complex {complex}      \def\s!start   {start}
\def\s!simple  {simple}       \def\s!stop    {stop}

%D \macros
%D   {@EA,expanded}
%D
%D When in unprotected mode, to be entered with 
%D \type{\unprotect}, one can use \type{\@EA} as equivalent 
%D of \type{\expandafter}.

\let\@EA=\expandafter

%D Sometimes we pass macros as arguments to commands that 
%D don't expand them before interpretation. Such commands can
%D be enclosed with \type{\expanded}, like: 
%D
%D \starttypen
%D \expanded{\setupsomething[\alfa]} 
%D \stoptypen
%D  
%D Such situations occur for instance when \type{\alfa} is a 
%D commalist or when data stored in macros is fed to index of
%D list commands. If needed, one should use \type{\noexpand} 
%D inside the argument. Later on we will meet some more clever 
%D alternatives to this command. 

\def\expanded#1%
  {\edef\@@expanded{\noexpand#1}\@@expanded}

%D \macros
%D   {gobbleoneargument,gobble...arguments}
%D
%D The next set of macros just do nothing, except that they
%D get rid of a number of arguments.

\long\def\gobbleoneargument    #1{}
\long\def\gobbletwoarguments   #1#2{}
\long\def\gobblethreearguments #1#2#3{}
\long\def\gobblefourarguments  #1#2#3#4{}
\long\def\gobblefivearguments  #1#2#3#4#5{}
\long\def\gobblesixarguments   #1#2#3#4#5#6{}
\long\def\gobblesevenarguments #1#2#3#4#5#6#7{}
\long\def\gobbleeightarguments #1#2#3#4#5#6#7#8{}
\long\def\gobbleninearguments  #1#2#3#4#5#6#7#8#9{}

%D \macros
%D   {doifnextcharelse}
%D
%D When we started using \TEX\ in the late eighties, our
%D first experiences with programming concerned a simple shell
%D around \LATEX. The commands probably use most at \PRAGMA,
%D are the itemizing ones. One of those few shell commands took
%D care of an optional argument, that enabled us to specify
%D what kind of item symbol we wanted. Without understanding
%D anything we were able to locate a \LATEX\ macro that could
%D be used to inspect the next character.
%D
%D It's this macro that the ancester of the next one presented
%D here. It executes one of two actions, dependant of the next
%D character. Disturbing spaces and line endings, which are
%D normally interpreted as spaces too, are skipped.
%D
%D \starttypen
%D \doifnextcharelse {karakter} {then ...} {else ...}
%D \stoptypen
%D
%D This macro differs from the original in testing on
%D \type{\endoflinetoken}, which of course we have to define
%D first. We also use \type{\localnext} because we don't want
%D clashes with \type{\next}.

\let\endoflinetoken=^^M

\long\def\doifnextcharelse#1#2#3%
   {\let\charactertoken=#1%
    \def\!!stringa{#2}%
    \def\!!stringb{#3}%
    \futurelet\nexttoken\inspectnextcharacter}

\def\inspectnextcharacter%
  {\ifx\nexttoken\blankspace
     \let\localnext\reinspectnextcharacter
   \else\ifx\!!stringc\endoflinetoken
     \let\localnext\reinspectnextcharacter
   \else\ifx\nexttoken\charactertoken
     \let\localnext\!!stringa
   \else
     \let\localnext\!!stringb
   \fi\fi\fi
   \localnext}

%D This macro uses some auxiliary macros. Although we were able
%D to program quite complicated things, I only understood these
%D after rereading the \TEX book. The trick is in using a
%D command with a one character name. Such commands differ from
%D the longer ones in the fact that trailing spaces are {\em
%D not} skipped. This enables us to indirectly define a long
%D named macro that gobbles a space.
%D
%D In the first line we define \type{\blankspace}. Next we
%D make \type{\:} equivalent to \type{\reinspect...}. This
%D one||character command is expanded before the next
%D \type{\def} comes into action. This way the space after
%D \type{\:} becomes a delimiter of the longer named
%D \type{\reinspectnextcharacter}. The chain reaction is
%D visually compatible with the next sequence:
%D
%D \starttypen
%D \expandafter\def\reinspectnextcharacter %
%D   {\futurelet\nexttoken\inspectnextcharacter}
%D \stoptypen
%D
%D However complicated it may look, I'm still glad I stumbled
%D into this construction.

\def\:{\let\blankspace= }  \:

\def\:{\reinspectnextcharacter}

\expandafter\def\: {\futurelet\nexttoken\inspectnextcharacter}

%D \macros
%D   {setvalue,setgvalue,setevalue,setxvalue,
%D    letvalue,
%D    getvalue,
%D    resetvalue}
%D
%D \TEX's primitive \type{\csname} can be used to construct
%D all kind of commands that cannot be defined with
%D \type{\def} and \type{\let}. Every macro programmer sooner
%D or later wants macros like these.
%D
%D \starttypen
%D \setvalue   {naam}{...} = \def\naam{...}
%D \setgvalue  {naam}{...} = \gdef\naam{...}
%D \setevalue  {naam}{...} = \edef\naam{...}
%D \setxvalue  {naam}{...} = \xdef\naam{...}
%D \letvalue   {naam}=\... = \let\naam=\...
%D \getvalue   {naam}      = \naam
%D \resetvalue {naam}      = \def\naam{}
%D \stoptypen
%D
%D As we will see, \CONTEXT\ uses these commands many times,
%D which is mainly due to its object oriented and parameter
%D driven character.

\def\setvalue#1%
  {\expandafter\def\csname#1\endcsname}

\def\setgvalue#1%
  {\expandafter\gdef\csname#1\endcsname}

\def\setevalue#1%
  {\expandafter\edef\csname#1\endcsname}

\def\setxvalue#1%
  {\expandafter\xdef\csname#1\endcsname}

\def\getvalue#1%
  {\csname#1\endcsname}

\def\letvalue#1%
  {\expandafter\let\csname#1\endcsname}

\def\resetvalue#1%
  {\expandafter\let\csname#1\endcsname\empty}

%D \macros
%D   {donottest,unexpanded}
%D
%D When expansion of a macro gives problems, we can precede it
%D by \type{\donottest}. It seems that protection is one of the
%D burdens of developers of packages, so maybe that's why in
%D \ETEX\ protection is solved in a more robust way.
%D
%D Sometimes prefixing the macro with \type{\donottest} leads
%D to defining an auxiliary macro, like
%D
%D \starttypen
%D \def\dosomecommand {... ... ...}
%D \def\somecommand   {\donottest\dosomecommand}
%D \stoptypen
%D
%D This double definition can be made transparant by using 
%D \type{\protecte}, as in:
%D 
%D \starttypen
%D \unexpanded\def\somecommand{... ... ...}
%D \stoptypen
%D 
%D The protection mechanism uses:

\beginTEX

\def\dontprocesstest#1% 
  {==}

\def\doprocesstest#1%
  {#1}

\let\donottest=\doprocesstest

\endTEX

\beginETEX \detokenize

\def\donottest#1{#1}%       \detokenize{#1}}

\endETEX

%D By the way, we use a placeholder because we don't want 
%D interference when testing on empty strings. Using a 
%D placeholder of 8~characters increases the processing time 
%D of simple \type{\doifelse} tests by about 10 \%. When we 
%D process the test, we have to remove the braces and 
%D therefore explictly gobble \type{#1}. 

%D The fact that many macros have the same prefix, could have
%D a negative impact on searching in the hash table. Because
%D some simple testing does not show differences, we just use:
%D 
%D \starttypen
%D \def\unexpanded#1#2%
%D   {\@EA#1\@EA#2\@EA{\@EA\donottest\csname\s!do\string#2\endcsname}%
%D    \@EA#1\csname\s!do\string#2\endcsname}
%D \stoptypen
%D 
%D Well, in fact we use the bit more versatile alternative: 

\beginTEX

\def\dosetunexpanded#1#2%
  {\@EA#1\@EA{\@EA#2\@EA}%
     \@EA{\@EA\donottest\csname\s!do\@EA\string\csname#2\endcsname\endcsname}%
   \@EA#1{\s!do\@EA\string\csname#2\endcsname}}

\def\docomunexpanded#1#2%
  {\@EA#1\@EA#2\@EA{\@EA\donottest\csname\s!do\string#2\endcsname}%
   \@EA#1\csname\s!do\string#2\endcsname}

\def\unexpanded#1%
  {\def\dounexpanded%
     {\ifx\next\bgroup
        \@EA\dosetunexpanded
      \else
        \@EA\docomunexpanded
      \fi#1}%
   \futurelet\next\dounexpanded}

\endTEX

\beginETEX \protected

\let\unexpanded\normalprotected

\endETEX

%D This one accepts the more direct \type{\def} and cousins 
%D as well as the \CONTEXT\ specific \type{\setvalue} ones. 
%D
%D And so the definition in our example turns out to be:
%D
%D \starttypen
%D \def\csname do\somecommand\endcsname{... ... ...}
%D \def\somecommand{\donottest\csname do\somecommand\endcsname}
%D \stoptypen
%D
%D In which \type{do\somecommand} is hidden from the user and 
%D cannot lead to confusion. It's still permitted to define 
%D auxiliary macros like \type{\dosomecommand}.  
%D
%D When we are going to use e-\TEX, we'll probably end up 
%D redefining some commands, but we can probably keep the 
%D \type{\unexpanded} ones unchanged. 

%D \macros
%D   {doifundefined,doifdefined,
%D    doifundefinedelse,doifdefinedelse,
%D    doifalldefinedelse}
%D
%D The standard way of testing if a macro is defined is
%D comparing its meaning with another undefined one, usually
%D \type{\undefined}. To garantee correct working of the next
%D set of macros, \type{\undefined} may never be defined!
%D
%D \starttypen
%D \doifundefined      {string}    {...}
%D \doifdefined        {string}    {...}
%D \doifundefinedelse  {string}    {then ...} {else ...}
%D \doifdefinedelse    {string}    {then ...} {else ...}
%D \doifalldefinedelse {commalist} {then ...} {else ...}
%D \stoptypen
%D
%D Every macroname that \TEX\ builds gets an entry in the hash
%D table, which is of limited size. It is expected that e-\TeX\
%D will offer a less memory||consuming alternative.

%D Although it will probably never be a big problem, it is good
%D to be aware of the difference between testing on a macro
%D name to be build by using \type{\csname} and
%D \type{\endcsname} and testing the \type{\name} directly. 
%D 
%D \starttypen
%D \expandafter\ifx\csname NameA\endcsname\relax ... \else ... \fi
%D 
%D \ifx\NameB\undefined ... \else ... \fi
%D \stoptypen
%D 
%D I became aware of this when I mistakenly testen the first
%D one against \type{\undefined}. When \TEX\ build a name using
%D \type{\csname} it automatically sets it to \type{\relax},
%D which is definitely not the same as \type{\undefined}. The
%D quickest way to check these things is asking \TEX\ to show
%D the meaning of the names: 
%D 
%D \starttypen
%D \expandafter\show\csname NameA\endcsname
%D 
%D \show\NameB
%D \stoptypen
%D
%D The main reason why this never will be a big problem is that
%D when one uses the \type{\csname} way, one probably has to do
%D with some macroname that always is dealt with that way.
%D Confusion can however arise when one applies both testing
%D methods to the same macroname. By the way, the assignment
%D of \type{\relax} obeys grouping. 

%D The first one gets rid of \type{#1}, but still expands to
%D something and the second one expands to \type{#1}. Because
%D we accept arguments between \type{{}}, we have to get rid
%D of one level of braces.
%D
%D Our first implementation of \type{\ifundefined} was
%D straightforward and readable:
%D
%D \starttypen
%D \def\ifundefined#1%
%D   {\expandafter\ifx\csname#1\endcsname\relax}%
%D
%D \def\doifundefinedelse#1#2#3%
%D   {\let\donottest=\dontprocesstest
%D    \ifundefined{#1}%
%D      \let\donottest=\doprocesstest#2%
%D    \else
%D      \let\donottest=\doprocesstest#3%
%D    \fi}
%D
%D \def\doifdefinedelse#1#2#3%
%D   {\doifundefinedelse{#1}{#3}{#2}}
%D
%D \def\doifundefined#1#2%
%D   {\doifundefinedelse{#1}{#2}{}}
%D
%D \def\doifdefined#1#2%
%D   {\doifundefinedelse{#1}{}{#2}}
%D
%D \def\doifalldefinedelse#1#2#3%
%D   {\bgroup
%D    \donetrue
%D    \def\checkcommand##1%
%D      {\doifundefined{##1}{\donefalse}}%
%D    \processcommalist[#1]\checkcommand
%D    \ifdone
%D      \egroup#2%
%D    \else
%D      \egroup#3%
%D    \fi}
%D \stoptypen
%D
%D When this module was optimized, timing showed that the
%D next alternative can be upto twice as fast, especially when
%D longer arguments are used.

\beginTEX

\def\ifundefined#1%
  {\expandafter\ifx\csname#1\endcsname\relax}

\def\p!doifundefined#1%
  {\let\donottest\dontprocesstest
   \expandafter\ifx\csname#1\endcsname\relax}

\def\doifundefinedelse#1#2#3%  
  {\p!doifundefined{#1}%
     \let\donottest\doprocesstest#2%
   \else
     \let\donottest\doprocesstest#3%
   \fi}

\def\doifdefinedelse#1#2#3%
  {\p!doifundefined{#1}%
     \let\donottest\doprocesstest#3%
   \else
     \let\donottest\doprocesstest#2%
   \fi}

\def\doifundefined#1#2%
  {\p!doifundefined{#1}%
     \let\donottest\doprocesstest#2%
   \else
     \let\donottest\doprocesstest
   \fi}

\def\doifdefined#1#2%
  {\p!doifundefined{#1}%
     \let\donottest\doprocesstest
   \else
     \let\donottest\doprocesstest#2%
   \fi}

\endTEX

\beginETEX \ifcsname

\def\ifundefined#1% ongelukkige naam 
  {\unless\ifcsname#1\endcsname}

\def\p!doifundefined#1%
  {\edef\p!defined{#1}% 
   \@EA\unless\@EA\ifcsname\@EA\detokenize\@EA{\p!defined}\endcsname}

\def\doifundefinedelse#1#2#3%  
  {\edef\p!defined{#1}%
   \@EA\ifcsname\@EA\detokenize\@EA{\p!defined}\endcsname#3\else#2\fi}

\def\doifdefinedelse#1#2#3%
  {\edef\p!defined{#1}%
   \@EA\ifcsname\@EA\detokenize\@EA{\p!defined}\endcsname#2\else#3\fi}

\def\doifundefined#1#2%  
  {\edef\p!defined{#1}%
   \@EA\ifcsname\@EA\detokenize\@EA{\p!defined}\endcsname\else#2\fi}

\def\doifdefined#1#2%
  {\edef\p!defined{#1}%
   \@EA\ifcsname\@EA\detokenize\@EA{\p!defined}\endcsname#2\fi}

\endETEX

%D \macros 
%D   {letbeundefined}
%D 
%D Testing for being undefined comes down to testing on \type
%D {\relax} when we use \type {\csname}, but when using \type
%D {\ifx}, we test on being \type {\undefined}! In \ETEX\ we
%D have \type {\ifcsname} and that way of testing on existance
%D is not the same as the one described here. Therefore we 
%D introduce: 

\beginTEX

\def\letbeundefined#1% 
  {\expandafter\let\csname#1\endcsname\relax}

\endTEX

\beginETEX

\def\letbeundefined#1% 
  {\expandafter\let\csname#1\endcsname\undefined}

\endETEX

%D Before we start using this variant, we used another one, 
%D which is even a bit faster. This one looked like: 
%D 
%D \starttypen
%D \def\p!doifundefined%
%D   {\begingroup
%D    \let\donottest=\dontprocesstest
%D    \ifundefined}
%D 
%D \def\doifundefinedelse#1#2#3%
%D   {\p!doifundefined{#1}%
%D      \endgroup#2%
%D    \else
%D      \endgroup#3%
%D    \fi}
%D \stoptypen
%D 
%D A even more previous version used \type{\bgroup} and 
%D \type{\egroup}. In math mode however, \type{$1{x}2$} differs
%D from \type{$1x2$}. This can been seen when one compares the 
%D output of: 
%D 
%D \starttypen 
%D $\kern10pt\showthe\lastkern$
%D $\kern10pt{\showthe\lastkern}$
%D $\kern10pt\begingroup\showthe\lastkern\endgroup$
%D \stoptypen
%D 
%D When we were developing the scientific units module, we 
%D encountered different behavior in text and math mode, which 
%D was due to this grouping subtilities. We therefore decided 
%D to use \type{\begingroup} instead of \type{\bgroup}. 
%D Later, when we had optimized some macro's the grouped 
%D solution turned out to be unsafe when typesetting this 
%D documentation, especially when using \type{\globaldefs}. 
%D
%D We still have to define \type{\doifalldefinedelse}. Watch 
%D the use of grouping, which garantees local use of the 
%D boolean \type{\ifdone}.

\beginTEX

\def\docheckonedefined#1%
  {\ifundefined{#1}%
     \donefalse
   \fi}

\def\doifalldefinedelse#1#2#3%
  {\begingroup
   \let\donottest\dontprocesstest
   \donetrue
   \processcommalist[#1]\docheckonedefined
   \ifdone
     \endgroup\let\donottest\doprocesstest#2%
   \else
     \endgroup\let\donottest\doprocesstest#3%
   \fi}

\endTEX

\beginETEX \ifcsname

\def\docheckonedefined#1%
  {\unless\ifcsname#1\endcsname
     \donefalse
   \fi}

\def\doifalldefinedelse#1#2#3%
  {\begingroup
   \donetrue
   \processcommalist[#1]\docheckonedefined
   \ifdone
     \endgroup#2%
   \else
     \endgroup#3%
   \fi}

\endETEX

%D \macros
%D   {doif,doifelse,doifnot,
%D    donottest}
%D
%D Programming in \TEX\ differs from programming in procedural
%D languages like \MODULA. This means that one --- well, let me
%D speek for myself --- tries to do the things in the well
%D known way. Therefore the next set of \type{\ifthenelse}
%D commands were between the first ones we needed. A few years
%D later, the opposite became true: when programming in
%D \MODULA, I sometimes miss handy things like grouping,
%D runtime redefinition, expansion etc. While \MODULA\ taught
%D me to structure, \TEX\ taught me to think recursive.
%D
%D \starttypen
%D \doif     {string1} {string2} {...}
%D \doifnot  {string1} {string2} {...}
%D \doifelse {string1} {string2} {then ...}{else ...}
%D \stoptypen
%D
%D When expansion gives problems, we can precede the
%D troublemaker with \type{\donottest}.
%D
%D This implementatie does not use the construction which is
%D more robust for nested conditionals.
%D
%D \starttypen
%D \ifx\!!stringa\!!stringb
%D   \def\next{#3}%
%D \else
%D   \def\next{#4}%
%D \fi
%D \next
%D \stoptypen
%D
%D In practice, this alternative is at least 20\% slower than
%D the alternative used here. The few cases in which we
%D really need the \type{\next} construction, often need some
%D other precautions and or adaptions too.

\beginTEX

\long\def\doif#1#2#3%
  {\let\donottest\dontprocesstest
   \edef\!!stringa{#1}%
   \edef\!!stringb{#2}%
   \let\donottest\doprocesstest
   \ifx\!!stringa\!!stringb
     #3%
   \fi}

\long\def\doifnot#1#2#3%
  {\let\donottest\dontprocesstest
   \edef\!!stringa{#1}%
   \edef\!!stringb{#2}%
   \let\donottest\doprocesstest
   \ifx\!!stringa\!!stringb
   \else
     #3%
   \fi}

\long\def\doifelse#1#2#3#4%
  {\let\donottest\dontprocesstest
   \edef\!!stringa{#1}%
   \edef\!!stringb{#2}%
   \let\donottest\doprocesstest
   \ifx\!!stringa\!!stringb
     #3%
   \else
     #4%
   \fi}

\endTEX

\beginETEX \protected

\long\def\doif#1#2#3%
  {\edef\!!stringa{#1}\edef\!!stringb{#2}%
   \ifx\!!stringa\!!stringb#3\fi}

\long\def\doifnot#1#2#3%
  {\edef\!!stringa{#1}\edef\!!stringb{#2}%
   \unless\ifx\!!stringa\!!stringb#3\fi}

\long\def\doifelse#1#2#3#4%
  {\edef\!!stringa{#1}\edef\!!stringb{#2}%
   \ifx\!!stringa\!!stringb#3\else#4\fi}

\endETEX

%D One could wonder why we don't follow the the same approach
%D as in \type{\doifdefined} c.s.\ and use \type{\begingroup} 
%D and \type{\endgroup}. In this case, this alternative is 
%D slower, which is probably due to the fact that more meanings
%D need to be restored.
%D
%D The in terms of memory more efficient alternative using a
%D auxiliary macro also proved to be slower, so we definitely
%D did not choose for:
%D
%D \starttypen
%D \def\p!doifelse#1#2%
%D   {\let\donottest=\dontprocesstest
%D    \edef\!!stringa{#1}%
%D    \edef\!!stringb{#2}%
%D    \let\donottest=\doprocesstest
%D    \ifx\!!stringa\!!stringb}
%D
%D \long\def\doif#1#2#3%
%D   {\p!doifelse{#1}{#2}#3\fi}
%D
%D \long\def\doifnot#1#2#3%
%D   {\p!doifelse{#1}{#2}\else#3\fi}
%D
%D \long\def\doifelse#1#2#3#4%
%D   {\p!doifelse{#1}{#2}#3\else#4\fi}
%D \stoptypen
%D
%D Optimizations like this are related of course to the
%D bottlenecks in \TEX. It seems that restoring saved meanings
%D and passing arguments takes some time.

%D \macros
%D   {doifempty,doifemptyelse,doifnotempty}
%D
%D We complete our set of conditionals with:
%D
%D \starttypen
%D \doifempty     {string} {...}
%D \doifnot       {string} {...}
%D \doifemptyelse {string} {then ...} {else ...}
%D \stoptypen
%D
%D This time, the string is not expanded.

\long\def\doifemptyelse#1#2#3%
  {\def\!!stringa{#1}%
   \ifx\!!stringa\empty
     #2%
   \else
     #3%
   \fi}

\long\def\doifempty#1#2%
  {\def\!!stringa{#1}%
   \ifx\!!stringa\empty
     #2%
   \fi}

\long\def\doifnotempty#1#2%
  {\def\!!stringa{#1}%
   \ifx\!!stringa\empty
   \else
     #2%
   \fi}

%D \macros
%D   {doifinset,doifnotinset,doifinsetelse}
%D
%D We can check if a string is present in a comma separated
%D set of strings. Depending on the result, some action is
%D taken.
%D
%D \starttypen
%D \doifinset     {string} {string,...} {...}
%D \doifnotinset  {string} {string,...} {...}
%D \doifinsetelse {string} {string,...} {then ...} {else ...}
%D \stoptypen
%D
%D The second argument is the comma separated set of strings.
%D
%D \starttypen
%D \long\def\doifinsetelse#1#2#3#4%
%D   {\doifelse{#1}{}
%D      {#4}
%D      {\donefalse
%D       \def\p!checkiteminset##1%
%D         {\doif{#1}{##1}
%D            {\donetrue
%D             \let\p!checkiteminset=\gobbleoneargument}}%
%D       \processcommalist[#2]\p!checkiteminset
%D       \ifdone
%D         #3%
%D       \else
%D         #4%
%D       \fi}}
%D
%D \long\def\doifinset#1#2#3%
%D   {\doifinsetelse{#1}{#2}{#3}{}}
%D
%D \long\def\doifnotinset#1#2#3%
%D   {\doifinsetelse{#1}{#2}{}{#3}}
%D \stoptypen
%D
%D Because this macro is called quite often we've spent some
%D time optimizing it. This time, the gain in speed is due to
%D (1)~defining an external auxiliary macro, (2)~not calling
%D any other macros and (3)~minimizing the passing of
%D arguments. The gain in speed is impressive.

% \def\p!dodocheckiteminset#1%
%   {\edef\!!stringb{#1}%
%    \ifx\!!stringa\!!stringb
%      \donetrue
%      \let\p!docheckiteminset\gobbleoneargument
%    \fi}
% 
% \beginTEX
% 
% \def\p!doifinsetelse#1#2%
%   {\let\donottest\dontprocesstest
%    \donefalse
%    \edef\!!stringa{#1}%
%    \ifx\!!stringa\empty
%    \else
%      \let\p!docheckiteminset\p!dodocheckiteminset
%      \processcommalist[#2]\p!docheckiteminset
%    \fi
%    \let\donottest\doprocesstest
%    \ifdone}
% 
% \endTEX
% 
% \beginETEX \protected
% 
% \def\p!doifinsetelse#1#2%
%   {\donefalse
%    \edef\!!stringa{#1}%
%    \ifx\!!stringa\empty
%    \else
%      \let\p!docheckiteminset\p!dodocheckiteminset
%      \processcommalist[#2]\p!docheckiteminset
%    \fi
%    \ifdone}
% 
% \endETEX

\def\p!docheckiteminset#1%
  {\edef\!!stringb{#1}%
   \ifx\!!stringa\!!stringb
     \donetrue
     \expandafter\quitcommalist
   \fi}

\beginTEX

\def\p!doifinsetelse#1#2%
  {\let\donottest\dontprocesstest
   \donefalse
   \edef\!!stringa{#1}%
   \ifx\!!stringa\empty
   \else
     \processcommalist[#2]\p!docheckiteminset
   \fi
   \let\donottest\doprocesstest
   \ifdone}

\endTEX

\beginETEX \protected

\def\p!doifinsetelse#1#2%
  {\donefalse
   \edef\!!stringa{#1}%
   \ifx\!!stringa\empty
   \else
     \processcommalist[#2]\p!docheckiteminset
   \fi
   \ifdone}

\endETEX

\long\def\doifinsetelse#1#2#3#4%
  {\p!doifinsetelse{#1}{#2}#3\else#4\fi}

\long\def\doifinset#1#2#3%
  {\p!doifinsetelse{#1}{#2}#3\fi}

\long\def\doifnotinset#1#2#3%
  {\p!doifinsetelse{#1}{#2}\else#3\fi}

%D \macros
%D   {doifcommon,doifnotcommon,doifcommonelse}
%D
%D Probably the most time consuming tests are those that test
%D for overlap in sets of strings.
%D
%D \starttypen
%D \doifcommon     {string,...} {string,...} {...}
%D \doifnotcommon  {string,...} {string,...} {...}
%D \doifcommonelse {string,...} {string,...} {then ...} {else ...}
%D \stoptypen
%D
%D We show the slower alternative first, because it shows us
%D how things are done.
%D
%D \starttypen
%D \long\def\doifcommonelse#1#2#3#4%
%D   {\donefalse
%D    \def\p!docommoncheck##1%
%D      {\def\p!dodocommoncheck####1%
%D         {\doif{####1}{##1}
%D            {\donetrue
%D             \def\commalistelement{##1}%
%D             \let\p!docommoncheck=\gobbleoneargument
%D             \let\p!dodocommoncheck=\gobbleoneargument}}%
%D       \processcommalist[#2]\p!dodocommoncheck}%
%D    \processcommalist[#1]\p!docommoncheck
%D    \ifdone
%D      #3%
%D    \else
%D      #4%
%D    \fi}
%D
%D \long\def\doifcommon#1#2#3%
%D   {\doifcommonelse{#1}{#2}{#3}{}}
%D
%D \long\def\doifnotcommon#1#2#3%
%D   {\doifcommonelse{#1}{#2}{}{#3}}
%D \stoptypen
%D
%D The processing time is shortened by getting the auxiliary
%D macro to the outermost level and using less \type{\edef}'s.
%D Sometimes it makes more sence to define local macro's not
%D only because this way we can be sure that they are not
%D redefined, but also because it shows the dependance. In
%D compiled languages, this is no problem at all. It can even
%D save us bytes and processing time. In interpreted languages
%D like \TEX\ it nearly always slows down processing.

% \def\p!dododocommoncheck#1%
%   {\edef\!!stringb{#1}%
%    \ifx\!!stringa\!!stringb
%      \donetrue
%      \let\p!docommoncheck\gobbleoneargument
%      \let\p!dodocommoncheck\gobbleoneargument
%    \fi}
% 
% \beginTEX
% 
% \def\p!doifcommonelse#1#2%
%   {\donefalse
%    \let\donottest\dontprocesstest
%    \let\p!dodocommoncheck\p!dododocommoncheck
%    \def\p!docommoncheck##1%
%      {\edef\!!stringa{##1}%
%       \def\commalistelement{##1}%
%       \processcommalist[#2]\p!dodocommoncheck}%
%    \processcommalist[#1]\p!docommoncheck
%    \let\donottest\doprocesstest
%    \ifdone}
% 
% \endTEX
% 
% \beginETEX \protected
% 
% \def\p!doifcommonelse#1#2%
%   {\donefalse
%    \let\p!dodocommoncheck\p!dododocommoncheck
%    \def\p!docommoncheck##1%
%      {\edef\!!stringa{##1}%
%       \def\commalistelement{##1}%
%       \processcommalist[#2]\p!dodocommoncheck}%
%    \processcommalist[#1]\p!docommoncheck
%    \ifdone}
% 
% \endETEX

\def\p!dodocommoncheck#1%
  {\edef\!!stringb{#1}%
   \ifx\!!stringa\!!stringb
     \donetrue
     \expandafter\quitprevcommalist
   \fi}

\beginTEX

\def\p!doifcommonelse#1#2%
  {\donefalse
   \let\donottest\dontprocesstest
   \def\p!docommoncheck##1%
     {\edef\!!stringa{##1}%
      \def\commalistelement{##1}%
      \processcommalist[#2]\p!dodocommoncheck}%
   \processcommalist[#1]\p!docommoncheck
   \let\donottest\doprocesstest
   \ifdone}

\endTEX

\beginETEX \protected

\def\p!doifcommonelse#1#2%
  {\donefalse
   \def\p!docommoncheck##1%
     {\edef\!!stringa{##1}%
      \def\commalistelement{##1}%
      \processcommalist[#2]\p!dodocommoncheck}%
   \processcommalist[#1]\p!docommoncheck
   \ifdone}

\endETEX

\long\def\doifcommonelse#1#2#3#4%
  {\p!doifcommonelse{#1}{#2}#3\else#4\fi}

\long\def\doifcommon#1#2#3%
  {\p!doifcommonelse{#1}{#2}#3\fi}

\long\def\doifnotcommon#1#2#3%
  {\p!doifcommonelse{#1}{#2}\else#3\fi}

%D \macros
%D   {processcommalist,processcommacommand,quitcommalist,
%D    processcommalistwithparameters}
%D
%D We've already seen some macros that take care of comma
%D separated lists. Such list can be processed with
%D
%D \starttypen
%D \processcommalist[string,string,...]\commando
%D \stoptypen
%D
%D The user supplied command \type{\commando} receives one
%D argument: the string. This command permits nesting and 
%D spaces after commas are skipped. Empty sets are no problem.
%D 
%D \startbuffer
%D \def\dosomething#1{(#1)}
%D
%D \processcommalist [\hbox{$a,b,c,d,e,f$}] \dosomething \par
%D \processcommalist [{a,b,c,d,e,f}]        \dosomething \par
%D \processcommalist [{a,b,c},d,e,f]        \dosomething \par
%D \processcommalist [a,b,{c,d,e},f]        \dosomething \par
%D \processcommalist [a{b,c},d,e,f]         \dosomething \par
%D \processcommalist [{a,b}c,d,e,f]         \dosomething \par
%D \processcommalist []                     \dosomething \par
%D \processcommalist [{[}]                  \dosomething \par
%D \stopbuffer
%D 
%D \typebuffer
%D
%D Before we show the result, we present the macro's: 

\newcount\commalevel

\def\dododoprocesscommaitem%
  {\csname\s!next\the\commalevel\endcsname}

\def\dodoprocesscommaitem%
  {\ifx\nexttoken\blankspace
     \let\nextcommaitem\redoprocesscommaitem
   \else\ifx\nexttoken]%
     \let\nextcommaitem\gobbleoneargument
   \else
     \let\nextcommaitem\dododoprocesscommaitem
   \fi\fi
   \nextcommaitem}

\def\doprocesscommaitem%
  {\futurelet\nexttoken\dodoprocesscommaitem}

\def\doprocesscommalistA#1#2]#3%
  {\global\advance\commalevel 1
   \long\expandafter\def\csname\s!next\the\commalevel\endcsname##1,%
     {#3{##1}\doprocesscommaitem}%
   \doprocesscommaitem{#1}#2,]\relax
   \global\advance\commalevel -1 }

\def\doprocesscommalistB#1]#2%
  {\global\advance\commalevel 1
   \long\expandafter\def\csname\s!next\the\commalevel\endcsname##1,%
     {#2{##1}\doprocesscommaitem}%
   \doprocesscommaitem#1,]\relax
   \global\advance\commalevel -1 }

%D Empty arguments are not processed. Empty items (\type{,,})
%D however are treated. We have to check for the special case
%D \type{[{a,b,c}]}.

\def\docheckcommaitem%
  {\ifx\nexttoken]%
     \let\nextcommaitem\gobbletwoarguments
   \else\ifx\nexttoken\bgroup
     \let\nextcommaitem\doprocesscommalistA
   \else
     \let\nextcommaitem\doprocesscommalistB
   \fi\fi
   \nextcommaitem}

\def\processcommalist[%
  {\futurelet\nexttoken\docheckcommaitem}

%D One way of quitting a commalist halfway is:

% \def\quitcommalist%
%   {\@EA\let\csname\s!next\the\commalevel\endcsname\gobbleoneargument}

\def\quitcommalist%
  {\bgroup\let\doprocesscommaitem\doquitcommalist}

\def\doquitcommalist#1]%
  {\egroup}

\def\quitprevcommalist%
  {\bgroup\let\doprocesscommaitem\doquitprevcommalist}

\def\doquitprevcommalist#1]%
  {\let\doprocesscommaitem\doquitcommalist}

%D The hack we used for checking the next character
%D \type {\doifnextcharelse} is also used here. 

\def\:{\redoprocesscommaitem}

\expandafter\def\: {\futurelet\nexttoken\dodoprocesscommaitem}

%D The previous examples lead to:
%D
%D \haalbuffer

%D When a list is saved in a macro, we can use a construction
%D like:
%D
%D \starttypen
%D \expandafter\processcommalist\expandafter[\list]\command
%D \stoptypen
%D
%D Such solutions suit most situations, but we wanted a bit
%D more.
%D
%D \starttypen
%D \processcommacommand[string,\stringset,string]\commando
%D \stoptypen
%D
%D where \type{\stringset} is a predefined set, like:
%D
%D \starttypen
%D \def\first{aap,noot,mies}
%D \def\second{laatste}
%D
%D \processcommacommand[\first]\message
%D \processcommacommand[\first,second,third]\message
%D \processcommacommand[\first,between,\second]\message
%D \stoptypen
%D
%D Commands that are part of the list are expanded, so the
%D use of this macro has its limits.

% why the \toks0? still needed? 

\def\processcommacommand[#1]%
  {\edef\commacommand{#1}% 
   \toks0=\expandafter{\expandafter[\commacommand]}%
   \expandafter\processcommalist\the\toks0 }

%D The argument to \type{\command} is not delimited. Because
%D we often use \type{[]} as delimiters, we also have:
%D
%D \starttypen
%D \processcommalistwithparameters[string,string,...]\command
%D \stoptypen
%D
%D where \type{\command} looks like:
%D
%D \starttypen
%D \def\command[#1]{... #1 ...}
%D \stoptypen

\def\processcommalistwithparameters[#1]#2%
  {\def\docommand##1{#2[##1]}%
   \processcommalist[#1]\docommand}

%D \macros
%D   {processaction,
%D    processfirstactioninset,
%D    processallactionsinset}
%D
%D \CONTEXT\ makes extensive use of a sort of case or switch
%D command. Depending of the presence of one or more provided
%D items, some actions is taken. These macros can be nested
%D without problems.
%D
%D \starttypen
%D \processaction           [x]     [a=>\a,b=>\b,c=>\c]
%D \processfirstactioninset [x,y,z] [a=>\a,b=>\b,c=>\c]
%D \processallactionsinset  [x,y,z] [a=>\a,b=>\b,c=>\c]
%D \stoptypen
%D
%D We can supply both a \type{default} action and an action
%D to be undertaken when an \type{unknown} value is met:
%D
%D \starttypen
%D \processallactionsinset
%D   [x,y,z]
%D   [      a=>\a,
%D          b=>\b,
%D          c=>\c,
%D    default=>\default,
%D    unknown=>\unknown{... \commalistelement ...}]
%D \stoptypen
%D 
%D When \type{#1} is empty, this macro scans list \type{#2} for
%D the keyword \type{default} and executed the related action
%D if present. When \type{#1} is non empty and not in the list,
%D the action related to \type{unknown} is executed. Both
%D keywords must be at the end of list \type{#2}. Afterwards,
%D the actually found keyword is available in
%D \type{\commalistelement}. An advanced example of the use of
%D this macro can be found in \PPCHTEX, where we completely
%D rely on \TEX\ for interpreting user supplied keywords like
%D \type{SB}, \type{SB1..6}, \type{SB125} etc. 
%D 
%D Even a quick glance at the macros below show some overlap,
%D which means that more efficient alternatives are possible.
%D Because these macro's are very sensitive to subtle changes,
%D we've decided to present the readable originals first
%D Maybe these these macros look complicated, but this is a
%D direct result of the support of nesting. Protection is only
%D applied in \type{\processaction}.
%D
%D \starttypen
%D \newcount\processlevel
%D
%D \def\processaction[#1]#2[#3]%
%D   {\doifelse{#1}{}
%D      {\def\p!compareprocessaction[##1=>##2]%
%D         {\edef\!!stringa{##1}%
%D          \ifx\!!stringa\s!default
%D            \def\commalistelement{#1}%
%D            ##2%
%D          \fi}}
%D      {\let\donottest=\dontprocesstest
%D       \edef\!!stringb{#1}%
%D       \let\donottest=\doprocesstest
%D       \def\p!compareprocessaction[##1=>##2]%
%D         {\edef\!!stringa{##1}%
%D          \ifx\!!stringa\!!stringb
%D            \def\commalistelement{#1}%
%D            ##2%
%D            \let\p!doprocessaction=\gobbleoneargument
%D          \else\ifx\!!stringa\s!unknown
%D            \def\commalistelement{#1}%
%D            ##2%
%D          \fi\fi}}%
%D    \def\p!doprocessaction##1%
%D      {\p!compareprocessaction[##1]}%
%D    \processcommalist[#3]\p!doprocessaction}
%D
%D \def\processfirstactioninset[#1]#2[#3]%
%D   {\doifelse{#1}{}
%D      {\processaction[][#3]}
%D      {\def\p!compareprocessaction[##1=>##2][##3]%
%D         {\edef\!!stringa{##1}%
%D          \edef\!!stringb{##3}%
%D          \ifx\!!stringa\!!stringb
%D            \def\commalistelement{##3}%
%D            ##2%
%D            \let\p!doprocessaction=\gobbleoneargument
%D            \let\p!dodoprocessaction=\gobbleoneargument
%D          \else\ifx\!!stringa\s!unknown
%D            \def\commalistelement{##3}%
%D            ##2%
%D          \fi\fi}%
%D       \def\p!doprocessaction##1%
%D         {\def\p!dodoprocessaction####1%
%D            {\p!compareprocessaction[####1][##1]}%
%D          \processcommalist[#3]\p!dodoprocessaction}%
%D       \processcommalist[#1]\p!doprocessaction}}
%D
%D \def\processallactionsinset[#1]#2[#3]%
%D   {\doifelse{#1}{}
%D      {\processaction[][#3]}
%D      {\advance\processlevel by 1\relax
%D       \def\p!compareprocessaction[##1=>##2][##3]%
%D         {\edef\!!stringa{##1}%
%D          \edef\!!stringb{##3}%
%D          \ifx\!!stringa\!!stringb
%D            \def\commalistelement{##3}%
%D            ##2%
%D            \let\p!dodoprocessaction=\gobbleoneargument
%D          \else\ifx\!!stringa\s!unknown
%D            \def\commalistelement{##3}%
%D            ##2%
%D          \fi\fi}%
%D       \setvalue{\s!do\the\processlevel}##1%
%D         {\def\p!dodoprocessaction####1%
%D            {\p!compareprocessaction[####1][##1]}%
%D          \processcommalist[#3]\p!dodoprocessaction}%
%D       \processcommalist[#1]{\getvalue{\s!do\the\processlevel}}%
%D       \advance\processlevel by -1\relax}}
%D \stoptypen
%D
%D The gain of speed in the (again) next implementation is around
%D 20\%, depending on the application.

\newcount\processlevel

\def\p!compareprocessactionA[#1=>#2][#3]%
  {\edef\!!stringb{#1}%
   \ifx\!!stringb\s!default
     \let\commalistelement\empty
     #2%
   \fi}

% \def\p!compareprocessactionB[#1=>#2][#3]%
%   {\expandedaction\!!stringb{#1}%
%    \ifx\!!stringa\!!stringb
%      \def\commalistelement{#3}%
%      #2%
%      \let\p!doprocessaction\gobbleoneargument
%    \else
%      \edef\!!stringb{#1}%
%      \ifx\!!stringb\s!unknown
%        \def\commalistelement{#3}% beware of loops 
%        #2%
%      \fi
%    \fi}

% met \quitcommalist tot meer dan 25\% sneller 

\def\p!compareprocessactionB[#1=>#2][#3]%
  {\expandedaction\!!stringb{#1}%
   \ifx\!!stringa\!!stringb
     \def\commalistelement{#3}%
     #2%
     \expandafter\quitcommalist
   \else
     \edef\!!stringb{#1}%
     \ifx\!!stringb\s!unknown
       \def\commalistelement{#3}% beware of loops 
       #2%
     \fi
   \fi}

\beginTEX

\def\processaction[#1]#2[#3]%
  {\let\donottest\dontprocesstest
   \expandedaction\!!stringa{#1}%
   \let\donottest\doprocesstest
   \ifx\!!stringa\empty
     \let\p!compareprocessaction\p!compareprocessactionA
   \else
     \let\p!compareprocessaction\p!compareprocessactionB
   \fi
   \def\p!doprocessaction##1%
     {\p!compareprocessaction[##1][#1]}%
   \processcommalist[#3]\p!doprocessaction
   \expandactions}

\endTEX

\beginETEX \protected

\def\processaction[#1]#2[#3]%
  {\expandedaction\!!stringa{#1}%
   \ifx\!!stringa\empty
     \let\p!compareprocessaction\p!compareprocessactionA
   \else
     \let\p!compareprocessaction\p!compareprocessactionB
   \fi
   \def\p!doprocessaction##1%
     {\p!compareprocessaction[##1][#1]}%
   \processcommalist[#3]\p!doprocessaction
   \expandactions}

\endETEX

% \def\p!compareprocessactionC[#1=>#2][#3]%
%   {\expandedaction\!!stringa{#1}%
%    \expandedaction\!!stringb{#3}%
%    \ifx\!!stringa\!!stringb
%      \def\commalistelement{#3}%
%      #2%
%      \let\p!doprocessaction\gobbleoneargument
%      \let\p!dodoprocessaction\gobbleoneargument
%    \else
%      \edef\!!stringa{#1}%
%      \ifx\!!stringa\s!unknown
%        \def\commalistelement{#3}%
%        #2%
%      \fi
%    \fi}

\def\p!compareprocessactionC[#1=>#2][#3]%
  {\expandedaction\!!stringa{#1}%
   \expandedaction\!!stringb{#3}%
   \ifx\!!stringa\!!stringb
     \def\commalistelement{#3}%
     #2%
     \expandafter\quitprevcommalist
   \else
     \edef\!!stringa{#1}%
     \ifx\!!stringa\s!unknown
       \def\commalistelement{#3}%
       #2%
     \fi
   \fi}

\def\processfirstactioninset[#1]#2[#3]%
  {\expandedaction\!!stringa{#1}%
   \ifx\!!stringa\empty
     \processaction[][#3]%
   \else
     \def\p!doprocessaction##1%
       {\def\p!dodoprocessaction####1%
          {\p!compareprocessactionC[####1][##1]}%
        \processcommalist[#3]\p!dodoprocessaction}%
     \processcommalist[#1]\p!doprocessaction
   \fi
   \expandactions}

% \def\p!compareprocessactionD[#1=>#2][#3]%
%   {\expandedaction\!!stringa{#1}%
%    \expandedaction\!!stringb{#3}%
%    \ifx\!!stringa\!!stringb
%      \def\commalistelement{#3}%
%      #2%
%      \let\p!dodoprocessaction\gobbleoneargument
%    \else
%      \edef\!!stringa{#1}%
%      \ifx\!!stringa\s!unknown
%        \def\commalistelement{#3}%
%        #2%
%      \fi
%    \fi}

\def\p!compareprocessactionD[#1=>#2][#3]%
  {\expandedaction\!!stringa{#1}%
   \expandedaction\!!stringb{#3}%
   \ifx\!!stringa\!!stringb
     \def\commalistelement{#3}%
     #2%
     \expandafter\quitcommalist
   \else
     \edef\!!stringa{#1}%
     \ifx\!!stringa\s!unknown
       \def\commalistelement{#3}%
       #2%
     \fi
   \fi}

\def\doprocessallactionsinset%
  {\csname\s!do\the\processlevel\endcsname}

\def\processallactionsinset[#1]#2[#3]%
  {\expandedaction\!!stringa{#1}%
   \ifx\!!stringa\empty
     \processaction[][#3]%
   \else
     \advance\processlevel 1
     \expandafter\def\csname\s!do\the\processlevel\endcsname##1%
       {\def\p!dodoprocessaction####1%
          {\p!compareprocessactionD[####1][##1]}%
        \processcommalist[#3]\p!dodoprocessaction}%
     \processcommalist[#1]\doprocessallactionsinset
     \advance\processlevel -1
   \fi 
   \expandactions}

%D \macros
%D   {unexpandedprocessaction,
%D    unexpandedprocessfirstactioninset,
%D    unexpandedprocessallactionsinset}
%D
%D Now what are those expansion commands doing there. Well, 
%D sometimes we want to compare actions that may consist off
%D commands (i.e. are no constants). In such occasions we can 
%D use the a bit slower alternatives: 

\def\unexpandedprocessfirstactioninset{\dontexpandactions\processfirstactioninset}
\def\unexpandedprocessaction          {\dontexpandactions\processaction}
\def\unexpandedprocessallactionsinset {\dontexpandactions\processallactionsinset}

%D By default we expand actions: 

\def\expandactions%
  {\let\expandedaction\edef}

\expandactions

%D But when needed we convert the strings to meaningful 
%D sequences of characters.  

\def\unexpandedaction#1>{}

\def\noexpandedaction#1#2%
  {\def\convertedargument{#2}%
   \@EA\edef\@EA#1\@EA{\@EA\unexpandedaction\meaning\convertedargument}}

\def\dontexpandactions%
  {\let\expandedaction\noexpandedaction}

%D \macros
%D   {getfirstcharacter,firstcharacter}
%D
%D Sometimes the action to be undertaken depends on the
%D next character. This macro get this character and puts it in
%D \type{\firstcharacter}.
%D
%D \starttypen
%D \getfirstcharacter {string}
%D \stoptypen
%D
%D A two step expansion is used to prevent problems with
%D complicated arguments, for instance arguments that
%D consist of two or more expandable tokens.

\def\dogetfirstcharacter#1#2\\%
  {\def\firstcharacter{#1}}

\def\getfirstcharacter#1%
  {\edef\!!stringa{#1}%
   \expandafter\dogetfirstcharacter\!!stringa\\}

%D \macros
%D   {doifinstringelse, doifincsnameelse}
%D
%D We can check for the presence of a substring in a given
%D sequence of characters.
%D
%D \starttypen
%D \doifinsetelse {substring} {string} {then ...} {else ...}
%D \stoptypen
%D
%D An application of this command can be found further on.
%D Like before, we first show some alternatives, like the one
%D we started with:
%D
%D \starttypen
%D \long\def\p!doifinstringelse#1#2#3#4%
%D   {\def\pp!doifinstringelse##1#1##2##3\war%
%D      {\if##2@%
%D         #4%
%D       \else
%D         #3%
%D       \fi}%
%D    \pp!doifinstringelse#2#1@@\war}
%D
%D \def\doifinstringelse%
%D   {\ExpandBothAfter\p!doifinstringelse}
%D \stoptypen
%D
%D After this we came to:
%D
%D \starttypen
%D \def\p!doifinstringelse#1#2%
%D   {\def\pp!doifinstringelse##1#1##2##3\war%
%D      {\if##2@}%
%D    \pp!doifinstringelse#2#1@@\war}
%D
%D \def\doifinstringelse#1#2#3#4%
%D   {\ExpandBothAfter\p!doifinstringelse{#1}{#2}%
%D      #4%
%D    \else
%D      #3%
%D    \fi}
%D \stoptypen
%D 
%D Sometimes the second argument is passed as a macro. By
%D postponing the expansion of this macro, we gain quite some
%D run time, simply because the less tokens we pass, the faster
%D \TEX\ runs. So finally the definition became: 

\def\p!doifinstringelse#1#2%
  {\def\pp!doifinstringelse##1#1##2##3\war%
    %{\csname\if##2@iffalse\else iftrue\fi\endcsname}%
     {\csname if\if##2@fals\else tru\fi e\endcsname}%
   \expanded{\pp!doifinstringelse#2#1@@\noexpand\war}} % expand #2 here

\long\def\doifinstringelse#1#2#3#4%
  {\edef\@@@instring{#1}% expand #1 here 
   \@EA\p!doifinstringelse\@EA{\@@@instring}{#2}% 
     #3%
   \else
     #4%
   \fi}

%D The next alternative proved to be upto twice as fast on
%D tasks like checking reserved words in pretty verbatim
%D typesetting! This is mainly due to the fact that passing
%D (expanded) strings is much slower that passing a macro. 
%D
%D \starttypen
%D \doifincsnameelse {substring} {\string} {then ...} {else ...}
%D \stoptypen
%D 
%D Where \type{\doifinstringelse} does as much expansion as 
%D possible, the latter alternative does minimal (one level) 
%D expansion. 

\def\p!doifincsnameelse#1#2%
  {\def\pp!doifincsnameelse##1#1##2##3\war%
     {\csname\if##2@iffalse\else iftrue\fi\endcsname}%
   \@EA\pp!doifincsnameelse#2#1@@\war}

\long\def\doifincsnameelse#1#2#3#4%
  {\edef\@@@instring{#1}%
   \@EA\p!doifincsnameelse\@EA{\@@@instring}{#2}%
     #3%
   \else
     #4%
   \fi}

%D \macros
%D   {doifnumberelse}
%D
%D The next macro executes a command depending of the outcome
%D of a test on numerals. This is probably one of the fastest
%D test possible, exept from a less robust 10||step
%D \type{\if}||ladder or some tricky \type{\lcode} checking.  
%D
%D \starttypen
%D \doifnumberelse {string} {then ...} {else ...}
%D \stoptypen
%D
%D The macro accepts \type{123}, \type{abc}, \type{{}},
%D \type{\getal} and \type{\the\count...}. This macro is a 
%D reather dirty one.

\long\def\doifnumberelse#1#2#3%
  {\bgroup\donefalse\ifcase1#1\or\or\or\or\or\or\or\or\or\else\donetrue\fi
   \ifdone\egroup#2\else\egroup#3\fi}

%D The previous implementation was:
%D 
%D \starttypen
%D \long\def\doifnumberelse#1#2#3%
%D   {\getfirstcharacter{#1}%
%D    \@EA\p!doifinstringelse\@EA{\firstcharacter}{1234567890}%
%D      #2%
%D    \else
%D      #3%
%D    \fi}
%D \starttypen
%D 
%D Before we had \type{\p!doifinstringelse} available, we used:
%D
%D \starttypen
%D \def\doifnumberelse#1%
%D   {\getfirstcharacter{#1}%
%D    \rawdoifinsetelse{\firstcharacter}{1,2,3,4,5,6,7,8,9,0}}
%D \stoptypen
%D
%D The current implementation is much faster. 

%D \macros
%D   {makerawcommalist,
%D    rawdoinsetelse,
%D    rawprocesscommalist,
%D    rawprocessaction}
%D
%D Some of the commands mentioned earlier are effective but
%D slow. When one is desperately in need of faster alternatives
%D and when the conditions are predictable safe, the \type{\raw}
%D alternatives come into focus. A major drawback is that
%D they do not take \type{\c!constants} into account, simply
%D because no expansion is done. This is no problem with
%D \type{\rawprocesscommalist}, because this macro does not 
%D compare anything. Expandable macros are permitted as search
%D string.
%D
%D \starttypen
%D \makerawcommalist[string,string,...]\stringlist
%D \rawdoifinsetelse{string}{string,...}{...}{...}
%D \rawprocesscommalist[string,string,...]\commando
%D \rawprocessaction[x][a=>\a,b=>\b,c=>\c]
%D \stoptypen
%D
%D Spaces embedded in the list, for instance after commas,
%D spoil the search process. The gain in speed depends on the 
%D length of the argument (the longer the argument, the less 
%D we gain). 
%D
%D The slow alternative looks like:
%D
%D \starttypen
%D \def\makerawcommalist[#1]#2%
%D   {\def\appendtocommalist##1%
%D      {\doifelse{#2}{}
%D         {\edef#2{##1}}
%D         {\edef#2{#2,##1}}}%
%D    \def#2{}%
%D    \processcommalist[#1]\appendtocommalist}
%D \stoptypen
%D
%D But we prefer:

\def\makerawcommalist[#1]#2%
  {\def\appendtocommalist##1%
     {\edef#2{##1}\def\appendtocommalist####1{\edef#2{#2,####1}}}%
   \let#2\empty
   \processcommalist[#1]\appendtocommalist}

\def\rawprocesscommaitem#1,%
  {\if]#1\else
     \csname\s!next\the\commalevel\endcsname{#1}%
     \expandafter\rawprocesscommaitem
   \fi}

\def\rawprocesscommalist[#1]#2%
  {\global\advance\commalevel 1
   \expandafter\let\csname\s!next\the\commalevel\endcsname#2%
   \expandafter\rawprocesscommaitem#1,],% \relax
   \global\advance\commalevel -1 }

\def\rawdoifinsetelse#1#2%
  {\doifinstringelse{,#1,}{,#2,}}

\def\p!rawprocessaction[#1][#2]%
  {\def\pp!rawprocessaction##1,#1=>##2,##3\war%
     {\if##3@\else
        \def\!!processaction{##2}%
      \fi}%
   \pp!rawprocessaction,#2,#1=>,@\war}

\def\rawprocessaction[#1]#2[#3]%
  {\edef\!!stringa{#1}%
   \edef\!!stringb{undefined}%
   \let\!!processaction\!!stringb
   \ifx\!!stringa\empty
     \@EA\p!rawprocessaction\@EA[\s!default][#3]%
   \else
      \expandafter\p!rawprocessaction\expandafter[\!!stringa][#3]%
      \ifx\!!processaction\!!stringb
        \@EA\p!rawprocessaction\@EA[\s!unknown][#3]%
      \fi
   \fi
   \ifx\!!processaction\!!stringb
   \else
     \!!processaction
   \fi}

%D When we process the list \type{a,b,c,d,e}, the raw routine 
%D takes over 30\% less time, when we feed $20+$ character 
%D strings we gain about 20\%. Alternatives which use 
%D \type{\futurelet} perform worse. Part of the speedup is 
%D due to the \type{\let} and \type{\expandafter} in the test. 

% %D \macros
% %D   {processunexpandedcommalist}
% %D
% %D When processing commalists, the arguments are expanded. The
% %D main reason for doing so lays in the fact that these 
% %D macros are used for interfacing. The next alternative can be used 
% %D for 
% %D
% %D \starttypen
% %D \processunexpandedcommalist
% %D   [\alfa\beta,\gamma,\delta\epsilon]
% %D   \handleitem
% %D \stoptypen
% %D
% %D This time nesting is not supported. 
% 
% %\def\processunexpandedcommaitem#1,%
% %  {\if]\noexpand#1%
% %     \let\nextcommaitem\relax
% %   \else
% %     \handleunexpandedcommaitem{#1}%
% %     \let\nextcommaitem\processunexpandedcommaitem
%   \fi
%   \nextcommaitem}
%
% faster:
% 
% \def\processunexpandedcommaitem#1,%
%   {\if]\noexpand#1\else
%      \handleunexpandedcommaitem{#1}%
%      \expandafter\processunexpandedcommaitem
%    \fi}
% 
% \def\processunexpandedcommalist[#1]#2%
%   {\def\handleunexpandedcommaitem{#2}%
%    \processunexpandedcommaitem#1,],}% \relax}
% 
% %D Or faster: 
% 
% \def\processunexpandedcommaitem#1,%
%   {\if]\noexpand#1\else
%      \handleunexpandedcommaitem{#1}%
%      \expandafter\processunexpandedcommaitem
%    \fi}

%D \macros
%D   {dosetvalue,dosetevalue,docopyvalue,doresetvalue,
%D    dogetvalue}
%D
%D When we are going to do assignments, we have to take
%D multi||linguality into account. For the moment we keep
%D things simple and single||lingual.
%D
%D \starttypen
%D \dosetvalue   {label}    {variable}   {value}
%D \dosetevalue  {label}    {variable}   {value}
%D \docopyvalue  {to label} {from label} {variable}
%D \doresetvalue {label}    {variable}
%D \stoptypen
%D
%D These macros are in fact auxiliary ones and are not meant
%D for use outside the assignment macros.

\def\dosetvalue#1#2% #3
  {\@EA\def\csname#1#2\endcsname} % {#3}}

\def\dosetevalue#1#2% #3
  {\@EA\edef\csname#1#2\endcsname} % {#3}}

\def\doresetvalue#1#2%
  {\@EA\def\csname#1#2\endcsname{}}

\def\docopyvalue#1#2#3%
  {\@EA\def\csname#1#3\endcsname{\csname#2#3\endcsname}}

%D \macros
%D   {doassign,undoassign,doassignempty}
%D
%D Assignments are the backbone of \CONTEXT. Abhorred by the
%D concept of style file hacking, we took a considerable effort
%D in building a parameterized system. Unfortunately there is a
%D price to pay in terms of speed. Compared to other packages
%D and taking the functionality of \CONTEXT\ into account, the
%D total size of the format file is still very acceptable. Now
%D how are these assignments done. 
%D
%D Assignments can be realized with:
%D
%D \starttypen
%D \doassign[label][variable=value]
%D \undoassign[label][variable=value]
%D \stoptypen
%D
%D and:
%D
%D \starttypen
%D \doassignempty[label][variable=value]
%D \stoptypen
%D
%D Assignments like \type{\doassign} are compatible with:
%D
%D \starttypen
%D \def\labelvariable{value}
%D \stoptypen
%D
%D We do check for the presence of an \type{=} and loudly
%D complain of it's missed. We will redefine this macro later
%D on, when a more advanced message mechanism is implemented. 

%\def\p!doassign#1[#2][#3=#4=#5]%
%  {\let\donottest\dontprocesstest
%   \edef\!!stringa{#5}%
%   \let\!!stringb\relax
%   \let\donottest\doprocesstest
%   \ifx\!!stringa\!!stringb
%     \writestatus
%       {setup}
%       {missing '=' after '#3' in line \the\inputlineno}%
%   \else
%     #1{#2}{#3}{#4}%
%   \fi}

\def\p!doassign#1[#2][#3=#4=#5]% redefined in mult-ini
  {\ifx\empty#3\else  % and definitely not \ifx#3\empty
     \ifx\relax#5%
     \writestatus
       {setup}
       {missing '=' after '#3' in line \the\inputlineno}%
     \else
       #1{#2}{#3}{#4}%
     \fi
   \fi}

\def\doassign[#1][#2]%
  {\p!doassign\dosetvalue[#1][#2==\relax]}

\def\doeassign[#1][#2]%
  {\p!doassign\dosetevalue[#1][#2==\relax]}

\def\undoassign[#1][#2]%
  {\p!doassign\doresetvalue[#1][#2==\relax]}

\def\doassignempty[#1][#2=#3]%
  {\doifundefined{#1#2}
     {\dosetvalue{#1}{#2}{#3}}}

%D \macros
%D   {getparameters,geteparameters} % ,forgetparameters}
%D
%D Using the assignment commands directly is not our
%D ideal of user friendly interfacing, so we take some further
%D steps.
%D
%D \starttypen
%D \getparameters    [label] [...=...,...=...]
% %D \forgetparameters [label] [...=...,...=...]
%D \stoptypen
%D
%D Again, the label identifies the category a variable
%D belongs to. The second argument can be a comma separated
%D list of assignments.
%D
%D \starttypen
%D \getparameters
%D   [demo]
%D   [alfa=1,
%D    beta=2]
%D \stoptypen
%D
%D is equivalent to
%D
%D \starttypen
%D \def\demoalfa{1}
%D \def\demobeta{2}
%D \stoptypen
%D
%D
%D In the pre||multi||lingual stadium \CONTEXT\ took the next
%D approach. With
%D
%D \starttypen
%D \def\??demo {@@demo}
%D \def\!!alfa {alfa}
%D \def\!!beta {beta}
%D \stoptypen
%D
%D calling
%D
%D \starttypen
%D \getparameters
%D   [\??demo]
%D   [\!!alfa=1,
%D    \!!beta=2]
%D \stoptypen
%D
%D lead to:
%D
%D \starttypen
%D \def\@@demoalfa{1}
%D \def\@@demobeta{2}
%D \stoptypen
%D
%D Because we want to be able to distinguish the \type{!!}
%D pre||tagged user supplied variables from internal
%D counterparts, we will introduce a slightly different tag in
%D the multi||lingual modules. There we will use \type{c!} or
%D \type{v!}, depending on the context.
%D
%D By calling \type{\p!doassign} directly, we save ourselves
%D some argument passing and gain some speed. Whatever
%D optimizations we do, this  command will always be one of the
%D bigger bottlenecks.
%D
%D The alternative \type{\geteparameters} --- it's funny to
%D see that this alternative saw the light so lately --- can be
%D used to do expanded assigments.

\def\dogetparameters#1[#2]#3[#4]%
  {\def\p!dogetparameter##1%
     {\p!doassign#1[#2][##1==\relax]}%
   \processcommalist[#4]\p!dogetparameter}

\def\getparameters%
  {\dogetparameters\dosetvalue}

\def\geteparameters%
  {\dogetparameters\dosetevalue}

% \def\forgetparameters%
%   {\dogetparameters\doresetvalue}

\let\getexpandedparameters=\geteparameters

%D \macros
%D   {getemptyparameters}
%D
%D Sometimes we explicitly want variables to default to an
%D empty string, so we welcome:
%D
%D \starttypen
%D \getemptyparameters [label] [...=...,...=...]
%D \stoptypen

\def\getemptyparameters[#1]#2[#3]%
  {\def\p!dogetemptyparameter##1%
     {\doassignempty[#1][##1]}%
   \processcommalist[#3]\p!dogetemptyparameter}

%D \macros
%D   {copyparameters}
%D
%D Some \CONTEXT\ commands take their default setups from
%D others. All commands that are able to provide backgounds
%D or rules around some content, for instance default to the
%D standard command for ruled boxes. Is situations like this 
%D we can use: 
%D
%D \starttypen
%D \copyparameters [to-label] [from-label] [name1,name2,...]
%D \stoptypen
%D
%D For instance
%D
%D \starttypen
%D \copyparameters
%D   [internal][external]
%D   [alfa,beta]
%D \stoptypen
%D
%D Leads to:
%D
%D \starttypen
%D \def\internalalfa {\externalalfa}
%D \def\internalbeta {\externalbeta}
%D \stoptypen
%D
%D By using \type{\docopyvalue} we've prepared this command
%D for use in a multi||lingual environment.

\def\copyparameters[#1]#2[#3]#4[#5]%
  {\doifnot{#1}{#3}
     {\def\docopyparameter##1%
        {\docopyvalue{#1}{#3}{##1}}%
      \processcommalist[#5]\docopyparameter}}

%D \macros
%D   {doifassignmentelse}
%D
%D A lot of \CONTEXT\ commands take optional arguments, for
%D instance:
%D
%D \starttypen
%D \dothisorthat[alfa,beta]
%D \dothisorthat[first=foo,second=bar]
%D \dothisorthat[alfa,beta][first=foo,second=bar]
%D \stoptypen
%D
%D Although a combined solution is possible, we prefer a
%D seperation. The next command takes care of propper
%D handling of such multi||faced commands.
%D
%D \starttypen
%D \doifassignmentelse {...} {then ...} {else ...}
%D \stoptypen

\def\doifassignmentelse%
  {\doifinstringelse{=}}

%D \macros
%D   {ifparameters,checkparameters}
%D
%D A slightly different one is \type{\checkparameters}, which
%D also checks on the presence of a~\type{=}.
%D
%D The boolean \type{\ifparameters} can be used afterwards.
%D Combining both in one \type{\if}||macro would lead to
%D problems with nested \type{\if}'s.
%D
%D \starttypen
%D \checkparameters[argument]
%D \stoptypen

\newif\ifparameters

\def\p!checkparameters#1=#2#3\war%
  {\if#2@\parametersfalse\else\parameterstrue\fi}

\def\checkparameters[#1]%
  {\p!checkparameters#1=@@\war}

%D \macros
%D   {getfromcommalist,getfromcommacommand,
%D    commalistelement,
%D    getcommalistsize,getcommacommandsize}
%D
%D It's possible to get an element from a commalist or a
%D command representing a commalist.
%D
%D \starttypen
%D \getfromcommalist    [string] [n]
%D \getfromcommacommand [string,\strings,string,...] [n]
%D \stoptypen
%D
%D The difference betwee the two of them is the same as the
%D difference between \type{\processcomma...}. The found string
%D is stored in \type{\commalistelement}.
%D
%D We can calculate the size of a comma separated list by
%D using:
%D
%D \starttypen
%D \getcommalistsize    [string,string,...]
%D \getcommacommandsize [string,\strings,string,...]
%D \stoptypen
%D
%D Afterwards, the length is available in the macro
%D \type{\commalistsize} (not a \COUNTER).

\newcount\commalistcounter

\def\commalistsize{0}

\def\p!dogetcommalistsize#1%
  {\advance\commalistcounter 1 }

\def\getcommalistsize[#1]%
  {\commalistcounter=0
   \processcommalist[#1]\p!dogetcommalistsize   % was [{#1}]
   \edef\commalistsize{\the\commalistcounter}}

\def\getcommacommandsize[#1]%
  {\edef\commacommand{#1}%
   \toks0=\expandafter{\expandafter[\commacommand]}%
   \expandafter\getcommalistsize\the\toks0 }

% \def\p!dogetfromcommalist#1%
%   {\advance\commalistcounter -1
%    \ifcase\commalistcounter
%      \def\commalistelement{#1}%
%      \bgroup\def\doprocesscommaitem##1]{\egroup}%
%    \fi}

\def\p!dogetfromcommalist#1%
  {\advance\commalistcounter -1
   \ifcase\commalistcounter
     \def\commalistelement{#1}%
     \expandafter\quitcommalist
   \fi}

\def\getfromcommalist[#1]#2[#3]%
  {\let\commalistelement\empty
   \commalistcounter=#3\relax
   \processcommalist[#1]\p!dogetfromcommalist}

\def\getfromcommacommand[#1]%
  {\edef\commacommand{#1}%
   \toks0=\expandafter{\expandafter[\commacommand]}%
   \expandafter\getfromcommalist\the\toks0 }

%D Because 0, 1 and~2 are often asked for, we optimize this 
%D macro for those cases. The indirect call however slows 
%D down the other cases. 

\def\p!dogetfirstfromcommalist    [#1,#2]{\def\commalistelement{#1}}
\def\p!dogetsecondfromcommalist[#1,#2,#3]{\def\commalistelement{#2}}
\let\p!dogetotherfromcommalist=\getfromcommalist

\def\getfromcommalist[#1]#2[#3]% optimized for 0,1,2 
  {\ifcase#3\relax
     \let\commalistelement\empty
   \or
     \p!dogetfirstfromcommalist[#1,]%    
   \or
     \p!dogetsecondfromcommalist[#1,,]%    
   \else
     \p!dogetotherfromcommalist[#1][#3]%
   \fi}

%D Watertight (and efficient) solutions are hard to find, due
%D to the handling of braces during parameters passing and 
%D scanning. Nevertheless:   
%D 
%D \startbuffer
%D \def\dosomething#1{(#1=\commalistsize) }
%D 
%D \getcommalistsize [\hbox{$a,b,c,d,e,f$}] \dosomething 1
%D \getcommalistsize [{a,b,c,d,e,f}]        \dosomething 1
%D \getcommalistsize [{a,b,c},d,e,f]        \dosomething 4
%D \getcommalistsize [a,b,{c,d,e},f]        \dosomething 4
%D \getcommalistsize [a{b,c},d,e,f]         \dosomething 4
%D \getcommalistsize [{a,b}c,d,e,f]         \dosomething 4
%D \getcommalistsize []                     \dosomething 0
%D \getcommalistsize [{[}]                  \dosomething 1
%D \stopbuffer
%D
%D \typebuffer 
%D
%D reports: 
%D 
%D \haalbuffer

%D \macros
%D   {dogetcommalistelement,dogetcommacommandelement}
%D 
%D For low level (fast) purposes, we can also use the next 
%D alternative, which can handle 8~elements at most. 
%D 
%D \starttypen
%D \dogetcommalistelement1\from a,b,c\to\commalistelement
%D \stoptypen

\def\dodogetcommalistelement#1\from#2,#3,#4,#5,#6,#7,#8\to#9%
  {\edef#9{\ifcase#1\relax\or#2\or#3\or#4\or#5\or#6\or#7\or#8\fi}}

\def\dogetcommalistelement#1\from#2\to%
  {\dodogetcommalistelement#1\from#2,,,,,,\to}

\def\dogetcommacommandelement#1\from#2\to%
  {\@EA\dodogetcommalistelement\@EA#1\@EA\from#2,,,,,,\to}

%D \macros
%D   {dosingleargument,dodoubleargument,dotripleargument,
%D    doquadrupleargument,doquintupleargument,dosixtupleargument,
%D    doseventupleargument}
%D
%D When working with delimited arguments, spaces and
%D lineendings can interfere. The next set of macros uses
%D \TEX' internal scanner for grabbing everything between
%D arguments. Forgive me the funny names. 
%D
%D \starttypen
%D \dosingleargument\commando    = \commando[#1]
%D \dodoubleargument\commando    = \commando[#1][#2]
%D \dotripleargument\commando    = \commando[#1][#2][#3]
%D \doquadrupleargument\commando = \commando[#1][#2][#3][#4]
%D \doquintupleargument\commando = \commando[#1][#2][#3][#4][#5]
%D \dosixtupleargument\commando  = \commando[#1][#2][#3][#4][#5][#6]
%D \doseventupleargument\commando= \commando[#1][#2][#3][#4][#5][#6][#7]
%D \stoptypen
%D
%D These macros are used in the following way:
%D
%D \starttypen
%D \def\dosetupsomething[#1][#2]%
%D   {... #1 ... #2 ...}
%D
%D \def\setupsomething%
%D   {\dodoubleargument\dosetupsomething}
%D \stoptypen
%D
%D The implementation can be surprisingly simple and needs no
%D further explanation, like:
%D
%D \starttypen
%D \def\dosingleargument#1[#2]%
%D   {#1[#2]}
%D \def\dotripleargument#1[#2]#3[#4]#5[#6]%
%D   {#1[#2][#4][#6]}
%D \def\doquintupleargument#1%
%D   {\def\dodoquintupleargument[##1]##2[##3]##4[##5]##6[##7]##8[##9]%
%D      {#1[##1][##3][##5][##7][##9]}%
%D    \dodoquintupleargument}
%D \stoptypen
%D
%D Because \TEX\ accepts 9~arguments at most, we have to use
%D two||step solution when getting five or more arguments.
%D
%D When developing more and more of the real \CONTEXT, we
%D started using some alternatives that provided empty
%D arguments (in fact optional ones) whenever the user failed
%D to supply them. Because this more complicated macros enable
%D us to do some checking, we reimplemented the non||empty 
%D ones.

\def\dosingleargument%
  {\chardef\expectedarguments=1
   \dosingleempty}

\def\dodoubleargument%
  {\chardef\expectedarguments=2
   \dodoubleempty}

\def\dotripleargument%
  {\chardef\expectedarguments=3
   \dotripleempty}

\def\doquadrupleargument%
  {\chardef\expectedarguments=4
   \doquadrupleempty}

\def\doquintupleargument%
  {\chardef\expectedarguments=5
   \doquintupleempty}

\def\dosixtupleargument%
  {\chardef\expectedarguments=6
   \dosixtupleempty}

\def\doseventupleargument%
  {\chardef\expectedarguments=7
   \doseventupleempty}

%D \macros
%D   {iffirstagument,ifsecondargument,ifthirdargument,
%D    iffourthargument,iffifthargument,ifsixthargument,
%D    ifseventhargument}
%D
%D We use some signals for telling the calling macros if all
%D wanted arguments are indeed supplied by the user.

\newif\iffirstargument
\newif\ifsecondargument
\newif\ifthirdargument
\newif\iffourthargument
\newif\iffifthargument
\newif\ifsixthargument
\newif\ifseventhargument

%D \macros
%D   {dosingleempty,dodoubleempty,dotripleempty,
%D    doquadrupleempty,doquintupleempty,dosixtupeempty,
%D    doseventupleempty}
%D
%D The empty argument supplying macros mentioned before, look
%D like:
%D
%D \starttypen
%D \dosingleempty    \command
%D \dodoubleempty    \command
%D \dotripleempty    \command
%D \doquadrupleempty \command
%D \doquintupleempty \command
%D \dosixtupleempty  \command
%D \doseventupleempty\command
%D \stoptypen
%D
%D So \type{\dodoubleempty} leades to:
%D
%D \starttypen
%D \command[#1][#2]
%D \command[#1][]
%D \command[][]
%D \stoptypen
%D
%D Depending of the generousity of the user. Afterwards one can
%D use the \type{\if...argument} boolean. For novice: watch
%D the stepwise doubling of \type{#}'s

% idea: \ignorespaces afterwards

\chardef\noexpectedarguments=0
\chardef\expectedarguments  =0

\def\dogetargument#1#2#3#4% redefined in mult-ini
  {\doifnextcharelse{#1}
     {\let\expectedarguments\noexpectedarguments
      #3\dodogetargument}
     {\ifnum\expectedarguments>\noexpectedarguments
        \writestatus
          {setup}
          {\expectedarguments\space argument(s) expected
           in line \the\inputlineno\space}%
      \fi
      \let\expectedarguments\noexpectedarguments
      #4\dodogetargument#1#2}}

%\def\getsingleempty#1#2#3%
%  {\def\dodogetargument%
%     {#3}%
%   \dogetargument#1#2\firstargumenttrue\firstargumentfalse}
%
%\def\getdoubleempty#1#2#3%
%  {\def\dodogetargument#1##1#2%
%     {\def\dodogetargument%
%        {#3#1##1#2}%
%      \dogetargument#1#2\secondargumenttrue\secondargumentfalse}%
%   \dogetargument#1#2\firstargumenttrue\firstargumentfalse}
%
%\def\gettripleempty#1#2#3%
%  {\def\dodogetargument#1##1#2%
%     {\def\dodogetargument#1####1#2%
%        {\def\dodogetargument%
%           {#3#1##1#2%
%              #1####1#2}%
%         \dogetargument#1#2\thirdargumenttrue\thirdargumentfalse}%
%      \dogetargument#1#2\secondargumenttrue\secondargumentfalse}%
%   \dogetargument#1#2\firstargumenttrue\firstargumentfalse}
%
%\def\getquadrupleempty#1#2#3%
%  {\def\dodogetargument#1##1#2%
%     {\def\dodogetargument#1####1#2%
%        {\def\dodogetargument#1########1#2%
%           {\def\dodogetargument%
%              {#3#1##1#2%
%                 #1####1#2%
%                 #1########1#2}%
%            \dogetargument#1#2\fourthargumenttrue\fourthargumentfalse}%
%         \dogetargument#1#2\thirdargumenttrue\thirdargumentfalse}%
%      \dogetargument#1#2\secondargumenttrue\secondargumentfalse}%
%   \dogetargument#1#2\firstargumenttrue\firstargumentfalse}
%
%\def\getquintupleempty#1#2#3%
%  {\def\dodogetargument#1##1#2%
%     {\def\dodogetargument#1####1#2%
%        {\def\dodogetargument#1########1#2%
%           {\def\dodogetargument#1################1#2%
%              {\def\dodogetargument%
%                 {#3#1##1#2%
%                    #1####1#2%
%                    #1########1#2%
%                    #1################1#2}%
%               \dogetargument#1#2\fifthargumenttrue\fifthargumentfalse}%
%            \dogetargument#1#2\fourthargumenttrue\fourthargumentfalse}%
%         \dogetargument#1#2\thirdargumenttrue\thirdargumentfalse}%
%      \dogetargument#1#2\secondargumenttrue\secondargumentfalse}%
%   \dogetargument#1#2\firstargumenttrue\firstargumentfalse}
%
%\def\getsixtupleempty#1#2#3%
%  {\def\dodogetargument#1##1#2%
%     {\def\dodogetargument#1####1#2%
%        {\def\dodogetargument#1########1#2%
%           {\def\dodogetargument#1################1#2%
%              {\def\dodogetargument#1################################1#2%
%                 {\def\dodogetargument%
%                    {#3#1##1#2%
%                       #1####1#2%
%                       #1########1#2%
%                       #1################1#2%
%                       #1################################1#2}%
%                 \dogetargument#1#2\sixthargumenttrue\sixthargumentfalse}%
%               \dogetargument#1#2\fifthargumenttrue\fifthargumentfalse}%
%            \dogetargument#1#2\fourthargumenttrue\fourthargumentfalse}%
%         \dogetargument#1#2\thirdargumenttrue\thirdargumentfalse}%
%      \dogetargument#1#2\secondargumenttrue\secondargumentfalse}%
%   \dogetargument#1#2\firstargumenttrue\firstargumentfalse}
%
%\def\getseventupleempty#1#2#3%
%  {\def\dodogetargument#1##1#2%
%     {\def\dodogetargument#1####1#2%
%        {\def\dodogetargument#1########1#2%
%           {\def\dodogetargument#1################1#2%
%              {\def\dodogetargument#1################################1#2%
%                 {\def\dodogetargument#1###############################%
%                                        ################################1#2%
%                    {\def\dodogetargument%
%                       {#3#1##1#2%
%                          #1####1#2%
%                          #1########1#2%
%                          #1################1#2%
%                          #1################################1#2%
%                          #1###############################%
%                            ################################1#2}%
%                   \dogetargument#1#2\seventhargumenttrue\seventhargumentfalse}%
%                 \dogetargument#1#2\sixthargumenttrue\sixthargumentfalse}%
%               \dogetargument#1#2\fifthargumenttrue\fifthargumentfalse}%
%            \dogetargument#1#2\fourthargumenttrue\fourthargumentfalse}%
%         \dogetargument#1#2\thirdargumenttrue\thirdargumentfalse}%
%      \dogetargument#1#2\secondargumenttrue\secondargumentfalse}%
%   \dogetargument#1#2\firstargumenttrue\firstargumentfalse}

\def\getsingleempty#1#2#3%
  {\def\dodogetargument%
     {#3}%
   \dogetargument#1#2\firstargumenttrue\firstargumentfalse}

\def\getdoubleempty#1#2#3%
  {\def\dodogetargument#1##1#2%
     {\def\dodogetargument%
        {#3#1{##1}#2}%
      \dogetargument#1#2\secondargumenttrue\secondargumentfalse}%
   \dogetargument#1#2\firstargumenttrue\firstargumentfalse}

\def\gettripleempty#1#2#3%
  {\def\dodogetargument#1##1#2%
     {\def\dodogetargument#1####1#2%
        {\def\dodogetargument%
           {#3#1{##1}#2%
              #1{####1}#2}%
         \dogetargument#1#2\thirdargumenttrue\thirdargumentfalse}%
      \dogetargument#1#2\secondargumenttrue\secondargumentfalse}%
   \dogetargument#1#2\firstargumenttrue\firstargumentfalse}

\def\getquadrupleempty#1#2#3%
  {\def\dodogetargument#1##1#2%
     {\def\dodogetargument#1####1#2%
        {\def\dodogetargument#1########1#2%
           {\def\dodogetargument%
              {#3#1{##1}#2%
                 #1{####1}#2%
                 #1{########1}#2}%
            \dogetargument#1#2\fourthargumenttrue\fourthargumentfalse}%
         \dogetargument#1#2\thirdargumenttrue\thirdargumentfalse}%
      \dogetargument#1#2\secondargumenttrue\secondargumentfalse}%
   \dogetargument#1#2\firstargumenttrue\firstargumentfalse}

\def\getquintupleempty#1#2#3%
  {\def\dodogetargument#1##1#2%
     {\def\dodogetargument#1####1#2%
        {\def\dodogetargument#1########1#2%
           {\def\dodogetargument#1################1#2%
              {\def\dodogetargument%
                 {#3#1{##1}#2%
                    #1{####1}#2%
                    #1{########1}#2%
                    #1{################1}#2}%
               \dogetargument#1#2\fifthargumenttrue\fifthargumentfalse}%
            \dogetargument#1#2\fourthargumenttrue\fourthargumentfalse}%
         \dogetargument#1#2\thirdargumenttrue\thirdargumentfalse}%
      \dogetargument#1#2\secondargumenttrue\secondargumentfalse}%
   \dogetargument#1#2\firstargumenttrue\firstargumentfalse}

\def\getsixtupleempty#1#2#3%
  {\def\dodogetargument#1##1#2%
     {\def\dodogetargument#1####1#2%
        {\def\dodogetargument#1########1#2%
           {\def\dodogetargument#1################1#2%
              {\def\dodogetargument#1################################1#2%
                 {\def\dodogetargument%
                    {#3#1{##1}#2%
                       #1{####1}#2%
                       #1{########1}#2%
                       #1{################1}#2%
                       #1{################################1}#2}%
                 \dogetargument#1#2\sixthargumenttrue\sixthargumentfalse}%
               \dogetargument#1#2\fifthargumenttrue\fifthargumentfalse}%
            \dogetargument#1#2\fourthargumenttrue\fourthargumentfalse}%
         \dogetargument#1#2\thirdargumenttrue\thirdargumentfalse}%
      \dogetargument#1#2\secondargumenttrue\secondargumentfalse}%
   \dogetargument#1#2\firstargumenttrue\firstargumentfalse}

\def\getseventupleempty#1#2#3%
  {\def\dodogetargument#1##1#2%
     {\def\dodogetargument#1####1#2%
        {\def\dodogetargument#1########1#2%
           {\def\dodogetargument#1################1#2%
              {\def\dodogetargument#1################################1#2%
                 {\def\dodogetargument#1###############################%
                                        ################################1#2%
                    {\def\dodogetargument%
                       {#3#1{##1}#2%
                          #1{####1}#2%
                          #1{########1}#2%
                          #1{################1}#2%
                          #1{################################1}#2%
                          #1{###############################%
                             ################################1}#2}%
                   \dogetargument#1#2\seventhargumenttrue\seventhargumentfalse}%
                 \dogetargument#1#2\sixthargumenttrue\sixthargumentfalse}%
               \dogetargument#1#2\fifthargumenttrue\fifthargumentfalse}%
            \dogetargument#1#2\fourthargumenttrue\fourthargumentfalse}%
         \dogetargument#1#2\thirdargumenttrue\thirdargumentfalse}%
      \dogetargument#1#2\secondargumenttrue\secondargumentfalse}%
   \dogetargument#1#2\firstargumenttrue\firstargumentfalse}

\def\dosingleempty    {\getsingleempty    []}
\def\dodoubleempty    {\getdoubleempty    []}
\def\dotripleempty    {\gettripleempty    []}
\def\doquadrupleempty {\getquadrupleempty []}
\def\doquintupleempty {\getquintupleempty []}
\def\dosixtupleempty  {\getsixtupleempty  []}
\def\doseventupleempty{\getseventupleempty[]}

%D \macros
%D   {dosingleargumentwithset,
%D    dodoubleargumentwithset,dodoubleemptywithset,
%D    dotripleargumentwithset,dotripleemptywithset}
%D
%D These maybe too mysterious macros enable us to handle more
%D than one setup at once.
%D
%D \starttypen
%D \dosingleargumentwithset \command[#1]
%D \dodoubleargumentwithset \command[#1][#2]
%D \dotripleargumentwithset \command[#1][#2][#3]
%D \dodoubleemptywithset    \command[#1][#2]
%D \dotripleemptywithset    \command[#1][#2][#3]
%D \stoptypen
%D
%D The first macro calls \type{\command[##1]} for each string
%D in the set~\type{#1}. The second one calls for
%D \type{\commando[##1][#2]} and the third, well one may guess.
%D These commands support constructions like:
%D
%D \starttypen
%D \def\dodefinesomething[#1][#2]%
%D   {\getparameters[\??xx#1][#2]}
%D
%D \def\definesomething%
%D   {\dodoubleargumentwithset\dodefinesomething}
%D \stoptypen
%D
%D Which accepts calls like:
%D
%D \starttypen
%D \definesomething[alfa,beta,...][variable=...,...]
%D \stoptypen
%D
%D Now a whole bunch of variables like \type{\@@xxalfavariable}
%D and \type{\@@xxbetavariable} is defined.

\def\dosingleargumentwithset#1%
  {\def\dodosinglewithset[##1]%
     {\def\dododosinglewithset####1%
        {#1[####1]}%
      \processcommalist[##1]\dododosinglewithset}%
   \dosingleargument\dodosinglewithset}%

\def\dodoublewithset#1#2%
  {\def\dododoublewithset[##1][##2]%
     {\doifnot{##1}{}
        {\def\dodododoublewithset####1%
           {#2[####1][##2]}%
         \processcommalist[##1]\dodododoublewithset}}%
   #1\dododoublewithset}%

\def\dodoubleemptywithset%
  {\dodoublewithset\dodoubleempty}

\def\dodoubleargumentwithset%
  {\dodoublewithset\dodoubleargument}

\def\dotriplewithset#1#2%
  {\def\dodotriplewithset[##1][##2][##3]%
     {\doifnot{##1}{}
        {\def\dododotriplewithset####1%
           {#2[####1][##2][##3]}%
         \processcommalist[##1]\dododotriplewithset}}%
   #1\dodotriplewithset}%

\def\dotripleemptywithset%
  {\dotriplewithset\dotripleempty}

\def\dotripleargumentwithset%
  {\dotriplewithset\dotripleargument}

%D \macros
%D   {complexorsimple,complexorsimpleempty}
%D
%D Setups can be optional. A command expecting a setup is
%D prefixed by \type{\complex}, a command without one gets the
%D prefix \type{\simple}. Commands like this can be defined by:
%D
%D \starttypen
%D \complexorsimple\command
%D \stoptypen
%D
%D When \type{\command} is followed by a \type{[setup]}, then
%D
%D \starttypen
%D \complexcommand [setup]
%D \stoptypen
%D
%D executes, else we get 
%D
%D \starttypen
%D \simplecommand
%D \stoptypen
%D
%D An alternative for \type{\complexorsimple} is:
%D
%D \starttypen
%D \complexorsimpleempty {command}
%D \stoptypen
%D
%D Depending on the presence of \type{[setup]}, this one 
%D leads to one of: 
%D
%D \starttypen
%D \complexcommando [setup]
%D \complexcommando []
%D \stoptypen
%D
%D Many \CONTEXT\ commands started as complex or simple ones, 
%D but changed into more versatile (more object oriented) ones
%D using the \type{\get..argument} commands. 

\def\complexorsimple#1%
  {\setnameofcommand{#1}%
   \doifnextcharelse{[}
     {\firstargumenttrue\getvalue{\s!complex\nameofcommand}}
     {\firstargumentfalse\getvalue{\s!simple\nameofcommand}}}

\def\complexorsimpleempty#1%
  {\setnameofcommand{#1}%
   \doifnextcharelse{[}
     {\firstargumenttrue\getvalue{\s!complex\nameofcommand}}
     {\firstargumentfalse\getvalue{\s!complex\nameofcommand}[]}}

\def\setnameofcommand#1%
  {\bgroup
   \escapechar=-1
   \globaldefs=0 % pretty important! 
   \xdef\nameofcommand{\string#1}%
   \egroup}

%D \macros
%D   {definecomplexorsimple,definecomplexorsimpleempty}
%D
%D The previous commands are used that often that we found it
%D worthwile to offer two more alternatives. Watch the build
%D in protection. 

\beginTEX

\def\definewithnameofcommand#1#2% 
  {\setnameofcommand{#2}%
   \@EA\def\@EA#2\@EA{\@EA\donottest\@EA#1\@EA{\nameofcommand}}}

\def\definecomplexorsimple%
  {\definewithnameofcommand\complexorsimple}

\def\definecomplexorsimpleempty%
  {\definewithnameofcommand\complexorsimpleempty}

\endTEX

\beginETEX \protected

\def\definecomplexorsimple#1%
  {\normalprotected\def#1{\complexorsimple#1}}

\def\definecomplexorsimpleempty#1%
  {\normalprotected\def#1{\complexorsimpleempty#1}}

\endETEX

%D These commands are called as:
%D
%D \starttypen
%D \definecomplexorsimple\command
%D \stoptypen
%D
%D Of course, we must have available
%D
%D \starttypen
%D \def\complexcommand[#1]{...}
%D \def\simplecommand     {...}
%D \stoptypen
%D
%D Using this construction saves a few string now and then.

%D \macros
%D   {definestartstopcommand}
%D
%D Those who get the creeps of expansion may skip the next 
%D one. It's one of the most recent additions and concerns 
%D \type{\start}||\type{\stop} pairs with complicated 
%D arguments.
%D
%D We won't go into details here, but the general form of 
%D this using this command is: 
%D
%D \starttypen
%D \definestartstopcommand\somecommand\e!specifier{arg}{arg}%
%D   {do something with arg} 
%D \stoptypen
%D
%D This expands to something like:
%D
%D \starttypen
%D \def\somecommand arg \startspecifier arg \stopspecifier%
%D   {do something with arg} 
%D \stoptypen
%D
%D The argumentss can be anything reasonable, but double 
%D \type{#}'s are needed in the specification part, like:
%D
%D \starttypen
%D \definestartstopcommand\somecommand\e!specifier{[##1][##2]}{##3}%
%D   {do #1 something #2 with #3 arg} 
%D \stoptypen
%D
%D which becomes: 
%D
%D \starttypen
%D \def\somecommand[#1][#2]\startspecifier#3\stopspecifier%
%D   {do #1 something #2 with #3 arg} 
%D \stoptypen
%D
%D We will see some real applications of this command in the 
%D core modules. 

\def\definestartstopcommand#1#2#3#4%
  {\def\!stringa{#3}%
   \def\!stringb{\e!start#2}%
   \def\!stringc{#4}%
   \def\!stringd{\e!stop#2}%
   \@EA\@EA\@EA\@EA\@EA\@EA\@EA\@EA\@EA\@EA\@EA\@EA\@EA\@EA\@EA
   \def\@EA\@EA\@EA\@EA\@EA\@EA\@EA\@EA\@EA\@EA\@EA\@EA\@EA\@EA\@EA
   #1\@EA\@EA\@EA\@EA\@EA\@EA\@EA
   \!stringa\@EA\@EA\@EA
   \csname\@EA\@EA\@EA\!stringb\@EA\@EA\@EA\endcsname\@EA
   \!stringc
   \csname\!stringd\endcsname}

%D \macros
%D   {dosinglegroupempty,dodoublegroupempty,dotriplegroupempty,
%D    doquadruplegroupempty}
%D 
%D We've already seen some commands that take care of 
%D optional arguments between \type{[]}. The next two commands 
%D handle the ones with \type{{}}. They are called as: 
%D 
%D \starttypen
%D \dosinglegroupempty \IneedONEargument
%D \dodoublegroupempty \IneedTWOarguments
%D \dotriplegroupempty \IneedTHREEarguments
%D \dotriplegroupempty \IneedFOURarguments
%D \stoptypen
%D 
%D where \type{\IneedONEargument} takes one and the others 
%D two and three arguments. These macro's were first needed in 
%D \PPCHTEX. 

\def\dogetgroupargument#1#2% redefined in mult-ini
  {\def\nextnextargument%
     {\ifx\nextargument\bgroup  
        \let\expectedarguments\noexpectedarguments
        \def\nextargument{#1\dodogetargument}%
      %\else\ifx\nextargument\lineending % this can be an option
      %  \def\nextargument{\bgroup\def\\ {\egroup\dogetgroupargument#1#2}\\}%
      %\else\ifx\nextargument\blankspace % but it may never be default
      %  \def\nextargument{\bgroup\def\\ {\egroup\dogetgroupargument#1#2}\\}%
      \else
        \ifnum\expectedarguments>\noexpectedarguments
          \writestatus
             {setup}
             {\the\expectedarguments\space argument(s) expected
              in line \the\inputlineno\space}%
        \fi
        \let\expectedarguments\noexpectedarguments
        \def\nextargument{#2\dodogetargument{}}%
      \fi%\fi\fi                 % so let's get rid of it
      \nextargument}%
   \futurelet\nextargument\nextnextargument}
 
\def\dosinglegroupempty#1%
  {\def\dodogetargument%
     {#1}%
   \dogetgroupargument\firstargumenttrue\firstargumentfalse}

\def\dodoublegroupempty#1%
  {\def\dodogetargument##1%
     {\def\dodogetargument%
        {#1{##1}}%
      \dogetgroupargument\secondargumenttrue\secondargumentfalse}%
   \dogetgroupargument\firstargumenttrue\firstargumentfalse}

\def\dotriplegroupempty#1%
  {\def\dodogetargument##1%
     {\def\dodogetargument####1%
        {\def\dodogetargument%
           {#1{##1}{####1}}%
         \dogetgroupargument\thirdargumenttrue\thirdargumentfalse}%
      \dogetgroupargument\secondargumenttrue\secondargumentfalse}%
   \dogetgroupargument\firstargumenttrue\firstargumentfalse}

\def\doquadruplegroupempty#1%
  {\def\dodogetargument##1%
     {\def\dodogetargument####1%
        {\def\dodogetargument########1%
           {\def\dodogetargument%
              {#1{##1}{####1}{########1}}%
            \dogetgroupargument\fourthargumenttrue\fourthargumentfalse}%   
         \dogetgroupargument\thirdargumenttrue\thirdargumentfalse}%
      \dogetgroupargument\secondargumenttrue\secondargumentfalse}%
   \dogetgroupargument\firstargumenttrue\firstargumentfalse}

%D These macros explictly take care of spaces, which means 
%D that the next definition and calls are valid:
%D 
%D \starttypen
%D \def\test#1#2#3{[#1#2#3]}
%D 
%D \dotriplegroupempty\test {a}{b}{c}
%D \dotriplegroupempty\test {a}{b} 
%D \dotriplegroupempty\test {a}
%D \dotriplegroupempty\test 
%D \dotriplegroupempty\test {a} {b} {c}
%D \dotriplegroupempty\test {a} {b} 
%D \dotriplegroupempty\test 
%D   {a} 
%D   {b} 
%D \stoptypen
%D
%D And alike.

%D \macros
%D   {wait}
%D
%D The next macro hardly needs explanation. Because no
%D nesting is to be expected, we can reuse \type{\wait} within
%D \type{\wait} itself.

\def\wait%
  {\bgroup
   \read16 to \wait
   \egroup}

%D \macros
%D   {writestring,writeline,
%D    writestatus,statuswidth}
%D
%D Maybe one didn't notice, but we've already introduced a
%D macro for showing messages. In the multi||lingual modules,
%D we will also introduce a mechanism for message passing. For
%D the moment we stick to the core macros:
%D
%D \starttypen
%D \writestring {string}
%D \writeline
%D \writestatus {category} {message}
%D \stoptypen
%D
%D Messages are formatted. One can provide the maximum with
%D of the identification string with the macro
%D \type{\statuswidth}.

% \chardef\statuswidth=15
% 
% \def\writestring%
%   {\immediate\write16}
% 
% \def\writeline%
%   {\writestring{}}
% 
% \def\dosplitstatus#1#2\end%
%   {\ifx#1?%
%      \loop
%        \advance\scratchcounter by 1
%        \ifnum\scratchcounter<\statuswidth
%          \edef\messagecontentA{\messagecontentA\space}%
%      \repeat
%    \else
%      \advance\scratchcounter by 1
%      \ifnum\scratchcounter<\statuswidth
%        \edef\messagecontentA{\messagecontentA#1}%
%      \fi
%      \dosplitstatus#2\end
%    \fi}
% 
% \def\writestatus#1#2%
%   {\bgroup
%    \let\messagecontentA=\empty
%    \edef\messagecontentB{#2}%  maybe it's \the\scratchcounter
%    \scratchcounter=0
%    \expandafter\dosplitstatus#1?\end
%    \writestring{\messagecontentA\space:\space\messagecontentB}%
%    \egroup}

\chardef\statuswidth=15

\def\writestring%
  {\immediate\write16}

\def\writeline%
  {\writestring{}}

\def\dosplitstatus#1%
  {\advance\scratchcounter 1
   \ifnum\scratchcounter<\statuswidth
     \edef\messagecontentA{\messagecontentA#1}%
     \expandafter\dosplitstatus
   \else
     \expandafter\nosplitstatus
   \fi}

\def\nosplitstatus#1\end%
  {}

\gdef\writestatus#1#2%
  {\bgroup
   \let\messagecontentA\empty
   \edef\messagecontentB{#2}%  maybe it's \the\scratchcounter
   \scratchcounter=0
   \expandafter\dosplitstatus#1%
     \space\space\space\space\space\space\space
     \space\space\space\space\space\space\space
     \space\space\space\space\space\space\end
   \writestring{\messagecontentA\space:\space\messagecontentB}%
   \egroup}

%D \macros
%D   {debuggerinfo}
%D
%D For debugging purposes we can enhance macros with the
%D next alternative. Here \type{debuggerinfo} stands for both
%D a macro accepting two arguments and a boolean (in fact a
%D few macro's too).

\newif\ifdebuggerinfo

\def\debuggerinfo#1#2%
  {\ifdebuggerinfo
     \writestatus{debugger}{#1:: #2}%
   \fi}

%D Finally we do what from now on will be done at the top of
%D the files: we tell the user what we are loading.

\writestatus{loading}{Context System Macros / General}

%D Well, the real final command is the one that resets the 
%D unprotected characters \type{@}, \type{?} and \type{!}.

\protect

\endinput
