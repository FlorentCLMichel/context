%D \module
%D   [       file=syst-ext,
%D        version=1995.10.10,
%D          title=\CONTEXT\ System Macros,
%D       subtitle=Extras,
%D         author=Hans Hagen,
%D           date=\currentdate,
%D      copyright={PRAGMA / Hans Hagen \& Ton Otten}]
%C
%C This module is part of the \CONTEXT\ macro||package and is
%C therefore copyrighted by \PRAGMA. Non||commercial use is 
%C granted. 

\writestatus{loading}{Context System Macro's / Extras}

%D In this second system module, we continue the definition of
%D some handy commands.

\unprotect

%D \macros
%D   {doglobal,
%D    redoglobal,dodoglobal,resetglobal}
%D
%D The two macros \type{\redoglobal} and \type{\dodoglobal} are
%D used in this and some other modules to enforce a user
%D specified \type{\doglobal} action. The last and often only
%D global assignment in a macro is done with
%D \type{\dodoglobal}, but all preceding ones with
%D \type{\redoglobal}. When using only alternatives, one can 
%D reset this mechanism with \type{\resetglobal}. 

\def\doglobal%
  {\let\redoglobal=\global
   \def\dodoglobal%
     {\resetglobal\global}}

\def\resetglobal%
  {\let\redoglobal=\relax
   \let\dodoglobal=\relax}

\resetglobal

%D \macros
%D   {newcounter,
%D    increment,decrement}
%D   {}
%D
%D Unfortunately the number of \COUNTERS\ in \TEX\ is limited,
%D but fortunately we can store numbers in a macro. We can
%D increment such pseudo \COUNTERS\ with \type{\increment}.
%D
%D \starttypen
%D \increment(\counter,20)
%D \increment(\counter,-4)
%D \increment(\counter)
%D \increment\counter
%D \stoptypen
%D
%D After this sequence of commands, the value of
%D \type{\counter} is 20, 16, 17 and~18. Of course there is
%D also the complementary command \type{\decrement}.
%D
%D Global assignments are possible too, using \type{\doglobal}:
%D
%D \starttypen
%D \doglobal\increment\counter
%D \stoptypen
%D
%D When \type{\counter} is undefined, it's value is initialized
%D at~0. It is nevertheless better to define a \COUNTER\
%D explicitly. One reason could be that the \COUNTER\ can be
%D part of a test with \type{\ifnum} and this conditional does
%D not accept undefined macro's. The \COUNTER\ in our example
%D can for instance be defined with:
%D
%D \starttypen
%D \newcounter\counter
%D \stoptypen
%D
%D The command \type{\newcounter} must not be confused with
%D \type{\newcount}! Of course this mechanism is much slower
%D than using \TEX's \COUNTERS\ directly. In practice
%D \COUNTERS\ (and therefore our pseudo counters too) are
%D seldom the bottleneck in the processing of a text. Apart
%D from some other incompatilities we want to mention a pitfal
%D when using \type{\ifnum}.
%D
%D \starttypen
%D \ifnum\normalcounter=\pseudocounter \doif \else \doelse \fi
%D \ifnum\pseudocounter=\normalcounter \doif \else \doelse \fi
%D \stoptypen
%D
%D In the first test, \TEX\ continues it's search for the
%D second number after reading  \type{\pseudocounter}, while 
%D in the second test, it stops reading after having
%D encountered a real one. Tests like the first one therefore
%D can give unexpected results, for instance execution
%D of \type{\doif} even if both numbers are unequal.

\def\newcounter#1%
  {\dodoglobal\def#1{0}}

\def\dodododoincrement(#1,#2)%
  {\ifx#1\undefined
     \def#1{0}%
   \fi
   \scratchcounter=#2\relax
   \scratchcounter=\incrementsign\scratchcounter
   \advance\scratchcounter by #1\relax
   \dodoglobal\edef#1{\the\scratchcounter}}

\def\dododoincrement#1%
  {\dodododoincrement(#1,1)}

\def\dodoincrement(#1%
  {\doifnextcharelse{,}
     {\dodododoincrement(#1}
     {\dodododoincrement(#1,1}}

\def\doincrement#1%
  {\def\incrementsign{#1}%
   \doifnextcharelse{(}
     {\dodoincrement}
     {\dododoincrement}}

\def\increment%
  {\doincrement+}

\def\decrement%
  {\doincrement-}

%D \macros
%D  {newsignal}
%D
%D When writing advanced macros, we cannot do without
%D signaling. A signal is a small (invisible) kern or penalty
%D that signals the next macro that something just happened.
%D This macro can take any action depending onthe previous
%D signal. Signals must be unique and the next macro takes care
%D of that. 
%D
%D \starttypen
%D \newsignal\somesignal
%D \stoptypen
%D
%D Signals old dimensions and can be used in skips, kerns and 
%D tests like \type{\ifdim}. 

\newdimen\currentsignal

\def\newsignal#1%
  {\advance\currentsignal by 0.00025pt 
   \edef#1{\the\currentsignal}}

%D \macros
%D   {newskimen}
%D 
%D \TEX\ offers 256 \DIMENSIONS\ and \SKIPS. Unfortunately this
%D amount is too small to suit certain packages. Therfore when
%D possible one should use 
%D 
%D \starttypen
%D \newskimen\tempskimen
%D \stoptypen
%D 
%D This commands allocates a \DIMENSION\ or a \SKIP, depending
%D on the availability. One should be aware of the difference
%D between both. When searching for some glue \TEX\ goes on
%D searching till it's sure that no other glue component if
%D found. This search can be canceled by using \type{\relax}
%D when possible and needed. 

\def\newskimen#1%
  {\ifx#1\undefined
     \ifnum\count11>\count12
       \newskip#1\relax
     \else
       \newdimen#1\relax
     \fi
   \fi} 

%D \macros
%D   {strippedcsname}
%D   {}
%D
%D The next macro can be very useful when using \type{\csname}
%D like in: 
%D
%D \starttypen
%D \csname if\strippedcsname\something\endcsname
%D \stoptypen
%D
%D This expands to \type{\ifsomething}. 

\def\strippedcsname%
  {\expandafter\gobbleoneargument\string}

%D \macros
%D   {newconditional, 
%D    settrue, setfalse, 
%D    ifconditional}
%D   {}
%D 
%D \TEX's lacks boolean variables, although the \PLAIN\ format 
%D implements \type{\newif}. The main disadvantage of this 
%D scheme is that it takes three hash table entries. A more 
%D memory saving alternative is presented here. A conditional 
%D is defined by: 
%D 
%D \starttypen
%D \newconditional\doublesided
%D \setfalse
%D 
%D Setting a conditional is done by \type{\settrue} and 
%D \type{\setfalse}: 
%D 
%D \starttypen
%D \settrue\doublesided
%D \setfalse
%D 
%D while testing is accomplished by: 
%D 
%D \starttypen
%D \ifconditional\doublesided  ... \else ... \fi
%D \setfalse
%D 
%D We cannot use the simple scheme: 
%D
%D \starttypen
%D \def\settrue#1{\let#1=\iftrue}
%D \def\settrue#1{\let#1=\iffalse}
%D \stoptypen
%D
%D Such an implementation gives problems with nested 
%D conditionals. The next implementation is abaou as fast 
%D and just as straightforward: 

\def\settrue#1%  
  {\chardef#1=0 }

\def\setfalse#1%  
  {\chardef#1=1 }

\let\newconditional = \setfalse
\let\ifconditional  = \ifcase

%D \macros
%D   {dorecurse,recurselevel,recursedepth
%D    dostepwiserecurse,
%D    for}
%D   {}
%D
%D \TEX\ does not offer us powerfull for||loop mechanisms. On 
%D the other hand its recursion engine is quite unique. We
%D therefore identify the for||looping macros by this method.
%D The most simple alternative is the one that only needs a
%D number. 
%D
%D \starttypen
%D \dorecurse {n} {whatever we want}
%D \stoptypen
%D
%D This macro can be nested without problems and therefore be
%D used in situations where \PLAIN\ \TEX's \type{\loop} macro
%D ungracefully fails. The current value of the counter is
%D available in \type{\recurselevel}, before as well as after
%D the \typ{whatever we wat} stuff.
%D
%D \starttypen
%D \dorecurse               % inner loop
%D   {10}
%D   {\recurselevel:          % outer value
%D      \dorecurse          % inner loop
%D        {\recurselevel}     % outer value
%D        {\recurselevel}     % inner value
%D      \dorecurse          % inner loop
%D        {\recurselevel}     % outer value
%D        {\recurselevel}     % inner value
%D    \endgraf}
%D \stoptypen
%D
%D In this example the first, second and fourth
%D \type{\recurselevel} concern the outer loop, while the third
%D and fifth one concern the inner loop. The depth of the
%D nesting is available for inspection in \type{\recursedepth}.
%D 
%D Both \type{\recurselevel} and \type{\recursedepth} are 
%D macros. The real \COUNTERS\ are hidden from the user because
%D we don't want any interference. 

\def\@@irecurse{@@irecurse}  % stepper
\def\@@nrecurse{@@nrecurse}  % number of steps 
\def\@@srecurse{@@srecurse}  % step 
\def\@@drecurse{@@drecurse}  % direction, < or > 
\def\@@arecurse{@@arecurse}  % action 

\newcount\outerrecurse
\newcount\innerrecurse

\def\recursedepth%
  {\the\outerrecurse}

\long\def\dostepwiserecurse#1#2#3#4%
  {\ifnum#2=0
     \def\recurselevel{0}% 
     \let\next=\relax
   \else
     \global\advance\outerrecurse by 1
     \innerrecurse=#1\setevalue{\@@irecurse\recursedepth}{\the\innerrecurse}%
     \innerrecurse=#2\setevalue{\@@nrecurse\recursedepth}{\the\innerrecurse}%
     \innerrecurse=#3\setevalue{\@@srecurse\recursedepth}{\the\innerrecurse}%
     \ifnum#3>0\relax\ifnum#2<#1\relax
     \else
       \setevalue{\@@drecurse\recursedepth}{>}%
       \long\setvalue{\@@arecurse\recursedepth}{#4}%
       \let\next=\dodorecurse
     \fi\fi
     \ifnum#3<0\relax\ifnum#1<#2\relax
     \else
       \setevalue{\@@drecurse\recursedepth}{<}%
       \long\setvalue{\@@arecurse\recursedepth}{#4}%
       \let\next=\dodorecurse
     \fi\fi
   \fi
   \next}

\def\donorecurse%
  {}

\def\dodonorecurse%
  {\global\advance\outerrecurse by -1\relax}

\def\dododorecurse%
  {\edef\recurselevel{\getvalue{\@@irecurse\recursedepth}}%
   \getvalue{\@@arecurse\recursedepth}%
   \edef\recurselevel{\getvalue{\@@irecurse\recursedepth}}%
   \innerrecurse=\recurselevel
   \advance\innerrecurse  by \getvalue{\@@srecurse\recursedepth}\relax
   \setevalue{\@@irecurse\recursedepth}{\the\innerrecurse}%
   \dodorecurse}

\def\dodorecurse%
  {\ifnum\getvalue{\@@irecurse\recursedepth}
         \getvalue{\@@drecurse\recursedepth}
         \getvalue{\@@nrecurse\recursedepth}\relax
     \expandafter\dodonorecurse
   \else
     \expandafter\dododorecurse
   \fi}

\def\dorecurse#1%
  {\dostepwiserecurse{1}{#1}{1}}

%D For those we like to offer visual beauty for efficiency we 
%D say however:\voetnoot{In this kind of macro's we tend to 
%D minimalize the overhead.}

\def\dodorecurse%
  {\ifnum\getvalue{\@@irecurse\recursedepth}
         \getvalue{\@@drecurse\recursedepth}
         \getvalue{\@@nrecurse\recursedepth}\relax
     \global\advance\outerrecurse by -1
   \else
     \expandafter\dododorecurse
   \fi}

%D As we can see here, the simple command \type{\dorecurse} is
%D a special case of the more general:
%D
%D \starttypen
%D \dostepwiserecurse {from} {to} {step} {action}
%D \stoptypen
%D
%D This commands accepts positive and negative steps. Illegal
%D values are handles as good as possible and the macro accepts
%D numbers and \COUNTERS.
%D
%D \starttypen
%D \dostepwiserecurse  {1} {10}  {2} {...}
%D \dostepwiserecurse {10}  {1} {-2} {...}
%D \stoptypen
%D
%D The third alternative looks a bit different and uses a
%D pseudo counter. When this macro is nested, we have to use
%D different counters. This time we use keywords. 
%D
%D \starttypen
%D \def\alfa{2} \def\beta{100} \def\gamma{3}
%D
%D \for \n=55    \to 100   \step  1      \do {... \n ...}
%D \for \n=\alfa \to \beta \step  \gamma \do {... \n ...}
%D \for \n=\n    \to 120   \step  1      \do {... \n ...}
%D \for \n=120   \to 100   \step -3      \do {... \n ...}
%D \for \n=55    \to 100   \step  2      \do {... \n ...}
%D \stoptypen
%D
%D Only in the third example we need to predefine \type{\n}. 
%D The use of \type{\od} as a dilimiter would have made nested
%D use more problematic. 

\def\for#1=#2\to#3\step#4\do#5%
  {\dostepwiserecurse{#2}{#3}{#4}
     {\edef#1{\recurselevel}%
      #5%
      \edef#1{\recurselevel}}}

%D \macros
%D   {doloop,exitloop}
%D   {}
%D
%D Sometimes loops are not determined by counters, but by 
%D (a combinations of) conditions. We therefore implement a 
%D straightforward loop, which can only be left when we 
%D explictly exit it. Nesting is supported. First we present 
%D a more extensive alternative. 
%D
%D \starttypen
%D \doloop
%D   {Some kind of typesetting punishment \par
%D    \ifnum\pageno>100 \exitloop \fi}
%D \stoptypen
%D 
%D When needed, one can call for \type{\looplevel} and 
%D \type{\loopdepth}.
%D
%D If we write this macros from scratch, we end up with 
%D something like the ones described above: 
%D 
%D \starttypen
%D \def\@@eloop{@@eloop}  % exit 
%D \def\@@iloop{@@iloop}  % stepper
%D \def\@@aloop{@@aloop}  % action
%D 
%D \newcount\outerloop
%D 
%D \def\loopdepth%
%D   {\the\outerloop}
%D 
%D \def\exitloop%
%D   {\setevalue{\@@eloop\loopdepth}{0}}
%D 
%D \long\def\doloop#1%
%D   {\global\advance\outerloop by 1
%D    \setevalue{\@@iloop\loopdepth}{1}%
%D    \setevalue{\@@eloop\loopdepth}{1}%
%D    \long\setvalue{\@@aloop\loopdepth}{#1}%
%D    \dodoloop}
%D 
%D \def\dodonoloop%
%D   {\global\advance\outerloop by -1\relax}
%D 
%D \def\dododoloop%
%D   {\edef\looplevel{\getvalue{\@@iloop\loopdepth}}%
%D    \innerrecurse=\looplevel
%D    \advance\innerrecurse by 1
%D    \setevalue{\@@iloop\loopdepth}{\the\innerrecurse}%
%D    \getvalue{\@@aloop\loopdepth}%
%D    \edef\looplevel{\getvalue{\@@iloop\loopdepth}}%
%D    \dodoloop}
%D 
%D \def\dodoloop%
%D   {\ifnum\getvalue{\@@eloop\loopdepth}=0
%D      \expandafter\dodonoloop
%D    \else
%D      \expandafter\dododoloop
%D    \fi}  
%D 
%D \def\doloop%
%D   {\dostepwiserecurse{1}{\maxdimen}{1}}
%D 
%D \def\exitloop
%D   {\setvalue{\@@irecurse\recursedepth}{\maxdimen}}
%D 
%D \def\looplevel{\recurselevel}
%D \def\loopdepth{\recursedepth}
%D \stoptypen
%D
%D We prefer however a more byte saving implementation, that 
%D executes of course a bit slower. 

\def\doloop%
  {\dostepwiserecurse{1}{\maxdimen}{1}}

\def\exitloop
  {\setvalue{\@@irecurse\recursedepth}{\maxdimen}}

%D We don't declare new counters for \type{\looplevel} and 
%D \type{\loopdepth} because one can use \type{\recurselevel} 
%D and \type{\recursedepth}. 
%D
%D The loop is executed at least once, so beware of situations
%D like: 
%D
%D \starttypen
%D \doloop {\exitloop some commands} 
%D \stoptypen
%D
%D It's just a matter of putting the text into the \type{\if}
%D statement that should be there anyway, like in: 
%D
%D \starttypen
%D \doloop {\ifwhatever \exitloop \else some commands\fi} 
%D \stoptypen

%D \macros
%D   {newevery,everyline,EveryLine,EveryPar}
%D   {}
%D
%D Lets skip to something quite different. It's common use
%D to use \type{\everypar} for special purposes. In \CONTEXT\
%D we use this primitive for locating sidefloats. This means
%D that when user assignments to \type{\everypar} can interfere
%D with those of the package. We therefore introduce
%D \type{\EveryPar}.
%D
%D The same goes for \type{\EveryLine}. Because \TEX\ offers
%D no \type{\everyline} primitive, we have to call for
%D \type{\everyline} when we are working on a line by line
%D basis. Just by calling \type{\EveryPar{}} and
%D \type{\EveryLine{}} we restore the old situation.
%D
%D The definition command \type{\DoWithEvery} will be quite 
%D unreadable, so let's first show an implementation that 
%D shows how things are done: 
%D
%D \starttypen 
%D \newtoks \everyline
%D \newtoks \oldeveryline
%D \newif   \ifeveryline
%D 
%D \def\DoWithEvery#1#2#3#4%
%D   {#3\else\edef\next{\noexpand#2={\the#1}}\next\fi
%D    \edef\next{\noexpand#1={\the#2\the\scratchtoks}}\next
%D    #4}
%D 
%D \def\doEveryLine%
%D   {\DoWithEvery\everyline\oldeveryline\ifeveryline\everylinetrue}
%D 
%D \def\EveryLine%
%D   {\afterassignment\doEveryLine\scratchtoks}
%D
%D The real implementation is a bit more complicated but we 
%D prefer something more versatile. 

\def\DoWithEvery#1%
  {\csname if\strippedcsname#1\endcsname \else
     \edef\next%
       {\@EA\noexpand\csname old\strippedcsname#1\endcsname=
          {\the#1}}%
     \next
   \fi
   \edef\next%
     {\noexpand#1=
        {\@EA\the\csname old\strippedcsname#1\endcsname\the\scratchtoks}}%
   \next
   \csname\strippedcsname#1true\endcsname}

\def\dowithevery#1%
  {\@EA\afterassignment\csname do\strippedcsname#1\endcsname\scratchtoks}

\def\newevery#1#2%
  {\ifx#2\undefined
     \ifx#1\undefined\newtoks#1\fi
     \@EA\newtoks\csname old\strippedcsname#1\endcsname
     \@EA\newif  \csname  if\strippedcsname#1\endcsname
     \@EA\def    \csname  do\strippedcsname#2\endcsname{\DoWithEvery#1}%
     \def#2{\dowithevery#2}%
   \fi}

%D This one permits definitions like: 

\newevery \everypar  \EveryPar
\newevery \everyline \EveryLine

%D and how about: 

\newevery \neverypar  \NeveryPar

%D Which indeed we're going to use indeed!

%D Technically spoken we could have used the method we are 
%D going to present in the visual debugger. First we save 
%D the primitive \type{\everypar}: 
%D 
%D \starttypen 
%D \let\normaleverypar=\everypar
%D \stoptypen
%D
%D Next we allocate a \TOKENLIST\ named \type{\everypar}, 
%D which means that \type{\everypar} is no longer a primitive 
%D but something like \type{\toks44}. 
%D
%D \starttypen 
%D \newtoks\everypar
%D \stoptypen
%D
%D Because \TEX\ now executes \type{\normaleverypar} instead 
%D of \type{\everypar}, we are ready to assign some tokens to 
%D this internally known and used \TOKENLIST.
%D 
%D \starttypen 
%D \normaleverypar={all the things the system wants to do \the\everypar}
%D \stoptypen
%D 
%D Where the user can provide his own tokens to be expanded 
%D every time he expects them to expand. 
%D 
%D \starttypen 
%D \everypar={something the user wants to do}
%D \stoptypen
%D 
%D We don't use this method because it undoubtly leads to 
%D confusing situations, especially when other packages are 
%D used, but it's this kind of tricks that make \TEX\ so 
%D powerful. 

%D \macros
%D   {convertargument,convertcommand}
%D   {}
%D
%D Some persistent experimenting led us to the next macro. This
%D macro converts a parameter or an expanded macro to it's
%D textual meaning.
%D
%D \starttypen
%D \convertargument ... \to \command
%D \stoptypen
%D
%D For example,
%D
%D \starttypen
%D \convertargument{one \two \three{four}}\to\ascii
%D \stoptypen
%D
%D The resulting macro \type{\ascii} can be written to a file
%D or the terminal without problems. In \CONTEXT\ we use this
%D macro for generating registers and tables of contents.
%D
%D The second conversion alternative accepts a command:
%D
%D \starttypen
%D \convertcommand\command\to\ascii
%D \stoptypen
%D
%D Both commands accept the prefix \type{\doglobal} for global
%D assignments.

\def\doconvertargument#1>{}

\def\convertedcommand%
  {\expandafter\doconvertargument\meaning}

\long\def\convertargument#1\to#2%
  {\long\def\convertedargument{#1}%
   \dodoglobal\edef#2%
     {\convertedcommand\convertedargument}}

\long\def\convertcommand#1\to#2%
  {\dodoglobal\edef#2%
     {\convertedcommand#1}}

%D This is typically a macro that one comes to after reading
%D the \TEX book carefully. Even then, the definite solution
%D was found after rereading the \TEX book. The first
%D implementation was:
%D
%D \starttypen
%D \def\doconvertargument#1->#2\\\\{#2}
%D \stoptypen
%D
%D The \type{-}, the delimiter \type{\\\\} and the the second 
%D argument are completely redundant.

%D \macros
%D   {ExpandFirstAfter,ExpandSecondAfter,ExpandBothAfter}
%D   {}
%D
%D These three commands support expansion of arguments before
%D executing the commands that uses them. We can best
%D illustrate this with an example.
%D
%D \starttypen
%D \def\first  {alfa,beta,gamma}
%D \def\second {alfa,epsilon,zeta}
%D
%D \ExpandFirstAfter  \doifcommon {\first} {alfa}    {\message{OK}}
%D \ExpandSecondAfter \doifcommon {alfa}   {\second} {\message{OK}}
%D \ExpandBothAfter   \doifcommon {\first} {\second} {\message{OK}}
%D
%D \ExpandFirstAfter\processcommalist[\first]\message
%D
%D \ExpandAfter       \doifcommon {\first} {alfa}    {\message{OK}}
%D \stoptypen
%D
%D The first three calls result in the threefold message
%D \type{OK}, the fourth one shows the three elements of 
%D \type{\first}. The command \type{\ExpandFirstAfter} takes 
%D care of (first) arguments that are delimited by \type{[ ]},
%D but the faster \type{\ExpandAfter} does not.

%D RECONSIDER

\def\simpleExpandFirstAfter#1%
  {\edef\!!stringa{#1}%
   \@EA\ExpandCommand\@EA{\!!stringa}}

\def\complexExpandFirstAfter[#1]%
  {\edef\!!stringa{#1}%
   \@EA\ExpandCommand\@EA[\!!stringa]}

\def\ExpandFirstAfter#1%
  {\def\ExpandCommand{#1}%
   \complexorsimple{ExpandFirstAfter}}

\def\ExpandSecondAfter#1#2#3%
  {\def\!!stringa{#2}%
   \edef\!!stringb{#3}%
   \@EA#1\@EA{\@EA\!!stringa\@EA}\@EA{\!!stringb}}

% \def\ExpandSecondAfter#1#2#3%
%   {\toks0={#2}%
%    \edef\!!stringa{#3}%
%    \@EA\@EA\@EA#1\@EA\@EA\@EA{\@EA\the\@EA\toks0\@EA}\@EA{\!!stringa}}

\def\ExpandBothAfter#1#2#3%
  {\edef\!!stringa{#2}%
   \edef\!!stringb{#3}%
   \@EA\@EA\@EA#1\@EA\@EA\@EA{\@EA\!!stringa\@EA}\@EA{\!!stringb}}

\def\ExpandAfter#1#2%
  {\edef\!!stringa{#2}%
   \@EA#1\@EA{\!!stringa}}

%D Now we can for instance redefine \type{\ifinstringelse} as:

\def\ifinstringelse%
  {\ExpandBothAfter\v!ifinstringelse}

%D \macros
%D   {ConvertToConstant,ConvertConstantAfter}
%D   {}
%D
%D When comparing arguments with a constant, we can get into
%D trouble when this argument consists of tricky expandable
%D commands. One solution for this is converting the
%D argument to a string of unexpandable characters. To make
%D comparison possible, we have to convert the constant too
%D
%D \starttypen
%D \ConvertToConstant\doifelse {...} {...} {then ...} {else ...}
%D \stoptypen
%D
%D This construction is only needed when the first argument
%D can give troubles. Misuse can slow down processing.
%D
%D \starttypen
%D \ConvertToConstant\doifelse{\c!alfa}        {\c!alfa}{...}{...}
%D \ConvertToConstant\doifelse{alfa}           {\c!alfa}{...}{...}
%D \ConvertToConstant\doifelse{alfa}           {alfa}   {...}{...}
%D \ConvertToConstant\doifelse{alfa \alfa test}{\c!alfa}{...}{...}
%D \stoptypen
%D
%D In examples~2 and~3 both arguments equal, in~1 and~4
%D they differ.

\def\ConvertToConstant#1#2#3%
  {\expandafter\convertargument\expandafter{#2}\to\!!stringa
   \expandafter\convertargument\expandafter{#3}\to\!!stringb
   #1{\!!stringa}{\!!stringb}}

%D When the argument \type{#1} consists of commands, we had
%D better use
%D
%D \starttypen
%D \ConvertConstantAfter\processaction[#1][...]
%D \ConvertConstantAfter\doifelse{#1}{\v!iets}{}{}
%D \stoptypen
%D
%D This commands accepts things like:
%D
%D \starttypen
%D \v!constant
%D constant
%D \hbox to \hsize{\rubish}
%D \stoptypen
%D
%D As we will see in the core moudles, this macro permits 
%D constructions like:
%D
%D \starttypen
%D \setupfoottexts[...][...]
%D \setupfoottexts[margin][...][...]
%D \setupfoottexts[\v!margin][...][...]
%D \stoptypen
%D
%D where \type{...} can be anything legally \TEX.

\def\CheckConstantAfter#1#2%
  {\@EA\convertargument\v!prefix!\to\ascii
   \convertargument#1\to#2\relax
   \doifinstringelse{\ascii}{#2}
     {\expandafter\convertargument#1\to#2}
     {}}

\def\simpleConvertConstantAfter#1#2%
  {\CheckConstantAfter{#1}\asciiA
   \CheckConstantAfter{#2}\asciiB
   \ConvertCommand{\asciiA}{\asciiB}}

\def\complexConvertConstantAfter[#1]%
  {\doConvertConstantAfter{#1}%
   \@EA\ConvertCommand\@EA[\!!stringa]}

\def\ConvertConstantAfter#1%
  {\def\ConvertCommand{#1}%
   \complexorsimple{ConvertConstantAfter}}

%D \macros
%D   {assignifempty}
%D   {}
%D
%D We can assign a default value to an empty macro using:
%D
%D \starttypen
%D \assignifempty \macro {default value}
%D \stoptypen
%D
%D We don't explicitly test if the macro is defined.

\def\assignifempty#1#2%
  {\doifnot{#1}{}
     {\def#1{#2}}}

%D \macros
%D   {gobbleuntil,grabuntil,processbetween}
%D   {}
%D
%D In \TEX\ gobbling usually stand for skipping arguments, so
%D here are our gobbling macros.
%D
%D In \CONTEXT\ we use a lot of \type{\start}||\type{\stop}
%D like constructions. Sometimes, the \type{\stop} is used as a
%D hard coded delimiter like in:
%D
%D \starttypen
%D \def\startcommand#1\stopcommand%
%D   {... #1 ...}
%D \stoptypen
%D
%D In many cases the \type{\start}||\type{\stop} pair is
%D defined at format generation time or during a job. This
%D means that we cannot hardcode the \type{\stop} criterium.
%D Only after completely understanding \type{\csname} and
%D \type{\expandafter} I was able to to implement a solution,
%D starting with:
%D
%D \starttypen
%D \grabuntil{stop}\command
%D \stoptypen
%D
%D This commands executed, after having encountered
%D \type{\stop} the command \type{\command}. This command
%D receives as argument the text preceding the \type{\stop}.
%D This means that:
%D
%D \starttypen
%D \def\starthello%
%D   {\grabuntil{stophello}\message}
%D
%D \starthello Hello world!\stophello
%D \stoptypen
%D
%D results in: \type{\message{Hello world!}}.

\def\dograbuntil#1#2%
  {\long\def\next##1#1{#2{##1}}\next}

\def\grabuntil#1%
  {\expandafter\dograbuntil\expandafter{\csname#1\endcsname}}

%D The next command build on this mechanism:
%D
%D \starttypen
%D \processbetween{string}\command
%D \stoptypen
%D
%D Here:
%D
%D \starttypen
%D \processbetween{hello}\message
%D \starthello Hello again!\stophello
%D \stoptypen
%D
%D leads to: \type{\message{Hello again!}}. The command
%D
%D \starttypen
%D \gobbleuntil\command
%D \stoptypen
%D
%D is related to these commands. This one simply throws away
%D everything preceding \type{\command}.

\long\def\processbetween#1#2%
  {\setvalue{\s!start#1}%
     {\grabuntil{\s!stop#1}{#2}}}

\def\gobbleuntil#1%
  {\long\def\next##1#1{}\next}

%D \macros
%D   {groupedcommand}
%D   {}
%D
%D Commands often manipulate argument as in:
%D
%D \starttypen
%D \def\doezomaarwat#1{....#1....}
%D \stoptypen
%D
%D A disadvantage of this approach is that the tokens that
%D form \type{#1} are fixed the the moment the argument is read
%D in. Normally this is no problem, but for instance verbatim
%D environments adapt the \CATCODES\ of characters and therefore
%D are not always happy with already fixed tokens.
%D
%D Another problem arises when the argument is grouped not by
%D \type{{}} but by \type{\bgroup} and \type{\egroup}. Such an
%D argument fails, because the \type{\bgroup} is een as the
%D argument (which is quite normal).
%D
%D The next macro offers a solution for both unwanted
%D situations:
%D
%D \starttypen
%D \groupedcommand {before} {after}
%D \stoptypen
%D
%D Which can be used like:
%D
%D \starttypen
%D \def\cite%
%D   {\groupedcommand{\rightquote\rightquote}{\leftquote\leftquote}}
%D \stoptypen
%D
%D This command is equivalent to, but more 'robust' than:
%D
%D \starttypen
%D \def\cite#1%
%D   {\rightquote\rightquote#1\leftquote\leftquote}
%D \stoptypen
%D
%D One should say that the next implementation would suffice:
%D
%D \starttypen
%D \def\groupedcommand#1#2%
%D   {\def\BeforeGroup{#1\ignorespaces}%
%D    \def\AfterGroup{\unskip#2\egroup}%
%D    \bgroup\bgroup
%D    \aftergroup\AfterGroup
%D    \afterassignment\BeforeGroup
%D    \let\next=}
%D \stoptypen
%D
%D It did indeed, but one day we decided to support the
%D processing of boxes too:
%D
%D \starttypen
%D \def\rightword%
%D   {\groupedcommand{\hfill\hbox}{\parfillskip\!!zeropoint}}
%D
%D .......... \rightword{the right way}
%D \stoptypen
%D
%D Here \TEX\ typesets \type{\bf the right way} unbreakable
%D at the end of the line. The solution mentioned before does
%D not work here.
%D 
%D \starttypen
%D \long\unexpanded\def\groupedcommand#1#2%
%D   {\bgroup
%D    \long\def\BeforeGroup%
%D      {\bgroup#1\bgroup\aftergroup\AfterGroup}%
%D    \long\def\AfterGroup%
%D      {#2\egroup\egroup}%
%D    \afterassignment\BeforeGroup
%D    \let\next=}
%D \stoptypen
%D 
%D We used this method some time until the next alternative 
%D was needed. From now on we support both 
%D 
%D \starttypen 
%D to be \bold{bold} or not, that's the question
%D \stoptypen
%D 
%D and 
%D 
%D \starttypen 
%D to be {\bold bold} or not, that's the question
%D \stoptypen
%D 
%D This alternative checks for a \type{\bgroup} token first.
%D The internal alternative does not accept the box handling
%D mentioned before, but further nesting works all right. The
%D extra \type{\bgroup}||\type{\egroup} is needed to keep
%D \type{\AfterGroup} both into sight and local. 

\long\def\HandleGroup#1#2%
  {\bgroup
   \long\def\BeforeGroup%
     {\bgroup#1\bgroup\aftergroup\AfterGroup}%
   \long\def\AfterGroup%
     {#2\egroup\egroup}%
   \afterassignment\BeforeGroup
   \let\next=}

\long\def\HandleNoGroup#1#2%
  {\long\def\AfterGroup{#2\egroup}%
   \bgroup\aftergroup\AfterGroup#1}

%D These macros come together in:
%D
%D \starttypen
%D \long\unexpanded\def\groupedcommand#1#2%
%D   {\def\dogroupedcommand%
%D      {\ifx\next\bgroup
%D         \let\next=\HandleGroup
%D       \else
%D         \let\next=\HandleNoGroup
%D       \fi
%D       \next{#1}{#2}}%
%D    \futurelet\next\dogroupedcommand}
%D \stoptypen
%D 
%D From the missing paragraph number one can deduce that the 
%D last macro is not the real one yet. I considered it a 
%D nuisance that 
%D 
%D \starttypen
%D \kleur[groen] 
%D   {as gras} 
%D \stoptypen
%D 
%D was not interpreted as one would expect. This is due to the
%D fact that \type{\futurelet} obeys blank spaces, and a
%D line||ending token is treated as a blank space. So the final
%D implementation became: 

\long\unexpanded\def\groupedcommand#1#2%
  {\bgroup
   \def\dogroupedcommand%
     {\ifx\next\bgroup
        \def\\{\egroup\HandleGroup{#1}{#2}}%
      \else\ifx\next\blankspace
        \def\\ {\egroup\groupedcommand{#1}{#2}}% 
      \else
        \def\\{\egroup\HandleNoGroup{#1}{#2}}%
      \fi\fi
      \\}%
   \futurelet\next\dogroupedcommand}

%D Users should be aware of the fact that grouping can 
%D interfere with ones paragraph settings that are executed 
%D after the paragraph is closed. One should therefore 
%D explictly close the paragraph with \type{\par}, else the 
%D settings will be forgotten and not applied. So it's:
%D
%D \starttypen
%D \def\BoldRaggedCenter%
%D   {\groupedcommand{\raggedcenter\bf}{\par}}
%D \stoptypen

%D \macros
%D   {checkdefined}
%D   {}
%D
%D The bigger the system, the greater the change that
%D user defined commands collide with those that are part of
%D the system. The next macro gives a warning when a command is
%D already defined. We considered blocking the definition, but
%D this is not always what we want.
%D
%D \starttypen
%D \checkdefined {category} {class} {command}
%D \stoptypen
%D
%D The user is warned with the suggestion to use
%D \type{CAPITALS}. This suggestion is feasible, because
%D \CONTEXT only defines lowcased macros.

\def\checkdefined#1#2#3% redefined in mult-ini
  {\doifdefined{#3}
     {\writestatus{#1}{#2 #3 replaces a macro, use CAPITALS!}}}

%D \macros
%D   {GotoPar,GetPar}
%D   {}
%D
%D Typesetting a paragraph in a special way can be done by
%D first grabbing the contents of the paragraph and processing
%D this contents grouped. The next macro for instance typesets
%D a paragraph in boldface.
%D
%D \starttypen
%D \def\remark#1\par%
%D   {\bgroup\bf#1\egroup}
%D \stoptypen
%D
%D This macro has to be called like
%D
%D \starttypen
%D \remark some text ... ending with \par
%D \stoptypen
%D
%D Instead of \type{\par} we can of course use an empty line.
%D When we started typesetting with \TEX, we already had
%D produced lots of text in plain \ASCII. In producing such
%D simple formatted texts, we adopted an open layout, and when
%D switching to \TEX, we continued this open habit. Although
%D \TEX\ permits a cramped and badly formatted source, it adds
%D to confusion and sometimes introduces errors. So we prefer:
%D
%D \starttypen
%D \remark
%D
%D some text ... ending with an empty line
%D \stoptypen
%D
%D We are going to implement a mechanism that allows such open
%D specifications. The definition of the macro handling
%D \type{\remark} becomes:
%D
%D \starttypen
%D \def\remark%
%D   {\BeforePar{\bgroup\bf}%
%D    \AfterPar{\egroup}%
%D    \GetPar}
%D \stoptypen
%D
%D A macro like \type{\GetPar} can be defined in several
%D ways. The recent version, the fourth one in a row,
%D originally was far more complicated, but some functionality
%D has been moved to other macros.
%D
%D We start with the more simple but in some cases more
%D appropriate alternative is \type{\GotoPar}. This one leaves
%D \type{\par} unchanged and is therefore more robust. On the
%D other hand, \type{\AfterPar} is not supported.

\newtoks\BeforePar
\newtoks\AfterPar

\def\doGotoPar%
  {\ifx\nextchar\blankspace
     \let\donext=\GotoPar
   \else\ifx\nextchar\endoflinetoken
     \let\donext=\GotoPar
   \else
     \def\donext%
       {\the\BeforePar
        \BeforePar{}%
        \nextchar}%
   \fi\fi
   \donext}

\def\GotoPar%
  {\afterassignment\doGotoPar\let\nextchar=}

%D Its big brother \type{\GetPar} redefines the \type{\par}
%D primitive, which can lead to unexpected results, depending
%D in the context.

\def\GetPar%
  {\edef\next%
     {\BeforePar
        {\the\BeforePar
         \BeforePar{}%
         \bgroup
         \def\par%
           {\egroup
            \par
            \the\AfterPar
            \BeforePar{}%
            \AfterPar{}}}}%
   \next
   \GotoPar}

%D \macros
%D   {dowithpargument,dowithwargument}
%D   {}
%D
%D The next macros are a variation on \type{\GetPar}. When
%D macros expect an argument, it interprets a grouped sequence
%D of characters a one token. While this adds to robustness and
%D less ambiguous situations, we sometimes want to be a bit
%D more flexible, or at least want to be a bit more tolerant
%D to user input.
%D
%D We start with a commands that acts on paragraphs. This
%D command is called as:
%D
%D \starttypen
%D \dowithpargument\command
%D \dowithpargument{\command ... }
%D \stoptypen
%D
%D In \CONTEXT\ we use this one to read in the titles of
%D chapters, sections etc. The commands responsible for these
%D activities accept several alternative ways of argument
%D passing. In these examples, the \type{\par} can be omitted
%D when an empty line is present.
%D
%D \starttypen
%D \command{...}
%D \command ... \par
%D \command
%D   {...}
%D \command
%D   ... \par
%D \stoptypen
%D
%D We show two implementations, of which for the moment the
%D we prefier to use the second one:
%D
%D \starttypen
%D \def\dowithpargument#1%
%D   {\def\dodowithpargument%
%D      {\ifx\next\bgroup
%D         \def\next{#1}%
%D       \else
%D         \def\next####1 \par{#1{####1}}%
%D       \fi
%D       \next}%
%D    \futurelet\next\dodowithpargument}
%D \stoptypen
%D
%D A second and better implementation was:
%D 
%D \starttypen
%D \def\dowithpargument#1%
%D   {\def\nextpar##1 \par{#1{##1}}%
%D    \def\nextarg##1{#1{##1}}%
%D    \doifnextcharelse{\bgroup}
%D      {\nextarg}
%D      {\nextpar}}
%D \stoptypen
%D 
%D We ended up with an alternative that also accepts en empty 
%D argument. This command permits for instance chapters to 
%D have no title. 

\def\dowithpargument#1%
  {\def\nextpar##1 \par{#1{##1}}%
   \def\nextarg##1{#1{##1}}%
   \doifnextcharelse{\bgroup}
     {\nextarg}
     {\doifnextcharelse{\par}
        {#1{}}
        {\nextpar}}}

%D The \type{p} in the previous command stands for paragraph.
%D When we want to act upon words we can use the \type{w}
%D alternative.
%D
%D \starttypen
%D \dowithwargument\command
%D \dowithwargument{... \command ...}
%D \stoptypen
%D
%D The main difference bwteen two alternatives is in the
%D handling of \type{\par}'s. This time the space token acts
%D as a delimiter.
%D
%D \starttypen
%D \command{...}
%D \command ...
%D \command
%D   {...}
%D \command
%D   ...
%D \stoptypen
%D
%D Again there are two implementations possible:
%D
%D \starttypen
%D \def\dowithwargument#1%
%D   {\def\dodowithwargument%
%D      {\ifx\next\bgroup
%D         \def\next{#1}%
%D       \else
%D         \def\next####1 {#1{####1}}%
%D       \fi
%D       \next}%
%D    \futurelet\next\dodowithwargument}
%D \stoptypen
%D
%D We've chosen:

\def\dowithwargument#1%
  {\def\nextwar##1 {#1{##1}}%
   \def\nextarg##1{#1{##1}}%
   \doifnextcharelse{\bgroup}
     {\nextarg}
     {\nextwar}}

%D \macros
%D   {dorepeat,dorepeatwithcommand}
%D   {}
%D
%D When doing repetitive tasks, we stromgly advice to use
%D \type{\dorecurse}. The next alternative however, suits
%D better some of the \CONTEXT\ interface commands.
%D
%D \starttypen
%D \dorepeat[n*\command]
%D \stoptypen
%D
%D The value of the used \COUNTER\ can be called within
%D \type{\command} by \type{\repeater}.
%D
%D A slightly different alternative is:
%D
%D \starttypen
%D \dorepeatwithcommand[n*{...}]\command
%D \stoptypen
%D
%D When we call for something like:
%D
%D \starttypen
%D \dorepeatwithcommand[3*{Hello}]\message
%D \stoptypen
%D
%D we get ourselves three \type{\message{Hello}} messages in
%D a row. In both commands, the \type{n*} is optional. When this
%D specification is missing, the command executes once.

\long\def\dodorepeat[#1*#2*#3*]%
  {\doifelse{#3}{}
     {#1}
     {\dorecurse{#1}{#2}}}

\long\def\dorepeat[#1]%
  {\dodorepeat[#1***]}

\def\repeater%
  {\recurselevel}

\def\dorepeatwithcommand[#1]#2%
  {\def\p!dorepeatnot%
     {#2{#1}}%
   \def\p!dorepeatyes[##1*##2]%
     {\dorecurse{##1}{#2{##2}}}%
   \doifinstringelse{*}{#1}
     {\doifnumberelse{#1}{\p!dorepeatyes[#1]}{\p!dorepeatnot}}%
     {\p!dorepeatnot}}

%D \macros
%D   {appendtoks,prependtoks,flushtoks,dotoks}
%D   {}
%D
%D We use \TOKENLISTS\ sparsely within \CONTEXT, because the
%D comma separated lists are more suitable for the user
%D interface. Nevertheless we have:
%D
%D \starttypen
%D (\doglobal) \appendtoks ... \to\tokenlist
%D (\doglobal) \prependtoks ... \to\tokenlist
%D (\doglobal) \flushtoks\tokenlist
%D             \dotoks\tokenlist
%D \stoptypen
%D
%D Er worden eerst enkele klad||registers gedefinieerd. These
%D macros are clones of the ones implemented in page~378 of
%D Knuth's \TeX book.

\def\appendtoks#1\to#2%
  {\scratchtoks={#1}%
   \edef\next{\noexpand#2={\the#2\the\scratchtoks}}%
   \next
   \doglobal#2=#2} 

\def\prependtoks#1\to#2%
  {\scratchtoks={#1}%
   \edef\next{\noexpand#2={\the\scratchtoks\the#2}}%
   \next
   \doglobal#2=#2} 

\def\flushtoks#1%
  {\scratchtoks=#1\relax
   \doglobal#1={}% 
   \the\scratchtoks\relax}

\let\dotoks=\the

%D \macros
%D   {makecounter,pluscounter,minuscounter,
%D    resetcounter,setcounter,countervalue}
%D   {}
%D
%D Declaring, setting and resetting \COUNTERS\ can be doen
%D with the next set of commands.
%D
%D \starttypen
%D \makecounter   {name}
%D \pluscounter   {name}
%D \minuscounter  {name}
%D \resetcounter  {name}
%D \setcounter    {name} {value}
%D \countervalue  {name}
%D \stoptypen
%D
%D We prefer the use of global counters. This means that we
%D have to load \PLAIN\ \TEX\ in a bit different way:
%D
%D \starttypen
%D \let\oldouter=\outer
%D \let\outer=\relax
%D \input plain.tex
%D \let\outer=\oldouter
%D
%D \def\newcount%
%D   {\alloc@0\count\countdef\insc@unt}
%D \stoptypen
%D
%D First we show a solution in which we use real \COUNTERS.
%D Apart from some expansion, nothing special is done.
%D
%D \starttypen
%D \def\makecounter#1%
%D   {\expandafter\newcount\csname#1\endcsname}
%D
%D \def\pluscounter#1%
%D   {\expandafter\global\expandafter\advance\csname#1\endcsname by 1 }
%D
%D \def\minuscounter#1%
%D   {\expandafter\global\expandafter\advance\csname#1\endcsname by -1 }
%D
%D \def\resetcounter#1%
%D   {\expandafter\global\csname#1\endcsname=0 }
%D
%D \def\setcounter#1#2%
%D   {\expandafter\global\csname#1\endcsname=#2 }
%D
%D \def\countervalue#1%
%D   {\the\getvalue{#1}}
%D \stoptypen
%D
%D Because these macros are already an indirect way of working
%D with counters, there is no harm in using pseudo \COUNTERS\
%D here:

\def\makecounter#1%
  {\setxvalue{#1}{0}}

\def\pluscounter#1%
  {\scratchcounter=\getvalue{#1}\relax
   \advance\scratchcounter by 1\relax
   \setxvalue{#1}{\the\scratchcounter}}

\def\minuscounter#1%
  {\scratchcounter=\getvalue{#1}\relax
   \advance\scratchcounter by -1\relax
   \setxvalue{#1}{\the\scratchcounter}}

\def\resetcounter#1%
  {\setxvalue{#1}{0}}

\def\setcounter#1#2%
  {\scratchcounter=#2\relax
   \setxvalue{#1}{\the\scratchcounter}}

\def\countervalue#1%
  {\getvalue{#1}}

%D \macros
%D   {savecounter,restorecounter}
%D 
%D These two commands can be used to save and restore counter 
%D values. Only one level is saved. 

\def\savecounter#1%
  {\expanded{\setgvalue{!#1}{\getvalue{#1}}}}

\def\restorecounter#1%
  {\expanded{\setgvalue{#1}{\getvalue{!#1}}}}

%D \macros
%D   {beforesplitstring,aftersplitstring}
%D   {}
%D
%D These both commands split a string at a given point in two
%D parts, so \type{x.y} becomes \type{x} or \type{y}.
%D
%D \starttypen
%D \beforesplitstring test.tex\at.\to\filename
%D \aftersplitstring  test.tex\at.\to\extension
%D \stoptypen
%D
%D The first routine looks (and is indeed) a bit simpler than
%D the second one. The alternative looking more or less like
%D the first one did not always give the results we needed.
%D Both implementations show some insight in the manipulation
%D of arguments.

\def\beforesplitstring#1\at#2\to#3%
  {\def\dosplitstring##1#2##2#2##3\\%
     {\def#3{##1}}%
   \@EA\dosplitstring#1#2#2\\}

\def\aftersplitstring#1\at#2\to#3%
  {\def\dosplitstring##1#2##2@@@##3\\%
     {\def#3{##2}}%
   \@EA\dosplitstring#1@@@#2@@@\\}

%D \macros
%D   {removesubstring}
%D   {}
%D
%D A first application of the two routines defined above is:
%D
%D \starttypen
%D \removesubstringtest-\from first-last\to\nothyphenated
%D \stoptypen
%D 
%D Which in terms of \TEX\ looks like: 

\def\removesubstring#1\from#2\to#3%
  {\doifinstringelse{#1}{#2}
     {\beforesplitstring#2\at#1\to\!!stringa
      \aftersplitstring #2\at#1\to\!!stringb
      \edef#3{\!!stringa\!!stringb}%
      \def\next{\removesubstring#1\from#3\to#3}}
     {\let\next=\relax}%
   \next}

%D \macros
%D   {addtocommalist,removefromcommalist}
%D   {}
%D
%D When working with comma separated lists, oen sooner or
%D later want the tools to append or remove items from such a
%D list. When we add an item, we first check if it's already
%D there. This means that every item in the list is unique.
%D
%D \starttypen
%D \addtocommalist      {alfa}  \naam
%D \addtocommalist      {beta}  \naam
%D \addtocommalist      {gamma} \naam
%D \removefromcommalist {beta}  \naam
%D \stoptypen
%D
%D These commands can be prefixed with \type{\doglobal}. The
%D implementation of the second command is more complecated,
%D because we have to take leading spaces into account. Keep in
%D mind that useres may provide lists with spaces after the
%D commas. When one item is left, we also have to get rid of
%D trailing spaces.
%D
%D \starttypen
%D \def\words{alfa, beta, gamma, delta}
%D \def\words{alfa,beta,gamma,delta}
%D \stoptypen
%D
%D Removing an item takes more time than adding one.

\def\addtocommalist#1#2%
  {\doifelse{#2}{}
     {\dodoglobal\edef#2{#1}}
     {\edef\!!stringa{#2,,}%
      \beforesplitstring#2\at,,\to#2\relax
      \ExpandBothAfter\doifnotinset{#1}{#2}
        {\dodoglobal\edef#2{#2,#1}}}}

\def\doremovefromcommalist#1#2#3%
  {\edef\!!stringa{,,#3,,}%
   \beforesplitstring\!!stringa\at,#1#2,\to\!!stringb
   \aftersplitstring\!!stringa\at,#1#2,\to\!!stringc
   \edef#3{\!!stringb,\!!stringc}%
   \aftersplitstring#3\at,,\to#3\relax
   \beforesplitstring#3\at,,\to#3}

\def\dodofrontstrip[#1#2]#3%
  {\ifx#1\space
     \def#3{#2}%
   \else
     \def#3{#1#2}%
   \fi}%

\def\dofrontstrip#1%
  {\edef\!!stringa{#1}%
   \ifx\!!stringa\empty
   \else
     \@EA\dodofrontstrip\@EA[#1]#1%
   \fi}

\def\removefromcommalist#1#2%
  {\doremovefromcommalist{ }{#1}{#2}%
   \doremovefromcommalist{}{#1}{#2}%
   \dofrontstrip#2%
   \dodoglobal\edef#2{#2}}

%D \macros
%D   {withoutunit,withoutpt,
%D    PtToCm,
%D    numberofpoints,dimensiontocount}
%D   {}
%D
%D We can convert point into centimeters with:
%D
%D \starttypen
%D \PtToCm{dimension}
%D \stoptypen
%D
%D Splitting the value and the unit is done by:

\def\withoutunit#1#2%
  {\bgroup
   \dimen0=#1\relax
   \@EA\convertargument\the\dimen0\to\asciiA
   \@EA\convertargument#2\to\asciiB
   \@EA\@EA\@EA\beforesplitstring\@EA\asciiA\@EA\at\asciiB\to\!!stringa%
   \!!stringa
   \egroup}

\def\withoutpt#1%
  {\withoutunit{#1}{pt}}

\def\withoutcm#1%
  {\withoutunit{#1}{cm}}

%D A bit faster alternative is one that manipulates the
%D \CATCODES.

{\catcode`\.=\@@other
 \catcode`\p=\@@other
 \catcode`\t=\@@other
 \gdef\WITHOUTPT#1pt{#1}}

\def\withoutpt#1%
  {\expandafter\WITHOUTPT#1}

%D The capitals are needed because \type{p} and \type{t} have
%D \CATCODE~12, while macronames only permit tokens with the
%D \CATCODE~11. As a result we cannot use the \type{.group}
%D primitives. Those who want to know more about this kind of
%D manipulations, we advice to study the \TEX book in detail.
%D Because this macro does not do any assignment, we can use it
%D in the following way too.

\def\PtToCm#1%
  {\bgroup
   \scratchdimen=#1\relax
   \scratchdimen=0.0351459804\scratchdimen % 2.54/72.27
   \withoutpt{\the\scratchdimen}cm%
   \egroup}

%D We also support:
%D
%D \starttypen
%D \numberofpoints   {dimension}
%D \dimensiontocount {dimension} {\count}
%D \stoptypen
%D
%D Both macros return a rounded number.

\def\numberofpoints#1%
  {\scratchdimen=#1\relax
   \advance\scratchdimen by .5pt\relax
   \withoutpt{\the\scratchdimen}}

\def\dimensiontocount#1#2%
  {\scratchdimen=#1\relax
   \advance\scratchdimen by .5pt\relax
   #2=\scratchdimen
   \divide#2 by \!!maxcard\relax}

%D \macros
%D   {swapdimens,swapmacros}
%D   {}
%D
%D Simple but effective are the next two macros. There name
%D exactly states their purpose. The \type{\scratchdimen} and
%D \type{\!!stringa} can only be swapped when being the first
%D argument.

\def\swapdimens#1#2%
  {\scratchdimen=#1\relax
   #1=#2\relax
   #2=\scratchdimen}

\def\swapmacros#1#2%
  {\let\!!stringa=#1\relax
   \let#1=#2\relax
   \let#2=\!!stringa\relax}

%D \macros
%D   {setlocalhsize}
%D   {}
%D
%D Sometimes we need to work with the \type{\hsize} that is
%D corrected for indentation and left and right skips. The
%D corrected value is available in \type{\localhsize}, which
%D needs to be calculated with \type{\setlocalhsize} first.
%D
%D \starttypen
%D \setlocalhsize        \hbox to \localhsize{...}
%D \setlocalhsize[-1em]  \hbox to \localhsize{...}
%D \setlocalhsize[.5ex]  \hbox to \localhsize{...}
%D \stoptypen
%D
%D These examples show us that an optional can be used. The
%D value provided is added to \type{\localhsize}.

\newdimen\localhsize

\def\complexsetlocalhsize[#1]%
  {\localhsize=\hsize
   \advance\localhsize by -\parindent
   \advance\localhsize by -\leftskip
   \advance\localhsize by -\rightskip
   \advance\localhsize by #1\relax}

\def\simplesetlocalhsize%
  {\complexsetlocalhsize[\!!zeropoint]}

\definecomplexorsimple\setlocalhsize

%D \macros
%D   {processtokens}
%D   {}
%D
%D We fully agree with (most) typogaphers that inter||letter
%D spacing is only permitted in fancy titles, we provide a
%D macro that can be used to do so. Because this is
%D (definitely and fortunately) no feature of \TEX, we have to
%D step through the token list ourselves.
%D
%D \starttypen
%D \processtokens {before} {between} {after} {space} {tokens}
%D \stoptypen
%D
%D An example of a call is:
%D
%D \startbuffer
%D \processtokens {[} {+} {]} {\space} {hello world}
%D \stopbuffer
%D
%D \typebuffer
%D
%D This results in:
%D
%D \haalbuffer
%D
%D The list of tokens may contain spaces, while \type{\\},
%D \type{{}} and \type{\ } are handled as space too.

\def\dodoprocesstokens%
  {\ifx\next\lastcharacter
     \after
     \let\next=\relax
   \else\ifx\next\bgroup
     \def\next%
       {\dowithnextbox
          {\before\box\nextbox
           \let\before=\between
           \doprocesstokens}
          \hbox\bgroup}%          
   \else
     \expandafter\if\space\next 
       \before\white
     \else
       \before\next
     \fi
     \let\before=\between
     \let\next=\doprocesstokens
   \fi\fi
   \next}

\def\doprocesstokens% the space after = is essential
  {\afterassignment\dodoprocesstokens\let\next= }

\def\processtokens#1#2#3#4#5%
  {\bgroup
   \def\lastcharacter{\lastcharacter}%
   \def\space{ }%
   \let\\=\space
   \def\before{#1}%
   \def\between{#2}%
   \def\after{#3}%
   \def\white{#4}%
   \doprocesstokens#5\lastcharacter
   \egroup}

%D \macros
%D   {doifvalue,doifnotvalue,doifelsevalue,
%D    doifnothing,doifsomething,doifelsenothing,
%D    doifvaluenothing,doifvaluesomething,doifelsevaluenothing}
%D   {}
%D
%D These long named \type{\if} commands can be used to access
%D macros (or variables) that are normally accessed by using
%D \type{\getvalue}. Using these alternatives safes us three
%D tokens per call. Anyone familiar with the not||values
%D ones, can derive their meaning from the definitions.

           \def\doifvalue#1{\doif{\getvalue{#1}}}
        \def\doifnotvalue#1{\doifnot{\getvalue{#1}}}
       \def\doifelsevalue#1{\doifelse{\getvalue{#1}}}

         \def\doifnothing#1{\doif{#1}{}}
       \def\doifsomething#1{\doifnot{#1}{}}
     \def\doifelsenothing#1{\doifelse{#1}{}}

    \def\doifvaluenothing#1{\doif{\getvalue{#1}}{}}
  \def\doifvaluesomething#1{\doifnot{\getvalue{#1}}{}}
\def\doifelsevaluenothing#1{\doifelse{\getvalue{#1}}{}}

%D \macros
%D   {DOIF,DOIFELSE,DOIFNOT}
%D   {}
%D
%D \TEX\ is case sensitive. When comparing arguments, this
%D feature sometimes is less desirable, for instance when we
%D compare filenames. The next three alternatives upcase their
%D arguments before comparing them.
%D
%D \starttypen
%D \DOIF     {string1} {string2} {...}
%D \DOIFNOT  {string1} {string2} {...}
%D \DOIFELSE {string1} {string2} {then ...}{else ...}
%D \stoptypen
%D
%D We have to use a two||step implementation, because the
%D expansion has to take place outside \type{\uppercase}.

\def\p!DOIF#1#2#3%
  {\uppercase{\ifinstringelse{$#1$}{$#2$}}%
     #3%
   \fi}

\def\p!DOIFNOT#1#2#3%
  {\uppercase{\ifinstringelse{$#1$}{$#2$}}%
   \else
     #3%
   \fi}

\def\p!DOIFELSE#1#2#3#4%
  {\uppercase{\ifinstringelse{$#1$}{$#2$}}%
     #3%
   \else
     #4%
   \fi}

\def\DOIF     {\ExpandBothAfter\p!DOIF}
\def\DOIFNOT  {\ExpandBothAfter\p!DOIFNOT}
\def\DOIFELSE {\ExpandBothAfter\p!DOIFELSE}

%D \macros
%D   {stripcharacters,stripspaces}
%D   {}
%D
%D The next command was needed first when we implemented
%D the \CONTEXT\ interactivity macros. When we use labeled
%D destinations, we often cannot use all the characters we
%D want. We therefore strip some of the troublemakers, like
%D spaces, from the labels before we write them to the
%D \DVI||file, which passes them to for instance a PostScript
%D file.
%D
%D \starttypen
%D \stripspaces\from\one\to\two
%D \stoptypen
%D
%D Both the old string \type{\one} and the new one \type{\two}
%D are expanded. This command is a special case of:
%D
%D \starttypen
%D \stripcharacter\char\from\one\to\two
%D \stoptypen
%D
%D As we can see below, spaces following a control sequence are
%D to enclosed in \type{{}}.

\def\stripcharacter#1\from#2\to#3%
  {\def\dostripcharacter##1#1##2\end%
     {\edef\p!strippedstring{\p!strippedstring##1}%
      \doifemptyelse{##2}
        {\let\next=\relax}
        {\def\next{\dostripcharacter##2\end}}%
      \next}%
   \let\p!strippedstring=\empty
   \edef\!!stringa{#2}%
   \@EA\dostripcharacter\!!stringa#1\end
   \let#3=\p!strippedstring}

\def\stripspaces\from#1\to#2%
  {\stripcharacter{ }\from#1\to#2}

%D \macros
%D   {executeifdefined}
%D   {}
%D
%D \CONTEXT\ uses one auxiliary file for all data concerning
%D tables of contents, references, two||pass optimizations,
%D sorted lists etc. This file is loaded as many times as
%D needed. During such a pass we skip the commands thate are of
%D no use at that moment. Because we don't want to come into
%D trouble with undefined auxiliary commands, we call the
%D macros in a way similar to \type{\getvalue}. The next macro
%D take care of such executions and when not defined, gobbles
%D the unwanted arguments.
%D
%D \starttypen
%D \executeifdefined{name}\gobbleoneargument
%D \stoptypen
%D
%D We can of course globble more arguments using the
%D appropriate globbling command.

\def\executeifdefined#1#2%
  {\ifundefined{#1}%
     \def\next{#2}%
   \else
     \def\next{\getvalue{#1}}%
   \fi
   \next}

%D We considered an alternative imlementation accepting
%D commands directly, like:
%D
%D \starttypen
%D \executeifdefined\naam\gobblefivearguments
%D \stoptypen
%D
%D For the moment we don't need this one, so we stick to the
%D faster one. The more versatile alternative is:
%D
%D \starttypen
%D \def\executeifdefined#1#2%
%D   {\setnameofcommand{#1}%
%D    \@EA\ifundefined\@EA{\nameofcommand}
%D      \def\next{#2}%
%D    \else
%D      \def\next{\getvalue{\nameofcommand}}%
%D    \fi
%D    \next}
%D \stoptypen

%D \macros
%D   {doifsomespaceelse}
%D   {}
%D
%D The next command checks a string on the presence of a space
%D and executed a command accordingly.
%D
%D \starttypen
%D \doifsomespaceelse {tekst} {then ...} {else ...}
%D \stoptypen
%D
%D We use this command in \CONTEXT\ for determing if an
%D argument must be broken into words when made interactive.
%D Watch the use of \type{\noexpand}.

\long\def\doifsomespaceelse#1#2#3%
  {\def\c!doifsomespaceelse##1 ##2##3\war%
     {\if\noexpand##2@%
        #3%
      \else
        #2%
      \fi}%
   \c!doifsomespaceelse#1 @ @\war}

%D \macros
%D   {adaptdimension,balancedimensions}
%D   {}
%D
%D Again we introduce some macros that are closely related to
%D an interface aspect of \CONTEXT. The first command can be
%D used to adapt a \DIMENSION.
%D
%D \starttypen
%D \adaptdimension {dimension} {value}
%D \stoptypen
%D
%D When the value is preceed by a \type{+} or minus, the
%D dimension is advanced accordingly, otherwise it gets the
%D value.

\def\doadaptdimension#1#2\\#3\\%
  {\if#1+%
     \dodoglobal\advance#3 by #1#2\relax
   \else\if##1-%
     \dodoglobal\advance#3 by #1#2\relax
   \else
     \dodoglobal#3=#1#2\relax
   \fi\fi}

\def\adaptdimension#1#2%
  {\expandafter\doadaptdimension#2\\#1\\}

%D A second command takes two \DIMENSIONS. Both are adapted,
%D depending on the sign of the given value.
%D maat. This time we take the value as it is, and don't look
%D explicitly at the preceding sign.
%D
%D \starttypen
%D \balancedimensions {dimension 1} {dimension 2} {value}
%D \stoptypen
%D
%D When a positive value is given, the first dimension is
%D incremented, the second ond is decremented. A negative value
%D has the opposite result.

\def\balancedimensions#1#2#3%
  {\scratchdimen=#3\relax
   \redoglobal\advance#1 by \scratchdimen\relax
   \dodoglobal\advance#2 by -\scratchdimen\relax}

%D Both commands can be preceded by \type{\doglobal}. Here we
%D use \type{\redo} first, because \type{\dodo} resets the 
%D global character.

%D \macros
%D   {processconcanatedlist}
%D   {}
%D
%D Maybe a bit late, but here is a more general version of the
%D \type{\processcommalist} command. This time we don't handle
%D nesting but accept arbitrary seperators.
%D
%D \starttypen
%D \processconcanatedlist[list][separator]\command
%D \stoptypen
%D
%D One can think of things like:
%D
%D \starttypen
%D \processconcanatedlist[alfa+beta+gamma][+]\message
%D \stoptypen

\def\processconcanatedlist[#1][#2]#3%
  {\def\doprocessconcanatedlist##1##2#2%
     {\if]##1%
        \let\next=\relax
      \else\if]##2%
        \let\next=\relax
      \else\ifx\blankspace##2%
        #3{##1}%
        \let\next=\doprocessconcanatedlist
      \else
        #3{##1##2}%
        \let\next=\doprocessconcanatedlist
      \fi\fi\fi
      \next}%
   \doprocessconcanatedlist#1#2]#2}

%D \macros
%D   {processassignlist}
%D   {}
%D
%D Is possible to combine an assignment list with one
%D containing keywords. Assignments are treated accordingly,
%D keywords are treated by \type{\command}.
%D
%D \starttypen
%D \processassignlist[...=...,...=...,...]\commando
%D \stoptypen
%D
%D This command can be integrated in \type{\getparameters}, but
%D we decided best not to do so.

\def\processassignlist#1[#2]#3%
  {\def\p!dodogetparameter[##1=##2=##3]%
     {\doifnot{##3}{\relax}{#3{##1}}}%
   \def\p!dogetparameter##1%
     {\p!dodogetparameter[##1==\relax]}%
   \processcommalist[#2]\p!dogetparameter}

%D \macros
%D   {DoAfterFi,DoAfterFiFi}
%D   {}
%D
%D Sometimes \type{\fi}'s can get into the way. We can reach
%D over such a troublemaker with:
%D
%D \starttypen
%D \DoAfterFi{some commands}
%D \DoAfterFiFi{some commands}
%D \stoptypen
%D
%D It saves us a \type{\next} construction. Skipping
%D \type{\else...\fi} is more tricky, so this one is not
%D provided.

\def\DoAfterFi#1\fi{\fi#1}
\def\DoAfterFiFi#1\fi#2\fi{\fi\fi#1}

%D \macros
%D   {untextargument
%D    untexcommand}
%D 
%D When manipulating data(bases) and for instance generating 
%D index entries, the next three macros can be of help: 
%D 
%D \starttypen
%D \untextargument{...}\to\name
%D \untexcommand  {...}\to\name
%D \stoptypen
%D 
%D They remove braces and backslashes and give us something to 
%D sort. 

\def\untexsomething%
  {\bgroup
   \catcode`\{=\@@ignore
   \catcode`\}=\@@ignore
   \escapechar=-1
   \dountexsomething}

\long\def\dountexsomething#1#2\to#3%
  {\doglobal#1#2\to\untexedargument
   \egroup
   \let#3=\untexedargument}

\def\untexargument%
  {\untexsomething\convertargument}

\def\untexcommand%
  {\untexsomething\convertcommand}

%D \macros
%D   {ScaledPointsToBigPoints,ScaledPointsToWholeBigPoints}
%D 
%D One characteristic of \POSTSCRIPT\ and \PDF\ is that both 
%D used big points (\TEX's bp). The next macros convert points 
%D and scaled points into big points. 
%D 
%D \starttypen
%D \ScaledPointsToBigPoints      {number} \target
%D \ScaledPointsToWholeBigPoints {number} \target
%D \stoptypen
%D 
%D The magic factor $72/72.27$ can be found in most \TEX\ 
%D related books. 

\def\ScaledPointsToBigPoints#1#2%       
  {\scratchdimen=#1sp\relax
   \scratchdimen=.996264\scratchdimen    
   \edef#2{\withoutpt{\the\scratchdimen}}}

\def\ScaledPointsToWholeBigPoints#1#2%  
  {\scratchdimen=#1sp\relax
   \scratchdimen=.996264\scratchdimen    
   \scratchcounter=\scratchdimen
   \advance\scratchcounter by \!!medcard
   \divide\scratchcounter by \!!maxcard
   \edef#2{\the\scratchcounter}}

%D \macros
%D   {PointsToReal}
%D 
%D Points can be stripped from their suffix by using 
%D \type{\withoutpt}. The next macro enveloppes this macro. 
%D 
%D \starttypen
%D \PointsToReal {dimension} \target
%D \stoptypen 

\def\PointsToReal#1#2%
  {\scratchdimen=#1%  
   \edef#2{\withoutpt{\the\scratchdimen}}}

%D \macros 
%D  {dontleavehmode}
%D 
%D Sometimes when we enter a paragraph with some command, the 
%D first token gets the whole first line. We can prevent this 
%D by saying:
%D 
%D \starttypen
%D \dontleavehmode
%D \stoptypen
%D 
%D This command is used in for instance the language module 
%D \type{lang-ini}.

\def\dontleavehmode{\ifmmode\else$ $\fi}

%D \macro
%D   {handletokens}
%D
%D With the next macro we enter a critical area of macro 
%D expansion. What we want is a macro that looks like:
%D 
%D \handletokens some tokens\with \somemacro
%D 
%D At first sight the next implementation will suffice, but 
%D running this one shows that we loose the spaces. This is no 
%D surprise because we grab arguments and spaces preceding those
%D are just ignored. 
%D 
%D \starttypen
%D \def\nohandletokens#1\end%
%D   {}
%D 
%D \def\dohandletokens#1#2\end%
%D   {\ifx#1\endoftoken 
%D      \expandafter\nohandletokens
%D    \else
%D      \docommando{#1}%
%D      \expandafter\dohandletokens
%D    \fi
%D    #2\end}
%D 
%D \long\def\handletokens#1\with#2%
%D   {\let\docommando=#2\relax
%D    \dohandletokens#1\endoftoken\end}
%D \stoptypen
%D 
%D A second approach therefore grabs the indicidual characters 
%D by using \type{\afterassignment}, in which case the space is 
%D read in as space.  
%D 
%D \starttypen
%D \def\dodohandletokens%
%D   {\ifx\next\end \else
%D      \docommando\next
%D      \expandafter\dohandletokens
%D    \fi}
%D 
%D \def\dohandletokens%
%D   {\afterassignment\dodohandletokens\let\next= }
%D 
%D \long\def\handletokens#1\with#2%
%D   {\let\docommando=#2%
%D    \dohandletokens#1\end}
%D \stoptypen

%D \macro
%D   {counttoken}
%D
%D For the few occasion sthat we want to know the number of 
%D specific tokens in a string, we can use: 
%D 
%D \starttypen
%D \counttoken token\in string\to \count
%D \stoptypen
%D 
%D This macro, that for instance is used in \type{cont-tab}, 
%D takes a real counter. The macro can be preceded by \type 
%D {\doglobal}.  

\def\counttoken#1\in#2\to#3%
  {\redoglobal#3=0
   \def\!!stringa{#1}%
   \def\!!stringb{\end}%
   \def\docounttoken##1% obeys {}
     {\def\!!stringc{##1}%
      \ifx\!!stringb\!!stringc \else
        \ifx\!!stringa\!!stringc
          \dodoglobal\advance#3 by 1
        \fi
        \expandafter\docounttoken
      \fi}%
   \docounttoken#2\end
   \resetglobal}

%D \macros
%D   {splitofftokens}
%D
%D Running this one not always gives the expected results. 
%D Consider for instance the macro for which I originally 
%D wrote this token handler. 

\long\def\splitofftokens#1\from#2\to#3%
  {\ifnum#1>0
     \scratchcounter=#1\relax
     \def\dosplitofftokens##1%
       {\ifnum\scratchcounter>0
          \advance\scratchcounter by -1
          \edef#3{#3##1}%
        \fi}%
     % \let#3=\empty % #3 can be #2, so:  
     \@EA\let\@EA#3\@EA\empty
     \@EA\handletokens#2\with\dosplitofftokens
   \else
     \edef#3{#2}%
   \fi}

%D This macro can be called like:
%D 
%D \startbuffer[example] 
%D \splitofftokens10\from01234567 890123456789\to\test [\test]
%D \stopbuffer
%D 
%D However, the characters that we expect to find in
%D \type{\test} just don;t show up there. The reason for this
%D is not that logical but follows from \TEX's sometimes
%D mysterious way of expanding. Look at this: 
%D 
%D \startbuffer[next]
%D \def\next{a} \edef\test{\next}                  [\test]
%D \let\next=b  \edef\test{\test\next}             [\test]
%D \let\next=c  \edef\test{\next}                  [\test]
%D \let\next=d  \edef\test{\test\next}             [\test]
%D \let\next=e  \@EA\edef\@EA\test\@EA{\test\next} [\test]
%D \stopbuffer
%D 
%D \typebuffer[next]
%D 
%D Careful reading shows that inside an \type{\edef} macro's 
%D that are \type{\let} are not expanded!
%D 
%D \haalbuffer[next]
%D 
%D That's why we finally end up with a macro that looks ahead
%D by using an assignment, this time by using
%D \type{\futurelet}, and grabbing an argument as well. That
%D way we can handle both the sentinal and the blank space. 

\def\dodohandletokens#1%
  {\ifx\next\blankspace
     \docommando{ }%
   \fi
   \ifx#1\end \else
     \docommando{#1}%
     \expandafter\dohandletokens
   \fi}

\def\dohandletokens% 
  {\futurelet\next\dodohandletokens}

\long\def\handletokens#1\with#2%
  {\let\docommando=#2% #2 can be \docommando itself 
   \dohandletokens#1\end}

%D So our example finaly shows up as:
%D 
%D \haalbuffer[example]

\protect

\endinput
