%D \module
%D   [       file=page-lay,
%D        version=2000.10.20, % copied from main-001
%D          title=\CONTEXT\ Page Macros,
%D       subtitle=Layout Specification,
%D         author=Hans Hagen,
%D           date=\currentdate,
%D      copyright={PRAGMA / Hans Hagen \& Ton Otten}]
%C
%C This module is part of the \CONTEXT\ macro||package and is
%C therefore copyrighted by \PRAGMA. See mreadme.pdf for
%C details.

\writestatus{loading}{Context Page Macros / Layout Specification}

% to be translated into english 

%D Before you start wondering why some of the page related
%D modules skip upward or left in order to place elements, you
%D must realize that the reference point is the top left
%D corner of the main typesetting area. One reason for this
%D choice is that it suited some viewers that displayed page
%D areas. Another reason is that margins, edges and top and
%D bottom areas are kind of virtual, while the header, text
%D and footer areas normally determine the text flow.

\unprotect

%D First we get rid of the funny \TEX\ offset defaults of one
%D inch by setting them to zero.

\voffset = 0pt % setting this to -1in let's go metapost crazy
\hoffset = 0pt % setting this to -1in let's go metapost crazy

%D The dimensions related to layout areas are represented by
%D real dimensions.

\newdimen\papierhoogte       \papierhoogte         = 297mm
\newdimen\papierbreedte      \papierbreedte        = 210mm

\newdimen\printpapierhoogte  \printpapierhoogte    = 297mm
\newdimen\printpapierbreedte \printpapierbreedte   = 210mm

\newdimen\zethoogte                                % calculated
\newdimen\zetbreedte                               % calculated

\newdimen\teksthoogte                              % calculated
\newdimen\tekstbreedte                             % calculated

\newdimen\kopwit              \kopwit              = 2cm
\newdimen\rugwit              \rugwit              = \kopwit

\newdimen\hoofdhoogte         \hoofdhoogte         = 2cm
\newdimen\voethoogte          \voethoogte          = \hoofdhoogte

\newdimen\kopoffset           \kopoffset           = 0pt
\newdimen\rugoffset           \rugoffset           = \kopoffset

\newdimen\linkermargebreedte  \linkermargebreedte  = 3cm
\newdimen\rechtermargebreedte \rechtermargebreedte = \linkermargebreedte

\newdimen\linkerrandbreedte   \linkerrandbreedte   = 3cm
\newdimen\rechterrandbreedte  \rechterrandbreedte  = \linkerrandbreedte

\newdimen\bovenhoogte         \bovenhoogte         = 0cm
\newdimen\onderhoogte         \onderhoogte         = \bovenhoogte

%D We can save some tokens and fuzzy parameters by using a
%D symbolic name for the current set of layout parameters.

\let\currentlayout\empty

\def\layoutparameter#1%
  {\ifundefined{\??ly\currentlayout#1}%
     \getvalue{\??ly#1}%
   \else
     \getvalue{\??ly\currentlayout#1}%
   \fi}

%D Look how ugly a speed up looks:

\beginTEX

\def\layoutparameter#1%
  {\csname\??ly\@EA\ifx\csname
     \??ly\currentlayout#1\endcsname\relax\else\currentlayout
   \fi#1\endcsname}

\endTEX

%D Its \ETEX\ counterpart is:

\beginETEX \ifcsname

\def\layoutparameter#1%
  {\csname\??ly\ifcsname
     \??ly\currentlayout#1\endcsname\currentlayout
   \fi#1\endcsname}

\endETEX

%D Beause normal \TEX\ has at most 256 dimensions (of which a
%D substantial part is already in use), we provide a way to
%D generate a format with macro based alternatives. For a long
%D time, this used to be the default case. Beware: only fixed
%D dimensions can be used in calculations! By the way, the
%D gain in speed can hardly be called impressive and is roughly
%D 1 second on a 35 second run of 850 empty pages with a
%D couple of backgrounds only (which is far less than one
%D percent on a normal document).

\newif\iffixedlayoutdimensions \fixedlayoutdimensionstrue

\iffixedlayoutdimensions
  \let\@the\the 
\else
  \let\@the\empty
\fi 

%D The next series of dimensions are complemented by left
%D and rights ones.

\iffixedlayoutdimensions

  \newdimen \margeafstand
  \newdimen \randafstand
  \newdimen \margebreedte
  \newdimen \randbreedte

\else

  \def\margeafstand{\layoutparameter\c!margeafstand}
  \def\randafstand {\layoutparameter\c!randafstand}
  \def\margebreedte{\layoutparameter\c!marge}
  \def\randbreedte {\layoutparameter\c!rand}

\fi

%D Because a distance does not really makes sense when there
%D is no area, we use a zero distance in case there is no
%D area.

\iffixedlayoutdimensions

  \def\layoutdistance#1#2%
    {\ifdim\zeropoint<#1\layoutparameter#2\else\zeropoint\fi}

\else

  \def\layoutdistance#1#2%
    {\ifdim\!!zeropoint<#1\layoutparameter#2\else\!!zeropoint\fi}

\fi

%D The horizontal distances are:

\iffixedlayoutdimensions

  \newdimen \linkerrandafstand
  \newdimen \rechterrandafstand
  \newdimen \linkermargeafstand
  \newdimen \rechtermargeafstand

\else

  \def\linkerrandafstand
    {\layoutdistance\linkerrandbreedte\c!linkerrandafstand}

  \def\rechterrandafstand%
    {\layoutdistance\rechterrandbreedte\c!rechterrandafstand}

  \def\linkermargeafstand%
    {\layoutdistance\linkermargebreedte\c!linkermargeafstand}

  \def\rechtermargeafstand%
    {\layoutdistance\rechtermargebreedte\c!rechtermargeafstand}

\fi

%D The vertical distances are:

\iffixedlayoutdimensions

  \newdimen \bovenafstand
  \newdimen \hoofdafstand
  \newdimen \voetafstand
  \newdimen \onderafstand

\else

  \def\bovenafstand{\layoutdistance\bovenhoogte\c!bovenafstand}
  \def\hoofdafstand{\layoutdistance\hoofdhoogte\c!hoofdafstand}
  \def\voetafstand {\layoutdistance\voethoogte \c!voetafstand }
  \def\onderafstand{\layoutdistance\onderhoogte\c!onderafstand}

\fi

%D When fixed dimensions are used, we need to calculate the
%D distances:

\iffixedlayoutdimensions

  \def\setlayoutdimensions%
    {\global\margebreedte\layoutparameter\c!marge
     \global\randbreedte \layoutparameter\c!rand
     \global\margeafstand\layoutparameter\c!margeafstand
     \global\randafstand \layoutparameter\c!randafstand
     \global\linkerrandafstand  \layoutdistance\linkerrandbreedte  \c!linkerrandafstand
     \global\rechterrandafstand \layoutdistance\rechterrandbreedte \c!rechterrandafstand
     \global\linkermargeafstand \layoutdistance\linkermargebreedte \c!linkermargeafstand
     \global\rechtermargeafstand\layoutdistance\rechtermargebreedte\c!rechtermargeafstand
     \global\bovenafstand\layoutdistance\bovenhoogte\c!bovenafstand
     \global\hoofdafstand\layoutdistance\hoofdhoogte\c!hoofdafstand
     \global\global\voetafstand \layoutdistance\voethoogte \c!voetafstand
     \global\onderafstand\layoutdistance\onderhoogte\c!onderafstand}

\else

  \let\setlayoutdimensions\relax

\fi

%D \macros 
%D   {definepapersize}
%D
%D Before we start calculating layout dimensions, we will
%D first take care of paper sizes. The first argument can be
%D either an assignment (for defaults) or an identifier, in 
%D which case the second argument is an assignment.
%D
%D \showsetup{\y!definepapersize}

\def\definepapersize%
  {\dodoubleempty\dodefinepapersize}

\def\dodefinepapersize[#1][#2]%
  {\ifsecondargument
     \getparameters
       [\??pp#1] % geen \c!schaal, scheelt hash ruimte
       [\c!breedte=\@@ppbreedte,
        \c!hoogte=\@@pphoogte,
        \c!offset=\@@ppoffset,
        #2]%
   \else
     \getparameters[\??pp][#1]%
     \setuppapersize
   \fi}

%D For the moment we need to fake this macro.  

\ifx\setuppapersize\undefined 
  \let\setuppapersize\relax   
\fi

%D We set the defaults to the dimensions of an A4 sheet of 
%D paper. 

\definepapersize
  [\c!breedte=210mm,\c!hoogte=297mm,\c!offset=\!!zeropoint]

%D \macros
%D   {setuppapersize}
%D
%D When setting up the papersize on which to typeset and
%D print, we can also determine some more characteristics. 
%D
%D \showsetup{\y!setuppapersize}
%D
%D We keep track of these features with the following
%D variables. 

\chardef\papermirror   =0  \chardef\printmirror   =0
\chardef\paperrotation =0  \chardef\printrotation =0
\chardef\paperreverse  =0  \chardef\printreverse  =0
\chardef\paperlandscape=0  \chardef\printlandscape=0

\let\papersize\empty \let\printpapersize\empty

\def\paperscale{1} \newif\ifnegateprintbox

\def\setuppapersize%
  {\dodoubleempty\dosetuppapersize}

\def\dosetuppapersize[#1][#2]%
  {\doifinstringelse{=}{#1}
     {\getparameters[\??pp][#1]}
     {\doifinstringelse{=}{#2}
        {\getparameters[\??pp#1][#2]}
        {\dodosetuppapersize[#1][#2]}}}

\def\dodosetuppapersize[#1][#2]%
  {\ifsecondargument
     \xdef\restorepapersize%
       {\noexpand\setuppapersize[#1][#2]}%
     \dostelpapierrichtingin{#1}\paperlandscape\paperrotation\paperreverse\papermirror
     \dostelpapierrichtingin{#2}\printlandscape\printrotation\printreverse\printmirror
     \def\docommando##1%
       {\doifsomething{##1}{\doifdefined{\??pp##1\c!breedte}
          {\global\papierbreedte=\getvalue{\??pp##1\c!breedte}%
           \global\papierhoogte=\getvalue{\??pp##1\c!hoogte}%
           \calculatepaperoffsets{##1}%
           \xdef\papersize{##1}}}}%
     \processcommacommand[#1]\docommando
     \doifdefinedelse{\??pp#1\c!schaal}
       {\edef\paperscale{\getvalue{\??pp#1\c!schaal}}}
       {\edef\paperscale{1}}%
     \def\docommando##1%
       {\doifsomething{##1}{\doifdefined{\??pp##1\c!breedte}
          {\global\printpapierbreedte=\getvalue{\??pp##1\c!breedte}%
           \global\printpapierhoogte=\getvalue{\??pp##1\c!hoogte}%
           \xdef\printpapersize{##1}}}}%
     \processcommacommand[#2]\docommando
     \global\setdimentoatleast\papierbreedte     \!!onepoint
     \global\setdimentoatleast\papierhoogte      \!!onepoint
     \global\setdimentoatleast\printpapierbreedte\!!onepoint
     \global\setdimentoatleast\printpapierhoogte \!!onepoint
     \ifcase\paperlandscape\else
       \doglobal\swapdimens\papierbreedte\papierhoogte
     \fi
     \ifcase\printlandscape\else
       \doglobal\swapdimens\printpapierbreedte\printpapierhoogte
     \fi
     \ifdim\papierhoogte>\printpapierhoogte
       \global\printpapierhoogte=\papierhoogte
     \fi
     \ifdim\papierbreedte>\printpapierbreedte
       \global\printpapierbreedte=\papierbreedte
     \fi
     \calculatehsizes
     \calculatevsizes
     \recalculatelogos
     \recalculatebackgrounds
     \recalculatelayout
   \else\iffirstargument
     \setuppapersize[#1][#2]%
   \else\ifx\papersize\undefined\else
     \restorepapersize
   \fi\fi\fi}

\def\dostelpapierrichtingin#1#2#3#4#5%
  {\global\chardef#2=0
   \global\chardef#5=0
   \gdef#3{0}%
   \gdef#4{0}%
   \global\negateprintboxfalse
   \processallactionsinset
     [#1]
     [   \v!liggend=>\global\chardef#2=1,
      \v!gespiegeld=>\global\chardef#5=1,
       \v!geroteerd=>\gdef#3{90}\gdef#4{270},
        \v!negatief=>\global\negateprintboxtrue,
                 90=>\gdef#3{90}\gdef#4{270},
                180=>\gdef#3{180}\gdef#4{0},
                270=>\gdef#3{270}\gdef#4{90}]}

\ifx\calculatepaperoffsets\undefined

  \def\calculatepaperoffsets#1%
    {\scratchdimen=\getvalue{\??pp#1\c!offset}%
     \global\advance\papierbreedte by -2\scratchdimen
     \global\advance\papierhoogte  by -2\scratchdimen}

\fi

\let\restorepapersize\relax

\def\checkforems[#1]%
  {\def\docommando##1%
     {\beforesplitstring##1\at em\to\asciia
      \doifnot{\asciia}{##1}
        {\aftersplitstring\asciia\at=\to\asciia
         \doifsomething{\asciia}
           {\showmessage{\m!systems}{10}{##1}}}}%
   \processcommalist[#1]\docommando}

\def\recalculatelayout%
  {\global\linkermargebreedte =\layoutparameter\c!linkermarge
   \global\rechtermargebreedte=\layoutparameter\c!rechtermarge
   \global\linkerrandbreedte  =\layoutparameter\c!linkerrand
   \global\rechterrandbreedte =\layoutparameter\c!rechterrand
   \global\hoofdhoogte        =\layoutparameter\c!hoofd
   \global\voethoogte         =\layoutparameter\c!voet
   \global\onderhoogte        =\layoutparameter\c!onder
   \global\bovenhoogte        =\layoutparameter\c!boven
   \global\rugwit             =\layoutparameter\c!rugwit
   \global\kopwit             =\layoutparameter\c!kopwit
   \setlayoutdimensions % the rest of the `dimensions'
   \doifelse{\@@lygrid}{\v!ja}
     {\gridsnappingtrue}
     {\gridsnappingfalse}%
   \ifgridsnapping
     \widowpenalty=0 % is gewoon beter
     \clubpenalty =0 % zeker bij grids
   \else
     \widowpenalty=\defaultwidowpenalty
     \clubpenalty =\defaultclubpenalty
   \fi
   \stelwitruimtein
   \stelblankoin
   \doifelse{\layoutparameter\c!breedte}{\v!midden}
     {\global\zetbreedte=\papierbreedte
      \global\advance\zetbreedte by -\rugwit
      \scratchdimen=\layoutparameter\c!snijwit\relax
      \ifdim\scratchdimen=\zeropoint
        \scratchdimen=\rugwit
      \fi
      \global\advance\zetbreedte by -\scratchdimen}
     {\doifelse{\layoutparameter\c!breedte}{\v!passend}
        {\global\zetbreedte=\papierbreedte
         \global\advance\zetbreedte by -\rugwit
         \scratchdimen=\rugwit
         \advance\scratchdimen by -\linkerrandbreedte
         \advance\scratchdimen by -\linkerrandafstand
         \advance\scratchdimen by -\linkermargebreedte
         \advance\scratchdimen by -\linkermargeafstand
         \ifdim\scratchdimen<\zeropoint
           \scratchdimen=\zeropoint
         \fi
         \global\advance\zetbreedte by -\rechtermargeafstand
         \global\advance\zetbreedte by -\rechtermargebreedte
         \global\advance\zetbreedte by -\rechterrandafstand
         \global\advance\zetbreedte by -\rechterrandbreedte
         \global\advance\zetbreedte by -\scratchdimen}
        {\global\zetbreedte=\layoutparameter\c!breedte}}%
   \doifelse{\layoutparameter\c!regels}{}
     {\doifelse{\layoutparameter\c!hoogte}{\v!midden}
        {\global\zethoogte=\papierhoogte
         \global\advance\zethoogte by -\kopwit
         \scratchdimen=\layoutparameter\c!bodemwit\relax
         \ifdim\scratchdimen=\zeropoint
           \scratchdimen=\kopwit
         \fi
         \global\advance\zethoogte by -\scratchdimen}
        {\doifelse{\layoutparameter\c!hoogte}{\v!passend}
           {\global\zethoogte=\papierhoogte
            \global\advance\zethoogte by -\kopwit
            \scratchdimen=\kopwit
            \advance\scratchdimen by -\bovenhoogte
            \advance\scratchdimen by -\bovenafstand
            \ifdim\scratchdimen<\zeropoint
              \scratchdimen=\zeropoint
            \fi
            \global\advance\zethoogte by -\onderafstand
            \global\advance\zethoogte by -\onderhoogte
            \global\advance\zethoogte by -\scratchdimen}
           {\global\zethoogte=\layoutparameter\c!hoogte}}}
     {\global\zethoogte=\layoutparameter\c!regels\lineheight
      \global\advance\zethoogte by \hoofdhoogte
      \global\advance\zethoogte by \voethoogte}%
   \rugoffset=\layoutparameter\c!rugoffset
   \kopoffset=\layoutparameter\c!kopoffset
\global\setdimentoatleast\zetbreedte\!!onepoint
\global\setdimentoatleast\zethoogte\!!onepoint
   \calculatehsizes
   \calculatevsizes
   \recalculatelogos
   \recalculatebackgrounds}

\def\checklayout%
  {\doifsomething{\layoutparameter\c!regels}
     {\ifdim\zethoogte=\layoutparameter\c!regels\lineheight \else \recalculatelayout \fi}}

\appendtoks \checklayout \to \everystarttext

\def\checkcurrentlayout% 
  {\doifundefinedelse{\??ly\number\realfolio\c!status}
     {\doifonevenpaginaelse
        {\doifdefined{\??ly\v!oneven\c!status}{\stellayoutin[\v!oneven]}}
        {\doifdefined{\??ly\v!even  \c!status}{\stellayoutin[\v!even  ]}}}
     {\stellayoutin[\number\realfolio]}}

\appendtoks \checkcurrentlayout \to \everyaftershipout

\newif\ifdoublesidedprint

\def\presetcenterpagebox% in \stellayoutin !!!!!!!!!!!!!!!!
  {\doublesidedprintfalse
   \ExpandFirstAfter\processallactionsinset
     [\@@lyplaats]
     [      \v!midden=>{\setuppapersize[\c!links=\hss,\c!rechts=\hss,\c!boven=\vss,\c!onder=\vss]},
             \v!links=>{\setuppapersize[\c!links=,\c!rechts=\hss]},
            \v!rechts=>{\setuppapersize[\c!links=\hss,\c!rechts=]},
             \v!onder=>{\setuppapersize[\c!boven=\vss,\c!onder=]},
             \v!boven=>{\setuppapersize[\c!boven=,\c!onder=\vss]},%
      \v!dubbelzijdig=>\doublesidedprinttrue,
       \v!enkelzijdig=>\doublesidedprintfalse]}

\def\definelayout%
  {\dodoubleargument\dodefinelayout}

\def\dodefinelayout[#1][#2]%
  {\getparameters[\??ly#1][\c!status=\v!start,#2]}

\def\dodostellayoutin[#1][#2]%
  {\ConvertToConstant\doifnot{#2}{\v!reset}
     {\getparameters[\??ly#1][#2]%
      \checkforems[#2]}}

\def\dostellayoutin[#1][#2]%
  {\let\currentlayout\empty
   \ifsecondargument
     \dodostellayoutin[#1][#2]%
   \else\iffirstargument
     \doifassignmentelse{#1}
       {\dodostellayoutin[][#1]}
       {\doifnot{#1}{\v!reset}{\def\currentlayout{#1}}}%
   \fi\fi
   \recalculatelayout
   \presetcenterpagebox}

\def\stellayoutin%
  {\dodoubleempty\dostellayoutin}

\let\@@zahoogte=\!!zeropoint

\def\dopushpagedimensions%
  {\xdef\oldteksthoogte{\the\teksthoogte}%
   \xdef\oldvoethoogte{\the\voethoogte}%
   \global\let\@@zahoogte=\@@zahoogte}

\def\dopoppagedimensions%
  {\global\teksthoogte=\oldteksthoogte
   \global\voethoogte=\oldvoethoogte
   \recalculatelayout
   \global\let\pushpagedimensions=\dopushpagedimensions
   \global\let\poppagedimensions=\relax}

\let\poppagedimensions  = \relax
\let\pushpagedimensions = \dopushpagedimensions

% Elke \csname ... \endcsname wordt ook aangemaakt, dus ook
% in een test met \doifdefined. Bij veel bladzijden kan dit
% te veel macro's kosten. Vandaar de set \adaptedpages. Het
% kost tijd, maar scheelt macro's.

\let\adaptedpages\empty

\def\adaptpagedimensions%
  {\rawdoifinsetelse{\realfolio}{\adaptedpages}
     {\getvalue{\??za\realfolio}%
      \letbeundefined{\??za\realfolio}}
     {}}

\def\checkpagedimensions%
  {\poppagedimensions
   \adaptpagedimensions}

\def\reportpagedimensions%
  {\ifx\poppagedimensions\relax
   \else
     \spatie\@@zahoogte\spatie-
   \fi
   \realfolio}

\def\dodopaslayoutaan[#1]%
  {\getparameters[\??za][\c!hoogte=,\c!regels=,#1]%
   \pushpagedimensions
   \doifelsenothing{\@@zaregels}
     {\showmessage{\m!layouts}{1}{\@@zahoogte,\realfolio}}
     {\showmessage{\m!layouts}{1}{\@@zaregels\space\v!regels,\realfolio}%
      \def\@@zahoogte{\@@zaregels\openlineheight}}%
   \doifelse{\@@zahoogte}{\v!max}
     {\balancedimensions{\teksthoogte}{\voethoogte}{\voethoogte}}
     {\balancedimensions{\teksthoogte}{\voethoogte}{\@@zahoogte}}%
   \ifdim\voethoogte<\zeropoint
     \global\advance\teksthoogte by \voethoogte
     \global\voethoogte=\zeropoint
     \global\xdef\@@zahoogte{\layoutparameter\c!voet\spatie(\v!max)}%
   \fi
   \setvsize
   \global\pagegoal=\vsize  % nog corrigeren voor insertions ?
   \recalculatelogos
   \recalculatebackgrounds
   \global\let\pushpagedimensions=\relax
   \global\let\poppagedimensions=\dopoppagedimensions}

\def\dopaslayoutaan[#1][#2]%
  {\doifelsenothing{#2}
     {\dodopaslayoutaan[#1]}
     {\def\docommando##1%
        {\addtocommalist{##1}\adaptedpages
         \setgvalue{\??za##1}{\dodopaslayoutaan[#2]}}%
      \processcommalist[#1]\docommando
      \adaptpagedimensions}}

\def\paslayoutaan%
  {\dodoubleempty\dopaslayoutaan}

% describe interface 

%D Centering the paper area on the print area is determined
%D by the \type {top}, \type {bottom}, \type {left} and \type
%D {right} parameters. 

\def\centerpagebox#1%
  {\printpapierbreedte=\paperscale\printpapierbreedte
   \printpapierhoogte =\paperscale\printpapierhoogte
   \setbox#1=\vbox to \printpapierhoogte
     {\@@ppboven
      \hbox to \printpapierbreedte
        {\ifdoublesidedprint
           \doifbothsides \@@pplinks \box#1\@@pprechts
           \orsideone     \@@pplinks \box#1\@@pprechts
           \orsidetwo     \@@pprechts\box#1\@@pplinks  
         \od \else        \@@pplinks \box#1\@@pprechts
         \fi}%
      \par
      \@@pponder}}

\def\offsetprintbox#1%
  {\ifdim\kopoffset=\zeropoint % \relax 
     \ifdim\rugoffset=\zeropoint 
       \donefalse
     \else 
       \donetrue 
     \fi 
   \else 
     \donetrue 
   \fi  
   \ifdone
     \edef\next{\wd#1\the\wd#1\ht#1\the\ht#1\dp#1\the\dp#1}%
     \setbox#1=\vbox
       {%\forgetall
        \offinterlineskip
        \vskip\kopoffset
        \doifbothsides
          \hskip\rugoffset
        \orsideone
          \hskip\rugoffset
        \orsidetwo
          \hskip-\rugoffset
        \od
        \box#1}%
     \next
   \fi}

% \def\replicatebox#1#2#3%
%   {\setbox#1=\vbox
%      {%\forgetall
%       \offinterlineskip
%       \dorecurse{#3}
%         {\hbox{\dorecurse{#2}{\copy#1\hskip\@@lydx}\unskip}%
%          \vskip\@@lydy}
%       \unskip}}
% 
% \def\replicatepagebox#1%
%   {\ifnum\@@lynx>0 \ifnum\@@lyny>0
%      \replicatebox{#1}{\@@lynx}{\@@lyny}%
%    \fi\fi}

\def\replicatepagebox#1%
  {\ifnum\@@lynx>1 
     \donetrue
   \else\ifnum\@@lyny>1
     \donetrue
   \else
     \donefalse
   \fi\fi
   \ifdone 
     \setbox#1=\vbox
       {%\forgetall
        \offinterlineskip
        \dorecurse{\@@lyny}
          {\hbox{\dorecurse{\@@lynx}{\copy#1\hskip\@@lydx}\unskip}%
           \vskip\@@lydy}
        \unskip}%
   \fi}

\def\rotatepagebodybox#1#2#3%
  {\ifnum#2#3>0
     \setbox#1=\vbox
       {\edef\somerotation%
          {\ifdubbelzijdig\ifodd\realpageno#2\else#3\fi\else#2\fi}%
        \dorotatebox\somerotation\hbox{\box#1}}%
   \fi}

\def\rotatepaperbox#1%
  {\rotatepagebodybox{#1}\paperrotation\paperreverse}

\def\rotateprintbox#1%
  {\rotatepagebodybox{#1}\printrotation\printreverse}

\def\mirrorpagebodybox#1#2%
  {\ifcase#2\or
     \setbox#1=\vbox
       {\domirrorbox\vbox{\box#1}}%
   \fi}

\def\mirrorpaperbox#1%
  {\mirrorpagebodybox{#1}\papermirror}

\def\mirrorprintbox#1%
  {\mirrorpagebodybox{#1}\printmirror}

\def\scalepagebox#1%
  {\ifdim\@@lyschaal pt=1pt \else
     \setbox#1=\vbox
       {\schaal[\c!sx=\@@lyschaal,\c!sy=\@@lyschaal]{\box#1}}%
     \papierbreedte=\@@lyschaal\papierbreedte
     \papierhoogte =\@@lyschaal\papierhoogte
   \fi}

\def\negateprintbox#1%
  {\ifnegateprintbox
     \negatecolorbox{#1}%
   \fi}

\def\pagecutmarksymbol%
  {\the\realpageno}%

\def\addpagecutmarks#1%
  {\doif{\@@lymarkering}{\v!aan}
     {\let\cutmarksymbol=\pagecutmarksymbol
      \makecutbox{#1}}}

\def\addpagecolormarks#1%
  {\doif{\@@lymarkering}{\v!kleur}
     {\let\cutmarksymbol=\pagecutmarksymbol
      \makecutbox{#1}%
      \ifnum\horizontalcutmarks>1 \chardef\colormarkoffset=4 \fi
      \ifnum\verticalcutmarks  >1 \chardef\colormarkoffset=4 \fi
      \colormarkbox{#1}}}

\newif\ifdubbelzijdig \dubbelzijdigfalse
\newif\ifenkelzijdig  \enkelzijdigtrue

\def\doifsometextlineelse#1#2#3% ! omgekeerd !
  {\doifinsetelse{\getvalue{\??tk#1\v!tekst\c!status}}{\v!geen,\v!hoog}
     {#3}{#2}}

% NOG EENS NAGAAN WANNEER NU GLOBAL EN WANNEER NIET

\def\calculatevsizes% global needed in \recalculatelayoutregel
  {\redoglobal\teksthoogte=\zethoogte
   \doifsometextlineelse \v!hoofd
     {\redoglobal\advance\teksthoogte by -\hoofdhoogte
      \redoglobal\advance\teksthoogte by -\hoofdafstand}
     \donothing
   \doifsometextlineelse \v!voet
     {\redoglobal\advance\teksthoogte by -\voethoogte
      \redoglobal\advance\teksthoogte by -\voetafstand}
     \donothing
   \resetglobal
   \setvsize}

\def\calculatereducedvsizes%
  {\teksthoogte=\zethoogte
   \doifsometextlineelse \v!hoofd
     {\advance\teksthoogte by -\hoofdhoogte
      \advance\teksthoogte by -\hoofdafstand}
     {\hoofdhoogte=\zeropoint}%
   \doifsometextlineelse \v!voet
     {\advance\teksthoogte by -\voethoogte
      \advance\teksthoogte by -\voetafstand}
     {\voethoogte=\zeropoint}}

\def\freezetextwidth%        % \zetbreedte may be set to \tekstbreedte
  {\tekstbreedte=\zetbreedte % which is a tricky but valid value
   \doifsomething{\layoutparameter\c!tekstbreedte}
     {\tekstbreedte=\layoutparaneter\c!tekstbreedte}}

\def\calculatehsizes%
  {\freezetextwidth
   \sethsize}

% De onderstaande macro voert commando's uit, afhankelijk van
% het karakter van het paginanummer.
%
% \doifonevenpaginaelse{then-commando}{else-commando}

%D When we start at an even page, we need to swap the layout
%D differently. We cannot adapt the real page number, since
%D it is used in cross referencing. The next switch is set
%D when we start at an even page.

\newif\ifshiftedrealpageno

\def\doifonevenpaginaelse#1#2%
  {\ifshiftedrealpageno
     \ifodd\realpageno#2\else#1\fi
   \else
     \ifodd\realpageno#1\else#2\fi
   \fi}

\def\doifbothsidesoverruled#1\orsideone#2\orsidetwo#3\od%
  {\ifdubbelzijdig
     \doifonevenpaginaelse{#2}{#3}\relax
   \else
     #1\relax
   \fi}

\def\doifbothsides#1\orsideone#2\orsidetwo#3\od
  {\ifdubbelzijdig
     \ifenkelzijdig
       #1\relax
     \else
       \doifonevenpaginaelse{#2}{#3}\relax
     \fi
   \else
     #1\relax
   \fi}

\newdimen\texthoffset

\def\settexthoffset%
  {\doifbothsides
     \texthoffset\rugwit
   \orsideone
     \texthoffset\rugwit
   \orsidetwo
     \texthoffset\papierbreedte
     \advance\texthoffset-\rugwit
     \advance\texthoffset-\zetbreedte
   \od}

\def\initializepaper%
  {\iflocation
     \dosetuppaper
       {\papersize}
       {\the\papierbreedte}
       {\the\papierhoogte}%
   \else
     \dosetuppaper
       {\printpapersize}
       {\the\printpapierbreedte}
       {\the\printpapierhoogte}%
   \fi}

\def\goleftonpage%
  {\hskip-\linkermargeafstand
   \hskip-\linkermargebreedte
   \hskip-\linkerrandafstand
   \hskip-\linkerrandbreedte}

\iffixedlayoutdimensions

  \def\doswapmargins%
    {\let\doswapmargins=\relax % to prevent local swapping
     \swapdimens\linkermargeafstand\rechtermargeafstand
     \swapdimens\linkerrandafstand \rechterrandafstand
     \swapdimens\linkermargebreedte\rechtermargebreedte
     \swapdimens\linkerrandbreedte \rechterrandbreedte}

\else

  \def\dodoswapmargins#1#2%
    {\edef\!!stringa{\layoutparameter#1}%
     \edef\!!stringb{\layoutparameter#2}%
     \letvalue{\??ly\currentlayout#1}\!!stringb
     \letvalue{\??ly\currentlayout#2}\!!stringa}

  \def\doswapmargins%
    {\let\doswapmargins=\relax % to prevent local swapping
     \dodoswapmargins\c!linkermargeafstand\c!rechtermargeafstand
     \dodoswapmargins\c!linkerrandafstand \c!rechterrandafstand
     \swapdimens\linkermargebreedte\rechtermargebreedte
     \swapdimens\linkerrandbreedte \rechterrandbreedte}

\fi

\def\doifmarginswapelse#1#2%
  {\doifbothsides#1\orsideone#1\orsidetwo#2\od}

\def\swapmargins%
  {\doifmarginswapelse{}{\doswapmargins}}

%D \macros
%D   {showprint, showframe, showlayout, showsetups}
%D
%D We predefine a couple of tracing macros.
%D
%D \showsetup{\y!showprint}
%D \showsetup{\y!showframe}
%D \showsetup{\y!showlayout}
%D \showsetup{\y!showsetups}

\fetchruntimecommand \showprint  {page-run}
\fetchruntimecommand \showframe  {page-run}
\fetchruntimecommand \showlayout {page-run}
\fetchruntimecommand \showsetups {page-run}

%D The default dimensions are quite old and will not change.
%D The funny fractions were introduced when we went from fixed
%D dimensions to relative ones. Since \CONTEXT\ is a dutch
%D package, the dimensions are based on the metric system. The
%D asymmetrical layout is kind of handy for short
%D quick||and||dirty stapled documents.
%D
%D Although valid, it is not a real good idea to use
%D dimensions based on the \type {em} unit. First of all,
%D since there are no fonts loaded yet, this dimension makes
%D no sense, and second, you would loose track of values,
%D since they could change while going to a new page,
%D depending on the current font setting.

\stellayoutin
  [             \c!kopwit=.08417508418\papierhoogte,  %  2.5cm
                 \c!boven=\!!zeropoint,
          \c!bovenafstand=\!!zeropoint,
                 \c!hoofd=.06734006734\papierhoogte,  %  2.0cm
          \c!hoofdafstand=\!!zeropoint,
                \c!hoogte=.84175084175\papierhoogte,  % 25.0cm
           \c!voetafstand=\layoutparameter\c!hoofdafstand,
                  \c!voet=.06734006734\papierhoogte,  %  2.0cm
          \c!onderafstand=\layoutparameter\c!bovenafstand,
                 \c!onder=\!!zeropoint,
                \c!rugwit=.11904761905\papierbreedte, %  2.5cm
                  \c!rand=\!!zeropoint,
           \c!randafstand=\layoutparameter\c!margeafstand,
                 \c!marge=.12649983170\papierbreedte, %  snijwit-2*afstand
          \c!margeafstand=.02008341748\papierbreedte, %  12.0pt
            \c!linkerrand=\layoutparameter\c!rand,
     \c!linkerrandafstand=\layoutparameter\c!randafstand,
           \c!linkermarge=\layoutparameter\c!marge,
    \c!linkermargeafstand=\layoutparameter\c!margeafstand,
               \c!breedte=.71428571429\papierbreedte, %  15.0cm
   \c!rechtermargeafstand=\layoutparameter\c!margeafstand,
          \c!rechtermarge=\layoutparameter\c!marge,
    \c!rechterrandafstand=\layoutparameter\c!randafstand,
           \c!rechterrand=\layoutparameter\c!rand,
             \c!kopoffset=\!!zeropoint,
             \c!rugoffset=\!!zeropoint,
          \c!tekstbreedte=, % dangerous option
                \c!letter=,
             \c!markering=\v!uit,
                \c!plaats=, % \v!enkelzijdig, but empty is signal
                \c!schaal=1,
                    \c!nx=1,
                    \c!ny=1,
                    \c!dx=\!!zeropoint,
                    \c!dy=\!!zeropoint,
                  \c!grid=\v!nee,
                \c!regels=,
               \c!snijwit=\!!zeropoint,
              \c!bodemwit=\!!zeropoint]

%D First we define a whole range of (DIN) papersizes,
%D of which the A-series makes most sense.

\definepapersize [A0] [\c!breedte=841mm,\c!hoogte=1189mm]
\definepapersize [A1] [\c!breedte=594mm,\c!hoogte=841mm]
\definepapersize [A2] [\c!breedte=420mm,\c!hoogte=594mm]
\definepapersize [A3] [\c!breedte=297mm,\c!hoogte=420mm]
\definepapersize [A4] [\c!breedte=210mm,\c!hoogte=297mm]
\definepapersize [A5] [\c!breedte=148mm,\c!hoogte=210mm]
\definepapersize [A6] [\c!breedte=105mm,\c!hoogte=148mm]
\definepapersize [A7] [\c!breedte=74mm,\c!hoogte=105mm]
\definepapersize [A8] [\c!breedte=52mm,\c!hoogte=74mm]
\definepapersize [A9] [\c!breedte=37mm,\c!hoogte=52mm]

\definepapersize [B0] [\c!breedte=1000mm,\c!hoogte=1414mm]
\definepapersize [B1] [\c!breedte=707mm,\c!hoogte=1000mm]
\definepapersize [B2] [\c!breedte=500mm,\c!hoogte=707mm]
\definepapersize [B3] [\c!breedte=354mm,\c!hoogte=500mm]
\definepapersize [B4] [\c!breedte=250mm,\c!hoogte=354mm]
\definepapersize [B5] [\c!breedte=177mm,\c!hoogte=250mm]
\definepapersize [B6] [\c!breedte=125mm,\c!hoogte=177mm]
\definepapersize [B7] [\c!breedte=88mm,\c!hoogte=125mm]
\definepapersize [B8] [\c!breedte=63mm,\c!hoogte=88mm]
\definepapersize [B9] [\c!breedte=44mm,\c!hoogte=63mm]

\definepapersize [C0] [\c!breedte=917mm,\c!hoogte=1297mm]
\definepapersize [C1] [\c!breedte=649mm,\c!hoogte=917mm]
\definepapersize [C2] [\c!breedte=459mm,\c!hoogte=649mm]
\definepapersize [C3] [\c!breedte=324mm,\c!hoogte=459mm]
\definepapersize [C4] [\c!breedte=229mm,\c!hoogte=324mm]
\definepapersize [C5] [\c!breedte=162mm,\c!hoogte=229mm]
\definepapersize [C6] [\c!breedte=115mm,\c!hoogte=162mm]
\definepapersize [C7] [\c!breedte=81mm,\c!hoogte=115mm]
\definepapersize [C8] [\c!breedte=57mm,\c!hoogte=81mm]
\definepapersize [C9] [\c!breedte=40mm,\c!hoogte=57mm]

%D Because there are no standardized screen sizes, we define
%D a bunch of sizes with $4:3$ ratios. The \type {S6} size is
%D nearly as wide as a sheet of \type {A4} paper.

\definepapersize [S3] [\c!breedte=300pt,\c!hoogte=225pt]
\definepapersize [S4] [\c!breedte=400pt,\c!hoogte=300pt]
\definepapersize [S5] [\c!breedte=500pt,\c!hoogte=375pt]
\definepapersize [S6] [\c!breedte=600pt,\c!hoogte=450pt]

%D These are handy too: 

\definepapersize [S33] [\c!breedte=300pt,\c!hoogte=300pt]
\definepapersize [S44] [\c!breedte=400pt,\c!hoogte=400pt]
\definepapersize [S55] [\c!breedte=500pt,\c!hoogte=500pt]
\definepapersize [S66] [\c!breedte=600pt,\c!hoogte=600pt]

%D One may wonder if \TEX\ should be used for typesetting
%D \CDROM\ covers, but it does not hurt to have the paper size
%D ready.

\definepapersize [CD] [\c!breedte=120mm, \c!hoogte=120mm]

%D The next series is for our English speaking friends who
%D decided to stick to non metric values.

\definepapersize [letter]      [\c!breedte=8.5in,  \c!hoogte=11in]
\definepapersize [2*letter]    [\c!breedte=11in,   \c!hoogte=17in]
\definepapersize [legal]       [\c!breedte=8.5in,  \c!hoogte=14in]
\definepapersize [folio]       [\c!breedte=8.5in,  \c!hoogte=13in]
\definepapersize [executive]   [\c!breedte=7.25in, \c!hoogte=10.5in]

%D This set is for Tobias Burnus, who gave me the sizes.

\definepapersize [envelope 9]  [\c!breedte=8.88in, \c!hoogte=3.88in]
\definepapersize [envelope 10] [\c!breedte=9.5in,  \c!hoogte=4.13in]
\definepapersize [envelope 11] [\c!breedte=10.38in,\c!hoogte=4.5in]
\definepapersize [envelope 12] [\c!breedte=11.0in, \c!hoogte=4.75in]
\definepapersize [envelope 14] [\c!breedte=11.5in, \c!hoogte=5.0in]
\definepapersize [monarch]     [\c!breedte=7.5in,  \c!hoogte=3.88in]
\definepapersize [check]       [\c!breedte=8.58in, \c!hoogte=3.88in]
\definepapersize [DL]          [\c!breedte=220mm,  \c!hoogte=110mm]

%D We can now default to a reasonable size. We match the print
%D paper size with the typeset paper size. This setting should
%D come after the first layout specification (already done).

\setuppapersize
  [A4][A4]

%D We also set some of the parameters that will be used when
%D positioning the typeset paper onto the print paper.

\setuppapersize
  [\c!boven=,
   \c!onder=\vss,
   \c!links=,
   \c!rechts=\hss]

\protect \endinput
