%I n=Afbreekregels
%I c=\taal,\vertaal
%I
%I Er kunnen verschillende afbreekregels worden gehanteerd.
%I Deze worden ingesteld met het commando:
%I
%I   \taal[taal]
%I
%I waarbij voor taal kan worden ingevuld:
%I
%I   nl  nederlands
%I   en  engels
%I   du  duits
%I   fa  frans
%I
%I Er zijn ook verkorte commando's beschikbaar:
%I
%I   \nl \en \du \fa
%P
%I Er kan automatisch van taal gewisseld worden met het
%I commando:
%I
%I   \vertaal[nl=,en=,nl=,fa=,...]
%I
%I Afhankelijk van de actuele taal, wordt de toegekende tekst
%I gezet: \en this is an \vertaal[nl=voorbeeld,en=example],
%I \nl of in goed nederlands: een \vertaal.
%I
%I Als niets wordt meegegeven, dan wordt de laatst opgegeven
%I waarde gebruikt.

\endinput
