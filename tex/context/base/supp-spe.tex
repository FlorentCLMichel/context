%D \module
%D   [       file=supp-spe,
%D        version=1997.07.05,
%D          title=\CONTEXT\ Support Macros,
%D       subtitle=Specials,
%D         author=Hans Hagen,
%D           date=\currentdate,
%D      copyright={PRAGMA / Hans Hagen \& Ton Otten}]
%C
%C This module is part of the \CONTEXT\ macro||package and is
%C therefore copyrighted by \PRAGMA. Non||commercial use is
%C granted.

\unprotect

%D This module implements some \type{\special} manipulation 
%D macros. I needed these when I implemented the code that 
%D handles the conversion of \TPIC\ specials to \PDF\ code. 

\writestatus{loading}{Context Support Macros / Specials}

%D When interpreting specials we need to do some basic scanning.
%D For the moment we distinguish between three cases. We need 
%D 
%D \starttypen
%D \special{tag: arguments}
%D \special{tag arguments}
%D \special{tag}
%D \stoptypen
%D 
%D We cannot be sure that the first case isn't 
%D 
%D \starttypen
%D \special{tag:arguments}
%D \stoptypen
%D 
%D So we have to take care of that one too. 

%D \macros
%D   {redefinespecial}
%D 
%D Specials that are to be interpreted are defined with 
%D commands like:
%D 
%D \startbuffer[tmp-1]
%D \redefinespecial a: \using#1\endspecial%
%D   {let's execute special 'a:' using '#1'}
%D 
%D \redefinespecial a  \using#1\endspecial%
%D   {let's execute special 'a' using '#1'}
%D 
%D \redefinespecial a  \using#1\endspecial%
%D   {let's execute special 'a' using nothing}
%D \stopbuffer
%D
%D \typebuffer[tmp-1]
%D 
%D The first two always take an argument, the last one not. 
%D The definition of this redefinition macro is not that 
%D complex. The names are internally tagged with \type{\@rds@} 
%D which saves both time and space.  

\def\@rds@{@rds@}

\def\redefinespecial #1 %
  {\setvalue{\@rds@#1}}

%D \macros
%D   {mimmickspecials}
%D
%D Mimmicking specials is activated by saying: 
%D
%D \starttypen
%D \mimmickspecials
%D \stoptypen
%D
%D This commands redefines the \PLAIN\ \TEX\ primitive 
%D \type{\special}. 

\def\mimmickspecials%
  {\let\special=\domimmickspecial}

%D The special mimmicking macro first looks if it can find an
%D colon terminated tag, next it searches for a tag that end
%D with a space. If both cannot find, the tag itself is treated
%D without argument. 

\def\domimmickspecial#1%
  {\domimmickcolonspecial#1:\relax/:\relax/\end}

\def\domimmickcolonspecial#1:#2#3:\relax/#4\end%
  {\ifx#2\relax
     \domimmickspacespecial#1 \relax/ \relax/\end
   \else
     \dodomimmickspecial#1:\using#2#3\endspecial
   \fi}

\def\domimmickspacespecial#1 #2#3 \relax/#4\end%
  {\ifx#2\relax
     \dodomimmickspecial#1\using\endspecial
   \else
     \dodomimmickspecial#1\using#2#3\endspecial
   \fi}

\def\dodomimmickspecial#1\using#2\endspecial%
  {\expandafter\ifx\csname\@rds@#1\endcsname\relax % \doifdefinedelse
     \defaultspecial{#2}%
   \else  
     %\message{[mimmick special #1 with #2#3]}% 
     \getvalue{\@rds@#1}\using#2\endspecial
   \fi}

%D Now let's show that things work the way we want, using the 
%D previous definitions of tag~a.
%D 
%D \startbuffer[tmp-2]
%D \mimmickspecials
%D \special{a: 1 2 3 4 5}
%D \special{a: 1 2 3 4 5}
%D \special{a}
%D \stopbuffer
%D 
%D \typebuffer[tmp-2]
%D 
%D Which results in:
%D 
%D \startregels
%D \haalbuffer[tmp-1]
%D \haalbuffer[tmp-2]
%D \stopregels

%D \macros
%D   {mimmickspecial}
%D   
%D When needed, one can call a mimmicked special directly by 
%D saying for instance: 
%D 
%D \starttypen 
%D \mimmickspecial a: \using...\endspecial
%D \stoptypen
%D
%D This can be handy when specials have much in common.

\def\mimmickspecial #1 %
  {\getvalue{\@rds@#1}}

%D \macros
%D   {normalspecial,defaultspecial}
%D 
%D Unknown specials are passed to the default special handler. 
%D One can for instance ignore all further specials by saying
%D \type{\normalspecial}:
%D 
%D \starttypen
%D \def\defaultspecial#1{\normalspecial}
%D \stoptypen
%D 
%D But here we default to idle. 

\let\normalspecial =\special
\let\defaultspecial=\special

\protect 

\endinput 
