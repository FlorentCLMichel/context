%D \module
%D   [       file=spec-ps,
%D        version=1996.01.25,
%D          title=\CONTEXT\ Special Macros,
%D       subtitle=Adobe PostScript,
%D         author=Hans Hagen,
%D           date=\currentdate,
%D      copyright={PRAGMA / Hans Hagen \& Ton Otten}]
%C
%C This module is part of the \CONTEXT\ macro||package and is
%C therefore copyrighted by \PRAGMA. See mreadme.pdf for 
%C details. 

%D \macros
%D   {dostartgraymode,dostopgraymode,
%D    dostartrgbcolormode,dostartcmykcolormode,dostartgraycolormode,dostopcolormode,
%D    dostartrotation,dostoprotation,
%D    dostartscaling,dostopscaling,
%D    dostartmirroring,dostopmirroring,
%D    dostartnegative,dostopnegative,
%D    doselectfirstpaperbin,
%D    doselectsecondpaperbin}
%D   {}
%D
%D This implementation is straightforward and can be used as 
%D a default with postscript||drivers. We use \type{ps:} as 
%D opening, because most drivers support this. 

\startspecials[postscript]

\def\@@insertpostscriptliteral {ps: }
\def\@@insertpostscriptretain  {" } 

\definespecial\dostartgraymode#1%
  {\special
     {\@@insertpostscriptliteral 
        #1\space setgray}}

\definespecial\dostopgraymode%
  {\special
     {\@@insertpostscriptliteral 
        0 setgray}}

\definespecial\dostartrgbcolormode#1#2#3%
  {\special
     {\@@insertpostscriptliteral 
        #1\space #2\space #3\space setrgbcolor}}

\definespecial\dostartcmykcolormode#1#2#3#4%
  {\special
     {\@@insertpostscriptliteral 
        #1\space #2\space #3\space #4\space setcmykcolor}}

\definespecial\dostartgraycolormode#1%
  {\special
     {\@@insertpostscriptliteral 
        #1\space setgray}}

\definespecial\dostopcolormode
  {\special
     {\@@insertpostscriptliteral 
        0 setgray}}

\definespecial\doselectfirstpaperbin%
  {\special
     {\@@insertpostscriptliteral 
        statusdict begin 1 setpapertray end}}  % checken 

\definespecial\doselectsecondpaperbin%
  {\special
     {\@@insertpostscriptliteral 
        statusdict begin 0 setpapertray end}}  % checken 

\definespecial\dostartrotation#1% straight from the YandY manual
  {\special
     {\@@insertpostscriptliteral  
        gsave currentpoint currentpoint translate
        #1 rotate neg exch neg exch translate}}

\definespecial\dostoprotation%
  {\special
     {\@@insertpostscriptliteral  
        currentfont currentpoint grestore moveto setfont}}

\definespecial\dostartscaling#1#2% conforming the YandY manual
  {\special
     {\@@insertpostscriptliteral 
        gsave currentpoint currentpoint translate 
        #1\space #2\space scale neg exch neg exch translate}}
 
\definespecial\dostopscaling%
  {\special
     {\@@insertpostscriptliteral 
        currentfont currentpoint grestore moveto setfont}}

\definespecial\dostartmirroring% some translate and exch stuff from YandY
  {\special
     {\@@insertpostscriptliteral  
        gsave currentpoint currentpoint translate
        [-1 0 0 1 0 0] concat neg exch neg exch translate}}

\definespecial\dostopmirroring%
  {\special
     {\@@insertpostscriptliteral  
        currentfont currentpoint grestore moveto setfont}}

\definespecial\dostartnegative% 
  {\special
     {\@@insertpostscriptliteral  
        gsave {1 exch sub} settransfer}}

\definespecial\dostopnegative%
  {\special
     {\@@insertpostscriptliteral  
        grestore}}

%D \macros
%D   {doovalbox}
%D   {}
%D
%D This implementation is a mixture of several possible 
%D implementations. We use some constants that may not be 
%D changed. It took some time to find them, but these values 
%D offer quite accurate results. The macro calls for 
%D \type{\forgetall}, which resets indentation, skips and 
%D \type{\everypar}.
%D
%D Because a stroke follows the line, we correct for half of
%D the linewidth. Furthermore we use scaling to overcome some 
%D limitations in the precision ($<1$~sp) and to prevent 
%D rounding errors. We also do some correction for large 
%D values. We let PostScript compare some arguments with 
%D \type{a b eq {action} fi}.  
%D
%D The path is based on a macro of J.~Hefferon cs. We also 
%D tried the D.~Salomon implementation, but this did not work 
%D well, just like some other alternatives. 

% #1 = width            #2 = height           #3 = depth 
% #4 = linewidth        #5 = radius 
% #5 = stroke (ja,nee)  #7 = fill (ja,nee)

\def\@@insertpostscriptliteral {ps: }
\def\@@insertpostscriptretain  {postscript } % unknown 

\def\dosomeovalcalc#1#2#3%
  {\dimen2=#1sp%
   \advance\dimen2 by #2%   
   \ScaledPointsToBigPoints{\number\dimen2}#3}

\def\dosomeovalbox#1#2#3#4#5#6#7% 
  {\bgroup
   \dimen0=#4sp\divide\dimen0 by 2
   \dosomeovalcalc {0}{+\dimen0}\xmin
   \dosomeovalcalc{#1}{-\dimen0}\xmax
   \dosomeovalcalc{#2}{-\dimen0}\ymax
   \dosomeovalcalc{-#3}{+\dimen0}\ymin
   \dosomeovalcalc{#4}{0pt}\stroke
   \dosomeovalcalc{#5}{0pt}\radius
   \edef\dostroke{#6}%
   \edef\dofill{#7}% 
   \setbox0=\hbox
     {\ifnum\dostroke\dofill>0
        \special 
          {\@@insertpostscriptretain
             gsave
               newpath
               \xmin\space \radius\space add \ymin\space moveto
               \xmax\space \ymin\space \xmax\space \ymax\space \radius\space arcto
               \xmax\space \ymax\space \xmin\space \ymax\space \radius\space arcto
               \xmin\space \ymax\space \xmin\space \ymin\space \radius\space arcto
               \xmin\space \ymin\space \xmax\space \ymin\space \radius\space arcto 
               \xmin\space \radius\space add \ymin\space moveto
               16 {pop} repeat
               closepath
               (\dostroke) (1) eq 
                 {\stroke\space 0 ne 
                   {gsave 
                     \stroke\space setlinewidth 
                      stroke 
                    grestore} if} if
               (\dofill) (1) eq 
                 {fill} if
             grestore}%
      \fi}% 
   \wd0=#1sp\ht0=#2sp\dp0=#3sp\box0
   \egroup}

%D The indirect call permits a (rather) dirty hack, needed 
%D for \DVIPS\ support. 

\definespecial\doovalbox% 
  {\dosomeovalbox}

%D \macros 
%D   {dostartclipping,dostopclipping}
%D
%D These two specials are not yet working as expected.  

\definespecial\dostartclipping#1#2#3%
  {\ScaledPointsToBigPoints{#2}\width
   \ScaledPointsToBigPoints{#3}\height
   \grabMPclippath{#1}{0}{\width}{\height}
     {0 0 moveto \width 0 lineto \width \height lineto 0 \height\space lineto}%    
   \special
     {\@@insertpostscriptliteral\space  
      save newpath \MPclippath\space closepath clip}}

\definespecial\dostopclipping%
  {\special{\@@insertpostscriptliteral\space restore}}

\stopspecials

\endinput
