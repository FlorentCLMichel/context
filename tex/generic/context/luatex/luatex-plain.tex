%D \module
%D   [       file=luatex-plain,
%D        version=2009.12.01,
%D          title=\LUATEX\ Macros,
%D       subtitle=Plain Format,
%D         author=Hans Hagen,
%D           date=\currentdate,
%D      copyright={PRAGMA ADE \& \CONTEXT\ Development Team}]

%D We found out that in some cases the cm* fonts are not found and we don't
%D want to load them anyway. So we skip the font definitions.

\catcode`\{=1 % begin group
\catcode`\}=2 % end group
\catcode`\#=6 % macro parameter

\let\normalfont            \font
%let\normalskewchar        \skewchar
\let\normaltextfont        \textfont
\let\normalscriptfont      \scriptfont
\let\normalscriptscriptfont\scriptscriptfont

\def\font           #1=#2 {\immediate\write99{ignoring \string#1=\string#2}\let#1\nullfont}
%def\skewchar       #1=#2 {\immediate\write99{ignoring \string#1=\string#2}}
\def\textfont        #1=#2{\immediate\write99{ignoring \string\textfont        \string#1=\string#2}}
\def\scriptfont      #1=#2{\immediate\write99{ignoring \string\scriptfont      \string#1=\string#2}}
\def\scriptscriptfont#1=#2{\immediate\write99{ignoring \string\scriptscriptfont\string#1=\string#2}}

\input plain

\let\font            \normalfont
%let\skewchar        \normalskewchar
\let\textfont        \normaltextfont
\let\scriptfont      \normalscriptfont
\let\scriptscriptfont\normalscriptscriptfont

\directlua {tex.enableprimitives('', tex.extraprimitives())}

% We assume that pdf is used.

\ifdefined\pdfextension
    \input luatex-pdf \relax
\fi

\outputmode 1

% \outputmode 0 \magnification\magstep5

% We set the page dimensions because otherwise the backend does weird things
% when we have for instance this on a line of its own:
%
%   \hbox to 100cm {\hss wide indeed\hss}
%
% The page dimension calculation is a fuzzy one as there are some compensations
% for the \hoffset and \voffset and such. I remember long discussions and much
% trial and error in figuring this out during pdftex development times. Where
% a dvi driver will project on a papersize (and thereby clip) the pdf backend
% has to deal with the lack of a page concept on tex by some guessing. Normally
% a macro package will set the dimensions to something reasonable anyway.

\pagewidth   8.5truein
\pageheight 11.0truein

% We load some code at runtime:

\everyjob \expandafter {%
    \the\everyjob
    %D \module
%D   [       file=luatex-basics,
%D        version=2009.12.01,
%D          title=\LUATEX\ Support Macros,
%D       subtitle=Attribute Allocation,
%D         author=Hans Hagen,
%D           date=\currentdate,
%D      copyright={PRAGMA ADE \& \CONTEXT\ Development Team}]

%D As soon as we feel the need this file will file will contain an extension
%D to the standard plain register allocation. For the moment we stick to a
%D rather dumb attribute allocator. We start at 256 because we don't want
%D any interference with the attributes used in the font handler.

\ifx\newattribute\undefined \else \endinput \fi

\newcount \lastallocatedattribute \lastallocatedattribute=255

\def\newattribute#1%
  {\global\advance\lastallocatedattribute 1
   \attributedef#1\lastallocatedattribute}

% maybe we will have luatex-basics.lua some day for instance when more
% (pdf) primitives have moved to macros)

\directlua {

    gadgets = gadgets or { } % reserved namespace

    gadgets.functions = { }
    local registered  = {}

    function gadgets.functions.reverve()
        local numb = token.scan_int()
        local name = token.scan_string()
        local okay = string.gsub(name,"[\string\\ ]","")
        registered[okay] = numb
        texio.write_nl("reserving lua function '"..okay.."' with number "..numb)
    end

    function gadgets.functions.register(name,f)
        local okay = string.gsub(name,"[\string\\ ]","")
        local numb = registered[okay]
        if numb then
            texio.write_nl("registering lua function '"..okay.."' with number "..numb)
            lua.get_functions_table()[numb] = f
        else
            texio.write_nl("lua function '"..okay.."' is not reserved")
        end
    end

}

\newcount\lastallocatedluafunction

\def\newluafunction#1%
  {\ifdefined#1\else
     \global\advance\lastallocatedluafunction 1
     \global\chardef#1\lastallocatedluafunction
     \directlua{gadgets.functions.reserve()}#1{\detokenize{#1}}%
   \fi}

% an example of usage (if we ever support it it will go to the plain gadgets module):
%
% \newluafunction\UcharcatLuaOne
% \newluafunction\UcharcatLuaTwo
%
% \directlua {
%
%     local cct = nil
%     local chr = nil
%
%     gadgets.functions.register("UcharcatLuaOne",function()
%         chr = token.scan_int()
%         cct = tex.getcatcode(chr)
%         tex.setcatcode(chr,token.scan_int())
%         tex.sprint(unicode.utf8.char(chr))
%     end)
%
%     gadgets.functions.register("UcharcatLuaTwo",function()
%         tex.setcatcode(chr,cct)
%     end)
%
% }
%
% \def\Ucharcat
%   {\expandafter\expandafter\expandafter\luafunction
%    \expandafter\expandafter\expandafter\UcharcatLuaTwo
%    \luafunction\UcharcatLuaOne}
%
% A:\the\catcode65:\Ucharcat 65 11:A:\the\catcode65\par
% A:\the\catcode65:\Ucharcat 65  5:A:\the\catcode65\par
% A:\the\catcode65:\Ucharcat 65 11:A:\the\catcode65\par

\endinput
%
    % language=uk

\environment luatex-style
\environment luatex-logos

\startcomponent luatex-fonts

\startchapter[reference=fonts,title={Font structure}]

\section {The font tables}

All \TEX\ fonts are represented to \LUA\ code as tables, and internally as
\CCODE~structures. All keys in the table below are saved in the internal font
structure if they are present in the table returned by the \type {define_font}
callback, or if they result from the normal \TFM|/|\VF\ reading routines if there
is no \type {define_font} callback defined.

The column \quote {\VF} means that this key will be created by the \type
{font.read_vf()} routine, \quote {\TFM} means that the key will be created by the
\type {font.read_tfm()} routine, and \quote{used} means whether or not the
\LUATEX\ engine itself will do something with the key.

The top|-|level keys in the table are as follows:

\starttabulate[|l|c|c|c|l|p|]
\BC key                     \BC vf  \BC tfm \BC used \BC value type \BC description \NC \NR
\NC \type{name}             \NC yes \NC yes \NC yes  \NC string     \NC metric (file) name \NC \NR
\NC \type{area}             \NC no  \NC yes \NC yes  \NC string     \NC (directory) location, typically empty \NC \NR
\NC \type{used}             \NC no  \NC yes \NC yes  \NC boolean    \NC indicates usage (initial: false) \NC \NR
\NC \type{characters}       \NC yes \NC yes \NC yes  \NC table      \NC the defined glyphs of this font \NC \NR
\NC \type{checksum}         \NC yes \NC yes \NC no   \NC number     \NC default: 0 \NC \NR
\NC \type{designsize}       \NC no  \NC yes \NC yes  \NC number     \NC expected size (default: 655360 == 10pt) \NC \NR
\NC \type{direction}        \NC no  \NC yes \NC yes  \NC number     \NC default: 0 \NC \NR
\NC \type{encodingbytes}    \NC no  \NC no  \NC yes  \NC number     \NC default: depends on \type {format} \NC \NR
\NC \type{encodingname}     \NC no  \NC no  \NC yes  \NC string     \NC encoding name \NC \NR
\NC \type{fonts}            \NC yes \NC no  \NC yes  \NC table      \NC locally used fonts \NC \NR
\NC \type{psname}           \NC no  \NC no  \NC yes  \NC string     \NC This is the \POSTSCRIPT\ fontname in the incoming font
                                                                        source, and it's used as fontname identifier in the \PDF\
                                                                        output. This has to be a valid string, e.g.\ no spaces
                                                                        and such, as the backend will not do a cleanup. This gives
                                                                        complete control to the loader. \NC \NR
\NC \type{fullname}         \NC no  \NC no  \NC yes  \NC string     \NC output font name, used as a fallback in the \PDF\ output
                                                                        if the \type {psname} is not set \NC \NR
\NC \type{header}           \NC yes \NC no  \NC no   \NC string     \NC header comments, if any \NC \NR
\NC \type{hyphenchar}       \NC no  \NC no  \NC yes  \NC number     \NC default: \TEX's \type {\hyphenchar} \NC \NR
\NC \type{parameters}       \NC no  \NC yes \NC yes  \NC hash       \NC default: 7 parameters, all zero \NC \NR
\NC \type{size}             \NC no  \NC yes \NC yes  \NC number     \NC loaded (at) size. (default: same as designsize) \NC \NR
\NC \type{skewchar}         \NC no  \NC no  \NC yes  \NC number     \NC default: \TEX's \type {\skewchar} \NC \NR
\NC \type{type}             \NC yes \NC no  \NC yes  \NC string     \NC basic type of this font \NC \NR
\NC \type{format}           \NC no  \NC no  \NC yes  \NC string     \NC disk format type \NC \NR
\NC \type{embedding}        \NC no  \NC no  \NC yes  \NC string     \NC \PDF\ inclusion  \NC \NR
\NC \type{filename}         \NC no  \NC no  \NC yes  \NC string     \NC the name of the font on disk \NC \NR
\NC \type{tounicode}        \NC no  \NC yes \NC yes  \NC number     \NC When this is set to~1 \LUATEX\ assumes per|-|glyph
                                                                        tounicode entries are present in the font. \NC \NR
\NC \type{stretch}          \NC no  \NC no  \NC yes  \NC number     \NC the \quote {stretch} value from \type
                                                                        {\expandglyphsinfont} \NC \NR
\NC \type{shrink}           \NC no  \NC no  \NC yes  \NC number     \NC the \quote {shrink} value from \type
                                                                        {\expandglyphsinfont} \NC \NR
\NC \type{step}             \NC no  \NC no  \NC yes  \NC number     \NC the \quote {step} value from \type
                                                                        {\expandglyphsinfont} \NC \NR
\NC \type{expansion_factor} \NC no  \NC no  \NC no   \NC number     \NC the actual expansion factor of an expanded font \NC \NR
\NC \type{attributes}       \NC no  \NC no  \NC yes  \NC string     \NC the \type {\pdffontattr} \NC \NR
\NC \type{cache}            \NC no  \NC no  \NC yes  \NC string     \NC This key controls caching of the \LUA\ table on the
                                                                        \TEX\ end where \type {yes} means: use a reference to
                                                                        the table that is passed to \LUATEX\ (this is the
                                                                        default), and no \type {no} means: don't store the
                                                                        table reference, don't cache any \LUA\ data for this
                                                                        font while \type {renew} means: don't store the table
                                                                        reference, but save a reference to the table that is
                                                                        created at the first access to one of its fields in font.
                                                                        Note: the saved reference is thread|-|local, so be
                                                                        careful when you are using coroutines: an error will be
                                                                        thrown if the table has been cached in one thread, but
                                                                        you reference it from another thread. \NC \NR
\NC \type{nomath}           \NC no  \NC no  \NC yes  \NC boolean    \NC This key allows a minor speedup for text fonts. If it
                                                                        is present and true, then \LUATEX\ will not check the
                                                                        character entries for math|-|specific keys. \NC \NR
\NC \type{oldmath}          \NC no  \NC no  \NC yes  \NC boolean    \NC This key flags a font as representing an old school \TEX\
                                                                        math font and disables the \OPENTYPE\ code path. \NC \NR
\NC \type{slant}            \NC no  \NC no  \NC yes  \NC number     \NC This has the same semantics as the \type {SlantFont}
                                                                        operator in font map files. \NC \NR
\NC \type{extent}           \NC no  \NC no  \NC yes  \NC number     \NC This has the same semantics as the \type {ExtendFont}
                                                                        operator in font map files. \NC \NR
\stoptabulate

The key \type {name} is always required. The keys \type {stretch}, \type
{shrink}, \type {step} only have meaning when used together: they can be used to
replace a post|-|loading \type {\expandglyphsinfont} command. The \type
{auto_expand} option is not supported in \LUATEX. In fact, the primitives
that create expanded or protruding copies are probably only useful when used with
traditional fonts because all these extra \OPENTYPE\ properties are kept out of
the picture. The \type {expansion_factor} is value that can be present inside a
font in \type {font.fonts}. It is the actual expansion factor (a value between
\type {-shrink} and \type {stretch}, with step \type {step}) of a font that was
automatically generated by the font expansion algorithm.

Because we store the actual state of expansion with each glyph and don't have
special font instances, we can change some font related parameters before lines
are constructed, like:

\starttyping
font.setexpansion(font.current(),100,100,20)
\stoptyping

This is mostly meant for experiments (or an optimizing routing written in \LUA)
so there is no primitive.

The key \type {attributes} can be used to set font attributes in the \PDF\ file.
The key \type {used} is set by the engine when a font is actively in use, this
makes sure that the font's definition is written to the output file (\DVI\ or
\PDF). The \TFM\ reader sets it to false. The \type {direction} is a number
signalling the \quote {normal} direction for this font. There are sixteen
possibilities:

\starttabulate[|c|c|c|c|]
\BC number   \BC meaning   \BC number    \BC meaning   \NC \NR
\NC \type{0} \NC \type{LT} \NC \type {8} \NC \type{TT} \NC \NR
\NC \type{1} \NC \type{LL} \NC \type {9} \NC \type{TL} \NC \NR
\NC \type{2} \NC \type{LB} \NC \type{10} \NC \type{TB} \NC \NR
\NC \type{3} \NC \type{LR} \NC \type{11} \NC \type{TR} \NC \NR
\NC \type{4} \NC \type{RT} \NC \type{12} \NC \type{BT} \NC \NR
\NC \type{5} \NC \type{RL} \NC \type{13} \NC \type{BL} \NC \NR
\NC \type{6} \NC \type{RB} \NC \type{14} \NC \type{BB} \NC \NR
\NC \type{7} \NC \type{RR} \NC \type{15} \NC \type{BR} \NC \NR
\stoptabulate

These are \OMEGA|-|style direction abbreviations: the first character indicates
the \quote {first} edge of the character glyphs (the edge that is seen first in
the writing direction), the second the \quote {top} side. Keep in mind that
\LUATEX\ has a bit different directional model so these values are not used for
anything.

The \type {parameters} is a hash with mixed key types. There are seven possible
string keys, as well as a number of integer indices (these start from 8 up). The
seven strings are actually used instead of the bottom seven indices, because that
gives a nicer user interface.

The names and their internal remapping are:

\starttabulate[|l|c|]
\BC name                  \BC remapping \NC \NR
\NC \type {slant}         \NC 1 \NC \NR
\NC \type {space}         \NC 2 \NC \NR
\NC \type {space_stretch} \NC 3 \NC \NR
\NC \type {space_shrink}  \NC 4 \NC \NR
\NC \type {x_height}      \NC 5 \NC \NR
\NC \type {quad}          \NC 6 \NC \NR
\NC \type {extra_space}   \NC 7 \NC \NR
\stoptabulate

The keys \type {type}, \type {format}, \type {embedding}, \type {fullname} and
\type {filename} are used to embed \OPENTYPE\ fonts in the result \PDF.

The \type {characters} table is a list of character hashes indexed by an integer
number. The number is the \quote {internal code} \TEX\ knows this character by.

Two very special string indexes can be used also: \type {left_boundary} is a
virtual character whose ligatures and kerns are used to handle word boundary
processing. \type {right_boundary} is similar but not actually used for anything
(yet).

Other index keys are ignored.

Each character hash itself is a hash. For example, here is the character \quote
{f} (decimal 102) in the font \type {cmr10 at 10pt}:

\starttyping
[102] = {
    ['width'] = 200250,
    ['height'] = 455111,
    ['depth'] = 0,
    ['italic'] = 50973,
    ['kerns'] = {
        [63] = 50973,
        [93] = 50973,
        [39] = 50973,
        [33] = 50973,
        [41] = 50973
    },
    ['ligatures'] = {
        [102] = {
            ['char'] = 11,
            ['type'] = 0
        },
        [108] = {
            ['char'] = 13,
            ['type'] = 0
        },
        [105] = {
            ['char'] = 12,
            ['type'] = 0
        }
    }
}
\stoptyping

The following top|-|level keys can be present inside a character hash:

\starttabulate[|l|c|c|c|l|p|]
\BC key              \BC vf  \BC tfm \BC used  \BC type    \BC description \NC\NR
\NC \type{width}            \NC yes \NC yes \NC yes   \NC number  \NC character's width, in sp (default 0) \NC\NR
\NC \type{height}           \NC no  \NC yes \NC yes   \NC number  \NC character's height, in sp (default 0) \NC\NR
\NC \type{depth}            \NC no  \NC yes \NC yes   \NC number  \NC character's depth, in sp (default 0) \NC\NR
\NC \type{italic}           \NC no  \NC yes \NC yes   \NC number  \NC character's italic correction, in sp (default zero) \NC\NR
\NC \type{top_accent}       \NC no  \NC no  \NC maybe \NC number  \NC character's top accent alignment place, in sp (default zero) \NC\NR
\NC \type{bot_accent}       \NC no  \NC no  \NC maybe \NC number  \NC character's bottom accent alignment place, in sp (default zero) \NC\NR
\NC \type{left_protruding}  \NC no  \NC no  \NC maybe \NC number  \NC character's \type {\lpcode} \NC\NR
\NC \type{right_protruding} \NC no  \NC no  \NC maybe \NC number  \NC character's \type {\rpcode} \NC\NR
\NC \type{expansion_factor} \NC no  \NC no  \NC maybe \NC number  \NC character's \type {\efcode} \NC\NR
\NC \type{tounicode}        \NC no  \NC no  \NC maybe \NC string  \NC character's \UNICODE\ equivalent(s), in \UTF|-|16BE hexadecimal format \NC\NR
\NC \type{next}             \NC no  \NC yes \NC yes   \NC number  \NC the \quote {next larger} character index \NC\NR
\NC \type{extensible}       \NC no  \NC yes \NC yes   \NC table   \NC the constituent parts of an extensible recipe \NC\NR
\NC \type{vert_variants}    \NC no  \NC no  \NC yes   \NC table   \NC constituent parts of a vertical variant set \NC \NR
\NC \type{horiz_variants}   \NC no  \NC no  \NC yes   \NC table   \NC constituent parts of a horizontal variant set \NC \NR
\NC \type{kerns}            \NC no  \NC yes \NC yes   \NC table   \NC kerning information \NC\NR
\NC \type{ligatures}        \NC no  \NC yes \NC yes   \NC table   \NC ligaturing information \NC\NR
\NC \type{commands}         \NC yes \NC no  \NC yes   \NC array   \NC virtual font commands \NC\NR
\NC \type{name}             \NC no  \NC no  \NC no    \NC string  \NC the character (\POSTSCRIPT) name \NC\NR
\NC \type{index}            \NC no  \NC no  \NC yes   \NC number  \NC the (\OPENTYPE\ or \TRUETYPE) font glyph index \NC\NR
\NC \type{used}             \NC no  \NC yes \NC yes   \NC boolean \NC typeset already (default: false)? \NC\NR
\NC \type{mathkern}         \NC no  \NC no  \NC yes   \NC table   \NC math cut-in specifications \NC\NR
\stoptabulate

The values of \type {top_accent}, \type {bot_accent} and \type {mathkern} are
used only for math accent and superscript placement, see the \at {math chapter}
[math] in this manual for details.

The values of \type {left_protruding} and \type {right_protruding} are used only
when \type {\protrudechars} is non-zero.

Whether or not \type {expansion_factor} is used depends on the font's global
expansion settings, as well as on the value of \type {\adjustspacing}.

The usage of \type {tounicode} is this: if this font specifies a \type
{tounicode=1} at the top level, then \LUATEX\ will construct a \type {/ToUnicode}
entry for the \PDF\ font (or font subset) based on the character|-|level \type
{tounicode} strings, where they are available. If a character does not have a
sensible \UNICODE\ equivalent, do not provide a string either (no empty strings).

If the font level \type {tounicode} is not set, then \LUATEX\ will build up \type
{/ToUnicode} based on the \TEX\ code points you used, and any character-level
\type {tounicodes} will be ignored. The string format is exactly the format that
is expected by Adobe \CMAP\ files (\UTF-16BE in hexadecimal encoding), minus the
enclosing angle brackets. For instance the \type {tounicode} for a \type {fi}
ligature would be \type {00660069}. When you pass a number the conversion will be
done for you.

The presence of \type {extensible} will overrule \type {next}, if that is also
present. It in in turn can be overruled by \type {vert_variants}.

The \type {extensible} table is very simple:

\starttabulate[|l|l|p|]
\BC key        \BC type   \BC description                \NC\NR
\NC \type{top} \NC number \NC top character index        \NC\NR
\NC \type{mid} \NC number \NC middle character index     \NC\NR
\NC \type{bot} \NC number \NC bottom character index     \NC\NR
\NC \type{rep} \NC number \NC repeatable character index \NC\NR
\stoptabulate

The \type {horiz_variants} and \type {vert_variants} are arrays of components.
Each of those components is itself a hash of up to five keys:

\starttabulate[|l|l|p|]
\BC key             \BC type   \BC explanation \NC \NR
\NC \type{glyph}    \NC number \NC The character index. Note that this is an encoding number, not a name. \NC \NR
\NC \type{extender} \NC number \NC One (1) if this part is repeatable, zero (0) otherwise. \NC \NR
\NC \type{start}    \NC number \NC The maximum overlap at the starting side (in scaled points). \NC \NR
\NC \type{end}      \NC number \NC The maximum overlap at the ending side (in scaled points). \NC \NR
\NC \type{advance}  \NC number \NC The total advance width of this item. It can be zero or missing,
                                   then the natural size of the glyph for character \type {component}
                                   is used. \NC \NR
\stoptabulate

The \type {kerns} table is a hash indexed by character index (and \quote
{character index} is defined as either a non|-|negative integer or the string
value \type {right_boundary}), with the values the kerning to be applied, in
scaled points.

The \type {ligatures} table is a hash indexed by character index (and \quote
{character index} is defined as either a non|-|negative integer or the string
value \type {right_boundary}), with the values being yet another small hash, with
two fields:

\starttabulate[|l|l|p|]
\BC key         \BC type   \BC description \NC \NR
\NC \type{type} \NC number \NC the type of this ligature command, default 0 \NC \NR
\NC \type{char} \NC number \NC the character index of the resultant ligature \NC \NR
\stoptabulate

The \type {char} field in a ligature is required.

The \type {type} field inside a ligature is the numerical or string value of one
of the eight possible ligature types supported by \TEX. When \TEX\ inserts a new
ligature, it puts the new glyph in the middle of the left and right glyphs. The
original left and right glyphs can optionally be retained, and when at least one
of them is kept, it is also possible to move the new \quote {insertion point}
forward one or two places. The glyph that ends up to the right of the insertion
point will become the next \quote {left}.

\starttabulate[|l|c|l|l|]
\BC textual (Knuth)       \BC number \BC string        \BC result      \NC\NR
\NC \type{l + r =: n}     \NC 0      \NC \type{=:}     \NC \type{|n}   \NC\NR
\NC \type{l + r =:| n}    \NC 1      \NC \type{=:|}    \NC \type{|nr}  \NC\NR
\NC \type{l + r |=: n}    \NC 2      \NC \type{|=:}    \NC \type{|ln}  \NC\NR
\NC \type{l + r |=:| n}   \NC 3      \NC \type{|=:|}   \NC \type{|lnr} \NC\NR
\NC \type{l + r  =:|> n}  \NC 5      \NC \type{=:|>}   \NC \type{n|r}  \NC\NR
\NC \type{l + r |=:> n}   \NC 6      \NC \type{|=:>}   \NC \type{l|n}  \NC\NR
\NC \type{l + r |=:|> n}  \NC 7      \NC \type{|=:|>}  \NC \type{l|nr} \NC\NR
\NC \type{l + r |=:|>> n} \NC 11     \NC \type{|=:|>>} \NC \type{ln|r} \NC\NR
\stoptabulate

The default value is~0, and can be left out. That signifies a \quote {normal}
ligature where the ligature replaces both original glyphs. In this table the~\type {|}
indicates the final insertion point.

The \type {commands} array is explained below.

\section {Real fonts}

Whether or not a \TEX\ font is a \quote {real} font that should be written to the
\PDF\ document is decided by the \type {type} value in the top|-|level font
structure. If the value is \type {real}, then this is a proper font, and the
inclusion mechanism will attempt to add the needed font object definitions to the
\PDF. Values for \type {type} are:

\starttabulate[|l|p|]
\BC value          \BC description            \NC\NR
\NC \type{real}    \NC this is a base font    \NC\NR
\NC \type{virtual} \NC this is a virtual font \NC\NR
\stoptabulate

The actions to be taken depend on a number of different variables:

\startitemize[packed]
\startitem
    Whether the used font fits in an 8-bit encoding scheme or not.
\stopitem
\startitem
    The type of the disk font file.
\stopitem
\startitem
    The level of embedding requested.
\stopitem
\stopitemize

A font that uses anything other than an 8-bit encoding vector has to be written
to the \PDF\ in a different way.

The rule is: if the font table has \type {encodingbytes} set to~2, then this is a
wide font, in all other cases it isn't. The value~2 is the default for \OPENTYPE\
and \TRUETYPE\ fonts loaded via \LUA. For \TYPEONE\ fonts, you have to set \type
{encodingbytes} to~2 explicitly. For \PK\ bitmap fonts, wide font encoding is not
supported at all.

If no special care is needed, \LUATEX\ currently falls back to the
mapfile|-|based solution used by \PDFTEX\ and \DVIPS. This behaviour might
silently be removed in the future, in which case the related primitives and \LUA\
functions will become no|-|ops.

If a \quote {wide} font is used, the new subsystem kicks in, and some
extra fields have to be present in the font structure. In this case, \LUATEX\
does not use a map file at all.

The extra fields are: \type {format}, \type {embedding}, \type {fullname}, \type
{cidinfo} (as explained above), \type {filename}, and the \type {index} key in
the separate characters.

Values for \type {format} are:

\starttabulate[|l|p|]
\BC value           \BC description                                               \NC \NR
\NC \type{type1}    \NC this is a \POSTSCRIPT\ \TYPEONE\ font                     \NC \NR
\NC \type{type3}    \NC this is a bitmapped (\PK) font                            \NC \NR
\NC \type{truetype} \NC this is a \TRUETYPE\ or \TRUETYPE|-|based \OPENTYPE\ font \NC \NR
\NC \type{opentype} \NC this is a \POSTSCRIPT|-|based \OPENTYPE\ font             \NC \NR
\stoptabulate

\type {type3} fonts are provided for backward compatibility only, and do not
support the new wide encoding options.

Values for \type {embedding} are:

\starttabulate[|l|p|]
\BC value         \BC description                             \NC \NR
\NC \type{no}     \NC don't embed the font at all             \NC \NR
\NC \type{subset} \NC include and atttempt to subset the font \NC \NR
\NC \type{full}   \NC include this font in its entirety       \NC \NR
\stoptabulate

The other fields are used as follows: The \type {fullname} will be the
\POSTSCRIPT|/|\PDF\ font name. The \type {cidinfo} will be used as the character
set (the CID \type {/Ordering} and \type {/Registry} keys). The \type {filename}
points to the actual font file. If you include the full path in the \type
{filename} or if the file is in the local directory, \LUATEX\ will run a little
bit more efficient because it will not have to re|-|run the \type {find_xxx_file}
callback in that case.

Be careful: when mixing old and new fonts in one document, it is possible to
create \POSTSCRIPT\ name clashes that can result in printing errors. When this
happens, you have to change the \type {fullname} of the font.

Typeset strings are written out in a wide format using 2~bytes per glyph, using
the \type {index} key in the character information as value. The overall effect
is like having an encoding based on numbers instead of traditional (\POSTSCRIPT)
name|-|based reencoding. The way to get the correct \type {index} numbers for
\TYPEONE\ fonts is by loading the font via \type {fontloader.open} and use the table
indices as \type {index} fields.

In order to make sure that cut and paste of the final document works okay you can
best make sure that there is a \type {tounicode} vector enforced.

\section[virtualfonts]{Virtual fonts}

\subsection{The structure}

You have to take the following steps if you want \LUATEX\ to treat the returned
table from \type {define_font} as a virtual font:

\startitemize[packed]
\startitem
    Set the top|-|level key \type {type} to \type {virtual}.
\stopitem
\startitem
    Make sure there is at least one valid entry in \type {fonts} (see below).
\stopitem
\startitem
    Give a \type {commands} array to every character (see below).
\stopitem
\stopitemize

The presence of the toplevel \type {type} key with the specific value \type
{virtual} will trigger handling of the rest of the special virtual font fields in
the table, but the mere existence of 'type' is enough to prevent \LUATEX\ from
looking for a virtual font on its own.

Therefore, this also works \quote {in reverse}: if you are absolutely certain
that a font is not a virtual font, assigning the value \type {base} or \type
{real} to \type {type} will inhibit \LUATEX\ from looking for a virtual font
file, thereby saving you a disk search.

The \type {fonts} is another \LUA\ array. The values are one- or two|-|key
hashes themselves, each entry indicating one of the base fonts in a virtual font.
In case your font is referring to itself, you can use the \type {font.nextid()}
function which returns the index of the next to be defined font which is probably
the currently defined one.

An example makes this easy to understand

\starttyping
fonts = {
    { name = 'ptmr8a', size = 655360 },
    { name = 'psyr', size = 600000 },
    { id = 38 }
}
\stoptyping

says that the first referenced font (index 1) in this virtual font is \type
{ptrmr8a} loaded at 10pt, and the second is \type {psyr} loaded at a little over
9pt. The third one is previously defined font that is known to \LUATEX\ as font id
\quote {38}.

The array index numbers are used by the character command definitions that are
part of each character.

The \type {commands} array is a hash where each item is another small array,
with the first entry representing a command and the extra items being the
parameters to that command. The allowed commands and their arguments are:

\starttabulate[|l|l|l|p|]
\BC command name   \BC arguments \BC type      \BC description \NC \NR
\NC \type{font}    \NC 1         \NC number    \NC select a new font from the local \type {fonts} table \NC \NR
\NC \type{char}    \NC 1         \NC number    \NC typeset this character number from the current font,
                                                   and move right by the character's width \NC \NR
\NC \type{node}    \NC 1         \NC node      \NC output this node (list), and move right
                                                   by the width of this list\NC \NR
\NC \type{slot}    \NC 2         \NC 2 numbers \NC a shortcut for the combination of a font and char command\NC \NR
\NC \type{push}    \NC 0         \NC           \NC save current position\NC \NR
\NC \type{nop}     \NC 0         \NC           \NC do nothing \NC \NR
\NC \type{pop}     \NC 0         \NC           \NC pop position \NC \NR
\NC \type{rule}    \NC 2         \NC 2 numbers \NC output a rule $ht*wd$, and move right. \NC \NR
\NC \type{down}    \NC 1         \NC number    \NC move down on the page \NC \NR
\NC \type{right}   \NC 1         \NC number    \NC move right on the page \NC \NR
\NC \type{special} \NC 1         \NC string    \NC output a \type {\special} command \NC \NR
\NC \type{pdf}     \NC 2         \NC 2 strings \NC output a \PDF\ literal, the first string is one of \type {origin},
                                                   \type {page}, \type {text}, \type {font}, \type {direct} or \type {raw}; if you
                                                   have one string only \type {origin} is assumed \NC \NR
\NC \type{lua}     \NC 1         \NC string    \NC execute a \LUA\ script (at \type {\latelua} time) \NC \NR
\NC \type{image}   \NC 1         \NC image     \NC output an image (the argument can be either an \type
                                                   {<image>} variable or an \type {image_spec} table) \NC \NR
\NC \type{comment} \NC any       \NC any       \NC the arguments of this command are ignored \NC \NR
\stoptabulate

When a font id is set to~0 then it will be replaced by the currently assigned
font id. This prevents the need for hackery with future id's (normally one could
use \type {font.nextid} but when more complex fonts are built in the meantime
other instances could have been loaded.

The \type {pdf} option also accepts a \type {mode} keyword in which case the
third argument sets the mode. That option will change the mode in an efficient
way (passing an empty string would result in an extra empty lines in the \PDF\
file. This option only makes sense for virtual fonts. The \type {font} mode only
makes sense in virtual fonts.

These modes are somewhat fuzzy and partially inherited from \PDFTEX.

\starttabulate[|l|p|]
\BC mode           \BC description \NC \NR
\NC \type {origin} \NC enter page mode and set the position \NC \NR
\NC \type {page}   \NC enter page mode \NC \NR
\NC \type {text}   \NC enter text mode \NC \NR
\NC \type {font}   \NC enter font mode (kind of text mode, only in virtual fonts) \NC \NR
\NC \type {always} \NC finish the current string and force a transform if needed \NC \NR
\NC \type {raw}    \NC finish the current string \NC \NR
\stoptabulate

You always need to check what \PDF\ code is generated because there can be all kind of
interferences with optimizations in the backend and fonts are complicated anyway.

Here is a rather elaborate glyph commands example:

\starttyping
...
commands = {
    { "push" },                     -- remember where we are
    { "right", 5000 },              -- move right about 0.08pt
    { "font", 3 },                  -- select the fonts[3] entry
    { "char", 97 },                 -- place character 97 (ASCII 'a')
    { "pop" },                      -- go all the way back
    { "down", -200000 },            -- move upwards by about 3pt
    { "special", "pdf: 1 0 0 rg" }  -- switch to red color
 -- { "pdf", "origin", "1 0 0 rg" } -- switch to red color (alternative)
    { "rule", 500000, 20000 }       -- draw a bar
    { "special", "pdf: 0 g" }       -- back to black
 -- { "pdf", "origin", "0 g" }      -- back to black (alternative)
}
...
\stoptyping

The default value for \type {font} is always~1 at the start of the
\type {commands} array. Therefore, if the virtual font is essentially only a
re|-|encoding, then you do usually not have create an explicit \quote {font}
command in the array.

Rules inside of \type {commands} arrays are built up using only two dimensions:
they do not have depth. For correct vertical placement, an extra \type {down}
command may be needed.

Regardless of the amount of movement you create within the \type {commands}, the
output pointer will always move by exactly the width that was given in the \type
{width} key of the character hash. Any movements that take place inside the \type
{commands} array are ignored on the upper level.

The special can have a \type {pdf:}, \type {pdf:origin:},  \type {pdf:page:},
\type {pdf:direct:} or  \type {pdf:raw:} prefix. When you have to concatenate
strings using the \type {pdf} command might be more efficient.

\subsection{Artificial fonts}

Even in a \quote {real} font, there can be virtual characters. When \LUATEX\
encounters a \type {commands} field inside a character when it becomes time to
typeset the character, it will interpret the commands, just like for a true
virtual character. In this case, if you have created no \quote {fonts} array,
then the default (and only) \quote {base} font is taken to be the current font
itself. In practice, this means that you can create virtual duplicates of
existing characters which is useful if you want to create composite characters.

Note: this feature does {\it not\/} work the other way around. There can not be
\quote {real} characters in a virtual font! You cannot use this technique for
font re-encoding either; you need a truly virtual font for that (because
characters that are already present cannot be altered).

\subsection{Example virtual font}

Finally, here is a plain \TEX\ input file with a virtual font demonstration:

\startbuffer
\directlua {
  callback.register('define_font',
    function (name,size)
      if name == 'cmr10-red' then
        f = font.read_tfm('cmr10',size)
        f.name = 'cmr10-red'
        f.type = 'virtual'
        f.fonts = {{ name = 'cmr10', size = size }}
        for i,v in pairs(f.characters) do
          if (string.char(i)):find('[tacohanshartmut]') then
             v.commands = {
               {'special','pdf: 1 0 0 rg'},
               {'char',i},
               {'special','pdf: 0 g'},
              }
          else
             v.commands = {{'char',i}}
          end
        end
      else
        f = font.read_tfm(name,size)
      end
      return f
    end
  )
}

\font\myfont = cmr10-red at 10pt \myfont  This is a line of text \par
\font\myfontx= cmr10     at 10pt \myfontx Here is another line of text \par
\stopbuffer

\typebuffer

\section{The \type {vf} library}

The \type {vf} library can be used when \LUA\ code, as defined in the \type
{commands} of the font, is executed. The functions provided are similar as the
commands: \type {char}, \type {down}, \type {fontid}, \type {image}, \type
{node}, \type {nop}, \type {pop}, \type {push}, \type {right}, \type {rule},
\type {special} and \type {pdf}. This library has been present for a while but
not been advertised and tested much, if only because it's easy to define an
invalid font (or mess up the \PDF\ stream). Keep in mind that the \LUA\ snippets
are executed each time when a character is output.


\section{The \type {font} library}

The font library provides the interface into the internals of the font system,
and also it contains helper functions to load traditional \TEX\ font metrics
formats. Other font loading functionality is provided by the \type {fontloader}
library that will be discussed in the next section.

\subsection{Loading a \TFM\ file}

The behavior documented in this subsection is considered stable in the sense that
there will not be backward-incompatible changes any more.

\startfunctioncall
<table> fnt =
    font.read_tfm(<string> name, <number> s)
\stopfunctioncall

The number is a bit special:

\startitemize
\startitem
    If it is positive, it specifies an \quote {at size} in scaled points.
\stopitem
\startitem
    If it is negative, its absolute value represents a \quote {scaled}
    setting relative to the designsize of the font.
\stopitem
\stopitemize

The internal structure of the metrics font table that is returned is explained in
\in {chapter} [fonts].

\subsection{Loading a \VF\ file}

The behavior documented in this subsection is considered stable in the sense that
there will not be backward-incompatible changes any more.

\startfunctioncall
<table> vf_fnt =
    font.read_vf(<string> name, <number> s)
\stopfunctioncall

The meaning of the number \type {s} and the format of the returned table are
similar to the ones in the \type {read_tfm()} function.

\subsection{The fonts array}

The whole table of \TEX\ fonts is accessible from \LUA\ using a virtual array.

\starttyping
font.fonts[n] = { ... }
<table> f = font.fonts[n]
\stoptyping

See \in {chapter} [fonts] for the structure of the tables. Because this is a
virtual array, you cannot call \type {pairs} on it, but see below for the \type
{font.each} iterator.

The two metatable functions implementing the virtual array are:

\startfunctioncall
<table> f = font.getfont(<number> n)
font.setfont(<number> n, <table> f)
\stopfunctioncall

Note that at the moment, each access to the \type {font.fonts} or call to \type
{font.getfont} creates a \LUA\ table for the whole font. This process can be quite
slow. In a later version of \LUATEX, this interface will change (it will start
using userdata objects instead of actual tables).

Also note the following: assignments can only be made to fonts that have already
been defined in \TEX, but have not been accessed {\it at all\/} since that
definition. This limits the usability of the write access to \type {font.fonts}
quite a lot, a less stringent ruleset will likely be implemented later.

\subsection{Checking a font's status}

You can test for the status of a font by calling this function:

\startfunctioncall
<boolean> f =
    font.frozen(<number> n)
\stopfunctioncall

The return value is one of \type {true} (unassignable), \type {false} (can be
changed) or \type {nil} (not a valid font at all).

\subsection{Defining a font directly}

You can define your own font into \type {font.fonts} by calling this function:

\startfunctioncall
<number> i =
    font.define(<table> f)
\stopfunctioncall

The return value is the internal id number of the defined font (the index into
\type {font.fonts}). If the font creation fails, an error is raised. The table
is a font structure, as explained in \in {chapter} [fonts]. An alternative call
is:

\startfunctioncall
<number> i =
    font.define(<number> n, <table> f)
\stopfunctioncall

Where the first argument is a reserved font id (see below).

\subsection{Extending a font}

Within reasonable bounds you can extend a font after it has been defined. Because
some properties are best left unchanged this is limited to adding characters.

\startfunctioncall
font.addcharacters(<number n>, <table> f)
\stopfunctioncall

The table passed can have the fields \type {characters} which is a (sub)table
like the one used in define, and for virtual fonts a \type {fonts} table can be
added. The characters defined in the \type {characters} table are added (when not
yet present) or replace an existing entry. Keep in mind that replacing can have
side effects because a character already can have been used. Instead of posing
restrictions we expect the user to be careful. (The \type {setfont} helper is
a more drastic replacer.)

\subsection{Projected next font id}

\startfunctioncall
<number> i =
    font.nextid()
\stopfunctioncall

This returns the font id number that would be returned by a \type {font.define}
call if it was executed at this spot in the code flow. This is useful for virtual
fonts that need to reference themselves. If you pass \type {true} as argument,
the id gets reserved and you can pass to \type {font.define} as first argument.
This can be handy when you create complex virtual fonts.

\startfunctioncall
<number> i =
    font.nextid(true)
\stopfunctioncall

\subsection{Font id}

\startfunctioncall
<number> i =
    font.id(<string> csname)
\stopfunctioncall

This returns the font id associated with \type {csname} string, or $-1$ if \type
{csname} is not defined.

\subsection{Currently active font}

\startfunctioncall
<number> i = font.current()
font.current(<number> i)
\stopfunctioncall

This gets or sets the currently used font number.

\subsection{Maximum font id}

\startfunctioncall
<number> i =
    font.max()
\stopfunctioncall

This is the largest used index in \type {font.fonts}.

\subsection{Iterating over all fonts}

\startfunctioncall
for i,v in font.each() do
  ...
end
\stopfunctioncall

This is an iterator over each of the defined \TEX\ fonts. The first returned
value is the index in \type {font.fonts}, the second the font itself, as a \LUA\
table. The indices are listed incrementally, but they do not always form an array
of consecutive numbers: in some cases there can be holes in the sequence.

\stopchapter

\stopcomponent
%
    % language=uk

\environment luatex-style
\environment luatex-logos

\startcomponent luatex-math

\startchapter[reference=math,title={Math}]

The handling of mathematics in \LUATEX\ differs quite a bit from how \TEX82 (and
therefore \PDFTEX) handles math. First, \LUATEX\ adds primitives and extends some
others so that \UNICODE\ input can be used easily. Second, all of \TEX82's
internal special values (for example for operator spacing) have been made
accessible and changeable via control sequences. Third, there are extensions that
make it easier to use \OPENTYPE\ math fonts. And finally, there are some
extensions that have been proposed or considered in the past that are now added
to the engine.

\section{The current math style}

It is possible to discover the math style that will be used for a formula in an
expandable fashion (while the math list is still being read). To make this
possible, \LUATEX\ adds the new primitive: \type {\mathstyle}. This is a \quote
{convert command} like e.g. \type {\romannumeral}: its value can only be read,
not set.

\subsection{\type {\mathstyle}}

The returned value is between 0 and 7 (in math mode), or $-1$ (all other modes).
For easy testing, the eight math style commands have been altered so that the can
be used as numeric values, so you can write code like this:

\starttyping
\ifnum\mathstyle=\textstyle
    \message{normal text style}
\else \ifnum\mathstyle=\crampedtextstyle
    \message{cramped text style}
\fi \fi
\stoptyping

\subsection{\type {\Ustack}}

There are a few math commands in \TEX\ where the style that will be used is not
known straight from the start. These commands (\type {\over}, \type {\atop},
\type {\overwithdelims}, \type {\atopwithdelims}) would therefore normally return
wrong values for \type {\mathstyle}. To fix this, \LUATEX\ introduces a special
prefix command: \type {\Ustack}:

\starttyping
$\Ustack {a \over b}$
\stoptyping

The \type {\Ustack} command will scan the next brace and start a new math group
with the correct (numerator) math style.

\section{Unicode math characters}

Character handling is now extended up to the full \UNICODE\ range (the \type {\U}
prefix), which is compatible with \XETEX.

The math primitives from \TEX\ are kept as they are, except for the ones that
convert from input to math commands: \type {mathcode}, and \type {delcode}. These
two now allow for a 21-bit character argument on the left hand side of the equals
sign.

Some of the new \LUATEX\ primitives read more than one separate value. This is
shown in the tables below by a plus sign in the second column.

The input for such primitives would look like this:

\starttyping
\def\overbrace{\Umathaccent 0 1 "23DE }
\stoptyping

The altered \TEX82 primitives are:

\starttabulate[|l|l|r|c|l|r|]
\NC \bf primitive     \NC \bf min \NC \bf max \NC \kern 2em  \NC \bf min \NC \bf max \NC \NR
\NC \type {\mathcode} \NC 0       \NC 10FFFF  \NC =          \NC 0       \NC 8000    \NC \NR
\NC \type {\delcode}  \NC 0       \NC 10FFFF  \NC =          \NC 0       \NC FFFFFF  \NC \NR
\stoptabulate

The unaltered ones are:

\starttabulate[|l|l|r|]
\NC \bf primitive        \NC \bf min \NC \bf max \NC \NR
\NC \type {\mathchardef} \NC 0       \NC    8000 \NC \NR
\NC \type {\mathchar}    \NC 0       \NC    7FFF \NC \NR
\NC \type {\mathaccent}  \NC 0       \NC    7FFF \NC \NR
\NC \type {\delimiter}   \NC 0       \NC 7FFFFFF \NC \NR
\NC \type {\radical}     \NC 0       \NC 7FFFFFF \NC \NR
\stoptabulate

For practical reasons \type {\mathchardef} will silently accept values larger
that \type {0x8000} and interpret it as \type {\Umathcharnumdef}. This is needed
to satisfy older macro packages.

The following new primitives are compatible with \XETEX:

% somewhat fuzzy:

\starttabulate[|l|l|r|c|l|r|]
\NC \bf primitive                           \NC \bf min   \NC \bf max             \NC \kern 2em  \NC \bf min   \NC \bf max                    \NC \NR
\NC \type {\Umathchardef}                   \NC 0+0+0     \NC 7+FF+10FFFF\rlap{\high{1}}   \NC   \NC           \NC                            \NC \NR
\NC \type {\Umathcharnumdef}\rlap{\high{5}} \NC -80000000 \NC    7FFFFFFF\rlap{\high{3}}   \NC   \NC           \NC                            \NC \NR
\NC \type {\Umathcode}                      \NC 0         \NC      10FFFF                  \NC = \NC 0+0+0     \NC 7+FF+10FFFF\rlap{\high{1}} \NC \NR
\NC \type {\Udelcode}                       \NC 0         \NC      10FFFF                  \NC = \NC 0+0       \NC   FF+10FFFF\rlap{\high{2}} \NC \NR
\NC \type {\Umathchar}                      \NC 0+0+0     \NC 7+FF+10FFFF                  \NC   \NC           \NC                            \NC \NR
\NC \type {\Umathaccent}                    \NC 0+0+0     \NC 7+FF+10FFFF\rlap{\high{2,4}} \NC   \NC           \NC                            \NC \NR
\NC \type {\Udelimiter}                     \NC 0+0+0     \NC 7+FF+10FFFF\rlap{\high{2}}   \NC   \NC           \NC                            \NC \NR
\NC \type {\Uradical}                       \NC 0+0       \NC   FF+10FFFF\rlap{\high{2}}   \NC   \NC           \NC                            \NC \NR
\NC \type {\Umathcharnum}                   \NC -80000000 \NC    7FFFFFFF\rlap{\high{3}}   \NC   \NC           \NC                            \NC \NR
\NC \type {\Umathcodenum}                   \NC 0         \NC      10FFFF                  \NC = \NC -80000000 \NC    7FFFFFFF\rlap{\high{3}} \NC \NR
\NC \type {\Udelcodenum}                    \NC 0         \NC      10FFFF                  \NC = \NC -80000000 \NC    7FFFFFFF\rlap{\high{3}} \NC \NR
\stoptabulate

Specifications typically look like:

\starttyping
\Umathchardef\xx="1"0"456
\Umathcode   123="1"0"789
\stoptyping

Note 1: The new primitives that deal with delimiter|-|style objects do not set up a
\quote {large family}. Selecting a suitable size for display purposes is expected
to be dealt with by the font via the \type {\Umathoperatorsize} parameter (more
information can be found in a following section).

Note 2: For these three primitives, all information is packed into a single
signed integer. For the first two (\type {\Umathcharnum} and \type
{\Umathcodenum}), the lowest 21 bits are the character code, the 3 bits above
that represent the math class, and the family data is kept in the topmost bits
(This means that the values for math families 128--255 are actually negative).
For \type {\Udelcodenum} there is no math class. The math family information is
stored in the bits directly on top of the character code. Using these three
commands is not as natural as using the two- and three|-|value commands, so
unless you know exactly what you are doing and absolutely require the speedup
resulting from the faster input scanning, it is better to use the verbose
commands instead.

Note 3: The \type {\Umathaccent} command accepts optional keywords to control
various details regarding math accents. See \in {section} [mathacc] below for
details.

New primitives that exist in \LUATEX\ only (all of these will be explained
in following sections):

\starttabulate[|l|l|l|l|]
\NC \bf primitive         \NC \bf value range (in hex) \NC \NR
\NC \type {\Uroot}           \NC 0+0--FF+10FFFF$^2$       \NC \NR
\NC \type {\Uoverdelimiter}  \NC 0+0--FF+10FFFF$^2$       \NC \NR
\NC \type {\Uunderdelimiter} \NC 0+0--FF+10FFFF$^2$       \NC \NR
\NC \type {\Udelimiterover}  \NC 0+0--FF+10FFFF$^2$       \NC \NR
\NC \type {\Udelimiterunder} \NC 0+0--FF+10FFFF$^2$       \NC \NR
\stoptabulate

\section{Cramped math styles}

\LUATEX\ has four new primitives to set the cramped math styles directly:

\starttyping
\crampeddisplaystyle
\crampedtextstyle
\crampedscriptstyle
\crampedscriptscriptstyle
\stoptyping

These additional commands are not all that valuable on their own, but they come
in handy as arguments to the math parameter settings that will be added shortly.

In Eijkhouts \quotation {\TEX\ by Topic} the rules for handling styles in scripts
are described as follows:

\startitemize
\startitem
    In any style superscripts and subscripts are taken from the next smaller style.
    Exception: in display style they are taken in script style.
\stopitem
\startitem
    Subscripts are always in the cramped variant of the style; superscripts are only
    cramped if the original style was cramped.
\stopitem
\startitem
    In an \type {..\over..} formula in any style the numerator and denominator are
    taken from the next smaller style.
\stopitem
\startitem
    The denominator is always in cramped style; the numerator is only in cramped
    style if the original style was cramped.
\stopitem
\startitem
    Formulas under a \type {\sqrt} or \type {\overline} are in cramped style.
\stopitem
\stopitemize

In \LUATEX\ one can set the styles in more detail which means that you sometimes
have to set both normal and cramped styles to get the effect you want. If we
force styles in the script using \type {\scriptstyle} and \type {\crampedscriptstyle}
we get this:

\startbuffer[demo]
\starttabulate
\NC default       \NC $b_{x=xx}^{x=xx}$ \NC \NR
\NC script        \NC $b_{\scriptstyle x=xx}^{\scriptstyle x=xx}$ \NC \NR
\NC crampedscript \NC $b_{\crampedscriptstyle x=xx}^{\crampedscriptstyle x=xx}$ \NC \NR
\stoptabulate
\stopbuffer

\getbuffer[demo]

Now we set the following parameters

\startbuffer[setup]
\Umathordrelspacing\scriptstyle=30mu
\Umathordordspacing\scriptstyle=30mu
\stopbuffer

\typebuffer[setup]

This gives:

\start\getbuffer[setup,demo]\stop

But, as this is not what is expected (visually) we should say:

\startbuffer[setup]
\Umathordrelspacing\scriptstyle=30mu
\Umathordordspacing\scriptstyle=30mu
\Umathordrelspacing\crampedscriptstyle=30mu
\Umathordordspacing\crampedscriptstyle=30mu
\stopbuffer

\typebuffer[setup]

Now we get:

\start\getbuffer[setup,demo]\stop

\section{Math parameter settings}

In \LUATEX, the font dimension parameters that \TEX\ used in math typesetting are
now accessible via primitive commands. In fact, refactoring of the math engine
has resulted in many more parameters than were accessible before.

\starttabulate
\NC \bf primitive name               \NC \bf description \NC \NR
\NC \type {\Umathquad}               \NC the width of 18 mu's \NC \NR
\NC \type {\Umathaxis}               \NC height of the vertical center axis of
                                         the math formula above the baseline \NC \NR
\NC \type {\Umathoperatorsize}       \NC minimum size of large operators in display mode \NC \NR
\NC \type {\Umathoverbarkern}        \NC vertical clearance above the rule \NC \NR
\NC \type {\Umathoverbarrule}        \NC the width of the rule \NC \NR
\NC \type {\Umathoverbarvgap}        \NC vertical clearance below the rule \NC \NR
\NC \type {\Umathunderbarkern}       \NC vertical clearance below the rule \NC \NR
\NC \type {\Umathunderbarrule}       \NC the width of the rule \NC \NR
\NC \type {\Umathunderbarvgap}       \NC vertical clearance above the rule \NC \NR
\NC \type {\Umathradicalkern}        \NC vertical clearance above the rule \NC \NR
\NC \type {\Umathradicalrule}        \NC the width of the rule \NC \NR
\NC \type {\Umathradicalvgap}        \NC vertical clearance below the rule \NC \NR
\NC \type {\Umathradicaldegreebefore}\NC the forward kern that takes place before placement of
                                         the radical degree \NC \NR
\NC \type {\Umathradicaldegreeafter} \NC the backward kern that takes place after placement of
                                         the radical degree \NC \NR
\NC \type {\Umathradicaldegreeraise} \NC this is the percentage of the total height and depth of
                                         the radical sign that the degree is raised by; it is
                                         expressed in \type {percents}, so 60\% is expressed as the
                                         integer $60$ \NC \NR
\NC \type {\Umathstackvgap}          \NC vertical clearance between the two
                                         elements in a \type {\atop} stack \NC \NR
\NC \type {\Umathstacknumup}         \NC numerator shift upward in \type {\atop} stack \NC \NR
\NC \type {\Umathstackdenomdown}     \NC denominator shift downward in \type {\atop} stack \NC \NR
\NC \type {\Umathfractionrule}       \NC the width of the rule in a \type {\over} \NC \NR
\NC \type {\Umathfractionnumvgap}    \NC vertical clearance between the numerator and the rule \NC \NR
\NC \type {\Umathfractionnumup}      \NC numerator shift upward in \type {\over} \NC \NR
\NC \type {\Umathfractiondenomvgap}  \NC vertical clearance between the denominator and the rule \NC \NR
\NC \type {\Umathfractiondenomdown}  \NC denominator shift downward in \type {\over} \NC \NR
\NC \type {\Umathfractiondelsize}    \NC minimum delimiter size for \type {\...withdelims} \NC \NR
\NC \type {\Umathlimitabovevgap}     \NC vertical clearance for limits above operators \NC \NR
\NC \type {\Umathlimitabovebgap}     \NC vertical baseline clearance for limits above operators \NC \NR
\NC \type {\Umathlimitabovekern}     \NC space reserved at the top of the limit \NC \NR
\NC \type {\Umathlimitbelowvgap}     \NC vertical clearance for limits below operators \NC \NR
\NC \type {\Umathlimitbelowbgap}     \NC vertical baseline clearance for limits below operators \NC \NR
\NC \type {\Umathlimitbelowkern}     \NC space reserved at the bottom of the limit \NC \NR
\NC \type {\Umathoverdelimitervgap}  \NC vertical clearance for limits above delimiters \NC \NR
\NC \type {\Umathoverdelimiterbgap}  \NC vertical baseline clearance for limits above delimiters \NC \NR
\NC \type {\Umathunderdelimitervgap} \NC vertical clearance for limits below delimiters \NC \NR
\NC \type {\Umathunderdelimiterbgap} \NC vertical baseline clearance for limits below delimiters \NC \NR
\NC \type {\Umathsubshiftdrop}       \NC subscript drop for boxes and subformulas \NC \NR
\NC \type {\Umathsubshiftdown}       \NC subscript drop for characters \NC \NR
\NC \type {\Umathsupshiftdrop}       \NC superscript drop (raise, actually) for boxes and subformulas \NC \NR
\NC \type {\Umathsupshiftup}         \NC superscript raise for characters \NC \NR
\NC \type {\Umathsubsupshiftdown}    \NC subscript drop in the presence of a superscript \NC \NR
\NC \type {\Umathsubtopmax}          \NC the top of standalone subscripts cannot be higher than this
                                         above the baseline \NC \NR
\NC \type {\Umathsupbottommin}       \NC the bottom of standalone superscripts cannot be less than
                                         this above the baseline \NC \NR
\NC \type {\Umathsupsubbottommax}    \NC the bottom of the superscript of a combined super- and subscript
                                         be at least as high as this above the baseline \NC \NR
\NC \type {\Umathsubsupvgap}         \NC vertical clearance between super- and subscript \NC \NR
\NC \type {\Umathspaceafterscript}   \NC additional space added after a super- or subscript \NC \NR
\NC \type {\Umathconnectoroverlapmin}\NC minimum overlap between parts in an extensible recipe \NC \NR
\stoptabulate

Each of the parameters in this section can be set by a command like this:

\starttyping
\Umathquad\displaystyle=1em
\stoptyping

they obey grouping, and you can use \type {\the\Umathquad\displaystyle} if
needed.

\section{Skips around display math}

The injection of \type {\abovedisplayskip} and \type {\belowdisplayskip} is not
symmetrical. An above one is always inserted, also when zero, but the below is
only inserted when larger than zero. Especially the later mkes it sometimes hard
to fully control spacing. Therefore \LUATEX\ comes with a new directive: \type
{\mathdisplayskipmode}. The following values apply:

\starttabulate
\NC 0 \NC normal \TEX\ behaviour: always above, only below when larger than zero \NC \NR
\NC 1 \NC always \NC \NR
\NC 2 \NC only when not zero \NC \NR
\NC 3 \NC never, not even when not zero \NC \NR
\stoptabulate

\section{Font-based Math Parameters}

While it is nice to have these math parameters available for tweaking, it would
be tedious to have to set each of them by hand. For this reason, \LUATEX\
initializes a bunch of these parameters whenever you assign a font identifier to
a math family based on either the traditional math font dimensions in the font
(for assignments to math family~2 and~3 using \TFM|-|based fonts like \type
{cmsy} and \type {cmex}), or based on the named values in a potential \type
{MathConstants} table when the font is loaded via Lua. If there is a \type
{MathConstants} table, this takes precedence over font dimensions, and in that
case no attention is paid to which family is being assigned to: the \type
{MathConstants} tables in the last assigned family sets all parameters.

In the table below, the one|-|letter style abbreviations and symbolic tfm font
dimension names match those using in the \TeX book. Assignments to \type
{\textfont} set the values for the cramped and uncramped display and text styles,
\type {\scriptfont} sets the script styles, and \type {\scriptscriptfont} sets
the scriptscript styles, so we have eight parameters for three font sizes. In the
\TFM\ case, assignments only happen in family~2 and family~3 (and of course only
for the parameters for which there are font dimensions).

Besides the parameters below, \LUATEX\ also looks at the \quote {space} font
dimension parameter. For math fonts, this should be set to zero.

\start

\switchtobodyfont[8pt]

\starttabulate[|l|l|l|p|]
\NC \bf variable                      \NC \bf style             \NC \bf default value opentype               \NC \bf default value tfm \NC \NR
\NC \type {\Umathaxis}                \NC --                    \NC AxisHeight                               \NC axis_height \NC \NR
\NC \type {\Umathoperatorsize}        \NC D, D'                 \NC DisplayOperatorMinHeight                 \NC $^6$ \NC \NR
\NC \type {\Umathfractiondelsize}     \NC D, D'                 \NC FractionDelimiterDisplayStyleSize$^9$    \NC delim1 \NC \NR
\NC                                   \NC T, T', S, S', SS, SS' \NC FractionDelimiterSize$^9$                \NC delim2 \NC \NR
\NC \type {\Umathfractiondenomdown}   \NC D, D'                 \NC FractionDenominatorDisplayStyleShiftDown \NC denom1 \NC \NR
\NC                                   \NC T, T', S, S', SS, SS' \NC FractionDenominatorShiftDown             \NC denom2 \NC \NR
\NC \type {\Umathfractiondenomvgap}   \NC D, D'                 \NC FractionDenominatorDisplayStyleGapMin    \NC 3*default_rule_thickness \NC \NR
\NC                                   \NC T, T', S, S', SS, SS' \NC FractionDenominatorGapMin                \NC default_rule_thickness \NC \NR
\NC \type {\Umathfractionnumup}       \NC D, D'                 \NC FractionNumeratorDisplayStyleShiftUp     \NC num1 \NC \NR
\NC                                   \NC T, T', S, S', SS, SS' \NC FractionNumeratorShiftUp                 \NC num2 \NC \NR
\NC \type {\Umathfractionnumvgap}     \NC D, D'                 \NC FractionNumeratorDisplayStyleGapMin      \NC 3*default_rule_thickness \NC \NR
\NC                                   \NC T, T', S, S', SS, SS' \NC FractionNumeratorGapMin                  \NC default_rule_thickness \NC \NR
\NC \type {\Umathfractionrule}        \NC --                    \NC FractionRuleThickness                    \NC default_rule_thickness \NC \NR
\NC \type {\Umathskewedfractionhgap}  \NC --                    \NC SkewedFractionHorizontalGap              \NC math_quad/2 \NC \NR
\NC \type {\Umathskewedfractionvgap}  \NC --                    \NC SkewedFractionVerticalGap                \NC math_x_height \NC \NR
\NC \type {\Umathlimitabovebgap}      \NC --                    \NC UpperLimitBaselineRiseMin                \NC big_op_spacing3 \NC \NR
\NC \type {\Umathlimitabovekern}      \NC --                    \NC 0$^1$                                    \NC big_op_spacing5 \NC \NR
\NC \type {\Umathlimitabovevgap}      \NC --                    \NC UpperLimitGapMin                         \NC big_op_spacing1 \NC \NR
\NC \type {\Umathlimitbelowbgap}      \NC --                    \NC LowerLimitBaselineDropMin                \NC big_op_spacing4 \NC \NR
\NC \type {\Umathlimitbelowkern}      \NC --                    \NC 0$^1$                                    \NC big_op_spacing5 \NC \NR
\NC \type {\Umathlimitbelowvgap}      \NC --                    \NC LowerLimitGapMin                         \NC big_op_spacing2 \NC \NR
\NC \type {\Umathoverdelimitervgap}   \NC --                    \NC StretchStackGapBelowMin                  \NC big_op_spacing1 \NC \NR
\NC \type {\Umathoverdelimiterbgap}   \NC --                    \NC StretchStackTopShiftUp                   \NC big_op_spacing3 \NC \NR
\NC \type {\Umathunderdelimitervgap}  \NC--                     \NC StretchStackGapAboveMin                  \NC big_op_spacing2 \NC \NR
\NC \type {\Umathunderdelimiterbgap}  \NC--                     \NC StretchStackBottomShiftDown              \NC big_op_spacing4 \NC \NR
\NC \type {\Umathoverbarkern}         \NC --                    \NC OverbarExtraAscender                     \NC default_rule_thickness \NC \NR
\NC \type {\Umathoverbarrule}         \NC --                    \NC OverbarRuleThickness                     \NC default_rule_thickness \NC \NR
\NC \type {\Umathoverbarvgap}         \NC --                    \NC OverbarVerticalGap                       \NC 3*default_rule_thickness \NC \NR
\NC \type {\Umathquad}                \NC --                    \NC <font_size(f)>$^1$                       \NC math_quad \NC \NR
\NC \type {\Umathradicalkern}         \NC --                    \NC RadicalExtraAscender                     \NC default_rule_thickness \NC \NR
\NC \type {\Umathradicalrule}         \NC --                    \NC RadicalRuleThickness                     \NC <not set>$^2$ \NC \NR
\NC \type {\Umathradicalvgap}         \NC D, D'                 \NC RadicalDisplayStyleVerticalGap           \NC (default_rule_thickness+\crlf
                                                                                                                 (abs(math_x_height)/4))$^3$ \NC \NR
\NC                                   \NC T, T', S, S', SS, SS' \NC RadicalVerticalGap                       \NC (default_rule_thickness+\crlf
                                                                                                                 (abs(default_rule_thickness)/4))$^3$ \NC \NR
\NC \type {\Umathradicaldegreebefore} \NC --                    \NC RadicalKernBeforeDegree                  \NC <not set>$^2$ \NC \NR
\NC \type {\Umathradicaldegreeafter}  \NC --                    \NC RadicalKernAfterDegree                   \NC <not set>$^2$ \NC \NR
\NC \type {\Umathradicaldegreeraise}  \NC --                    \NC RadicalDegreeBottomRaisePercent          \NC <not set>$^{2,7}$ \NC \NR
\NC \type {\Umathspaceafterscript}    \NC --                    \NC SpaceAfterScript                         \NC script_space$^4$ \NC \NR
\NC \type {\Umathstackdenomdown}      \NC D, D'                 \NC StackBottomDisplayStyleShiftDown         \NC denom1 \NC \NR
\NC                                   \NC T, T', S, S', SS, SS' \NC StackBottomShiftDown                     \NC denom2 \NC \NR
\NC \type {\Umathstacknumup}          \NC D, D'                 \NC StackTopDisplayStyleShiftUp              \NC num1 \NC \NR
\NC                                   \NC T, T', S, S', SS, SS' \NC StackTopShiftUp                          \NC num3 \NC \NR
\NC \type {\Umathstackvgap}           \NC D, D'                 \NC StackDisplayStyleGapMin                  \NC 7*default_rule_thickness \NC \NR
\NC                                   \NC T, T', S, S', SS, SS' \NC StackGapMin                              \NC 3*default_rule_thickness \NC \NR
\NC \type {\Umathsubshiftdown}        \NC --                    \NC SubscriptShiftDown                       \NC sub1 \NC \NR
\NC \type {\Umathsubshiftdrop}        \NC --                    \NC SubscriptBaselineDropMin                 \NC sub_drop \NC \NR
\NC \type {\Umathsubsupshiftdown}     \NC --                    \NC SubscriptShiftDownWithSuperscript$^8$    \NC \NC \NR
\NC                                   \NC                       \NC \quad\ or SubscriptShiftDown             \NC sub2 \NC \NR
\NC \type {\Umathsubtopmax}           \NC --                    \NC SubscriptTopMax                          \NC (abs(math_x_height * 4) / 5) \NC \NR
\NC \type {\Umathsubsupvgap}          \NC --                    \NC SubSuperscriptGapMin                     \NC 4*default_rule_thickness \NC \NR
\NC \type {\Umathsupbottommin}        \NC --                    \NC SuperscriptBottomMin                     \NC (abs(math_x_height) / 4) \NC \NR
\NC \type {\Umathsupshiftdrop}        \NC --                    \NC SuperscriptBaselineDropMax               \NC sup_drop \NC \NR
\NC \type {\Umathsupshiftup}          \NC D                     \NC SuperscriptShiftUp                       \NC sup1 \NC \NR
\NC                                   \NC T, S, SS,             \NC SuperscriptShiftUp                       \NC sup2 \NC \NR
\NC                                   \NC D', T', S', SS'       \NC SuperscriptShiftUpCramped                \NC sup3 \NC \NR
\NC \type {\Umathsupsubbottommax}     \NC --                    \NC SuperscriptBottomMaxWithSubscript        \NC (abs(math_x_height * 4) / 5) \NC \NR
\NC \type {\Umathunderbarkern}        \NC --                    \NC UnderbarExtraDescender                   \NC default_rule_thickness \NC \NR
\NC \type {\Umathunderbarrule}        \NC --                    \NC UnderbarRuleThickness                    \NC default_rule_thickness \NC \NR
\NC \type {\Umathunderbarvgap}        \NC --                    \NC UnderbarVerticalGap                      \NC 3*default_rule_thickness \NC \NR
\NC \type {\Umathconnectoroverlapmin} \NC --                    \NC MinConnectorOverlap                      \NC 0$^5$ \NC \NR
\stoptabulate

\stop

Note 1: \OPENTYPE\ fonts set \type {\Umathlimitabovekern} and \type
{\Umathlimitbelowkern} to zero and set \type {\Umathquad} to the font size of the
used font, because these are not supported in the \type {MATH} table,

Note 2: Traditional \TFM\ fonts do not set \type {\Umathradicalrule} because
\TEX82\ uses the height of the radical instead. When this parameter is indeed not
set when \LUATEX\ has to typeset a radical, a backward compatibility mode will
kick in that assumes that an oldstyle \TEX\ font is used. Also, they do not set
\type {\Umathradicaldegreebefore}, \type {\Umathradicaldegreeafter}, and \type
{\Umathradicaldegreeraise}. These are then automatically initialized to
$5/18$quad, $-10/18$quad, and 60.

Note 3: If \TFM\ fonts are used, then the \type {\Umathradicalvgap} is not set
until the first time \LUATEX\ has to typeset a formula because this needs
parameters from both family~2 and family~3. This provides a partial backward
compatibility with \TEX82, but that compatibility is only partial: once the \type
{\Umathradicalvgap} is set, it will not be recalculated any more.

Note 4: When \TFM\ fonts are used a similar situation arises with respect to
\type {\Umathspaceafterscript}: it is not set until the first time \LUATEX\ has
to typeset a formula. This provides some backward compatibility with \TEX82. But
once the \type {\Umathspaceafterscript} is set, \type {\scriptspace} will never
be looked at again.

Note 5: Traditional \TFM\ fonts set \type {\Umathconnectoroverlapmin} to zero
because \TEX82\ always stacks extensibles without any overlap.

Note 6: The \type {\Umathoperatorsize} is only used in \type {\displaystyle}, and
is only set in \OPENTYPE\ fonts. In \TFM\ font mode, it is artificially set to
one scaled point more than the initial attempt's size, so that always the \quote
{first next} will be tried, just like in \TEX82.

Note 7: The \type {\Umathradicaldegreeraise} is a special case because it is the
only parameter that is expressed in a percentage instead of as a number of scaled
points.

Note 8: \type {SubscriptShiftDownWithSuperscript} does not actually exist in the
\quote {standard} \OPENTYPE\ math font Cambria, but it is useful enough to be
added.

Note 9: \type {FractionDelimiterDisplayStyleSize} and \type
{FractionDelimiterSize} do not actually exist in the \quote {standard} \OPENTYPE\
math font Cambria, but were useful enough to be added.

\section{Math spacing setting}

Besides the parameters mentioned in the previous sections, there are also 64 new
primitives to control the math spacing table (as explained in Chapter~18 of the
\TEX book). The primitive names are a simple matter of combining two math atom
types, but for completeness' sake, here is the whole list:

\starttwocolumns
\starttyping
\Umathordordspacing
\Umathordopspacing
\Umathordbinspacing
\Umathordrelspacing
\Umathordopenspacing
\Umathordclosespacing
\Umathordpunctspacing
\Umathordinnerspacing
\Umathopordspacing
\Umathopopspacing
\Umathopbinspacing
\Umathoprelspacing
\Umathopopenspacing
\Umathopclosespacing
\Umathoppunctspacing
\Umathopinnerspacing
\Umathbinordspacing
\Umathbinopspacing
\Umathbinbinspacing
\Umathbinrelspacing
\Umathbinopenspacing
\Umathbinclosespacing
\Umathbinpunctspacing
\Umathbininnerspacing
\Umathrelordspacing
\Umathrelopspacing
\Umathrelbinspacing
\Umathrelrelspacing
\Umathrelopenspacing
\Umathrelclosespacing
\Umathrelpunctspacing
\Umathrelinnerspacing
\Umathopenordspacing
\Umathopenopspacing
\Umathopenbinspacing
\Umathopenrelspacing
\Umathopenopenspacing
\Umathopenclosespacing
\Umathopenpunctspacing
\Umathopeninnerspacing
\Umathcloseordspacing
\Umathcloseopspacing
\Umathclosebinspacing
\Umathcloserelspacing
\Umathcloseopenspacing
\Umathcloseclosespacing
\Umathclosepunctspacing
\Umathcloseinnerspacing
\Umathpunctordspacing
\Umathpunctopspacing
\Umathpunctbinspacing
\Umathpunctrelspacing
\Umathpunctopenspacing
\Umathpunctclosespacing
\Umathpunctpunctspacing
\Umathpunctinnerspacing
\Umathinnerordspacing
\Umathinneropspacing
\Umathinnerbinspacing
\Umathinnerrelspacing
\Umathinneropenspacing
\Umathinnerclosespacing
\Umathinnerpunctspacing
\Umathinnerinnerspacing
\stoptyping
\stoptwocolumns

These parameters are of type \type {\muskip}, so setting a parameter can be done
like this:

\starttyping
\Umathopordspacing\displaystyle=4mu plus 2mu
\stoptyping

They are all initialized by \type {initex} to the values mentioned in the table
in Chapter~18 of the \TEX book.

Note 1: for ease of use as well as for backward compatibility, \type
{\thinmuskip}, \type {\medmuskip} and \type {\thickmuskip} are treated
especially. In their case a pointer to the corresponding internal parameter is
saved, not the actual \type {\muskip} value. This means that any later changes to
one of these three parameters will be taken into account.

Note 2: Careful readers will realise that there are also primitives for the items
marked \type {*} in the \TEX book. These will not actually be used as those
combinations of atoms cannot actually happen, but it seemed better not to break
orthogonality. They are initialized to zero.

\section[mathacc]{Math accent handling}

\LUATEX\ supports both top accents and bottom accents in math mode, and math
accents stretch automatically (if this is supported by the font the accent comes
from, of course). Bottom and combined accents as well as fixed-width math accents
are controlled by optional keywords following \type {\Umathaccent}.

The keyword \type {bottom} after \type {\Umathaccent} signals that a bottom accent
is needed, and the keyword \type {both} signals that both a top and a bottom
accent are needed (in this case two accents need to be specified, of course).

Then the set of three integers defining the accent is read. This set of integers
can be prefixed by the \type {fixed} keyword to indicate that a non-stretching
variant is requested (in case of both accents, this step is repeated).

A simple example:

\starttyping
\Umathaccent both fixed 0 0 "20D7 fixed 0 0 "20D7 {example}
\stoptyping

If a math top accent has to be placed and the accentee is a character and has a
non-zero \type {top_accent} value, then this value will be used to place the
accent instead of the \type {\skewchar} kern used by \TEX82.

The \type {top_accent} value represents a vertical line somewhere in the
accentee. The accent will be shifted horizontally such that its own \type
{top_accent} line coincides with the one from the accentee. If the \type
{top_accent} value of the accent is zero, then half the width of the accent
followed by its italic correction is used instead.

The vertical placement of a top accent depends on the \type {x_height} of the
font of the accentee (as explained in the \TEX book), but if value that turns out
to be zero and the font had a \type {MathConstants} table, then \type
{AccentBaseHeight} is used instead.

The vertical placement of a bottom accent is straight below the accentee, no
correction takes place.

Possible locations are \type {top}, \type {bottom}, \type {both} and \type
{center}. When no location is given \type {top} is assumed. An additional
parameter \type {fraction} can be specified followed by a number; a value of for
instance 1200 means that the criterium is 1.2 times the width of the nucleus. The
fraction only applies to the stepwise selected shapes and is mostly meant for the
\type {overlay} location. It also works for the other locations but then it
concerns the width.

\section{Math root extension}

The new primitive \type {\Uroot} allows the construction of a radical noad
including a degree field. Its syntax is an extension of \type {\Uradical}:

\starttyping
\Uradical <fam integer> <char integer> <radicand>
\Uroot    <fam integer> <char integer> <degree> <radicand>
\stoptyping

The placement of the degree is controlled by the math parameters \type
{\Umathradicaldegreebefore}, \type {\Umathradicaldegreeafter}, and \type
{\Umathradicaldegreeraise}. The degree will be typeset in \type
{\scriptscriptstyle}.

\section{Math kerning in super- and subscripts}

The character fields in a \LUA|-|loaded \OPENTYPE\ math font can have a \quote
{mathkern} table. The format of this table is the same as the \quote {mathkern}
table that is returned by the \type {fontloader} library, except that all height
and kern values have to be specified in actual scaled points.

When a super- or subscript has to be placed next to a math item, \LUATEX\ checks
whether the super- or subscript and the nucleus are both simple character items.
If they are, and if the fonts of both character items are \OPENTYPE\ fonts (as
opposed to legacy \TEX\ fonts), then \LUATEX\ will use the \OPENTYPE\ math
algorithm for deciding on the horizontal placement of the super- or subscript.

This works as follows:

\startitemize
    \startitem
        The vertical position of the script is calculated.
    \stopitem
    \startitem
        The default horizontal position is flat next to the base character.
    \stopitem
    \startitem
        For superscripts, the italic correction of the base character is added.
    \stopitem
    \startitem
        For a superscript, two vertical values are calculated: the bottom of the
        script (after shifting up), and the top of the base. For a subscript, the two
        values are the top of the (shifted down) script, and the bottom of the base.
    \stopitem
    \startitem
        For each of these two locations:
        \startitemize
            \startitem
                find the math kern value at this height for the base (for a subscript
                placement, this is the bottom_right corner, for a superscript
                placement the top_right corner)
            \stopitem
            \startitem
                find the math kern value at this height for the script (for a
                subscript placement, this is the top_left corner, for a superscript
                placement the bottom_left corner)
            \stopitem
            \startitem
                add the found values together to get a preliminary result.
            \stopitem
        \stopitemize
    \stopitem
    \startitem
        The horizontal kern to be applied is the smallest of the two results from
        previous step.
    \stopitem
\stopitemize

The math kern value at a specific height is the kern value that is specified by the
next higher height and kern pair, or the highest one in the character (if there is no
value high enough in the character), or simply zero (if the character has no math kern
pairs at all).

\section{Scripts on horizontally extensible items like arrows}

The primitives \type {\Uunderdelimiter} and \type {\Uoverdelimiter} allow the
placement of a subscript or superscript on an automatically extensible item and
\type {\Udelimiterunder} and \type {\Udelimiterover} allow the placement of an
automatically extensible item as a subscript or superscript on a nucleus. The
input:

% these produce radical noads .. in fact the code base has the numbers wrong for
% quite a while, so no one seems to use this

\startbuffer
$\Uoverdelimiter  0 "2194 {\hbox{\strut  overdelimiter}}$
$\Uunderdelimiter 0 "2194 {\hbox{\strut underdelimiter}}$
$\Udelimiterover  0 "2194 {\hbox{\strut  delimiterover}}$
$\Udelimiterunder 0 "2194 {\hbox{\strut delimiterunder}}$
\stopbuffer

\typebuffer will render this:

\blank \startnarrower \getbuffer \stopnarrower \blank

The vertical placements are controlled by \type {\Umathunderdelimiterbgap}, \type
{\Umathunderdelimitervgap}, \type {\Umathoverdelimiterbgap}, and \type
{\Umathoverdelimitervgap} in a similar way as limit placements on large operators.
The superscript in \type {\Uoverdelimiter} is typeset in a suitable scripted style,
the subscript in \type {\Uunderdelimiter} is cramped as well.

These primitives accepts an option \type {width} specification. When used the
also optional keywords \type {left}, \type {middle} and \type {right} will
determine what happens when a requested size can't be met (which can happen when
we step to successive larger variants).

An extra primitive \type {\Uhextensible} is available that can be used like this:

\startbuffer
$\Uhextensible width 10cm 0 "2194$
\stopbuffer

\typebuffer This will render this:

\blank \startnarrower \getbuffer \stopnarrower \blank

Here you can also pass options, like:

\startbuffer
$\Uhextensible width 1pt middle 0 "2194$
\stopbuffer

\typebuffer This gives:

\blank \startnarrower \getbuffer \stopnarrower \blank

\LUATEX\ internally uses a structure that supports \OPENTYPE\ \quote
{MathVariants} as well as \TFM\ \quote {extensible recipes}. In most cases where
font metrics are involved we have a different code path for traditional fonts end
\OPENTYPE\ fonts.

\section {Extracting values}

You can extract the components of a math character. Say that we have defined:

\starttyping
\Umathcode 1 2 3 4
\stoptyping

then

\starttyping
[\Umathcharclass1] [\Umathcharfam1] [\Umathcharslot1]
\stoptyping

will return:

\starttyping
[2] [3] [4]
\stoptyping

These commands are provides as convenience. Before they came available you could
do the following:

\starttyping
\def\Umathcharclass{\directlua{tex.print(tex.getmathcode(token.scan_int())[1])}}
\def\Umathcharfam  {\directlua{tex.print(tex.getmathcode(token.scan_int())[2])}}
\def\Umathcharslot {\directlua{tex.print(tex.getmathcode(token.scan_int())[3])}}
\stoptyping

\section{fractions}

The \type {\abovewithdelims} command accepts a keyword \type {exact}. When issued
the extra space relative to the rule thickness is not added. One can of course
use the \type {\Umathfraction..gap} commands to influence the spacing. Also the
rule is still positioned around the math axis.

\starttyping
$$ { {a} \abovewithdelims() exact 4pt {b} }$$
\stoptyping

The math parameter table contains some parameters that specify a horizontal and
vertical gap for skewed fractions. Of course some guessing is needed in order to
implement something that uses them. And so we now provide a primitive similar to the
other fraction related ones but with a few options so that one can influence the
rendering. Of course a user can also mess around a bit with the parameters
\type {\Umathskewedfractionhgap} and \type {\Umathskewedfractionvgap}.

The syntax used here is:

\starttyping
{ {1} \Uskewed / <options> {2} }
{ {1} \Uskewedwithdelims / () <options> {2} }
\stoptyping

where the options can be \type {noaxis} and \type {exact}. By default we add half
the axis to the shifts and by default we zero the width of the middle character.
For Latin Modern The result looks as follows:

\def\ShowA#1#2#3{$x + { {#1} \Uskewed           /    #3 {#2} } + x$}
\def\ShowB#1#2#3{$x + { {#1} \Uskewedwithdelims / () #3 {#2} } + x$}

\start
    \switchtobodyfont[modern]
    \starttabulate[||||||]
        \NC \NC
            \ShowA{a}{b}{} \NC
            \ShowA{1}{2}{} \NC
            \ShowB{a}{b}{} \NC
            \ShowB{1}{2}{} \NC
        \NR
        \NC \type{exact} \NC
            \ShowA{a}{b}{exact} \NC
            \ShowA{1}{2}{exact} \NC
            \ShowB{a}{b}{exact} \NC
            \ShowB{1}{2}{exact} \NC
        \NR
        \NC \type{noaxis} \NC
            \ShowA{a}{b}{noaxis} \NC
            \ShowA{1}{2}{noaxis} \NC
            \ShowB{a}{b}{noaxis} \NC
            \ShowB{1}{2}{noaxis} \NC
        \NR
        \NC \type{exact noaxis} \NC
            \ShowA{a}{b}{exact noaxis} \NC
            \ShowA{1}{2}{exact noaxis} \NC
            \ShowB{a}{b}{exact noaxis} \NC
            \ShowB{1}{2}{exact noaxis} \NC
        \NR
    \stoptabulate
\stop

\section {Other Math changes}

\subsection {Verbose versions of single-character math commands}

\LUATEX\ defines six new primitives that have the same function as
\type {^}, \type {_}, \type {$}, and \type {$$}: %$

\starttabulate[|l|l|l|l|]
\NC \bf primitive           \NC \bf explanation \NC \NR
\NC \type {\Usuperscript}      \NC Duplicates the functionality of \type {^} \NC \NR
\NC \type {\Usubscript}        \NC Duplicates the functionality of \type {_} \NC \NR
\NC \type {\Ustartmath}        \NC Duplicates the functionality of \type {$}, % $
                                   when used in non-math mode. \NC \NR
\NC \type {\Ustopmath}         \NC Duplicates the functionality of \type {$}, % $
                                   when used in inline math mode. \NC \NR
\NC \type {\Ustartdisplaymath} \NC Duplicates the functionality of \type {$$}, % $$
                                   when used in non-math mode. \NC \NR
\NC \type {\Ustopdisplaymath}  \NC Duplicates the functionality of \type {$$}, % $$
                                   when used in display math mode. \NC \NR
\stoptabulate

The \type {\Ustopmath} and \type {\Ustopdisplaymath} primitives check if the current
math mode is the correct one (inline vs.\ displayed), but you can freely intermix
the four mathon|/|mathoff commands with explicit dollar sign(s).

\subsection{Allowed math commands in non-math modes}

The commands \type {\mathchar}, and \type {\Umathchar} and control sequences that
are the result of \type {\mathchardef} or \type {\Umathchardef} are also
acceptable in the horizontal and vertical modes. In those cases, the \type
{\textfont} from the requested math family is used.

\section{Math surrounding skips}

Inline math is surrounded by (optional) \type {\mathsurround} spacing but that is fixed
dimension. There is now an additional parameter \type {\mathsurroundskip}. When set to a
non|-|zero value (or zero with some stretch or shrink) this parameter will replace
\type {\mathsurround}. By using an additional parameter instead of changing the nature
of \type {\mathsurround}, we can remain compatible.

% \section{Math todo}
%
% The following items are still todo.
%
% \startitemize
% \startitem
%     Pre-scripts.
% \stopitem
% \startitem
%     Multi-story stacks.
% \stopitem
% \startitem
%     Flattened accents for high characters (maybe).
% \stopitem
% \startitem
%     Better control over the spacing around displays and handling of equation numbers.
% \stopitem
% \startitem
%     Support for multi|-|line displays using \MATHML\ style alignment points.
% \stopitem
% \stopitemize

\subsection {Delimiters: \type{\Uleft}, \type {\Umiddle} and \type {\Uright}}

Normally you will force delimiters to certain sizes by putting an empty box or
rule next to it. The resulting delimiter will either be a character from the
stepwise size range or an extensible. The latter can be quite differently
positioned that the characters as it depends on the fit as well as the fact if
the used characters in the font have depth or height. Commands like (plain \TEX
s) \type {\big} need use this feature. In \LUATEX\ we provide a bit more control
by three variants that supporting optional parameters \type {height}, \type
{depth} and \type {axis}. The following example uses this:

\startbuffer
\Uleft   height 30pt depth 10pt      \Udelimiter "0 "0 "000028
\quad x\quad
\Umiddle height 40pt depth 15pt      \Udelimiter "0 "0 "002016
\quad x\quad
\Uright  height 30pt depth 10pt      \Udelimiter "0 "0 "000029
\quad \quad \quad
\Uleft   height 30pt depth 10pt axis \Udelimiter "0 "0 "000028
\quad x\quad
\Umiddle height 40pt depth 15pt axis \Udelimiter "0 "0 "002016
\quad x\quad
\Uright  height 30pt depth 10pt axis \Udelimiter "0 "0 "000029
\stopbuffer

\typebuffer

\startlinecorrection
\ruledhbox{\mathematics{\getbuffer}}
\stoplinecorrection

The keyword \type {exact} can be used as directive that the real dimensions
should be applied when the criteria can't be met which can happen when we're
still stepping through the successively larger variants. When no dimensions are
given the \type {noaxis} command can be used to prevent shifting over the axis.

You can influence the final class with the keyword \type {class} which will
influence the spacing.

\subsection{Fixed scripts}

We have three parameters that are used for this fixed anchoring:

\starttabulate[|l|l|]
\NC $d$ \NC \type {\Umathsubshiftdown}    \NC \NR
\NC $u$ \NC \type {\Umathsupshiftup}      \NC \NR
\NC $s$ \NC \type {\Umathsubsupshiftdown} \NC \NR
\stoptabulate

When we set \type {\mathscriptsmode} to a value other than zero these are used
for calculating fixed positions. This is something that is needed for instance
for chemistry. You can manipulate the mentioned variables to achive different
effects.

\def\SampleMath#1%
  {$\mathscriptsmode#1\mathupright CH_2 + CH^+_2 + CH^2_2$}

\starttabulate[|c|c|c|l|]
\NC \bf mode \NC \bf down      \NC \bf up        \NC                \NC \NR
\NC 0        \NC dynamic       \NC dynamic       \NC \SampleMath{0} \NC \NR
\NC 1        \NC $d$           \NC $u$           \NC \SampleMath{1} \NC \NR
\NC 2        \NC $s$           \NC $u$           \NC \SampleMath{2} \NC \NR
\NC 3        \NC $s$           \NC $u + s - d$   \NC \SampleMath{3} \NC \NR
\NC 4        \NC $d + (s-d)/2$ \NC $u + (s-d)/2$ \NC \SampleMath{4} \NC \NR
\NC 5        \NC $d$           \NC $u + s - d$   \NC \SampleMath{5} \NC \NR
\stoptabulate

The value of this parameter obeys grouping but applies to the whole current
formula.

% if needed we can put the value in stylenodes but maybe more should go there

\subsection {Tracing}

Because there are quite some math related parameters and values, it is possible
to limit tracing. Only when \type {tracingassigns} and|/|or \type
{tracingrestores} are set to~2 or more they will be traced.

\subsection {Math options}

The logic in the math engine is rather complex and there are often no universal
solutions (read: what works out well for one font, fails for another). Therefore
some variations in the implementation will be driven by options for which a new
primitive \type {\mathoption} has been introduced (so that we don't end up with
many new commands). The approach of options also permits us to see what effect a
specific solution has.

\subsubsection {\type {\mathoption noitaliccompensation}}

This option compensates placement for characters with a built|-|in italic
correction.

\startbuffer
{\showboxes\int}\quad
{\showboxes\int_{|}^{|}}\quad
{\showboxes\int\limits_{|}^{|}}
\stopbuffer

\typebuffer

Gives (with computer modern that has such italics):

\startlinecorrection[blank]
    \switchtobodyfont[modern]
    \startcombination[nx=2,ny=2,distance=5em]
        {\mathoption noitaliccompensation 0\relax \mathematics{\getbuffer}}
            {\nohyphens\type{0:inline}}
        {\mathoption noitaliccompensation 0\relax \mathematics{\displaymath\getbuffer}}
            {\nohyphens\type{0:display}}
        {\mathoption noitaliccompensation 1\relax \mathematics{\getbuffer}}
            {\nohyphens\type{1:inline}}
        {\mathoption noitaliccompensation 1\relax \mathematics{\displaymath\getbuffer}}
            {\nohyphens\type{1:display}}
    \stopcombination
\stoplinecorrection

\subsubsection {\type {\mathoption nocharitalic}}

When two characters follow each other italic correction can interfere. The
following example shows what this option does:

\startbuffer
\catcode"1D443=11
\catcode"1D444=11
\catcode"1D445=11
P( PP PQR
\stopbuffer

\typebuffer

Gives (with computer modern that has such italics):

\startlinecorrection[blank]
    \switchtobodyfont[modern]
    \startcombination[nx=2,ny=2,distance=5em]
        {\mathoption nocharitalic 0\relax \mathematics{\getbuffer}}
            {\nohyphens\type{0:inline}}
        {\mathoption nocharitalic 0\relax \mathematics{\displaymath\getbuffer}}
            {\nohyphens\type{0:display}}
        {\mathoption nocharitalic 1\relax \mathematics{\getbuffer}}
            {\nohyphens\type{1:inline}}
        {\mathoption nocharitalic 1\relax \mathematics{\displaymath\getbuffer}}
            {\nohyphens\type{1:display}}
    \stopcombination
\stoplinecorrection

\subsubsection {\type {\mathoption useoldfractionscaling}}

This option has been introduced as solution for tracker item 604 for fuzzy cases
around either or not present fraction related settings for new fonts.

\stopchapter

\stopcomponent
%
    % language=uk

\environment luatex-style
\environment luatex-logos

\startcomponent luatex-languages

\startchapter[reference=languages,title={Languages, characters, fonts and glyphs}]

\LUATEX's internal handling of the characters and glyphs that eventually become
typeset is quite different from the way \TEX82 handles those same objects. The
easiest way to explain the difference is to focus on unrestricted horizontal mode
(i.e.\ paragraphs) and hyphenation first. Later on, it will be easy to deal
with the differences that occur in horizontal and math modes.

In \TEX82, the characters you type are converted into \type {char_node} records
when they are encountered by the main control loop. \TEX\ attaches and processes
the font information while creating those records, so that the resulting \quote
{horizontal list} contains the final forms of ligatures and implicit kerning.
This packaging is needed because we may want to get the effective width of for
instance a horizontal box.

When it becomes necessary to hyphenate words in a paragraph, \TEX\ converts (one
word at time) the \type {char_node} records into a string by replacing ligatures
with their components and ignoring the kerning. Then it runs the hyphenation
algorithm on this string, and converts the hyphenated result back into a \quote
{horizontal list} that is consecutively spliced back into the paragraph stream.
Keep in mind that the paragraph may contain unboxed horizontal material, which
then already contains ligatures and kerns and the words therein are part of the
hyphenation process.

Those \type {char_node} records are somewhat misnamed, as they are glyph
positions in specific fonts, and therefore not really \quote {characters} in the
linguistic sense. There is no language information inside the \type {char_node}
records at all. Instead, language information is passed along using \type
{language whatsit} records inside the horizontal list.

In \LUATEX, the situation is quite different. The characters you type are always
converted into \type {glyph_node} records with a special subtype to identify them
as being intended as linguistic characters. \LUATEX\ stores the needed language
information in those records, but does not do any font|-|related processing at
the time of node creation. It only stores the index of the current font and a
reference to a character in that font.

When it becomes necessary to typeset a paragraph, \LUATEX\ first inserts all
hyphenation points right into the whole node list. Next, it processes all the
font information in the whole list (creating ligatures and adjusting kerning),
and finally it adjusts all the subtype identifiers so that the records are \quote
{glyph nodes} from now on.

\section[charsandglyphs]{Characters and glyphs}

\TEX82 (including \PDFTEX) differentiates between \type {char_node}s and \type
{lig_node}s. The former are simple items that contained nothing but a \quote
{character} and a \quote {font} field, and they lived in the same memory as
tokens did. The latter also contained a list of components, and a subtype
indicating whether this ligature was the result of a word boundary, and it was
stored in the same place as other nodes like boxes and kerns and glues.

In \LUATEX, these two types are merged into one, somewhat larger structure called
a \type {glyph_node}. Besides having the old character, font, and component
fields, and the new special fields like \quote {attr} (see~\in {section}
[glyphnodes]), these nodes also contain:

\startitemize

\startitem A subtype, split into four main types:

    \startitemize
        \startitem
            \type {character}, for characters to be hyphenated: the lowest bit
            (bit 0) is set to 1.
        \stopitem
        \startitem
            \type {glyph}, for specific font glyphs: the lowest bit (bit 0) is
            not set.
        \stopitem
        \startitem
            \type {ligature}, for ligatures (bit 1 is set)
        \stopitem
        \startitem
            \type {ghost}, for \quote {ghost objects} (bit 2 is set)
        \stopitem
    \stopitemize

    The latter two make further use of two extra fields (bits 3 and 4):

    \startitemize
        \startitem
            \type {left}, for ligatures created from a left word boundary and for
            ghosts created from \type {\leftghost}
        \stopitem
        \startitem
            \type {right}, for ligatures created from a right word boundary and
            for ghosts created from \type {\rightghost}
        \stopitem
   \stopitemize

   For ligatures, both bits can be set at the same time (in case of a
   single|-|glyph word).

\stopitem

\startitem
    \type {glyph_node}s of type \quote {character} also contain language data,
    split into four items that were current when the node was created: the
    \type {\setlanguage} (15 bits), \type {\lefthyphenmin} (8 bits), \type
    {\righthyphenmin} (8 bits), and \type {\uchyph} (1 bit).
\stopitem

\stopitemize

Incidentally, \LUATEX\ allows 16383 separate languages, and words can be 256
characters long. The language is stored with each character. You can set
\type {\firstvalidlanguage} to for instance~1 and make thereby language~0
an ignored hyphenation language.

The new primitive \type {\hyphenationmin} can be used to signal the minimal length
of a word. This value stored with the (current) language.

Because the \type {\uchyph} value is saved in the actual nodes, its handling is
subtly different from \TEX82: changes to \type {\uchyph} become effective
immediately, not at the end of the current partial paragraph.

Typeset boxes now always have their language information embedded in the nodes
themselves, so there is no longer a possible dependency on the surrounding
language settings. In \TEX82, a mid-paragraph statement like \type {\unhbox0} would
process the box using the current paragraph language unless there was a
\type {\setlanguage} issued inside the box. In \LUATEX, all language variables are
already frozen.

In traditional \TEX\ the process of hyphenation is driven by \type {lccode}s. In
\LUATEX\ we made this dependency less strong. There are several strategies
possible. When you do nothing, the currently used \type {lccode}s are used, when
loading patterns, setting exceptions or hyphenating a list.

When you set \type {\savinghyphcodes} to a value larger than zero the current set
of \type {lccode}s will be saved with the language. In that case changing a \type
{lccode} afterwards has no effect. However, you can adapt the set with:

\starttyping
\hjcode`a=`a
\stoptyping

This change is global which makes sense if you keep in mind that the moment that
hyphenation happens is (normally) when the paragraph or a horizontal box is
constructed. When \type {\savinghyphcodes} was zero when the language got
initialized you start out with nothing, otherwise you already have a set.

When a \type {\hjcode} is larger than $0$ but smaller than $32$ is indicates the
to be used length. In the following example we map a character (\type {x}) onto
another one in the patterns and tell the engine that \type {œ} counts as one
character. Because traditionally zero itself is reserved for inhibiting
hyphenation, a value of $32$ counts as zero.

\starttyping
% assuming french patterns:
foobar % foo-bar

\hjcode`x=`o

fxxbar % fxx-bar

\lefthyphenmin3

œdipus % œdi-pus

\lefthyphenmin4

œdipus % œdipus

\hjcode`œ=2

œdipus % œdi-pus

\hjcode`i=32
\hjcode`d=32

œdipus % œdipus
\stoptyping

Carrying all this information with each glyph would give too much overhead and
also make the process of setting up thee codes more complex. A solution with
\type {hjcode} sets was considered but rejected because in practice the current
approach is sufficient and it would not be compatible anyway.

Beware: the values are always saved in the format, independent of the setting
of \type {\savinghyphcodes} at the moment the format is dumped.

A boundary node normally would mark the end of a word which interferes with for
instance discretionary injection. For this you can use the \type {\wordboundary}
as trigger. Here are a few examples of usage:

\startbuffer
    discrete---discrete
\stopbuffer
\typebuffer \start \dontcomplain \hsize 1pt \getbuffer \par \stop
\startbuffer
    discrete\discretionary{}{}{---}discrete
\stopbuffer
\typebuffer \start \dontcomplain \hsize 1pt \getbuffer \par \stop
\startbuffer
    discrete\wordboundary\discretionary{}{}{---}discrete
\stopbuffer
\typebuffer \start \dontcomplain \hsize 1pt \getbuffer \par \stop
\startbuffer
    discrete\wordboundary\discretionary{}{}{---}\wordboundary discrete
\stopbuffer
\typebuffer \start \dontcomplain \hsize 1pt \getbuffer \par \stop
\startbuffer
    discrete\wordboundary\discretionary{---}{}{}\wordboundary discrete
\stopbuffer
\typebuffer \start \dontcomplain \hsize 1pt \getbuffer \par \stop

We only accept an explicit hyphen when there is a preceding glyph and we skip a
sequence of explicit hyphens as that normally indicates a \type {--} or \type
{---} ligature in which case we can in a worse case usage get bad node lists
later on due to messed up ligature building as these dashes are ligatures in base
fonts. This is a side effect of the separating the hyphenation, ligaturing and
kerning steps.

The start and end of a characters is signalled by a glue, penalty, kern or boundary
node. But by default also a hlist, vlist, rule, dir, whatsit, ins, and adjust node
indicate a start or end. You can omit the last set from the test by setting
\type {\hyphenationbounds} to a non|-|zero value:

\starttabulate[|l|l|]
\NC \type{0} \NC not strict \NC \NR
\NC \type{1} \NC strict start \NC \NR
\NC \type{2} \NC strict end \NC \NR
\NC \type{3} \NC strict start and strict end \NC \NR
\stoptabulate

The word start is determined as follows:

\starttabulate[|l|l|]
\BC boundary  \NC yes when wordboundary \NC \NR
\BC hlist     \NC when hyphenationbounds 1 or 3 \NC \NR
\BC vlist     \NC when hyphenationbounds 1 or 3 \NC \NR
\BC rule      \NC when hyphenationbounds 1 or 3 \NC \NR
\BC dir       \NC when hyphenationbounds 1 or 3 \NC \NR
\BC whatsit   \NC when hyphenationbounds 1 or 3 \NC \NR
\BC glue      \NC yes \NC \NR
\BC math      \NC skipped \NC \NR
\BC glyph     \NC exhyphenchar (one only) : yes (so no -- ---) \NC \NR
\BC otherwise \NC yes \NC \NR
\stoptabulate

The word end is determined as follows:

\starttabulate[|l|l|]
\BC boundary  \NC yes \NC \NR
\BC glyph     \NC yes when different language \NC \NR
\BC glue      \NC yes \NC \NR
\BC penalty   \NC yes \NC \NR
\BC kern      \NC yes when not italic (for some historic reason) \NC \NR
\BC hlist     \NC when hyphenationbounds 2 or 3 \NC \NR
\BC vlist     \NC when hyphenationbounds 2 or 3 \NC \NR
\BC rule      \NC when hyphenationbounds 2 or 3 \NC \NR
\BC dir       \NC when hyphenationbounds 2 or 3 \NC \NR
\BC whatsit   \NC when hyphenationbounds 2 or 3 \NC \NR
\BC ins       \NC when hyphenationbounds 2 or 3 \NC \NR
\BC adjust    \NC when hyphenationbounds 2 or 3 \NC \NR
\stoptabulate

\in{Figures}[hb:1] upto \in[hb:5] show some examples. In all cases we set the min
values to 1 and make sure that the words hyphenate at each character.

\hyphenation{o-n-e t-w-o}

\def\SomeTest#1#2%
  {\lefthyphenmin  \plusone
   \righthyphenmin \plusone
   \parindent      \zeropoint
   \everypar       \emptytoks
   \dontcomplain
   \hbox to 2cm {%
     \vtop {%
       \hsize 1pt
       \hyphenationbounds#1
       #2
       \par}}}

\startplacefigure[reference=hb:1,title={\type{one}}]
    \startcombination[4*1]
        {\SomeTest{0}{one}}          {\type{0}}
        {\SomeTest{1}{one}}          {\type{1}}
        {\SomeTest{2}{one}}          {\type{2}}
        {\SomeTest{3}{one}}          {\type{3}}
    \stopcombination
\stopplacefigure
\startplacefigure[reference=hb:2,title={\type{one\null two}}]
    \startcombination[4*1]
        {\SomeTest{0}{one\null two}} {\type{0}}
        {\SomeTest{1}{one\null two}} {\type{1}}
        {\SomeTest{2}{one\null two}} {\type{2}}
        {\SomeTest{3}{one\null two}} {\type{3}}
    \stopcombination
\stopplacefigure
\startplacefigure[reference=hb:3,title={\type{\null one\null two}}]
    \startcombination[4*1]
        {\SomeTest{0}{\null one\null two}} {\type{0}}
        {\SomeTest{1}{\null one\null two}} {\type{1}}
        {\SomeTest{2}{\null one\null two}} {\type{2}}
        {\SomeTest{3}{\null one\null two}} {\type{3}}
    \stopcombination
\stopplacefigure
\startplacefigure[reference=hb:4,title={\type{one\null two\null}}]
    \startcombination[4*1]
        {\SomeTest{0}{one\null two\null}} {\type{0}}
        {\SomeTest{1}{one\null two\null}} {\type{1}}
        {\SomeTest{2}{one\null two\null}} {\type{2}}
        {\SomeTest{3}{one\null two\null}} {\type{3}}
    \stopcombination
\stopplacefigure
\startplacefigure[reference=hb:5,title={\type{\null one\null two\null}}]
    \startcombination[4*1]
        {\SomeTest{0}{\null one\null two\null}} {\type{0}}
        {\SomeTest{1}{\null one\null two\null}} {\type{1}}
        {\SomeTest{2}{\null one\null two\null}} {\type{2}}
        {\SomeTest{3}{\null one\null two\null}} {\type{3}}
    \stopcombination
\stopplacefigure

% (Future versions of \LUATEX\ might provide more granularity.)

In traditional \TEX\ ligature building and hyphenation are interwoven with the
line break mechanism. In \LUATEX\ these phases are isolated. As a consequence we
deal differently with (a sequence of) explicit hyphens. We already have added
some control over aspects of the hyphenation and yet another one concerns
automatic hyphens (e.g.\ \type {-} characters in the input).

When \type {\automatichyphenmode} has a value of 0, a hyphen will be turned into
an automatic discretionary. The snippets before and after it will not be
hyphenated. A side effect is that a leading hyphen can lead to a split but one
will seldom run into that situation. Setting a pre and post character makes this
more prominent. A value of 1 will prevent this side effect and a value of 2 will
not turn the hyphen into a discretionary. Experiments with other options, like
permitting hyphenation of the words on both sides were discarded.

\startbuffer[a]
before-after \par
before--after \par
before---after \par
\stopbuffer

\startbuffer[b]
-before \par
after- \par
--before \par
after-- \par
---before \par
after--- \par
\stopbuffer

\startbuffer[c]
before-after \par
before--after \par
before---after \par
\stopbuffer

We show three samples:

Input A: \typebuffer[a]
Input B: \typebuffer[b]
Input C: \typebuffer[c]

\startbuffer[demo]
\startcombination[nx=4,ny=3,location=top]
    {\framed[align=normal,strut=no,top=\vskip.5ex,bottom=\vskip.5ex]{\automatichyphenmode\zerocount \hsize6em \getbuffer[a]}} {A~0~6em}
    {\framed[align=normal,strut=no,top=\vskip.5ex,bottom=\vskip.5ex]{\automatichyphenmode\zerocount \hsize2pt \getbuffer[a]}} {A~0~2pt}
    {\framed[align=normal,strut=no,top=\vskip.5ex,bottom=\vskip.5ex]{\automatichyphenmode\plusone   \hsize2pt \getbuffer[a]}} {A~1~2pt}
    {\framed[align=normal,strut=no,top=\vskip.5ex,bottom=\vskip.5ex]{\automatichyphenmode\plustwo   \hsize2pt \getbuffer[a]}} {A~2~2pt}
    {\framed[align=normal,strut=no,top=\vskip.5ex,bottom=\vskip.5ex]{\automatichyphenmode\zerocount \hsize6em \getbuffer[b]}} {B~0~6em}
    {\framed[align=normal,strut=no,top=\vskip.5ex,bottom=\vskip.5ex]{\automatichyphenmode\zerocount \hsize2pt \getbuffer[b]}} {B~0~2pt}
    {\framed[align=normal,strut=no,top=\vskip.5ex,bottom=\vskip.5ex]{\automatichyphenmode\plusone   \hsize2pt \getbuffer[b]}} {B~1~2pt}
    {\framed[align=normal,strut=no,top=\vskip.5ex,bottom=\vskip.5ex]{\automatichyphenmode\plustwo   \hsize2pt \getbuffer[b]}} {B~2~2pt}
    {\framed[align=normal,strut=no,top=\vskip.5ex,bottom=\vskip.5ex]{\automatichyphenmode\zerocount \hsize6em \getbuffer[c]}} {C~0~6em}
    {\framed[align=normal,strut=no,top=\vskip.5ex,bottom=\vskip.5ex]{\automatichyphenmode\zerocount \hsize2pt \getbuffer[c]}} {C~0~2pt}
    {\framed[align=normal,strut=no,top=\vskip.5ex,bottom=\vskip.5ex]{\automatichyphenmode\plusone   \hsize2pt \getbuffer[c]}} {C~1~2pt}
    {\framed[align=normal,strut=no,top=\vskip.5ex,bottom=\vskip.5ex]{\automatichyphenmode\plustwo   \hsize2pt \getbuffer[c]}} {C~2~2pt}
\stopcombination
\stopbuffer

\startplacefigure[reference=automatic:1,title={The automatic modes \type {0} (default), \type {1} and \type {2}, with a \type {\hsize}
of 6em and 2pt (which triggers a linebreak).}]
    \dontcomplain \tt \getbuffer[demo]
\stopplacefigure

\startplacefigure[reference=automatic:2,title={The automatic modes \type {0} (default), \type {1} and \type {2}, with \type
{\preexhyphenchar} and \type {\postexhyphenchar} set to characters \type {A} and \type {B}.}]
    \postexhyphenchar`A\relax
    \preexhyphenchar `B\relax
    \dontcomplain \tt \getbuffer[demo]
\stopplacefigure

As with primitive companions of other single character commands, the \type {\-}
command has a more verbose primitive version in \type {\explicitdiscretionary}
and the normally intercepted in the hyphenator character \type {-} (or whatever
is configured) is available as \type {\automaticdiscretionary}.

\section{The main control loop}

In \LUATEX's main loop, almost all input characters that are to be typeset are
converted into \type {glyph} node records with subtype \quote {character}, but
there are a few exceptions.

First, the \type {\accent} primitives creates nodes with subtype \quote {glyph}
instead of \quote {character}: one for the actual accent and one for the
accentee. The primary reason for this is that \type {\accent} in \TEX82 is
explicitly dependent on the current font encoding, so it would not make much
sense to attach a new meaning to the primitive's name, as that would invalidate
many old documents and macro packages. \footnote {Of course, modern packages will
not use the \type {\accent} primitive at all but try to map directly on composed
characters.} A secondary reason is that in \TEX82, \type {\accent} prohibits
hyphenation of the current word. Since in \LUATEX\ hyphenation only takes place
on \quote {character} nodes, it is possible to achieve the same effect.

This change of meaning did happen with \type {\char}, that now generates \quote
{glyph} nodes with a character subtype. In traditional \TEX\ there was a strong
relationship between the 8|-|bit input encoding, hyphenation and glyphs taken
from a font. In \LUATEX\ we have \UTF\ input, and in most cases this maps
directly to a character in a font, apart from glyph replacement in the font
engine. If you want to access arbitrary glyphs in a font directly you can always
use \LUA\ to do so, because fonts are available as \LUA\ table.

Second, all the results of processing in math mode eventually become nodes with
\quote {glyph} subtypes.

Third, the \ALEPH|-|derived commands \type {\leftghost} and \type {\rightghost}
create nodes of a third subtype: \quote {ghost}. These nodes are ignored
completely by all further processing until the stage where inter|-|glyph kerning
is added.

Fourth, automatic discretionaries are handled differently. \TEX82 inserts an
empty discretionary after sensing an input character that matches the \type
{\hyphenchar} in the current font. This test is wrong in our opinion: whether or
not hyphenation takes place should not depend on the current font, it is a
language property. \footnote {When \TEX\ showed up we didn't have \UNICODE\ yet
and being limited to eight bits meant that one sometimes had to compromise
between supporting character input, glyph rendering, hyphenation.}

In \LUATEX, it works like this: if \LUATEX\ senses a string of input characters
that matches the value of the new integer parameter \type {\exhyphenchar}, it will
insert an explicit discretionary after that series of nodes. Initex sets the \type
{\exhyphenchar=`\-}. Incidentally, this is a global parameter instead of a
language-specific one because it may be useful to change the value depending on
the document structure instead of the text language.

The insertion of discretionaries after a sequence of explicit hyphens happens at
the same time as the other hyphenation processing, {\it not\/} inside the main
control loop.

The only use \LUATEX\ has for \type {\hyphenchar} is at the check whether a word
should be considered for hyphenation at all. If the \type {\hyphenchar} of the
font attached to the first character node in a word is negative, then hyphenation
of that word is abandoned immediately. This behaviour is added for backward
compatibility only, and the use of \type {\hyphenchar=-1} as a means of
preventing hyphenation should not be used in new \LUATEX\ documents.

Fifth, \type {\setlanguage} no longer creates whatsits. The meaning of \type
{\setlanguage} is changed so that it is now an integer parameter like all others.
That integer parameter is used in \type {\glyph_node} creation to add language
information to the glyph nodes. In conjunction, the \type {\language} primitive is
extended so that it always also updates the value of \type {\setlanguage}.

Sixth, the \type {\noboundary} command (that prohibits word boundary processing
where that would normally take place) now does create nodes. These nodes are
needed because the exact place of the \type {\noboundary} command in the input
stream has to be retained until after the ligature and font processing stages.

Finally, there is no longer a \type {main_loop} label in the code. Remember that
\TEX82 did quite a lot of processing while adding \type {char_nodes} to the
horizontal list? For speed reasons, it handled that processing code outside of
the \quote {main control} loop, and only the first character of any \quote {word}
was handled by that \quote {main control} loop. In \LUATEX, there is no longer a
need for that (all hard work is done later), and the (now very small) bits of
character|-|handling code have been moved back inline. When \type
{\tracingcommands} is on, this is visible because the full word is reported,
instead of just the initial character.

Because we tend to make hard codes behaviour configurable a few new primitives
have been added:

\starttyping
\hyphenpenaltymode
\automatichyphenpenalty
\explicithyphenpenalty
\stoptyping

The first parameter has the following consequences for automatic discs (the ones
resulting from an \type {\exhyphenchar}:

\starttabulate[|c|l|l|]
\BC mode     \BC automatic disc \type{-}         \BC explicit disc \type{\-}         \NC \NR
\HL
\NC \type{0} \NC \type {\exhyphenpenalty}        \NC \type {\exhyphenpenalty}        \NC \NR
\NC \type{1} \NC \type {\hyphenpenalty}          \NC \type {\hyphenpenalty}          \NC \NR
\NC \type{2} \NC \type {\exhyphenpenalty}        \NC \type {\hyphenpenalty}          \NC \NR
\NC \type{3} \NC \type {\hyphenpenalty}          \NC \type {\exhyphenpenalty}        \NC \NR
\NC \type{4} \NC \type {\automatichyphenpenalty} \NC \type {\explicithyphenpenalty}  \NC \NR
\NC \type{5} \NC \type {\exhyphenpenalty}        \NC \type {\explicithyphenpenalty}  \NC \NR
\NC \type{6} \NC \type {\hyphenpenalty}          \NC \type {\explicithyphenpenalty}  \NC \NR
\NC \type{7} \NC \type {\automatichyphenpenalty} \NC \type {\exhyphenpenalty}        \NC \NR
\NC \type{8} \NC \type {\automatichyphenpenalty} \NC \type {\hyphenpenalty}          \NC \NR
\stoptabulate

other values do what we always did in \LUATEX: insert \type {\exhyphenpenalty}.

\section[patternsexceptions]{Loading patterns and exceptions}

The hyphenation algorithm in \LUATEX\ is quite different from the one in \TEX82,
although it uses essentially the same user input.

After expansion, the argument for \type {\patterns} has to be proper \UTF8 with
individual patterns separated by spaces, no \type {\char} or \type {\chardef}d
commands are allowed. The current implementation quite strict and will reject all
non|-|\UNICODE\ characters.

Likewise, the expanded argument for \type {\hyphenation} also has to be proper
\UTF8, but here a bit of extra syntax is provided:

\startitemize[n]
\startitem
    Three sets of arguments in curly braces (\type {{}{}{}}) indicates a desired
    complex discretionary, with arguments as in \type {\discretionary}'s command in
    normal document input.
\stopitem
\startitem
    A \type {-} indicates a desired simple discretionary, cf.\ \type {\-} and \type
    {\discretionary{-}{}{}} in normal document input.
\stopitem
\startitem
    Internal command names are ignored. This rule is provided especially for \type
    {\discretionary}, but it also helps to deal with \type {\relax} commands that
    may sneak in.
\stopitem
\startitem
    An \type {=} indicates a (non|-|discretionary) hyphen in the document input.
\stopitem
\stopitemize

The expanded argument is first converted back to a space-separated string while
dropping the internal command names. This string is then converted into a
dictionary by a routine that creates key|-|value pairs by converting the other
listed items. It is important to note that the keys in an exception dictionary
can always be generated from the values. Here are a few examples:

\starttabulate[|l|l|l|]
\BC value                  \BC implied key (input) \NC effect \NC\NR
\NC \type {ta-ble}         \NC table               \NC \type {ta\-ble} ($=$ \type {ta\discretionary{-}{}{}ble}) \NC\NR
\NC \type {ba{k-}{}{c}ken} \NC backen              \NC \type {ba\discretionary{k-}{}{c}ken} \NC\NR
\stoptabulate

The resultant patterns and exception dictionary will be stored under the language
code that is the present value of \type {\language}.

In the last line of the table, you see there is no \type {\discretionary} command
in the value: the command is optional in the \TEX-based input syntax. The
underlying reason for that is that it is conceivable that a whole dictionary of
words is stored as a plain text file and loaded into \LUATEX\ using one of the
functions in the \LUA\ \type {lang} library. This loading method is quite a bit
faster than going through the \TEX\ language primitives, but some (most?) of that
speed gain would be lost if it had to interpret command sequences while doing so.

It is possible to specify extra hyphenation points in compound words by using
\type {{-}{}{-}} for the explicit hyphen character (replace \type {-} by the
actual explicit hyphen character if needed). For example, this matches the word
\quote {multi|-|word|-|boundaries} and allows an extra break inbetween \quote
{boun} and \quote {daries}:

\starttyping
\hyphenation{multi{-}{}{-}word{-}{}{-}boun-daries}
\stoptyping

The motivation behind the \ETEX\ extension \type {\savinghyphcodes} was that
hyphenation heavily depended on font encodings. This is no longer true in
\LUATEX, and the corresponding primitive is basically ignored. Because we now
have \type {hjcode}, the case relate codes can be used exclusively for \type
{\uppercase} and \type {\lowercase}.

\section{Applying hyphenation}

The internal structures \LUATEX\ uses for the insertion of discretionaries in
words is very different from the ones in \TEX82, and that means there are some
noticeable differences in handling as well.

First and foremost, there is no \quote {compressed trie} involved in hyphenation.
The algorithm still reads \PATGEN-generated pattern files, but \LUATEX\ uses a
finite state hash to match the patterns against the word to be hyphenated. This
algorithm is based on the \quote {libhnj} library used by \OPENOFFICE, which in
turn is inspired by \TEX.

There are a few differences between \LUATEX\ and \TEX82 that are a direct result
of the implementation:

\startitemize
\startitem
    \LUATEX\ happily hyphenates the full \UNICODE\ character range.
\stopitem
\startitem
    Pattern and exception dictionary size is limited by the available memory
    only, all allocations are done dynamically. The trie|-|related settings in
    \type {texmf.cnf} are ignored.
\stopitem
\startitem
    Because there is no \quote {trie preparation} stage, language patterns never
    become frozen. This means that the primitive \type {\patterns} (and its \LUA\
    counterpart \type {lang.patterns}) can be used at any time, not only in
    ini\TEX.
\stopitem
\startitem
    Only the string representation of \type {\patterns} and \type {\hyphenation} is
    stored in the format file. At format load time, they are simply
    re|-|evaluated. It follows that there is no real reason to preload languages
    in the format file. In fact, it is usually not a good idea to do so. It is
    much smarter to load patterns no sooner than the first time they are actually
    needed.
\stopitem
\startitem
    \LUATEX\ uses the language-specific variables \type {\prehyphenchar} and \type
    {\posthyphenchar} in the creation of implicit discretionaries, instead of
    \TEX82's \type {\hyphenchar}, and the values of the language|-|specific variables
    \type {\preexhyphenchar} and \type {\postexhyphenchar} for explicit
    discretionaries (instead of \TEX82's empty discretionary).
\stopitem
\startitem
    The value of the two counters related to hyphenation, \type {\hyphenpenalty}
    and \type {\exhyphenpenalty}, are now stored in the discretionary nodes. This
    permits a local overload for explicit \type {\discretionary} commands. The
    value current when the hyphenation pass is applied is used. When no callbacks
    are used this is compatible with traditional \TEX. When you apply the \LUA\
    \type {lang.hyphenate} function the current values are used.
\stopitem
\stopitemize

Because we store penalties in the disc node the \type {\discretionary} command has
been extended to accept an optional penalty specification, so you can do the
following:

\startbuffer
\hsize1mm
1:foo{\hyphenpenalty 10000\discretionary{}{}{}}bar\par
2:foo\discretionary penalty 10000 {}{}{}bar\par
3:foo\discretionary{}{}{}bar\par
\stopbuffer

\typebuffer

This results in:

\blank \start \getbuffer \stop \blank

Inserted characters and ligatures inherit their attributes from the nearest glyph
node item (usually the preceding one, but the following one for the items
inserted at the left-hand side of a word).

Word boundaries are no longer implied by font switches, but by language switches.
One word can have two separate fonts and still be hyphenated correctly (but it
can not have two different languages, the \type {\setlanguage} command forces a
word boundary).

All languages start out with \type {\prehyphenchar=`\-}, \type {\posthyphenchar=0},
\type {\preexhyphenchar=0} and \type {\postexhyphenchar=0}. When you assign the
values of one of these four parameters, you are actually changing the settings
for the current \type {\language}, this behaviour is compatible with \type {\patterns}
and \type {\hyphenation}.

\LUATEX\ also hyphenates the first word in a paragraph. Words can be up to 256
characters long (up from 64 in \TEX82). Longer words generate an error right now,
but eventually either the limitation will be removed or perhaps it will become
possible to silently ignore the excess characters (this is what happens in
\TEX82, but there the behaviour cannot be controlled).

If you are using the \LUA\ function \type {lang.hyphenate}, you should be aware
that this function expects to receive a list of \quote {character} nodes. It will
not operate properly in the presence of \quote {glyph}, \quote {ligature}, or
\quote {ghost} nodes, nor does it know how to deal with kerning.

The hyphenation exception dictionary is maintained as key|-|value hash, and that
is also dynamic, so the \type {hyph_size} setting is not used either.

\section{Applying ligatures and kerning}

After all possible hyphenation points have been inserted in the list, \LUATEX\
will process the list to convert the \quote {character} nodes into \quote {glyph}
and \quote {ligature} nodes. This is actually done in two stages: first all
ligatures are processed, then all kerning information is applied to the result
list. But those two stages are somewhat dependent on each other: If the used font
makes it possible to do so, the ligaturing stage adds virtual \quote {character}
nodes to the word boundaries in the list. While doing so, it removes and
interprets \type {\noboundary} nodes. The kerning stage deletes those word
boundary items after it is done with them, and it does the same for \quote
{ghost} nodes. Finally, at the end of the kerning stage, all remaining \quote
{character} nodes are converted to \quote {glyph} nodes.

This work separation is worth mentioning because, if you overrule from \LUA\ only
one of the two callbacks related to font handling, then you have to make sure you
perform the tasks normally done by \LUATEX\ itself in order to make sure that the
other, non|-|overruled, routine continues to function properly.

Work in this area is not yet complete, but most of the possible cases are handled
by our rewritten ligaturing engine. At some point all of the possible inputs will
become supported. \footnote {Not all of this makes sense because we nowadays have
\OPENTYPE\ fonts and ligature building can happen in ,any different ways there.}

For example, take the word \type {office}, hyphenated \type {of-fice}, using a
\quote {normal} font with all the \type {f}-\type {f} and \type {f}-\type {i}
type ligatures:

\starttabulate[|l|l|]
\NC initial              \NC \type {{o}{f}{f}{i}{c}{e}}             \NC\NR
\NC after hyphenation    \NC \type {{o}{f}{{-},{},{}}{f}{i}{c}{e}}  \NC\NR
\NC first ligature stage \NC \type {{o}{{f-},{f},{<ff>}}{i}{c}{e}}  \NC\NR
\NC final result         \NC \type {{o}{{f-},{<fi>},{<ffi>}}{c}{e}} \NC\NR
\stoptabulate

That's bad enough, but let us assume that there is also a hyphenation point
between the \type {f} and the \type {i}, to create \type {of-f-ice}. Then the
final result should be:

\starttyping
{o}{{f-},
    {{f-},
     {i},
     {<fi>}},
    {{<ff>-},
     {i},
     {<ffi>}}}{c}{e}
\stoptyping

with discretionaries in the post-break text as well as in the replacement text of
the top-level discretionary that resulted from the first hyphenation point.

Here is that nested solution again, in a different representation:

\starttabulate[|l|c|c|c|c|c|c|]
\NC         \BC pre           \BC     \BC post      \BC       \BC replace       \BC       \NC \NR
\NC topdisc \NC \type {f-}    \NC (1) \NC           \NC sub 1 \NC               \NC sub 2 \NC \NR
\NC sub 1   \NC \type {f-}    \NC (2) \NC \type {i} \NC (3)   \NC \type {<fi>}  \NC (4)   \NC \NR
\NC sub 2   \NC \type {<ff>-} \NC (5) \NC \type {i} \NC (6)   \NC \type {<ffi>} \NC (7)   \NC \NR
\stoptabulate

When line breaking is choosing its breakpoints, the following fields will
eventually be selected:

\starttabulate[|l|c|c|]
\NC \type {of-f-ice} \NC \type {f-}    \NC (1) \NC \NR
\NC                  \NC \type {f-}    \NC (2) \NC \NR
\NC                  \NC \type {i}     \NC (3) \NC \NR
\NC \type {of-fice}  \NC \type {f-}    \NC (1) \NC \NR
\NC                  \NC \type {<fi>}  \NC (4) \NC \NR
\NC \type {off-ice}  \NC \type {<ff>-} \NC (5) \NC \NR
\NC                  \NC \type {i}     \NC (6) \NC \NR
\NC \type {office}   \NC \type {<ffi>} \NC (7) \NC \NR
\stoptabulate

The current solution in \LUATEX\ is not able to handle nested discretionaries,
but it is in fact smart enough to handle this fictional \type {of-f-ice} example.
It does so by combining two sequential discretionary nodes as if they were a
single object (where the second discretionary node is treated as an extension of
the first node).

One can observe that the \type {of-f-ice} and \type {off-ice} cases both end with
the same actual post replacement list (\type {i}), and that this would be the
case even if that \type {i} was the first item of a potential following ligature
like \type {ic}. This allows \LUATEX\ to do away with one of the fields, and thus
make the whole stuff fit into just two discretionary nodes.

The mapping of the seven list fields to the six fields in this discretionary node
pair is as follows:

\starttabulate[|l|c|c|]
\BC field                 \BC description   \NC       \NC \NR
\NC \type {disc1.pre}     \NC \type {f-}    \NC (1)   \NC \NR
\NC \type {disc1.post}    \NC \type {<fi>}  \NC (4)   \NC \NR
\NC \type {disc1.replace} \NC \type {<ffi>} \NC (7)   \NC \NR
\NC \type {disc2.pre}     \NC \type {f-}    \NC (2)   \NC \NR
\NC \type {disc2.post}    \NC \type {i}     \NC (3,6) \NC \NR
\NC \type {disc2.replace} \NC \type {<ff>-} \NC (5)   \NC \NR
\stoptabulate

What is actually generated after ligaturing has been applied is therefore:

\starttyping
{o}{{f-},
    {<fi>},
    {<ffi>}}
   {{f-},
    {i},
    {<ff>-}}{c}{e}
\stoptyping

The two discretionaries have different subtypes from a discretionary appearing on
its own: the first has subtype 4, and the second has subtype 5. The need for
these special subtypes stems from the fact that not all of the fields appear in
their \quote {normal} location. The second discretionary especially looks odd,
with things like the \type {<ff>-} appearing in \type {disc2.replace}. The fact
that some of the fields have different meanings (and different processing code
internally) is what makes it necessary to have different subtypes: this enables
\LUATEX\ to distinguish this sequence of two joined discretionary nodes from the
case of two standalone discretionaries appearing in a row.

Of course there is still that relationship with fonts: ligatures can be implemented by
mapping a sequence of glyphs onto one glyph, but also by selective replacement and
kerning. This means that the above examples are just representing the traditional
approach.

\section{Breaking paragraphs into lines}

This code is still almost unchanged, but because of the above|-|mentioned changes
with respect to discretionaries and ligatures, line breaking will potentially be
different from traditional \TEX. The actual line breaking code is still based on
the \TEX82 algorithms, and it does not expect there to be discretionaries inside
of discretionaries.

But that situation is now fairly common in \LUATEX, due to the changes to the
ligaturing mechanism. And also, the \LUATEX\ discretionary nodes are implemented
slightly different from the \TEX82 nodes: the \type {no_break} text is now
embedded inside the disc node, where previously these nodes kept their place in
the horizontal list. In traditional \TEX\ the discretionary node contains a
counter indicating how many nodes to skip, but in \LUATEX\ we store the pre, post
and replace text in the discretionary node.

The combined effect of these two differences is that \LUATEX\ does not always use
all of the potential breakpoints in a paragraph, especially when fonts with many
ligatures are used. Of course kerning also complicates matters here.

\section{The \type {lang} library}

This library provides the interface to \LUATEX's structure
representing a language, and the associated functions.

\startfunctioncall
<language> l = lang.new()
<language> l = lang.new(<number> id)
\stopfunctioncall

This function creates a new userdata object. An object of type \type {<language>}
is the first argument to most of the other functions in the \type {lang}
library. These functions can also be used as if they were object methods, using
the colon syntax.

Without an argument, the next available internal id number will be assigned to
this object. With argument, an object will be created that links to the internal
language with that id number.

\startfunctioncall
<number> n = lang.id(<language> l)
\stopfunctioncall

returns the internal \type {\language} id number this object refers to.

\startfunctioncall
<string> n = lang.hyphenation(<language> l)
lang.hyphenation(<language> l, <string> n)
\stopfunctioncall

Either returns the current hyphenation exceptions for this language, or adds new
ones. The syntax of the string is explained in~\in {section}
[patternsexceptions].

\startfunctioncall
lang.clear_hyphenation(<language> l)
\stopfunctioncall

Clears the exception dictionary (string) for this language.

\startfunctioncall
<string> n = lang.clean(<language> l, <string> o)
<string> n = lang.clean(<string> o)
\stopfunctioncall

Creates a hyphenation key from the supplied hyphenation value. The syntax of the
argument string is explained in~\in {section} [patternsexceptions]. This function
is useful if you want to do something else based on the words in a dictionary
file, like spell|-|checking.

\startfunctioncall
<string> n = lang.patterns(<language> l)
lang.patterns(<language> l, <string> n)
\stopfunctioncall

Adds additional patterns for this language object, or returns the current set.
The syntax of this string is explained in~\in {section} [patternsexceptions].

\startfunctioncall
lang.clear_patterns(<language> l)
\stopfunctioncall

Clears the pattern dictionary for this language.

\startfunctioncall
<number> n = lang.prehyphenchar(<language> l)
lang.prehyphenchar(<language> l, <number> n)
\stopfunctioncall

Gets or sets the \quote {pre|-|break} hyphen character for implicit hyphenation
in this language (initially the hyphen, decimal 45).

\startfunctioncall
<number> n = lang.posthyphenchar(<language> l)
lang.posthyphenchar(<language> l, <number> n)
\stopfunctioncall

Gets or sets the \quote {post|-|break} hyphen character for implicit hyphenation
in this language (initially null, decimal~0, indicating emptiness).

\startfunctioncall
<number> n = lang.preexhyphenchar(<language> l)
lang.preexhyphenchar(<language> l, <number> n)
\stopfunctioncall

Gets or sets the \quote {pre|-|break} hyphen character for explicit hyphenation
in this language (initially null, decimal~0, indicating emptiness).

\startfunctioncall
<number> n = lang.postexhyphenchar(<language> l)
lang.postexhyphenchar(<language> l, <number> n)
\stopfunctioncall

Gets or sets the \quote {post|-|break} hyphen character for explicit hyphenation
in this language (initially null, decimal~0, indicating emptiness).

\startfunctioncall
<boolean> success = lang.hyphenate(<node> head)
<boolean> success = lang.hyphenate(<node> head, <node> tail)
\stopfunctioncall

Inserts hyphenation points (discretionary nodes) in a node list. If \type {tail}
is given as argument, processing stops on that node. Currently, \type {success}
is always true if \type {head} (and \type {tail}, if specified) are proper nodes,
regardless of possible other errors.

Hyphenation works only on \quote {characters}, a special subtype of all the glyph
nodes with the node subtype having the value \type {1}. Glyph modes with
different subtypes are not processed. See \in {section~} [charsandglyphs] for
more details.

The following two commands can be used to set or query hj codes:

\startfunctioncall
lang.sethjcode(<language> l, <number> char, <number> usedchar)
<number> usedchar = lang.gethjcode(<language> l, <number> char)
\stopfunctioncall

When you set a hjcode the current sets get initialized unless the set was already
initialized due to \type {\savinghyphcodes} being larger than zero.

\stopchapter

\stopcomponent

% \parindent0pt \hsize=1.1cm
% 12-34-56 \par
% 12-34-\hbox{56} \par
% 12-34-\vrule width 1em height 1.5ex \par
% 12-\hbox{34}-56 \par
% 12-\vrule width 1em height 1.5ex-56 \par
% \hjcode`\1=`\1 \hjcode`\2=`\2 \hjcode`\3=`\3 \hjcode`\4=`\4 \vskip.5cm
% 12-34-56 \par
% 12-34-\hbox{56} \par
% 12-34-\vrule width 1em height 1.5ex \par
% 12-\hbox{34}-56 \par
% 12-\vrule width 1em height 1.5ex-56 \par

%
    %D \module
%D   [       file=luatex-mplib,
%D        version=2009.12.01,
%D          title=\LUATEX\ Support Macros,
%D       subtitle=\METAPOST\ to \PDF\ conversion,
%D         author=Taco Hoekwater \& Hans Hagen,
%D      copyright=see context related readme files]

%D This is the companion to the \LUA\ module \type {supp-mpl.lua}. Further
%D embedding is up to others. A simple example of usage in plain \TEX\ is:
%D
%D \starttyping
%D \pdfoutput=1
%D
%D \input luatex-mplib.tex
%D
%D \setmplibformat{plain}
%D
%D \mplibcode
%D   beginfig(1);
%D     draw fullcircle
%D       scaled 10cm
%D       withcolor red
%D       withpen pencircle xscaled 4mm yscaled 2mm rotated 30 ;
%D   endfig;
%D \endmplibcode
%D
%D \end
%D \stoptyping

\def\setmplibformat#1{\def\mplibformat{#1}}

\def\setupmplibcatcodes
  {\catcode`\{=12 \catcode`\}=12 \catcode`\#=12 \catcode`\^=12 \catcode`\~=12
   \catcode`\_=12 \catcode`\%=12 \catcode`\&=12 \catcode`\$=12 }

\def\mplibcode
  {\bgroup
   \setupmplibcatcodes
   \domplibcode}

\long\def\domplibcode#1\endmplibcode
  {\egroup
   \directlua{metapost.process('\mplibformat',[[#1]])}}

%D We default to \type {plain} \METAPOST:

\def\mplibformat{plain}

%D We use a dedicated scratchbox:

\ifx\mplibscratchbox\undefined \newbox\mplibscratchbox \fi

%D Now load the needed \LUA\ code.

\directlua{dofile(kpse.find_file('luatex-mplib.lua'))}

%D The following code takes care of encapsulating the literals:

\def\startMPLIBtoPDF#1#2#3#4%
  {\hbox\bgroup
   \xdef\MPllx{#1}\xdef\MPlly{#2}%
   \xdef\MPurx{#3}\xdef\MPury{#4}%
   \xdef\MPwidth{\the\dimexpr#3bp-#1bp\relax}%
   \xdef\MPheight{\the\dimexpr#4bp-#2bp\relax}%
   \parskip0pt%
   \leftskip0pt%
   \parindent0pt%
   \everypar{}%
   \setbox\mplibscratchbox\vbox\bgroup
   \noindent}

\def\stopMPLIBtoPDF
  {\egroup
   \setbox\mplibscratchbox\hbox
     {\hskip-\MPllx bp%
      \raise-\MPlly bp%
      \box\mplibscratchbox}%
   \setbox\mplibscratchbox\vbox to \MPheight
     {\vfill
      \hsize\MPwidth
      \wd\mplibscratchbox0pt%
      \ht\mplibscratchbox0pt%
      \dp\mplibscratchbox0pt%
      \box\mplibscratchbox}%
   \wd\mplibscratchbox\MPwidth
   \ht\mplibscratchbox\MPheight
   \box\mplibscratchbox
   \egroup}

%D The body of picture, except for text items, is taken care of by:

\ifnum\pdfoutput>0
    \let\MPLIBtoPDF\pdfliteral
\else
    \def\MPLIBtoPDF#1{\special{pdf:literal direct #1}} % not ok yet
\fi

%D Text items have a special handler:

\def\MPLIBtextext#1#2#3#4#5%
  {\begingroup
   \setbox\mplibscratchbox\hbox
     {\font\temp=#1 at #2bp%
      \temp
      #3}%
   \setbox\mplibscratchbox\hbox
     {\hskip#4 bp%
      \raise#5 bp%
      \box\mplibscratchbox}%
   \wd\mplibscratchbox0pt%
   \ht\mplibscratchbox0pt%
   \dp\mplibscratchbox0pt%
   \box\mplibscratchbox
   \endgroup}

\endinput
%
    %D \module
%D   [       file=luatex-gadgets,
%D        version=2015.05.12,
%D          title=\LUATEX\ Support Macros,
%D       subtitle=Useful stuff from articles,
%D         author=Hans Hagen,
%D           date=\currentdate,
%D      copyright={PRAGMA ADE \& \CONTEXT\ Development Team}]

\directlua{dofile(resolvers.findfile('luatex-gadgets.lua'))}

% optional removal of marked content
%
% before\marksomething{gone}{\em HERE}\unsomething{gone}after
% before\marksomething{kept}{\em HERE}\unsomething{gone}after
% \marksomething{gone}{\em HERE}\unsomething{gone}last
% \marksomething{kept}{\em HERE}\unsomething{gone}last

\def\setmarksignal  #1{\directlua{gadgets.marking.setsignal(\number#1)}}
\def\marksomething#1#2{{\directlua{gadgets.marking.mark("#1")}{#2}}}
\def\unsomething    #1{\directlua{gadgets.marking.remove("#1")}}

\newattribute\gadgetmarkattribute \setmarksignal\gadgetmarkattribute

\endinput
%
}

% We also patch the version number:

\edef\fmtversion{\fmtversion+luatex}

\automatichyphenmode=1

\dump
